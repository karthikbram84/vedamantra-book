\chapt{काण्डम् ३}
%3.1.0.0
% {\anuvakamend[{प्र॒जा\-प॑तिरकामयतै॒ष ते॑ य॒ज्ञं वै प्र॒जा\-प॑ते॒र्जाय॑मानाः प्राजाप॒त्या यो वा अय॑थादेवतमि॒ष्टर्गो॑ निग्रा॒भ्याः᳚ स्थ॒ यो वै दे॒वां जुष्टो॒\-ऽग्निना॑ र॒यिमेका॑\-दश}]}%॥11॥ 
\sect{प्रथमः प्रश्नः}\setcounter{anuvakam}{0}
\dnsub{तैत्तिरीयसंहितायां तृतीयकाण्डे प्रथमः प्रश्नः}
%3.1.1.0 
%3.1.1.1
प्र॒जा\-प॑तिरकामयत प्र॒जाः सृ॑जे॒येति॒ स तपो॑\-ऽतप्यत॒ स स॒र्पान॑सृजत॒ सो॑\-ऽकामयत प्र॒जाः सृ॑जे॒येति॒ स द्वि॒तीय॑म\-तप्यत॒ स वयाꣴ॑स्यसृजत॒ सो॑\-ऽकामयत प्र॒जाः सृ॑जे॒येति॒ स तृ॒तीय॑मतप्यत॒ स ए॒तं दी᳚क्षितवा॒दम॑पश्य॒त्तम॑वद॒त्ततो॒ वै स प्र॒जा अ॑सृजत॒ यत्तप॑स्त॒प्त्वा दी᳚क्षितवा॒दं वद॑ति प्र॒जा ए॒व तद्यज॑मानः~(१)

%3.1.1.2
सृजते॒ यद्वै दी᳚क्षि॒तो॑\-ऽमे॒ध्यं पश्य॒त्यपा᳚स्माद्दी॒क्षा क्रा॑मति॒ नील॑मस्य॒ हरो॒ व्ये᳚त्यब॑द्धं॒ मनो॑ द॒रिद्रं॒ चक्षुः॒ सूर्यो॒ ज्योति॑षा॒ꣴ॒ श्रेष्ठो॒ दीक्षे॒ मा मा॑ हासी॒रित्या॑ह॒ नास्मा᳚द्दी॒क्षाऽप॑ क्रामति॒ नास्य॒ नीलं॒ न हरो॒ व्ये॑ति॒ यद्वै दी᳚क्षि॒तम॑भि॒वर्\mbox{}ष॑ति दि॒व्या आपो\-ऽशा᳚न्ता॒ ओजो॒ बलं॑ दी॒क्षां~(२)

%3.1.1.3
तपो᳚\-ऽस्य॒ निर्घ्न॑न्त्युन्द॒तीर्बलं॑ ध॒त्तौजो॑ धत्त॒ बलं॑ धत्त॒ मा मे॑ दी॒क्षां मा तपो॒ निर्व॑धि॒ष्टेत्या॑है॒तदे॒व सर्व॑मा॒त्मन्ध॑त्ते॒ नास्यौजो॒ बलं॒ न दी॒क्षां न तपो॒ निर्घ्न॑न्त्य॒ग्निर्वै दी᳚क्षि॒तस्य॑ दे॒वता॒ सो᳚\-ऽस्मादे॒तर्\mbox{}हि॑ ति॒र इ॑व॒ यर्\mbox{}हि॒ याति॒ तमी᳚श्व॒रꣳ रक्षाꣳ॑सि॒ हन्तो᳚र्-~(३)

%3.1.1.4
भ॒द्राद॒भि श्रेयः॒ प्रेहि॒ बृह॒स्पतिः॑ पुरए॒ता ते॑ अ॒स्त्वित्या॑ह॒ ब्रह्म॒ वै दे॒वानां॒ बृह॒स्पति॒स्तमे॒वान्वार॑भते॒ स ए॑न॒ꣳ॒ सं पा॑रय॒त्येदम॑गन्म देव॒यज॑नं पृथि॒व्या इत्या॑ह देव॒यज॑न॒ꣴ॒ ह्ये॑ष पृ॑थि॒व्या आ॒गच्छ॑ति॒ यो यज॑ते॒ विश्वे॑ दे॒वा यदजु॑षन्त॒ पूर्व॒ इत्या॑ह॒ विश्वे॒ ह्ये॑तद्दे॒वा जो॒षय॑न्ते॒ यद्ब्रा᳚ह्म॒णा ऋ॑ख्सा॒माभ्यां॒ यजु॑षा स॒न्तर॑न्त॒ इत्या॑हर्ख्सा॒माभ्या॒ꣴ॒ ह्ये॑ष यजु॑षा स॒न्तर॑ति॒ यो यज॑ते रा॒यस्पोषे॑ण॒ समि॒षा म॑दे॒मेत्या॑हा॒ऽऽशिष॑मे॒वैतामा शा᳚स्ते॥~(४)

%3.1.2.0
{\anuvakamend[{यज॑मानो दी॒क्षाꣳ हन्तो᳚र्ब्राह्म॒णाश्चतु॑र्विꣳशतिश्च}]}%~(१)

%3.1.2.1
ए॒ष ते॑ गाय॒त्रो भा॒ग इति॑ मे॒ सोमा॑य ब्रूतादे॒ष ते॒ त्रैष्टु॑भो॒ जाग॑तो भा॒ग इति॑ मे॒ सोमा॑य ब्रूताच्छन्दो॒माना॒ꣳ॒ साम्रा᳚ज्यं ग॒च्छेति॑ मे॒ सोमा॑य ब्रूता॒द्यो वै सोम॒ꣳ॒ राजा॑न॒ꣳ॒ साम्रा᳚ज्यं लो॒कं ग॑मयि॒त्वा क्री॒णाति॒ गच्छ॑ति॒ स्वाना॒ꣳ॒ साम्रा᳚ज्यं॒ छन्दाꣳ॑सि॒ खलु॒ वै सोम॑स्य॒ राज्ञः॒ साम्रा᳚ज्यो लो॒कः पु॒रस्ता॒थ्सोम॑स्य क्र॒यादे॒वम॒भि म॑न्त्रयेत॒ साम्रा᳚ज्यमे॒वै-~(५)

%3.1.2.2
नं॑ लो॒कं ग॑मयि॒त्वा क्री॑णाति॒ गच्छ॑ति॒ स्वाना॒ꣳ॒ साम्रा᳚ज्यं॒ यो वै ता॑नून॒प्त्रस्य॑ प्रति॒ष्ठां वेद॒ प्रत्ये॒व ति॑ष्ठति ब्रह्मवा॒दिनो॑ वदन्ति॒ न प्रा॒श्ञन्ति॒ न जु॑ह्व॒त्यथ॒ क्व॑ तानून॒प्त्रं प्रति॑ तिष्ठ॒तीति॑ प्र॒जा\-प॑तौ॒ मन॒सीति॑ ब्रूया॒त्त्रिरव॑ जिघ्रेत्प्र॒जा\-प॑तौ त्वा॒ मन॑सि जुहो॒मीत्ये॒षा वै ता॑नून॒प्त्रस्य॑ प्रति॒ष्ठा य ए॒वं वेद॒ प्रत्ये॒व ति॑ष्ठति॒ यो~(६)

%3.1.2.3
वा अ॑ध्व॒र्योः प्र॑ति॒ष्ठां वेद॒ प्रत्ये॒व ति॑ष्ठति॒ यतो॒ मन्ये॒तान॑भिक्रम्य होष्या॒मीति॒ तत्तिष्ठ॒न्ना श्रा॑वयेदे॒षा वा अ॑ध्व॒र्योः प्र॑ति॒ष्ठा य ए॒वं वेद॒ प्रत्ये॒व ति॑ष्ठति॒ यद॑भि॒क्रम्य॑ जुहु॒यात्प्र॑ति॒ष्ठाया॑ इया॒त्तस्मा᳚थ्समा॒नत्र॒ तिष्ठ॑ता होत॒व्यं॑ प्रति॑ष्ठित्यै॒ यो वा अ॑ध्व॒र्योः स्वं वेद॒ स्ववा॑ने॒व भ॑वति॒ स्रुग्वा अ॑स्य॒ स्वं वा॑य॒व्य॑मस्य॒~(७)

%3.1.2.4
स्वं च॑म॒सो᳚\-ऽस्य॒ स्वं यद्वा॑य॒व्यं॑ वा चम॒सं वा\-ऽन॑न्वारभ्याऽऽश्रा॒वये॒थ्\-स्वादि॑या॒त् तस्मा॑दन्वा॒रभ्या॒ऽऽश्राव्य॒ꣴ॒ स्वादे॒व नैति॒ यो वै सोम॒मप्र॑तिष्ठाप्य स्तो॒त्रमु॑पाक॒रोत्यप्र॑तिष्ठितः॒ सोमो॒ भव॒त्यप्र॑तिष्ठितः॒ स्तोमो\-ऽप्र॑तिष्ठितान्यु॒क्थान्यप्र॑तिष्ठितो॒ यज॑मा॒नो\-ऽप्र॑तिष्ठितो\-ऽध्व॒र्युर्वा॑य॒व्यं॑ वै सोम॑स्य प्रति॒ष्ठा च॑म॒सो᳚\-ऽस्य प्रति॒ष्ठा सोमः॒ स्तोम॑स्य॒ स्तोम॑ उ॒क्थानां॒ ग्रहं॑ वा गृही॒त्वा च॑म॒सं वो॒न्नीय॑ स्तो॒त्रमु॒पाकु॑र्या॒त्प्रत्ये॒व सोमꣴ॑ स्था॒पय॑ति॒ प्रति॒ स्तोमं॒ प्रत्यु॒क्थानि॒ प्रति॒ यज॑मान॒स्तिष्ठ॑ति॒ प्रत्य॑ध्व॒र्युः॥~(८)

%3.1.3.0
{\anuvakamend[{ए॒व ति॑ष्ठति॒ यो वा॑य॒व्य॑मस्य॒ ग्रहं॒ वैका॒न्नविꣳ॑श॒तिश्च॑}]}%~(२)

%3.1.3.1
य॒ज्ञं वा ए॒तथ्सम्भ॑रन्ति॒ यथ्सो॑म॒क्रय॑ण्यै प॒दं य॑ज्ञमु॒खꣳ ह॑वि॒र्धाने॒ यर्\mbox{}हि॑ हवि॒र्धाने॒ प्राची᳚ प्रव॒र्तये॑यु॒स्तर्\mbox{}हि॒ तेनाक्ष॒मुपा᳚ञ्ज्याद्यज्ञ\-मु॒ख ए॒व य॒ज्ञमनु॒ सं त॑नोति॒ प्राञ्च॑म॒ग्निं प्र ह॑र॒न्त्युत्पत्नी॒मा न॑य॒न्त्यन्वनाꣳ॑सि॒ प्र व॑र्तय॒न्त्यथ॒ वा अ॑स्यै॒ष धिष्णि॑यो हीयते॒ सो\-ऽनु॑ ध्यायति॒ स ई᳚श्व॒रो रु॒द्रो भू॒त्वा~(९)

%3.1.3.2
प्र॒जां प॒शून् यज॑मानस्य॒ शम॑यितो॒र्यर्\mbox{}हि॑ प॒शुमाप्री॑त॒मुद॑ञ्चं॒ नय॑न्ति॒ तर्\mbox{}हि॒ तस्य॑ पशु॒श्रप॑णꣳ हरे॒त्तेनै॒वैनं॑ भा॒गिनं॑ करोति॒ यज॑मानो॒ वा आ॑हव॒नीयो॒ यज॑मानं॒ वा ए॒तद्वि क॑र्\mbox{}षन्ते॒ यदा॑हव॒नीया᳚त्पशु॒श्रप॑ण॒ꣳ॒ हर॑न्ति॒ स वै॒व स्यान्नि॑र्म॒न्थ्यं॑ वा कुर्या॒द्यज॑मानस्य सात्म॒त्वाय॒ यदि॑ प॒शोर॑व॒दानं॒ नश्ये॒दाज्य॑स्य प्रत्या॒ख्याय॒मव॑ द्ये॒थ्सैव ततः॒ प्राय॑श्चित्ति॒र्ये प॒शुं वि॑मथ्नी॒रन् यस्तान्का॒मये॒तार्ति॒मार्च्छे॑यु॒रिति॑ कु॒विद॒ङ्गेति॒ नमो॑वृक्तिवत्य॒र्चाऽऽग्नी᳚ध्रे जुहुया॒न्नमो॑वृक्तिमे॒वैषां᳚ वृङ्क्ते ता॒जगार्ति॒मार्च्छ॑न्ति॥~(१०)

%3.1.4.0
{\anuvakamend[{भू॒त्वा ततः॒ षड्विꣳ॑शतिश्च}]}%~(३)

%3.1.4.1
प्र॒जा\-प॑ते॒र्जाय॑मानाः प्र॒जा जा॒ताश्च॒ या इ॒माः। तस्मै॒ प्रति॒ प्र वे॑दय चिकि॒त्वाꣳ अनु॑ मन्यताम्। इ॒मं प॒शुं प॑शुपते ते अ॒द्य ब॒ध्नाम्य॑ग्ने सुकृ॒तस्य॒ मध्ये᳚। अनु॑ मन्यस्व सु॒यजा॑ यजाम॒ जुष्टं॑ दे॒वाना॑मि॒दम॑स्तु ह॒व्यम्। प्र॒जा॒नन्तः॒ प्रति॑ गृह्णन्ति॒ पूर्वे᳚ प्रा॒णमङ्गे᳚भ्यः॒ पर्या॒चर॑न्तम्। सु॒व॒र्गं या॑हि प॒थिभि॑र्देव॒यानै॒रोष॑धीषु॒ प्रति॑ तिष्ठा॒ शरी॑रैः। येषा॒मीशे॑~(११)

%3.1.4.2
पशु॒पतिः॑ पशू॒नां चतु॑ष्पदामु॒त च॑ द्वि॒पदा᳚म्। निष्क्री॑तो॒\-ऽयं य॒ज्ञियं॑ भा॒गमे॑तु रा॒यस्पोषा॒ यज॑मानस्य सन्तु। ये ब॒ध्यमा॑न॒मनु॑ ब॒ध्यमा॑ना अ॒भ्यैक्ष॑न्त॒ मन॑सा॒ चक्षु॑षा च। अ॒ग्निस्ताꣳ अग्रे॒ प्र मु॑मोक्तु दे॒वः प्र॒जा\-प॑तिः प्र॒जया॑ संविदा॒नः। य आ॑र॒ण्याः प॒शवो॑ वि॒श्वरू॑पा॒ विरू॑पाः॒ सन्तो॑ बहु॒धैक॑रूपाः। वा॒युस्ताꣳ अग्रे॒ प्र मु॑मोक्तु दे॒वः प्र॒जा\-प॑तिः प्र॒जया॑ संविदा॒नः। प्र॒मु॒ञ्चमा॑ना॒~(१२)

%3.1.4.3
भुव॑नस्य॒ रेतो॑ गा॒तुं ध॑त्त॒ यज॑मानाय देवाः। उ॒पाकृ॑तꣳ शशमा॒नं यदस्था᳚ज्जी॒वं दे॒वाना॒मप्ये॑तु॒ पाथः॑। नाना᳚ प्रा॒णो यज॑मानस्य प॒शुना॑ य॒ज्ञो दे॒वेभिः॑ स॒ह दे॑व॒यानः॑। जी॒वं दे॒वाना॒मप्ये॑तु॒ पाथः॑ स॒त्याः स॑न्तु॒ यज॑मानस्य॒ कामाः᳚। यत्प॒शुर्मा॒युमकृ॒तोरो॑ वा प॒द्भिरा॑ह॒ते। अ॒ग्निर्मा॒ तस्मा॒देन॑सो॒ विश्वा᳚न्मुञ्च॒त्वꣳह॑सः। शमि॑तार उ॒पेत॑न य॒ज्ञं~(१३)

%3.1.4.4
दे॒वेभि॑रिन्वि॒तम्। पाशा᳚त्प॒शुं प्र मु॑ञ्चत ब॒न्धाद्य॒ज्ञप॑तिं॒ परि॑। अदि॑तिः॒ पाशं॒ प्र मु॑मोक्त्वे॒तं नमः॑ प॒शुभ्यः॑ पशु॒पत॑ये करोमि। अ॒रा॒ती॒यन्त॒मध॑रं कृणोमि॒ यं द्वि॒ष्मस्तस्मि॒न्प्रति॑ मुञ्चामि॒ पाशम्᳚। त्वामु॒ ते द॑धिरे हव्य॒वाहꣳ॑ शृतं क॒र्तार॑मु॒त य॒ज्ञियं॑ च। अग्ने॒ सद॑क्षः॒ सत॑नु॒र्॒\mbox{}हि भू॒त्वा\-ऽथ॑ ह॒व्या जा॑तवेदो जुषस्व। जात॑वेदो व॒पया॑ गच्छ दे॒वान्त्वꣳ हि होता᳚ प्रथ॒मो ब॒भूथ॑। घृ॒तेन॒ त्वं त॒नुवो॑ वर्धयस्व॒ स्वाहा॑कृतꣳ ह॒विर॑दन्तु दे॒वाः। स्वाहा॑ दे॒वेभ्यो॑ दे॒वेभ्यः॒ स्वाहा᳚॥~(१४)

%3.1.5.0
{\anuvakamend[{ईशे᳚ प्रमु॒ञ्चमा॑ना य॒ज्ञन्त्वꣳ षोड॑श च}]}%~(४)

%3.1.5.1
प्रा॒जा॒प॒त्या वै प॒शव॒स्तेषाꣳ॑ रु॒द्रो\-ऽधि॑पति॒र्यदे॒ताभ्या॑मुपाक॒रोति॒ ताभ्या॑मे॒वैनं॑ प्रति॒प्रोच्या ल॑भत आ॒त्मनो\-ऽना᳚व्रस्काय॒ द्वाभ्या॑मु॒पा\-क॑रोति द्वि॒पाद्यज॑मानः॒ प्रति॑ष्ठित्या उपा॒कृत्य॒ पञ्च॑ जुहोति॒ पाङ्क्ताः᳚ प॒शवः॑ प॒शूने॒वाव॑ रुन्धे मृ॒त्यवे॒ वा ए॒ष नी॑यते॒ यत्प॒शुस्तं यद॑न्वा॒रभे॑त प्र॒मायु॑को॒ यज॑मानः स्या॒न्नाना᳚ प्रा॒णो यज॑मानस्य प॒शुनेत्या॑ह॒ व्यावृ॑त्त्यै॒~(१५)

%3.1.5.2
यत्प॒शुर्मा॒युमकृ॒तेति॑ जुहोति॒ शान्त्यै॒ शमि॑तार उ॒पेत॒नेत्या॑ह यथाय॒जुरे॒वैतद्व॒पायां॒ वा आ᳚ह्रि॒यमा॑णायाम॒ग्नेर्मेधो\-ऽप॑ क्रामति॒ त्वामु॒ ते द॑धिरे हव्य॒वाह॒मिति॑ व॒पाम॒भि जु॑होत्य॒ग्नेरे॒व मेध॒मव॑ रु॒न्धे\-ऽथो॑ शृत॒त्वाय॑ पु॒रस्ता᳚थ्\-स्वाहाकृतयो॒ वा अ॒न्ये दे॒वा उ॒परि॑ष्टाथ्\-स्वाहाकृतयो॒\-ऽन्ये स्वाहा॑ दे॒वेभ्यो॑ दे॒वेभ्यः॒ स्वाहेत्य॒भितो॑ व॒पां जु॑होति॒ ताने॒वोभया᳚न्प्रीणाति॥~(१६)

%3.1.6.0
{\anuvakamend[{व्यावृ॑त्त्या अ॒भितो॑ व॒पां पञ्च॑ च}]}%~(५)

%3.1.6.1
यो वा अय॑थादेवतं य॒ज्ञमु॑प॒चर॒त्या दे॒वता᳚भ्यो वृश्च्यते॒ पापी॑यान्भवति॒ यो य॑थादेव॒तं न दे॒वता᳚भ्य॒ आ वृ॑श्च्यते॒ वसी॑यान्भवत्याग्ने॒य्यर्चाग्नी᳚ध्रम॒भि मृ॑शेद्वैष्ण॒व्या ह॑वि॒र्धान॑माग्ने॒य्या स्रुचो॑ वाय॒व्य॑या वाय॒व्या᳚न्यैन्द्रि॒या सदो॑ यथादेव॒तमे॒व य॒ज्ञमुप॑ चरति॒ न दे॒वता᳚भ्य॒ आ वृ॑श्च्यते॒ वसी॑यान्भवति यु॒नज्मि॑ ते पृथि॒वीं ज्योति॑षा स॒ह यु॒नज्मि॑ वा॒युम॒न्तरि॑क्षेण~(१७)

%3.1.6.2
ते स॒ह यु॒नज्मि॒ वाचꣳ॑ स॒ह सूर्ये॑ण ते यु॒नज्मि॑ ति॒स्रो वि॒पृचः॒ सूर्य॑स्य ते। अ॒ग्निर्दे॒वता॑ गाय॒त्री छन्द॑ उपा॒ꣳ॒शोः पात्र॑मसि॒ सोमो॑ दे॒वता᳚ त्रि॒ष्टुप्छन्दो᳚\-ऽन्तर्या॒मस्य॒ पात्र॑म॒सीन्द्रो॑ दे॒वता॒ जग॑ती॒ छन्द॑ इन्द्रवायु॒वोः पात्र॑मसि॒ बृह॒स्पति॑र्दे॒वता॑\-ऽनु॒ष्टुप्छन्दो॑ मि॒त्रावरु॑णयोः॒ पात्र॑मस्य॒श्विनौ॑ दे॒वता॑ प॒ङ्क्तिश्छन्दो॒\-ऽश्विनोः॒ पात्र॑मसि॒ सूर्यो॑ दे॒वता॑ बृह॒ती~(१८)

%3.1.6.3
छन्दः॑ शु॒क्रस्य॒ पात्र॑मसि च॒न्द्रमा॑ दे॒वता॑ स॒तोबृ॑हती॒ छन्दो॑ म॒न्थिनः॒ पात्र॑मसि॒ विश्वे॑ दे॒वा दे॒वतो॒ष्णिहा॒ छन्द॑ आग्रय॒णस्य॒ पात्र॑म॒सीन्द्रो॑ दे॒वता॑ क॒कुच्छन्द॑ उ॒क्थानां॒ पात्र॑मसि पृथि॒वी दे॒वता॑ वि॒राट्छन्दो᳚ ध्रु॒वस्य॒ पात्र॑मसि॥~(१९)

%3.1.7.0
{\anuvakamend[{अ॒न्तरि॑क्षेण बृह॒ती त्रय॑स्त्रिꣳशच्च}]}%~(६)

%3.1.7.1
इ॒ष्टर्गो॒ वा अ॑ध्व॒र्युर्यज॑मानस्ये॒ष्टर्गः॒ खलु॒ वै पूर्वो॒\-ऽर्ष्टुः क्षी॑यत आस॒न्या᳚न्मा॒ मन्त्रा᳚त्पाहि॒ कस्या᳚श्चिद॒भिश॑स्त्या॒ इति॑ पु॒रा प्रा॑तरनुवा॒काज्जु॑हुयादा॒त्मन॑ ए॒व तद॑ध्व॒र्युः पु॒रस्ता॒च्छर्म॑ नह्य॒ते\-ऽना᳚र्त्यै संवे॒शाय॑ त्वोपवे॒शाय॑ त्वा गायत्रि॒यास्त्रि॒ष्टुभो॒ जग॑त्या अ॒भिभू᳚त्यै॒ स्वाहा॒ प्राणा॑पानौ मृ॒त्योर्मा॑ पातं॒ प्राणा॑पानौ॒ मा मा॑ हासिष्टं दे॒वता॑सु॒ वा ए॒ते प्रा॑णापा॒नयो॒र्-~(२०)

%3.1.7.2
व्याय॑च्छन्ते॒ येषा॒ꣳ॒ सोमः॑ समृ॒च्छते॑ संवे॒शाय॑ त्वोपवे॒शाय॒ त्वेत्या॑ह॒ छन्दाꣳ॑सि॒ वै सं॑वे॒श उ॑पवे॒शश्छन्दो॑भिरे॒वास्य॒ छन्दाꣳ॑सि वृङ्क्ते॒ प्रेति॑व॒न्त्याज्या॑नि भवन्त्य॒भिजि॑त्यै म॒रुत्व॑तीः प्रति॒पदो॒ विजि॑त्या उ॒भे बृ॑हद्रथन्त॒रे भ॑वत इ॒यं वाव र॑थन्त॒रम॒सौ बृ॒हदा॒भ्यामे॒वैन॑म॒न्तरे᳚त्य॒द्य वाव र॑थन्त॒रꣴ श्वो बृ॒हद॑द्या॒श्वादे॒वैन॑म॒न्तरे॑ति भू॒तं~(२१)

%3.1.7.3
वाव र॑थन्त॒रं भ॑वि॒ष्यद्बृ॒हद्भू॒ताच्चै॒वैनं॑ भविष्य॒तश्चा॒न्तरे॑ति॒ परि॑मितं॒ वाव र॑थन्त॒रमप॑रिमितं बृ॒हत्परि॑मिताच्चै॒वैन॒मप॑रि\-मिताच्चा॒न्तरे॑ति विश्वामित्रजमद॒ग्नी वसि॑ष्ठेनास्पर्धेता॒ꣳ॒ स ए॒तज्ज॒मद॑ग्निर्विह॒व्य॑मपश्य॒त्तेन॒ वै स वसि॑ष्ठस्येन्द्रि॒यं वी॒र्य॑मवृङ्क्त॒ यद्वि॑ह॒व्यꣳ॑ श॒स्यत॑ इन्द्रि॒यमे॒व तद्वी॒र्यं॑ यज॑मानो॒ भ्रातृ॑व्यस्य वृङ्क्ते॒ यस्य॒ भूयाꣳ॑सो यज्ञक्र॒तव॒ इत्या॑हुः॒ स दे॒वता॑ वृङ्क्त॒ इति॒ यद्य॑ग्निष्टो॒मः सोमः॑ प॒रस्ता॒थ्स्यादु॒क्थ्यं॑ कुर्वीत॒ यद्यु॒क्थ्यः॑ स्याद॑तिरा॒त्रं कु॑र्वीत यज्ञक्र॒तुभि॑रे॒वास्य॑ दे॒वता॑ वृङ्क्ते॒ वसी॑यान्भवति॥~(२२)

%3.1.8.0
{\anuvakamend[{प्रा॒णा॒पा॒नयो᳚र्भू॒तं वृ॑ङ्क्ते॒\-ऽष्टाविꣳ॑शतिश्च}]}%~(७)

%3.1.8.1
नि॒ग्रा॒भ्याः᳚ स्थ देव॒श्रुत॒ आयु॑र्मे तर्पयत प्रा॒णं मे॑ तर्पयतापा॒नं मे॑ तर्पयत व्या॒नं मे॑ तर्पयत॒ चक्षु॑र्मे तर्पयत॒ श्रोत्रं॑ मे तर्पयत॒ मनो॑ मे तर्पयत॒ वाचं॑ मे तर्पयता॒ऽऽत्मानं॑ मे तर्पय॒ताङ्गा॑नि मे तर्पयत प्र॒जां मे॑ तर्पयत प॒शून्मे॑ तर्पयत गृ॒हान्मे॑ तर्पयत ग॒णान्मे॑ तर्पयत स॒र्वग॑णं मा तर्पयत त॒र्पय॑त मा~(२३)

%3.1.8.2
ग॒णा मे॒ मा वि तृ॑ष॒न्नोष॑धयो॒ वै सोम॑स्य॒ विशो॒ विशः॒ खलु॒ वै राज्ञः॒ प्रदा॑तोरीश्व॒रा ऐ॒न्द्रः सोमो\-ऽवी॑वृधं वो॒ मन॑सा सुजाता॒ ऋत॑प्रजाता॒ भग॒ इद्वः॑ स्याम। इन्द्रे॑ण दे॒वीर्वी॒रुधः॑ संविदा॒ना अनु॑ मन्यन्ता॒ꣳ॒ सव॑नाय॒ सोम॒मित्या॒हौष॑धीभ्य ए॒वैन॒ꣴ॒ स्वायै॑ वि॒शः स्वायै॑ दे॒वता॑यै नि॒र्याच्या॒भि षु॑णोति॒ यो वै सोम॑स्याभिषू॒यमा॑णस्य~(२४)

%3.1.8.3
प्रथ॒मो\-ऽꣳ॑शुः स्कन्द॑ति॒ स ई᳚श्व॒र इ॑न्द्रि॒यं वी॒र्यं॑ प्र॒जां प॒शून् यज॑मानस्य॒ निर्\mbox{}ह॑न्तो॒स्तम॒भि म॑न्त्रये॒ताऽऽ मा᳚स्कान्थ्स॒ह प्र॒जया॑ स॒ह रा॒यस्पोषे॑णेन्द्रि॒यं मे॑ वी॒र्यं॑ मा निर्व॑धी॒रित्या॒शिष॑मे॒वैतामा शा᳚स्त इन्द्रि॒यस्य॑ वी॒र्य॑स्य प्र॒जायै॑ पशू॒नामनि॑र्घाताय द्र॒फ्सश्च॑स्कन्द पृथि॒वीमनु॒ द्यामि॒मं च॒ योनि॒मनु॒ यश्च॒ पूर्वः॑। तृ॒तीयं॒ योनि॒मनु॑ स॒ञ्चर॑न्तं द्र॒फ्सं जु॑हो॒म्यनु॑ स॒प्त होत्राः᳚॥~(२५)

%3.1.9.0
{\anuvakamend[{त॒र्पय॑त मा\-ऽभिषू॒यमा॑णस्य॒ यश्च॒ दश॑ च}]}%~(८)

%3.1.9.1
यो वै दे॒वान्दे॑वयश॒सेना॒र्पय॑ति मनु॒ष्या᳚न्मनुष्ययश॒सेन॑ देव\-यश॒स्ये॑व दे॒वेषु॒ भव॑ति मनुष्ययश॒सी म॑नु॒ष्ये॑षु॒ यान्प्रा॒चीन॑\-माग्रय॒णाद् ग्रहा᳚न्गृह्णी॒यात् तानु॑पा॒ꣳ॒शु गृ॑ह्णीया॒द्यानू॒र्ध्वाꣴस्तानु॑पब्दि॒मतो॑ दे॒वाने॒व तद्दे॑वयश॒सेना᳚र्पयति मनु॒ष्या᳚न्मनुष्ययश॒सेन॑ देवयश॒स्ये॑व दे॒वेषु॑ भवति मनुष्ययश॒सी म॑नु॒ष्ये᳚ष्व॒ग्निः प्रा॑तःसव॒ने पा᳚त्व॒स्मान् वै᳚श्वान॒रो म॑हि॒ना वि॒श्वश॑म्भूः। स नः॑ पाव॒को द्रवि॑णं दधा॒-~(२६)

%3.1.9.2
त्वायु॑ष्मन्तः स॒हभ॑क्षाः स्याम। विश्वे॑ दे॒वा म॒रुत॒ इन्द्रो॑ अ॒स्मान॒स्मिन्द्वि॒तीये॒ सव॑ने॒ न ज॑ह्युः। आयु॑ष्मन्तः प्रि॒यमे॑षां॒ वद॑न्तो व॒यं दे॒वानाꣳ॑ सुम॒तौ स्या॑म। इ॒दं तृ॒तीय॒ꣳ॒ सव॑नं कवी॒नामृ॒तेन॒ ये च॑म॒समैर॑यन्त। ते सौ॑धन्व॒नाः सुव॑रानशा॒नाः स्वि॑ष्टिं नो अ॒भि वसी॑यो नयन्तु। आ॒यत॑नवती॒र्वा अ॒न्या आहु॑तयो हू॒यन्ते॑\-ऽनायत॒ना अ॒न्या या आ॑घा॒रव॑ती॒स्ता आ॒यत॑नवती॒र्याः~(२७)

%3.1.9.3
सौ॒म्यास्ता अ॑नायत॒ना ऐ᳚न्द्रवाय॒वमा॒दाया॑ऽऽघा॒रमा घा॑रयेदध्व॒रो य॒ज्ञो॑\-ऽयम॑स्तु देवा॒ ओष॑धीभ्यः प॒शवे॑ नो॒ जना॑य॒ विश्व॑स्मै भू॒ताया᳚ध्व॒रो॑\-ऽसि॒ स पि॑न्वस्व घृ॒तव॑द्देव सो॒मेति॑ सौ॒म्या ए॒व तदाहु॑तीरा॒यत॑नवतीः करोत्या॒यत॑नवान्भवति॒ य ए॒वं वेदाथो॒ द्यावा॑पृथि॒वी ए॒व घृ॒तेन॒ व्यु॑नत्ति॒ ते व्यु॑त्ते उपजीव॒नीये॑ भवत उपजीव॒नीयो॑ भवति॒~(२८)

%3.1.9.4
य ए॒वं वेदै॒ष ते॑ रुद्र भा॒गो यं नि॒रया॑चथा॒स्तं जु॑षस्व वि॒देर्गौ॑प॒त्यꣳ रा॒यस्पोषꣳ॑ सु॒वीर्यꣳ॑ संवथ्स॒रीणाꣴ॑ स्व॒स्तिम्। मनुः॑ पु॒त्रेभ्यो॑ दा॒यं व्य॑भज॒थ्स नाभा॒नेदि॑ष्ठं ब्रह्म॒चर्यं॒ वस॑न्तं॒ निर॑भज॒थ्स आग॑च्छ॒थ्सो᳚\-ऽब्रवीत्क॒था मा॒ निर॑भा॒गिति॒ न त्वा॒ निर॑भाक्ष॒मित्य॑ब्रवी॒दङ्गि॑रस इ॒मे स॒त्तमा॑सते॒ ते~(२९)

%3.1.9.5
सु॑व॒र्गं लो॒कं न प्र जा॑नन्ति॒ तेभ्य॑ इ॒दं ब्राह्म॑णं ब्रूहि॒ ते सु॑व॒र्गं लो॒कं यन्तो॒ य ए॑षां प॒शव॒स्ताꣴस्ते॑ दास्य॒न्तीति॒ तदे᳚भ्यो\-ऽब्रवी॒त्ते सु॑व॒र्गं लो॒कं यन्तो॒ य ए॑षां प॒शव॒ आस॒न्तान॑स्मा अददु॒स्तं प॒शुभि॒श्चर॑न्तं यज्ञवा॒स्तौ रु॒द्र आग॑च्छ॒थ्सो᳚\-ऽब्रवी॒न्मम॒ वा इ॒मे प॒शव॒ इत्यदु॒र्वै~(३०)

%3.1.9.6
मह्य॑मि॒मानित्य॑ब्रवी॒न्न वै तस्य॒ त ई॑शत॒ इत्य॑ब्रवी॒द्यद्य॑ज्ञवा॒स्तौ हीय॑ते॒ मम॒ वै तदिति॒ तस्मा᳚द्यज्ञवा॒स्तु नाभ्य॒वेत्य॒ꣳ॒ सो᳚\-ऽब्रवीद्य॒ज्ञे मा भ॒जाथ॑ ते प॒शून्नाभि मꣴ॑स्य॒ इति॒ तस्मा॑ ए॒तं म॒न्थिनः॑ सꣴस्रा॒वम॑जुहो॒त्ततो॒ वै तस्य॑ रु॒द्रः प॒शून्नाभ्य॑मन्यत॒ यत्रै॒तमे॒वं वि॒द्वान्म॒न्थिनः॑ सꣴस्रा॒वं जु॒होति॒ न तत्र॑ रु॒द्रः प॒शून॒भि म॑न्यते॥~(३१)

%3.1.10.0
{\anuvakamend[{द॒धा॒त्वा॒यत॑नवती॒र्या उ॑पजीव॒नीयो॑ भवति॒ ते\-ऽदु॒र्वै यत्रै॒तमेका॑\-दश च}]}%~(९)

%3.1.10.1
जुष्टो॑ वा॒चो भू॑यासं॒ जुष्टो॑ वा॒चस्पत॑ये॒ देवि॑ वाक्। यद्वा॒चो मधु॑म॒त्तस्मि॑न्मा धाः॒ स्वाहा॒ सर॑स्वत्यै। ऋ॒चा स्तोम॒ꣳ॒ सम॑र्धय गाय॒त्रेण॑ रथन्त॒रम्। बृ॒हद्गा॑य॒त्रव॑र्तनि। यस्ते᳚ द्र॒फ्सः स्कन्द॑ति॒ यस्ते॑ अ॒ꣳ॒शुर्बा॒हुच्यु॑तो धि॒षण॑योरु॒पस्था᳚त्। अ॒ध्व॒र्योर्वा॒ परि॒ यस्ते॑ प॒वित्रा॒थ्\-स्वाहा॑कृत॒मिन्द्रा॑य॒ तं जु॑होमि। यो द्र॒फ्सो अ॒ꣳ॒शुः प॑ति॒तः पृ॑थि॒व्यां प॑रिवा॒पात्~(३२)

%3.1.10.2
पु॑रो॒डाशा᳚त्कर॒म्भात्। धा॒ना॒सो॒मान्म॒न्थिन॑ इन्द्र शु॒क्राथ्\-स्वाहा॑कृत॒मिन्द्रा॑य॒ तं जु॑होमि। यस्ते᳚ द्र॒फ्सो मधु॑माꣳ इन्द्रि॒यावा॒न्थ्\-स्वाहा॑कृतः॒ पुन॑र॒प्येति॑ दे॒वान्। दि॒वः पृ॑थि॒व्याः पर्य॒न्तरि॑क्षा॒थ्\-स्वाहा॑कृत॒मिन्द्रा॑य॒ तं जु॑होमि। अ॒ध्व॒र्युर्वा ऋ॒त्विजां᳚ प्रथ॒मो यु॑ज्यते॒ तेन॒ स्तोमो॑ योक्त॒व्य॑ इत्या॑हु॒र्वाग॑ग्रे॒गा अग्र॑ एत्वृजु॒गा दे॒वेभ्यो॒ यशो॒ मयि॒ दध॑ती प्रा॒णान्प॒शुषु॑ प्र॒जां मयि॑~(३३)

%3.1.10.3
च॒ यज॑माने॒ चेत्या॑ह॒ वाच॑मे॒व तद्य॑ज्ञमु॒खे यु॑नक्ति॒ वास्तु॒ वा ए॒तद्य॒ज्ञस्य॑ क्रियते॒ यद्ग्र॒हा᳚न्गृही॒त्वा ब॑हिष्पवमा॒नꣳ सर्प॑न्ति॒ परा᳚ञ्चो॒ हि यन्ति॒ परा॑चीभिः स्तु॒वते॑ वैष्ण॒व्यर्चा पुन॒रेत्योप॑ तिष्ठते य॒ज्ञो वै विष्णु॑र्य॒ज्ञमे॒वाक॒र्विष्णो॒ त्वं नो॒ अन्त॑मः॒ शर्म॑ यच्छ सहन्त्य। प्र ते॒ धारा॑ मधु॒श्चुत॒ उथ्सं॑ दुह्रते॒ अक्षि॑त॒मित्या॑ह॒ यदे॒वास्य॒ शया॑नस्योप॒शुष्य॑ति॒ तदे॒वास्यै॒तेना प्या॑ययति॥~(३४)

%3.1.11.0
{\anuvakamend[{प॒रि॒वा॒पात्प्र॒जां मयि॑ दुह्रते॒ चतु॑र्दश च}]}%॥10॥

%3.1.11.1
अ॒ग्निना॑ र॒यिम॑श्ञव॒त्पोष॑मे॒व दि॒वेदि॑वे। य॒शसं॑ वी॒रव॑त्तमम्॥ गोमाꣳ॑ अ॒ग्ने\-ऽवि॑माꣳ अ॒श्वी य॒ज्ञो नृ॒वथ्स॑खा॒ सद॒मिद॑प्रमृ॒ष्यः। इडा॑वाꣳ ए॒षो अ॑सुर प्र॒जावा᳚न्दी॒र्घो र॒यिः पृ॑थुबु॒ध्नः स॒भावान्॑॥ आ प्या॑यस्व॒ सं ते᳚॥ इ॒ह त्वष्टा॑रमग्रि॒यं वि॒श्वरू॑प॒मुप॑ ह्वये। अ॒स्माक॑मस्तु॒ केव॑लः॥ तन्न॑स्तु॒रीप॒मध॑ पोषयि॒त्नु देव॑ त्वष्ट॒र्वि र॑रा॒णः स्य॑स्व। यतो॑ वी॒रः~(३५)

%3.1.11.2
क॑र्म॒ण्यः॑ सु॒दक्षो॑ यु॒क्तग्रा॑वा॒ जाय॑ते दे॒वका॑मः। शि॒वस्त्व॑ष्टरि॒हा ग॑हि वि॒भुः पोष॑ उ॒त त्मना᳚। य॒ज्ञेय॑ज्ञे न॒ उद॑व। पि॒शङ्ग॑रूपः सु॒भरो॑ वयो॒धाः श्रु॒ष्टी वी॒रो जा॑यते दे॒वका॑मः। प्र॒जां त्वष्टा॒ वि ष्य॑तु॒ नाभि॑म॒स्मे अथा॑ दे॒वाना॒मप्ये॑तु॒ पाथः॑। प्र णो॑ दे॒व्या नो॑ दि॒वः। पी॒पि॒वाꣳस॒ꣳ॒ सर॑स्वतः॒ स्तनं॒ यो वि॒श्वद॑र्\mbox{}शतः। धु॒क्षी॒महि॑ प्र॒जामिषम्᳚।~(३६)

%3.1.11.3
ये ते॑ सरस्व ऊ॒र्मयो॒ मधु॑मन्तो घृत॒श्चुतः॑। तेषां᳚ ते सु॒म्नमी॑महे। यस्य॑ व्र॒तं प॒शवो॒ यन्ति॒ सर्वे॒ यस्य॑ व्र॒तमु॑प॒तिष्ठ॑न्त॒ आपः॑। यस्य॑ व्र॒ते पु॑ष्टि॒पति॒र्निवि॑ष्ट॒स्तꣳ सर॑स्वन्त॒मव॑से हुवेम। दि॒व्यꣳ सु॑प॒र्णं व॑य॒सं बृ॒हन्त॑म॒पां गर्भं॑ वृष॒भमोष॑धीनाम्। अ॒भी॒प॒तो वृ॒ष्ट्या त॒र्पय॑न्तं॒ तꣳ सर॑स्वन्त॒मव॑से हुवेम। सिनी॑वालि॒ पृथु॑ष्टुके॒ या दे॒वाना॒मसि॒ स्वसा᳚। जु॒षस्व॑ ह॒व्य-~(३७)

%3.1.11.4
माहु॑तं प्र॒जां दे॑वि दिदिड्ढि नः। या सु॑पा॒णिः स्व॑ङ्गु॒रिः सु॒षूमा॑ बहु॒सूव॑री। तस्यै॑ वि॒श्पत्नि॑यै ह॒विः सि॑नीवा॒ल्यै जु॑होतन। इन्द्रं॑ वो वि॒श्वत॒स्परीन्द्रं॒ नरः॑। असि॑तवर्णा॒ हर॑यः सुप॒र्णा मिहो॒ वसा॑ना॒ दिव॒मुत्प॑तन्ति। त आ\-ऽव॑वृत्र॒न्थ्सद॑नानि कृ॒त्वादित्पृ॑थि॒वी घृ॒तैर्व्यु॑द्यते। हिर॑ण्यकेशो॒ रज॑सो विसा॒रे\-ऽहि॒र्धुनि॒र्वात॑ इव॒ ध्रजी॑मान्। शुचि॑भ्राजा उ॒षसो॒~(३८)

%3.1.11.5
नवे॑दा॒ यश॑स्वतीरप॒स्युवो॒ न स॒त्याः। आ ते॑ सुप॒र्णा अ॑मिनन्त॒ एवैः᳚ कृ॒ष्णो नो॑नाव वृष॒भो यदी॒दम्। शि॒वाभि॒र्न स्मय॑\-माना\-भि॒रागा॒त्पत॑न्ति॒ मिहः॑ स्त॒नय॑न्त्य॒भ्रा। वा॒श्रेव॑ वि॒द्युन्मि॑माति व॒थ्सं न मा॒ता सि॑षक्ति। यदे॑षां वृ॒ष्टिरस॑र्जि। पर्व॑तश्चि॒न्महि॑ वृ॒द्धो बि॑भाय दि॒वश्चि॒थ्सानु॑ रेजत स्व॒ने वः॑। यत्क्रीड॑थ मरुत~-~(३९)

%3.1.11.6
ऋष्टि॒मन्त॒ आप॑ इव स॒ध्रिय॑ञ्चो धवध्वे। अ॒भि क्र॑न्द स्त॒नय॒ गर्भ॒मा धा॑ उद॒न्वता॒ परि॑ दीया॒ रथे॑न। दृति॒ꣳ॒ सु क॑र्\mbox{}ष॒ विषि॑तं॒ न्य॑ञ्चꣳ स॒मा भ॑वन्तू॒द्वता॑ निपा॒दाः। त्वं त्या चि॒दच्यु॒ताग्ने॑ प॒शुर्न यव॑से। धामा॑ ह॒ यत्ते॑ अजर॒ वना॑ वृ॒श्चन्ति॒ शिक्व॑सः। अग्ने॒ भूरी॑णि॒ तव॑ जातवेदो॒ देव॑ स्वधावो॒\-ऽमृत॑स्य॒ धाम॑। याश्च॑~(४०)

%3.1.11.7
मा॒या मा॒यिनां᳚ विश्वमिन्व॒ त्वे पू॒र्वीः स॑न्द॒धुः पृ॑ष्टबन्धो। दि॒वो नो॑ वृ॒ष्टिं म॑रुतो ररीध्वं॒ प्र पि॑न्वत॒ वृष्णो॒ अश्व॑स्य॒ धाराः᳚। अ॒र्वाङे॒तेन॑ स्तनयि॒त्नुनेह्य॒पो नि॑षि॒ञ्चन्नसु॑रः पि॒ता नः॑। पिन्व॑न्त्य॒पो म॒रुतः॑ सु॒\-दान॑वः॒ पयो॑ घृ॒तव॑द्वि॒दथे᳚ष्वा॒भुवः॑। अत्यं॒ न मि॒हे वि न॑यन्ति वा॒जिन॒\-मुथ्सं॑ दुहन्ति स्त॒नय॑न्त॒मक्षि॑तम्। उ॒द॒प्रुतो॑ मरुत॒स्ताꣳ इ॑यर्त॒ वृष्टिं॒~(४१)

%3.1.11.8
ये विश्वे॑ म॒रुतो॑ जु॒नन्ति॑। क्रोशा॑ति॒ गर्दा॑ क॒न्ये॑व तु॒न्ना पेरुं॑ तुञ्जा॒ना पत्ये॑व जा॒या। घृ॒तेन॒ द्यावा॑पृथि॒वी मधु॑ना॒ समु॑क्षत॒ पय॑स्वतीः कृणु॒ताऽऽप॒ ओष॑धीः। ऊर्जं॑ च॒ तत्र॑ सुम॒तिं च॑ पिन्वथ॒ यत्रा॑ नरो मरुतः सि॒ञ्चथा॒ मधु॑। उदु॒ त्यं चि॒त्रम्। औ॒र्व॒भृ॒गु॒वच्छुचि॑मप्नवान॒वदा हु॑वे। अ॒ग्निꣳ स॑मु॒द्रवा॑ससम्। आ स॒वꣳ स॑वि॒तुर्य॑था॒ भग॑स्येव भु॒जिꣳ हु॑वे। अ॒ग्निꣳ स॑मु॒द्रवा॑ससम्। हु॒वे वात॑स्वनं क॒विं प॒र्जन्य॑क्रन्द्य॒ꣳ॒ सहः॑। अ॒ग्निꣳ स॑मु॒द्रवा॑ससम्॥~(४२)

{\anuvakamend[{वी॒र इषꣳ॑ ह॒व्यमु॒षसो॑ मरुतश्च॒ वृष्टिं॒ भग॑स्य॒ द्वाद॑श च}]}%॥11॥

{\prashnaend[यो वै पव॑मानाना॒न्त्रीणि॑ परि॒भूः स्फ्यः स्व॒स्तिर्भक्षेहि॑ मही॒नां पयो॑\-ऽसि॒ देव॑ सवितरे॒तत्ते᳚ श्ये॒नाय॒ यद्वै होतो॑पयाम॒गृ॑हीतो\-ऽसि वाक्ष॒सत्प्र सो अ॑ग्न॒ एका॑\-दश। प्र॒जा\-प॑तिरकामयत प्र॒जा\-प॑ते॒र्जाय॑माना॒ व्याय॑च्छन्ते॒ मह्य॑मि॒मान्मा॒या मा॒यिनां॒ द्विच॑त्वारिꣳशत्॥42॥ प्र॒जा\-प॑तिरकामयता॒ग्निꣳ स॑मु॒द्रवा॑ससम्॥]}
%%% END PRASHNA

\sect{द्वितीयः प्रश्नः}\setcounter{anuvakam}{0}
%3.2.0.0
\dnsub{तैत्तिरीयसंहितायां तृतीयकाण्डे द्वितीयः प्रश्नः}
%3.2.1.0
%3.2.1.1
यो वै पव॑मानानामन्वारो॒हान् वि॒द्वान् यज॒ते\-ऽनु॒ पव॑माना॒ना रो॑हति॒ न पव॑माने॒भ्यो\-ऽव॑च्छिद्यते श्ये॒नो॑\-ऽसि गाय॒त्रछ॑न्दा॒ अनु॒ त्वा र॑भे स्व॒स्ति मा॒ सं पा॑रय सुप॒र्णो॑\-ऽसि त्रि॒ष्टुप्छ॑न्दा॒ अनु॒ त्वा र॑भे स्व॒स्ति मा॒ सं पा॑रय॒ सघा॑सि॒ जग॑तीछन्दा॒ अनु॒ त्वा र॑भे स्व॒स्ति मा॒ सं पा॑र॒येत्या॑है॒ते~(१)

%3.2.1.2
वै पव॑मानानामन्वारो॒हास्तान् य ए॒वं वि॒द्वान् यज॒ते\-ऽनु॒ पव॑माना॒ना रो॑हति॒ न पव॑माने॒भ्यो\-ऽव॑च्छिद्यते॒ यो वै पव॑मानस्य॒ सन्त॑तिं॒ वेद॒ सर्व॒मायु॑रेति॒ न पु॒राऽऽयु॑षः॒ प्र मी॑यते पशु॒मान्भ॑वति वि॒न्दते᳚ प्र॒जां पव॑मानस्य॒ ग्रहा॑ गृह्य॒न्ते\-ऽथ॒ वा अ॑स्यै॒ते\-ऽगृ॑हीता द्रोणकल॒श आ॑धव॒नीयः॑ पूत॒भृत्तान् यदगृ॑हीत्वोपाकु॒र्यात्पव॑मानं॒ वि~(२)

%3.2.1.3
च्छि॑न्द्या॒त् तं वि॒च्छिद्य॑मानमध्व॒र्योः प्रा॒णो\-ऽनु॒ विच्छि॑द्येतोप\-या॒मगृ॑हीतो\-ऽसि प्र॒जा\-प॑तये॒ त्वेति॑ द्रोणकल॒शम॒भि मृ॑शे॒दिन्द्रा॑य॒ त्वेत्या॑धव॒नीयं॒ विश्वे᳚भ्यस्त्वा दे॒वेभ्य॒ इति॑ पूत॒भृतं॒ पव॑मानमे॒व तथ्सं त॑नोति॒ सर्व॒मायु॑रेति॒ न पु॒राऽऽयु॑षः॒ प्र मी॑यते पशु॒मान्भ॑वति वि॒न्दते᳚ प्र॒जाम्॥~(३)

%3.2.2.0
{\anuvakamend[{ए॒ते वि द्विच॑त्वारिꣳशच्च}]}%~(१)

%3.2.2.1
त्रीणि॒ वाव सव॑ना॒न्यथ॑ तृ॒तीय॒ꣳ॒ सव॑न॒मव॑ लुम्पन्त्यन॒ꣳ॒शु कु॒र्वन्त॑ उपा॒ꣳ॒शुꣳ हु॒त्वोपाꣳ॑शुपा॒त्रे\-ऽꣳ॑शुम॒वास्य॒ तं तृ॑तीयसव॒ने॑\-ऽपि॒सृज्या॒भि षु॑णुया॒द्यदा᳚प्या॒यय॑ति॒ तेनाꣳ॑शु॒मद्यद॑भिषु॒णोति॒ तेन॑र्जी॒षि सर्वा᳚ण्ये॒व तथ्सव॑नान्यꣳशु॒मन्ति॑ शु॒क्रव॑न्ति स॒माव॑द्वीर्याणि करोति॒ द्वौ स॑मु॒द्रौ वित॑तावजू॒र्यौ प॒र्याव॑र्तेते ज॒ठरे॑व॒ पादाः᳚। तयोः॒ पश्य॑न्तो॒ अति॑ यन्त्य॒न्यमप॑श्यन्तः॒~(४)

%3.2.2.2
सेतु॒नाऽति॑ यन्त्य॒न्यम्। द्वे द्रध॑सी स॒तती॑ वस्त॒ एकः॑ के॒शी विश्वा॒ भुव॑नानि वि॒द्वान्। ति॒रो॒धायै॒त्यसि॑तं॒ वसा॑नः शु॒क्रमा द॑त्ते अनु॒हाय॑ जा॒र्यै। दे॒वा वै यद्य॒ज्ञे\-ऽकु॑र्वत॒ तदसु॑रा अकुर्वत॒ ते दे॒वा ए॒तं म॑हाय॒ज्ञम॑पश्य॒न्तम॑तन्वता\-ऽ\-ग्निहो॒त्रं व्र॒तम॑कुर्वत॒ तस्मा॒द् द्विव्र॑तः स्या॒द् द्विर्\mbox{}ह्य॑ग्निहो॒त्रं जुह्व॑ति पौर्णमा॒सं य॒ज्ञम॑ग्नीषो॒मीयं॑~(५)

%3.2.2.3
प॒शुम॑कुर्वत दा॒र्श्यं य॒ज्ञमा᳚ग्ने॒यं प॒शुम॑कुर्वत वैश्वदे॒वं प्रा॑तःसव॒नम॑कुर्वत वरुणप्रघा॒सान्माध्य॑न्दिन॒ꣳ॒ सव॑नꣳ साक\-मे॒धान्पि॑तृय॒ज्ञं त्र्य॑म्बकाꣴस्तृतीयसव॒नम॑कुर्वत॒ तमे॑षा॒मसु॑रा य॒ज्ञम॒न्व\-वा॑जिगाꣳ\-स॒न्तं नान्ववा॑य॒न्ते᳚\-ऽब्रुवन्नध्वर्त॒व्या वा इ॒मे दे॒वा अ॑भूव॒न्निति॒ तद॑ध्व॒रस्या᳚ध्वर॒त्वं ततो॑ दे॒वा अभ॑व॒न्परासु॑रा॒ य ए॒वं वि॒द्वान्थ्सोमे॑न॒ यज॑ते॒ भव॑त्या॒त्मना॒ परा᳚स्य॒ भ्रातृ॑व्यो भवति॥~(६)

%3.2.3.0
{\anuvakamend[{अप॑श्यन्तो\-ऽग्नीषो॒मीय॑मा॒त्मना॒ परा॒ त्रीणि॑ च}]}%~(२)

%3.2.3.1
प॒रि॒भूर॒ग्निं प॑रि॒भूरिन्द्रं॑ परि॒भूर्विश्वा᳚न् दे॒वान्प॑रि॒भूर्माꣳ स॒ह ब्र॑ह्मवर्च॒सेन॒ स नः॑ पवस्व॒ शं गवे॒ शं जना॑य॒ शमर्व॑ते॒ शꣳ रा॑ज॒न्नोष॑धी॒भ्यो\-ऽच्छि॑न्नस्य ते रयिपते सु॒वीर्य॑स्य रा॒यस्पोष॑स्य ददि॒तारः॑ स्याम। तस्य॑ मे रास्व॒ तस्य॑ ते भक्षीय॒ तस्य॑ त इ॒दमुन्मृ॑जे। प्रा॒णाय॑ मे वर्चो॒दा वर्च॑से पवस्वापा॒नाय॑ व्या॒नाय॑ वा॒चे~(७)

%3.2.3.2
द॑क्षक्र॒तुभ्यां॒ चक्षु॑र्भ्यां मे वर्चो॒दौ वर्च॑से पवेथा॒ꣴ॒ श्रोत्रा॑या॒\-ऽऽ\-त्मने\-ऽङ्गे᳚भ्य॒ आयु॑षे वी॒र्या॑य॒ विष्णो॒रिन्द्र॑स्य॒ विश्वे॑षां दे॒वानां᳚ ज॒ठर॑मसि वर्चो॒दा मे॒ वर्च॑से पवस्व॒ को॑\-ऽसि॒ को नाम॒ कस्मै᳚ त्वा॒ काय॑ त्वा॒ यं त्वा॒ सोमे॒नाती॑तृपं॒ यं त्वा॒ सोमे॒नामी॑मदꣳ सुप्र॒जाः प्र॒जया॑ भूयासꣳ सु॒वीरो॑ वी॒रैः सु॒वर्चा॒ वर्च॑सा सु॒पोषः॒ पोषै॒र्विश्वे᳚भ्यो मे रू॒पेभ्यो॑ वर्चो॒दा~-~(८)

%3.2.3.3
वर्च॑से पवस्व॒ तस्य॑ मे रास्व॒ तस्य॑ ते भक्षीय॒ तस्य॑ त इ॒दमुन्मृ॑जे। बुभू॑ष॒न्नवे᳚क्षेतै॒ष वै पात्रि॑यः प्र॒जा\-प॑तिर्य॒ज्ञः प्र॒जा\-प॑ति॒स्तमे॒व त॑र्पयति॒ स ए॑नं तृ॒प्तो भूत्या॒\-ऽभि प॑वते ब्रह्मवर्च॒सका॒मो\-ऽवे᳚क्षेतै॒ष वै पात्रि॑यः प्र॒जा\-प॑तिर्य॒ज्ञः प्र॒जा\-प॑ति॒स्तमे॒व त॑र्पयति॒ स ए॑नं तृ॒प्तो ब्र॑ह्मवर्च॒सेना॒भि प॑वत आमया॒व्य-~(९)

%3.2.3.4
वे᳚क्षेतै॒ष वै पात्रि॑यः प्र॒जा\-प॑तिर्य॒ज्ञः प्र॒जा\-प॑ति॒स्तमे॒व त॑र्पयति॒ स ए॑नं तृ॒प्त आयु॑षा॒ऽभि प॑वते\-ऽभि॒चर॒न्नवे᳚क्षेतै॒ष वै पात्रि॑यः प्र॒जा\-प॑तिर्य॒ज्ञः प्र॒जा\-प॑ति॒स्तमे॒व त॑र्पयति॒ स ए॑नं तृ॒प्तः प्रा॑णापा॒ना\-भ्यां᳚ वा॒चो द॑क्षक्र॒तुभ्यां॒ चक्षु॑र्भ्या॒ꣴ॒ श्रोत्रा᳚भ्यामा॒त्मनो\-ऽङ्गे᳚भ्य॒ आयु॑षो॒\-ऽन्तरे॑ति ता॒जक्प्र ध॑न्वति॥~(१०)

%3.2.4.0
{\anuvakamend[{वा॒चे रू॒पेभ्यो॑ वर्चो॒दा आ॑मया॒वी पञ्च॑चत्वारिꣳशच्च}]}%~(३)

%3.2.4.1
स्फ्यः स्व॒स्तिर्वि॑घ॒नः स्व॒स्तिः पर्\mbox{}शु॒र्वेदिः॑ पर॒शुर्नः॑ स्व॒स्तिः। य॒ज्ञिया॑ यज्ञ॒कृतः॑ स्थ॒ ते मा॒ऽस्मिन् य॒ज्ञ उप॑ ह्वयध्व॒मुप॑ मा॒ द्यावा॑पृथि॒वी ह्व॑येता॒मुपा᳚स्ता॒वः क॒लशः॒ सोमो॑ अ॒ग्निरुप॑ दे॒वा उप॑ य॒ज्ञ उप॑ मा॒ होत्रा॑ उपह॒वे ह्व॑यन्तां॒ नमो॒\-ऽग्नये॑ मख॒घ्ने म॒खस्य॑ मा॒ यशो᳚\-ऽर्या॒दित्या॑हव॒नीय॒मुप॑ तिष्ठते य॒ज्ञो वै म॒खो~(११)

%3.2.4.2
य॒ज्ञं वाव स तद॑ह॒न्तस्मा॑ ए॒व न॑म॒स्कृत्य॒ सदः॒ प्र स॑र्पत्या॒त्मनो\-ऽना᳚र्त्यै॒ नमो॑ रु॒द्राय॑ मख॒घ्ने नम॑स्कृत्या मा पा॒हीत्याग्नी᳚ध्रं॒ तस्मा॑ ए॒व न॑म॒स्कृत्य॒ सदः॒ प्र स॑र्पत्या॒त्मनो\-ऽना᳚र्त्यै॒ नम॒ इन्द्रा॑य मख॒घ्न इ॑न्द्रि॒यं मे॑ वी॒र्यं॑ मा निर्व॑धी॒रिति॑ हो॒त्रीय॑मा॒शिष॑मे॒वैतामा शा᳚स्त इन्द्रि॒यस्य॑ वी॒र्य॑स्यानि॑र्घाताय॒ या वै~(१२)

%3.2.4.3
दे॒वताः॒ सद॒स्यार्ति॑मा॒र्पय॑न्ति॒ यस्ता वि॒द्वान्प्र॒सर्प॑ति॒ न सद॒स्यार्ति॒मार्च्छ॑ति॒ नमो॒\-ऽग्नये॑ मख॒घ्न इत्या॑है॒ता वै दे॒वताः॒ सद॒स्यार्ति॒मार्प॑यन्ति॒ ता य ए॒वं वि॒द्वान्प्र॒सर्प॑ति॒ न सद॒स्यार्ति॒मार्च्छ॑ति दृ॒ढे स्थः॑ शिथि॒रे स॒मीची॒ माꣳह॑सस्पात॒ꣳ॒ सूर्यो॑ मा दे॒वो दि॒व्यादꣳह॑सस्पातु वा॒युर॒न्तरि॑क्षा-~(१३)

%3.2.4.4
द॒ग्निः पृ॑थि॒व्या य॒मः पि॒तृभ्यः॒ सर॑स्वती मनु॒ष्ये᳚भ्यो॒ देवी᳚ द्वारौ॒ मा मा॒ सन्ता᳚प्तं॒ नमः॒ सद॑से॒ नमः॒ सद॑स॒स्पत॑ये॒ नमः॒ सखी॑नां पुरो॒गाणां॒ चक्षु॑षे॒ नमो॑ दि॒वे नमः॑ पृथि॒व्या अहे॑ दैधिष॒व्योदत॑स्तिष्ठा॒न्यस्य॒ सद॑ने सीद॒ यो᳚\-ऽस्मत्पाक॑तर॒ उन्नि॒वत॒ उदु॒द्वत॑श्च गेषं पा॒तं मा᳚ द्यावा\-पृथिवी अ॒द्याह्नः॒ सदो॒ वै प्र॒सर्प॑न्तं~(१४)

%3.2.4.5
पि॒तरो\-ऽनु॒ प्र स॑र्पन्ति॒ त ए॑नमीश्व॒रा हिꣳसि॑तोः॒ सदः॑ प्र॒सृप्य॑ दक्षिणा॒र्धं परे᳚क्षे॒ताग॑न्त पितरः पितृ॒मान॒हं यु॒ष्माभि॑र्भूयासꣳ सुप्र॒जसो॒ मया॑ यू॒यं भू॑या॒स्तेति॒ तेभ्य॑ ए॒व न॑म॒स्कृत्य॒ सदः॒ प्र स॑र्पत्या॒त्मनो\-ऽना᳚र्त्यै॥~(१५)

%3.2.5.0
{\anuvakamend[{म॒खो वा अ॒न्तरि॑क्षात्प्र॒सर्प॑न्त॒न्त्रय॑स्त्रिꣳशच्च}]}%~(४)

%3.2.5.1
भक्षेहि॒ मा वि॑श दीर्घायु॒त्वाय॑ शन्तनु॒त्वाय॑ रा॒यस्पोषा॑य॒ वर्च॑से सुप्रजा॒स्त्वायेहि॑ वसो पुरोवसो प्रि॒यो मे॑ हृ॒दो᳚\-ऽस्य॒श्विनो᳚स्त्वा बा॒हुभ्याꣳ॑ सघ्यासं नृ॒चक्ष॑सं त्वा देव सोम सु॒चक्षा॒ अव॑ ख्येषं म॒न्द्राभिभू॑तिः के॒तुर्य॒ज्ञानां॒ वाग्जु॑षा॒णा सोम॑स्य तृप्यतु म॒न्द्रा स्व॑र्वा॒च्यदि॑ति॒रना॑हतशीर्ष्णी॒ वाग्जु॑षा॒णा सोम॑स्य तृप्य॒त्वेहि॑ विश्वचर्\mbox{}षणे~(१६)

%3.2.5.2
श॒म्भूर्म॑यो॒भूः स्व॒स्ति मा॑ हरिवर्ण॒ प्र च॑र॒ क्रत्वे॒ दक्षा॑य रा॒यस्पोषा॑य सुवी॒रता॑यै॒ मा मा॑ राज॒न्वि बी॑भिषो॒ मा मे॒ हार्दि॑ त्वि॒षा व॑धीः। वृष॑णे॒ शुष्मा॒याऽऽयु॑षे॒ वर्च॑से॥ वसु॑मद्गणस्य सोम देव ते मति॒विदः॑ प्रातःसव॒नस्य॑ गाय॒त्रछ॑न्दस॒ इन्द्र॑पीतस्य॒ नरा॒शꣳस॑पीतस्य पि॒तृपी॑तस्य॒ मधु॑मत॒ उप॑हूत॒स्योप॑हूतो भक्षयामि रु॒द्रव॑द्गणस्य सोम देव ते मति॒विदो॒ माध्य॑न्दिनस्य॒ सव॑नस्य त्रि॒ष्टुप्छ॑न्दस॒ इन्द्र॑पीतस्य॒ नरा॒शꣳस॑पीतस्य~(१७)

%3.2.5.3
पि॒तृपी॑तस्य॒ मधु॑मत॒ उप॑हूत॒स्योप॑हूतो भक्षयाम्यादि॒त्यव॑द्गणस्य सोम देव ते मति॒विद॑स्तृ॒तीय॑स्य॒ सव॑नस्य॒ जग॑तीछन्दस॒ इन्द्र॑पीतस्य॒ नरा॒शꣳस॑पीतस्य पि॒तृपी॑तस्य॒ मधु॑मत॒ उप॑हूत॒स्योप॑हूतो भक्षयामि। आ प्या॑यस्व॒ समे॑तु ते वि॒श्वतः॑ सोम॒ वृष्णि॑यम्। भवा॒ वाज॑स्य सङ्ग॒थे। हिन्व॑ मे॒ गात्रा॑ हरिवो ग॒णान्मे॒ मा वि ती॑तृषः। शि॒वो मे॑ सप्त॒र्॒\mbox{}षीनुप॑ तिष्ठस्व॒ मा मे\-ऽवा॒ङ्नाभि॒मति॑~(१८)

%3.2.5.4
गाः। अपा॑म॒ सोम॑म॒मृता॑ अभू॒माद॑र्श्म॒ ज्योति॒रवि॑दाम दे॒वान्। किम॒स्मान्कृ॑णव॒दरा॑तिः॒ किमु॑ धू॒र्तिर॑मृत॒ मर्त्य॑स्य। यन्म॑ आ॒त्मनो॑ मि॒न्दाभू॑द॒ग्निस्तत्पुन॒राहा᳚र्जा॒तवे॑दा॒ विच॑र्\mbox{}षणिः। पुन॑र॒ग्निश्चक्षु॑रदा॒त्पुन॒रिन्द्रो॒ बृह॒स्पतिः॑। पुन॑र्मे अश्विना यु॒वं चक्षु॒रा ध॑त्तम॒क्ष्योः। इ॒ष्टय॑जुषस्ते देव सोम स्तु॒तस्तो॑मस्य~(१९)

%3.2.5.5
श॒स्तोक्थ॑स्य॒ हरि॑वत॒ इन्द्र॑पीतस्य॒ मधु॑मत॒ उप॑हूत॒स्योप॑हूतो भक्षयामि। आ॒पूर्याः॒ स्था मा॑ पूरयत प्र॒जया॑ च॒ धने॑न च। ए॒तत्ते॑ तत॒ ये च॒ त्वामन्वे॒तत्ते॑ पितामह प्रपितामह॒ ये च॒ त्वामन्वत्र॑ पितरो यथाभा॒गं म॑न्दध्वं॒ नमो॑ वः पितरो॒ रसा॑य॒ नमो॑ वः पितरः॒ शुष्मा॑य॒ नमो॑ वः पितरो जी॒वाय॒ नमो॑ वः पितरः~(२०)

%3.2.5.6
स्व॒धायै॒ नमो॑ वः पितरो म॒न्यवे॒ नमो॑ वः पितरो घो॒राय॒ पित॑रो॒ नमो॑ वो॒ य ए॒तस्मिँ॑ल्लो॒के स्थ यु॒ष्माꣴस्ते\-ऽनु॒ ये᳚\-ऽस्मिँल्लो॒के मां ते\-ऽनु॒ य ए॒तस्मिँ॑ल्लो॒के स्थ यू॒यं तेषां॒ वसि॑ष्ठा भूयास्त॒ ये᳚\-ऽस्मिँल्लो॒के॑\-ऽहं तेषां॒ वसि॑ष्ठो भूयासं॒ प्रजा॑पते॒ न त्वदे॒तान्य॒न्यो विश्वा॑ जा॒तानि॒ परि॒ ता ब॑भूव।~(२१)

%3.2.5.7
यत्का॑मास्ते जुहु॒मस्तन्नो॑ अस्तु व॒यꣴ स्या॑म॒ पत॑यो रयी॒णाम्। दे॒वकृ॑त॒स्यैन॑सो\-ऽव॒यज॑नमसि मनु॒ष्य॑कृत॒स्यैन॑सो\-ऽ\-व॒यज॑नमसि पि॒तृकृ॑त॒स्यैन॑सो\-ऽव॒यज॑नमस्य॒फ्सु धौ॒तस्य॑ सोम देव ते॒ नृभिः॑ सु॒तस्ये॒ष्टय॑जुषः स्तु॒तस्तो॑मस्य श॒स्तोक्थ॑स्य॒ यो भ॒क्षो अ॑श्व॒सनि॒र्यो गो॒सनि॒स्तस्य॑ ते पि॒तृभि॑र्भ॒क्षं कृ॑त॒स्योप॑हूत॒स्योप॑हूतो भक्षयामि॥~(२२)

%3.2.6.0
{\anuvakamend[{वि॒श्व॒च॒र्॒\mbox{}ष॒णे॒ त्रि॒ष्टुफ्छ॑न्दस॒ इन्द्र॑पीतस्य॒ नरा॒शꣳस॑पीत॒स्याति॑ स्तु॒तस्तो॑मस्य जी॒वाय॒ नमो॑ वः पितरो बभूव॒ चतु॑श्चत्वारिꣳशच्च}]}%~(५)

%3.2.6.1
म॒ही॒नां पयो॑\-ऽसि॒ विश्वे॑षां दे॒वानां᳚ त॒नूर्\mbox{}ऋ॒ध्यास॑म॒द्य पृ॑षतीनां॒ ग्रहं॒ पृष॑तीनां॒ ग्रहो॑\-ऽसि॒ विष्णो॒र्॒\mbox{}हृद॑यम॒स्येक॑मिष॒ विष्णु॒स्त्वाऽनु॒ वि च॑क्रमे भू॒तिर्द॒ध्ना घृ॒तेन॑ वर्धतां॒ तस्य॑ मे॒ष्टस्य॑ वी॒तस्य॒ द्रवि॑ण॒मा ग॑म्या॒ज्ज्योति॑रसि वैश्वान॒रं पृश्ञि॑यै दु॒ग्धं याव॑ती॒ द्यावा॑पृथि॒वी म॑हि॒त्वा याव॑च्च स॒प्त सिन्ध॑वो वित॒स्थुः। ताव॑न्तमिन्द्र ते॒~(२३)

%3.2.6.2
ग्रहꣳ॑ स॒होर्जा गृ॑ह्णा॒म्यस्तृ॑तम्। यत्कृ॑ष्णशकु॒नः पृ॑षदा॒ज्यम॑व\-मृ॒शेच्छू॒द्रा अ॑स्य प्र॒मायु॑काः स्यु॒र्यच्छ्वा\-ऽव॑मृ॒शेच्चतु॑ष्पादो\-ऽस्य प॒शवः॑ प्र॒मायु॑काः स्यु॒र्यथ्स्कन्दे॒द्यज॑मानः प्र॒मायु॑कः स्यात्प॒शवो॒ वै पृ॑षदा॒ज्यं प॒शवो॒ वा ए॒तस्य॑ स्कन्दन्ति॒ यस्य॑ पृषदा॒ज्यꣴ स्कन्द॑ति॒ यत्पृ॑षदा॒ज्यं पुन॑र्गृ॒ह्णाति॑ प॒शूने॒वास्मै॒ पुन॑र्गृह्णाति प्रा॒णो वै पृ॑षदा॒ज्यं प्रा॒णो वा~(२४)

%3.2.6.3
ए॒तस्य॑ स्कन्दति॒ यस्य॑ पृषदा॒ज्यꣴ स्कन्द॑ति॒ यत्पृ॑षदा॒ज्यं पुन॑र्गृ॒ह्णाति॑ प्रा॒णमे॒वास्मै॒ पुन॑र्गृह्णाति॒ हिर॑ण्यमव॒धाय॑ गृह्णात्य॒मृतं॒ वै हिर॑ण्यं प्रा॒णः पृ॑षदा॒ज्यम॒मृत॑मे॒वास्य॑ प्रा॒णे द॑धाति श॒तमा॑नं भवति श॒तायुः॒ पुरु॑षः श॒तेन्द्रि॑य॒ आयु॑ष्ये॒वेन्द्रि॒ये प्रति॑ तिष्ठ॒त्यश्व॒मव॑ घ्रापयति प्राजाप॒त्यो वा अश्वः॑ प्राजाप॒त्यः प्रा॒णः स्वादे॒वास्मै॒ योनेः᳚ प्रा॒णं निर्मि॑मीते॒ वि वा ए॒तस्य॑ य॒ज्ञश्छि॑द्यते॒ यस्य॑ पृषदा॒ज्यꣴ स्कन्द॑ति वैष्ण॒व्यर्चा पुन॑र्गृह्णाति य॒ज्ञो वै विष्णु॑र्य॒ज्ञेनै॒व य॒ज्ञꣳ सं त॑नोति॥~(२५)

%3.2.7.0
{\anuvakamend[{ते॒ पृ॒ष॒दा॒ज्यं प्रा॒णो वै योनेः᳚ प्रा॒णं द्वाविꣳ॑शतिश्च}]}%~(६)

%3.2.7.1
देव॑ सवितरे॒तत्ते॒ प्राऽऽह॒ तत्प्र च॑ सु॒व प्र च॑ यज॒ बृह॒स्पति॑र्ब्र॒ह्माऽऽयु॑ष्मत्या ऋ॒चो मा गा॑त तनू॒पाथ्साम्नः॑ स॒त्या व॑ आ॒शिषः॑ सन्तु स॒त्या आकू॑तय ऋ॒तं च॑ स॒त्यं च॑ वदत स्तु॒त दे॒वस्य॑ सवि॒तुः प्र॑स॒वे स्तु॒तस्य॑ स्तु॒तम॒स्यूर्जं॒ मह्यꣴ॑ स्तु॒तं दु॑हा॒मा मा᳚ स्तु॒तस्य॑ स्तु॒तं ग॑म्याच्छ॒स्त्रस्य॑ श॒स्त्र-~(२६)

%3.2.7.2
म॒स्यूर्जं॒ मह्यꣳ॑ श॒स्त्रं दु॑हा॒मा मा॑ श॒स्त्रस्य॑ श॒स्त्रं ग॑म्यादिन्द्रि॒याव॑न्तो वनामहे धुक्षी॒महि॑ प्र॒जामिषम्᳚। सा मे॑ स॒त्याशीर्दे॒वेषु॑ भूयाद् ब्रह्मवर्च॒सं मा ग॑म्यात्। य॒ज्ञो ब॑भूव॒ स आ ब॑भूव॒ स प्र ज॑ज्ञे॒ स वा॑वृधे। स दे॒वाना॒मधि॑पतिर्बभूव॒ सो अ॒स्माꣳ अधि॑पतीन्करोतु व॒यꣴ स्या॑म॒ पत॑यो रयी॒णाम्। य॒ज्ञो वा॒ वै~(२७)

%3.2.7.3
य॒ज्ञप॑तिं दु॒हे य॒ज्ञप॑तिर्वा य॒ज्ञं दु॑हे॒ स यः स्तु॑तश॒स्त्रयो॒र्दोह॒म\-वि॑द्वा॒न्॒ यज॑ते॒ तं य॒ज्ञो दु॑हे॒ स इ॒ष्ट्वा पापी॑यान्भवति॒ य ए॑नयो॒र्दोहं॑ वि॒द्वान् यज॑ते॒ स य॒ज्ञं दु॑हे॒ स इ॒ष्ट्वा वसी॑यान्भवति स्तु॒तस्य॑ स्तु॒तम॒स्यूर्जं॒ मह्यꣴ॑ स्तु॒तं दु॑हा॒मा मा᳚ स्तु॒तस्य॑ स्तु॒तं ग॑म्याच्छ॒स्त्रस्य॑ श॒स्त्रम॒स्यूर्जं॒ मह्यꣳ॑ श॒स्त्रं दु॑हा॒मा मा॑ श॒स्त्रस्य॑ श॒स्त्रं ग॑म्या॒दित्या॑है॒ष वै स्तु॑तश॒स्त्रयो॒र्दोह॒स्तं य ए॒वं वि॒द्वान् यज॑ते दु॒ह ए॒व य॒ज्ञमि॒ष्ट्वा वसी॑यान्भवति॥~(२८)

%3.2.8.0
{\anuvakamend[{श॒स्त्रं वै श॒स्त्रन्दु॑हा॒न्द्वाविꣳ॑शतिश्च}]}%~(७)

%3.2.8.1
श्ये॒नाय॒ पत्व॑ने॒ स्वाहा॒ वट्थ्स्व॒य\-म॑भि\-गूर्ताय॒\\
नमो॑ विष्ट॒म्भाय॒ धर्म॑णे॒ स्वाहा॒ वट्थ्स्व॒य\-म॑भि\-गूर्ताय॒\\
नमः॑ परि॒धये॑ जन॒प्रथ॑नाय॒ स्वाहा॒ वट्थ्स्व॒य\-म॑भि\-गूर्ताय॒\\
नम॑ ऊ॒र्जे होत्रा॑णा॒ꣴ॒ स्वाहा॒ वट्थ्स्व॒य\-म॑भि\-गूर्ताय॒\\
नमः॒ पय॑से॒ होत्रा॑णा॒ꣴ॒ स्वाहा॒ वट्थ्स्व॒य\-म॑भि\-गूर्ताय॒\\
नमः॑ प्र॒जा\-प॑तये॒ मन॑वे॒ स्वाहा॒ वट्थ्स्व॒य\-म॑भि\-गूर्ताय॒\\
नम॑ ऋ॒तमृ॑तपाः सुवर्वा॒ट्थ्\-स्वाहा॒ वट्थ्स्व॒य\-म॑भि\-गूर्ताय॒\\
नम॑स्तृ॒म्पन्ता॒ꣳ॒ होत्रा॒ मधो᳚र्घृ॒तस्य॑ य॒ज्ञप॑ति॒मृष॑य॒ एन॑सा~(२९)

%3.2.8.2
ऽऽहुः। प्र॒जा निर्भ॑क्ता अनुत॒प्यमा॑ना मध॒व्यौ᳚ स्तो॒कावप॒ तौ र॑राध। सं न॒स्ताभ्याꣳ॑ सृजतु वि॒श्वक॑र्मा घो॒रा ऋष॑यो॒ नमो॑ अस्त्वेभ्यः। चक्षु॑ष एषां॒ मन॑सश्च स॒न्धौ बृह॒स्पत॑ये॒ महि॒ षद्द्यु॒मन्नमः॑। नमो॑ वि॒श्वक॑र्मणे॒ स उ॑ पात्व॒स्मान॑न॒न्यान्थ्सो॑म॒पान्मन्य॑मानः। प्रा॒णस्य॑ वि॒द्वान्थ्स॑म॒रे न धीर॒ एन॑श्चकृ॒वान्महि॑ ब॒द्ध ए॑षाम्। तं वि॑श्वकर्म॒न्~(३०)

%3.2.8.3
प्र मु॑ञ्चा स्व॒स्तये॒ ये भ॒क्षय॑न्तो॒ न वसू᳚न्यानृ॒हुः। यान॒ग्नयो॒\-ऽन्वत॑प्यन्त॒ धिष्णि॑या इ॒यं तेषा॑मव॒या दुरि॑ष्ट्यै॒ स्वि॑ष्टिं न॒स्तां कृ॑णोतु वि॒श्वक॑र्मा। नमः॑ पि॒तृभ्यो॑ अ॒भि ये नो॒ अख्य॑न् यज्ञ॒कृतो॑ य॒ज्ञका॑माः सुदे॒वा अ॑का॒मा वो॒ दक्षि॑णां॒ न नी॑निम॒ मा न॒स्तस्मा॒देन॑सः पापयिष्ट। याव॑न्तो॒ वै स॑द॒स्या᳚स्ते सर्वे॑ दक्षि॒ण्या᳚स्तेभ्यो॒ यो दक्षि॑णां॒ न~(३१)

%3.2.8.4
नये॒दैभ्यो॑ वृश्च्येत॒ यद्वै᳚श्वकर्म॒णानि॑ जु॒होति॑ सद॒स्या॑ने॒व तत्प्री॑णात्य॒स्मे दे॑वासो॒ वपु॑षे चिकिथ्सत॒ यमा॒शिरा॒ दम्प॑ती वा॒मम॑श्ञु॒तः। पुमा᳚न्पु॒त्रो जा॑यते वि॒न्दते॒ वस्वथ॒ विश्वे॑ अर॒पा ए॑धते गृ॒हः। आ॒शी॒र्दा॒या दम्प॑ती वा॒मम॑श्ञुता॒मरि॑ष्टो॒ रायः॑ सचता॒ꣳ॒ समो॑कसा। य आसि॑च॒थ्सन्दु॑ग्धं कु॒म्भ्या स॒हेष्टेन॒ याम॒न्नम॑तिं जहातु॒ सः। स॒र्पि॒र्ग्री॒वी~(३२)


%3.2.8.5
पीव॑र्यस्य जा॒या पीवा॑नः पु॒त्रा अकृ॑शासो अस्य। स॒हजा॑नि॒र्यः सु॑मख॒स्यमा॑न॒ इन्द्रा॑या॒शिरꣳ॑ स॒ह कु॒म्भ्याऽदा᳚त्। आ॒शीर्म॒ ऊर्ज॑मु॒त सु॑प्रजा॒स्त्वमिषं॑ दधातु॒ द्रवि॑ण॒ꣳ॒ सव॑र्चसम्। स॒ञ्जय॒न्क्षेत्रा॑णि॒ सह॑सा॒ऽहमि॑न्द्र कृण्वा॒नो अ॒न्याꣳ अध॑रान्थ्स॒पत्नान्॑। भू॒तम॑सि भू॒ते मा॑ धा॒ मुख॑मसि॒ मुखं॑ भूयासं॒ द्यावा॑पृथि॒वी\-भ्यां᳚ त्वा॒ परि॑ गृह्णामि॒ विश्वे᳚ त्वा दे॒वा वै᳚श्वान॒राः~(३३)

%3.2.8.6
प्र च्या॑वयन्तु दि॒वि दे॒वां दृꣳ॑हा॒न्तरि॑क्षे॒ वयाꣳ॑सि पृथि॒व्यां पार्थि॑वान्ध्रु॒वं ध्रु॒वेण॑ ह॒विषाऽव॒ सोमं॑ नयामसि। यथा॑ नः॒ सर्व॒मिज्जग॑दय॒क्ष्मꣳ सु॒मना॒ अस॑त्। यथा॑ न॒ इन्द्र॒ इद्विशः॒ केव॑लीः॒ सर्वाः॒ सम॑नसः॒ कर॑त्। यथा॑ नः॒ सर्वा॒ इद्दिशो॒\-ऽस्माकं॒ केव॑ली॒रसन्न्॑॥~(३४)

%3.2.9.0
{\anuvakamend[{एन॑सा विश्वकर्म॒न्॒ यो दक्षि॑णां॒ न स॑र्पिर्ग्री॒वी वै᳚श्वान॒राश्च॑त्वारि॒ꣳ॒शच्च॑}]}%~(८)

%3.2.9.1
यद्वै होता᳚ध्व॒र्युम॑भ्या॒ह्वय॑ते॒ वज्र॑मेनम॒भि प्र व॑र्तय॒त्युक्थ॑शा॒ इत्या॑ह प्रातःसव॒नं प्र॑ति॒गीर्य॒ त्रीण्ये॒तान्य॒क्षरा॑णि त्रि॒पदा॑ गाय॒त्री गा॑य॒त्रं प्रा॑तःसव॒नं गा॑यत्रि॒यैव प्रा॑तःसव॒ने वज्र॑म॒न्तर्ध॑त्त उ॒क्थं वा॒चीत्या॑ह॒ माध्य॑न्दिन॒ꣳ॒ सव॑नं प्रति॒गीर्य॑ च॒त्वार्ये॒तान्य॒क्षरा॑णि॒ चतु॑ष्पदा त्रि॒ष्टुप्त्रैष्टु॑भं॒ माध्य॑न्दिन॒ꣳ॒ सव॑नं त्रि॒ष्टुभै॒व माध्य॑न्दिने॒ सव॑ने॒ वज्र॑म॒न्तर्ध॑त्त~-~(३५)

%3.2.9.2
उ॒क्थं वा॒चीन्द्रा॒येत्या॑ह तृतीयसव॒नं प्र॑ति॒गीर्य॑ स॒प्तैतान्य॒क्षरा॑णि स॒प्तप॑दा॒ शक्व॑री शाक्व॒रो वज्रो॒ वज्रे॑णै॒व तृ॑तीयसव॒ने वज्र॑म॒न्तर्ध॑त्ते ब्रह्मवा॒दिनो॑ वदन्ति॒ स त्वा अ॑ध्व॒र्युः स्या॒द्यो य॑थासव॒नं प्र॑तिग॒रे छन्दाꣳ॑सि सम्पा॒दये॒त्तेजः॑ प्रातःसव॒न आ॒त्मन्दधी॑तेन्द्रि॒यं माध्य॑न्दिने॒ सव॑ने प॒शूꣴस्तृ॑तीयसव॒न इत्युक्थ॑शा॒ इत्या॑ह प्रातःसव॒नं प्र॑ति॒गीर्य॒ त्रीण्ये॒तान्य॒क्षरा॑णि~(३६)

%3.2.9.3
त्रि॒पदा॑ गाय॒त्री गा॑य॒त्रं प्रा॑तःसव॒नं प्रा॑तःसव॒न ए॒व प्र॑तिग॒रे छन्दाꣳ॑सि॒ सम्पा॑दय॒त्यथो॒ तेजो॒ वै गा॑य॒त्री तेजः॑ प्रातःसव॒नं तेज॑ ए॒व प्रा॑तःसव॒न आ॒त्मन्ध॑त्त उ॒क्थं वा॒चीत्या॑ह॒ माध्य॑न्दिन॒ꣳ॒ सव॑नं प्रति॒गीर्य॑ च॒त्वार्ये॒तान्य॒क्षरा॑णि॒ चतु॑ष्पदा त्रि॒ष्टुप्त्रैष्टु॑भं॒ माध्य॑न्दिन॒ꣳ॒ सव॑नं॒ माध्य॑न्दिन ए॒व सव॑ने प्रतिग॒रे छन्दाꣳ॑सि॒ सम्पा॑दय॒त्यथो॑ इन्द्रि॒यं वै त्रि॒ष्टुगि॑न्द्रि॒यं माध्य॑न्दिन॒ꣳ॒ सव॑न-~(३७)

%3.2.9.4
मिन्द्रि॒यमे॒व माध्य॑न्दिने॒ सव॑न आ॒त्मन्ध॑त्त उ॒क्थं वा॒चीन्द्रा॒येत्या॑ह तृतीयसव॒नं प्र॑ति॒गीर्य॑ स॒प्तैतान्य॒क्षरा॑णि स॒प्तप॑दा॒ शक्व॑री शाक्व॒राः प॒शवो॒ जाग॑तं तृतीयसव॒नं तृ॑तीयसव॒न ए॒व प्र॑तिग॒रे छन्दाꣳ॑सि॒ सम्पा॑दय॒त्यथो॑ प॒शवो॒ वै जग॑ती प॒शव॑स्तृतीयसव॒नं प॒शूने॒व तृ॑तीयसव॒न आ॒त्मन्ध॑त्ते॒ यद्वै होता᳚ध्व॒र्युम॑भ्या॒ह्वय॑त आ॒व्य॑मस्मिन्दधाति॒ तद्यन्ना~(३८)

%3.2.9.5
ऽप॒हनी॑त पु॒रास्य॑ संवथ्स॒राद्गृ॒ह आ वे॑वीर॒ञ्छोꣳसा॒ मोद॑ इ॒वेति॑ प्र॒त्याह्व॑यते॒ तेनै॒व तदप॑ हते॒ यथा॒ वा आय॑तां प्र॒तीक्ष॑त ए॒वम॑ध्व॒र्युः प्र॑तिग॒रं प्रती᳚क्षते॒ यद॑भिप्रतिगृणी॒याद्यथाय॑तया समृ॒च्छते॑ ता॒दृगे॒व तद्यद॑र्ध॒र्चाल्लुप्ये॑त॒ यथा॒ धाव॑द्भ्यो॒ हीय॑ते ता॒दृगे॒व तत्प्र॒बाहु॒ग्वा ऋ॒त्विजा॑मुद्गी॒था उ॑द्गी॒थ ए॒वोद्गा॑तृ॒णा-~(३९)

%3.2.9.6
मृ॒चः प्र॑ण॒व उ॑क्थश॒ꣳ॒सिनां᳚ प्रतिग॒रो᳚\-ऽध्वर्यू॒णां य ए॒वं वि॒द्वान्प्र॑तिगृ॒णात्य॑न्ना॒द ए॒व भ॑व॒त्यास्य॑ प्र॒जायां᳚ वा॒जी जा॑यत इ॒यं वै होता॒साव॑ध्व॒र्युर्यदासी॑नः॒ शꣳस॑त्य॒स्या ए॒व तद्धोता॒ नैत्यास्त॑ इव॒ हीयमथो॑ इ॒मामे॒व तेन॒ यज॑मानो दुहे॒ यत्तिष्ठ॑न्प्रतिगृ॒णात्य॒मुष्या॑ ए॒व तद॑ध्व॒र्युर्नैति॒~(४०)

%3.2.9.7
तिष्ठ॑तीव॒ ह्य॑सावथो॑ अ॒मूमे॒व तेन॒ यज॑मानो दुहे॒ यदासी॑नः॒ शꣳस॑ति॒ तस्मा॑दि॒तःप्र॑दानं दे॒वा उप॑ जीवन्ति॒ यत्तिष्ठ॑न्प्रतिगृ॒णाति॒ तस्मा॑द॒मुतः॑प्रदानं मनु॒ष्या॑ उप॑ जीवन्ति॒ यत्प्राङासी॑नः॒ शꣳस॑ति प्र॒त्यङ्तिष्ठ॑न्प्रतिगृ॒णाति॒ तस्मा᳚त्प्रा॒चीन॒ꣳ॒ रेतो॑ धीयते प्र॒तीचीः᳚ प्र॒जा जा॑यन्ते॒ यद्वै होता᳚ध्व॒र्युम॑भ्या॒ह्वय॑ते॒ वज्र॑मेनम॒भि प्र व॑र्तयति॒ परा॒ङा व॑र्तते॒ वज्र॑मे॒व तन्नि क॑रोति॥~(४१)

%3.2.10.0
{\anuvakamend[{सव॑ने॒ वज्र॑म॒न्तर्ध॑त्ते॒ त्रीण्ये॒तान्य॒क्षरा॑णीन्द्रि॒यं माध्य॑न्दिन॒ꣳ॒ सव॑न॒न्नोद्गा॑तृ॒णाम॑ध्व॒र्युर्नैति॑ वर्तयत्य॒ष्टौ च॑}]}%~(९)

%3.2.10.1
उ॒प॒या॒मगृ॑हीतो\-ऽसि वाक्ष॒सद॑सि वा॒क्पा\-भ्यां᳚ त्वा क्रतु॒पाभ्या॑म॒स्य य॒ज्ञस्य॑ ध्रु॒वस्याध्य॑क्षाभ्यां गृह्णाम्युपया॒मगृ॑हीतो\-ऽस्यृत॒सद॑सि चक्षु॒ष्पा\-भ्यां᳚ त्वा क्रतु॒पाभ्या॑म॒स्य य॒ज्ञस्य॑ ध्रु॒वस्याध्य॑क्षाभ्यां गृह्णाम्युपया॒मगृ॑हीतो\-ऽसि श्रुत॒सद॑सि श्रोत्र॒पा\-भ्यां᳚ त्वा क्रतु॒पाभ्या॑म॒स्य य॒ज्ञस्य॑ ध्रु॒वस्याध्य॑क्षाभ्यां गृह्णामि दे॒वेभ्य॑स्त्वा वि॒श्वदे॑वेभ्यस्त्वा॒ विश्वे᳚भ्यस्त्वा दे॒वेभ्यो॒ विष्ण॑वुरुक्रमै॒ष ते॒ सोम॒स्तꣳ र॑क्षस्व॒~(४२)

%3.2.10.2
तं ते॑ दु॒श्चक्षा॒ माव॑ ख्य॒न्मयि॒ वसुः॑ पुरो॒वसु॑र्वा॒क्पा वाचं॑ मे पाहि॒ मयि॒ वसु॑र्वि॒दद्व॑सुश्चक्षु॒ष्पाश्चक्षु॑र्मे पाहि॒ मयि॒ वसुः॑ सं॒यद्व॑सुः श्रोत्र॒पाः श्रोत्रं॑ मे पाहि॒ भूर॑सि॒ श्रेष्ठो॑ रश्मी॒नां प्रा॑ण॒पाः प्रा॒णं मे॑ पाहि॒ धूर॑सि॒ श्रेष्ठो॑ रश्मी॒नाम॑पान॒पा अ॑पा॒नं मे॑ पाहि॒ यो न॑ इन्द्रवायू मित्रावरुणावश्विनावभि॒दास॑ति॒ भ्रातृ॑व्य उ॒त्पिपी॑ते शुभस्पती इ॒दम॒हं तमध॑रं पादयामि॒ यथे᳚न्द्रा॒हमु॑त्त॒मश्चे॒तया॑नि॥~(४३)

%3.2.11.0
{\anuvakamend[{र॒क्ष॒स्व॒ भ्रातृ॑व्य॒स्त्रयो॑दश च}]}%॥10॥

%3.2.11.1
प्र सो अ॑ग्ने॒ तवो॒तिभिः॑ सु॒वीरा॑भिस्तरति॒ वाज॑कर्मभिः। यस्य॒ त्वꣳ स॒ख्यमावि॑थ। प्र होत्रे॑ पू॒र्व्यं वचो॒\-ऽग्नये॑ भरता बृ॒हत्। वि॒पां ज्योतीꣳ॑षि॒ बिभ्र॑ते॒ न वे॒धसे᳚। अग्ने॒ त्री ते॒ वाजि॑ना॒ त्री ष॒धस्था॑ ति॒स्रस्ते॑ जि॒ह्वा ऋ॑तजात पू॒र्वीः। ति॒स्र उ॑ ते त॒नुवो॑ दे॒ववा॑ता॒स्ताभि॑र्नः पाहि॒ गिरो॒ अप्र॑युच्छन्न्। सं वां॒ कर्म॑णा॒ समि॒षा~(४४)

%3.2.11.2
हि॑नो॒मीन्द्रा॑विष्णू॒ अप॑सस्पा॒रे अ॒स्य। जु॒षेथां᳚ य॒ज्ञं द्रवि॑णं च धत्त॒मरि॑ष्टैर्नः प॒थिभिः॑ पा॒रय॑न्ता। उ॒भा जि॑ग्यथु॒र्न परा॑ जयेथे॒ न परा॑ जिग्ये कत॒रश्च॒नैनोः᳚। इन्द्र॑श्च विष्णो॒ यदप॑स्पृधेथां त्रे॒धा स॒हस्रं॒ वि तदै॑रयेथाम्। त्रीण्यायूꣳ॑षि॒ तव॑ जातवेदस्ति॒स्र आ॒जानी॑रु॒षस॑स्ते अग्ने। ताभि॑र्दे॒वाना॒मवो॑ यक्षि वि॒द्वानथा॑~(४५)

%3.2.11.3
भव॒ यज॑मानाय॒ शं योः। अ॒ग्निस्त्रीणि॑ त्रि॒धातू॒न्या क्षे॑ति वि॒दथा॑ क॒विः। स त्रीꣳरे॑काद॒शाꣳ इ॒ह। यक्ष॑च्च पि॒प्रय॑च्च नो॒ विप्रो॑ दू॒तः परि॑ष्कृतः। नभ॑न्तामन्य॒के स॑मे। इन्द्रा॑विष्णू दृꣳहि॒ताः शम्ब॑रस्य॒ नव॒ पुरो॑ नव॒तिं च॑ श्ञथिष्टम्। श॒तं व॒र्चिनः॑ स॒हस्रं॑ च सा॒कꣳ ह॒थो अ॑प्र॒त्यसु॑रस्य वी॒रान्। उ॒त मा॒ता म॑हि॒षमन्व॑वेनद॒मी त्वा॑ जहति पुत्र दे॒वाः। अथा᳚ब्रवीद्वृ॒त्रमिन्द्रो॑ हनि॒ष्यन्थ्सखे॑ विष्णो वित॒रं वि क्र॑मस्व॥~(४६)

%3.3.0.0

%3.3.0.0
{\anuvakamend[{इ॒षा\-ऽथ॑ त्वा॒ त्रयो॑दश च}]}%॥11॥

{\prashnaend[{यो वै स्फ्यः स्व॒स्तिः स्व॒धायै॒ नमः॒ प्र मु़॑ञ्च॒ तिष्ठ॑तीव॒ षट्च॑त्वारिꣳशत्॥46॥ यो वै पव॑मानानां॒ वि क्र॑मस्व॥}]}
%%% END PRASHNA

\sect{तृतीयः प्रश्नः}\setcounter{anuvakam}{0}
\dnsub{तैत्तिरीयसंहितायां तृतीयकाण्डे तृतीयः प्रश्नः}
%3.3.1.0
%3.3.1.1
अग्ने॑ तेजस्विन्तेज॒स्वी त्वं दे॒वेषु॑ भूया॒स्तेज॑स्वन्तं॒ मामायु॑ष्मन्तं॒ वर्च॑स्वन्तं मनु॒ष्ये॑षु कुरु दी॒क्षायै॑ च त्वा॒ तप॑सश्च॒ तेज॑से जुहोमि तेजो॒विद॑सि॒ तेजो॑ मा॒ मा हा॑सी॒न्मा\-ऽहं तेजो॑ हासिषं॒ मा मां तेजो॑ हासी॒दिन्द्रौ॑जस्विन्नोज॒स्वी त्वं दे॒वेषु॑ भूया॒ ओज॑स्वन्तं॒ मामायु॑ष्मन्तं॒ वर्च॑स्वन्तं मनु॒ष्ये॑षु कुरु॒ ब्रह्म॑णश्च त्वा क्ष॒त्रस्य॒ चौ-~(१)

%3.3.1.2
ज॑से जुहोम्योजो॒विद॒स्योजो॑ मा॒ मा हा॑सी॒न्माऽहमोजो॑ हासिषं॒ मा मामोजो॑ हासी॒थ्सूर्य॑ भ्राजस्विन्भ्राज॒स्वी त्वं दे॒वेषु॑ भूया॒ भ्राज॑स्वन्तं॒ मामायु॑ष्मन्तं॒ वर्च॑स्वन्तं मनु॒ष्ये॑षु कुरु वा॒योश्च॑ त्वा॒\-ऽपां च॒ भ्राज॑से जुहोमि सुव॒र्विद॑सि॒ सुव॑र्मा॒ मा हा॑सी॒न्माऽहꣳ सुव॑र्\mbox{}हासिषं॒ मा माꣳ सुव॑र्\mbox{}हासी॒न्मयि॑ मे॒धां मयि॑ प्र॒जां मय्य॒ग्निस्तेजो॑ दधातु॒ मयि॑ मे॒धां मयि॑ प्र॒जां मयीन्द्र॑ इन्द्रि॒यं द॑धातु॒ मयि॑ मे॒धां मयि॑ प्र॒जां मयि॒ सूर्यो॒ भ्राजो॑ दधातु॥~(२)

%3.3.2.0
{\anuvakamend[{क्ष॒त्रस्य॑ च॒ मयि॒ त्रयो॑विꣳशतिश्च}]}%~(१)

%3.3.2.1
वा॒युर्\mbox{}हि॑ङ्क॒र्ता\-ऽग्निः प्र॑स्तो॒ता प्र॒जा\-प॑तिः॒ साम॒ बृह॒स्पति॑रुद्गा॒ता विश्वे॑ दे॒वा उ॑पगा॒तारो॑ म॒रुतः॑ प्रतिह॒र्तार॒ इन्द्रो॑ नि॒धनं॒ ते दे॒वाः प्रा॑ण॒भृतः॑ प्रा॒णं मयि॑ दधत्वे॒तद्वै सर्व॑मध्व॒र्युरु॑पाकु॒र्वन्नु॑द्गा॒तृभ्य॑ उ॒पा\-क॑रोति॒ ते दे॒वाः प्रा॑ण॒भृतः॑ प्रा॒णं मयि॑ दध॒त्वित्या॑है॒तदे॒व सर्व॑मा॒त्मन्ध॑त्त॒ इडा॑ देव॒हूर्मनु॑र्यज्ञ॒नीर्बृह॒स्पति॑रुक्थाम॒दानि॑ शꣳसिष॒द्विश्वे॑ दे॒वाः~(३)

%3.3.2.2
सू᳚क्त॒वाचः॒ पृथि॑वि मात॒र्मा मा॑ हिꣳसी॒र्मधु॑ मनिष्ये॒ मधु॑ जनिष्ये॒ मधु॑ वक्ष्यामि॒ मधु॑ वदिष्यामि॒ मधु॑मतीं दे॒वेभ्यो॒ वाच॑मुद्यासꣳ शुश्रू॒षेण्यां᳚ मनु॒ष्ये᳚भ्य॒स्तं मा॑ दे॒वा अ॑वन्तु शो॒भायै॑ पि॒तरो\-ऽनु॑ मदन्तु॥~(४)

%3.3.3.0
{\anuvakamend[{श॒ꣳ॒सि॒ष॒द्विश्वे॑ दे॒वा अ॒ष्टाविꣳ॑शतिश्च}]}%~(२)

%3.3.3.1
वस॑वस्त्वा॒ प्र वृ॑हन्तु गाय॒त्रेण॒ छन्द॑सा॒\-ऽग्नेः प्रि॒यं पाथ॒ उपे॑हि रु॒द्रास्त्वा॒ प्र वृ॑हन्तु॒ त्रैष्टु॑भेन॒ छन्द॒सेन्द्र॑स्य प्रि॒यं पाथ॒ उपे᳚ह्यादि॒त्यास्त्वा॒ प्र वृ॑हन्तु॒ जाग॑तेन॒ छन्द॑सा॒ विश्वे॑षां दे॒वानां᳚ प्रि॒यं पाथ॒ उपे॑हि॒ मान्दा॑सु ते शुक्र शु॒क्रमा धू॑नोमि भ॒न्दना॑सु॒ कोत॑नासु॒ नूत॑नासु॒ रेशी॑षु॒ मेषी॑षु॒ वाशी॑षु विश्व॒भृथ्सु॒ माध्वी॑षु ककु॒हासु॒ शक्व॑रीषु~(५)

%3.3.3.2
शु॒क्रासु॑ ते शुक्र शु॒क्रमा धू॑नोमि शु॒क्रं ते॑ शु॒क्रेण॑ गृह्णा॒म्यह्नो॑ रू॒पेण॒ सूर्य॑स्य र॒श्मिभिः॑। आ\-ऽस्मि॑न्नु॒ग्रा अ॑चुच्यवुर्दि॒वो धारा॑ असश्चत। क॒कु॒हꣳ रू॒पं वृ॑ष॒भस्य॑ रोचते बृ॒हथ्सोमः॒ सोम॑स्य पुरो॒गाः शु॒क्रः शु॒क्रस्य॑ पुरो॒गाः। यत्ते॑ सो॒मादा᳚भ्यं॒ नाम॒ जागृ॑वि॒ तस्मै॑ ते सोम॒ सोमा॑य॒ स्वाहो॒शिक्त्वं दे॑व सोम गाय॒त्रेण॒ छन्द॑सा॒\-ऽग्नेः~(६)

%3.3.3.3
प्रि॒यं पाथो॒ अपी॑हि व॒शी त्वं दे॑व सोम॒ त्रैष्टु॑भेन॒ छन्द॒सेन्द्र॑स्य प्रि॒यं पाथो॒ अपी᳚ह्य॒स्मथ्स॑खा॒ त्वं दे॑व सोम॒ जाग॑तेन॒ छन्द॑सा॒ विश्वे॑षां दे॒वानां᳚ प्रि॒यं पाथो॒ अपी॒ह्या नः॑ प्रा॒ण ए॑तु परा॒वत॒ आन्तरि॑क्षाद्दि॒वस्परि॑। आयुः॑ पृथि॒व्या अध्य॒मृत॑मसि प्रा॒णाय॑ त्वा। इ॒न्द्रा॒ग्नी मे॒ वर्चः॑ कृणुतां॒ वर्चः॒ सोमो॒ बृह॒स्पतिः॑। वर्चो॑ मे॒ विश्वे॑ दे॒वा वर्चो॑ मे धत्तमश्विना। द॒ध॒न्वे वा॒ यदी॒मनु॒ वोच॒द्ब्रह्मा॑णि॒ वेरु॒ तत्। परि॒ विश्वा॑नि॒ काव्या॑ ने॒मिश्च॒क्रमि॑वाभवत्॥~(७)

%3.3.4.0
{\anuvakamend[{शक्व॑रीष्व॒ग्नेर्बृह॒स्पतिः॒ पञ्च॑विꣳशतिश्च}]}%~(३)

%3.3.4.1
ए॒तद्वा अ॒पां ना॑म॒धेयं॒ गुह्यं॒ यदा॑धा॒वा मान्दा॑सु ते शुक्र शु॒क्रमा धू॑नो॒मीत्या॑हा॒पामे॒व ना॑म॒धेये॑न॒ गुह्ये॑न दि॒वो वृष्टि॒मव॑ रुन्धे शु॒क्रं ते॑ शु॒क्रेण॑ गृह्णा॒मीत्या॑है॒तद्वा अह्नो॑ रू॒पं यद्रात्रिः॒ सूर्य॑स्य र॒श्मयो॒ वृष्ट्या॑ ईश॒ते\-ऽह्न॑ ए॒व रू॒पेण॒ सूर्य॑स्य र॒श्मिभि॑र्दि॒वो वृ॑ष्टिं च्यावय॒त्या\-ऽस्मि॑न्नु॒ग्राः~(८)

%3.3.4.2
अ॑चु॒च्य॒वु॒रित्या॑ह यथाय॒जुरे॒वैतत्क॑कु॒हꣳ रू॒पं वृ॑ष॒भस्य॑ रोचते बृ॒हदित्या॑है॒तद्वा अ॑स्य ककु॒हꣳ रू॒पं यद्वृष्टी॑ रू॒पेणै॒व वृष्टि॒मव॑ रुन्धे॒ यत्ते॑ सो॒मादा᳚भ्यं॒ नाम॒ जागृ॒वीत्या॑है॒ष ह॒ वै ह॒विषा॑ ह॒विर्य॑जति॒ यो\-ऽदा᳚भ्यं गृही॒त्वा सोमा॑य जु॒होति॒ परा॒ वा ए॒तस्यायुः॑ प्रा॒ण ए॑ति॒~(९)

%3.3.4.3
यो\-ऽꣳ॑शुं गृ॒ह्णात्या नः॑ प्रा॒ण ए॑तु परा॒वत॒ इत्या॒हायु॑रे॒व प्रा॒णमा॒त्मन्ध॑त्ते॒\-ऽमृत॑मसि प्रा॒णाय॒ त्वेति॒ हिर॑ण्यम॒भि व्य॑नित्य॒मृतं॒ वै हिर॑ण्य॒मायुः॑ प्रा॒णो॑\-ऽमृते॑नै॒वायु॑रा॒त्मन्ध॑त्ते श॒तमा॑नं भवति श॒तायुः॒ पुरु॑षः श॒तेन्द्रि॑य॒ आयु॑ष्ये॒वेन्द्रि॒ये प्रति॑ तिष्ठत्य॒प उप॑ स्पृशति भेष॒जं वा आपो॑ भेष॒जमे॒व कु॑रुते॥~(१०)

%3.3.5.0
{\anuvakamend[{उ॒ग्रा ए॒त्याप॒स्त्रीणि॑ च}]}%~(४)

%3.3.5.1
वा॒युर॑सि प्रा॒णो नाम॑ सवि॒तुराधि॑पत्ये\-ऽपा॒नं मे॑ दा॒श्चक्षु॑रसि॒ श्रोत्रं॒ नाम॑ धा॒तुराधि॑पत्य॒ आयु॑र्मे दा रू॒पम॑सि॒ वर्णो॒ नाम॒ बृह॒स्पते॒राधि॑पत्ये प्र॒जां मे॑ दा ऋ॒तम॑सि स॒त्यं नामेन्द्र॒स्याधि॑पत्ये क्ष॒त्रं मे॑ दा भू॒तम॑सि॒ भव्यं॒ नाम॑ पितृ॒णामाधि॑पत्ये॒\-ऽपामोष॑धीनां॒ गर्भं॑ धा ऋ॒तस्य॑ त्वा॒ व्यो॑मन ऋ॒तस्य॑~(११)

%3.3.5.2
त्वा॒ विभू॑मन ऋ॒तस्य॑ त्वा॒ विध॑र्मण ऋ॒तस्य॑ त्वा स॒त्याय॒र्तस्य॑ त्वा॒ ज्योति॑षे प्र॒जा\-प॑तिर्वि॒राज॑मपश्य॒त्तया॑ भू॒तं च॒ भव्यं॑ चासृजत॒ तामृषि॑भ्यस्ति॒रो॑\-ऽदधा॒त्तां ज॒मद॑ग्नि॒स्तप॑सा\-ऽपश्य॒त्तया॒ वै स पृश्ञी॒न्कामा॑नसृजत॒ तत्पृ॑श्ञीनां पृश्ञि॒त्वं यत्पृश्ञ॑यो गृ॒ह्यन्ते॒ पृश्ञी॑ने॒व तैः कामा॒न्॒ यज॑मा॒नो\-ऽव॑ रुन्धे वा॒युर॑सि प्रा॒णो~(१२)

%3.3.5.3
नामेत्या॑ह प्राणापा॒नावे॒वाव॑ रुन्धे॒ चक्षु॑रसि॒ श्रोत्रं॒ नामेत्या॒हायु॑रे॒वाव॑ रुन्धे रू॒पम॑सि॒ वर्णो॒ नामेत्या॑ह प्र॒जामे॒वाव॑ रुन्ध ऋ॒तम॑सि स॒त्यं नामेत्या॑ह क्ष॒त्रमे॒वाव॑ रुन्धे भू॒तम॑सि॒ भव्यं॒ नामेत्या॑ह प॒शवो॒ वा अ॒पामोष॑धीनां॒ गर्भः॑ प॒शूने॒वा-~(१३)

%3.3.5.4
ऽव॑ रुन्ध ए॒ताव॒द्वै पुरु॑षं प॒रित॒स्तदे॒वाव॑ रुन्ध ऋ॒तस्य॑ त्वा॒ व्यो॑मन॒ इत्या॑हे॒यं वा ऋ॒तस्य॒ व्यो॑मे॒मामे॒वाभि ज॑यत्यृ॒तस्य॑ त्वा॒ विभू॑मन॒ इत्या॑हा॒न्तरि॑क्षं॒ वा ऋ॒तस्य॒ विभू॑मा॒न्तरि॑क्षमे॒वाभि ज॑यत्यृ॒तस्य॑ त्वा॒ विध॑र्मण॒ इत्या॑ह॒ द्यौर्वा ऋ॒तस्य॒ विध॑र्म॒ दिव॑मे॒वाभि ज॑यत्यृ॒तस्य॑~(१४)

%3.3.5.5
त्वा॒ स॒त्यायेत्या॑ह॒ दिशो॒ वा ऋ॒तस्य॑ स॒त्यं दिश॑ ए॒वाभि ज॑यत्यृ॒तस्य॑ त्वा॒ ज्योति॑ष॒ इत्या॑ह सुव॒र्गो वै लो॒क ऋ॒तस्य॒ ज्योतिः॑ सुव॒र्गमे॒व लो॒कम॒भि ज॑यत्ये॒ताव॑न्तो॒ वै दे॑वलो॒कास्ताने॒वाभि ज॑यति दश॒ सम्प॑द्यन्ते॒ दशा᳚क्षरा वि॒राडन्नं॑ वि॒राड्वि॒राज्ये॒वान्नाद्ये॒ प्रति॑ तिष्ठति॥~(१५)

%3.3.6.0
{\anuvakamend[{व्यो॑मन ऋ॒तस्य॑ प्रा॒णः प॒शूने॒व विध॑र्म॒ दिव॑मे॒वाभि ज॑यत्यृ॒तस्य॒ षट्च॑त्वारिꣳशच्च}]}%~(५)

%3.3.6.1
दे॒वा वै यद्य॒ज्ञेन॒ नावारु॑न्धत॒ तत्परै॒रवा॑रुन्धत॒ तत्परा॑णां पर॒त्वं यत्परे॑ गृ॒ह्यन्ते॒ यदे॒व य॒ज्ञेन॒ नाव॑रु॒न्धे तस्याव॑रुद्ध्यै॒ यं प्र॑थ॒मं गृ॒ह्णाती॒ममे॒व तेन॑ लो॒कम॒भि ज॑यति॒ यं द्वि॒तीय॑म॒न्तरि॑क्षं॒ तेन॒ यं तृ॒तीय॑म॒मुमे॒व तेन॑ लो॒कम॒भि ज॑यति यदे॒ते गृ॒ह्यन्त॑ ए॒षां लो॒काना॑म॒भिजि॑त्यै~(१६)

%3.3.6.2
उत्त॑रे॒ष्वहः॑स्व॒मुतो॒\-ऽर्वाञ्चो॑ गृह्यन्ते\-ऽभि॒जित्यै॒वेमाँल्लो॒कान्पुन॑रि॒मं लो॒कं प्र॒त्यव॑रोहन्ति॒ यत्पूर्वे॒ष्वहः॑स्वि॒तः परा᳚ञ्चो गृ॒ह्यन्ते॒ तस्मा॑दि॒तः परा᳚ञ्च इ॒मे लो॒का यदुत्त॑रे॒ष्वहः॑स्व॒मुतो॒\-ऽर्वाञ्चो॑ गृ॒ह्यन्ते॒ तस्मा॑द॒मुतो॒\-ऽर्वां च॑ इ॒मे लो॒कास्तस्मा॒द\-या॑तयाम्नो लो॒कान्म॑नु॒ष्या॑ उप॑ जीवन्ति ब्रह्मवा॒दिनो॑ वदन्ति॒ कस्मा᳚थ्स॒त्याद॒द्भ्य ओष॑धयः॒ सम्भ॑व॒न्त्योष॑धयः~(१७)

%3.3.6.3
म॒नु॒ष्या॑णा॒मन्नं॑ प्र॒जा\-प॑तिं प्र॒जा अनु॒ प्र जा॑यन्त॒ इति॒ परा॒नन्विति॑ ब्रूया॒द्यद्गृ॒ह्णात्य॒द्भ्यस्त्वौष॑धीभ्यो गृह्णा॒मीति॒ तस्मा॑द॒द्भ्य ओष॑धयः॒ सम्भ॑वन्ति॒ यद्गृ॒ह्णात्योष॑धीभ्यस्त्वा प्र॒जाभ्यो॑ गृह्णा॒मीति॒ तस्मा॒दोष॑धयो मनु॒ष्या॑णा॒मन्नं॒ यद्गृ॒ह्णाति॑ प्र॒जाभ्य॑स्त्वा प्र॒जा\-प॑तये गृह्णा॒मीति॒ तस्मा᳚त्प्र॒जा\-प॑तिं प्र॒जा अनु॒ प्र जा॑यन्ते॥~(१८)

%3.3.7.0
{\anuvakamend[{अ॒भिजि॑त्या॒ ओष॑धयो॒\-ऽष्टाच॑त्वारिꣳशच्च}]}%~(६)

%3.3.7.1
प्र॒जा\-प॑तिर्देवासु॒रान॑सृजत॒ तदनु॑ य॒ज्ञो॑\-ऽसृज्यत य॒ज्ञं छन्दाꣳ॑सि॒ ते विष्व॑ञ्चो॒ व्य॑क्राम॒न्थ्सो\-ऽसु॑रा॒ननु॑ य॒ज्ञो\-ऽपा᳚क्रामद्य॒ज्ञं छन्दाꣳ॑सि॒ ते दे॒वा अ॑मन्यन्ता॒मी वा इ॒दम॑भूव॒न्॒ यद्व॒यꣴ स्म इति॒ ते प्र॒जा\-प॑ति॒मुपा॑धाव॒न्थ्सो᳚\-ऽब्रवीत्प्र॒जा\-प॑ति॒श्छन्द॑सां वी॒र्य॑मा॒दाय॒ तद्वः॒ प्र दा᳚स्या॒मीति॒ स छन्द॑सां वी॒र्यम्᳚~(१९)

%3.3.7.2
आ॒दाय॒ तदे᳚भ्यः॒ प्राय॑च्छ॒त्तदनु॒ छन्दा॒ꣴ॒स्यपा᳚क्राम॒ञ्छन्दाꣳ॑सि य॒ज्ञस्ततो॑ दे॒वा अभ॑व॒न्परासु॑रा॒ य ए॒वं छन्द॑सां वी॒र्यं॑ वेदा श्रा॑व॒यास्तु॒ श्रौष॒ड्यज॒ ये यजा॑महे वषट्का॒रो भव॑त्या॒त्मना॒ परा᳚\-ऽस्य॒ भ्रातृ॑व्यो भवति ब्रह्मवा॒दिनो॑ वदन्ति॒ कस्मै॒ कम॑ध्व॒र्युरा श्रा॑वय॒तीति॒ छन्द॑सां वी॒र्या॑येति॑ ब्रूयादे॒तद्वै~(२०)

%3.3.7.3
छन्द॑सां वी॒र्य॑मा श्रा॑व॒यास्तु॒ श्रौष॒ड्यज॒ ये यजा॑महे वषट्का॒रो य ए॒वं वेद॒ सवी᳚र्यैरे॒व छन्दो॑भिरर्चति॒ यत्किं चार्च॑ति॒ यदिन्द्रो॑ वृ॒त्रमह॑न्नमे॒ध्यं तद्यद्यती॑न॒पाव॑पदमे॒ध्यं तदथ॒ कस्मा॑दै॒न्द्रो य॒ज्ञ आ सꣴस्था॑तो॒रित्या॑हु॒रिन्द्र॑स्य॒ वा ए॒षा य॒ज्ञिया॑ त॒नूर्यद्य॒ज्ञस्तामे॒व तद्य॑जन्ति॒ य ए॒वं वेदोपै॑नं य॒ज्ञो न॑मति॥~(२१)

%3.3.8.0
{\anuvakamend[{स छन्द॑सां वी॒र्यं॑ वा ए॒व तद॒ष्टौ च॑}]}%~(७)

%3.3.8.1
आ॒यु॒र्दा अ॑ग्ने ह॒विषो॑ जुषा॒णो घृ॒तप्र॑तीको घृ॒तयो॑निरेधि। घृ॒तं पी॒त्वा मधु॒ चारु॒ गव्यं॑ पि॒तेव॑ पु॒त्रम॒भि र॑क्षतादि॒मम्। आ वृ॑श्च्यते॒ वा ए॒तद्यज॑मानो॒\-ऽग्निभ्यां॒ यदे॑नयोः शृतं॒कृत्याथा॒न्यत्रा॑वभृ॒थम॒वैत्या॑यु॒र्दा अ॑ग्ने ह॒विषो॑ जुषा॒ण इत्य॑वभृ॒थम॑वै॒ष्यञ्जु॑हुया॒दाहु॑त्यै॒वैनौ॑ शमयति॒ नार्ति॒मार्च्छ॑ति॒ यज॑मानो॒ यत्कुसी॑दम्~(२२)

%3.3.8.2
अप्र॑तीत्तं॒ मयि॒ येन॑ य॒मस्य॑ ब॒लिना॒ चरा॑मि। इ॒हैव सन्नि॒रव॑दये॒ तदे॒तत्तद॑ग्ने अनृ॒णो भ॑वामि। विश्व॑लोप विश्वदा॒वस्य॑ त्वा॒सञ्जु॑होम्य॒ग्धादेको॑\-ऽहु॒तादेकः॑ समस॒नादेकः॑। ते नः॑ कृण्वन्तु भेष॒जꣳ सदः॒ सहो॒ वरे᳚ण्यम्। अ॒यं नो॒ नभ॑सा पु॒रः स॒ꣴ॒स्फानो॑ अ॒भि र॑क्षतु। गृ॒हाणा॒मस॑मर्त्यै ब॒हवो॑ नो गृ॒हा अ॑सन्न्। स त्वं नः॑~(२३)

%3.3.8.3
न॒भ॒स॒स्प॒त॒ ऊर्जं॑ नो धेहि भ॒द्रया᳚। पुन॑र्नो न॒ष्टमा कृ॑धि॒ पुन॑र्नो र॒यिमा कृ॑धि। देव॑ सꣴस्फान सहस्रपो॒षस्ये॑शिषे॒ स नो॑ रा॒स्वाज्या॑निꣳ रा॒यस्पोषꣳ॑ सु॒वीर्यꣳ॑ संवथ्स॒रीणाꣴ॑ स्व॒स्तिम्। अ॒ग्निर्वाव य॒म इ॒यं य॒मी कुसी॑दं॒ वा ए॒तद्य॒मस्य॒ यज॑मान॒ आ द॑त्ते॒ यदोष॑धीभि॒र्वेदिꣴ॑ स्तृ॒णाति॒ यदनु॑पौष्य प्रया॒याद्ग्री॑वब॒द्धमे॑नम्~(२४)

%3.3.8.4
अ॒मुष्मिँ॑ल्लो॒के ने॑नीयेर॒न्॒ यत्कुसी॑द॒मप्र॑तीत्तं॒ मयीत्युपौ॑षती॒हैव सन् य॒मं कुसी॑दं निरव॒दाया॑नृ॒णः सु॑व॒र्गं लो॒कमे॑ति॒ यदि॑ मि॒श्रमि॑व॒ चरे॑दञ्ज॒लिना॒ सक्तू᳚न्प्रदा॒व्ये॑ जुहुयादे॒ष वा अ॒ग्निर्वै᳚श्वान॒रो यत्प्र॑दा॒व्यः॑ स ए॒वैनꣴ॑ स्वदय॒त्यह्नां᳚ वि॒धान्या॑मेकाष्ट॒काया॑मपू॒पं चतुः॑शरावं प॒क्त्वा प्रा॒तरे॒तेन॒ कक्ष॒मुपौ॑षे॒द्यदि॑~(२५)

%3.3.8.5
दह॑ति पुण्य॒समं॑ भवति॒ यदि॒ न दह॑ति पाप॒सम॑मे॒तेन॑ ह स्म॒ वा ऋ॑षयः पु॒रा वि॒ज्ञाने॑न दीर्घस॒त्रमुप॑ यन्ति॒ यो वा उ॑पद्र॒ष्टार॑मुपश्रो॒तार॑मनुख्या॒तारं॑ वि॒द्वान् यज॑ते॒ सम॒मुष्मिँ॑ल्लो॒क इ॑ष्टापू॒र्तेन॑ गच्छते॒\-ऽग्निर्वा उ॑पद्र॒ष्टा वा॒युरु॑पश्रो॒ता\-ऽ\-ऽ\-दि॒त्यो॑\-ऽनुख्या॒ता तान् य ए॒वं वि॒द्वान् यज॑ते॒ सम॒मुष्मिँ॑ल्लो॒क इ॑ष्टापू॒र्तेन॑ गच्छते॒\-ऽयं नो॒ नभ॑सा पु॒रः~(२६)

%3.3.8.6
इत्या॑हा॒\-ऽग्निर्वै नभ॑सा पु॒रो᳚\-ऽग्निमे॒व तदा॑है॒तन्मे॑ गोपा॒येति॒ स त्वं नो॑ नभसस्पत॒ इत्या॑ह वा॒युर्वै नभ॑स॒स्पति॑र्वा॒युमे॒व तदा॑है॒तन्मे॑ गोपा॒येति॒ देव॑ सꣴस्फा॒नेत्या॑हा॒सौ वा आ॑दि॒त्यो दे॒वः स॒ꣴ॒स्फान॑ आदि॒त्यमे॒व तदा॑है॒तन्मे॑ गोपा॒येति॑॥~(२७)

%3.3.9.0
{\anuvakamend[{कुसी॑द॒न्त्वन्न॑ एनमोषे॒द्यदि॑ पु॒र आ॑दि॒त्यमे॒व तदा॑है॒तन्मे॑ गोपा॒येति॑}]}%~(८)

%3.3.9.1
ए॒तं युवा॑नं॒ परि॑ वो ददामि॒ तेन॒ क्रीड॑न्तीश्चरत प्रि॒येण॑। मा नः॑ शाप्त ज॒नुषा॑ सुभागा रा॒यस्पोषे॑ण॒ समि॒षा म॑देम। नमो॑ महि॒म्न उ॒त चक्षु॑षे ते॒ मरु॑तां पित॒स्तद॒हं गृ॑णामि। अनु॑ मन्यस्व सु॒यजा॑ यजाम॒ जुष्टं॑ दे॒वाना॑मि॒दम॑स्तु ह॒व्यम्। दे॒वाना॑मे॒ष उ॑पना॒ह आ॑सीद॒पां गर्भ॒ ओष॑धीषु॒ न्य॑क्तः। सोम॑स्य द्र॒फ्सम॑वृणीत पू॒षा~(२८)

%3.3.9.2
बृ॒हन्नद्रि॑रभव॒त्तदे॑षाम्। पि॒ता व॒थ्सानां॒ पति॑रघ्नि॒याना॒मथो॑ पि॒ता म॑ह॒तां गर्ग॑राणाम्। व॒थ्सो ज॒रायु॑ प्रति॒धुक्पी॒यूष॑ आ॒मिक्षा॒ मस्तु॑ घृ॒तम॑स्य॒ रेतः॑। त्वां गावो॑\-ऽवृणत रा॒ज्याय॒ त्वाꣳ ह॑वन्त म॒रुतः॑ स्व॒र्काः। वर्ष्म॑न्क्ष॒त्रस्य॑ क॒कुभि॑ शिश्रिया॒णस्ततो॑ न उ॒ग्रो वि भ॑जा॒ वसू॑नि। व्यृ॑द्धेन॒ वा ए॒ष प॒शुना॑ यजते॒ यस्यै॒तानि॒ न क्रि॒यन्त॑ ए॒ष ह॒ त्वै समृ॑द्धेन यजते॒ यस्यै॒तानि॑ क्रि॒यन्ते᳚॥~(२९)

%3.3.10.0
{\anuvakamend[{पू॒षा क्रि॒यन्त॑ ए॒षो᳚\-ऽष्टौ च॑}]}%~(९)

%3.3.10.1
सूर्यो॑ दे॒वो दि॑वि॒षद्भ्यो॑ धा॒ता क्ष॒त्राय॑ वा॒युः प्र॒जाभ्यः॑। बृह॒स्पति॑स्त्वा प्र॒जा\-प॑तये॒ ज्योति॑ष्मतीं जुहोतु। यस्या᳚स्ते॒ हरि॑तो॒ गर्भो\-ऽथो॒ योनि॑र्\mbox{}हिर॒ण्ययी᳚। अङ्गा॒न्यह्रु॑ता॒ यस्यै॒ तां दे॒वैः सम॑जीगमम्। आ व॑र्तन वर्तय॒ नि नि॑वर्तन वर्त॒येन्द्र॑ नर्दबुद। भूम्या॒श्चत॑स्रः प्र॒दिश॒स्ताभि॒रा व॑र्तया॒ पुनः॑। वि ते॑ भिनद्मि तक॒रीं वि योनिं॒ वि ग॑वी॒न्यौ᳚। वि~(३०)

%3.3.10.2
मा॒तरं॑ च पु॒त्रं च॒ वि गर्भं॑ च ज॒रायु॑ च। ब॒हिस्ते॑ अस्तु॒ बालिति॑। उ॒रु॒द्र॒फ्सो वि॒श्वरू॑प॒ इन्दुः॒ पव॑मानो॒ धीर॑ आनञ्ज॒ गर्भम्᳚। एक॑पदी द्वि॒पदी᳚ त्रि॒पदी॒ चतु॑ष्पदी॒ पञ्च॑पदी॒ षट्प॑दी स॒प्तप॑द्य॒ष्टाप॑दी॒ भुव॒नानु॑ प्रथता॒ꣴ॒ स्वाहा᳚। म॒ही द्यौः पृ॑थि॒वी च॑ न इ॒मं य॒ज्ञं मि॑मिक्षताम्। पि॒पृ॒तां नो॒ भरी॑मभिः॥~(३१)

%3.3.11.0
{\anuvakamend[{ग॒वी॒न्यौ॑ वि चतु॑श्चत्वारिꣳशच्च}]}%॥10॥

%3.3.11.1
इ॒दं वा॑मा॒स्ये॑ ह॒विः प्रि॒यमि॑न्द्राबृहस्पती। उ॒क्थं मद॑श्च शस्यते। अ॒यं वां॒ परि॑ षिच्यते॒ सोम॑ इन्द्राबृहस्पती। चारु॒र्मदा॑य पी॒तये᳚। अ॒स्मे इ॑न्द्राबृहस्पती र॒यिं ध॑त्तꣳ शत॒ग्विनम्᳚। अश्वा॑वन्तꣳ सह॒स्रिणम्᳚। बृह॒स्पति॑र्नः॒ परि॑ पातु प॒श्चादु॒तोत्त॑रस्मा॒दध॑रादघा॒योः। इन्द्रः॑ पु॒रस्ता॑दु॒त म॑ध्य॒तो नः॒ सखा॒ सखि॑भ्यो॒ वरि॑वः कृणोतु। वि ते॒ विष्व॒ग्वात॑जूतासो अग्ने॒ भामा॑सः~(३२)

%3.3.11.2
शु॒चे॒ शुच॑यश्चरन्ति। तु॒वि॒म्र॒क्षासो॑ दि॒व्या नव॑ग्वा॒ वना॑ वनन्ति धृष॒ता रु॒जन्तः॑। त्वाम॑ग्ने॒ मानु॑षीरीडते॒ विशो॑ होत्रा॒विदं॒ विवि॑चिꣳ रत्न॒धात॑मम्। गुहा॒ सन्तꣳ॑ सुभग वि॒श्वद॑र्\mbox{}शतं तुविष्म॒णसꣳ॑ सु॒यजं॑ घृत॒श्रियम्᳚। धा॒ता द॑दातु नो र॒यिमीशा॑नो॒ जग॑त॒स्पतिः॑। स नः॑ पू॒र्णेन॑ वावनत्। धा॒ता प्र॒जाया॑ उ॒त रा॒य ई॑शे धा॒तेदं विश्व॒म्भुव॑नं जजान। धा॒ता पु॒त्रं यज॑मानाय॒ दाता᳚~(३३)

%3.3.11.3
तस्मा॑ उ ह॒व्यं घृ॒तव॑द्विधेम। धा॒ता द॑दातु नो र॒यिं प्राचीं᳚ जी॒वातु॒मक्षि॑ताम्। व॒यं दे॒वस्य॑ धीमहि सुम॒तिꣳ स॒त्यरा॑धसः। धा॒ता द॑दातु दा॒शुषे॒ वसू॑नि प्र॒जाका॑माय मी॒ढुषे॑ दुरो॒णे। तस्मै॑ दे॒वा अ॒मृताः॒ सं व्य॑यन्तां॒ विश्वे॑ दे॒वासो॒ अदि॑तिः स॒जोषाः᳚। अनु॑ नो॒\-ऽद्यानु॑मतिर्य॒ज्ञं दे॒वेषु॑ मन्यताम्। अ॒ग्निश्च॑ हव्य॒वाह॑नो॒ भव॑तां दा॒शुषे॒ मयः॑। अन्विद॑नुमते॒ त्वम्~(३४)

%3.3.11.4
मन्या॑सै॒ शं च॑ नः कृधि। क्रत्वे॒ दक्षा॑य नो हिनु॒ प्र ण॒ आयूꣳ॑षि तारिषः। अनु॑ मन्यतामनु॒मन्य॑माना प्र॒जाव॑न्तꣳ र॒यिमक्षी॑यमाणम्। तस्यै॑ व॒यꣳ हेड॑सि॒ मापि॑ भूम॒ सा नो॑ दे॒वी सु॒हवा॒ शर्म॑ यच्छतु। यस्या॑मि॒दं प्र॒दिशि॒ यद्वि॒रोच॒ते\-ऽनु॑मतिं॒ प्रति॑ भूषन्त्या॒यवः॑। यस्या॑ उ॒पस्थ॑ उ॒र्व॑न्तरि॑क्ष॒ꣳ॒ सा नो॑ दे॒वी सु॒हवा॒ शर्म॑ यच्छतु~(३५)

%3.3.11.5
रा॒काम॒हꣳ सु॒हवाꣳ॑ सुष्टु॒ती हु॑वे शृ॒णोतु॑ नः सु॒भगा॒ बोध॑तु॒ त्मना᳚। सीव्य॒त्वपः॑ सू॒च्या\-ऽच्छि॑द्यमानया॒ ददा॑तु वी॒रꣳ श॒तदा॑यमु॒क्थ्यम्᳚। यास्ते॑ राके सुम॒तयः॑ सु॒पेश॑सो॒ याभि॒र्ददा॑सि दा॒शुषे॒ वसू॑नि। ताभि॑र्नो अ॒द्य सु॒मना॑ उ॒पाग॑हि सहस्रपो॒षꣳ सु॑भगे॒ ररा॑णा। सिनी॑वालि॒ या सु॑पा॒णिः। कु॒हूम॒हꣳ सु॒भगां᳚ विद्म॒नाप॑सम॒स्मिन् य॒ज्ञे सु॒हवां᳚ जोहवीमि। सा नो॑ ददातु॒ श्रव॑णं पितृ॒णां तस्या᳚स्ते देवि ह॒विषा॑ विधेम। कु॒हूर्दे॒वाना॑म॒मृत॑स्य॒ पत्नी॒ हव्या॑ नो अ॒स्य ह॒विष॑श्चिकेतु। सं दा॒शुषे॑ कि॒रतु॒ भूरि॑ वा॒मꣳ रा॒यस्पोषं॑ चिकि॒तुषे॑ दधातु॥~(३६)

%3.4.0.0
{\anuvakamend[{भामा॑सो॒ दाता॒ त्वम॒न्तरि॑क्ष॒ꣳ॒ सा नो॑ दे॒वी सु॒हवा॒ शर्म॑ यच्छतु॒ श्रव॑णं॒ चतु॑र्विꣳशतिश्च}]}%॥11॥

%3.4.0.0
{\anuvakamend[{अग्ने॑ तेजस्विन्वा॒युर्वस॑वस्त्वै॒तद्वा अ॒पां वा॒युर॑सि प्रा॒णो नाम॑ दे॒वा वै यद्य॒ज्ञेन॒ न प्र॒जा\-प॑तिर्देवासु॒राना॑यु॒र्दा ए॒तं युवा॑न॒ꣳ॒ सूर्यो॑ दे॒व इ॒दं वा॒मेका॑\-दश}]}%॥11॥ अग्ने॑ तेजस्विन्वा॒युर॑सि॒ छन्द॑सां वी॒र्यं॑ मा॒तरं॑ च॒ षट्त्रिꣳ॑शत्॥36॥ अग्ने॑ तेजस्विꣴश्चिकि॒तुषे॑ दधातु॥
%%% END PRASHNA

\sect{चतुर्थः प्रश्नः}\setcounter{anuvakam}{0}
{\anuvakamend[{वि वा ए॒तस्या वा॑यो इ॒मे वै चि॒त्तञ्चा॒ग्निर्भू॒तानां᳚ दे॒वा वा अ॑भ्याता॒नानृ॑ता॒षाड्रा॒ष्ट्रका॑माय॒ देवि॑का॒ वास्तो᳚ष्पते॒ त्वम॑ग्ने बृ॒हदेका॑\-दश}]}%॥11॥
{\prashnaend{ वि वा ए॒तस्येत्या॑ह मृ॒त्युर्ग॑न्ध॒र्वो\-ऽव॑ रुन्धे मध्य॒तस्त्वम॑ग्ने बृ॒हथ्षट्च॑त्वारिꣳशत्॥46॥ वि वा ए॒तस्य॑ प्रि॒यासः॑॥}}
\dnsub{तैत्तिरीयसंहितायां तृतीयकाण्डे चतुर्थः प्रश्नः}
%3.4.1.0
%3.4.1.1
वि वा ए॒तस्य॑ य॒ज्ञ ऋ॑ध्यते॒ यस्य॑ ह॒विर॑ति॒रिच्य॑ते॒ सूर्यो॑ दे॒वो दि॑वि॒षद्भ्य॒ इत्या॑ह॒ बृह॒स्पति॑ना चै॒वास्य॑ प्र॒जा\-प॑तिना च य॒ज्ञस्य॒ व्यृ॑द्ध॒मपि॑ वपति॒ रक्षाꣳ॑सि॒ वा ए॒तत्प॒शुꣳ स॑चन्ते॒ यदे॑कदेव॒त्य॑ आल॑ब्धो॒ भूया॒न्भव॑ति॒ यस्या᳚स्ते॒ हरि॑तो॒ गर्भ॒ इत्या॑ह देव॒त्रैवैनां᳚ गमयति॒ रक्ष॑सा॒मप॑हत्या॒ आ व॑र्तन वर्त॒येत्या॑ह~(१)

%3.4.1.2
ब्रह्म॑णै॒वैन॒मा व॑र्तयति॒ वि ते॑ भिनद्मि तक॒रीमित्या॑ह यथाय॒जुरे॒वैतदु॑रुद्र॒फ्सो वि॒श्वरू॑प॒ इन्दु॒रित्या॑ह प्र॒जा वै प॒शव॒ इन्दुः॑ प्र॒जयै॒वैनं॑ प॒शुभिः॒ सम॑र्धयति॒ दिवं॒ वै य॒ज्ञस्य॒ व्यृ॑द्धं गच्छति पृथि॒वीमति॑रिक्त॒न्तद्यन्न श॒मये॒दार्ति॒मार्च्छे॒द्यज॑मानो म॒ही द्यौः पृ॑थि॒वी च॑ न॒ इति॑~(२)

%3.4.1.3
आ॒ह॒ द्यावा॑पृथि॒वीभ्या॑मे॒व य॒ज्ञस्य॒ व्यृ॑द्धं॒ चाति॑रिक्तं च शमयति॒ नार्ति॒मार्च्छ॑ति॒ यज॑मानो॒ भस्म॑ना॒ऽभि समू॑हति स्व॒गाकृ॑त्या॒ अथो॑ अ॒नयो॒र्वा ए॒ष गर्भो॒\-ऽनयो॑रे॒वैनं॑ दधाति॒ यद॑व॒द्येदति॒ तद्रे॑चये॒द्यन्नाव॒द्येत्प॒शोराल॑ब्धस्य॒ नाव॑ द्येत् पु॒रस्ता॒न्नाभ्या॑ अ॒न्यद॑व॒द्येदु॒परि॑ष्टाद॒न्यत्पु॒रस्ता॒द्वै नाभ्यै᳚~(३)

%3.4.1.4
प्रा॒ण उ॒परि॑ष्टादपा॒नो यावा॑ने॒व प॒शुस्तस्याव॑ द्यति॒ विष्ण॑वे शिपिवि॒ष्टाय॑ जुहोति॒ यद्वै य॒ज्ञस्या॑ति॒रिच्य॑ते॒ यः प॒शोर्भू॒मा या पुष्टि॒स्तद्विष्णुः॑ शिपिवि॒ष्टो\-ऽति॑रिक्त ए॒वाति॑रिक्तं दधा॒त्यति॑रिक्तस्य॒ शान्त्या॑ अ॒ष्टाप्रू॒ड्ढिर॑ण्यं॒ दक्षि॑णा॒\-ऽष्टाप॑दी॒ ह्ये॑षात्मा न॑व॒मः प॒शोराप्त्या॑ अन्तरको॒श उ॒ष्णीषे॒णावि॑ष्टितं भवत्ये॒वमि॑व॒ हि प॒शुरुल्ब॑मिव॒ चर्मे॑व मा॒ꣳ॒समि॒वास्थी॑व॒ यावा॑ने॒व प॒शुस्तमा॒प्त्वाव॑ रुन्धे॒ यस्यै॒षा य॒ज्ञे प्राय॑श्चित्तिः क्रि॒यत॑ इ॒ष्ट्वा वसी॑यान्भवति॥~(४)

%3.4.2.0
{\anuvakamend[{व॒र्त॒येत्या॑ह न॒ इति॒ वै नाभ्या॒ उल्ब॑मि॒वैक॑विꣳशतिश्च}]}%~(१)

%3.4.2.1
आ वा॑यो भूष शुचिपा॒ उप॑ नः स॒हस्रं॑ ते नि॒युतो॑ विश्ववार। उपो॑ ते॒ अन्धो॒ मद्य॑मयामि॒ यस्य॑ देव दधि॒षे पू᳚र्व॒पेयम्᳚। आकू᳚त्यै त्वा॒ कामा॑य त्वा स॒मृधे᳚ त्वा किक्कि॒टा ते॒ मनः॑ प्र॒जा\-प॑तये॒ स्वाहा॑ किक्कि॒टा ते᳚ प्रा॒णं वा॒यवे॒ स्वाहा॑ किक्कि॒टा ते॒ चक्षुः॒ सूर्या॑य॒ स्वाहा॑ किक्कि॒टा ते॒ श्रोत्रं॒ द्यावा॑पृथि॒वीभ्या॒ꣴ॒ स्वाहा॑ किक्कि॒टा ते॒ वाच॒ꣳ॒ सर॑स्वत्यै॒ स्वाहा᳚~(५)

%3.4.2.2
त्वं तु॒रीया॑ व॒शिनी॑ व॒शासि॑ स॒कृद्यत्त्वा॒ मन॑सा॒ गर्भ॒ आश॑यत्। व॒शा त्वं व॒शिनी॑ गच्छ दे॒वान्थ्स॒त्याः स॑न्तु॒ यज॑मानस्य॒ कामाः᳚। अ॒जासि॑ रयि॒ष्ठा पृ॑थि॒व्याꣳ सी॑दो॒र्ध्वान्तरि॑क्ष॒मुप॑ तिष्ठस्व दि॒वि ते॑ बृ॒हद्भाः। तन्तुं॑ त॒न्वन्रज॑सो भा॒नुमन्वि॑हि॒ ज्योति॑ष्मतः प॒थो र॑क्ष धि॒या कृ॒तान्। अ॒नु॒ल्ब॒णं व॑यत॒ जोगु॑वा॒मपो॒ मनु॑र्भव ज॒नया॒ दैव्यं॒ जनम्᳚। मन॑सो ह॒विर॑सि प्र॒जा\-प॑ते॒र्वर्णो॒ गात्रा॑णां ते गात्र॒भाजो॑ भूयास्म॥~(६)

%3.4.3.0
{\anuvakamend[{सर॑स्वत्यै॒ स्वाहा॒ मनु॒स्त्रयो॑दश च}]}%~(२)

%3.4.3.1
इ॒मे वै स॒हास्ता॒न्ते वा॒युर्व्य॑वा॒त्ते गर्भ॑मदधाता॒न्तꣳ सोमः॒ प्राज॑नयद॒ग्निर॑ग्रसत॒ स ए॒तं प्र॒जा\-प॑तिराग्ने॒यम॒ष्टाक॑पाल\-मपश्य॒त्तं निर॑वप॒त्तेनै॒वैना॑म॒ग्नेरधि॒ निर॑क्रीणा॒त्तस्मा॒दप्य॑न्यदेव॒त्या॑मा॒लभ॑मान आग्ने॒यम॒ष्टाक॑पालं पु॒रस्ता॒न्निर्व॑पेद॒ग्ने\-रे॒वैना॒मधि॑ नि॒ष्क्रीया ल॑भते॒ यत्~(७)

%3.4.3.2
वा॒युर्व्यवा॒त्तस्मा᳚द्वाय॒व्या॑ यदि॒मे गर्भ॒मद॑धातां॒ तस्मा᳚द्द्यावा\-पृथि॒व्या॑ यथ्सोमः॒ प्राज॑नयद॒ग्निरग्र॑सत॒ तस्मा॑दग्नीषो॒मीया॒ यद॒नयो᳚र्विय॒त्योर्वागव॑द॒त्तस्मा᳚थ्सारस्व॒ती यत्प्र॒जा\-प॑तिर॒ग्नेरधि॑ नि॒रक्री॑णा॒त्तस्मा᳚त्प्राजाप॒त्या सा वा ए॒षा स॑र्वदेव॒त्या॑ यद॒जा व॒शा वा॑य॒व्या॑मा \-ल॑भेत॒ भूति॑कामो वा॒युर्वै क्षेपि॑ष्ठा दे॒वता॑ वा॒युमे॒व स्वेन॑~(८)

%3.4.3.3
भा॒ग॒धेये॒नोप॑ धावति॒ स ए॒वैनं॒ भूतिं॑ गमयति द्यावा\-पृथि॒व्या॑मा ल॑भेत कृ॒षमा॑णः प्रति॒ष्ठाका॑मो दि॒व ए॒वास्मै॑ प॒र्जन्यो॑ वर्\mbox{}षति॒ व्य॑स्यामोष॑धयो रोहन्ति स॒मर्धु॑कमस्य स॒स्यं भ॑वत्यग्नीषो॒मीया॒मा \-ल॑भेत॒ यः का॒मये॒तान्न॑वानन्ना॒दः स्या॒मित्य॒ग्निनै॒वान्न॒मव॑ रुन्धे॒ सोमे॑ना॒न्नाद्य॒मन्न॑वाने॒वान्ना॒दो भ॑वति सारस्व॒तीमा \-ल॑भेत॒ यः~(९)

%3.4.3.4
ई॒श्व॒रो वा॒चो वदि॑तोः॒ सन्वाचं॒ न वदे॒द्वाग्वै सर॑स्वती॒ सर॑स्वतीमे॒व स्वेन॑ भाग॒धेये॒नोप॑ धावति॒ सैवास्मि॒न्वाचं॑ दधाति प्राजाप॒त्यामा \-ल॑भेत॒ यः का॒मये॒तान॑भिजितम॒भि ज॑येय॒मिति॑ प्र॒जा\-प॑तिः॒ सर्वा॑ दे॒वता॑ दे॒वता॑भिरे॒वान॑भि\-जितम॒भि ज॑यति वाय॒व्य॑यो॒पा\-क॑रोति वा॒योरे॒वैना॑मव॒रुध्या ल॑भत॒ आकू᳚त्यै त्वा॒ कामा॑य त्वा~(१०)

%3.4.3.5
इत्या॑ह यथाय॒जुरे॒वैतत्कि॑क्किटा॒कारं॑ जुहोति किक्किटाका॒रेण॒ वै ग्रा॒म्याः प॒शवो॑ रमन्ते॒ प्रार॒ण्याः प॑तन्ति॒ यत्कि॑क्किटा॒कारं॑ जु॒होति॑ ग्रा॒म्याणां᳚ पशू॒नां धृत्यै॒ पर्य॑ग्नौ क्रि॒यमा॑णे जुहोति॒ जीव॑न्तीमे॒वैनाꣳ॑ सुव॒र्गं लो॒कङ्ग॑मयति॒ त्वं तु॒रीया॑ व॒शिनी॑ व॒शासीत्या॑ह देव॒त्रैवैनां᳚ गमयति स॒त्याः स॑न्तु॒ यज॑मानस्य॒ कामा॒ इत्या॑है॒ष वै कामः॑~(११)

%3.4.3.6
यज॑मानस्य॒ यदना᳚र्त उ॒दृचं॒ गच्छ॑ति॒ तस्मा॑दे॒वमा॑हा॒जासि॑ रयि॒ष्ठेत्या॑है॒ष्वे॑वैनां᳚ लो॒केषु॒ प्रति॑\-ष्ठापयति दि॒वि ते॑ बृ॒हद्भा इत्या॑ह सुव॒र्ग ए॒वास्मै॑ लो॒के ज्योति॑र्दधाति॒ तन्तुं॑ त॒न्वन्रज॑सो भा॒नुमन्वि॒हीत्या॑हे॒माने॒वास्मै॑ लो॒कां ज्योति॑ष्मतः करोत्यनुल्ब॒णं व॑यत॒ जोगु॑वा॒मप॒ इति॑~(१२)

%3.4.3.7
आ॒ह॒ यदे॒व य॒ज्ञ उ॒ल्बणं॑ क्रि॒यते॒ तस्यै॒वैषा शान्ति॒र्मनु॑र्भव ज॒नया॒ दैव्यं॒ जन॒मित्या॑ह मान॒व्यो॑ वै प्र॒जास्ता ए॒वाद्याः᳚ कुरुते॒ मन॑सो ह॒विर॒सीत्या॑ह स्व॒गाकृ॑त्यै॒ गात्रा॑णां ते गात्र॒भाजो॑ भूया॒स्मेत्या॑हा॒ऽऽशिष॑मे॒वैतामा शा᳚स्ते॒ तस्यै॒ वा ए॒तस्या॒ एक॑मे॒वादे॑वयजनं॒ यदाल॑ब्धायाम॒भ्रः~(१३)

%3.4.3.8
भव॑ति॒ यदाल॑ब्धायाम॒भ्रः स्याद॒फ्सु वा᳚ प्रवे॒शये॒थ्सर्वां᳚ वा॒ प्राश्ञी॑या॒द्यद॒फ्सु प्र॑वे॒शये᳚द्यज्ञवेश॒सं कु॑र्या॒थ्सर्वा॑मे॒व प्राश्ञी॑यादिन्द्रि॒यमे॒वात्मन्ध॑त्ते॒ सा वा ए॒षा त्र॑या॒णामे॒वाव॑रुद्धा संवथ्सर॒सदः॑ सहस्रया॒जिनो॑ गृहमे॒धिन॒स्त ए॒वैतया॑ यजेर॒न्तेषा॑मे॒वैषाप्ता॥14॥

%3.4.4.0
{\anuvakamend[{यथ्स्वेन॑ सारस्व॒तीमा \-ल॑भेत॒ यः कामा॑य त्वा॒ कामो\-ऽप॒ इत्य॒भ्रो द्विच॑त्वारिꣳशच्च}]}%~(३)

%3.4.4.1
चि॒त्तं च॒ चित्ति॒श्चाकू॑तं॒ चाकू॑तिश्च॒ विज्ञा॑तं च वि॒ज्ञानं॑ च॒ मन॑श्च॒ शक्व॑रीश्च॒ दर्\mbox{}श॑श्च पू॒र्णमा॑सश्च बृ॒हच्च॑ रथन्त॒रं च॑ प्र॒जा\-प॑ति॒र्जया॒निन्द्रा॑य॒ वृष्णे॒ प्राय॑च्छदु॒ग्रः पृ॑त॒नाज्ये॑षु॒ तस्मै॒ विशः॒ सम॑नमन्त॒ सर्वाः॒ स उ॒ग्रः स हि हव्यो॑ ब॒भूव॑ देवासु॒राः संय॑त्ता आस॒न्थ्स इन्द्रः॑ प्र॒जा\-प॑ति॒मुपा॑धाव॒त्तस्मा॑ ए॒ताञ्जया॒न्प्राय॑च्छ॒त्तान॑जुहो॒त्ततो॒ वै दे॒वा असु॑रानजय॒न्॒ य\-दज॑य॒न्तज्जया॑नां जय॒त्वꣴ स्पर्ध॑मानेनै॒ते हो॑त॒व्या॑ जय॑त्ये॒व तां पृत॑नाम्॥~(१५)

%3.4.5.0
{\anuvakamend[{उप॒ पञ्च॑विꣳशतिश्च}]}%~(४)

%3.4.5.1
अ॒ग्निर्भू॒ताना॒मधि॑पतिः॒ स मा॑व॒त्विन्द्रो᳚ ज्ये॒ष्ठानां᳚ य॒मः पृ॑थि॒व्या वा॒युर॒न्तरि॑क्षस्य॒ सूर्यो॑ दि॒वश्च॒न्द्रमा॒ नक्ष॑त्राणां॒ बृह॒स्पति॒र्ब्रह्म॑णो मि॒त्रः स॒त्यानां॒ वरु॑णो॒\-ऽपाꣳ स॑मु॒द्रः स्रो॒त्याना॒मन्न॒ꣳ॒ साम्रा᳚ज्याना॒मधि॑पति॒ तन्मा॑वतु॒ सोम॒ ओष॑धीनाꣳ सवि॒ता प्र॑स॒वानाꣳ॑ रु॒द्रः प॑शू॒नां त्वष्टा॑ रू॒पाणां॒ विष्णुः॒ पर्व॑तानां म॒रुतो॑ ग॒णाना॒मधि॑पतय॒स्ते मा॑वन्तु॒ पित॑रः पितामहाः परे\-ऽवरे॒ तता᳚स्ततामहा इ॒ह मा॑वत। अ॒स्मिन्ब्रह्म॑न्न॒स्मिन्क्ष॒त्रे᳚\-ऽस्यामा॒शिष्य॒स्यां पु॑रो॒धाया॑\-म॒स्मिन्कर्म॑न्न॒स्यां दे॒वहू᳚त्याम्॥~(१६)

%3.4.6.0
{\anuvakamend[{अ॒व॒रे॒ स॒प्तद॑श च}]}%~(५)

%3.4.6.1
दे॒वा वै यद्य॒ज्ञे\-ऽकु॑र्वत॒ तदसु॑रा अकुर्वत॒ ते दे॒वा ए॒तान॑भ्याता॒नान॑पश्य॒न्तान॒भ्यात॑न्वत॒ यद्दे॒वानां॒ कर्मासी॒दार्ध्य॑त॒ तद्यदसु॑राणां॒ न तदा᳚र्ध्यत॒ येन॒ कर्म॒णेर्थ्से॒त्तत्र॑ होत॒व्या॑ ऋ॒ध्नोत्ये॒व तेन॒ कर्म॑णा॒ यद्विश्वे॑ दे॒वाः स॒मभ॑र॒न्तस्मा॑दभ्याता॒ना वै᳚श्वदे॒वा यत्प्र॒जा\-प॑ति॒र्जया॒न्प्राय॑च्छ॒त्तस्मा॒ज्जयाः᳚ प्राजाप॒त्याः~(१७)

%3.4.6.2
यद्रा᳚ष्ट्र॒भृद्भी॑ रा॒ष्ट्रमाद॑दत॒ तद्रा᳚ष्ट्र॒भृताꣳ॑ राष्ट्रभृ॒त्त्वन्ते दे॒वा अ॑भ्याता॒नैरसु॑रान॒भ्यात॑न्वत॒ जयै॑रजयन्राष्ट्र॒भृद्भी॑ रा॒ष्ट्रमाद॑दत॒ यद्दे॒वा अ॑भ्याता॒नैरसु॑रान॒भ्यात॑न्वत॒ तद॑भ्याता॒नाना॑मभ्यातान॒त्वं यज्जयै॒रज॑य॒न्तज्जया॑नां जय॒त्वं यद्रा᳚ष्ट्र॒भृद्भी॑ रा॒ष्ट्रमाद॑दत॒ तद्रा᳚ष्ट्र॒भृताꣳ॑ राष्ट्रभृ॒त्त्वं ततो॑ दे॒वा अभ॑व॒न्परासु॑रा॒ यो भ्रातृ॑व्यवा॒न्थ्स्याथ्स ए॒ताञ्जु॑हुयादभ्याता॒नैरे॒व भ्रातृ॑व्यान॒भ्यात॑नुते॒ जयै᳚र्जयति राष्ट्र॒भृद्भी॑ रा॒ष्ट्रमा द॑त्ते॒ भव॑त्या॒त्मना॒ परा᳚स्य॒ भ्रातृ॑व्यो भवति॥~(१८)

%3.4.7.0
{\anuvakamend[{प्रा॒जा॒प॒त्याः सो᳚\-ऽष्टाद॑श च}]}%~(६)

%3.4.7.1
ऋ॒ता॒षाडृ॒तधा॑मा॒\-ऽग्निर्ग॑न्ध॒र्वस्तस्यौष॑धयो\-ऽफ्स॒रस॒ ऊर्जो॒ नाम॒ स इ॒दं ब्रह्म॑ क्ष॒त्रं पा॑तु॒ ता इ॒दं ब्रह्म॑ क्ष॒त्रं पा᳚न्तु॒ तस्मै॒ स्वाहा॒ ताभ्यः॒ स्वाहा॑ सꣳहि॒तो वि॒श्वसा॑मा॒ सूर्यो॑ गन्ध॒र्वस्तस्य॒ मरी॑चयो\-ऽफ्स॒रस॑ आ॒युवः॑ सुषु॒म्नः सूर्य॑रश्मिश्च॒न्द्रमा॑ गन्ध॒र्वस्तस्य॒ नक्ष॑त्राण्यफ्स॒रसो॑ बे॒कुर॑यो भु॒ज्युः सु॑प॒र्णो य॒ज्ञो ग॑न्ध॒र्वस्तस्य॒ दक्षि॑णा अफ्स॒रसः॑ स्त॒वाः प्र॒जा\-प॑तिर्वि॒श्वक॑र्मा॒ मनः॑~(१९)

%3.4.7.2
ग॒न्ध॒र्वस्तस्य॑र्ख्सा॒मान्य॑फ्स॒रसो॒ वह्न॑य इषि॒रो वि॒श्वव्य॑चा॒ वातो॑ गन्ध॒र्वस्तस्यापो᳚\-ऽफ्स॒रसो॑ मु॒दा भुव॑नस्य पते॒ यस्य॑ त उ॒परि॑ गृ॒हा इ॒ह च॑। स नो॑ रा॒स्वाज्या॑निꣳ रा॒यस्पोषꣳ॑ सु॒वीर्यꣳ॑ संवथ्स॒रीणाꣴ॑ स्व॒स्तिम्। प॒र॒मे॒ष्ठ्यधि॑पति\-र्मृ॒त्युर्ग॑न्ध॒र्वस्तस्य॒ विश्व॑मफ्स॒रसो॒ भुवः॑ सुक्षि॒तिः सुभू॑तिर्भद्र॒कृथ्सुव॑र्वान्प॒र्जन्यो॑ गन्ध॒र्वस्तस्य॑ वि॒द्युतो᳚\-ऽफ्स॒रसो॒ रुचो॑ दू॒रेहे॑तिरमृड॒यः~(२०)

%3.4.7.3
मृ॒त्युर्ग॑न्ध॒र्वस्तस्य॑ प्र॒जा अ॑फ्स॒रसो॑ भी॒रुव॒श्चारुः॑ कृपणका॒शी कामो॑ गन्ध॒र्वस्तस्या॒धयो᳚\-ऽफ्स॒रसः॑ शो॒चय॑न्ती॒र्नाम॒ स इ॒दं ब्रह्म॑ क्ष॒त्रं पा॑तु॒ ता इ॒दं ब्रह्म॑ क्ष॒त्रं पा᳚न्तु॒ तस्मै॒ स्वाहा॒ ताभ्यः॒ स्वाहा॒ स नो॑ भुवनस्य पते॒ यस्य॑ त उ॒परि॑ गृ॒हा इ॒ह च॑। उ॒रु ब्रह्म॑णे॒\-ऽस्मै क्ष॒त्राय॒ महि॒ शर्म॑ यच्छ॥~(२१)

%3.4.8.0
{\anuvakamend[{मनो॑\-ऽमृड॒यः षट्च॑त्वारिꣳशच्च}]}%~(७)

%3.4.8.1
रा॒ष्ट्रका॑माय होत॒व्या॑ रा॒ष्ट्रं वै रा᳚ष्ट्र॒भृतो॑ रा॒ष्ट्रेणै॒वास्मै॑ रा॒ष्ट्रमव॑ रुन्धे रा॒ष्ट्रमे॒व भ॑वत्या॒त्मने॑ होत॒व्या॑ रा॒ष्ट्रं वै रा᳚ष्ट्र॒भृतो॑ रा॒ष्ट्रं प्र॒जा रा॒ष्ट्रं प॒शवो॑ रा॒ष्ट्रं यच्छ्रेष्ठो॒ भव॑ति रा॒ष्ट्रेणै॒व रा॒ष्ट्रमव॑ रुन्धे॒ वसि॑ष्ठः समा॒नानां᳚ भवति॒ ग्राम॑कामाय होत॒व्या॑ रा॒ष्ट्रं वै रा᳚ष्ट्र॒भृतो॑ रा॒ष्ट्रꣳ स॑जा॒ता रा॒ष्ट्रेणै॒वास्मै॑ रा॒ष्ट्रꣳ स॑जा॒तानव॑ रुन्धे ग्रा॒मी~(२२)

%3.4.8.2
ए॒व भ॑वत्यधि॒देव॑ने जुहोत्यधि॒देव॑न ए॒वास्मै॑ सजा॒तानव॑ रुन्धे॒ त ए॑न॒मव॑रुद्धा॒ उप॑ तिष्ठन्ते रथमु॒ख ओज॑स्कामस्य होत॒व्या॑ ओजो॒ वै रा᳚ष्ट्र॒भृत॒ ओजो॒ रथ॒ ओज॑सै॒वास्मा॒ ओजो\-ऽव॑ रुन्ध ओज॒स्व्ये॑व भ॑वति॒ यो रा॒ष्ट्रादप॑भूतः॒ स्यात्तस्मै॑ होत॒व्या॑ याव॑न्तो\-ऽस्य॒ रथाः॒ स्युस्तान्ब्रू॑याद्यु॒ङ्ग्ध्वमिति॑ रा॒ष्ट्रमे॒वास्मै॑ युनक्ति~(२३)

%3.4.8.3
आहु॑तयो॒ वा ए॒तस्याकॢ॑प्ता॒ यस्य॑ रा॒ष्ट्रं न कल्प॑ते स्वर॒थस्य॒ दक्षि॑णं च॒क्रं प्र॒वृह्य॑ ना॒डीम॒भि जु॑हुया॒दाहु॑तीरे॒वास्य॑ कल्पयति॒ ता अ॑स्य॒ कल्प॑माना रा॒ष्ट्रमनु॑ कल्पते सङ्ग्रा॒मे संय॑त्ते होत॒व्या॑ रा॒ष्ट्रं वै रा᳚ष्ट्र॒भृतो॑ रा॒ष्ट्रे खलु॒ वा ए॒ते व्याय॑च्छन्ते॒ ये स॑ङ्ग्रमꣳ सं॒यन्ति॒ यस्य॒ पूर्व॑स्य॒ जुह्व॑ति॒ स ए॒व भ॑वति॒ जय॑ति॒ तं स॑ङ्ग्रमं मा᳚न्धु॒क इ॒ध्मः~(२४)

%3.4.8.4
भ॒व॒त्यङ्गा॑रा ए॒व प्र॑ति॒वेष्ट॑माना अ॒मित्रा॑णामस्य॒ सेनां॒ प्रति॑ वेष्टयन्ति॒ य उ॒न्माद्ये॒त्तस्मै॑ होत॒व्या॑ गन्धर्वाफ्स॒रसो॒ वा ए॒तमुन्मा॑दयन्ति॒ य उ॒न्माद्य॑त्ये॒ते खलु॒ वै ग॑न्धर्वाफ्स॒रसो॒ यद्रा᳚ष्ट्र॒भृत॒स्तस्मै॒ स्वाहा॒ ताभ्यः॒ स्वाहेति॑ जुहोति॒ तेनै॒वैना᳚ञ्छमयति॒ नैय॑ग्रोध॒ औदु॑म्बर॒ आश्व॑त्थः॒ प्लाक्ष॒ इती॒ध्मो भ॑वत्ये॒ते वै ग॑न्धर्वाफ्स॒रसां᳚ गृ॒हाः स्व ए॒वैनान्॑~(२५)

%3.4.8.5
आ॒यत॑ने शमयत्यभि॒चर॑ता प्रतिलो॒मꣳ हो॑त॒व्याः᳚ प्रा॒णाने॒वास्य॑ प्र॒तीचः॒ प्रति॑ यौति॒ तं ततो॒ येन॒ केन॑ च स्तृणुते॒ स्वकृ॑त॒ इरि॑णे जुहोति प्रद॒रे वै॒तद्वा अ॒स्यै निर्\mbox{}ऋ॑तिगृहीतं॒ निर्\mbox{}ऋ॑तिगृहीत ए॒वैनं॒ निर्\mbox{}ऋ॑त्या ग्राहयति॒ यद्वा॒चः क्रू॒रन्तेन॒ वष॑ट्करोति वा॒च ए॒वैनं॑ क्रू॒रेण॒ प्र वृ॑श्चति ता॒जगार्ति॒मार्च्छ॑ति॒ यस्य॑ का॒मये॑ता॒न्नाद्यम्᳚~(२६)

%3.4.8.6
आ द॑दी॒येति॒ तस्य॑ स॒भाया॑मुत्ता॒नो नि॒पद्य॒ भुव॑नस्य पत॒ इति॒ तृणा॑नि॒ सं गृ॑ह्णीयात्प्र॒जा\-प॑तिर्वै भुव॑नस्य॒ पतिः॑ प्र॒जा\-प॑तिनै॒वास्या॒न्नाद्य॒मा द॑त्त इ॒दम॒हम॒मुष्या॑मुष्याय॒णस्या॒न्नाद्यꣳ॑ हरा॒मीत्या॑हा॒न्नाद्य॑मे॒वास्य॑ हरति ष॒ड्भिर्\mbox{}ह॑रति॒ षड्वा ऋ॒तवः॑ प्र॒जा\-प॑तिनै॒वास्या॒न्नाद्य॑मा॒दाय॒र्तवो᳚\-ऽस्मा॒ अनु॒ प्र य॑च्छन्ति~(२७)

%3.4.8.7
यो ज्ये॒ष्ठब॑न्धु॒रप॑भूतः॒ स्यात्तꣴ स्थले॑\-ऽव॒साय्य॑ ब्रह्मौद॒नं चतुः॑शरावं प॒क्त्वा तस्मै॑ होत॒व्या॑ वर्ष्म॒ वै रा᳚ष्ट्र॒भृतो॒ वर्ष्म॒ स्थलं॒ वर्ष्म॑णै॒वैनं॒ वर्ष्म॑ समा॒नानां᳚ गमयति॒ चतुः॑शरावो भवति दि॒क्ष्वे॑व प्रति॑ तिष्ठति क्षी॒रे भ॑वति॒ रुच॑मे॒वास्मि॑\-न्दधा॒त्युद्ध॑रति शृत॒त्वाय॑ स॒र्पिष्वा᳚न्भवति मेध्य॒त्वाय॑ च॒त्वार॑ आर्\mbox{}षे॒याः प्राश्ञ॑न्ति दि॒शामे॒व ज्योति॑षि जुहोति॥~(२८)

%3.4.9.0
{\anuvakamend[{ग्रा॒मी यु॑नक्ती॒ध्मः स्व ए॒वैना॑न॒न्नाद्यं॑ यच्छ॒न्त्येका॒न्नप॑ञ्चा॒शच्च॑}]}%~(८)

%3.4.9.1
देवि॑का॒ निर्व॑पेत्प्र॒जाका॑म॒श्छन्दाꣳ॑सि॒ वै देवि॑का॒श्छन्दाꣳ॑सीव॒ खलु॒ वै प्र॒जाश्छन्दो॑भिरे॒वास्मै᳚ प्र॒जाः प्र ज॑नयति प्रथ॒मं धा॒तारं॑ करोति मिथु॒नी ए॒व तेन॑ करो॒त्यन्वे॒वास्मा॒ अनु॑मतिर्मन्यते रा॒ते रा॒का प्र सि॑नीवा॒ली ज॑नयति प्र॒जास्वे॒व प्रजा॑तासु कु॒ह्वा॑ वाचं॑ दधात्ये॒ता ए॒व निर्व॑पेत्प॒शुका॑म॒श्छन्दाꣳ॑सि॒ वै देवि॑का॒श्छन्दाꣳ॑सि~(२९)

%3.4.9.2
इ॒व॒ खलु॒ वै प॒शव॒श्छन्दो॑भिरे॒वास्मै॑ प॒शून्प्र ज॑नयति प्रथ॒मं धा॒तारं॑ करोति॒ प्रैव तेन॑ वापय॒त्यन्वे॒वास्मा॒ अनु॑मतिर्मन्यते रा॒ते रा॒का प्र सि॑नीवा॒ली ज॑नयति प॒शूने॒व प्रजा॑तान्कु॒ह्वा᳚ प्रति॑\-ष्ठापयत्ये॒ता ए॒व निर्व॑पे॒द्ग्राम॑काम॒श्छन्दाꣳ॑सि॒ वै देवि॑का॒श्छन्दाꣳ॑सीव॒ खलु॒ वै ग्राम॒श्छन्दो॑भिरे॒वास्मै॒ ग्रामम्᳚~(३०)

%3.4.9.3
अव॑ रुन्धे मध्य॒तो धा॒तारं॑ करोति मध्य॒त ए॒वैनं॒ ग्राम॑स्य दधात्ये॒ता ए॒व निर्व॑पे॒ज्ज्योगा॑मयावी॒ छन्दाꣳ॑सि॒ वै देवि॑का॒श्छन्दाꣳ॑सि॒ खलु॒ वा ए॒तम॒भि म॑न्यन्ते॒ यस्य॒ ज्योगा॒मय॑ति॒ छन्दो॑भिरे॒वैन॑मग॒दं क॑रोति मध्य॒तो धा॒तारं॑ करोति मध्य॒तो वा ए॒तस्याकॢ॑प्तं॒ यस्य॒ ज्योगा॒मय॑ति मध्य॒त ए॒वास्य॒ तेन॑ कल्पयत्ये॒ता ए॒व निः~(३१)

%3.4.9.4
व॒पे॒द्यं य॒ज्ञो नोप॒नमे॒च्छन्दाꣳ॑सि॒ वै देवि॑का॒श्छन्दाꣳ॑सि॒ खलु॒ वा ए॒तं नोप॑ नमन्ति॒ यं य॒ज्ञो नोप॒नम॑ति प्रथ॒मं धा॒तारं॑ करोति मुख॒त ए॒वास्मै॒ छन्दाꣳ॑सि दधा॒त्युपै॑नं य॒ज्ञो न॑मत्ये॒ता ए॒व निर्व॑पेदीजा॒नश्छन्दाꣳ॑सि॒ वै देवि॑का या॒तया॑मानीव॒ खलु॒ वा ए॒तस्य॒ छन्दाꣳ॑सि॒ य ई॑जा॒न उ॑त्त॒मं धा॒तारं॑ करोति~(३२)

%3.4.9.5
उ॒परि॑ष्टादे॒वास्मै॒ छन्दा॒ꣴ॒स्यया॑तयामा॒न्यव॑ रुन्ध॒ उपै॑न॒मुत्त॑रो य॒ज्ञो न॑मत्ये॒ता ए॒व निर्व॑पे॒द्यं मे॒धा नोप॒नमे॒च्छन्दाꣳ॑सि॒ वै देवि॑का॒श्छन्दाꣳ॑सि॒ खलु॒ वा ए॒तं नोप॑ नमन्ति॒ यं मे॒धा नोप॒नम॑ति प्रथ॒मं धा॒तारं॑ करोति मुख॒त ए॒वास्मै॒ छन्दाꣳ॑सि दधा॒त्युपै॑नं मे॒धा न॑मत्ये॒ता ए॒व निर्व॑पेत्~(३३)

%3.4.9.6
रुक्का॑म॒श्छन्दाꣳ॑सि॒ वै देवि॑का॒श्छन्दाꣳ॑सीव॒ खलु॒ वै रुक्छन्दो॑भिरे॒वास्मि॒न्रुचं॑ दधाति क्षी॒रे भ॑वन्ति॒ रुच॑मे॒वास्मि॑न्दधति मध्य॒तो धा॒तारं॑ करोति मध्य॒त ए॒वैनꣳ॑ रु॒चो द॑धाति गाय॒त्री वा अनु॑मतिस्त्रि॒ष्टुग्रा॒का जग॑ती सिनीवाल्य॑नु॒ष्टुप्कु॒हूर्धा॒ता व॑षट्का॒रः पू᳚र्वप॒क्षो रा॒काप॑रप॒क्षः कु॒हूर॑मावा॒स्या॑ सिनीवा॒ली पौ᳚र्णमा॒स्यनु॑मतिश्च॒न्द्रमा॑ धा॒ता\-ऽष्टौ~(३४)

%3.4.9.7
वस॑वो॒\-ऽष्टाक्ष॑रा गाय॒त्र्येका॑\-दश रु॒द्रा एका॑\-दशाक्षरा त्रि॒ष्टुब्द्वाद॑शादि॒त्या द्वाद॑शाक्षरा॒ जग॑ती प्र॒जा\-प॑तिरनु॒ष्टुब्धा॒ता व॑षट्का॒र ए॒तद्वै देवि॑काः॒ सर्वा॑णि च॒ छन्दाꣳ॑सि॒ सर्वा᳚श्च दे॒वता॑ वषट्का॒रस्ता यथ्स॒ह सर्वा॑ नि॒र्वपे॑दीश्व॒रा ए॑नं प्र॒दहो॒ द्वे प्र॑थ॒मे नि॒रुप्य॑ धा॒तुस्तृ॒तीयं॒ निर्व॑पे॒त्तथो॑ ए॒वोत्त॑रे॒ निर्व॑पे॒त्तथै॑नं॒ न प्र द॑ह॒न्त्यथो॒ यस्मै॒ कामा॑य निरु॒प्यन्ते॒ तमे॒वाभि॒रुपा᳚प्नोति॥~(३५)

%3.4.10.0
{\anuvakamend[{प॒शुका॑म॒श्छन्दाꣳ॑सि॒ वै देवि॑का॒श्छन्दाꣳ॑सि॒ ग्राम॑ङ्कल्पयत्ये॒ता ए॒व निरु॑त्त॒मन्धा॒तारं॑ करोति मे॒धा न॑मत्ये॒ता ए॒व निर्व॑पेद॒ष्टौ द॑हन्ति॒ नव॑ च~(९) देविकाः प्रजाकामो मिथुनी पशुकाम}]}

%3.4.10.1
वास्तो᳚ष्पते॒ प्रति॑ जानीह्य॒स्मान्थ्\-स्वा॑वे॒शो अ॑नमी॒वो भ॑वा नः। यत्त्वेम॑हे॒ प्रति॒ तन्नो॑ जुषस्व॒ शं न॑ एधि द्वि॒पदे॒ शं चतु॑ष्पदे। वास्तो᳚ष्पते श॒ग्मया॑ स॒ꣳ॒सदा॑ ते सक्षी॒महि॑ र॒ण्वया॑ गातु॒मत्या᳚। आवः॒ क्षेम॑ उ॒त योगे॒ वरं॑ नो यू॒यं पा॑त स्व॒स्तिभिः॒ सदा॑ नः। यथ्सा॒यं प्रा॑तरग्निहो॒त्रं जु॒होत्या॑हुतीष्ट॒का ए॒व ता उप॑ धत्ते~(३६)

%3.4.10.2
यज॑मानो\-ऽहोरा॒त्राणि॒ वा ए॒तस्येष्ट॑का॒ य आहि॑ताग्नि॒र्यथ्सा॒यं प्रा॑तर्जु॒होत्य॑होरा॒त्राण्ये॒वाप्त्वेष्ट॑काः कृ॒त्वोप॑ धत्ते॒ दश॑ समा॒नत्र॑ जुहोति॒ दशा᳚क्षरा वि॒राड्वि॒राज॑मे॒वाप्त्वेष्ट॑कां कृ॒त्वोप॑ ध॒त्ते\-ऽथो॑ वि॒राज्ये॒व य॒ज्ञमा᳚प्नोति॒ चित्य॑श्चित्यो\-ऽस्य भवति॒ तस्मा॒द्यत्र॒ दशो॑षि॒त्वा प्र॒याति॒ तद्य॑ज्ञवा॒स्त्ववा᳚स्त्वे॒व तद्यत्ततो᳚\-ऽर्वा॒चीनम्᳚~(३७)

%3.4.10.3
रु॒द्रः खलु॒ वै वा᳚स्तोष्प॒तिर्यदहु॑त्वा वास्तोष्प॒तीयं॑ प्रया॒याद्रु॒द्र ए॑नं भू॒त्वाऽग्निर॑नू॒त्थाय॑ हन्याद्वास्तोष्प॒तीयं॑ जुहोति भाग॒धेये॑नै॒वैनꣳ॑ शमयति॒ नार्ति॒मार्च्छ॑ति॒ यज॑मानो॒ यद्यु॒क्ते जु॑हु॒याद्यथा॒ प्रया॑ते॒ वास्ता॒वाहु॑तिं जु॒होति॑ ता॒दृगे॒व तद्यदयु॑क्ते जुहु॒याद्यथा॒ क्षेम॒ आहु॑तिं जु॒होति॑ ता॒दृगे॒व तदहु॑तमस्य वास्तोष्प॒तीयꣴ॑ स्यात्~(३८)

%3.4.10.4
दक्षि॑णो यु॒क्तो भव॑ति स॒व्यो\-ऽयु॒क्तो\-ऽथ॑ वास्तोष्प॒तीयं॑ जुहोत्यु॒भय॑मे॒वाक॒रप॑रिवर्गमे॒वैनꣳ॑ शमयति॒ यदेक॑या जुहु॒याद्द॑र्विहो॒मं कु॑र्यात्पुरोनुवा॒क्या॑म॒नूच्य॑ या॒ज्य॑या जुहोति सदेव॒त्वाय॒ यद्धु॒त आ॑द॒ध्याद्रु॒द्रं गृ॒हान॒न्वारो॑हये॒द्यद॑व॒\-क्षाणा॒न्यसं॑ प्रक्षाप्य प्रया॒याद्यथा॑ यज्ञवेश॒सं वा॒दह॑नं वा ता॒दृगे॒व तद॒यं ते॒ योनि॑र्\mbox{}ऋ॒त्विय॒ इत्य॒रण्योः᳚ स॒मारो॑हयति~(३९)

%3.4.10.5
ए॒ष वा अ॒ग्नेर्योनिः॒ स्व ए॒वैनं॒ योनौ॑ स॒मारो॑हय॒त्यथो॒ खल्वा॑हु॒र्यद॒रण्योः᳚ स॒मारू॑ढो॒ नश्ये॒दुद॑स्या॒ग्निः सी॑देत्पुनरा॒धेयः॑ स्या॒दिति॒ या ते॑ अग्ने य॒ज्ञिया॑ त॒नूस्तयेह्या रो॒हेत्या॒त्मन्थ्स॒मारो॑हयते॒ यज॑मानो॒ वा अ॒ग्नेर्योनिः॒ स्वाया॑मे॒वैनं॒ योन्याꣳ॑ स॒मारो॑हयते॥~(४०)

%3.4.11.0
{\anuvakamend[{ध॒त्ते॒\-ऽर्वा॒चीनꣴ॑ स्याथ्स॒मारो॑हयति॒ पञ्च॑चत्वारिꣳशच्च}]}%॥10॥

%3.4.11.1
त्वम॑ग्ने बृ॒हद्वयो॒ दधा॑सि देव दा॒शुषे᳚। क॒विर्गृ॒हप॑ति॒र्युवा᳚॥ ह॒व्य॒वाड॒ग्निर॒जरः॑ पि॒ता नो॑ वि॒भुर्वि॒भावा॑ सु॒दृशी॑को अ॒स्मे। सु॒गा॒र्॒\mbox{}ह॒प॒त्याः समिषो॑ दिदीह्यस्म॒द्रिय॒ख्सम्मि॑मीहि॒ श्रवाꣳ॑सि। त्वं च॑ सोम नो॒ वशो॑ जी॒वातुं॒ न म॑रामहे। प्रि॒यस्तो᳚त्रो॒ वन॒स्पतिः॑। ब्र॒ह्मा दे॒वानां᳚ पद॒वीः क॑वी॒नामृषि॒र्विप्रा॑णां महि॒षो मृ॒गाणा᳚म्। श्ये॒नो गृ॑ध्राणा॒ꣴ॒ स्वधि॑ति॒र्वना॑ना॒ꣳ॒ सोमः॑~(४१)

%3.4.11.2
प॒वित्र॒मत्ये॑ति॒ रेभन्न्॑। आ वि॒श्वदे॑व॒ꣳ॒ सत्प॑तिꣳ सू॒क्तैर॒द्या वृ॑णीमहे। स॒त्यस॑वꣳ सवि॒तारम्᳚॥ आ स॒त्येन॒ रज॑सा॒ वर्त॑मानो निवे॒शय॑न्न॒मृतं॒ मर्त्यं॑ च। हि॒र॒ण्यये॑न सवि॒ता रथे॒ना दे॒वो या॑ति॒ भुव॑ना वि॒पश्यन्न्॑। यथा॑ नो॒ अदि॑तिः॒ कर॒त्पश्वे॒ नृभ्यो॒ यथा॒ गवे᳚। यथा॑ तो॒काय॑ रु॒द्रियम्᳚। मा न॑स्तो॒के तन॑ये॒ मा न॒ आयु॑षि॒ मा नो॒ गोषु॒ मा~(४२)

%3.4.11.3
नो॒ अश्वे॑षु रीरिषः। वी॒रान्मा नो॑ रुद्र भामि॒तो व॑धीर्\mbox{}ह॒विष्म॑न्तो॒ नम॑सा विधेम ते। उ॒द॒प्रुतो॒ न वयो॒ रक्ष॑माणा॒ वाव॑दतो अ॒भ्रिय॑स्येव॒ घोषाः᳚। गि॒रि॒भ्रजो॒ नोर्मयो॒ मद॑न्तो॒ बृह॒स्पति॑म॒भ्य॑र्का अ॑नावन्न्। ह॒ꣳ॒सैरि॑व॒ सखि॑भि॒र्वाव॑दद्भिरश्म॒न्मया॑नि॒ नह॑ना॒ व्यस्यन्न्॑। बृह॒स्पति॑रभि॒ कनि॑क्रद॒द्गा उ॒त प्रास्तौ॒दुच्च॑ वि॒द्वाꣳ अ॑गायत्। एन्द्र॑ सान॒सिꣳ र॒यिम्~(४३)

%3.4.11.4
स॒जित्वा॑नꣳ सदा॒सहम्᳚। वर्\mbox{}षि॑ष्ठमू॒तये॑ भर। प्र स॑साहिषे पुरुहूत॒ शत्रू॒ञ्ज्येष्ठ॑स्ते॒ शुष्म॑ इ॒ह रा॒तिर॑स्तु। इन्द्रा भ॑र॒ दक्षि॑णेना॒ वसू॑नि॒ पतिः॒ सिन्धू॑नामसि रे॒वती॑नाम्। त्वꣳ सु॒तस्य॑ पी॒तये॑ स॒द्यो वृ॒द्धो अ॑जायथाः। इन्द्र॒ ज्यैष्ठ्या॑य सुक्रतो। भु॒वस्त्वमि॑न्द्र॒ ब्रह्म॑णा म॒हान्भुवो॒ विश्वे॑षु॒ सव॑नेषु य॒ज्ञियः॑। भुवो॒ नॄꣴश्च्यौ॒त्नो विश्व॑स्मि॒न्भरे॒ ज्येष्ठ॑श्च॒ मन्त्रः॑~(४४)

%3.4.11.5
वि॒श्व॒च॒र्॒\mbox{}ष॒णे॒। मि॒त्रस्य॑ चर्\mbox{}षणी॒धृतः॒ श्रवो॑ दे॒वस्य॑ सान॒सिम्। स॒त्यं चि॒त्रश्र॑वस्तमम्। मि॒त्रो जनान्॑ यातयति प्रजा॒नन्मि॒त्रो दा॑धार पृथि॒वीमु॒त द्याम्। मि॒त्रः कृ॒ष्टीरनि॑मिषा॒भि च॑ष्टे स॒त्याय॑ ह॒व्यं घृ॒तव॑द्विधेम। प्र स मि॑त्र॒ मर्तो॑ अस्तु॒ प्रय॑स्वा॒न्॒ यस्त॑ आदित्य॒ शिक्ष॑ति व्र॒तेन॑। न ह॑न्यते॒ न जी॑यते॒ त्वोतो॒ नैन॒मꣳहो॑ अश्ञो॒त्यन्ति॑तो॒ न दू॒रात्। यत्~(४५)

%3.4.11.6
चि॒द्धि ते॒ विशो॑ यथा॒ प्र दे॑व वरुण व्र॒तम्। मि॒नी॒मसि॒ द्यवि॑द्यवि। यत्किं चे॒दं व॑रुण॒ दैव्ये॒ जने॑\-ऽभिद्रो॒हं म॑नु॒ष्या᳚श्चरा॑मसि। अचि॑त्ती॒ यत्तव॒ धर्मा॑ युयोपि॒म मा न॒स्तस्मा॒देन॑सो देव रीरिषः। कि॒त॒वासो॒ यद्रि॑रि॒पुर्न दी॒वि यद्वा॑ घा स॒त्यमु॒त यन्न वि॒द्म। सर्वा॒ ता वि ष्य॑ शिथि॒रेव॑ दे॒वाथा॑ ते स्याम वरुण प्रि॒यासः॑॥~(४६)

%3.5.0.0
{\anuvakamend[{सोमो॒ गोषु॒ मा र॒यिं मन्त्रो॒ यच्छि॑थि॒रा स॒प्त च॑}]}%॥11॥

%3.5.0.0

{\anuvakamend[{पू॒र्णर्\mbox{}ष॑यो॒\-ऽग्निना॒ ये दे॒वाः सूर्यो॑ मा॒ सन्त्वा॑ नह्यामि वषट्का॒रः स ख॑दि॒र उ॑पया॒मगृ॑हीतो\-ऽसि॒ यां वै त्वे क्रतुं॒ प्र दे॒वमेका॑\-दश}]}%॥11॥
{\prashnaend{ पू॒र्णा स॑ह॒जान्तवा᳚ग्ने प्रा॒णैरे॒व षट्त्रिꣳ॑शत्॥36॥ पू॒र्णा सन्ति॑ दे॒वाः॥}}
%%% END PRASHNA

\sect{पञ्चमः प्रश्नः}\setcounter{anuvakam}{0}
\dnsub{तैत्तिरीयसंहितायां तृतीयकाण्डे पञ्चमः प्रश्नः}
%3.5.1.0
%3.5.1.1
पू॒र्णा प॒श्चादु॒त पू॒र्णा पु॒रस्ता॒दुन्म॑ध्य॒तः पौ᳚र्णमा॒सी जि॑गाय। तस्यां᳚ दे॒वा अधि॑ सं॒वस॑न्त उत्त॒मे नाक॑ इ॒ह मा॑दयन्ताम्। यत्ते॑ दे॒वा अद॑धुर्भाग॒धेय॒ममा॑वास्ये सं॒वस॑न्तो महि॒त्वा। सा नो॑ य॒ज्ञं पि॑पृहि विश्ववारे र॒यिं नो॑ धेहि सुभगे सु॒वीरम्᳚। नि॒वेश॑नी सं॒गम॑नी॒ वसू॑नां॒ विश्वा॑ रू॒पाणि॒ वसू᳚न्यावे॒शय॑न्ती। स॒ह॒स्र॒पो॒षꣳ सु॒भगा॒ ररा॑णा॒ सा न॒ आ ग॒न्वर्च॑सा~(१)

%3.5.1.2
सं॒वि॒दा॒ना। अग्नी॑षोमौ प्रथ॒मौ वी॒र्ये॑ण॒ वसू᳚न्रु॒द्राना॑दि॒त्यानि॒ह जि॑न्वतम्। मा॒ध्यꣳ हि पौ᳚र्णमा॒सं जु॒षेथां॒ ब्रह्म॑णा वृ॒द्धौ सु॑कृ॒तेन॑ सा॒तावथा॒स्मभ्यꣳ॑ स॒हवी॑राꣳ र॒यिं नि य॑च्छतम्। आ॒दि॒त्याश्चाङ्गि॑रसश्चा॒ग्नीनाद॑धत॒ ते द॑र्\mbox{}शपूर्णमा॒सौ प्रैफ्स॒न्तेषा॒मङ्गि॑रसां॒ निरु॑प्तꣳ ह॒विरासी॒दथा॑दि॒त्या ए॒तौ होमा॑वपश्य॒न्ताव॑जुहवु॒स्ततो॒ वै ते द॑र्\mbox{}शपूर्णमा॒सौ~(२)

%3.5.1.3
पूर्व॒ आल॑भन्त दर्\mbox{}श\-पूर्ण\-मा॒सावा॒लभ॑मान ए॒तौ होमौ॑ पु॒रस्ता᳚ज्जुहुयाथ्सा॒क्षादे॒व द॑र्\mbox{}शपूर्णमा॒सावा ल॑भते ब्रह्मवा॒दिनो॑ वदन्ति॒ स त्वै द॑र्\mbox{}शपूर्णमा॒सावा\-ल॑भेत॒ य ए॑नयोरनुलो॒मं च॑ प्रतिलो॒मं च॑ वि॒द्यादित्य॑मावा॒स्या॑या ऊ॒र्ध्वं तद॑नुलो॒मं पौ᳚र्णमा॒स्यै प्र॑ती॒चीनं॒ तत्प्र॑तिलो॒मं यत्पौ᳚र्णमा॒सीं पूर्वा॑मा॒लभे॑त प्रतिलो॒ममे॑ना॒वा ल॑भेता॒मुम॑प॒क्षीय॑माण॒मन्वप॑~(३)

%3.5.1.4
क्षी॒ये॒त॒ सा॒र॒स्व॒तौ होमौ॑ पु॒रस्ता᳚ज्जुहुयादमावा॒स्या॑ वै सर॑स्वत्यनुलो॒ममे॒वैना॒वा ल॑भते॒\-ऽमुमा॒प्याय॑मान॒मन्वा प्या॑यत आग्नावैष्ण॒वमेका॑\-दश\-कपालं पु॒रस्ता॒न्निर्व॑पे॒थ्सर॑स्वत्यै च॒रुꣳ सर॑स्वते॒ द्वाद॑श\-कपालं॒ यदा᳚ग्ने॒यो भव॑त्य॒ग्निर्वै य॑ज्ञमु॒खं य॑ज्ञमु॒खमे॒वर्द्धिं॑ पु॒रस्ता᳚द्धत्ते॒ यद्वै᳚ष्ण॒वो भव॑ति य॒ज्ञो वै विष्णु॑र्य॒ज्ञमे॒वारभ्य॒ प्र त॑नुते॒ सर॑स्वत्यै च॒रुर्भ॑वति॒ सर॑स्वते॒ द्वाद॑श\-कपालो\-ऽमावा॒स्या॑ वै सर॑स्वती पू॒र्णमा॑सः॒ सर॑स्वा॒न्तावे॒व सा॒क्षादा र॑भत ऋ॒ध्नोत्या᳚भ्यां॒ द्वाद॑श\-कपालः॒ सर॑स्वते भवति मिथुन॒त्वाय॒ प्रजा᳚त्यै मिथु॒नौ गावौ॒ दक्षि॑णा॒ समृ॑द्ध्यै॥~(४)

%3.5.2.0
{\anuvakamend[{वर्च॑सा॒ वै ते द॑र्\mbox{}शपूर्णमा॒सावप॑ तनुते॒ सर॑स्वत्यै॒ पञ्च॑विꣳशतिश्च}]}%~(१)

%3.5.2.1
ऋष॑यो॒ वा इन्द्रं॑ प्र॒त्यक्षं॒ नाप॑श्य॒न्तं वसि॑ष्ठः प्र॒त्यक्षं॑ पश्य॒थ्सो᳚\-ऽब्रवी॒द्ब्राह्म॑णं ते वक्ष्यामि॒ यथा॒ त्वत्पु॑रोहिताः प्र॒जाः प्र॑जनि॒ष्यन्ते\-ऽथ॒ मेत॑रेभ्य॒ ऋषि॑भ्यो॒ मा प्र वो॑च॒ इति॒ तस्मा॑ ए॒तान्थ्स्तोम॑भागानब्रवी॒त्ततो॒ वसि॑ष्ठपुरोहिताः प्र॒जाः प्राजा॑यन्त॒ तस्मा᳚द्वासि॒ष्ठो ब्र॒ह्मा का॒र्यः॑ प्रैव जा॑यते र॒श्मिर॑सि॒ क्षया॑य त्वा॒ क्षयं॑ जि॒न्वेति॑~(५)

%3.5.2.2
आ॒ह॒ दे॒वा वै क्षयो॑ दे॒वेभ्य॑ ए॒व य॒ज्ञं प्राऽऽह॒ प्रेति॑रसि॒ धर्मा॑य त्वा॒ धर्मं॑ जि॒न्वेत्या॑ह मनु॒ष्या॑ वै धर्मो॑ मनु॒ष्ये᳚भ्य ए॒व य॒ज्ञं प्राहान्वि॑तिरसि दि॒वे त्वा॒ दिवं॑ जि॒न्वेत्या॑है॒भ्य ए॒व लो॒केभ्यो॑ य॒ज्ञं प्राह॑ विष्ट॒म्भो॑\-ऽसि॒ वृष्ट्यै᳚ त्वा॒ वृष्टिं॑ जि॒न्वेत्या॑ह॒ वृष्टि॑मे॒वाव॑~(६)

%3.5.2.3
रु॒न्धे॒ प्र॒वास्य॑नु॒वासीत्या॑ह मिथुन॒त्वायो॒शिग॑सि॒ वसु॑भ्यस्त्वा॒ वसू᳚ञ्जि॒न्वेत्या॑हा॒ष्टौ वस॑व॒ एका॑\-दश रु॒द्रा द्वाद॑शादि॒त्या ए॒ताव॑न्तो॒ वै दे॒वास्तेभ्य॑ ए॒व य॒ज्ञं प्राहौजो॑\-ऽसि पि॒तृभ्य॑स्त्वा पि॒तॄञ्जि॒न्वेत्या॑ह दे॒वाने॒व पि॒तॄननु॒ सं त॑नोति॒ तन्तु॑रसि प्र॒जाभ्य॑स्त्वा प्र॒जा जि॑न्व~(७)

%3.5.2.4
इत्या॑ह पि॒तॄने॒व प्र॒जा अनु॒ सं त॑नोति पृतना॒षाड॑सि प॒शुभ्य॑स्त्वा प॒शूञ्जि॒न्वेत्या॑ह प्र॒जा ए॒व प॒शूननु॒ सं त॑नोति रे॒वद॒स्योष॑धीभ्य॒स्त्वौष॑धीर्जि॒न्वेत्या॒हौष॑धीष्वे॒व प॒शून्प्रति॑\-ष्ठापयत्यभि॒जिद॑सि यु॒क्तग्रा॒वेन्द्रा॑य॒ त्वेन्द्रं॑ जि॒न्वेत्या॑हा॒भिजि॑त्या॒ अधि॑पतिरसि प्रा॒णाय॑ त्वा प्रा॒णम्~(८)

%3.5.2.5
जि॒न्वेत्या॑ह प्र॒जास्वे॒व प्रा॒णान्द॑धाति त्रि॒वृद॑सि प्र॒वृद॒सीत्या॑ह मिथुन॒त्वाय॑ सꣳरो॒हो॑\-ऽसि नीरो॒हो॑\-ऽसीत्या॑ह॒ प्रजा᳚त्यै वसु॒को॑\-ऽसि॒ वेष॑श्रिरसि॒ वस्य॑ष्टिर॒सीत्या॑ह॒ प्रति॑ष्ठित्यै॥~(९)

%3.5.3.0
{\anuvakamend[{जि॒न्वेत्यव॑ प्र॒जा जि॑न्व प्रा॒णन्त्रि॒ꣳ॒शच्च॑}]}%~(२)

%3.5.3.1
अ॒ग्निना॑ दे॒वेन॒ पृत॑ना जयामि गाय॒त्रेण॒ छन्द॑सा त्रि॒वृता॒ स्तोमे॑न रथन्त॒रेण॒ साम्ना॑ वषट्का॒रेण॒ वज्रे॑ण पूर्व॒जान्भ्रातृ॑व्या॒नध॑रान्पादया॒म्यवै॑नान्बाधे॒ प्रत्ये॑नान्नुदे॒\-ऽस्मिन्क्षये॒\-ऽस्मिन्भू॑मिलो॒के यो᳚\-ऽस्मान्द्वेष्टि॒ यं च॑ व॒यं द्वि॒ष्मो विष्णोः॒ क्रमे॒णात्ये॑नान्क्रामा॒मीन्द्रे॑ण दे॒वेन॒ पृत॑ना जयामि॒ त्रैष्टु॑भेन॒ छन्द॑सा पञ्चद॒शेन॒ स्तोमे॑न बृह॒ता साम्ना॑ वषट्का॒रेण॒ वज्रे॑ण~(१०)

%3.5.3.2
स॒ह॒जान् विश्वे॑भिर्दे॒वेभिः॒ पृत॑ना जयामि॒ जाग॑तेन॒ छन्द॑सा सप्तद॒शेन॒ स्तोमे॑न वामदे॒व्येन॒ साम्ना॑ वषट्का॒रेण॒ वज्रे॑णापर॒जानिन्द्रे॑ण स॒युजो॑ व॒यꣳ सा॑स॒ह्याम॑ पृतन्य॒तः। घ्नन्तो॑ वृ॒त्राण्य॑प्र॒ति। यत्ते॑ अग्ने॒ तेज॒स्तेना॒हं ते॑ज॒स्वी भू॑यासं॒ यत्ते॑ अग्ने॒ वर्च॒स्तेना॒हं व॑च॒स्वी भू॑यासं॒ यत्ते॑ अग्ने॒ हर॒स्तेना॒हꣳ ह॑र॒स्वी भू॑यासम्॥~(११)

%3.5.4.0
{\anuvakamend[{बृ॒ह॒ता साम्ना॑ वषट्का॒रेण॒ वज्रे॑ण॒ षट्च॑त्वारिꣳशच्च}]}%~(३)

%3.5.4.1
ये दे॒वा य॑ज्ञ॒हनो॑ यज्ञ॒मुषः॑ पृथि॒व्यामध्यास॑ते। अ॒ग्निर्मा॒ तेभ्यो॑ रक्षतु॒ गच्छे॑म सु॒कृतो॑ व॒यम्। आग॑न्म मित्रावरुणा वरेण्या॒ रात्री॑णां भा॒गो यु॒वयो॒र्यो अस्ति॑। नाकं॑ गृह्णा॒नाः सु॑कृ॒तस्य॑ लो॒के तृ॒तीये॑ पृ॒ष्ठे अधि॑ रोच॒ने दि॒वः। ये दे॒वा य॑ज्ञ॒हनो॑ यज्ञ॒मुषो॒\-ऽन्तरि॒क्षे\-ऽध्यास॑ते। वा॒युर्मा॒ तेभ्यो॑ रक्षतु॒ गच्छे॑म सु॒कृतो॑ व॒यम्। यास्ते॒ रात्रीः᳚ सवितः~(१२)

%3.5.4.2
दे॒व॒यानी॑रन्त॒रा द्यावा॑पृथि॒वी वि॒यन्ति॑। गृ॒हैश्च॒ सर्वैः᳚ प्र॒जया॒ न्वग्रे॒ सुवो॒ रुहा॑णास्तरता॒ रजाꣳ॑सि। ये दे॒वा य॑ज्ञ॒हनो॑ यज्ञ॒मुषो॑ दि॒व्यध्यास॑ते। सूर्यो॑ मा॒ तेभ्यो॑ रक्षतु॒ गच्छे॑म सु॒कृतो॑ व॒यम्। येनेन्द्रा॑य स॒मभ॑रः॒ पयाꣴ॑स्युत्त॒\-मेन॑ ह॒विषा॑ जातवेदः। तेना᳚ग्ने॒ त्वमु॒त व॑र्धये॒मꣳ स॑जा॒ताना॒ꣴ॒ श्रैष्ठ्य॒ आ धे᳚ह्येनम्। य॒ज्ञ॒हनो॒ वै दे॒वा य॑ज्ञ॒मुषः॑~(१३)

%3.5.4.3
स॒न्ति॒ त ए॒षु लो॒केष्वा॑सत आ॒ददा॑ना विमथ्ना॒ना यो ददा॑ति॒ यो यज॑ते॒ तस्य॑। ये दे॒वा य॑ज्ञ॒हनः॑ पृथि॒व्यामध्यास॑ते॒ ये अ॒न्तरि॑क्षे॒ ये दि॒वीत्या॑हे॒माने॒व लो॒काꣴस्ती॒र्त्वा सगृ॑हः॒ सप॑शुः सुव॒र्गं लो॒कमे॒त्यप॒ वै सोमे॑नेजा॒नाद्दे॒वता᳚श्च य॒ज्ञश्च॑ क्रामन्त्याग्ने॒यं पञ्च॑कपालमुदवसा॒नीयं॒ निर्व॑पेद॒ग्निः सर्वा॑ दे॒वताः᳚~(१४)

%3.5.4.4
पाङ्क्तो॑ य॒ज्ञो दे॒वता᳚श्चै॒व य॒ज्ञं चाव॑ रुन्धे गाय॒त्रो वा अ॒ग्निर्गा॑य॒त्रछ॑न्दा॒स्तं छन्द॑सा॒ व्य॑र्धयति॒ यत्पञ्च॑कपालं क॒रोत्य॒ष्टाक॑पालः का॒र्यो᳚\-ऽष्टाक्ष॑रा गाय॒त्री गा॑य॒त्रो᳚\-ऽग्निर्गा॑य॒त्रछ॑न्दाः॒ स्वेनै॒वैनं॒ छन्द॑सा॒ सम॑र्धयति प॒ङ्क्त्यौ॑ याज्यानुवा॒क्ये॑ भवतः॒ पाङ्क्तो॑ य॒ज्ञस्तेनै॒व य॒ज्ञान्नैति॑॥~(१५)

%3.5.5.0
{\anuvakamend[{स॒वि॒त॒र्दे॒वा य॑ज्ञ॒मुषः॒ सर्वा॑ दे॒वता॒स्त्रिच॑त्वारिꣳशच्च}]}%~(४)

%3.5.5.1
सूर्यो॑ मा दे॒वो दे॒वेभ्यः॑ पातु वा॒युर॒न्तरि॑क्षा॒द्यज॑मानो॒\-ऽग्निर्मा॑ पातु॒ चक्षु॑षः। सक्ष॒ शूष॒ सवि॑त॒र्विश्व॑चर्\mbox{}षण ए॒तेभिः॑ सोम॒ नाम॑भिर्विधेम ते॒ तेभिः॑ सोम॒ नाम॑भिर्विधेम ते। अ॒हं प॒रस्ता॑द॒हम॒वस्ता॑द॒हं ज्योति॑षा॒ वि तमो॑ ववार। यद॒न्तरि॑क्षं॒ तदु॑ मे पि॒ताभू॑द॒हꣳ सूर्य॑मुभ॒यतो॑ ददर्\mbox{}शा॒हं भू॑यासमुत्त॒मः स॑मा॒नाना᳚म्~(१६)

%3.5.5.2
आ स॑मु॒द्रादा\-ऽन्तरि॑क्षात्प्र॒जा\-प॑तिरुद॒धिं च्या॑वया॒तीन्द्रः॒ प्र स्नौ॑तु म॒रुतो॑ वर्\mbox{}षय॒न्तून्न॑म्भय पृथि॒वीं भि॒न्द्धीदं दि॒व्यं नभः॑। उ॒द्नो दि॒व्यस्य॑ नो दे॒हीशा॑नो॒ वि सृ॑जा॒ दृतिम्᳚। प॒शवो॒ वा ए॒ते यदा॑दि॒त्य ए॒ष रु॒द्रो यद॒ग्निरोष॑धीः॒ प्रास्या॒ग्नावा॑दि॒त्यं जु॑होति रु॒द्रादे॒व प॒शून॒न्तर्द॑धा॒त्यथो॒ ओष॑धीष्वे॒व प॒शून्~(१७)

%3.5.5.3
प्रति॑\-ष्ठापयति क॒विर्य॒ज्ञस्य॒ वि त॑नोति॒ पन्थां॒ नाक॑स्य पृ॒ष्ठे अधि॑ रोच॒ने दि॒वः। येन॑ ह॒व्यं वह॑सि॒ यासि॑ दू॒त इ॒तः प्रचे॑ता अ॒मुतः॒ सनी॑यान्। यास्ते॒ विश्वाः᳚ स॒मिधः॒ सन्त्य॑ग्ने॒ याः पृ॑थि॒व्यां ब॒र्॒\mbox{}हिषि॒ सूर्ये॒ याः। तास्ते॑ गच्छ॒न्त्वाहु॑तिं घृ॒तस्य॑ देवाय॒ते यज॑मानाय॒ शर्म॑। आ॒शासा॑नः सु॒वीर्यꣳ॑ रा॒यस्पोष॒ꣴ॒ स्वश्वि॑यम्। बृह॒स्पति॑ना रा॒या स्व॒गाकृ॑तो॒ मह्यं॒ यज॑मानाय तिष्ठ॥~(१८)

%3.5.6.0
{\anuvakamend[{स॒मा॒नाना॒मोष॑धीष्वे॒व प॒शून्मह्यं॒ यज॑माना॒यैक॑ञ्च}]}%~(५)

%3.5.6.1
सं त्वा॑ नह्यामि॒ पय॑सा घृ॒तेन॒ सं त्वा॑ नह्याम्य॒प ओष॑धीभिः। सं त्वा॑ नह्यामि प्र॒जया॒हम॒द्य सा दी᳚क्षि॒ता स॑नवो॒ वाज॑म॒स्मे। प्रैतु॒ ब्रह्म॑ण॒स्पत्नी॒ वेदिं॒ वर्णे॑न सीदतु। अथा॒हम॑नुका॒मिनी॒ स्वे लो॒के वि॒शा इ॒ह। सु॒प्र॒जस॑स्त्वा व॒यꣳ सु॒पत्नी॒रुप॑ सेदिम। अग्ने॑ सपत्न॒दं भ॑न॒मद॑ब्धासो॒ अदा᳚भ्यम्। इ॒मं वि ष्या॑मि॒ वरु॑णस्य॒ पाशम्᳚~(१९)

%3.5.6.2
यमब॑ध्नीत सवि॒ता सु॒केतः॑। धा॒तुश्च॒ योनौ॑ सुकृ॒तस्य॑ लो॒के स्यो॒नं मे॑ स॒ह पत्या॑ करोमि। प्रेह्यु॒देह्यृ॒तस्य॑ वा॒मीरन्व॒ग्निस्ते\-ऽग्रं॑ नय॒त्वदि॑ति॒र्मध्यं॑ ददताꣳ रु॒द्राव॑सृष्टासि यु॒वा नाम॒ मा मा॑ हिꣳसी॒र्वसु॑भ्यो रु॒द्रेभ्य॑ आदि॒त्येभ्यो॒ विश्वे᳚भ्यो वो दे॒वेभ्यः॑ प॒न्नेज॑नीर्गृह्णामि य॒ज्ञाय॑ वः प॒न्नेज॑नीः सादयामि॒ विश्व॑स्य ते॒ विश्वा॑वतो॒ वृष्णि॑यावतः~(२०)

%3.5.6.3
तवा᳚ग्ने वा॒मीरनु॑ स॒न्दृशि॒ विश्वा॒ रेताꣳ॑सि धिषी॒यागं॑ दे॒वान् य॒ज्ञो नि दे॒वीर्दे॒वेभ्यो॑ य॒ज्ञम॑शिषन्न॒स्मिन्थ्सु॑न्व॒ति यज॑मान आ॒शिषः॒ स्वाहा॑कृताः समुद्रे॒ष्ठा ग॑न्ध॒र्वमा ति॑ष्ठ॒ता\-ऽनु॑। वात॑स्य॒ पत्म॑न्नि॒ड ई॑डि॒ताः॥~(२१)

%3.5.7.0
{\anuvakamend[{पाशं॒ वृष्णि॑यावतस्त्रि॒ꣳ॒शच्च॑}]}%~(६)

%3.5.7.1
व॒ष॒ट्का॒रो वै गा॑यत्रि॒यै शिरो᳚\-ऽच्छिन॒त्तस्यै॒ रसः॒ परा॑पत॒थ्स पृ॑थि॒वीं प्रावि॑श॒थ्स ख॑दि॒रो॑\-ऽभव॒द्यस्य॑ खादि॒रः स्रु॒वो भव॑ति॒ छन्द॑सामे॒व रसे॒नाव॑ द्यति॒ सर॑सा अ॒स्याहु॑तयो भवन्ति तृ॒तीय॑स्यामि॒तो दि॒वि सोम॑ आसी॒त्तं गा॑य॒त्र्याह॑र॒त्तस्य॑ प॒र्णम॑च्छिद्यत॒ तत्प॒र्णो॑\-ऽभव॒त्तत्प॒र्णस्य॑ पर्ण॒त्वं यस्य॑ पर्ण॒मयी॑ जु॒हूः~(२२)

%3.5.7.2
भव॑ति सौ॒म्या अ॒स्याहु॑तयो भवन्ति जु॒षन्ते᳚\-ऽस्य दे॒वा आहु॑तीर्दे॒वा वै ब्रह्म॑न्नवदन्त॒ तत्प॒र्ण उपा॑शृणोथ्सु॒श्रवा॒ वै नाम॒ यस्य॑ पर्ण॒मयी॑ जु॒हूर्भव॑ति॒ न पा॒पꣴ श्लोकꣳ॑ शृणोति॒ ब्रह्म॒ वै प॒र्णो विण्म॒रुतो\-ऽन्नं॒ विण्मा॑रु॒तो᳚\-ऽश्व॒त्थो यस्य॑ पर्ण॒मयी॑ जु॒हूर्भव॒त्याश्व॑त्थ्युप॒भृद्ब्रह्म॑णै॒वान्न॒मव॑ रु॒न्धे\-ऽथो॒ ब्रह्म॑~(२३)

%3.5.7.3
ए॒व वि॒श्यध्यू॑हति रा॒ष्ट्रं वै प॒र्णो विड॑श्व॒त्थो यत्प॑र्ण॒मयी॑ जु॒हूर्भव॒त्याश्व॑त्थ्युप॒भृद्रा॒ष्ट्रमे॒व वि॒श्यध्यू॑हति प्र॒जा\-प॑ति॒र्वा अ॑जुहो॒थ्सा यत्राहु॑तिः प्र॒त्यति॑ष्ठ॒त्ततो॒ विक॑ङ्कत॒ उद॑तिष्ठ॒त्ततः॑ प्र॒जा अ॑सृजत॒ यस्य॒ वैक॑ङ्कती ध्रु॒वा भव॑ति॒ प्रत्ये॒वास्याहु॑तयस्तिष्ठ॒न्त्यथो॒ प्रैव जा॑यत ए॒तद्वै स्रु॒चाꣳ रू॒पं यस्यै॒वꣳरू॑पाः॒ स्रुचो॒ भव॑न्ति॒ सर्वा᳚ण्ये॒वैनꣳ॑ रू॒पाणि॑ पशू॒नामुप॑ तिष्ठन्ते॒ नास्याप॑रूपमा॒त्मञ्जा॑यते॥~(२४)

%3.5.8.0
{\anuvakamend[{जु॒हूरथो॒ ब्रह्म॑ स्रु॒चाꣳ स॒प्तद॑श च}]}%~(७)

%3.5.8.1
उ॒प॒या॒मगृ॑हीतो\-ऽसि प्र॒जा\-प॑तये त्वा॒ ज्योति॑ष्मते॒ ज्योति॑ष्मन्तं गृह्णामि॒ दक्षा॑य दक्ष॒वृधे॑ रा॒तं दे॒वेभ्यो᳚\-ऽग्निजि॒ह्वेभ्य॑\-स्त्वर्ता॒युभ्य॒ इन्द्र॑ज्येष्ठेभ्यो॒ वरु॑णराजभ्यो॒ वाता॑पिभ्यः प॒र्जन्या᳚त्मभ्यो दि॒वे त्वा॒न्तरि॑क्षाय त्वा पृथि॒व्यै त्वापे᳚न्द्र द्विष॒तो मनो\-ऽप॒ जिज्या॑सतो ज॒ह्यप॒ यो नो॑\-ऽराती॒यति॒ तं ज॑हि प्रा॒णाय॑ त्वापा॒नाय॑ त्वा व्या॒नाय॑ त्वा स॒ते त्वास॑ते त्वा॒द्भ्यस्त्वौष॑धीभ्यो॒ विश्वे᳚भ्यस्त्वा भू॒तेभ्यो॒ यतः॑ प्र॒जा अक्खि॑द्रा॒ अजा॑यन्त॒ तस्मै᳚ त्वा प्र॒जा\-प॑तये विभू॒दाव्ने॒ ज्योति॑ष्मते॒ ज्योति॑ष्मन्तं जुहोमि॥~(२५)

%3.5.9.0
{\anuvakamend[{ओष॑धीभ्य॒श्चतु॑र्दश च}]}%~(८)

%3.5.9.1
यां वा अ॑ध्व॒र्युश्च॒ यज॑मानश्च दे॒वता॑मन्तरि॒तस्तस्या॒ आ वृ॑श्च्येते प्राजाप॒त्यं द॑धिग्र॒हं गृ॑ह्णीयात्प्र॒जा\-प॑तिः॒ सर्वा॑ दे॒वता॑ दे॒वता᳚भ्य ए॒व नि ह्नु॑वाते ज्ये॒ष्ठो वा ए॒ष ग्रहा॑णां॒ यस्यै॒ष गृ॒ह्यते॒ ज्यैष्ठ्य॑मे॒व ग॑च्छति॒ सर्वा॑सां॒ वा ए॒तद्दे॒वता॑नाꣳ रू॒पं यदे॒ष ग्रहो॒ यस्यै॒ष गृ॒ह्यते॒ सर्वा᳚ण्ये॒वैनꣳ॑ रू॒पाणि॑ पशू॒नामुप॑ तिष्ठन्त उपया॒मगृ॑हीतः~(२६)

%3.5.9.2
अ॒सि॒ प्र॒जा\-प॑तये त्वा॒ ज्योति॑ष्मते॒ ज्योति॑ष्मन्तं गृह्णा॒मीत्या॑ह॒ ज्योति॑रे॒वैनꣳ॑ समा॒नानां᳚ करोत्यग्निजि॒ह्वेभ्य॑स्त्वर्ता॒युभ्य॒ इत्या॑है॒ताव॑ती॒र्वै दे॒वता॒स्ताभ्य॑ ए॒वैन॒ꣳ॒ सर्वा᳚भ्यो गृह्णा॒त्यपे᳚न्द्र द्विष॒तो मन॒ इत्या॑ह॒ भ्रातृ॑व्यापनुत्त्यै प्रा॒णाय॑ त्वापा॒नाय॒ त्वेत्या॑ह प्रा॒णाने॒व यज॑माने दधाति॒ तस्मै᳚ त्वा प्र॒जा\-प॑तये विभू॒दाव्ने॒ ज्योति॑ष्मते॒ ज्योति॑ष्मन्तं जुहोमि~(२७)

%3.5.9.3
इत्या॑ह प्र॒जा\-प॑तिः॒ सर्वा॑ दे॒वताः॒ सर्वा᳚भ्य ए॒वैनं॑ दे॒वता᳚भ्यो जुहोत्याज्यग्र॒हं गृ॑ह्णीया॒त्तेज॑स्कामस्य॒ तेजो॒ वा आज्यं॑ तेज॒स्व्ये॑व भ॑वति सोमग्र॒हं गृ॑ह्णीयाद्ब्रह्मवर्च॒सका॑मस्य ब्रह्मवर्च॒सं वै सोमो᳚ ब्रह्मवर्च॒स्ये॑व भ॑वति दधिग्र॒हं गृ॑ह्णीयात्प॒शुका॑म॒स्योर्ग्वै दध्यूर्क्प॒शव॑ ऊ॒र्जैवास्मा॒ ऊर्जं॑ प॒शूनव॑ रुन्धे॥~(२८)

%3.5.10.0
{\anuvakamend[{उ॒प॒या॒मगृ॑हीतो जुहोमि॒ त्रिच॑त्वारिꣳशच्च}]}%~(९)

%3.5.10.1
त्वे क्रतु॒मपि॑ वृञ्जन्ति॒ विश्वे॒ द्विर्यदे॒ते त्रिर्भव॒न्त्यूमाः᳚। स्वा॒दोः स्वादी॑यः स्वा॒दुना॑ सृजा॒ समत॑ ऊ॒ षु मधु॒ मधु॑ना॒भि यो॑धि। उ॒प॒या॒मगृ॑हीतो\-ऽसि प्र॒जा\-प॑तये त्वा॒ जुष्टं॑ गृह्णाम्ये॒ष ते॒ योनिः॑ प्र॒जा\-प॑तये त्वा। प्रा॒ण॒ग्र॒हान्गृ॑ह्णात्ये॒ताव॒द्वा अ॑स्ति॒ याव॑दे॒ते ग्रहाः॒ स्तोमा॒श्छन्दाꣳ॑सि पृ॒ष्ठानि॒ दिशो॒ याव॑दे॒वास्ति॒ तत्~(२९)

%3.5.10.2
अव॑ रुन्धे ज्ये॒ष्ठा वा ए॒तान्ब्रा᳚ह्म॒णाः पु॒रा वि॒द्वाम॑क्र॒न्तस्मा॒त्तेषा॒ꣳ॒ सर्वा॒ दिशो॒\-ऽभिजि॑ता अभूव॒न्॒ यस्यै॒ते गृ॒ह्यन्ते॒ ज्यैष्ठ्य॑मे॒व ग॑च्छत्य॒भि दिशो॑ जयति॒ पञ्च॑ गृह्यन्ते॒ पञ्च॒ दिशः॒ सर्वा᳚स्वे॒व दि॒क्ष्वृ॑ध्नुवन्ति॒ नव॑नव गृह्यन्ते॒ नव॒ वै पुरु॑षे प्रा॒णाः प्रा॒णाने॒व यज॑मानेषु दधति प्राय॒णीये॑ चोदय॒नीये॑ च गृह्यन्ते प्रा॒णा वै प्रा॑णग्र॒हाः~(३०)

%3.5.10.3
प्रा॒णैरे॒व प्र॒यन्ति॑ प्रा॒णैरुद्य॑न्ति दश॒मे\-ऽह॑न्गृह्यन्ते प्रा॒णा वै प्रा॑णग्र॒हाः प्रा॒णेभ्यः॒ खलु॒ वा ए॒तत्प्र॒जा य॑न्ति॒ यद्वा॑मदे॒व्यं योने॒श्च्यव॑ते दश॒मे\-ऽह॑न्वामदे॒व्यं योने᳚श्च्यवते॒ यद्द॑श॒मे\-ऽह॑न्गृ॒ह्यन्ते᳚ प्रा॒णेभ्य॑ ए॒व तत्प्र॒जा न य॑न्ति॥~(३१)

%3.5.11.0
{\anuvakamend[{तत्प्रा॑णग्र॒हाः स॒प्तत्रिꣳ॑शच्च}]}%॥10॥

%3.5.11.1
प्र दे॒वं दे॒व्या धि॒या भर॑ता जा॒तवे॑दसम्। ह॒व्या नो॑ वक्षदानु॒षक्। अ॒यमु॒ ष्य प्र दे॑व॒युर्\mbox{}होता॑ य॒ज्ञाय॑ नीयते। रथो॒ न योर॒भीवृ॑तो॒ घृणी॑वाञ्चेतति॒ त्मना᳚। अ॒यम॒ग्निरु॑रुष्यत्य॒मृता॑दिव॒ जन्म॑नः। सह॑सश्चि॒थ्सही॑यां दे॒वो जी॒वात॑वे कृ॒तः। इडा॑यास्त्वा प॒दे व॒यं नाभा॑ पृथि॒व्या अधि॑। जात॑वेदो॒ नि धी॑म॒ह्यग्ने॑ ह॒व्याय॒ वोढ॑वे।~(३२)

%3.5.11.2
अग्ने॒ विश्वे॑भिः स्वनीक दे॒वैरूर्णा॑वन्तं प्रथ॒मः सी॑द॒ योनिम्᳚। कु॒ला॒यिनं॑ घृ॒तव॑न्तꣳ सवि॒त्रे य॒ज्ञं न॑य॒ यज॑मानाय सा॒धु। सीद॑ होतः॒ स्व उ॑ लो॒के चि॑कि॒त्वान्थ्सा॒दया॑ य॒ज्ञꣳ सु॑कृ॒तस्य॒ योनौ᳚। दे॒वा॒वीर्दे॒वान् ह॒विषा॑ यजा॒स्यग्ने॑ बृ॒हद्यज॑माने॒ वयो॑ धाः। नि होता॑ होतृ॒षद॑ने॒ विदा॑नस्त्वे॒षो दी॑दि॒वाꣳ अ॑सदथ्सु॒दक्षः॑। अद॑ब्धव्रतप्रमति॒र्वसि॑ष्ठः सहस्रं भ॒रः शुचि॑जिह्वो अ॒ग्निः। त्वं दू॒तस्त्वम्~(३३)

%3.5.11.3
उ॒ नः॒ प॒र॒स्पास्त्वं वस्य॒ आ वृ॑षभ प्रणे॒ता। अग्ने॑ तो॒कस्य॑ न॒स्तने॑ त॒नूना॒मप्र॑युच्छ॒न्दीद्य॑द्बोधि गो॒पाः। अ॒भि त्वा॑ देव सवित॒रीशा॑नं॒ वार्या॑णाम्। सदा॑वन्भा॒गमी॑महे। म॒ही द्यौः पृ॑थि॒वी च॑ न इ॒मं य॒ज्ञं मि॑मिक्षताम्। पि॒पृ॒तां नो॒ भरी॑मभिः। त्वाम॑ग्ने॒ पुष्क॑रा॒दध्यथ॑र्वा॒ निर॑मन्थत। मू॒र्ध्नो विश्व॑स्य वा॒घतः॑। तमु॑~(३४)

%3.5.11.4
त्वा॒ द॒ध्यङ्ङृषिः॑ पु॒त्र ई॑धे॒ अथ॑र्वणः। वृ॒त्र॒हणं॑ पुरन्द॒रम्। तमु॑ त्वा पा॒थ्यो वृषा॒ समी॑धे दस्यु॒हन्त॑मम्। ध॒नं॒ज॒यꣳ रणे॑रणे। उ॒त ब्रु॑वन्तु ज॒न्तव॒ उद॒ग्निर्वृ॑त्र॒हाज॑नि। ध॒नं॒ज॒यो रणे॑रणे। आ यꣳ हस्ते॒ न खा॒दिन॒ꣳ॒ शिशुं॑ जा॒तं न बिभ्र॑ति। वि॒शाम॒ग्निꣴ स्व॑ध्व॒रम्। प्र दे॒वं दे॒ववी॑तये॒ भर॑ता वसु॒वित्त॑मम्। आ स्वे योनौ॒ नि षी॑दतु। आ~(३५)


%3.5.11.5
जा॒तं जा॒तवे॑दसि प्रि॒यꣳ शि॑शी॒ताति॑थिम्। स्यो॒न आ गृ॒हप॑तिम्। अ॒ग्निना॒ऽग्निः समि॑ध्यते क॒विर्गृ॒हप॑ति॒र्युवा᳚। ह॒व्य॒वाड्जु॒ह्वा᳚स्यः। त्वꣴ ह्य॑ग्ने अ॒ग्निना॒ विप्रो॒ विप्रे॑ण॒ सन्थ्स॒ता। सखा॒ सख्या॑ समि॒ध्यसे᳚। तं म॑र्जयन्त सु॒क्रतुं॑ पुरो॒यावा॑नमा॒जिषु॑। स्वेषु॒ क्षये॑षु वा॒जिनम्᳚। य॒ज्ञेन॑ य॒ज्ञम॑यजन्त दे॒वास्तानि॒ धर्मा॑णि प्रथ॒मान्या॑सन्न्। ते ह॒ नाकं॑ महि॒मानः॑ सचन्ते॒ यत्र॒ पूर्वे॑ सा॒ध्याः सन्ति॑ दे॒वाः~(३६)
{\anuvakamend[{वोढ॑वे दू॒तस्त्वन्तमु॑ सीद॒त्वा यत्र॑ च॒त्वारि॑ च}]}%॥11॥

%%% END PRASHNA
%%% END KANDAM
