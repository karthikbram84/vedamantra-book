\chapt{काण्डम् ७}
\sect{चतुर्थः प्रश्नः}\setcounter{anuvakam}{0}
\dnsub{तैत्तिरीयसंहितायां सप्तमकाण्डे चतुर्थः प्रश्नः}
%7.4.1.1
बृह॒स्पति॑रकामयत॒ श्रन्मे॑ दे॒वा दधी॑र॒न्गच्छे॑यं पुरो॒धामिति॒ स ए॒तं च॑तुर्विꣳशतिरा॒त्रम॑पश्य॒त्तमाह॑र॒त्तेना॑यजत॒ ततो॒ वै तस्मै॒ श्रद्दे॒वा अद॑ध॒ताग॑च्छत्पुरो॒धां य ए॒वं वि॒द्वाꣳस॑श्चतुर्विꣳशतिरा॒त्रमास॑ते॒ श्रदे᳚भ्यो मनु॒ष्या॑ दधते॒ गच्छ॑न्ति पुरो॒धां ज्योति॒र्गौरायु॒रिति॑ त्र्य॒हा भ॑वन्ती॒यं वाव ज्योति॑र॒न्तरि॑क्षं॒ गौर॒सावायुः॑~(१)

%7.4.1.2
इ॒माने॒व लो॒कान॒भ्यारो॑हन्त्यभिपू॒र्वं त्र्य॒हा भ॑वन्त्यभिपू॒र्वमे॒व सु॑व॒र्गं लो॒कम॒भ्यारो॑ह॒न्त्यस॑त्त्रं॒ वा ए॒तद्यद॑छन्दो॒मं यच्छ॑न्दो॒मा भव॑न्ति॒ तेन॑ स॒त्रं दे॒वता॑ ए॒व पृ॒ष्ठैरव॑ रुन्धते प॒शूञ्छ॑न्दो॒मैरोजो॒ वै वी॒र्यं॑ पृ॒ष्ठानि॑ प॒शव॑श्छन्दो॒मा ओज॑स्ये॒व वी॒र्ये॑ प॒शुषु॒ प्रति॑ तिष्ठन्ति बृहद्रथन्त॒रा\-भ्यां᳚ यन्ती॒यं वाव र॑थन्त॒रम॒सौ बृ॒हदा॒भ्यामे॒व~(२)

%7.4.1.3
य॒न्त्यथो॑ अ॒नयो॑रे॒व प्रति॑ तिष्ठन्त्ये॒ते वै य॒ज्ञस्या᳚ञ्ज॒साय॑नी स्रु॒ती ताभ्या॑मे॒व सु॑व॒र्गं लो॒कं य॑न्ति चतुर्विꣳशतिरा॒त्रो भ॑वति॒ चतु॑र्विꣳशतिरर्धमा॒साः सं॑वथ्स॒रः सं॑वथ्स॒रः सु॑व॒र्गो लो॒कः सं॑वथ्स॒र ए॒व सु॑व॒र्गे लो॒के प्रति॑ तिष्ठ॒न्त्यथो॒ चतु॑र्विꣳशत्यक्षरा गाय॒त्री गा॑य॒त्री ब्र॑ह्मवर्च॒सङ्गा॑यत्रि॒यैव ब्र॑ह्मवर्च॒समव॑ रुन्धते\-ऽतिरा॒त्राव॒भितो॑ भवतो ब्रह्मवर्च॒सस्य॒ परि॑गृहीत्यै॥~(३)

{\anuvakamend[{अ॒सावायु॑रा॒भ्यामे॒व पञ्च॑चत्वारिꣳशच्च}]}%~(१)

%7.4.2.1
यथा॒ वै म॑नु॒ष्या॑ ए॒वं दे॒वा अग्र॑ आस॒न्ते॑\-ऽकामय॒न्ताव॑र्तिं पा॒प्मानं॑ मृ॒त्युम॑प॒हत्य॒ दैवीꣳ॑ स॒ꣳ॒सदं॑ गच्छे॒मेति॒ त ए॒तं च॑तुर्विꣳशतिरा॒त्रम॑पश्य॒न्तमाह॑र॒न्तेना॑यजन्त॒ ततो॒ वै ते\-ऽव॑र्तिं पा॒प्मानं॑ मृ॒त्युम॑प॒हत्य॒ दैवीꣳ॑ स॒ꣳ॒सद॑मगच्छ॒न्॒ य ए॒वं वि॒द्वाꣳ॑सश्चतुर्विꣳशतिरा॒त्रमास॒ते\-ऽव॑र्तिमे॒व पा॒प्मान॑मप॒हत्य॒ श्रियं॑ गच्छन्ति॒ श्रीर्\mbox{}हि म॑नु॒ष्य॑स्य~(४)

%7.4.2.2
दैवी॑ स॒ꣳ॒सज्ज्योति॑रतिरा॒त्रो भ॑वति सुव॒र्गस्य॑ लो॒कस्यानु॑ख्यात्यै॒ पृष्ठ्यः॑ षड॒हो भ॑वति॒ षड्वा ऋ॒तवः॑ संवथ्स॒रस्तं मासा॑ अर्धमा॒सा ऋ॒तवः॑ प्र॒विश्य॒ दैवीꣳ॑ स॒ꣳ॒सद॑मगच्छ॒न्॒ य ए॒वं वि॒द्वाꣳ॑सश्चतुर्विꣳशतिरा॒त्रमास॑ते संवथ्स॒रमे॒व प्र॒विश्य॒ वस्य॑सीꣳ स॒ꣳ॒सदं॑ गच्छन्ति॒ त्रय॑स्त्रयस्त्रि॒ꣳ॒शा अ॒वस्ता᳚द्भवन्ति॒ त्रय॑स्त्रयस्त्रि॒ꣳ॒शाः प॒रस्ता᳚त्त्रयस्त्रि॒ꣳ॒शैरे॒वोभ॒यतो\-ऽव॑र्तिं पा॒प्मान॑मप॒हत्य॒ दैवीꣳ॑ स॒ꣳ॒सदं॑ मध्य॒तः~(५)

%7.4.2.3
ग॒च्छ॒न्ति॒ पृ॒ष्ठानि॒ हि दैवी॑ स॒ꣳ॒सज्जा॒मि वा ए॒तत्कु॑र्वन्ति॒ यत्त्रय॑स्त्रयस्त्रि॒ꣳ॒शा अ॒न्वञ्चो॒ मध्ये\-ऽनि॑रुक्तो भवति॒ तेनाजा᳚म्यू॒र्ध्वानि॑ पृ॒ष्ठानि॑ भवन्त्यू॒र्ध्वाश्छ॑न्दो॒मा उ॒भाभ्याꣳ॑ रू॒पाभ्याꣳ॑ सुव॒र्गं लो॒कं य॒न्त्यस॑त्त्रं॒ वा ए॒तद्यद॑छन्दो॒मं यच्छ॑न्दो॒मा भव॑न्ति॒ तेन॑ स॒त्रं दे॒वता॑ ए॒व पृ॒त्ष्ठैरव॑ रुन्धते प॒शूञ्छ॑न्दो॒मैरोजो॒ वै वी॒र्यं॑ पृ॒ष्ठानि॑ प॒शवः॑~(६)

%7.4.2.4
छ॒न्दो॒मा ओज॑स्ये॒व वी॒र्ये॑ प॒शुषु॒ प्रति॑ तिष्ठन्ति॒ त्रय॑स्त्रयस्त्रि॒ꣳ॒शा अ॒वस्ता᳚द्भवन्ति॒ त्रय॑स्त्रयस्त्रि॒ꣳ॒शाः प॒रस्ता॒न्मध्ये॑ पृ॒ष्ठान्युरो॒ वै त्र॑यस्त्रि॒ꣳ॒शा आ॒त्मा पृ॒ष्ठान्या॒त्मन॑ ए॒व तद्यज॑मानाः॒ शर्म॑ नह्य॒न्ते\-ऽना᳚र्त्यै बृहद्रथन्त॒रा\-भ्यां᳚ यन्ती॒यं वाव र॑थन्त॒रम॒सौ बृ॒हदा॒भ्यामे॒व य॒न्त्यथो॑ अ॒नयो॑रे॒व प्रति॑ तिष्ठन्त्ये॒ते वै य॒ज्ञस्या᳚ञ्ज॒साय॑नी स्रु॒ती ताभ्या॑मे॒व~(७)

%7.4.2.5
सु॒व॒र्गं लो॒कं य॑न्ति॒ परा᳚ञ्चो॒ वा ए॒ते सु॑व॒र्गं लो॒कम॒भ्यारो॑हन्ति॒ ये प॑रा॒चीना॑नि पृ॒ष्ठान्यु॑प॒यन्ति॑ प्र॒त्यङ्क्ष॑ड॒हो भ॑वति प्र॒त्यव॑रूढ्या॒ अथो॒ प्रति॑ष्ठित्या उ॒भयो᳚र्लो॒कयोर्॑\mbox{}ऋ॒द्ध्वोत्ति॑ष्ठन्ति त्रि॒वृतो\-ऽधि॑ त्रि॒वृत॒मुप॑ यन्ति॒ स्तोमा॑ना॒ꣳ॒ सम्प॑त्त्यै प्रभ॒वाय॒ ज्योति॑रग्निष्टो॒मो भ॑वत्य॒यं वाव स क्षयो॒\-ऽस्मादे॒व तेन॒ क्षया॒न्न य॑न्ति चतुर्विꣳशतिरा॒त्रो भ॑वति॒ चतु॑र्विꣳशतिरर्धमा॒साः सं॑वथ्स॒रः सं॑वथ्स॒रः सु॑व॒र्गो लो॒कः सं॑वथ्स॒र ए॒व सु॑व॒र्गे लो॒के प्रति॑ तिष्ठ॒न्त्यथो॒ चतु॑र्विꣳशत्यक्षरा गाय॒त्री गा॑य॒त्री ब्र॑ह्मवर्च॒सङ्गा॑यत्रि॒यैव ब्र॑ह्मवर्च॒समव॑ रुन्धते\-ऽतिरा॒त्राव॒भितो॑ भवतो ब्रह्मवर्च॒सस्य॒ परि॑गृहीत्यै॥~(८)

{\anuvakamend[{म॒नु॒ष्य॑स्य मध्य॒तः प॒शव॒स्ताभ्या॑मे॒व सं॑वथ्स॒रश्चतु॑र्विꣳशतिश्च}]}%~(२)

%7.4.3.1
ऋ॒क्षा वा इ॒यम॑लो॒मका॑सी॒थ्साका॑मय॒तौष॑धीभि॒र्वन॒स्पति॑भिः॒ प्र जा॑ये॒येति॒ सैतास्त्रि॒ꣳ॒शत॒ꣳ॒ रात्री॑रपश्य॒त्ततो॒ वा इ॒यमोष॑धीभि॒र्वन॒स्पति॑भिः॒ प्राजा॑यत॒ ये प्र॒जाका॑माः प॒शुका॑माः॒ स्युस्त ए॒ता आ॑सीर॒न्प्रैव जा॑यन्ते प्र॒जया॑ प॒शुभि॑रि॒यं वा अ॑क्षुध्य॒थ्सैतां वि॒राज॑मपश्य॒त्तामा॒त्मन्धि॒त्वान्नाद्य॒मवा॑रु॒न्धौष॑धीः~(९)

%7.4.3.2
वन॒स्पती᳚न्प्र॒जां प॒शून्तेना॑वर्धत॒ सा जे॒मानं॑ महि॒मान॑मगच्छ॒द्य ए॒वं वि॒द्वाꣳस॑ ए॒ता आस॑ते वि॒राज॑मे॒वा\-ऽऽ\-त्मन्धि॒त्वा\-ऽन्नाद्य॒मव॑ रुन्धते॒ वर्ध॑न्ते प्र॒जया॑ प॒शुभि॑र्जे॒मानं॑ महि॒मानं॑ गच्छन्ति॒ ज्योति॑रतिरा॒त्रो भ॑वति सुव॒र्गस्य॑ लो॒कस्यानु॑\-ख्यात्यै॒ पृष्ठ्यः॑ षड॒हो भ॑वति॒ षड्वा ऋ॒तवः॒ षट्पृ॒ष्ठानि॑ पृ॒ष्ठैरे॒वर्तून॒न्वारो॑हन्त्यृ॒तुभिः॑ संवथ्स॒रन्ते सं॑वथ्स॒र ए॒व~(१०)

%7.4.3.3
प्रति॑ तिष्ठन्ति त्रयस्त्रि॒ꣳ॒शात्त्र॑यस्त्रि॒ꣳ॒शमुप॑ यन्ति य॒ज्ञस्य॒ सन्त॑त्या॒ अथो᳚ प्र॒जा\-प॑ति॒र्वै त्र॑यस्त्रि॒ꣳ॒शः प्र॒जा\-प॑तिमे॒वा र॑भन्ते॒ प्रति॑ष्ठित्यै त्रिण॒वो भ॑वति॒ विजि॑त्या एकवि॒ꣳ॒शो भ॑वति॒ प्रति॑ष्ठित्या॒ अथो॒ रुच॑मे॒वा\-ऽऽ\-त्मन्द॑धते त्रि॒वृद॑ग्नि॒ष्टुद्भ॑वति पा॒प्मान॑मे॒व तेन॒ निर्द॑ह॒न्ते\-ऽथो॒ तेजो॒ वै त्रि॒वृत्तेज॑ ए॒वा\-ऽऽ\-त्मन्द॑धते पञ्चद॒श इ॑न्द्रस्तो॒मो भ॑वतीन्द्रि॒यमे॒वाव॑~(११)

%7.4.3.4
रु॒न्ध॒ते॒ स॒प्त॒द॒शो भ॑वत्य॒न्नाद्य॒स्याव॑रुद्ध्या॒ अथो॒ प्रैव तेन॑ जायन्त एकवि॒ꣳ॒शो भ॑वति॒ प्रति॑ष्ठित्या॒ अथो॒ रुच॑मे॒वा\-ऽऽ\-त्मन्द॑धते चतुर्वि॒ꣳ॒शो भ॑वति॒ चतु॑र्विꣳशतिरर्धमा॒साः सं॑वथ्स॒रः सं॑वथ्स॒रः सु॑व॒र्गो लो॒कः सं॑वथ्स॒र ए॒व सु॑व॒र्गे लो॒के प्रति॑ तिष्ठ॒न्त्यथो॑ ए॒ष वै वि॑षू॒वान् वि॑षू॒वन्तो॑ भवन्ति॒ य ए॒वं वि॒द्वाꣳस॑ ए॒ता आस॑ते चतुर्वि॒ꣳ॒शात्पृ॒ष्ठान्युप॑ यन्ति संवथ्स॒र ए॒व प्र॑ति॒ष्ठाय॑~(१२)

%7.4.3.5
दे॒वता॑ अ॒भ्यारो॑हन्ति त्रयस्त्रि॒ꣳ॒शात्त्र॑यस्त्रि॒ꣳ॒शमुप॑ यन्ति॒ त्रय॑स्त्रिꣳश॒द्वै दे॒वता॑ दे॒वता᳚स्वे॒व प्रति॑ तिष्ठन्ति त्रिण॒वो भ॑वती॒मे वै लो॒कास्त्रि॑ण॒व ए॒ष्वे॑व लो॒केषु॒ प्रति॑ तिष्ठन्ति॒ द्वावे॑कवि॒ꣳ॒शौ भ॑वतः॒ प्रति॑ष्ठित्या॒ अथो॒ रुच॑मे॒वा\-ऽऽ\-त्मन्द॑धते ब॒हवः॑ षोड॒शिनो॑ भवन्ति॒ तस्मा᳚द्ब॒हवः॑ प्र॒जासु॒ वृषा॑णो॒ यदे॒ते स्तोमा॒ व्यति॑षक्ता॒ भव॑न्ति॒ तस्मा॑दि॒यमोष॑धीभि॒र्वन॒स्पति॑भि॒र्व्यति॑षक्ता~(१३)

%7.4.3.6
व्यति॑षज्यन्ते प्र॒जया॑ प॒शुभि॒र्य ए॒वं वि॒द्वाꣳस॑ ए॒ता आस॒ते\-ऽकॢ॑प्ता॒ वा ए॒ते सु॑व॒र्गं लो॒कं य॑न्त्युच्चाव॒चान् हि स्तोमा॑नुप॒यन्ति॒ यदे॒त ऊ॒र्ध्वाः कॢ॒प्ताः स्तोमा॒ भव॑न्ति कॢ॒प्ता ए॒व सु॑व॒र्गं लो॒कं य॑न्त्यु॒भयो॑रेभ्यो लो॒कयोः᳚ कल्पते त्रि॒ꣳ॒शदे॒तास्त्रि॒ꣳ॒शद॑क्षरा वि॒राडन्नं॑ वि॒राड्वि॒राजै॒वान्नाद्य॒मव॑ रुन्धते\-ऽतिरा॒त्राव॒भितो॑ भवतो॒\-ऽन्नाद्य॑स्य॒ परि॑गृहीत्यै॥~(१४)

{\anuvakamend[{ओष॑धीः संवथ्स॒र ए॒वाव॑ प्रति॒ष्ठाय॒ व्यति॑ष॒क्तैका॒न्नप॑ञ्चा॒शच्च॑}]}%~(३)

%7.4.4.1
प्र॒जा\-प॑तिः सुव॒र्गं लो॒कमै॒त्तं दे॒वा येन॑येन॒ छन्द॒सानु॒ प्रायु॑ञ्जत॒ तेन॒ नाप्नु॑व॒न्त ए॒ता द्वात्रिꣳ॑शत॒ꣳ॒ रात्री॑रपश्य॒न् द्वात्रिꣳ॑शदक्षरानु॒ष्टुगानु॑ष्टुभः प्र॒जा\-प॑तिः॒ स्वेनै॒व छन्द॑सा प्र॒जा\-प॑तिमा॒प्त्वाभ्या॒रुह्य॑ सुव॒र्गं लो॒कमा॑य॒न्॒ य ए॒वं वि॒द्वाꣳस॑ ए॒ता आस॑ते॒ द्वात्रिꣳ॑शदे॒ता द्वात्रिꣳ॑शदक्षरानु॒ष्टुगानु॑ष्टुभः प्र॒जा\-प॑तिः॒ स्वेनै॒व छन्द॑सा प्र॒जा\-प॑तिमा॒प्त्वा श्रियं॑ गच्छन्ति~(१५)

%7.4.4.2
श्रीर्\mbox{}हि म॑नु॒ष्य॑स्य सुव॒र्गो लो॒को द्वात्रिꣳ॑शदे॒ता द्वात्रिꣳ॑शदक्षरानु॒ष्टुग्वाग॑नु॒ष्टुफ्सर्वा॑मे॒व वाच॑माप्नुवन्ति॒ सर्वे॑ वा॒चो व॑दि॒तारो॑ भवन्ति॒ सर्वे॒ हि श्रियं॒ गच्छ॑न्ति॒ ज्योति॒र्गौरायु॒रिति॑ त्र्य॒हा भ॑वन्ती॒यं वाव ज्योति॑र॒न्तरि॑क्षं॒ गौर॒सावायु॑\-रि॒माने॒व लो॒कान॒भ्यारो॑हन्त्यभिपू॒र्वं त्र्य॒हा भ॑वन्त्यभिपू॒र्वमे॒व सु॑व॒र्गं लो॒कम॒भ्यारो॑हन्ति बृहद्रथन्त॒रा\-भ्यां᳚ यन्ति~(१६)

%7.4.4.3
इ॒यं वाव र॑थन्त॒रम॒सौ बृ॒हदा॒भ्यामे॒व य॒न्त्यथो॑ अ॒नयो॑रे॒व प्रति॑ तिष्ठन्त्ये॒ते वै य॒ज्ञस्या᳚ञ्ज॒साय॑नी स्रु॒ती ताभ्या॑मे॒व सु॑व॒र्गं लो॒कं य॑न्ति॒ परा᳚ञ्चो॒ वा ए॒ते सु॑व॒र्गं लो॒कम॒भ्यारो॑हन्ति॒ ये परा॑चस्त्र्य॒हानु॑प॒यन्ति॑ प्र॒त्यङ्त्र्य॒हो भ॑वति॒ प्र॒त्यव॑रूढ्या॒ अथो॒ प्रति॑ष्ठित्या उ॒भयो᳚र्लो॒कयोर्॑\mbox{}॑ऋद्ध्वोत्ति॑ष्ठन्ति॒ द्वात्रिꣳ॑शदे॒तास्तासां॒ यास्त्रि॒ꣳ॒शत्त्रि॒ꣳ॒शद॑क्षरा वि॒राडन्नं॑ वि॒राड्वि॒राजै॒वान्नाद्य॒मव॑ रुन्धते॒ ये द्वे अ॑होरा॒त्रे ए॒व ते उ॒भाभ्याꣳ॑ रू॒पाभ्याꣳ॑ सुव॒र्गं लो॒कं य॑न्त्यतिरा॒त्राव॒भितो॑ भवतः॒ परि॑गृहीत्यै॥~(१७)

{\anuvakamend[{ग॒च्छ॒न्ति॒ य॒न्ति॒ त्रि॒ꣳ॒शद॑क्षरा॒ द्वाविꣳ॑शतिश्च}]}%~(४)

%7.4.5.1
द्वे वाव दे॑वस॒त्रे द्वा॑दशा॒हश्चै॒व त्र॑यस्त्रिꣳशद॒हश्च॒ य ए॒वं वि॒द्वाꣳस॑स्त्रयस्त्रिꣳशद॒हमास॑ते सा॒क्षादे॒व दे॒वता॑ अ॒भ्यारो॑हन्ति॒ यथा॒ खलु॒ वै श्रेया॑न॒भ्यारू॑ढः का॒मय॑ते॒ तथा॑ करोति॒ यद्य॑व॒विध्य॑ति॒ पापी॑यान्भवति॒ यदि॒ नाव॒विध्य॑ति स॒दृङ्य ए॒वं वि॒द्वाꣳस॑स्त्रयस्त्रिꣳशद॒हमास॑ते॒ वि पा॒प्मना॒ भ्रातृ॑व्ये॒णा व॑र्तन्ते\-ऽह॒र्भाजो॒ वा ए॒ता दे॒वा अग्र॒ आह॑रन्न्~(१८)

%7.4.5.2
अह॒रेको\-ऽभ॑ज॒ताह॒रेक॒स्ताभि॒र्वै ते प्र॒बाहु॑गार्ध्नुव॒न्॒ य ए॒वं वि॒द्वाꣳस॑स्त्रयस्त्रिꣳशद॒हमास॑ते॒ सर्व॑ ए॒व प्र॒बाहु॑गृध्नुवन्ति॒ सर्वे॒ ग्राम॑णीयं॒ प्राप्नु॑वन्ति पञ्चा॒हा भ॑वन्ति॒ पञ्च॒ वा ऋ॒तवः॑ संवथ्स॒र ऋ॒तुष्वे॒व सं॑वथ्स॒रे प्रति॑ तिष्ठ॒न्त्यथो॒ पञ्चा᳚क्षरा प॒ङ्क्तिः पाङ्क्तो॑ य॒ज्ञ य॒ज्ञमे॒वाव॑ रुन्धते॒ त्रीण्या᳚श्वि॒नानि॑ भवन्ति॒ त्रय॑ इ॒मे लो॒का ए॒षु~(१९)

%7.4.5.3
ए॒व लो॒केषु॒ प्रति॑ तिष्ठ॒न्त्यथो॒ त्रीणि॒ वै य॒ज्ञस्ये᳚न्द्रि॒याणि॒ तान्ये॒वाव॑ रुन्धते विश्व॒जिद्भ॑वत्य॒न्नाद्य॒स्याव॑रुद्ध्यै॒ सर्व॑पृष्ठो भवति॒ सर्व॑स्या॒भिजि॑त्यै॒ वाग्वै द्वा॑दशा॒हो यत्पु॒रस्ता᳚द्द्वादशा॒हमु॑पे॒युरना᳚प्तां॒ वाच॒मुपे॑युरुप॒दासु॑कैषां॒ वाख्स्या॑दु॒परि॑ष्टाद्द्वादशा॒हमुप॑ यन्त्या॒प्तामे॒व वाच॒मुप॑ यन्ति॒ तस्मा॑दु॒परि॑ष्टाद्वा॒चा व॑दामो\-ऽवान्त॒रम्~(२०)

%7.4.5.4
वै द॑शरा॒त्रेण॑ प्र॒जा\-प॑तिः प्र॒जा अ॑सृजत॒ यद्द॑शरा॒त्रो भव॑ति प्र॒जा ए॒व तद्यज॑मानाः सृजन्त ए॒ताꣳ ह॒ वा उ॑द॒ङ्कः शौ᳚ल्बाय॒नः स॒त्रस्यर्द्धि॑मुवाच॒ यद्द॑शरा॒त्रो यद्द॑शरा॒त्रो भव॑ति स॒त्रस्यर्द्ध्या॒ अथो॒ यदे॒व पूर्वे॒ष्वहः॑सु॒ विलो॑म क्रि॒यते॒ तस्यै॒वैषा शान्ति॑र्द्व्यनी॒का वा ए॒ता रात्र॑यो॒ यज॑माना विश्व॒जिथ्स॒हाति॑रा॒त्रेण॒ पूर्वाः॒ षोड॑श स॒हाति॑रा॒त्रेणोत्त॑राः॒ षोड॑श॒ य ए॒वं वि॒द्वाꣳस॑स्त्रयस्त्रिꣳशद॒हमास॑त॒ ऐ॑षां᳚ द्व्यनी॒का प्र॒जा जा॑यते\-ऽतिरा॒त्राव॒भितो॑ भवतः॒ परि॑गृहीत्यै॥~(२१)

{\anuvakamend[{अ॒ह॒र॒न्ने॒ष्व॑वान्त॒रꣳ षोड॑श स॒ह स॒प्तद॑श च}]}%~(५)

%7.4.6.1
आ॒दि॒त्या अ॑कामयन्त सुव॒र्गं लो॒कमि॑या॒मेति॒ ते सु॑व॒र्गं लो॒कं न प्राजा॑न॒न्न सु॑व॒र्गं लो॒कमा॑य॒न्त ए॒तꣳ ष॑ट्त्रिꣳशद्रा॒त्रम॑पश्य॒न्तमाह॑र॒न्तेना॑यजन्त॒ ततो॒ वै ते सु॑व॒र्गं लो॒कं प्राजा॑नन्थ्सुव॒र्गं लो॒कमा॑य॒न्॒ य ए॒वं वि॒द्वाꣳसः॑ षट्त्रिꣳशद्रा॒त्रमास॑ते सुव॒र्गमे॒व लो॒कं प्र जा॑नन्ति सुव॒र्गं लो॒कं य॑न्ति॒ ज्योति॑रतिरा॒त्रः~(२२)

%7.4.6.2
भ॒व॒ति॒ ज्योति॑रे॒व पु॒रस्ता᳚द्दधते सुव॒र्गस्य॑ लो॒कस्यानु॑ख्यात्यै षड॒हा भ॑वन्ति॒ षड्वा ऋ॒तव॑ ऋ॒तुष्वे॒व प्रति॑ तिष्ठन्ति च॒त्वारो॑ भवन्ति॒ चत॑स्रो॒ दिशो॑ दि॒क्ष्वे॑व प्रति॑ तिष्ठ॒न्त्यस॑त्त्रं॒ वा ए॒तद्यद॑छन्दो॒मं यच्छ॑न्दो॒मा भव॑न्ति॒ तेन॑ स॒त्रं दे॒वता॑ ए॒व पृ॒ष्ठैरव॑ रुन्धते प॒शूञ्छ॑न्दो॒मैरोजो॒ वै वी॒र्यं॑ पृ॒ष्ठानि॑ प॒शव॑श्छन्दो॒मा ओज॑स्ये॒व~(२३)

%7.4.6.3
वी॒र्ये॑ प॒शुषु॒ प्रति॑ तिष्ठन्ति षट्त्रिꣳशद्रा॒त्रो भ॑वति॒ षट्त्रिꣳ॑शदक्षरा बृह॒ती बार्\mbox{}ह॑ताः प॒शवो॑ बृह॒त्यैव प॒शूनव॑ रुन्धते बृह॒ती छन्द॑सा॒ꣴ॒ स्वारा᳚ज्यमाश्ञुताश्ञु॒वते॒ स्वारा᳚ज्यं॒ य ए॒वं वि॒द्वाꣳसः॑ षट्त्रिꣳशद्रा॒त्रमास॑ते सुव॒र्गमे॒व लो॒कं य॑न्त्यतिरा॒त्राव॒भितो॑ भवतः सुव॒र्गस्य॑ लो॒कस्य॒ परि॑गृहीत्यै॥~(२४)

{\anuvakamend[{अ॒ति॒रा॒त्र ओज॑स्ये॒व षट्त्रिꣳ॑शच्च}]}%~(६)

%7.4.7.1
वसि॑ष्ठो ह॒तपु॑त्रो\-ऽकामयत वि॒न्देय॑ प्र॒जाम॒भि सौ॑दा॒सान्भ॑वेय॒मिति॒ स ए॒तमे॑कस्मान्नपञ्चा॒शम॑पश्य॒त्तमाह॑र॒त्तेना॑यजत॒ ततो॒ वै सो\-ऽवि॑न्दत प्र॒जाम॒भि सौ॑दा॒सान॑भव॒द्य ए॒वं वि॒द्वाꣳस॑ एकस्मान्नपञ्चा॒शमास॑ते वि॒न्दन्ते᳚ प्र॒जाम॒भि भ्रातृ॑व्यान्भवन्ति॒ त्रय॑स्त्रि॒वृतो᳚\-ऽग्निष्टो॒मा भ॑वन्ति॒ वज्र॑स्यै॒व मुख॒ꣳ॒ सꣴ श्य॑न्ति॒ दश॑ पञ्चद॒शा भ॑वन्ति पञ्चद॒शो वज्रः॑~(२५)

%7.4.7.2
वज्र॑मे॒व भ्रातृ॑व्येभ्यः॒ प्र ह॑रन्ति षोडशि॒मद्द॑श॒ममह॑र्भवति॒ वज्र॑ ए॒व वी॒र्यं॑ दधति॒ द्वाद॑श सप्तद॒शा भ॑वन्त्य॒न्नाद्य॒स्याव॑रुद्ध्या॒ अथो॒ प्रैव तैर्जा॑यन्ते॒ पृष्ठ्यः॑ षड॒हो भ॑वति॒ षड्वा ऋ॒तवः॒ षट्पृ॒ष्ठानि॑ पृ॒ष्ठैरे॒वर्तून॒न्वारो॑हन्त्यृ॒तुभिः॑ संवथ्स॒रन्ते सं॑वथ्स॒र ए॒व प्रति॑ तिष्ठन्ति॒ द्वाद॑शैकवि॒ꣳ॒शा भ॑वन्ति॒ प्रति॑ष्ठित्या॒ अथो॒ रुच॑मे॒वा\-ऽऽ\-त्मन्न्~(२६)

%7.4.7.3
द॒ध॒ते॒ ब॒हवः॑ षोड॒शिनो॑ भवन्ति॒ विजि॑त्यै॒ षडा᳚श्वि॒नानि॑ भवन्ति॒ षड्वा ऋ॒तव॑ ऋ॒तुष्वे॒व प्रति॑ तिष्ठन्त्यूनातिरि॒क्ता वा ए॒ता रात्र॑य ऊ॒नास्तद्यदेक॑स्यै॒ न प॑ञ्चा॒शदति॑रिक्ता॒स्तद्यद्भूय॑सीर॒ष्टाच॑त्वारिꣳशत ऊ॒नाच्च॒ खलु॒ वा अति॑रिक्ताच्च प्र॒जा\-प॑तिः॒ प्राजा॑यत॒ ये प्र॒जाका॑माः प॒शुका॑माः॒ स्युस्त ए॒ता आ॑सीर॒न्प्रैव जा॑यन्ते प्र॒जया॑ प॒शुभि॑र्वैरा॒जो वा ए॒ष य॒ज्ञो यदे॑कस्मान्नपञ्चा॒शो य ए॒वं वि॒द्वाꣳस॑ एकस्मान्नपञ्चा॒शमास॑ते वि॒राज॑मे॒व ग॑च्छन्त्यन्ना॒दा भ॑वन्त्यतिरा॒त्राव॒भितो॑ भवतो॒\-ऽन्नाद्य॑स्य॒ परि॑गृहीत्यै॥~(२७)

{\anuvakamend[{वज्र॑ आ॒त्मन्प्र॒जया॒ द्वाविꣳ॑शतिश्च}]}%~(७)

%7.4.8.1
सं॒व॒थ्स॒राय॑ दीक्षि॒ष्यमा॑णा एकाष्ट॒कायां᳚ दीक्षेरन्ने॒षा वै सं॑वथ्स॒रस्य॒ पत्नी॒ यदे॑काष्ट॒कैतस्यां॒ वा ए॒ष ए॒ताꣳ रात्रिं॑ वसति सा॒क्षादे॒व सं॑वथ्स॒रमा॒रभ्य॑ दीक्षन्त॒ आर्तं॒ वा ए॒ते सं॑वथ्स॒रस्या॒भि दी᳚क्षन्ते॒ य ए॑काष्ट॒कायां॒ दीक्ष॒न्ते\-ऽन्त॑नामानावृ॒तू भ॑वतो॒ व्य॑स्तं॒ वा ए॒ते सं॑वथ्स॒रस्या॒भि दी᳚क्षन्ते॒ य ए॑काष्ट॒कायां॒ दीक्ष॒न्ते\-ऽन्त॑नामानावृ॒तू भ॑वतः फल्गुनीपूर्णमा॒से दी᳚क्षेर॒न्मुखं॒ वा ए॒तत्~(२८)

%7.4.8.2
सं॒व॒थ्स॒रस्य॒ यत्फ॑ल्गुनीपूर्णमा॒सो मु॑ख॒त ए॒व सं॑वथ्स॒रमा॒रभ्य॑ दीक्षन्ते॒ तस्यैकै॒व नि॒र्या यथ्सां मे᳚घ्ये विषू॒वान्थ्स॒म्पद्य॑ते चित्रापूर्णमा॒से दी᳚क्षेर॒न्मुखं॒ वा ए॒तथ्सं॑वथ्स॒रस्य॒ यच्चि॑त्रापूर्णमा॒सो मु॑ख॒त ए॒व सं॑वथ्स॒रमा॒रभ्य॑ दीक्षन्ते॒ तस्य॒ न का च॒न नि॒र्या भ॑वति चतुर॒हे पु॒रस्ता᳚त्पौर्णमा॒स्यै दी᳚क्षेर॒न्तेषा॑मेकाष्ट॒कायां᳚ क्र॒यः सम्प॑द्यते॒ तेनै॑काष्ट॒कां न छ॒म्बट्कु॑र्वन्ति॒ तेषा᳚म्~(२९)

%7.4.8.3
पू॒र्व॒प॒क्षे सु॒त्या सम्प॑द्यते पूर्वप॒क्षं मासा॑ अ॒भि सं प॑द्यन्ते॒ ते पू᳚र्वप॒क्ष उत्ति॑ष्ठन्ति॒ तानु॒त्तिष्ठ॑त॒ ओष॑धयो॒ वन॒स्पत॒यो\-ऽनूत्ति॑ष्ठन्ति॒ तान्क॑ल्या॒णी की॒र्तिरनूत्ति॑ष्ठ॒त्यरा᳚थ्सुरि॒मे यज॑माना॒ इति॒ तदनु॒ सर्वे॑ राध्नुवन्ति॥~(३०)

{\anuvakamend[{ए॒तच्छ॒म्बट्कु॑र्वन्ति॒ तेषा॒ञ्चतु॑स्त्रिꣳशच्च}]}%~(८)

%7.4.9.1
सु॒व॒र्गं वा ए॒ते लो॒कं य॑न्ति॒ ये स॒त्रमु॑प॒यन्त्य॒भीन्ध॑त ए॒व दी॒क्षाभि॑रा॒त्मानꣴ॑ श्रपयन्त उप॒सद्भि॒र्द्वाभ्यां॒ लोमाव॑ द्यन्ति॒ द्वाभ्या॒न्त्वचं॒ द्वाभ्या॒मसृ॒द्द्वाभ्यां᳚ मा॒ꣳ॒सं द्वाभ्या॒मस्थि॒ द्वाभ्यां᳚ म॒ज्जान॑मा॒त्मद॑क्षिणं॒ वै स॒त्रमा॒त्मान॑मे॒व दक्षि॑णां नी॒त्वा सु॑व॒र्गं लो॒कं य॑न्ति॒ शिखा॒मनु॒ प्र व॑पन्त॒ ऋद्ध्या॒ अथो॒ रघी॑याꣳसः सुव॒र्गं लो॒कम॑या॒मेति॑॥~(३१)

{\anuvakamend[{सु॒व॒र्गं प॑ञ्चा॒शत्}]}%~(९)

%7.4.10.1
ब्र॒ह्म॒वा॒दिनो॑ वदन्त्यतिरा॒त्रः प॑र॒मो य॑ज्ञक्रतू॒नां कस्मा॒त्तं प्र॑थ॒ममुप॑ य॒न्तीत्ये॒तद्वा अ॑ग्निष्टो॒मं प्र॑थ॒ममुप॑ य॒न्त्यथो॒क्थ्य॑मथ॑ षोड॒शिन॒मथा॑तिरा॒त्रम॑नुपू॒र्वमे॒वैतद्य॑ज्ञक्र॒तूनु॒पेत्य॒ ताना॒लभ्य॑ परि॒गृह्य॒ सोम॑मे॒वैतत्पिब॑न्त आसते॒ ज्योति॑ष्टोमं प्रथ॒ममुप॑ यन्ति॒ ज्योति॑ष्टोमो॒ वै स्तोमा॑नां॒ मुखं॑ मुख॒त ए॒व स्तोमा॒न्प्र यु॑ञ्जते॒ ते~(३२)

%7.4.10.2
सꣴस्तु॑ता वि॒राज॑म॒भि सं प॑द्यन्ते॒ द्वे चर्चा॒वति॑ रिच्येते॒ एक॑या॒ गौरति॑रिक्त॒ एक॒यायु॑रू॒नः सु॑व॒र्गो वै लो॒को ज्योति॒रूर्ग्वि॒राट्सु॑व॒र्गमे॒व तेन॑ लो॒कं य॑न्ति रथन्त॒रं दिवा॒ भव॑ति रथन्त॒रं नक्त॒मित्या॑हुर्ब्रह्मवा॒दिनः॒ केन॒ तदजा॒मीति॑ सौभ॒रं तृ॑तीयसव॒ने ब्र॑ह्मसा॒मं बृ॒हत्तन्म॑ध्य॒तो द॑धति॒ विधृ॑त्यै॒ तेनाजा॑मि॥~(३३)

{\anuvakamend[{त एका॒न्नप॑ञ्चा॒शच्च॑}]}%॥10॥

%7.4.11.1
ज्योति॑ष्टोमं प्रथ॒ममुप॑ यन्त्य॒स्मिन्ने॒व तेन॑ लो॒के प्रति॑ तिष्ठन्ति॒ गोष्टो॑मं द्वि॒तीय॒मुप॑ यन्त्य॒न्तरि॑क्ष ए॒व तेन॒ प्रति॑ तिष्ठ॒न्त्यायु॑ष्टोमं तृ॒तीय॒मुप॑ यन्त्य॒मुष्मि॑न्ने॒व तेन॑ लो॒के प्रति॑ तिष्ठन्ती॒यं वाव ज्योति॑र॒न्तरि॑क्षं॒ गौर॒सावायु॒र्यदे॒तान्थ्स्तोमा॑नुप॒यन्त्ये॒ष्वे॑व तल्लो॒केषु॑ स॒त्रिणः॑ प्रति॒तिष्ठ॑न्तो यन्ति॒ ते सꣴस्तु॑ता वि॒राजम्᳚~(३४)

%7.4.11.2
अ॒भि सं प॑द्यन्ते॒ द्वे चर्चा॒वति॑ रिच्येते॒ एक॑या॒ गौरति॑रिक्त॒ एक॒यायु॑रू॒नः सु॑व॒र्गो वै लो॒को ज्योति॒रूर्ग्वि॒राडूर्ज॑मे॒वाव॑ रुन्धते॒ ते न क्षु॒धार्ति॒मार्च्छ॒न्त्यक्षो॑धुका भवन्ति॒ क्षुथ्स॑म्बाधा इव॒ हि स॒त्रिणो᳚\-ऽग्निष्टो॒माव॒भितः॑ प्र॒धी तावु॒क्थ्या॑ मध्ये॒ नभ्यं॒ तत्तदे॒तत्प॑रि॒यद्दे॑वच॒क्रं यदे॒तेन॑~(३५)

%7.4.11.3
ष॒ड॒हेन॒ यन्ति॑ देवच॒क्रमे॒व स॒मारो॑ह॒न्त्यरि॑ष्ट्यै॒ ते स्व॒स्ति सम॑श्ञुवते षड॒हेन॑ यन्ति॒ षड्वा ऋ॒तव॑ ऋ॒तुष्वे॒व प्रति॑ तिष्ठन्त्युभ॒यतो᳚ज्योतिषा यन्त्युभ॒यत॑ ए॒व सु॑व॒र्गे लो॒के प्र॑ति॒तिष्ठ॑न्तो यन्ति॒ द्वौ ष॑ड॒हौ भ॑वत॒स्तानि॒ द्वाद॒शाहा॑नि॒ सम्प॑द्यन्ते द्वाद॒शो वै पुरु॑षो॒ द्वे स॒क्थ्यौ᳚ द्वौ बा॒हू आ॒त्मा च॒ शिर॑श्च च॒त्वार्यङ्गा॑नि॒ स्तनौ᳚ द्वाद॒शौ~(३६)

%7.4.11.4
तत्पुरु॑ष॒मनु॑ प॒र्याव॑र्तन्ते॒ त्रयः॑ षड॒हा भ॑वन्ति॒ तान्य॒ष्टाद॒शाहा॑नि॒ सं प॑द्यन्ते॒ नवा॒न्यानि॒ नवा॒न्यानि॒ नव॒ वै पुरु॑षे प्रा॒णास्तत्प्रा॒णाननु॑ प॒र्याव॑र्तन्ते च॒त्वारः॑ षड॒हा भ॑वन्ति॒ तानि॒ चतु॑र्विꣳशति॒रहा॑नि॒ सं प॑द्यन्ते॒ चतु॑र्विꣳशतिरर्धमा॒साः सं॑वथ्स॒रस्तथ्सं॑वथ्स॒रमनु॑ प॒र्याव॑र्त॒न्ते\-ऽप्र॑तिष्ठितः संवथ्स॒र इति॒ खलु॒ वा आ॑हु॒र्वर्\mbox{}षी॑यान्प्रति॒ष्ठाया॒ इत्ये॒ताव॒द्वै सं॑वथ्स॒रस्य॒ ब्राह्म॑णं॒ याव॑न्मा॒सो मा॒सिमा᳚स्ये॒व प्र॑ति॒तिष्ठ॑न्तो यन्ति॥~(३७)

{\anuvakamend[{वि॒राज॑मे॒तेन॑ द्वाद॒शावे॒ताव॒द्वा अ॒ष्टौ च॑}]}%॥11॥

%7.4.12.1
मे॒षस्त्वा॑ पच॒तैर॑वतु॒ लोहि॑तग्रीव॒श्छागैः᳚ शल्म॒लिर्वृद्ध्या॑ प॒र्णो ब्रह्म॑णा प्ल॒क्षो मेधे॑न न्य॒ग्रोध॑श्चम॒सैरु॑दु॒म्बर॑ ऊ॒र्जा गा॑य॒त्री छन्दो॑भिस्त्रि॒वृथ्स्तोमै॒रव॑न्तीः॒ स्थाव॑न्तीस्त्वावन्तु प्रि॒यं त्वा᳚ प्रि॒याणां॒ वर्\mbox{}षि॑ष्ठ॒माप्या॑नां निधी॒नां त्वा॑ निधि॒पतिꣳ॑ हवामहे वसो मम॥~(३८)

{\anuvakamend[{मे॒षः षट्त्रिꣳ॑शत्}]}%॥12॥

%7.4.13.1
कूप्या᳚भ्यः॒ स्वाहा॒ कूल्या᳚भ्यः॒ स्वाहा॑ विक॒र्या᳚भ्यः॒ स्वाहा॑\-ऽव॒ट्या᳚भ्यः॒ स्वाहा॒ खन्या᳚भ्यः॒ स्वाहा॒ ह्रद्या᳚भ्यः॒ स्वाहा॒ सूद्या᳚भ्यः॒ स्वाहा॑ सर॒स्या᳚भ्यः॒ स्वाहा॑ वैश॒न्तीभ्यः॒ स्वाहा॑ पल्व॒ल्या᳚भ्यः॒ स्वाहा॒ वर्ष्या᳚भ्यः॒ स्वाहा॑\-ऽव॒र्ष्याभ्यः॒ स्वाहा᳚ ह्रा॒दुनी᳚भ्यः॒ स्वाहा॒ पृष्वा᳚भ्यः॒ स्वाहा॒ स्यन्द॑मानाभ्यः॒ स्वाहा᳚ स्थाव॒राभ्यः॒ स्वाहा॑ नादे॒यीभ्यः॒ स्वाहा॑ सैन्ध॒वीभ्यः॒ स्वाहा॑ समु॒द्रिया᳚भ्यः॒ स्वाहा॒ सर्वा᳚भ्यः॒ स्वाहा᳚॥~(३९)

{\anuvakamend[{कूप्या᳚भ्यश्चत्वारि॒ꣳ॒शत्}]}%॥13॥

%7.4.14.1
अ॒द्भ्यः स्वाहा॒ वह॑न्तीभ्यः॒ स्वाहा॑ परि॒वह॑न्तीभ्यः॒ स्वाहा॑ सम॒न्तं वह॑न्तीभ्यः॒ स्वाहा॒ शीघ्रं॒ वह॑न्तीभ्यः॒ स्वाहा॒ शीभं॒ वह॑न्तीभ्यः॒ स्वाहो॒ग्रं वह॑न्तीभ्यः॒ स्वाहा॑ भी॒मं वह॑न्तीभ्यः॒ स्वाहा\-ऽम्भो᳚भ्यः॒ स्वाहा॒ नभो᳚भ्यः॒ स्वाहा॒ महो᳚भ्यः॒ स्वाहा॒ सर्व॑स्मै॒ स्वाहा᳚॥~(४०)

{\anuvakamend[{अ॒द्भ्य एका॒न्नत्रि॒ꣳ॒शत्}]}%॥14॥

%7.4.15.1
यो अर्व॑न्तं॒ जिघाꣳ॑सति॒ तम॒भ्य॑मीति॒ वरु॑णः। प॒रो मर्तः॑ प॒रः श्वा। अ॒हं च॒ त्वं च॑ वृत्रह॒न्थ्सम्ब॑भूव स॒निभ्य॒ आ। अ॒रा॒ती॒वा चि॑दद्रि॒वो\-ऽनु॑ नौ शूर मꣳसतै भ॒द्रा इन्द्र॑स्य रा॒तयः॑। अ॒भि क्रत्वे᳚न्द्र भू॒रध॒ ज्मन्न ते॑ विव्यङ्महि॒मान॒ꣳ॒ रजाꣳ॑सि। स्वेना॒ हि वृ॒त्रꣳ शव॑सा ज॒घन्थ॒ न शत्रु॒रन्तं॑ विविदद्यु॒धा ते᳚॥~(४१)

{\anuvakamend[{वि॒वि॒दद्द्वे च॑}]}%॥15॥

%7.4.16.1
नमो॒ राज्ञे॒ नमो॒ वरु॑णाय॒ नमो\-ऽश्वा॑य॒ नमः॑ प्र॒जा\-प॑तये॒ नमो\-ऽधि॑पत॒ये\-ऽधि॑पतिर॒स्यधि॑पतिं मा कु॒र्वधि॑पतिर॒हं प्र॒जानां᳚ भूयास॒म्मां धे॑हि॒ मयि॑ धेह्यु॒पाकृ॑ताय॒ स्वाहा\-ऽऽ\-ल॑ब्धाय॒ स्वाहा॑ हु॒ताय॒ स्वाहा᳚॥~(४२)

{\anuvakamend[{नम॒ एका॒न्नत्रि॒ꣳ॒शत्}]}%॥16॥

%7.4.17.1
म॒यो॒भूर्वातो॑ अ॒भि वा॑तू॒स्रा ऊर्ज॑स्वती॒रोष॑धी॒रा रि॑शन्ताम्। पीव॑स्वतीर्जी॒वध॑न्याः पिबन्त्वव॒साय॑ प॒द्वते॑ रुद्र मृड। याः सरू॑पा॒ विरू॑पा॒ एक॑रूपा॒ यासा॑म॒ग्निरिष्ट्या॒ नामा॑नि॒ वेद॑। या अङ्गि॑रस॒स्तप॑से॒ह च॒क्रुस्ताभ्यः॑ पर्जन्य॒ महि॒ शर्म॑ यच्छ। या दे॒वेषु॑ त॒नुव॒मैर॑यन्त॒ यासा॒ꣳ॒ सोमो॒ विश्वा॑ रू॒पाणि॒ वेद॑। ता अ॒स्मभ्यं॒ पय॑सा॒ पिन्व॑मानाः प्र॒जाव॑तीरिन्द्र~(४३)

%7.4.17.2
गो॒ष्ठे रि॑रीहि। प्र॒जा\-प॑ति॒र्मह्य॑मे॒ता ररा॑णो॒ विश्वै᳚र्दे॒वैः पि॒तृभिः॑ संविदा॒नः। शि॒वाः स॒तीरुप॑ नो गो॒ष्ठमाक॒स्तासां᳚ व॒यं प्र॒जया॒ सꣳ स॑देम। इ॒ह धृतिः॒ स्वाहे॒ह विधृ॑तिः॒ स्वाहे॒ह रन्तिः॒ स्वाहे॒ह रम॑तिः॒ स्वाहा॑ म॒हीमू॒ षु सु॒त्रामा॑णम्॥~(४४)

{\anuvakamend[{इ॒न्द्रा॒ष्टात्रिꣳ॑शच्च}]}%॥17॥

%7.4.18.1
किꣴ स्वि॑दासीत्पू॒र्वचि॑त्तिः॒ किꣴ स्वि॑दासीद्बृ॒हद्वयः॑। किꣴ स्वि॑दासीत्पिशङ्गि॒ला किꣴ स्वि॑दासीत्पिलिप्पि॒ला। द्यौरा॑सीत्पू॒र्वचि॑त्ति॒रश्व॑ आसीद्बृ॒हद्वयः॑। रात्रि॑रासीत्पिशङ्गि॒लावि॑रासीत्पिलिप्पि॒ला। कः स्वि॑देका॒की च॑रति॒ क उ॑ स्विज्जायते॒ पुनः॑। किꣴ स्वि॑द्धि॒मस्य॑ भेष॒जं किꣴ स्वि॑दा॒वप॑नं म॒हत्। सूर्य॑ एका॒की च॑रति~(४५)

%7.4.18.2
च॒न्द्रमा॑ जायते॒ पुनः॑। अ॒ग्निर्\mbox{}हि॒मस्य॑ भेष॒जं भूमि॑रा॒वप॑नं म॒हत्। पृ॒च्छामि॑ त्वा॒ पर॒मन्तं॑ पृथि॒व्याः पृ॒च्छामि॑ त्वा॒ भुव॑नस्य॒ नाभिम्᳚। पृ॒च्छामि॑ त्वा॒ वृष्णो॒ अश्व॑स्य॒ रेतः॑ पृ॒च्छामि॑ वा॒चः प॑र॒मं व्यो॑म। वेदि॑माहुः॒ पर॒मन्तं॑ पृथि॒व्या य॒ज्ञमा॑हु॒र्भुव॑नस्य॒ नाभिम्᳚। सोम॑माहु॒र्वृष्णो॒ अश्व॑स्य॒ रेतो॒ ब्रह्मै॒व वा॒चः प॑र॒मं व्यो॑म॥~(४६)

{\anuvakamend[{सूर्य॑ एका॒की च॑रति॒ षट्च॑त्वारिꣳशच्च}]}%॥18॥

%7.4.19.1
अम्बे॒ अम्बा॒ल्यम्बि॑के॒ न मा॑ नयति॒ कश्च॒न। स॒सस्त्य॑श्व॒कः। सुभ॑गे॒ कां पी॑लवासिनि सुव॒र्गे लो॒के सं प्रोर्ण्वा॑थाम्। आहम॑जानि गर्भ॒धमा त्वम॑जासि गर्भ॒धम्। तौ स॒ह च॒तुरः॑ प॒दः सम्प्र सा॑रयावहै। वृषा॑ वाꣳ रेतो॒धा रेतो॑ दधा॒तूथ्स॒क्थ्यो᳚र्गृ॒दं धे᳚ह्य॒ञ्जिमुद॑ञ्जि॒मन्व॑ज। यः स्त्री॒णां जी॑व॒भोज॑नो॒ य आ॑साम्~(४७)

%7.4.19.2
बि॒ल॒धाव॑नः। प्रि॒यः स्त्री॒णाम॑पी॒च्यः॑। य आ॑सां कृ॒ष्णे लक्ष्म॑णि॒ सर्दि॑गृदिं प॒राव॑धीत्। अम्बे॒ अम्बा॒ल्यम्बि॑के॒ न मा॑ यभति॒ कश्च॒न। स॒सस्त्य॑श्व॒कः। ऊ॒र्ध्वामे॑ना॒मुच्छ्र॑यताद्वेणुभा॒रं गि॒रावि॑व। अथा᳚स्या॒ मध्य॑मेधताꣳ शी॒ते वाते॑ पु॒नन्नि॑व। अम्बे॒ अम्बा॒ल्यम्बि॑के॒ न मा॑ यभति॒ कश्च॒न। स॒सस्त्य॑श्व॒कः। यद्ध॑रि॒णी यव॒मत्ति॒ न~(४८)

%7.4.19.3
पु॒ष्टं प॒शु म॑न्यते। शू॒द्रा यदर्य॑जारा॒ न पोषा॑य धनायति। अम्बे॒ अम्बा॒ल्यम्बि॑के॒ न मा॑ यभति॒ कश्च॒न। स॒सस्त्य॑श्व॒कः। इ॒यं य॒का श॑कुन्ति॒काहल॒मिति॒ सर्प॑ति। आह॑तं ग॒भे पसो॒ नि ज॑ल्गुलीति॒ धाणि॑का। अम्बे॒ अम्बा॒ल्यम्बि॑के॒ न मा॑ यभति॒ कश्च॒न। स॒सस्त्य॑श्व॒कः। मा॒ता च॑ ते पि॒ता च॒ ते\-ऽग्रं॑ वृ॒क्षस्य॑ रोहतः।~(४९)

%7.4.19.4
प्र सु॑ला॒मीति॑ ते पि॒ता ग॒भे मु॒ष्टिम॑तꣳसयत्। द॒धि॒क्राव्ण्णो॑ अकारिषं जि॒ष्णोरश्व॑स्य वा॒जिनः॑। सु॒र॒भि नो॒ मुखा॑ कर॒त्प्र ण॒ आयूꣳ॑षि तारिषत्। आपो॒ हि\-ष्ठा म॑यो॒भुव॒स्ता न॑ ऊ॒र्जे द॑धातन। म॒हे रणा॑य॒ चक्ष॑से। यो वः॑ शि॒वत॑मो॒ रस॒स्तस्य॑ भाजयते॒ह नः॑। उ॒श॒तीरि॑व मा॒तरः॑। तस्मा॒ अरं॑ गमाम वो॒ यस्य॒ क्षया॑य॒ जिन्व॑थ। आपो॑ ज॒नय॑था च नः॥~(५०)

{\anuvakamend[{आ॒सा॒मत्ति॒ न रो॑हतो॒ जिन्व॑थ च॒त्वारि॑ च}]}%॥19॥

%7.4.20.1
भूर्भुवः॒ सुव॒र्वस॑वस्त्वाञ्जन्तु गाय॒त्रेण॒ छन्द॑सा रु॒द्रास्त्वा᳚ञ्जन्तु॒ त्रैष्टु॑भेन॒ छन्द॑सादि॒त्यास्त्वा᳚ञ्जन्तु॒ जाग॑तेन॒ छन्द॑सा॒ यद्वातो॑ अ॒पो अग॑म॒दिन्द्र॑स्य त॒नुवं॑ प्रि॒याम्। ए॒तꣴ स्तो॑तरे॒तेन॑ प॒था पुन॒रश्व॒मा व॑र्तयासि नः। लाजी~(३) ञ्छाची~(३) न् यशो॑ म॒मा~(४)म्। य॒व्यायै॑ ग॒व्याया॑ ए॒तद्दे॑वा॒ अन्न॑मत्तै॒तदन्न॑मद्धि प्रजापते। यु॒ञ्जन्ति॑ ब्र॒ध्नम॑रु॒षं चर॑न्तं॒ परि॑ त॒स्थुषः॑। रोच॑न्ते रोच॒ना दि॒वि। यु॒ञ्जन्त्य॑स्य॒ काम्या॒ हरी॒ विप॑क्षसा॒ रथे᳚। शोणा॑ धृ॒ष्णू नृ॒वाह॑सा। के॒तुं कृ॒ण्वन्न॑के॒तवे॒ पेशो॑ मर्या अपे॒शसे᳚। समु॒षद्भि॑रजायथाः॥~(५१)

{\anuvakamend[{ब्र॒ध्नं पञ्च॑विꣳशतिश्च}]}%॥20॥

%7.4.21.1
प्रा॒णाय॒ स्वाहा᳚ व्या॒नाय॒ स्वाहा॑\-ऽपा॒नाय॒ स्वाहा॒ स्नाव॑भ्यः॒ स्वाहा॑ सन्ता॒नेभ्यः॒ स्वाहा॒ परि॑सन्तानेभ्यः॒ स्वाहा॒ पर्व॑भ्यः॒ स्वाहा॑ स॒न्धाने᳚भ्यः॒ स्वाहा॒ शरी॑रेभ्यः॒ स्वाहा॑ य॒ज्ञाय॒ स्वाहा॒ दक्षि॑णाभ्यः॒ स्वाहा॑ सुव॒र्गाय॒ स्वाहा॑ लो॒काय॒ स्वाहा॒ सर्व॑स्मै॒ स्वाहा᳚॥~(५२)

{\anuvakamend[{प्रा॒णाया॒ष्टाविꣳ॑शतिः}]}%॥21॥

%7.4.22.1
सि॒ताय॒ स्वाहा\-ऽसि॑ताय॒ स्वाहा॒\-ऽभिहि॑ताय॒ स्वाहा\-ऽन॑भिहिताय॒ स्वाहा॑ यु॒क्ताय॒ स्वाहाH\-ऽयु॑क्ताय॒ स्वाहा॒ सुयु॑क्ताय॒ स्वाहोद्यु॑क्ताय॒ स्वाहा॒ विमु॑क्ताय॒ स्वाहा॒ प्रमु॑क्ताय॒ स्वाहा॒ वञ्च॑ते॒ स्वाहा॑ परि॒वञ्च॑ते॒ स्वाहा॑ सं॒वञ्च॑ते॒ स्वाहा॑\-ऽनु॒वञ्च॑ते॒ स्वाहो॒द्वञ्च॑ते॒ स्वाहा॑ य॒ते स्वाहा॒ धाव॑ते॒ स्वाहा॒ तिष्ठ॑ते॒ स्वाहा॒ सर्व॑स्मै॒ स्वाहा᳚॥~(५३)

{\anuvakamend[{सि॒ताया॒ष्टात्रिꣳ॑शत्}]}%॥22॥

{\anuvakamend[{बृह॒स्पतिः॒ श्रद्यथा॒ वा ऋ॒क्षा वै प्र॒जा\-प॑ति॒र्येन॑येन॒ द्वे वाव दे॑वस॒त्रे आ॑दि॒त्या अ॑कामयन्त सुव॒र्गं वसि॑ष्ठः संवथ्स॒राय॑ सुव॒र्गं ये स॒त्रं ब्र॑ह्मवा॒दिनो॑\-ऽतिरा॒त्रो ज्योति॑ष्टोमं मे॒षः कूप्या᳚भ्यो॒\-ऽद्भ्यो यो नमो॑ मयो॒भूः किꣴ स्वि॒दम्बे॒ भूः प्रा॒णाय॑ सि॒ताय॒ द्वाविꣳ॑शतिः}]%॥22॥

{\prashnaend[{बृह॒स्पतिः॒ प्रति॑तिष्ठन्ति॒ वै द॑शरा॒त्रेण॑ सुव॒र्गं यो अर्व॑न्तं॒ भूस्त्रिप़॑ञ्चा॒शत्॥53॥ बृह॒स्पतिः॒ सर्व॑स्मै॒ स्वाहा᳚॥}]}


%%% END PRASHNA
