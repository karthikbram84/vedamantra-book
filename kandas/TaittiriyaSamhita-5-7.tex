\chapt{काण्डम् ५}
\sect{सप्तमः प्रश्नः}\setcounter{anuvakam}{0}
\dnsub{तैत्तिरीयसंहितायां पञ्चमकाण्डे सप्तमः प्रश्नः}
%5.7.1.1
यो वा अय॑थादेवतम॒ग्निं चि॑नु॒त आ दे॒वता᳚भ्यो वृश्च्यते॒ पापी॑यान्भवति॒ यो य॑थादेव॒तं न दे॒वता᳚भ्य॒ आ वृ॑श्च्यते॒ वसी॑यान्भवत्याग्ने॒य्या गा॑यत्रि॒या प्र॑थ॒मां चिति॑म॒भि मृ॑शेत्त्रि॒ष्टुभा᳚ द्वि॒तीयां॒ जग॑त्या तृ॒तीया॑मनु॒ष्टुभा॑ चतु॒र्थीं प॒ङ्क्त्या प॑ञ्च॒मीं य॑थादेव॒तमे॒वाग्निं चि॑नुते॒ न दे॒वता᳚भ्य॒ आ वृ॑श्च्यते॒ वसी॑यान्भव॒तीडा॑यै॒ वा ए॒षा विभ॑क्तिः प॒शव॒ इडा॑ प॒शुभि॑रेनम्~(१)

%5.7.1.2
चि॒नु॒ते॒ यो वै प्र॒जा\-प॑तये प्रति॒प्रोच्या॒ग्निं चि॒नोति॒ नार्ति॒मार्च्छ॒त्यश्वा॑व॒भित॑स्तिष्ठेतां कृ॒ष्ण उ॑त्तर॒तः श्वे॒तो दक्षि॑ण॒\-स्तावा॒लभ्येष्ट॑का॒ उप॑ दध्यादे॒तद्वै प्र॒जा\-प॑ते रू॒पं प्रा॑जाप॒त्यो\-ऽश्वः॑ सा॒क्षादे॒व प्र॒जा\-प॑तये प्रति॒प्रोच्या॒ग्निं चि॑नोति॒ नार्ति॒मार्च्छ॑त्ये॒तद्वा अह्नो॑ रू॒पं यच्छ्वे॒तो\-ऽश्वो॒ रात्रि॑यै कृ॒ष्ण ए॒तदह्नः॑~(२)

%5.7.1.3
रू॒पं यदिष्ट॑का॒ रात्रि॑यै॒ पुरी॑ष॒मिष्ट॑का उपधा॒स्यञ्छ्वे॒तमश्व॑म॒भि मृ॑शे॒त्पुरी॑षमुपधा॒स्यन्कृ॒ष्णम॑होरा॒त्राभ्या॑मे॒वैनं॑ चिनुते हिरण्यपा॒त्रं मधोः᳚ पू॒र्णं द॑दाति मध॒व्यो॑\-ऽसा॒नीति॑ सौ॒र्या चि॒त्रव॒त्यावे᳚क्षते चि॒त्रमे॒व भ॑वति म॒ध्यन्दि॒ने\-ऽश्व॒मव॑ घ्रापयत्य॒सौ वा आ॑दि॒त्य इन्द्र॑ ए॒ष प्र॒जा\-प॑तिः प्राजाप॒त्यो\-ऽश्व॒स्तमे॒व सा॒क्षादृ॑ध्नोति॥~(३)

{\anuvakamend[{ए॒नमे॒तदह्नो॒\-ऽष्टाच॑त्वारिꣳशच्च}]}%~(१)

%5.7.2.1
त्वाम॑ग्ने वृष॒भं चेकि॑तानं॒ पुन॒र्युवा॑नञ्ज॒नय॑न्नु॒पागा᳚म्। अ॒स्थू॒रि णो॒ गार्\mbox{}ह॑पत्यानि सन्तु ति॒ग्मेन॑ नो॒ ब्रह्म॑णा॒ सꣳ शि॑शाधि। प॒शवो॒ वा ए॒ते यदिष्ट॑का॒श्चित्यां᳚चित्यामृष॒भमुप॑ दधाति मिथु॒नमे॒वास्य॒ तद्य॒ज्ञे क॑रोति प्र॒जन॑नाय॒ तस्मा᳚द्यू॒थेयू॑थ ऋष॒भः। सं॒व॒थ्स॒रस्य॑ प्रति॒मां यां त्वा॑ रात्र्यु॒पास॑ते। प्र॒जाꣳ सु॒वीरां᳚ कृ॒त्वा विश्व॒मायु॒र्व्य॑श्ञवत्। प्रा॒जा॒प॒त्याम्~(४)

%5.7.2.2
ए॒तामुप॑ दधाती॒यं वावैषैका᳚ष्ट॒का यदे॒वैका᳚ष्ट॒काया॒मन्नं॑ क्रि॒यते॒ तदे॒वैतयाव॑ रुन्ध ए॒षा वै प्र॒जा\-प॑तेः काम॒दुघा॒ तयै॒व यज॑मानो॒\-ऽमुष्मिँ॑ल्लो॒के᳚\-ऽग्निं दु॑हे॒ येन॑ दे॒वा ज्योति॑षो॒र्ध्वा उ॒दाय॒न्॒ येना॑दि॒त्या वस॑वो॒ येन॑ रु॒द्राः। येनाङ्गि॑रसो महि॒मान॑मान॒शुस्तेनै॑तु॒ यज॑मानः स्व॒स्ति। सु॒व॒र्गाय॒ वा ए॒ष लो॒काय॑~(५)

%5.7.2.3
ची॒य॒ते॒ यद॒ग्निर्येन॑ दे॒वा ज्योति॑षो॒र्ध्वा उ॒दाय॒न्नित्युख्य॒ꣳ॒ समि॑न्द्ध॒ इष्ट॑का ए॒वैता उप॑ धत्ते वानस्प॒त्याः सु॑व॒र्गस्य॑ लो॒कस्य॒ सम॑ष्ट्यै श॒तायु॑धाय श॒तवी᳚र्याय श॒तोत॑ये\-ऽभिमाति॒षाहे᳚। श॒तं यो नः॑ श॒रदो॒ अजी॑ता॒निन्द्रो॑ नेष॒दति॑ दुरि॒तानि॒ विश्वा᳚। ये च॒त्वारः॑ प॒थयो॑ देव॒याना॑ अन्त॒रा द्यावा॑पृथि॒वी वि॒यन्ति॑। तेषां॒ यो अज्या॑नि॒मजी॑तिमा॒ वहा॒त्तस्मै॑ नो देवाः~(६)

%5.7.2.4
परि॑ दत्ते॒ह सर्वे᳚। ग्री॒ष्मो हे॑म॒न्त उ॒त नो॑ वस॒न्तः श॒रद्व॒र्॒\mbox{}षाः सु॑वि॒तं नो॑ अस्तु। तेषा॑मृतू॒नाꣳ श॒तशा॑रदानां निवा॒त ए॑षा॒मभ॑ये स्याम। इ॒दु॒व॒थ्स॒राय॑ परिवथ्स॒राय॑ संवथ्स॒राय॑ कृणुता बृ॒हन्नमः॑। तेषां᳚ व॒यꣳ सु॑म॒तौ य॒ज्ञिया॑नां॒ ज्योगजी॑ता॒ अह॑ताः स्याम। भ॒द्रान्नः॒ श्रेयः॒ सम॑नैष्ट देवा॒स्त्वया॑व॒सेन॒ सम॑शीमहि त्वा। स नो॑ मयो॒भूः पि॑तो~(७)

%5.7.2.5
आ वि॑शस्व॒ शं तो॒काय॑ त॒नुवे᳚ स्यो॒नः। अज्या॑नीरे॒ता उप॑ दधात्ये॒ता वै दे॒वता॒ अप॑राजिता॒स्ता ए॒व प्र वि॑शति॒ नैव जी॑यते ब्रह्मवा॒दिनो॑ वदन्ति॒ यद॑र्धमा॒सा मासा॑ ऋ॒तवः॑ संवथ्स॒र ओष॑धीः॒ पच॒न्त्यथ॒ कस्मा॑द॒न्याभ्यो॑ दे॒वता᳚भ्य आग्रय॒णं निरु॑प्यत॒ इत्ये॒ता हि तद्दे॒वता॑ उ॒दज॑य॒न्॒ यदृ॒तुभ्यो॑ नि॒र्वपे᳚द्दे॒वता᳚भ्यः स॒मदं॑ दध्यादाग्रय॒णं नि॒रुप्यै॒ता आहु॑तीर्जुहोत्यर्धमा॒साने॒व मासा॑नृ॒तून्थ्सं॑वथ्स॒रं प्री॑णाति॒ न दे॒वता᳚भ्यः स॒मदं॑ दधाति भ॒द्रान्नः॒ श्रेयः॒ सम॑नैष्ट देवा॒ इत्या॑ह हु॒ताद्या॑य॒ यज॑मान॒स्याप॑राभावाय॥~(८)

{\anuvakamend[{प्रा॒जा॒प॒त्यां लो॒काय॑ देवाः पितो दध्यादाग्रय॒णं पञ्च॑विꣳशतिश्च}]}%~(२)

%5.7.3.1
इन्द्र॑स्य॒ वज्रो॑\-ऽसि॒ वार्त्र॑घ्नस्तनू॒पा नः॑ प्रतिस्प॒शः। यो नः॑ पु॒रस्ता᳚द्दक्षिण॒तः प॒श्चादु॑त्तर॒तो॑\-ऽघा॒युर॑भि॒दास॑त्ये॒तꣳ सो\-ऽश्मा॑नमृच्छतु। दे॒वा॒सु॒राः संय॑त्ता आस॒न्ते\-ऽसु॑रा दि॒ग्भ्य आबा॑धन्त॒ तां दे॒वा इष्वा॑ च॒ वज्रे॑ण॒ चापा॑नुदन्त॒ यद्व॒ज्रिणी॑रुप॒दधा॒तीष्वा॑ चै॒व तद्वज्रे॑ण च॒ यज॑मानो॒ भ्रातृ॑व्या॒नप॑ नुदते दि॒क्षूप॑~(९)

%5.7.3.2
द॒धा॒ति॒ दे॒व॒पु॒रा ए॒वैतास्त॑नू॒पानीः॒ पर्यू॑ह॒ते\-ऽग्ना॑विष्णू स॒जोष॑से॒मा व॑र्धन्तु वां॒ गिरः॑। द्यु॒म्नैर्वाजे॑भि॒रा ग॑तम्। ब्र॒ह्म॒वा॒दिनो॑ वदन्ति॒ यन्न दे॒वता॑यै॒ जुह्व॒त्यथ॑ किन्देव॒त्या॑ वसो॒र्धारेत्य॒ग्निर्वसु॒स्तस्यै॒षा धारा॒ विष्णु॒र्वसु॒स्तस्यै॒षा धारा᳚ग्नावैष्ण॒व्यर्चा वसो॒र्धारां᳚ जुहोति भाग॒धेये॑नै॒वैनौ॒ सम॑र्धय॒त्यथो॑ ए॒ताम्~(१०)

%5.7.3.3
ए॒वाहु॑तिमा॒यत॑नवतीं करोति॒ यत्का॑म एनां जु॒होति॒ तदे॒वाव॑ रुन्धे रु॒द्रो वा ए॒ष यद॒ग्निस्तस्यै॒ते त॒नुवौ॑ घो॒रान्या शि॒वान्या यच्छ॑तरु॒द्रीयं॑ जु॒होति॒ यैवास्य॑ घो॒रा त॒नूस्तां तेन॑ शमयति॒ यद्वसो॒र्धारां᳚ जु॒होति॒ यैवास्य॑ शि॒वा त॒नूस्तां तेन॑ प्रीणाति॒ यो वै वसो॒र्धारा॑यै~(११)

%5.7.3.4
प्र॒ति॒ष्ठां वेद॒ प्रत्ये॒व ति॑ष्ठति॒ यदाज्य॑मु॒च्छिष्ये॑त॒ तस्मि॑न्ब्रह्मौद॒नं प॑चे॒त्तं ब्रा᳚ह्म॒णाश्च॒त्वारः॒ प्राश्ञी॑युरे॒ष वा अ॒ग्निर्वै᳚श्वान॒रो यद्ब्रा᳚ह्म॒ण ए॒षा खलु॒ वा अ॒ग्नेः प्रि॒या त॒नूर्यद्वै᳚श्वान॒रः प्रि॒याया॑मे॒वैनां᳚ त॒नुवां॒ प्रति॑\-ष्ठापयति॒ चत॑स्रो धे॒नूर्द॑द्या॒त्ताभि॑रे॒व यज॑मानो॒\-ऽमुष्मिँ॑ल्लो॒के᳚\-ऽग्निं दु॑हे॥~(१२)

{\anuvakamend[{उपै॒तान्धारा॑यै॒ षट्च॑त्वारिꣳशच्च}]}%~(३)

%5.7.4.1
चित्ति॑ञ्जुहोमि॒ मन॑सा घृ॒तेनेत्या॒हादा᳚भ्या॒ वै नामै॒षाहु॑तिर्वैश्वकर्म॒णी नैनं॑ चिक्या॒नं भ्रातृ॑व्यो दभ्नो॒त्यथो॑ दे॒वता॑ ए॒वाव॑ रु॒न्धे\-ऽग्ने॒ तम॒द्येति॑ प॒ङ्क्त्या जु॑होति प॒ङ्क्त्याहु॑त्या यज्ञमु॒खमार॑भते स॒प्त ते॑ अग्ने स॒मिधः॑ स॒प्त जि॒ह्वा इत्या॑ह॒ होत्रा॑ ए॒वाव॑ रुन्धे॒\-ऽग्निर्दे॒वेभ्यो\-ऽपा᳚क्रामद्भाग॒धेयम्᳚~(१३)

%5.7.4.2
इ॒च्छमा॑न॒स्तस्मा॑ ए॒तद्भा॑ग॒धेयं॒ प्राय॑च्छन्ने॒तद्वा अ॒ग्नेर॑ग्निहो॒त्रमे॒तर्\mbox{}हि॒ खलु॒ वा ए॒ष जा॒तो यर्\mbox{}हि॒ सर्व॑श्चि॒तो जा॒तायै॒वास्मा॒ अन्न॒मपि॑ दधाति॒ स ए॑नं प्री॒तः प्री॑णाति॒ वसी॑यान्भवति ब्रह्मवा॒दिनो॑ वदन्ति॒ यदे॒ष गार्\mbox{}ह॑पत्यश्ची॒यते\-ऽथ॒ क्वा᳚स्याहव॒नीय॒ इत्य॒सावा॑दि॒त्य इति॑ ब्रूयादे॒तस्मि॒न् हि सर्वा᳚भ्यो दे॒वता᳚भ्यो॒ जुह्व॑ति~(१४)

%5.7.4.3
य ए॒वं वि॒द्वान॒ग्निं चि॑नु॒ते सा॒क्षादे॒व दे॒वता॑ ऋध्नो॒त्यग्ने॑ यशस्वि॒न्॒ यश॑से॒मम॑र्प॒येन्द्रा॑वती॒मप॑चितीमि॒हा व॑ह। अ॒यं मू॒र्धा प॑रमे॒ष्ठी सु॒वर्चाः᳚ समा॒नाना॑मुत्त॒मश्लो॑को अस्तु। भ॒द्रं पश्य॑न्त॒ उप॑ सेदु॒रग्रे॒ तपो॑ दी॒क्षामृष॑यः सुव॒र्विदः॑। ततः॑ क्ष॒त्रं बल॒मोज॑श्च जा॒तं तद॒स्मै दे॒वा अ॒भि सं न॑मन्तु। धा॒ता वि॑धा॒ता प॑र॒मा~(१५)

%5.7.4.4
उ॒त स॒न्दृक्प्र॒जा\-प॑तिः परमे॒ष्ठी वि॒राजा᳚। स्तोमा॒श्छन्दाꣳ॑सि नि॒विदो॑ म आहुरे॒तस्मै॑ रा॒ष्ट्रम॒भि सं न॑माम। अ॒भ्याव॑र्तध्व॒मुप॒ मेत॑ सा॒कम॒यꣳ शा॒स्ताधि॑पतिर्वो अस्तु। अ॒स्य वि॒ज्ञान॒मनु॒ सꣳ र॑भध्वमि॒मं प॒श्चादनु॑ जीवाथ॒ सर्वे᳚। रा॒ष्ट्रभृत॑ ए॒ता उप॑ दधात्ये॒षा वा अ॒ग्नेश्चिती॑ राष्ट्र॒भृत्तयै॒वास्मि॑न्रा॒ष्ट्रं द॑धाति रा॒ष्ट्रमे॒व भ॑वति॒ नास्मा᳚द्रा॒ष्ट्रं भ्रꣳ॑शते॥~(१६)

{\anuvakamend[{भा॒ग॒धेय॒ञ्जुह्व॑ति पर॒मा रा॒ष्ट्रं द॑धाति स॒प्त च॑}]}%~(४)

%5.7.5.1
यथा॒ वै पु॒त्रो जा॒तो म्रि॒यत॑ ए॒वं वा ए॒ष म्रि॑यते॒ यस्या॒ग्निरुख्य॑ उ॒द्वाय॑ति॒ यन्नि॑र्म॒न्थ्यं॑ कु॒र्याद्विच्छि॑न्द्या॒द्भ्रातृ॑व्यमस्मै जनये॒थ्स ए॒व पुनः॑ प॒रीध्यः॒ स्वादे॒वैनं॒ योने᳚र्जनयति॒ नास्मै॒ भ्रातृ॑व्यं जनयति॒ तमो॒ वा ए॒तं गृ॑ह्णाति॒ यस्या॒ग्निरुख्य॑ उ॒द्वाय॑ति मृ॒त्युस्तमः॑ कृ॒ष्णं वासः॑ कृ॒ष्णा धे॒नुर्दक्षि॑णा॒ तम॑सा~(१७)

%5.7.5.2
ए॒व तमो॑ मृ॒त्युमप॑ हते॒ हिर॑ण्यं ददाति॒ ज्योति॒र्वै हिर॑ण्यं॒ ज्योति॑षै॒व तमो\-ऽप॑ ह॒ते\-ऽथो॒ तेजो॒ वै हिर॑ण्य॒न्तेज॑ ए॒वा\-ऽऽ\-त्मन्ध॑त्ते॒ सुव॒र्न घ॒र्मः स्वाहा॒ सुव॒र्नार्कः स्वाहा॒ सुव॒र्न शु॒क्रः स्वाहा॒ सुव॒र्न ज्योतिः॒ स्वाहा॒ सुव॒र्न सूर्यः॒ स्वाहा॒र्को वा ए॒ष यद॒ग्निर॒सावा॑दि॒त्यः~(१८)

%5.7.5.3
अ॒श्व॒मे॒धो यदे॒ता आहु॑तीर्जु॒होत्य॑र्काश्वमे॒धयो॑रे॒व ज्योतीꣳ॑षि॒ सं द॑धात्ये॒ष ह॒ त्वा अ॑र्काश्वमे॒धी यस्यै॒तद॒ग्नौ क्रि॒यत॒ आपो॒ वा इ॒दमग्रे॑ सलि॒लमा॑सी॒थ्स ए॒तां प्र॒जा\-प॑तिः प्रथ॒मां चिति॑मपश्य॒त्तामुपा॑धत्त॒ तदि॒यम॑भव॒त्तं वि॒श्वक॑र्माब्रवी॒दुप॒ त्वाया॒नीति॒ नेह लो॒को᳚\-ऽस्तीति॑~(१९)

%5.7.5.4
अ॒ब्र॒वी॒थ्स ए॒तां द्वि॒तीयां॒ चिति॑मपश्य॒त्तामुपा॑धत्त॒ तद॒न्तरि॑क्षमभव॒थ्स य॒ज्ञः प्र॒जा\-प॑तिमब्रवी॒दुप॒ त्वाया॒नीति॒ नेह लो॒को᳚\-ऽस्तीत्य॑ब्रवी॒थ्स वि॒श्वक॑र्माणमब्रवी॒दुप॒ त्वाया॒नीति॒ केन॑ मो॒पैष्य॒सीति॒ दिश्या॑भि॒रित्य॑ब्रवी॒त्तन्दिश्या॑भिरु॒पैत्ता उपा॑धत्त॒ ता दिशः॑~(२०)

%5.7.5.5
अ॒भ॒व॒न्थ्स प॑रमे॒ष्ठी प्र॒जा\-प॑तिमब्रवी॒दुप॒ त्वाया॒नीति॒ नेह लो॒को᳚\-ऽस्तीत्य॑ब्रवी॒थ्स वि॒श्वक॑र्माणं च य॒ज्ञं चा᳚ब्रवी॒दुप॑ वा॒माया॒नीति॒ नेह लो॒को᳚\-ऽस्तीत्य॑ब्रूता॒ꣳ॒ स ए॒तां तृ॒तीयां॒ चिति॑मपश्य॒त्तामुपा॑धत्त॒ तद॒साव॑भव॒थ्स आ॑दि॒त्यः प्र॒जा\-प॑तिमब्रवी॒दुप॑ त्वा~(२१)

%5.7.5.6
आ॒या॒नीति॒ नेह लो॒को᳚\-ऽस्तीत्य॑ब्रवी॒थ्स वि॒श्वक॑र्माणं च य॒ज्ञं चा᳚ब्रवी॒दुप॑ वा॒माया॒नीति॒ नेह लो॒को᳚\-ऽस्तीत्य॑ब्रूता॒ꣳ॒ स प॑रमे॒ष्ठिन॑मब्रवी॒दुप॒ त्वाया॒नीति॒ केन॑ मो॒पैष्य॒सीति॑ लोकं पृ॒णयेत्य॑ब्रवी॒त्तं लो॑कं पृ॒णयो॒पैत्तस्मा॒दया॑तयाम्नी लोकं पृ॒णा\-ऽया॑तयामा॒ ह्य॑सौ~(२२)

%5.7.5.7
आ॒दि॒त्यस्तानृष॑यो\-ऽब्रुव॒न्नुप॑ व॒ आया॒मेति॒ केन॑ न उ॒पैष्य॒थेति॑ भू॒म्नेत्य॑ब्रुव॒न्तां द्वाभ्यां॒ चिती᳚भ्यामु॒पाय॒न्थ्स पञ्च॑चितीकः॒ सम॑पद्यत॒ य ए॒वं वि॒द्वान॒ग्निं चि॑नु॒ते भूया॑ने॒व भ॑वत्य॒भीमाँल्लो॒काञ्ज॑यति वि॒दुरे॑नं दे॒वा अथो॑ ए॒तासा॑मे॒व दे॒वता॑ना॒ꣳ॒ सायु॑ज्यं गच्छति॥~(२३)

{\anuvakamend[{तम॑सा\-ऽऽ\-दि॒त्यो᳚\-ऽस्तीति॒ दिश॑ आदि॒त्यः प्र॒जा\-प॑तिमब्रवी॒दुप॑ त्वा॒\-ऽसौ पञ्च॑चत्वारिꣳशच्च}]}%~(६)

%5.7.6.1
वयो॒ वा अ॒ग्निर्यद॑ग्नि॒चित्प॒क्षिणो᳚\-ऽश्ञी॒यात्तमे॒वाग्निम॑द्या॒दार्ति॒मार्च्छे᳚थ्संवथ्स॒रं व्र॒तं च॑रेथ्संवथ्स॒रꣳ हि व्र॒तं नाति॑ प॒शुर्वा ए॒ष यद॒ग्निर्\mbox{}हि॒नस्ति॒ खलु॒ वै तं प॒शुर्य ए॑नं पु॒रस्ता᳚त्प्र॒त्यञ्च॑मुप॒चर॑ति॒ तस्मा᳚त्प॒श्चात्प्राङु॑प॒चर्य॑ आ॒त्मनो\-ऽहिꣳ॑सायै॒ तेजो॑\-ऽसि॒ तेजो॑ मे यच्छ पृथि॒वीं य॑च्छ~(२४)

%5.7.6.2
पृ॒थि॒व्यै मा॑ पाहि॒ ज्योति॑रसि॒ ज्योति॑र्मे यच्छा॒न्तरि॑क्षं यच्छा॒न्तरि॑क्षान्मा पाहि॒ सुव॑रसि॒ सुव॑र्मे यच्छ॒ दिवं॑ यच्छ दि॒वो मा॑ पा॒हीत्या॑है॒ताभि॒र्वा इ॒मे लो॒का विधृ॑ता॒ यदे॒ता उ॑प॒दधा᳚त्ये॒षां लो॒कानां॒ विधृ॑त्यै स्वयमातृ॒ण्णा उ॑प॒धाय॑ हिरण्येष्ट॒का उप॑ दधाती॒मे वै लो॒काः स्व॑यमातृ॒ण्णा ज्योति॒र्॒\mbox{}हिर॑ण्यं॒ यथ्स्व॑यमातृ॒ण्णा उ॑प॒धाय॑~(२५)

%5.7.6.3
हि॒र॒ण्ये॒ष्ट॒का उ॑प॒दधा॑ती॒माने॒वैताभि॑र्लो॒कां ज्योति॑ष्मतः कुरु॒ते\-ऽथो॑ ए॒ताभि॑रे॒वास्मा॑ इ॒मे लो॒काः प्र भा᳚न्ति॒ यास्ते॑ अग्ने॒ सूर्ये॒ रुच॑ उद्य॒तो दिव॑मात॒न्वन्ति॑ र॒श्मिभिः॑। ताभिः॒ सर्वा॑भी रु॒चे जना॑य नस्कृधि। या वो॑ देवाः॒ सूर्ये॒ रुचो॒ गोष्वश्वे॑षु॒ या रुचः॑। इन्द्रा᳚ग्नी॒ ताभिः॒ सर्वा॑भी॒ रुचं॑ नो धत्त बृहस्पते। रुचं॑ नो धेहि~(२६)

%5.7.6.4
ब्रा॒ह्म॒णेषु॒ रुच॒ꣳ॒ राज॑सु नस्कृधि। रुचं॑ वि॒श्ये॑षु शू॒द्रेषु॒ मयि॑ धेहि रु॒चा रुचम्᳚। द्वे॒धा वा अ॒ग्निं चि॑क्या॒नस्य॒ यश॑ इन्द्रि॒यं ग॑च्छत्य॒ग्निं वा॑ चि॒तमी॑जा॒नं वा॒ यदे॒ता आहु॑तीर्जु॒होत्या॒त्मन्ने॒व यश॑ इन्द्रि॒यं ध॑त्त ईश्व॒रो वा ए॒ष आर्ति॒मार्तो॒र्यो᳚\-ऽग्निं चि॒न्वन्न॑धि॒क्राम॑ति॒ तत्त्वा॑ यामि॒ ब्रह्म॑णा॒ वन्द॑मान॒ इति॑ वारु॒ण्यर्चा~(२७)

%5.7.6.5
जु॒हु॒या॒च्छान्ति॑रे॒वैषाग्नेर्गुप्ति॑रा॒त्मनो॑ ह॒विष्कृ॑तो॒ वा ए॒ष यो᳚\-ऽग्निं चि॑नु॒ते यथा॒ वै ह॒विः स्कन्द॑त्ये॒वं वा ए॒ष स्क॑न्दति॒ यो᳚\-ऽग्निं चि॒त्वा स्त्रिय॑मु॒पैति॑ मैत्रावरु॒ण्यामिक्ष॑या यजेत मैत्रावरु॒णता॑मे॒वोपै᳚त्या॒त्मनो\-ऽस्क॑न्दाय॒ यो वा अ॒ग्निमृ॑तु॒स्थां वेद॒र्तुर्\mbox{}ऋ॑तुरस्मै॒ कल्प॑मान एति॒ प्रत्ये॒व ति॑ष्ठति संवथ्स॒रो वा अ॒ग्निः~(२८)

%5.7.6.6
ऋ॒तु॒स्थास्तस्य॑ वस॑न्तः॒ शिरो᳚ ग्री॒ष्मो दक्षि॑णः प॒क्षो व॒र्॒\mbox{}षाः पुच्छꣳ॑ श॒रदुत्त॑रः प॒क्षो हे॑म॒न्तो मध्यं॑ पूर्वप॒क्षाश्चित॑यो\-ऽपरप॒क्षाः पुरी॑षमहोरा॒त्राणीष्ट॑का ए॒ष वा अ॒ग्निर्\mbox{}ऋ॑तु॒स्था य ए॒वं वेद॒र्तुर्\mbox{}ऋ॑तुरस्मै॒ कल्प॑मान एति॒ प्रत्ये॒व ति॑ष्ठति प्र॒जा\-प॑ति॒र्वा ए॒तं ज्यैष्ठ्य॑कामो॒ न्य॑धत्त॒ ततो॒ वै स ज्यैष्ठ्य॑मगच्छ॒द्य ए॒वं वि॒द्वान॒ग्निं चि॑नु॒ते ज्यैष्ठ्य॑मे॒व ग॑च्छति॥~(२९)

{\anuvakamend[{पृ॒थि॒वीं य॑च्छ॒ यथ्स्व॑यमातृ॒ण्णा उ॑प॒धाय॑ धेह्यृ॒चाग्निश्चि॑नु॒ते त्रीणि॑ च}]}%~(७)

%5.7.7.1
यदाकू॑ताथ्स॒मसु॑स्रोद्धृ॒दो वा॒ मन॑सो वा॒ सम्भृ॑तं॒ चक्षु॑षो वा। तमनु॒ प्रेहि॑ सुकृ॒तस्य॑ लो॒कं यत्रर्\mbox{}ष॑यः प्रथम॒जा ये पु॑रा॒णाः। ए॒तꣳ स॑धस्थ॒ परि॑ ते ददामि॒ यमा॒वहा᳚च्छेव॒धिं जा॒तवे॑दाः। अ॒न्वा॒ग॒न्ता य॒ज्ञप॑तिर्वो॒ अत्र॒ तꣴ स्म॑ जानीत पर॒मे व्यो॑मन्न्। जा॒नी॒तादे॑नं पर॒मे व्यो॑म॒न्देवाः᳚ सधस्था वि॒द रू॒पम॑स्य। यदा॒गच्छा᳚त्~(३०)

%5.7.7.2
प॒थिभि॑र्देव॒यानै॑रिष्टापू॒र्ते कृ॑णुतादा॒विर॑स्मै। सम्प्र च्य॑वध्व॒मनु॒ सम्प्र या॒ताग्ने॑ प॒थो दे॑व॒याना᳚न्कृणुध्वम्। अ॒स्मिन्थ्स॒धस्थे॒ अध्युत्त॑रस्मि॒न्विश्वे॑ देवा॒ यज॑मानश्च सीदत। प्र॒स्त॒रेण॑ परि॒धिना᳚ स्रु॒चा वेद्या॑ च ब॒र्॒\mbox{}हिषा᳚। ऋ॒चेमं य॒ज्ञं नो॑ वह॒ सुव॑र्दे॒वेषु॒ गन्त॑वे। यदि॒ष्टं यत्प॑रा॒दानं॒ यद्द॒त्तं या च॒ दक्षि॑णा। तत्~(३१)

%5.7.7.3
अ॒ग्निर्वै᳚श्वकर्म॒णः सुव॑र्दे॒वेषु॑ नो दधत्। येना॑ स॒हस्रं॒ वह॑सि॒ येना᳚ग्ने सर्ववेद॒सम्। तेने॒मं य॒ज्ञं नो॑ वह॒ सुव॑र्दे॒वेषु॒ गन्त॑वे। येना᳚ग्ने॒ दक्षि॑णा यु॒क्ता य॒ज्ञं वह॑न्त्यृ॒त्विजः॑। तेने॒मं य॒ज्ञं नो॑ वह॒ सुव॑र्दे॒वेषु॒ गन्त॑वे। येना᳚ग्ने सु॒कृतः॑ प॒था मधो॒र्धारा᳚ व्यान॒शुः। तेने॒मं य॒ज्ञं नो॑ वह॒ सुव॑र्दे॒वेषु॒ गन्त॑वे। यत्र॒ धारा॒ अन॑पेता॒ मधो᳚र्घृ॒तस्य॑ च॒ याः। तद॒ग्निर्वै᳚श्वकर्म॒णः सुव॑र्दे॒वेषु॑ नो दधत्॥~(३२)

{\anuvakamend[{आ॒गच्छा॒त्तद्व्या॑न॒शुस्तेने॒मं य॒ज्ञं नो॑ वह॒ सुव॑र्दे॒वेषु॒ गन्त॑वे॒ चतु॑र्दश च}]}%~(७)

%5.7.8.1
यास्ते॑ अग्ने स॒मिधो॒ यानि॒ धाम॒ या जि॒ह्वा जा॑तवेदो॒ यो अ॒र्चिः। ये ते॑ अग्ने मे॒डयो॒ य इन्द॑व॒स्तेभि॑रा॒त्मानं॑ चिनुहि प्रजा॒नन्न्। उ॒थ्स॒न्न॒य॒ज्ञो वा ए॒ष यद॒ग्निः किं वाहै॒तस्य॑ क्रि॒यते॒ किं वा॒ न यद्वा अ॑ध्व॒र्युर॒ग्नेश्चि॒न्वन्न॑न्त॒रेत्या॒त्मनो॒ वै तद॒न्तरे॑ति॒ यास्ते॑ अग्ने स॒मिधो॒ यानि॑~(३३)

%5.7.8.2
धामेत्या॑है॒षा वा अ॒ग्नेः स्व॑यञ्चि॒तिर॒ग्निरे॒व तद॒ग्निं चि॑नोति॒ नाध्व॒र्युरा॒त्मनो॒\-ऽन्तरे॑ति॒ चत॑स्र॒ आशाः॒ प्र च॑रन्त्व॒ग्नय॑ इ॒मं नो॑ य॒ज्ञं न॑यतु प्रजा॒नन्न्। घृ॒तं पिन्व॑न्न॒जरꣳ॑ सु॒वीरं॒ ब्रह्म॑ स॒मिद्भ॑व॒त्याहु॑तीनाम्। सु॒व॒र्गाय॒ वा ए॒ष लो॒कायोप॑ धीयते॒ यत्कू॒र्मश्चत॑स्र॒ आशाः॒ प्र च॑रन्त्व॒ग्नय॒ इत्या॑ह~(३४)

%5.7.8.3
दिश॑ ए॒वैतेन॒ प्र जा॑नाती॒मं नो॑ य॒ज्ञं न॑यतु प्रजा॒नन्नित्या॑ह सुव॒र्गस्य॑ लो॒कस्या॒भ᳚नी॑त्यै॒ ब्रह्म॑ स॒मिद्भ॑व॒त्याहु॑तीना॒मित्या॑ह॒ ब्रह्म॑णा॒ वै दे॒वाः सु॑व॒र्गं लो॒कमा॑य॒न्॒ यद्ब्रह्म॑ण्वत्योप॒दधा॑ति॒ ब्रह्म॑णै॒व तद्यज॑मानः सुव॒र्गं लो॒कमे॑ति प्र॒जा\-प॑ति॒र्वा ए॒ष यद॒ग्निस्तस्य॑ प्र॒जाः प॒शव॒श्छन्दाꣳ॑सि रू॒पꣳ सर्वा॒न् वर्णा॒निष्ट॑कानां कुर्याद्रू॒पेणै॒व प्र॒जां प॒शूञ्छन्दा॒ꣴ॒स्यव॑ रु॒न्धे\-ऽथो᳚ प्र॒जाभ्य॑ ए॒वैनं॑ प॒शुभ्य॒श्छन्दो᳚भ्यो\-ऽव॒रुद्ध्य॑ चिनुते॥~(३५)

{\anuvakamend[{यान्य॒ग्नय॒ इत्या॒हेष्ट॑काना॒ꣳ॒ षोड॑श च}]}%~(८)

%5.7.9.1
मयि॑ गृह्णा॒म्यग्रे॑ अ॒ग्निꣳ रा॒यस्पोषा॑य सुप्रजा॒स्त्वाय॑ सु॒वीर्या॑य। मयि॑ प्र॒जां मयि॒ वर्चो॑ दधा॒म्यरि॑ष्टाः स्याम त॒नुवा॑ सु॒वीराः᳚। यो नो॑ अ॒ग्निः पि॑तरो हृ॒थ्स्व॑न्तरम॑र्त्यो॒ मर्त्याꣳ॑ आवि॒वेश॑। तमा॒त्मन्परि॑ गृह्णीमहे व॒यं मा सो अ॒स्माꣳ अ॑व॒हाय॒ परा॑ गात्। यद॑ध्व॒र्युरा॒त्मन्न॒ग्निमगृ॑हीत्वा॒ग्निं चि॑नु॒याद्यो᳚\-ऽस्य॒ स्वो᳚\-ऽग्निस्तमपि॑~(३६)

%5.7.9.2
यज॑मानाय चिनुयाद॒ग्निं खलु॒ वै प॒शवो\-ऽनूप॑ तिष्ठन्ते\-ऽप॒क्रामु॑का अस्मात्प॒शवः॑ स्यु॒र्मयि॑ गृह्णाम्यग्रे॑ अ॒ग्निमित्या॑हा॒त्मन्ने॒व स्वम॒ग्निं दा॑धार॒ नास्मा᳚त्प॒शवो\-ऽप॑ क्रामन्ति ब्रह्मवा॒दिनो॑ वदन्ति॒ यन्मृच्चाप॑श्चा॒ग्नेर॑ना॒द्यमथ॒ कस्मा᳚न्मृ॒दा चा॒द्भिश्चा॒ग्निश्ची॑यत॒ इति॒ यद॒द्भिः सं॒यौति॑~(३७)

%5.7.9.3
आपो॒ वै सर्वा॑ दे॒वता॑ दे॒वता॑भिरे॒वैन॒ꣳ॒ सꣳ सृ॑जति॒ यन्मृ॒दा चि॒नोती॒यं वा अ॒ग्निर्वै᳚श्वान॒रो᳚\-ऽग्निनै॒व तद॒ग्निं चि॑नोति ब्रह्मवा॒दिनो॑ वदन्ति॒ यन्मृ॒दा चा॒द्भिश्चा॒ग्निश्ची॒यते\-ऽथ॒ कस्मा॑द॒ग्निरु॑च्यत॒ इति॒ यच्छन्दो॑भिश्चि॒नोत्य॒ग्नयो॒ वै छन्दाꣳ॑सि॒ तस्मा॑द॒ग्निरु॑च्य॒ते\-ऽथो॑ इ॒यं वा अ॒ग्निर्वै᳚श्वान॒रो यत्~(३८)

%5.7.9.4
मृ॒दा चि॒नोति॒ तस्मा॑द॒ग्निरु॑च्यते हिरण्येष्ट॒का उप॑ दधाति॒ ज्योति॒र्वै हिर॑ण्यं॒ ज्योति॑रे॒वास्मि॑न्दधा॒त्यथो॒ तेजो॒ वै हिर॑ण्यं॒ तेज॑ ए॒वा\-ऽऽ\-त्मन्ध॑त्ते॒ यो वा अ॒ग्निꣳ स॒र्वतो॑मुखं चिनु॒ते सर्वा॑सु प्र॒जास्वन्न॑मत्ति॒ सर्वा॒ दिशो॒\-ऽभि ज॑यति गाय॒त्रीं पु॒रस्ता॒दुप॑ दधाति त्रिष्टुभं॑ दक्षिण॒तो जग॑तीं प॒श्चाद॑नु॒ष्टुभ॑मुत्तर॒तः प॒ङ्क्तिं मध्य॑ ए॒ष वा अ॒ग्निः स॒र्वतो॑मुख॒स्तं य ए॒वं वि॒द्वाꣴश्चि॑नु॒ते सर्वा॑सु प्र॒जास्वन्न॑मत्ति॒ सर्वा॒ दिशो॒\-ऽभि ज॑य॒त्यथो॑ दि॒श्ये॑व दिशं॒ प्र व॑यति॒ तस्मा᳚द्दि॒शि दिक्प्रोता᳚॥~(३९)

{\anuvakamend[{अपि॑ सं॒ यौति॑ वैश्वान॒रो यदे॒ष वै पञ्च॑विꣳशतिश्च}]}%~(९)

%5.7.10.1
प्र॒जा\-प॑तिर॒ग्निम॑सृजत॒ सो᳚\-ऽस्माथ्सृ॒ष्टः प्राङ्प्राद्र॑व॒त्तस्मा॒ अश्वं॒ प्रत्या᳚स्य॒थ्स द॑क्षि॒णाव॑र्तत॒ तस्मै॑ वृ॒ष्णिं प्रत्या᳚स्य॒थ्स प्र॒त्यङ्ङाव॑र्तत॒ तस्मा॑ ऋष॒भं प्रत्या᳚स्य॒थ्स उद॒ङ्ङाव॑र्तत॒ तस्मै॑ ब॒स्तं प्रत्या᳚स्य॒थ्स ऊ॒र्ध्वो᳚\-ऽद्रव॒त्तस्मै॒ पुरु॑षं॒ प्रत्या᳚स्य॒त् यत्प॑शुशी॒र्॒\mbox{}षाण्यु॑प॒दधा॑ति स॒र्वत॑ ए॒वैनम्᳚~(४०)

%5.7.10.2
अ॒व॒रुध्य॑ चिनुत ए॒ता वै प्रा॑ण॒भृत॒श्चक्षु॑ष्मती॒रिष्ट॑का॒ यत्प॑शुशी॒र्॒\mbox{}षाणि॒ यत्प॑शुशी॒र्॒\mbox{}षाण्यु॑प॒दधा॑ति॒ ताभि॑रे॒व यज॑मानो॒\-ऽमुष्मिँ॑ल्लो॒के प्राणि॒त्यथो॒ ताभि॑रे॒वास्मा॑ इ॒मे लो॒काः प्र भा᳚न्ति मृ॒दाभि॒लिप्योप॑ दधाति मेध्य॒त्वाय॑ प॒शुर्वा ए॒ष यद॒ग्निरन्नं॑ प॒शव॑ ए॒ष खलु॒ वा अ॒ग्निर्यत्प॑शुशी॒र्॒\mbox{}षाणि॒ यं का॒मये॑त॒ कनी॑यो॒\-ऽस्यान्नम्᳚~(४१)

%5.7.10.3
स्या॒दिति॑ सन्त॒रां तस्य॑ पशुशी॒र्॒\mbox{}षाण्युप॑ दध्या॒त्कनी॑य ए॒वास्यान्न॑म्भवति॒ यं का॒मये॑त स॒माव॑द॒स्यान्नꣴ॑ स्या॒दिति॑ मध्य॒तस्तस्योप॑ दध्याथ्स॒माव॑दे॒वास्यान्न॑म्भवति॒ यं का॒मये॑त॒ भूयो॒\-ऽस्यान्नꣴ॑ स्या॒दित्यन्ते॑षु॒ तस्य॑ व्यु॒दूह्योप॑ दध्यादन्त॒त ए॒वास्मा॒ अन्न॒मव॑ रुन्धे॒ भूयो॒\-ऽस्यान्न॑म्भवति॥~(४२)

{\anuvakamend[{ए॒न॒म॒स्यान्न॒म्भूयो॒स्यान्न॑म्भवति}]}%॥10॥

%5.7.11.1
स्ते॒गान्दꣴष्ट्रा᳚भ्यां म॒ण्डूका॒ञ्जम्भ्ये॑भि॒राद॑कां खा॒देनोर्जꣳ॑ सꣳसू॒देनार॑ण्यं॒ जाम्बी॑लेन॒ मृद॑म्ब॒र्स्वे॑भिः॒ शर्क॑राभि॒रव॑का॒मव॑काभिः॒ शर्क॑रामुथ्सा॒देन॑ जि॒ह्वाम॑वक्र॒न्देन॒ तालु॒ꣳ॒ सर॑स्वतीं जिह्वा॒ग्रेण॑॥~(४३)

{\anuvakamend[{स्ते॒गान्द्वाविꣳ॑शतिः}]}%॥11॥

%5.7.12.1
वाज॒ꣳ॒ हनू᳚भ्याम॒प आ॒स्ये॑नादि॒त्याञ्छ्मश्रु॑भिरुपया॒ममध॑रे॒णोष्ठे॑न॒ सदुत्त॑रे॒णान्त॑रेणानूका॒शं प्र॑का॒शेन॒ बाह्यꣴ॑ स्तनयि॒त्नुं नि॑र्बा॒धेन॑ सूर्या॒ग्नी चक्षु॑र्भ्यां वि॒द्युतौ॑ क॒नान॑काभ्याम॒शनिं॑ म॒स्तिष्के॑ण॒ बलं॑ म॒ज्जभिः॑॥~(४४)

{\anuvakamend[{वाजं॒ पञ्च॑विꣳशतिः}]}%॥12॥

%5.7.13.1
कू॒र्माञ्छ॒फैर॒च्छला॑भिः क॒पिञ्ज॑ला॒न्थ्साम॒ कुष्ठि॑काभिर्ज॒वं जङ्घा॑भिरग॒दं जानु॑भ्यां वी॒र्यं॑ कु॒हा\-भ्यां᳚ भ॒यं प्र॑चा॒लाभ्यां॒ गुहो॑पप॒क्षाभ्या॑म॒श्विना॒वꣳसा᳚भ्या॒मदि॑तिꣳ शी॒र्ष्णा निर्\mbox{}ऋ॑तिं॒ निर्जा᳚ल्मकेन शी॒र्ष्णा॥~(४५)

{\anuvakamend[{कू॒र्मान्त्रयो॑विꣳशतिः}]}%॥13॥

%5.7.14.1
योक्त्रं॒ गृध्रा॑भिर्यु॒गमान॑तेन चि॒त्तं मन्या॑भिः सङ्क्रो॒शान्प्रा॒णैः प्र॑का॒शेन॒ त्वचं॑ पराका॒शेनान्त॑रां म॒शका॒न्केशै॒रिन्द्र॒ꣴ॒ स्वप॑सा॒ वहे॑न॒ बृह॒स्पतिꣳ॑ शकुनिसा॒देन॒ रथ॑मु॒ष्णिहा॑भिः॥~(४६)

{\anuvakamend[{योक्त्र॒मेक॑विꣳशतिः}]}%॥14॥

%5.7.15.1
मि॒त्रावरु॑णौ॒ श्रोणी᳚भ्यामिन्द्रा॒ग्नी शि॑ख॒ण्डाभ्या॒मिन्द्रा॒बृह॒स्पती॑ ऊ॒रुभ्या॒मिन्द्रा॒विष्णू॑ अष्ठी॒वद्भ्याꣳ॑ सवि॒तारं॒ पुच्छे॑न गन्ध॒र्वाञ्छेपे॑नाफ्स॒रसो॑ मु॒ष्काभ्यां॒ पव॑मानं पा॒युना॑ प॒वित्रं॒ पोत्रा᳚भ्यामा॒क्रम॑णꣴ स्थू॒रा\-भ्यां᳚ प्रति॒क्रम॑णं॒ कुष्ठा᳚भ्याम्॥~(४७)

{\anuvakamend[{}]}%॥15॥

%5.7.16.1
इन्द्र॑स्य क्रो॒डो\-ऽदि॑त्यै पाज॒स्य॑न्दि॒शां ज॒त्रवो॑ जी॒मूता᳚न्हृदयौप॒शाभ्या॑म॒न्तरि॑क्षं पुरि॒तता॒ नभ॑ उद॒र्ये॑णेन्द्रा॒णीं प्ली॒ह्ना व॒ल्मीका᳚न्क्लो॒म्ना गि॒रीन्प्ला॒शिभिः॑ समु॒द्रमु॒दरे॑ण वैश्वान॒रं भस्म॑ना॥~(४८)

{\anuvakamend[{मि॒त्रावरु॑णा॒विन्द्र॑स्य॒ द्वाविꣳ॑शति॒र्द्वाविꣳ॑शतिः}]}%॥16॥

%5.7.17.1
पू॒ष्णो व॑नि॒ष्ठुर॑न्धा॒हेः स्थू॑रगु॒दा स॒र्पान्गुदा॑भिर्\mbox{}ऋ॒तून्पृ॒ष्टीभि॒र्दिवं॑ पृ॒ष्ठेन॒ वसू॑नां प्रथ॒मा कीक॑सा रु॒द्राणां᳚ द्वि॒तीया॑दि॒त्यानां᳚ तृ॒तीयाङ्गि॑रसां चतु॒र्थी सा॒ध्यानां᳚ पञ्च॒मी विश्वे॑षां दे॒वानाꣳ॑ ष॒ष्ठी॥~(४९)

{\anuvakamend[{पू॒ष्णश्चतु॑र्विꣳशतिः}]}%॥17॥

%5.7.18.1
ओजो᳚ ग्री॒वाभि॒र्निर्\mbox{}ऋ॑तिम॒स्थभि॒रिन्द्र॒ꣴ॒ स्वप॑सा॒ वहे॑न रु॒द्रस्य॑ विच॒लः स्क॒न्धो॑\-ऽहोरा॒त्रयो᳚र्द्वि॒तीयो᳚\-ऽर्धमा॒सानां᳚ तृ॒तीयो॑ मा॒सां च॑तु॒र्थ ऋ॑तू॒नां प॑ञ्च॒मः सं॑वथ्स॒रस्य॑ ष॒ष्ठः॥~(५०)

{\anuvakamend[{ओजो॑ विꣳश॒तिः}]}%॥18॥

%5.7.19.1
आ॒न॒न्दं न॒न्दथु॑ना॒ कामं॑ प्रत्या॒सा\-भ्यां᳚ भ॒यꣳ शि॑ती॒म\-भ्यां᳚ प्र॒शिषं॑ प्रशा॒साभ्याꣳ॑ सूर्याचन्द्र॒मसौ॒ वृक्या᳚भ्याꣴ श्यामशब॒लौ मत॑स्नाभ्या॒व्व्युँ॑ष्टिꣳ रू॒पेण॒ निम्रु॑क्ति॒मरू॑पेण॥~(५१)

{\anuvakamend[{आ॒न॒न्दꣳ षोड॑श}]}%॥19॥

%5.7.20.1
अह॑र्मा॒ꣳ॒सेन॒ रात्रिं॒ पीव॑सा॒पो यू॒षेण॑ घृ॒तꣳ रसे॑न॒ श्यां वस॑या दू॒षीका॑भिर्\mbox{}ह्रा॒दुनि॒मश्रु॑भिः॒ पृष्वा॒न्दिवꣳ॑ रू॒पेण॒ नक्ष॑त्राणि॒ प्रति॑रूपेण पृथि॒वीं चर्म॑णा छ॒वीं छ॒व्यो॑पाकृ॑ताय॒ स्वाहाल॑ब्धाय॒ स्वाहा॑ हु॒ताय॒ स्वाहा᳚॥~(५२)

{\anuvakamend[{अह॑र॒ष्टाविꣳ॑शतिः}]}%॥20॥

%5.7.21.1
अ॒ग्नेः प॑क्ष॒तिः सर॑स्वत्यै॒ निप॑क्षतिः॒ सोम॑स्य तृ॒तीया॒पां च॑तु॒र्थ्योष॑धीनां पञ्च॒मी सं॑वथ्स॒रस्य॑ ष॒ष्ठी म॒रुताꣳ॑ सप्त॒मी बृह॒स्पते॑रष्ट॒मी मि॒त्रस्य॑ नव॒मी वरु॑णस्य दश॒मीन्द्र॑स्यैकाद॒शी विश्वे॑षां दे॒वानां᳚ द्वाद॒शी द्यावा॑पृथि॒व्योः पा॒र्श्वं य॒मस्य॑ पाटू॒रः॥~(५३)

{\anuvakamend[{अ॒ग्नेरेका॒न्नत्रि॒ꣳ॒शत्}]}%॥21॥

%5.7.22.1
वा॒योः प॑क्ष॒तिः सर॑स्वतो॒ निप॑क्षतिश्च॒न्द्रम॑सस्तृ॒तीया॒ नक्ष॑त्राणां चतु॒र्थी स॑वि॒तुः प॑ञ्च॒मी रु॒द्रस्य॑ ष॒ष्ठी स॒र्पाणाꣳ॑ सप्त॒म्य॑र्य॒म्णो᳚\-ऽष्ट॒मी त्वष्टु॑र्नव॒मी धा॒तुर्द॑श॒मीन्द्रा॒ण्या ए॑काद॒श्यदि॑त्यै द्वाद॒शी द्यावा॑पृथि॒व्योः पा॒र्श्वं य॒म्यै॑ पाटू॒रः॥~(५४)

{\anuvakamend[{वा॒योर॒ष्टाविꣳ॑शतिः}]}%॥22॥

%5.7.23.1
पन्था॑मनू॒वृग्भ्या॒ꣳ॒ सन्त॑तिꣴ स्नाव॒न्या᳚भ्या॒ꣳ॒ शुका᳚न्पि॒त्तेन॑ हरि॒माणं॑ य॒क्ना हली᳚क्ष्णान्पापवा॒तेन॑ कू॒श्माञ्छक॑भिः शव॒र्तानूव॑ध्येन॒ शुनो॑ वि॒शस॑नेन स॒र्पाल्लोँ॑हितग॒न्धेन॒ वयाꣳ॑सि पक्वग॒न्धेन॑ पि॒पीलि॑काः प्रशा॒देन॑॥~(५)

{\anuvakamend[{पन्था॒न्द्वाविꣳ॑शतिः}]}%॥23॥

%5.7.24.1
क्रमै॒रत्य॑क्रमीद्वा॒जी विश्वै᳚र्दे॒वैर्य॒ज्ञियैः᳚ संविदा॒नः। स नो॑ नय सुकृ॒तस्य॑ लो॒कं तस्य॑ ते व॒यꣴ स्व॒धया॑ मदेम॥~(५६)

{\anuvakamend[{क्रमै॑र॒ष्टाद॑श}]}%॥24॥

%5.7.25.1
द्यौस्ते॑ पृ॒ष्ठं पृ॑थि॒वी स॒धस्थ॑मा॒त्मान्तरि॑क्षꣳ समु॒द्रो योनिः॒ सूर्य॑स्ते॒ चक्षु॒र्वातः॑ प्रा॒णश्च॒न्द्रमाः॒ श्रोत्रं॒ मासा᳚श्चार्धमा॒साश्च॒ पर्वा᳚ण्यृ॒तवोङ्गा॑नि संवथ्स॒रो म॑हि॒मा॥~(५७)

{\anuvakamend[{द्यौः पञ्च॑विꣳशतिः}]}%॥25॥

%5.7.26.1
अ॒ग्निः प॒शुरा॑सी॒त्तेना॑यजन्त॒ स ए॒तं लो॒कम॑जय॒द्यस्मि॑न्न॒ग्निः स ते॑ लो॒कस्तं जे᳚ष्य॒स्यथाव॑ जिघ्र वा॒युः प॒शुरा॑सी॒त्तेना॑यजन्त॒ स ए॒तं लो॒कम॑जय॒द्यस्मि॑न्वा॒युः स ते॑ लो॒कस्तस्मा᳚त्त्वा॒न्तरे᳚ष्यामि॒ यदि॒ नाव॒जिघ्र॑स्यादि॒त्यः प॒शुरा॑सी॒त्तेना॑यजन्त॒ स ए॒तं लो॒कम॑जय॒द्यस्मि॑न्नादि॒त्यः स ते॑ लो॒कस्तं जे᳚ष्यसि॒ यद्य॑व॒जिघ्र॑सि॥~(५८)

{\anuvakamend[{यस्मि॑न्न॒ष्टौ च॑}]}%॥26॥

{\prashnaend[{प्रा॒चीन॑वꣳशं॒ याव॑न्त ऋख्सा॒मे वाग्वै दे॒वेभ्यो॑ दे॒वा वै दे॑व॒यज॑नङ्क॒द्रूश्च॒ तद्धिर॑ण्य॒ꣳ॒ षट्प॒दानि॑ ब्रह्मवा॒दिनो॑ वि॒चित्यो॒ यत्क॒लया॑ ते वारु॒णो वै क्री॒तः सोम॒ एका॑\-दश॥११॥ प्रा॒चीन॑वꣳश॒ꣴ॒ स्वाहेत्या॑ह॒ ये᳚\-ऽन्तः श॒रा ह्ये॑ष सं तप॑सा च॒ यत्क॑र्णगृही॒तेति॑ लोम॒तो वा॑रु॒णः षट्थ्स॑प्ततिः॥७६॥ प्रा॒चीन॑वꣳशं॒ परि॑चरति॥}]}%%५-७

\centerline{॥हरिः॑ ॐ॥}

\centerline{॥कृष्ण-यजुर्वेदीय-तैत्तिरीय-संहितायां पञ्चम्काण्डे सप्तमः प्रश्नः समाप्तः॥५-७॥}
%%% END PRASHNA
