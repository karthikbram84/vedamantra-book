\chapt{काण्डम् ६}
\sect{तृतीयः प्रश्नः}\setcounter{anuvakam}{0}
\dnsub{तैत्तिरीयसंहितायां षष्ठमकाण्डे तृतीयः प्रश्नः}
%6.3.1.1
चात्वा॑ला॒द्धिष्णि॑या॒नुप॑ वपति॒ योनि॒र्वै य॒ज्ञस्य॒ चात्वा॑लं य॒ज्ञस्य॑ सयोनि॒त्वाय॑ दे॒वा वै य॒ज्ञं परा॑जयन्त॒ तमाग्नी᳚ध्रा॒त्पुन॒रपा॑जयन्ने॒तद्वै य॒ज्ञस्याप॑राजितं॒ यदाग्नी᳚ध्रं॒ यदाग्नी᳚ध्रा॒द्धिष्णि॑यान् वि॒हर॑ति॒ यदे॒व य॒ज्ञस्याप॑राजितं॒ तत॑ ए॒वैनं॒ पुन॑स्तनुते परा॒जित्ये॑व॒ खलु॒ वा ए॒ते य॑न्ति॒ ये ब॑हिष्पवमा॒नꣳ सर्प॑न्ति बहिष्पवमा॒ने स्तु॒ते~(१)

%6.3.1.2
आ॒हाग्नी॑द॒ग्नीन् वि ह॑र ब॒र्॒\mbox{}हिः स्तृ॑णाहि पुरो॒डाशा॒ꣳ॒ अलं॑ कु॒र्विति॑ य॒ज्ञमे॒वाप॒जित्य॒ पुन॑स्तन्वा॒ना य॒न्त्यङ्गा॑रै॒र्द्वे सव॑ने॒ वि ह॑रति श॒लाका॑भिस्तृ॒तीयꣳ॑ सशुक्र॒त्वायाथो॒ सम्भ॑रत्ये॒वैन॒द्धिष्णि॑या॒ वा अ॒मुष्मिँ॑ल्लो॒के सोम॑मरक्ष॒न्तेभ्यो\-ऽधि॒ सोम॒माह॑र॒न्तम॑न्व॒वाय॒न्तं पर्य॑विश॒न्॒ य ए॒वं वेद॑ वि॒न्दते᳚~(२)

%6.3.1.3
प॒रि॒वे॒ष्टार॒न्ते सो॑मपी॒थेन॒ व्या᳚र्ध्यन्त॒ ते दे॒वेषु॑ सोमपी॒थमै᳚च्छन्त॒ तां दे॒वा अ॑ब्रुव॒न्द्वेद्वे॒ नाम॑नी कुरुध्व॒मथ॒ प्र वा॒फ्स्यथ॒ न वेत्य॒ग्नयो॒ वा अथ॒ धिष्णि॑या॒स्तस्मा᳚द्द्वि॒नामा᳚ ब्राह्म॒णो\-ऽर्धु॑क॒स्तेषां॒ ये नेदि॑ष्ठं प॒र्यवि॑श॒न्ते सो॑मपी॒थं प्राप्नु॑वन्नाहव॒नीय॑ आग्नी॒ध्रीयो॑ हो॒त्रीयो॑ मार्जा॒लीय॒स्तस्मा॒त्तेषु॑ जुह्वत्यति॒हाय॒ वष॑ट्करोति॒ वि हि~(३)

%6.3.1.4
ए॒ते सो॑मपी॒थेनार्ध्य॑न्त दे॒वा वै याः प्राची॒राहु॑ती॒रजु॑हवु॒र्ये पु॒रस्ता॒दसु॑रा॒ आस॒न्ताꣴस्ताभिः॒ प्राणु॑दन्त॒ याः प्र॒तीची॒र्ये प॒श्चादसु॑रा॒ आस॒न्ताꣴस्ताभि॒रपा॑नुदन्त॒ प्राची॑र॒न्या आहु॑तयो हू॒यन्ते᳚ प्र॒त्यङ्ङासी॑नो॒ धिष्णि॑या॒न्व्याघा॑रयति प॒श्चाच्चै॒व पु॒रस्ता᳚च्च॒ यज॑मानो॒ भ्रातृ॑व्या॒न्प्र णु॑दते॒ तस्मा॒त्परा॑चीः प्र॒जाः प्र वी॑यन्ते प्र॒तीचीः᳚~(४)

%6.3.1.5
जा॒य॒न्ते॒ प्रा॒णा वा ए॒ते यद्धिष्णि॑या॒ यद॑ध्व॒र्युः प्र॒त्यङ्धिष्णि॑यानति॒सर्पे᳚त्प्रा॒णान्थ्सं क॑र्\mbox{}षेत्प्र॒मायु॑कः स्या॒न्नाभि॒र्वा ए॒षा य॒ज्ञस्य॒ यद्धोतो॒र्ध्वः खलु॒ वै नाभ्यै᳚ प्रा॒णो\-ऽवा॑ङपा॒नो यद॑ध्व॒र्युः प्र॒त्यङ्होता॑रमति॒सर्पे॑दपा॒ने प्रा॒णं द॑ध्यात् प्र॒मायु॑कः स्या॒न्नाध्व॒र्युरुप॑ गाये॒द्वाग्वी᳚र्यो॒ वा अ॑ध्व॒र्युर्यद॑ध्व॒र्युरु॑प॒गाये॑दुद्गा॒त्रे~(५)

%6.3.1.6
वाच॒ꣳ॒ सम्प्र य॑च्छेदुप॒दासु॑कास्य॒ वाख्स्या᳚द्ब्रह्मवा॒दिनो॑ वदन्ति॒ नासꣴ॑स्थिते॒ सोमे᳚\-ऽध्व॒र्युः प्र॒त्यङ्ख्सदो\-ऽती॑या॒दथ॑ क॒था दा᳚क्षि॒णानि॒ होतु॑मेति॒ यामो॒ हि स तेषां॒ कस्मा॒ अह॑ दे॒वा यामं॒ वाया॑मं॒ वानु॑ ज्ञास्य॒न्तीत्युत्त॑रे॒णाग्नी᳚ध्रं प॒रीत्य॑ जुहोति दाक्षि॒णानि॒ न प्रा॒णान्थ्सं क॑र्\mbox{}षति॒ न्य॑न्ये धिष्णि॑या उ॒प्यन्ते॒ नान्ये यान्नि॒वप॑ति॒ तेन॒ तान्प्री॑णाति॒ यान्न नि॒वप॑ति॒ यद॑नुदि॒शति॒ तेन॒ तान्॥~(६)

{\anuvakamend[{स्तु॒ते वि॒न्दते॒ हि वी॑यन्ते प्र॒तीची॑रुद्गा॒त्र उ॒प्यन्ते॒ चतु॑र्दश च}]}%~(१)

%6.3.2.1
सु॒व॒र्गाय॒ वा ए॒तानि॑ लो॒काय॑ हूयन्ते॒ यद्वै॑सर्ज॒नानि॒ द्वाभ्यां॒ गार्\mbox{}ह॑पत्ये जुहोति द्वि॒पाद्यज॑मानः॒ प्रति॑ष्ठित्या॒ आग्नी᳚ध्रे जुहोत्य॒न्तरि॑क्ष ए॒वा क्र॑मत आहव॒नीये॑ जुहोति सुव॒र्गमे॒वैनं॑ लो॒कं ग॑मयति दे॒वान् वै सु॑व॒र्गं लो॒कं य॒तो रक्षाꣴ॑स्यजिघाꣳस॒न्ते सोमे॑न॒ राज्ञा॒ रक्षाꣴ॑स्यप॒हत्या॒प्तुमा॒त्मानं॑ कृ॒त्वा सु॑व॒र्गं लो॒कमा॑य॒न्रक्ष॑सा॒मनु॑पलाभा॒यात्तः॒ सोमो॑ भव॒त्यथ॑~(७)

%6.3.2.2
वै॒स॒र्ज॒नानि॑ जुहोति॒ रक्ष॑सा॒मप॑हत्यै॒ त्वꣳ सो॑म तनू॒कृद्भ्य॒ इत्या॑ह तनू॒कृद्ध्ये॑ष द्वेषो᳚भ्यो॒\-ऽन्यकृ॑तेभ्य॒ इत्या॑हा॒न्यकृ॑तानि॒ हि रक्षाꣴ॑स्यु॒रु य॒न्तासि॒ वरू॑थ॒मित्या॑हो॒रु ण॑स्कृ॒धीति॒ वावैतदा॑ह जुषा॒णो अ॒प्तुराज्य॑स्य वे॒त्वित्या॑हा॒प्तुमे॒व यज॑मानं कृ॒त्वा सु॑व॒र्गं लो॒कं ग॑मयति॒ रक्ष॑सा॒मनु॑पलाभा॒या सोमं॑ ददते~(८)

%6.3.2.3
आ ग्राव्ण्ण॒ आ वा॑य॒व्या᳚न्या द्रो॑णकल॒शमुत्पत्नी॒मा न॑य॒न्त्यन्वनाꣳ॑सि॒ प्र व॑र्तयन्ति॒ याव॑दे॒वास्यास्ति॒ तेन॑ स॒ह सु॑व॒र्गं लो॒कमे॑ति॒ नय॑वत्य॒र्चाग्नी᳚ध्रे जुहोति सुव॒र्गस्य॑ लो॒कस्या॒भिनी᳚त्यै॒ ग्राव्ण्णो॑ वाय॒व्या॑नि द्रोणकल॒शमाग्नी᳚ध्र॒ उप॑ वासयति॒ वि ह्ये॑नं॒ तैर्गृ॒ह्णते॒ यथ्स॒होप॑वा॒सये॑दपुवा॒येत॑ सौ॒म्यर्चा प्र पा॑दयति॒ स्वया᳚~(९)

%6.3.2.4
ए॒वैनं॑ दे॒वत॑या॒ प्र पा॑दय॒त्यदि॑त्याः॒ सदो॒\-ऽस्यदि॑त्याः॒ सद॒ आ सी॒देत्या॑ह यथाय॒जुरे॒वैतद्यज॑मानो॒ वा ए॒तस्य॑ पु॒रा गो॒प्ता भ॑वत्ये॒ष वो॑ देव सवितः॒ सोम॒ इत्या॑ह सवि॒तृप्र॑सूत ए॒वैनं॑ दे॒वता᳚भ्यः॒ सम्प्र य॑च्छत्ये॒तत्त्वꣳ सो॑म दे॒वो दे॒वानुपा॑गा॒ इत्या॑ह दे॒वो ह्ये॑ष सन्~(१०)

%6.3.2.5
दे॒वानु॒पैती॒दम॒हं म॑नु॒ष्यो॑ मनु॒ष्या॑नित्या॑ह मनु॒ष्यो  ह्ये॑ष सन्म॑नु॒ष्या॑नु॒पैति॒ यदे॒तद्यजु॒र्न ब्रू॒यादप्र॑जा अप॒शुर्यज॑मानः स्याथ्स॒ह प्र॒जया॑ स॒ह रा॒यस्पोषे॒णेत्या॑ह प्र॒जयै॒व प॒शुभिः॑ स॒हेमं लो॒कमु॒पाव॑र्तते॒ नमो॑ दे॒वेभ्य॒ इत्या॑ह नमस्का॒रो हि दे॒वानाꣴ॑ स्व॒धा पि॒तृभ्य॒ इत्या॑ह स्वधाका॒रो हि~(११)

%6.3.2.6
पि॒तृ॒णामि॒दम॒हं निर्वरु॑णस्य॒ पाशा॒दित्या॑ह वरुणपा॒शादे॒व निर्मु॑च्य॒ते\-ऽग्ने᳚ व्रतपत आ॒त्मनः॒ पूर्वा॑ त॒नूरा॒देयेत्या॑हुः॒ को हि तद्वेद॒ यद्वसी॑या॒न्थ्स्वे वशे॑ भू॒ते पुन॑र्वा॒ ददा॑ति॒ न वेति॒ ग्रावा॑णो॒ वै सोम॑स्य॒ राज्ञो॑ मलिम्लुसे॒ना य ए॒वं वि॒द्वान्ग्राव्ण्ण॒ आग्नी᳚ध्र उपवा॒सय॑ति॒ नैनं॑ मलिम्लुसे॒ना वि॑न्दति॥~(१२)

{\anuvakamend[{अथ॑ ददते॒ स्वया॒ सन्थ्स्व॑धाका॒रो हि वि॑न्दति}]}%~(२)

%6.3.3.1
वै॒ष्ण॒व्यर्चा हु॒त्वा यूप॒मच्छै॑ति वैष्ण॒वो वै दे॒वत॑या॒ यूपः॒ स्वयै॒वैनं॑ दे॒वत॒याच्छै॒त्यत्य॒न्यानगां॒ नान्यानुपा॑गा॒मित्या॒हाति॒ ह्य॑न्यानेति॒ नान्यानु॒पैत्य॒र्वाक्त्वा॒ परै॑रविदं प॒रो\-ऽव॑रै॒रित्या॑हा॒र्वाग्घ्ये॑नं॒ परै᳚र्वि॒न्दति॑ प॒रोव॑रै॒स्तं त्वा॑ जुषे~(१३)

%6.3.3.2
वै॒ष्ण॒वं दे॑वय॒ज्याया॒ इत्या॑ह देवय॒ज्यायै॒ ह्ये॑नं जु॒षते॑ दे॒वस्त्वा॑ सवि॒ता मध्वा॑न॒क्त्वित्या॑ह॒ तेज॑सै॒वैन॑मन॒क्त्योष॑धे॒ त्राय॑स्वैन॒ꣴ॒ स्वधि॑ते॒ मैनꣳ॑ हिꣳसी॒रित्या॑ह॒ वज्रो॒ वै स्वधि॑तिः॒ शान्त्यै॒ स्वधि॑तेर्वृ॒क्षस्य॒ बिभ्य॑तः प्रथ॒मेन॒ शक॑लेन स॒ह तेजः॒ परा॑ पतति॒ यः प्र॑थ॒मः शक॑लः परा॒पते॒त्तमप्या ह॑रे॒थ्सते॑जसम्~(१४)

%6.3.3.3
ए॒वैन॒मा ह॑रती॒मे वै लो॒का यूपा᳚त्प्रय॒तो बि॑भ्यति॒ दिव॒मग्रे॑ण॒ मा ले॑खीर॒न्तरि॑क्षं॒ मध्ये॑न॒ मा हिꣳ॑सी॒रित्या॑है॒भ्य ए॒वैनं॑ लो॒केभ्यः॑ शमयति॒ वन॑स्पते श॒तव॑ल्\mbox{}शो॒ वि रो॒हेत्या॒व्रश्च॑ने जुहोति॒ तस्मा॑दा॒व्रश्च॑नाद्वृ॒क्षाणां॒ भूयाꣳ॑स॒ उत्ति॑ष्ठन्ति स॒हस्र॑वल्\mbox{}शा॒ वि व॒यꣳ रु॑हे॒मेत्या॑हा॒\-ऽऽ\-शिष॑मे॒वैतामा शा॒स्ते\-ऽन॑क्षसङ्गम्~(१५)

%6.3.3.4
वृ॒श्चे॒द्यद॑क्षस॒ङ्गं वृ॒श्चेद॑धई॒षं यज॑मानस्य प्र॒मायु॑कꣴ स्या॒द्यं का॒मये॒ताप्र॑तिष्ठितः स्या॒दित्या॑रो॒हं तस्मै॑ वृश्चेदे॒ष वै वन॒स्पती॑ना॒मप्र॑तिष्ठि॒तो\-ऽप्र॑तिष्ठित ए॒व भ॑वति॒ यं का॒मये॑ताप॒शुः स्या॒दित्य॑प॒र्णं तस्मै॒ शुष्का᳚ग्रं वृश्चेदे॒ष वै वन॒स्पती॑नामपश॒व्यो॑\-ऽप॒शुरे॒व भ॑वति॒ यं का॒मये॑त पशु॒मान्थ्स्या॒दिति॑ बहुप॒र्णं तस्मै॑ बहुशा॒खं वृ॑श्चेदे॒ष वै~(१६)

%6.3.3.5
वन॒स्पती॑नां पश॒व्यः॑ पशु॒माने॒व भ॑वति॒ प्रति॑ष्ठितं वृश्चेत्प्रति॒ष्ठाका॑मस्यै॒ष वै वन॒स्पती॑नां॒ प्रति॑ष्ठितो॒ यः स॒मे भूम्यै॒ स्वाद्योने॑ रू॒ढः प्रत्ये॒व ति॑ष्ठति॒ यः प्र॒त्यङ्ङुप॑नत॒स्तं वृ॑श्चे॒थ्स हि मेध॑म॒भ्युप॑नतः॒ पञ्चा॑रत्निं॒ तस्मै॑ वृश्चे॒द्यं का॒मये॒तोपै॑न॒मुत्त॑रो य॒ज्ञो न॑मे॒दिति॒ पञ्चा᳚क्षरा प॒ङ्क्तिः पाङ्क्तो॑ य॒ज्ञ उपै॑न॒मुत्त॑रो य॒ज्ञः~(१७)

%6.3.3.6
न॒म॒ति॒ षड॑रत्निं प्रति॒ष्ठाका॑मस्य॒ षड्वा ऋ॒तव॑ ऋ॒तुष्वे॒व प्रति॑ तिष्ठति स॒प्तार॑त्निं प॒शुका॑मस्य स॒प्तप॑दा॒ शक्व॑री प॒शवः॒ शक्व॑री प॒शूने॒वाव॑ रुन्धे॒ नवा॑रत्निं॒ तेज॑स्कामस्य त्रि॒वृता॒ स्तोमे॑न॒ सम्मि॑तं॒ तेज॑स्त्रि॒वृत्ते॑ज॒स्व्ये॑व भ॑व॒त्येका॑\-दशारत्नि\-मिन्द्रि॒यका॑म॒स्यैका॑\-दशाक्षरा त्रि॒ष्टुगि॑न्द्रि॒यं त्रि॒ष्टुगि॑न्द्रिया॒व्ये॑व भ॑वति॒ पञ्च॑दशारत्नि॒म्भ्रातृ॑व्यवतः पञ्चद॒शो वज्रो॒ भ्रातृ॑व्याभिभूत्यै स॒प्तद॑शारत्निं प्र॒जाका॑मस्य सप्तद॒शः प्र॒जा\-प॑तिः प्र॒जा\-प॑ते॒राप्त्या॒ एक॑विꣳशत्यरत्निं प्रति॒ष्ठाका॑मस्यैक\-वि॒ꣳ॒शः स्तोमा॑नां प्रति॒ष्ठा प्रति॑ष्ठित्या अ॒ष्टाश्रि॑र्भवत्य॒ष्टाक्ष॑रा गाय॒त्री तेजो॑ गाय॒त्री गा॑य॒त्री य॑ज्ञमु॒खं तेज॑सै॒व गा॑यत्रि॒या य॑ज्ञमु॒खेन॒ सम्मि॑तः॥~(१८)

{\anuvakamend[{जु॒षे॒ सते॑जस॒मन॑क्षसङ्गं बहुशा॒खं वृ॑श्चेदे॒ष वै य॒ज्ञ उपै॑न॒मुत्त॑रो य॒ज्ञ आप्त्या॒ एका॒न्नविꣳ॑श॒तिश्च॑}]}%~(३)

%6.3.4.1
पृ॒थि॒व्यै त्वा॒ऽन्तरि॑क्षाय त्वा दि॒वे त्वेत्या॑है॒भ्य ए॒वैनं॑ लो॒केभ्यः॒ प्रोक्ष॑ति॒ परा᳚ञ्चं॒ प्रोक्ष॑ति॒ परा॑ङिव॒ हि सु॑व॒र्गो लो॒कः क्रू॒रमि॑व॒ वा ए॒तत्क॑रोति॒ यत्खन॑त्य॒पोव॑ नयति॒ शान्त्यै॒ यव॑मती॒रव॑ नय॒त्यूर्ग्वै यवो॒ यज॑मानेन॒ यूपः॒ सम्मि॑तो॒ यावा॑ने॒व यज॑मान॒स्ताव॑तीमे॒वास्मि॒न्नूर्जं॑ दधाति~(१९)

%6.3.4.2
पि॒तृ॒णाꣳ सद॑नम॒सीति॑ ब॒र्॒\mbox{}हिरव॑ स्तृणाति पितृदेव॒त्य ꣴ॒ ह्ये॑तद्यन्निखा॑तं॒ यद्ब॒र्॒\mbox{}हिरन॑वस्तीर्य मिनु॒यात्पि॑तृदेव॒त्यो॑ निखा॑तः स्याद्ब॒र्॒\mbox{}हिर॑व॒स्तीर्य॑ मिनोत्य॒स्यामे॒वैन॑म्मिनोति यूपशक॒लमवा᳚स्यति॒ सते॑जसमे॒वैन॑म्मिनोति दे॒वस्त्वा॑ सवि॒ता मध्वा॑न॒क्त्वित्या॑ह॒ तेज॑सै॒वैन॑मनक्ति सुपिप्प॒लाभ्य॒स्त्वौष॑धीभ्य॒ इति॑ च॒षालं॒ प्रति॑~(२०)

%6.3.4.3
मु॒ञ्च॒ति॒ तस्मा᳚च्छीर्\mbox{}ष॒त ओष॑धयः॒ फलं॑ गृह्णन्त्य॒नक्ति॒ तेजो॒ वा आज्यं॒ यज॑मानेनाग्नि॒ष्ठाश्रिः॒ सम्मि॑ता॒ यद॑ग्नि॒ष्ठा\-मश्रि॑म॒नक्ति॒ यज॑मानमे॒व तेज॑सानक्त्या॒न्तम॑नक्त्या॒न्तमे॒व यज॑मानं॒ तेज॑सानक्ति स॒र्वतः॒ परि॑ मृश॒त्यप॑रिवर्गमे॒वा\-स्मि॒न्तेजो॑ दधा॒त्युद्दिवꣴ॑ स्तभा॒नान्तरि॑क्षं पृ॒णेत्या॑है॒षां लो॒कानां॒ विधृ॑त्यै वैष्ण॒व्यर्चा~(२१)

%6.3.4.4
क॒ल्प॒य॒ति॒ वै॒ष्ण॒वो वै दे॒वत॑या॒ यूपः॒ स्वयै॒वैनं॑ दे॒वत॑या कल्पयति॒ द्वा\-भ्यां᳚ कल्पयति द्वि॒पाद्यज॑मानः॒ प्रति॑ष्ठित्यै॒ यं का॒मये॑त॒ तेज॑सैनं दे॒वता॑भिरिन्द्रि॒येण॒ व्य॑र्धयेय॒मित्य॑ग्नि॒ष्ठां तस्याश्रि॑माहव॒नीया॑दि॒त्थं वे॒त्थं वाति॑ नावये॒त्तेज॑सै॒वैनं॑ दे॒वता॑भिरिन्द्रि॒येण॒ व्य॑र्धयति॒ यं का॒मये॑त॒ तेज॑सैनं दे॒वता॑भिरिन्द्रि॒येण॒ सम॑र्धयेय॒मिति॑~(२२)

%6.3.4.5
अ॒ग्नि॒ष्ठां तस्याश्रि॑माहव॒नीये॑न॒ सम्मि॑नुया॒त्तेज॑सै॒वैनं॑ दे॒वता॑भिरिन्द्रि॒येण॒ सम॑र्धयति ब्रह्म॒वनिं॑ त्वा क्षत्र॒वनि॒मित्या॑ह यथाय॒जुरे॒वैतत्परि॑ व्यय॒त्यूर्ग्वै र॑श॒ना यज॑मानेन॒ यूपः॒ सम्मि॑तो॒ यज॑मानमे॒वोर्जा सम॑र्धयति नाभिद॒घ्ने परि॑ व्ययति नाभिद॒घ्न ए॒वास्मा॒ ऊर्जं॑ दधाति॒ तस्मा᳚न्नाभिद॒घ्न ऊ॒र्जा भु॑ञ्जते॒ यं का॒मये॑तो॒र्जैनम्᳚~(२३)

%6.3.4.6
व्य॑र्धयेय॒मित्यू॒र्ध्वां वा॒ तस्यावा॑चीं॒ वावो॑हेदू॒र्जैवैनं॒ व्य॑र्धयति॒ यदि॑ का॒मये॑त॒ वर्\mbox{}षु॑कः प॒र्जन्यः॑ स्या॒दित्यवा॑ची॒मवो॑हे॒\-द्वृष्टि॑मे॒व नि य॑च्छति॒ यदि॑ का॒मये॒ताव॑र्\mbox{}षुकः स्या॒दित्यू॒र्ध्वामुदू॑हे॒द्वृष्टि॑मे॒वोद्य॑च्छति पितृ॒णां निखा॑तं मनु॒ष्या॑णामू॒र्ध्वं निखा॑ता॒दा र॑श॒नाया॒ ओष॑धीनाꣳ रश॒ना विश्वे॑षाम्~(२४)

%6.3.4.7
दे॒वाना॑मू॒र्ध्वꣳ र॑श॒नाया॒ आ च॒षाला॒दिन्द्र॑स्य च॒षालꣳ॑ सा॒ध्याना॒मति॑रिक्त॒ꣳ॒ स वा ए॒ष स॑र्वदेव॒त्यो॑ यद्यूपो॒ यद्यूपं॑ मि॒नोति॒ सर्वा॑ ए॒व दे॒वताः᳚ प्रीणाति य॒ज्ञेन॒ वै दे॒वाः सु॑व॒र्गं लो॒कमा॑य॒न्ते॑\-ऽमन्यन्त मनु॒ष्या॑ नो॒\-ऽन्वाभ॑विष्य॒न्तीति॒ ते यूपे॑न योपयि॒त्वा सु॑व॒र्गं लो॒कमा॑य॒न्तमृष॑यो॒ यूपे॑नै॒वानु॒ प्राजा॑न॒न्तद्यूप॑स्य यूप॒त्वम्~(२५)

%6.3.4.8
यद्यूपं॑ मि॒नोति॑ सुव॒र्गस्य॑ लो॒कस्य॒ प्रज्ञा᳚त्यै पु॒रस्ता᳚न्मिनोति पु॒रस्ता॒द्धि य॒ज्ञस्य॑ प्रज्ञा॒यते\-ऽप्र॑ज्ञात॒ꣳ॒ हि तद्यदति॑पन्न आ॒हुरि॒दं का॒र्य॑मासी॒दिति॑ सा॒ध्या वै दे॒वा य॒ज्ञमत्य॑मन्यन्त॒ तान् य॒ज्ञो नास्पृ॑श॒त्तान् यद्य॒ज्ञस्याति॑रिक्त॒मासी॒त्तद॑स्पृश॒\-दति॑रिक्तं॒ वा ए॒तद्य॒ज्ञस्य॒ यद॒ग्नाव॒ग्निं म॑थि॒त्वा प्र॒हर॒त्यति॑रिक्तमे॒तत्~(२६)

%6.3.4.9
यूप॑स्य॒ यदू॒र्ध्वं च॒षाला॒त्तेषां॒ तद्भा॑ग॒धेयं॒ ताने॒व तेन॑ प्रीणाति दे॒वा वै सꣴस्थि॑ते॒ सोमे॒ प्र स्रुचोह॑र॒न्प्र यूपं॒ ते॑\-ऽमन्यन्त यज्ञवेश॒सं वा इ॒दं कु॑र्म॒ इति॒ ते प्र॑स्त॒रꣴ स्रु॒चान्नि॒ष्क्रय॑णमपश्य॒न्थ्स्वरुं॒ यूप॑स्य॒ सꣴस्थि॑ते॒ सोमे॒ प्र प्र॑स्त॒रꣳ हर॑ति जु॒होति॒ स्व॒रुमय॑ज्ञवेशसाय॥~(२७)

{\anuvakamend[{द॒धा॒ति॒ प्रत्यृ॒चा सम॑र्धयेय॒मित्यू॒र्जैनं॒ विश्वे॑षां यूप॒त्वमति॑रिक्तमे॒तद्द्विच॑त्वारिꣳशच्च}]}%~(४)

%6.3.5.1
सा॒ध्या वै दे॒वा अ॒स्मिँल्लो॒क आ॑स॒न्नान्यत्किञ्च॒न मि॒षत्ते᳚\-ऽग्निमे॒वाग्नये॒ मेधा॒याल॑भन्त॒ न ह्य॑न्यदा॑ल॒म्भ्य॑मवि॑न्द॒न्ततो॒ वा इ॒माः प्र॒जाः प्राजा॑यन्त॒ यद॒ग्नाव॒ग्निं म॑थि॒त्वा प्र॒हर॑ति प्र॒जानां᳚ प्र॒जन॑नाय रु॒द्रो वा ए॒ष यद॒ग्निर्यज॑मानः प॒शुर्यत्प॒शुमा॒लभ्या॒ग्निं मन्थे᳚द्रु॒द्राय॒ यज॑मानम्~(२८)

%6.3.5.2
अपि॑ दध्यात्प्र॒मायु॑कः स्या॒दथो॒ खल्वा॑हुर॒ग्निः सर्वा॑ दे॒वता॑ ह॒विरे॒तद्यत्प॒शुरिति॒ यत्प॒शुमा॒लभ्या॒ग्निं मन्थ॑ति ह॒व्यायै॒वास॑न्नाय॒ सर्वा॑ दे॒वता॑ जनयत्युपा॒कृत्यै॒व मन्थ्य॒स्तन्नेवाल॑ब्धं॒ नेवाना॑लब्धम॒ग्नेर्ज॒नित्र॑म॒सीत्या॑हा॒ग्नेर्\mbox{}ह्ये॑तज्ज॒नित्रं॒ वृष॑णौ स्थ॒ इत्या॑ह॒ वृष॑णौ~(२९)

%6.3.5.3
ह्ये॑तावु॒र्वश्य॑स्या॒युर॒सीत्या॑ह मिथुन॒त्वाय॑ घृ॒तेना॒क्ते वृष॑णं दधाथा॒मित्या॑ह॒ वृष॑ण॒ꣴ॒ ह्ये॑ते दधा॑ते॒ ये अ॒ग्निङ्गा॑य॒त्रं छन्दो\-ऽनु॒ प्र जा॑य॒स्वेत्या॑ह॒ छन्दो॑भिरे॒वैनं॒ प्र ज॑नयत्य॒ग्नये॑ म॒थ्यमा॑ना॒यानु॑ ब्रू॒हीत्या॑ह सावि॒त्रीमृच॒मन्वा॑ह सवि॒तृप्र॑सूत ए॒वैनं॑ मन्थति जा॒ताया॑नु ब्रूहि~(३०)

%6.3.5.4
प्र॒ह्रि॒यमा॑णा॒यानु॑ ब्रू॒हीत्या॑ह॒ काण्डे॑काण्ड ए॒वैनं॑ क्रियमा॑णे॒ सम॑र्धयति गाय॒त्रीः सर्वा॒ अन्वा॑ह गाय॒त्रछ॑न्दा॒ वा अ॒ग्निः स्वेनै॒वैनं॒ छन्द॑सा॒ सम॑र्धयत्य॒ग्निः पु॒रा भव॑त्य॒ग्निं म॑थि॒त्वा प्र ह॑रति॒ तौ स॒म्भव॑न्तौ॒ यज॑मानम॒भि सम्भ॑वतो॒ भव॑तं नः॒ सम॑नसा॒वित्या॑ह॒ शान्त्यै᳚ प्र॒हृत्य॑ जुहोति जा॒तायै॒वास्मा॒ अन्न॒मपि॑ दधा॒त्याज्ये॑न जुहोत्ये॒तद्वा अ॒ग्नेः प्रि॒यं धाम॒ यदाज्यं॑ प्रि॒येणै॒वैनं॒ धाम्ना॒ सम॑र्धय॒त्यथो॒ तेज॑सा॥~(३१)

{\anuvakamend[{यज॑मानमाह॒ वृष॑णौ जाता॒यानु॑ब्रू॒ह्यप्य॒ष्टाद॑श च}]}%~(५)

%6.3.6.1
इ॒षे त्वेति॑ ब॒र्॒\mbox{}हिरा द॑त्त इ॒च्छत॑ इव॒ ह्ये॑ष यो यज॑त उप॒वीर॒सीत्या॒होप॒ ह्ये॑नानाक॒रोत्युपो॑ दे॒वान्दैवी॒र्विशः॒ प्रागु॒रित्या॑ह॒ दैवी॒र्॒\mbox{}ह्ये॑ता विशः॑ स॒तीर्दे॒वानु॑प॒यन्ति॒ वह्नी॑रु॒शिज॒ इत्या॑ह॒र्त्विजो॒ वै वह्न॑य उ॒शिज॒स्तस्मा॑दे॒वमा॑ह॒ बृह॑स्पते धा॒रया॒ वसू॒नीति॑~(३२)

%6.3.6.2
आ॒ह॒ ब्रह्म॒ वै दे॒वानां॒ बृह॒स्पति॒र्ब्रह्म॑णै॒वास्मै॑ प॒शूनव॑ रुन्धे ह॒व्या ते᳚ स्वदन्ता॒मित्या॑ह स्व॒दय॑त्ये॒वैना॒न्देव॑ त्वष्ट॒र्वसु॑ र॒ण्वेत्या॑ह॒ त्वष्टा॒ वै प॑शू॒नां मि॑थु॒नानाꣳ॑ रूप॒कृद्रू॒पमे॒व प॒शुषु॑ दधाति॒ रेव॑ती॒ रम॑ध्व॒मित्या॑ह प॒शवो॒ वै रे॒वतीः᳚ प॒शूने॒वास्मै॑ रमयति दे॒वस्य॑ त्वा सवि॒तुः प्र॑स॒व इति॑~(३३)

%6.3.6.3
र॒श॒नामा द॑त्ते॒ प्रसू᳚त्या अ॒श्विनो᳚र्बा॒हुभ्या॒मित्या॑हा॒श्विनौ॒ हि दे॒वाना॑मध्व॒र्यू आस्तां᳚ पू॒ष्णो हस्ता᳚भ्या॒मित्या॑ह॒ यत्या॑ ऋ॒तस्य॑ त्वा देवहविः॒ पाशे॒ना र॑भ॒ इत्या॑ह स॒त्यं वा ऋ॒तꣳ स॒त्येनै॒वैन॑मृ॒तेना र॑भते\-ऽक्ष्ण॒या परि॑ हरति॒ वध्य॒ꣳ॒ हि प्र॒त्यञ्चं॑ प्रतिमु॒ञ्चन्ति॒ व्यावृ॑त्त्यै॒ धर्\mbox{}षा॒ मानु॑षा॒निति॒ नि यु॑नक्ति॒ धृत्या॑ अ॒द्भ्यः~(३४)

%6.3.6.4
त्वौष॑धीभ्यः॒ प्रोक्षा॒मीत्या॑हा॒द्भ्यो ह्ये॑ष ओष॑धीभ्यः स॒म्भव॑ति॒ यत्प॒शुर॒पां पे॒रुर॒सीत्या॑है॒ष ह्य॑पां पा॒ता यो मेधा॑यार॒भ्यते᳚ स्वा॒त्तं चि॒थ्सदे॑वꣳ ह॒व्यमापो॑ देवीः॒ स्वद॑तैन॒मित्या॑ह स्व॒दय॑त्ये॒वैन॑मु॒परि॑ष्टा॒त्प्रोक्ष॑त्यु॒परि॑ष्टादे॒वैनं॒ मेध्यं॑ करोति पा॒यय॑त्यन्तर॒त ए॒वैनं॒ मेध्यं॑ करोत्य॒धस्ता॒दुपो᳚क्षति स॒र्वत॑ ए॒वैनं॒ मेध्यं॑ करोति॥~(३५)

{\anuvakamend[{वसू॒निति॑ प्रस॒व इत्य॒द्भ्यो᳚\-ऽन्तर॒त ए॒वैन॒न्दश॑ च}]}%~(६)

%6.3.7.1
अ॒ग्निना॒ वै होत्रा॑ दे॒वा असु॑रान॒भ्य॑भवन्न॒ग्नये॑ समि॒ध्यमा॑ना॒यानु॑ ब्रू॒हीत्या॑ह॒ भ्रातृ॑व्याभिभूत्यै स॒प्तद॑श सामिधे॒नीरन्वा॑ह सप्तद॒शः प्र॒जा\-प॑तिः प्र॒जा\-प॑ते॒राप्त्यै॑ स॒प्तद॒शान्वा॑ह॒ द्वाद॑श॒ मासाः॒ पञ्च॒र्तवः॒ स सं॑वथ्स॒रः सं॑वथ्स॒रं प्र॒जा अनु॒ प्र जा॑यन्ते प्र॒जानां᳚ प्र॒जन॑नाय दे॒वा वै सा॑मिधे॒नीर॒नूच्य॑ य॒ज्ञं नान्व॑पश्य॒न्थ्स प्र॒जा\-प॑तिस्तू॒ष्णीमा॑घा॒रम्~(३६)

%6.3.7.2
आघा॑रय॒त्ततो॒ वै दे॒वा य॒ज्ञमन्व॑पश्य॒न्॒ यत्तू॒ष्णीमा॑घा॒रमा॑घा॒रय॑ति य॒ज्ञस्यानु॑ख्यात्या॒ असु॑रेषु॒ वै य॒ज्ञ आ॑सी॒त्तं दे॒वास्तू᳚ष्णीꣳहो॒मेना॑वृञ्जत॒ यत्तू॒ष्णीमा॑घा॒रमा॑घा॒रय॑ति॒ भ्रातृ॑व्यस्यै॒व तद्य॒ज्ञं वृ॑ङ्क्ते परि॒धीन्थ्सम्मा᳚र्ष्टि पु॒नात्ये॒वैना॒न्त्रिस्त्रिः॒ सम्मा᳚र्ष्टि॒ त्र्या॑वृ॒द्धि य॒ज्ञो\-ऽथो॒ रक्ष॑सा॒मप॑हत्यै॒ द्वाद॑श॒ सं प॑द्यन्ते॒ द्वाद॑श~(३७)

%6.3.7.3
मासाः᳚ संवथ्स॒रः सं॑वथ्स॒रमे॒व प्री॑णा॒त्यथो॑ संवथ्स॒रमे॒वास्मा॒ उप॑ दधाति सुव॒र्गस्य॑ लो॒कस्य॒ सम॑ष्ट्यै॒ शिरो॒ वा ए॒तद्य॒ज्ञस्य॒ यदा॑घा॒रो᳚\-ऽग्निः सर्वा॑ दे॒वता॒ यदा॑घा॒रमा॑घा॒रय॑ति शीर्\mbox{}ष॒त ए॒व य॒ज्ञस्य॒ यज॑मानः॒ सर्वा॑ दे॒वता॒ अव॑ रुन्धे॒ शिरो॒ वा ए॒तद्य॒ज्ञस्य॒ यदा॑घा॒र आ॒त्मा प॒शुरा॑घा॒रमा॒घार्य॑ प॒शुꣳ सम॑नक्त्या॒त्मन्ने॒व य॒ज्ञस्य॑~(३८)

%6.3.7.4
शिरः॒ प्रति॑ दधाति॒ सं ते᳚ प्रा॒णो वा॒युना॑ गच्छता॒मित्या॑ह वायुदेव॒त्यो॑ वै प्रा॒णो वा॒यावे॒वास्य॑ प्रा॒णं जु॑होति॒ सं यज॑त्रै॒रङ्गा॑नि॒ सं य॒ज्ञप॑तिरा॒शिषेत्या॑ह य॒ज्ञप॑तिमे॒वास्या॒\-ऽऽ\-शिषं॑ गमयति वि॒श्वरू॑पो॒ वै त्वा॒ष्ट्र उ॒परि॑ष्टात्प॒शुम॒भ्य॑वमी॒त्तस्मा॑दु॒परि॑ष्टात्प॒शोर्नाव॑ द्यन्ति॒ यदु॒परि॑ष्टात्प॒शुꣳ स॑म॒नक्ति॒ मेध्य॑मे॒व~(३९)

%6.3.7.5
ए॒नं॒ क॒रो॒त्यृ॒त्विजो॑ वृणीते॒ छन्दाꣴ॑स्ये॒व वृ॑णीते स॒प्त वृ॑णीते स॒प्त ग्रा॒म्याः प॒शवः॑ स॒प्तार॒ण्याः स॒प्त छन्दाꣴ॑स्यु॒भय॒स्याव॑रुद्ध्या॒ एका॑\-दश प्रया॒जान् य॑जति॒ दश॒ वै प॒शोः प्रा॒णा आ॒त्मैका॑द॒शो यावा॑ने॒व प॒शुस्तं प्र य॑जति व॒पामेकः॒ परि॑ शय आ॒त्मैवात्मानं॒ परि॑ शये॒ वज्रो॒ वै स्वधि॑ति॒र्वज्रो॑ यूपशक॒लो घृ॒तं खलु॒ वै दे॒वा वज्रं॑ कृ॒त्वा सोम॑मघ्नन्घृ॒तेना॒क्तौ प॒शं त्रा॑येथा॒मित्या॑ह॒ वज्रे॑णै॒वैनं॒ वशे॑ कृ॒त्वा ल॑भते॥~(४०)

{\anuvakamend[{आ॒घा॒रं प॑द्यन्ते॒ द्वाद॑शा॒त्मन्ने॒व य॒ज्ञस्य॒ मेध्य॑मे॒व खलु॒ वा अ॒ष्टाद॑श च}]}%~(७)

%6.3.8.1
पर्य॑ग्नि करोति सर्व॒हुत॑मे॒वैनं॑ करो॒त्यस्क॑न्दा॒यास्क॑न्न॒ꣳ॒ हि तद्यद्धु॒तस्य॒ स्कन्द॑ति॒ त्रिः पर्य॑ग्नि करोति॒ त्र्या॑वृ॒द्धि य॒ज्ञो\-ऽथो॒ रक्ष॑सा॒मप॑हत्यै ब्रह्मवा॒दिनो॑ वदन्त्यन्वा॒रभ्यः॑ प॒शू~(३) र्नान्वा॒रभ्या~(३) इति॑ मृ॒त्यवे॒ वा ए॒ष नी॑यते॒ यत्प॒शुस्तं यद॑न्वा॒रभे॑त प्र॒मायु॑को॒ यज॑मानः स्या॒दथो॒ खल्वा॑हुः सुव॒र्गाय॒ वा ए॒ष लो॒काय॑ नीयते॒ यत्~(४१)

%6.3.8.2
प॒शुरिति॒ यन्नान्वा॒रभे॑त सुव॒र्गाल्लो॒काद्यज॑मानो हीयेत वपा॒श्रप॑णीभ्याम॒न्वार॑भते॒ तन्नेवा॒न्वार॑ब्धं॒ नेवान॑न्वारब्ध॒मुप॒ प्रेष्य॑ होतर्\mbox{}ह॒व्या दे॒वेभ्य॒ इत्या॑हेषि॒तꣳ हि कर्म॑ क्रि॒यते॒ रेव॑तीर्य॒ज्ञप॑तिं प्रिय॒धा वि॑श॒तेत्या॑ह यथाय॒जुरे॒वैतद॒ग्निना॑ पु॒रस्ता॑देति॒ रक्ष॑सा॒मप॑हत्यै पृथि॒व्याः स॒म्पृचः॑ पा॒हीति॑ ब॒र्॒\mbox{}हिः~(४२)

%6.3.8.3
उपा᳚स्य॒त्यस्क॑न्दा॒यास्क॑न्न॒ꣳ॒ हि तद्यद्ब॒र्॒\mbox{}हिषि॒ स्कन्द॒त्यथो॑ बर्\mbox{}हि॒षद॑मे॒वैनं॑ करोति॒ परा॒ङा व॑र्तते\-ऽध्व॒र्युः प॒शोः सं᳚ज्ञ॒प्यमा॑नात्प॒शुभ्य॑ ए॒व तन्नि ह्नु॑त आ॒त्मनोना᳚व्रस्काय॒ गच्छ॑ति॒ श्रियं प्र प॒शूना᳚प्नोति॒ य ए॒वं वेद॑ प॒श्चाल्लो॑का॒ वा ए॒षा प्राच्यु॒दानी॑यते॒ यत्पत्नी॒ नम॑स्त आता॒नेत्या॑हादि॒त्यस्य॒ वै र॒श्मयः॑~(४३)

%6.3.8.4
आ॒ता॒नास्तेभ्य॑ ए॒व नम॑स्करोत्यन॒र्वा प्रेहीत्या॑ह॒ भ्रातृ॑व्यो॒ वा अर्वा॒ भ्रातृ॑व्यापनुत्त्यै घृ॒तस्य॑ कु॒ल्यामनु॑ स॒ह प्र॒जया॑ स॒ह रा॒यस्पोषे॒णेत्या॑हा॒\-ऽऽ\-शिष॑मे॒वैतामा शा᳚स्त॒ आपो॑ देवीः शुद्धायुव॒ इत्या॑ह यथाय॒जुरे॒वैतत्॥~(४४)

{\anuvakamend[{लो॒काय॑ नीयते॒ यद्ब॒र्॒\mbox{}ही र॒श्मयः॑ स॒प्तत्रिꣳ॑शच्च}]}%~(८)

%6.3.9.1
प॒शोर्वा आल॑ब्धस्य प्रा॒णाञ्छुगृ॑च्छति॒ वाक्त॒ आ प्या॑यतां प्रा॒णस्त॒ आ प्या॑यता॒मित्या॑ह प्रा॒णेभ्य॑ ए॒वास्य॒ शुचꣳ॑ शमयति॒ सा प्रा॒णेभ्यो\-ऽधि॑ पृथि॒वीꣳ शुक्प्र वि॑शति॒ शमहो᳚भ्या॒मिति॒ नि न॑यत्यहोरा॒त्राभ्या॑मे॒व पृ॑थि॒व्यै शुचꣳ॑ शमय॒त्योष॑धे॒ त्राय॑स्वैन॒ꣴ॒ स्वधि॑ते॒ मैनꣳ॑ हिꣳसी॒रित्या॑ह॒ वज्रो॒ वै स्वधि॑तिः~(४५)

%6.3.9.2
शान्त्यै॑ पार्श्व॒त आच्छ्य॑ति मध्य॒तो हि म॑नु॒ष्या॑ आ॒च्छ्यन्ति॑ तिर॒श्चीन॒मा च्छ्य॑त्यनू॒चीन॒ꣳ॒ हि म॑नु॒ष्या॑ आ॒च्छ्यन्ति॒ व्यावृ॑त्त्यै॒ रक्ष॑सां भा॒गो॑\-ऽसीति॑ स्थविम॒तो ब॒र्॒\mbox{}हिर॒क्त्वापा᳚स्यत्य॒स्नैव रक्षाꣳ॑सि नि॒रव॑दयत इ॒दम॒हꣳ रक्षो॑\-ऽध॒मं तमो॑ नयामि॒ यो᳚\-ऽस्मान्द्वेष्टि॒ यं च॑ व॒यं द्वि॒ष्म इत्या॑ह॒ द्वौ वाव पुरु॑षौ॒ यं चै॒व~(४६)

%6.3.9.3
द्वे॒ष्टि॒ यश्चै॑नं॒ द्वेष्टि॒ तावु॒भाव॑ध॒मं तमो॑ नयती॒षे त्वेति॑ व॒पामुत्खि॑दती॒च्छत॑ इव॒ ह्ये॑ष यो यज॑ते॒ यदु॑पतृ॒न्द्याद्रु॒द्रो᳚\-ऽस्य प॒शून्घातु॑कः स्या॒द्यन्नोप॑तृ॒न्द्यादय॑ता स्याद॒न्ययो॑पतृ॒णत्त्य॒न्यया॒ न धृत्यै॑ घृ॒तेन॑ द्यावा\-पृथिवी॒ प्रोर्ण्वा॑था॒मित्या॑ह॒ द्यावा॑पृथि॒वी ए॒व रसे॑नान॒क्त्यछि॑न्नः~(४७)

%6.3.9.4
रायः॑ सु॒वीर॒ इत्या॑ह यथाय॒जुरे॒वैतत्क्रू॒रमि॑व॒ वा ए॒तत्क॑रोति॒ यद्व॒पामु॑त्खि॒दत्यु॒र्व॑न्तरि॑क्ष॒मन्वि॒हीत्या॑ह॒ शान्त्यै॒ प्र वा ए॒षो᳚\-ऽस्माल्लो॒काच्च्य॑वते॒ यः प॒शुं मृ॒त्यवे॑ नी॒यमा॑नमन्वा॒रभ॑ते वपा॒श्रप॑णी॒ पुन॑र॒न्वार॑भते॒\-ऽस्मिन्ने॒व लो॒के प्रति॑ तिष्ठत्य॒ग्निना॑ पु॒रस्ता॑देति॒ रक्ष॑सा॒मप॑हत्या॒ अथो॑ दे॒वता॑ ए॒व ह॒व्येन॑~(४८)

%6.3.9.5
अन्वे॑ति॒ नान्त॒ममङ्गा॑र॒मति॑ हरे॒द्यद॑न्त॒ममङ्गा॑रमति॒हरे᳚द्दे॒वता॒ अति॑ मन्येत॒ वायो॒ वीहि॑ स्तो॒काना॒मित्या॑ह॒ तस्मा॒द्विभ॑क्ताः स्तो॒का अव॑ पद्य॒न्ते\-ऽग्रं॒ वा ए॒तत्प॑शू॒नां यद्व॒पाग्र॒मोष॑धीनां ब॒र्॒\mbox{}हिरग्रे॑णै॒वाग्र॒ꣳ॒ सम॑र्धय॒त्यथो॒ ओष॑धीष्वे॒व प॒शून्प्रति॑\-ष्ठापयति॒ स्वाहा॑कृतीभ्यः॒ प्रेष्येत्या॑ह~(४९)

%6.3.9.6
य॒ज्ञस्य॒ समि॑ष्ट्यै प्राणापा॒नौ वा ए॒तौ प॑शू॒नां यत्पृ॑षदा॒ज्यमा॒त्मा व॒पा पृ॑षदा॒ज्यम॑भि॒घार्य॑ व॒पाम॒भि घा॑रयत्या॒त्मन्ने॒व प॑शू॒नां प्रा॑णापा॒नौ द॑धाति॒ स्वाहो॒र्ध्वन॑भसम्मारु॒तं ग॑च्छत॒मित्या॑हो॒र्ध्वन॑भा ह स्म॒ वै मा॑रु॒तो दे॒वानां᳚ वपा॒श्रप॑णी॒ प्रह॑रति॒ तेनै॒वैने॒ प्र ह॑रति॒ विषू॑ची॒ प्र ह॑रति॒ तस्मा॒द्विष्व॑ञ्चौ प्राणापा॒नौ॥~(५०)

{\anuvakamend[{स्वधि॑तिश्चै॒वाच्छि॑न्नो ह॒व्येने॒ष्येत्या॑ह॒ षट्च॑त्वारिꣳशच्च}]}%~(९)

%6.3.10.1
प॒शुमा॒लभ्य॑ पुरो॒डाशं॒ निर्व॑पति॒ समे॑धमे॒वैन॒मा ल॑भते व॒पया᳚ प्र॒चर्य॑ पुरो॒डाशे॑न॒ प्र च॑र॒त्यूर्ग्वै पु॑रो॒डाश॒ ऊर्ज॑मे॒व प॑शू॒नां म॑ध्य॒तो द॑धा॒त्यथो॑ प॒शोरे॒व छि॒द्रमपि॑ दधाति पृषदा॒ज्यस्यो॑प॒हत्य॒ त्रिः पृ॑च्छति शृ॒तꣳ ह॒वीः~(३) श॑मित॒रिति॒ त्रिष॑त्या॒ हि दे॒वा यो\-ऽशृ॑तꣳ शृ॒तमाह॒ स एन॑सा प्राणापा॒नौ वा ए॒तौ प॑शू॒नाम्~(५१)

%6.3.10.2
यत्पृ॑षदा॒ज्यं प॒शोः खलु॒ वा आल॑ब्धस्य॒ हृद॑यमा॒त्माभि समे॑ति॒ यत्पृ॑षदा॒ज्येन॒ हृद॑यमभिघा॒रय॑त्या॒त्मन्ने॒व प॑शू॒नां प्रा॑णापा॒नौ द॑धाति प॒शुना॒ वै दे॒वाः सु॑व॒र्गं लो॒कमा॑य॒न्ते॑\-ऽमन्यन्त मनु॒ष्या॑ नो॒\-ऽन्वाभ॑विष्य॒न्तीति॒ तस्य॒ शिर॑श्छि॒त्त्वा मेधं॒ प्राक्षा॑रय॒न्थ्स प्र॒क्षो॑\-ऽभव॒त्तत्प्र॒क्षस्य॑ प्रक्ष॒त्वं यत्प्ल॑क्षशा॒खोत्त॑रब॒र्॒\mbox{}हिर्भव॑ति॒ समे॑धस्यै॒व~(५२)

%6.3.10.3
प॒शोरव॑ द्यति प॒शुं वै ह्रि॒यमा॑ण॒ꣳ॒ रक्षा॒ꣴ॒स्यनु॑ सचन्ते\-ऽन्त॒रा यूपं॑ चाहव॒नीयं॑ च हरति॒ रक्ष॑सा॒मप॑हत्यै प॒शोर्वा आल॑ब्धस्य॒ मनो\-ऽप॑ क्रामति म॒नोता॑यै ह॒विषो॑\-ऽवदी॒यमा॑न॒स्यानु॑ ब्रू॒हीत्या॑ह॒ मन॑ ए॒वास्याव॑ रुन्ध॒ एका॑\-दशाव॒दाना॒न्यव॑ द्यति॒ दश॒ वै प॒शोः प्रा॒णा आ॒त्मैका॑द॒शो यावा॑ने॒व प॒शुस्तस्याव॑~(५३)

%6.3.10.4
द्य॒ति॒ हृद॑य॒स्याग्रे\-ऽव॑ द्य॒त्यथ॑ जि॒ह्वाया॒ अथ॒ वक्ष॑सो॒ यद्वै हृद॑येनाभि॒गच्छ॑ति॒ तज्जि॒ह्वया॑ वदति॒ यज्जि॒ह्वया॒ वद॑ति॒ तदुर॒सो\-ऽधि॒ निर्व॑दत्ये॒तद्वै प॒शोर्य॑थापू॒र्वं यस्यै॒वम॑व॒दाय॑ यथा॒काम॒मुत्त॑रेषामव॒द्यति॑ यथापू॒र्वमे॒वास्य॑ प॒शोरव॑त्तं भवति मध्य॒तो गु॒दस्याव॑ द्यति मध्य॒तो हि प्रा॒ण उ॑त्त॒मस्याव॑ द्यति~(५४)

%6.3.10.5
उ॒त्त॒मो हि प्रा॒णो यदीत॑रं॒ यदीत॑रमु॒भय॑मे॒वाजा॑मि॒ जाय॑मानो॒ वै ब्रा᳚ह्म॒णस्त्रि॒भिर्\mbox{}ऋ॑ण॒वा जा॑यते ब्रह्म॒चर्ये॒णर्\mbox{}षि॑भ्यो य॒ज्ञेन॑ दे॒वेभ्यः॑ प्र॒जया॑ पि॒तृभ्य॑ ए॒ष वा अ॑नृ॒णो यः पु॒त्री यज्वा᳚ ब्रह्मचारिवा॒सी तद॑व॒दानै॑रे॒वाव॑ दयते॒ तद॑व॒दाना॑नामवदान॒त्वन्दे॑वासु॒राः संय॑त्ता आस॒न्ते दे॒वा अ॒ग्निम॑ब्रुव॒न्त्वया॑ वी॒रेणासु॑रान॒भि भ॑वा॒मेति॑~(५)

%6.3.10.6
सो᳚\-ऽब्रवी॒द्वरं॑ वृणै प॒शोरु॑द्धा॒रमुद्ध॑रा॒ इति॒ स ए॒तमु॑द्धा॒रमुद॑हरत॒ दोः पू᳚र्वा॒र्धस्य॑ गु॒दं म॑ध्य॒तः श्रोणिं॑ जघना॒र्धस्य॒ ततो॑ दे॒वा अभ॑व॒न्परासु॑रा॒ यत्त्र्य॒ङ्गाणाꣳ॑ समव॒द्यति॒ भ्रातृ॑व्या॒भिभूत्यै॒ भव॑त्या॒त्मना॒ परा᳚स्य॒ भ्रातृ॑व्यो भवत्यक्ष्ण॒याव॑ द्यति॒ तस्मा॑दक्ष्ण॒या प॒शवो\-ऽङ्गा॑नि॒ प्र ह॑रन्ति॒ प्रति॑ष्ठित्यै॥~(५६)

{\anuvakamend[{ए॒तौ प॑शू॒नाꣳ समे॑धस्यै॒व तस्यावो᳚त्त॒मस्याव॑ द्य॒तीति॒ पञ्च॑चत्वारिꣳशच्च}]}%॥10॥

%6.3.11.1
मेद॑सा॒ स्रुचौ॒ प्रोर्णो॑ति॒ मेदो॑रूपा॒ वै प॒शवो॑ रू॒पमे॒व प॒शुषु॑ दधाति यू॒षन्न॑व॒धाय॒ प्रोर्णो॑ति॒ रसो॒ वा ए॒ष प॑शू॒नां यद्यू रस॑मे॒व प॒शुषु॑ दधाति पा॒र्श्वेन॑ वसाहो॒मं प्र यौ॑ति॒ मध्यं॒ वा ए॒तत्प॑शू॒नां यत्पा॒र्श्वꣳ रस॑ ए॒ष प॑शू॒नां यद्वसा॒ यत्पा॒र्श्वेन॑ वसाहो॒मं प्र॒यौति॑ मध्य॒त ए॒व प॑शू॒नाꣳ रसं॑ दधाति॒ घ्नन्ति॑~(५७)

%6.3.11.2
वा ए॒तत्प॒शुं यथ्सं᳚ज्ञ॒पय॑न्त्यै॒न्द्रः खलु॒ वै दे॒वत॑या प्रा॒ण ऐ॒न्द्रो॑\-ऽपा॒न ऐ॒न्द्रः प्रा॒णो अङ्गे॑अङ्गे॒ नि दे᳚ध्य॒दित्या॑ह प्राणापा॒नावे॒व प॒शुषु॑ दधाति॒ देव॑ त्वष्ट॒र्भूरि॑ ते॒ सꣳस॑मे॒त्वित्या॑ह त्वा॒ष्ट्रा हि दे॒वत॑या प॒शवो॒ विषु॑रूपा॒ यथ्सल॑क्ष्माणो॒ भव॒थेत्या॑ह॒ विषु॑रूपा॒ ह्ये॑ते सन्तः॒ सल॑क्ष्माण ए॒तर्\mbox{}हि॒ भव॑न्ति देव॒त्रा यन्तम्᳚~(५८)

%6.3.11.3
अव॑से॒ सखा॒यो\-ऽनु॑ त्वा मा॒ता पि॒तरो॑ मद॒न्त्वित्या॒हानु॑मतमे॒वैनं॑ मा॒त्रा पि॒त्रा सु॑व॒र्गं लो॒कं ग॑मयत्यर्ध॒र्चे व॑साहो॒मं जु॑होत्य॒सौ वा अ॑र्ध॒र्च इ॒यम॑र्ध॒र्च इ॒मे ए॒व रसे॑नानक्ति॒ दिशो॑ जुहोति॒ दिश॑ ए॒व रसे॑नान॒क्त्यथो॑ दि॒ग्भ्य ए॒वोर्ज॒ꣳ॒ रस॒मव॑ रुन्धे प्राणापा॒नौ वा ए॒तौ प॑शू॒नां यत्पृ॑षदा॒ज्यं वा॑नस्प॒त्याः खलु॑~(५९)

%6.3.11.4
वै दे॒वत॑या प॒शवो॒ यत्पृ॑षदा॒ज्यस्यो॑प॒हत्याह॒ वन॒स्पत॒ये\-ऽनु॑ ब्रूहि॒ वन॒स्पत॑ये॒ प्रेष्येति॑ प्राणापा॒नावे॒व प॒शुषु॑ दधात्य॒न्यस्या᳚न्यस्य समव॒त्तꣳ स॒मव॑द्यति॒ तस्मा॒न्नाना॑रूपाः प॒शवो॑ यू॒ष्णोप॑ सिञ्चति॒ रसो॒ वा ए॒ष प॑शू॒नां यद्यू रस॑मे॒व प॒शुषु॑ दधा॒तीडा॒मुप॑ ह्वयते प॒शवो॒ वा इडा॑ प॒शूने॒वोप॑ ह्वयते च॒तुरुप॑ ह्वयते~(६०)

%6.3.11.5
चतु॑ष्पादो॒ हि प॒शवो॒ यं का॒मये॑ताप॒शुः स्या॒दित्य॑मे॒दस्कं॒ तस्मा॒ आ द॑ध्या॒न्मेदो॑रूपा॒ वै प॒शवो॑ रू॒पेणै॒वैनं॑ प॒शुभ्यो॒ निर्भ॑जत्यप॒शुरे॒व भ॑वति॒ यं का॒मये॑त पशु॒मान्थ्स्या॒दिति॒ मेद॑स्व॒त्तस्मा॒ आ द॑ध्या॒न्मेदो॑रूपा॒ वै प॒शवो॑ रू॒पेणै॒वास्मै॑ प॒शूनव॑ रुन्धे पशु॒माने॒व भ॑वति प्र॒जा\-प॑तिर्य॒ज्ञम॑सृजत॒ स आज्यम्᳚~(६१)

%6.3.11.6
पु॒रस्ता॑दसृजत प॒शुं म॑ध्य॒तः पृ॑षदा॒ज्यं प॒श्चात्तस्मा॒दाज्ये॑न प्रया॒जा इ॑ज्यन्ते प॒शुना॑ मध्य॒तः पृ॑षदा॒ज्येना॑नूया॒जास्तस्मा॑दे॒तन्मि॒श्रमि॑व पश्चाथ्सृ॒ष्टꣴ ह्येका॑\-दशानूया॒जान् य॑जति॒ दश॒ वै प॒शोः प्रा॒णा आ॒त्मैका॑द॒शो यावा॑ने॒व प॒शुस्तमनु॑ यजति॒ घ्नन्ति॒ वा ए॒तत्प॒शुं यथ्सं᳚ज्ञ॒पय॑न्ति प्राणापा॒नौ खलु॒ वा ए॒तौ प॑शू॒नां यत्पृ॑षदा॒ज्यं यत्पृ॑षदा॒ज्येना॑नूया॒जान् यज॑ति प्राणापा॒नावे॒व प॒शुषु॑ दधाति॥~(६२)

{\anuvakamend[{घ्नन्ति॒ यन्तं॒ खलु॑ च॒तुरुप॑ ह्वयत॒ आज्यं॒ यत्पृ॑षदा॒ज्येन॒ षट्च॑}]}%॥11॥

\prashnaend{चात्वा॑लाथ्सुव॒र्गाय॒ यद्वै॑सर्ज॒नानि॑ वैष्ण॒व्यर्चा पृ॑थि॒व्यै सा॒ध्या इ॒षे त्वेत्य॒ग्निना॒ पर्य॑ग्नि प॒शोः प॒शुमा॒लभ्य॒ मेद॑सा॒ स्रुचा॒वेका॑\-दश॥११॥}{चात्वा॑लाद्दे॒वानु॒पैति॑ मुञ्चति प्रह्रि॒यमा॑णाय॒ पर्य॑ग्नि प॒शुमा॒लभ्य॒ चतु॑ष्पादो॒ द्विष॑ष्टिः॥६२॥}{चात्वा॑लात्प॒शुषु॑ दधाति॥}%%६-३
{हरिः॑ ॐ}{॥कृष्ण-यजुर्वेदीय-तैत्तिरीय-संहितायां षष्ठकाण्डे तृतीयः प्रश्नः समाप्तः॥६-३॥}
%%% END PRASHNA
