\chapt{काण्डम् ३}
\sect{द्वितीयः प्रश्नः}\setcounter{anuvakam}{0}
\dnsub{तैत्तिरीयसंहितायां तृतीयकाण्डे द्वितीयः प्रश्नः}
%3.2.1.1
यो वै पव॑मानानामन्वारो॒हान् वि॒द्वान् यज॒ते\-ऽनु॒ पव॑माना॒ना रो॑हति॒ न पव॑माने॒भ्यो\-ऽव॑च्छिद्यते श्ये॒नो॑\-ऽसि गाय॒त्रछ॑न्दा॒ अनु॒ त्वा र॑भे स्व॒स्ति मा॒ सं पा॑रय सुप॒र्णो॑\-ऽसि त्रि॒ष्टुप्छ॑न्दा॒ अनु॒ त्वा र॑भे स्व॒स्ति मा॒ सं पा॑रय॒ सघा॑सि॒ जग॑तीछन्दा॒ अनु॒ त्वा र॑भे स्व॒स्ति मा॒ सं पा॑र॒येत्या॑है॒ते~(१)

%3.2.1.2
वै पव॑मानानामन्वारो॒हास्तान् य ए॒वं वि॒द्वान् यज॒ते\-ऽनु॒ पव॑माना॒ना रो॑हति॒ न पव॑माने॒भ्यो\-ऽव॑च्छिद्यते॒ यो वै पव॑मानस्य॒ सन्त॑तिं॒ वेद॒ सर्व॒मायु॑रेति॒ न पु॒रा\-ऽऽ\-यु॑षः॒ प्र मी॑यते पशु॒मान्भ॑वति वि॒न्दते᳚ प्र॒जां पव॑मानस्य॒ ग्रहा॑ गृह्य॒न्ते\-ऽथ॒ वा अ॑स्यै॒ते\-ऽगृ॑हीता द्रोणकल॒श आ॑धव॒नीयः॑ पूत॒भृत्तान् यदगृ॑हीत्वोपाकु॒र्यात्पव॑मानं॒ वि~(२)

%3.2.1.3
च्छि॑न्द्या॒त् तं वि॒च्छिद्य॑मानमध्व॒र्योः प्रा॒णो\-ऽनु॒ विच्छि॑द्येतोप\-या॒मगृ॑हीतो\-ऽसि प्र॒जा\-प॑तये॒ त्वेति॑ द्रोणकल॒शम॒भि मृ॑शे॒दिन्द्रा॑य॒ त्वेत्या॑धव॒नीयं॒ विश्वे᳚भ्यस्त्वा दे॒वेभ्य॒ इति॑ पूत॒भृतं॒ पव॑मानमे॒व तथ्सं त॑नोति॒ सर्व॒मायु॑रेति॒ न पु॒रा\-ऽऽ\-यु॑षः॒ प्र मी॑यते पशु॒मान्भ॑वति वि॒न्दते᳚ प्र॒जाम्॥~(३)

{\anuvakamend[{ए॒ते वि द्विच॑त्वारिꣳशच्च}]}%~(१)

%3.2.2.1
त्रीणि॒ वाव सव॑ना॒न्यथ॑ तृ॒तीय॒ꣳ॒ सव॑न॒मव॑ लुम्पन्त्यन॒ꣳ॒शु कु॒र्वन्त॑ उपा॒ꣳ॒शुꣳ हु॒त्वोपाꣳ॑शुपा॒त्रे\-ऽꣳ॑शुम॒वास्य॒ तं तृ॑तीयसव॒ने॑\-ऽपि॒सृज्या॒भि षु॑णुया॒द्यदा᳚प्या॒यय॑ति॒ तेनाꣳ॑शु॒मद्यद॑भिषु॒णोति॒ तेन॑र्जी॒षि सर्वा᳚ण्ये॒व तथ्सव॑नान्यꣳशु॒मन्ति॑ शु॒क्रव॑न्ति स॒माव॑द्वीर्याणि करोति॒ द्वौ स॑मु॒द्रौ वित॑तावजू॒र्यौ प॒र्याव॑र्तेते ज॒ठरे॑व॒ पादाः᳚। तयोः॒ पश्य॑न्तो॒ अति॑ यन्त्य॒न्यमप॑श्यन्तः॒~(४)

%3.2.2.2
सेतु॒ना\-ऽति॑ यन्त्य॒न्यम्। द्वे द्रध॑सी स॒तती॑ वस्त॒ एकः॑ के॒शी विश्वा॒ भुव॑नानि वि॒द्वान्। ति॒रो॒धायै॒त्यसि॑तं॒ वसा॑नः शु॒क्रमा द॑त्ते अनु॒हाय॑ जा॒र्यै। दे॒वा वै यद्य॒ज्ञे\-ऽकु॑र्वत॒ तदसु॑रा अकुर्वत॒ ते दे॒वा ए॒तं म॑हाय॒ज्ञम॑पश्य॒न्तम॑तन्वता\-ऽ\-ग्निहो॒त्रं व्र॒तम॑कुर्वत॒ तस्मा॒द् द्विव्र॑तः स्या॒द् द्विर्\mbox{}ह्य॑ग्निहो॒त्रं जुह्व॑ति पौर्णमा॒सं य॒ज्ञम॑ग्नीषो॒मीयं॑~(५)

%3.2.2.3
प॒शुम॑कुर्वत दा॒र्श्यं य॒ज्ञमा᳚ग्ने॒यं प॒शुम॑कुर्वत वैश्वदे॒वं प्रा॑तःसव॒नम॑कुर्वत वरुणप्रघा॒सान्माध्य॑न्दिन॒ꣳ॒ सव॑नꣳ साक\-मे॒धान्पि॑तृय॒ज्ञं त्र्य॑म्बकाꣴस्तृतीयसव॒नम॑कुर्वत॒ तमे॑षा॒मसु॑रा य॒ज्ञम॒न्व\-वा॑जिगाꣳ\-स॒न्तं नान्ववा॑य॒न्ते᳚\-ऽब्रुवन्नध्वर्त॒व्या वा इ॒मे दे॒वा अ॑भूव॒न्निति॒ तद॑ध्व॒रस्या᳚ध्वर॒त्वं ततो॑ दे॒वा अभ॑व॒न्परासु॑रा॒ य ए॒वं वि॒द्वान्थ्सोमे॑न॒ यज॑ते॒ भव॑त्या॒त्मना॒ परा᳚स्य॒ भ्रातृ॑व्यो भवति॥~(६)

{\anuvakamend[{अप॑श्यन्तो\-ऽग्नीषो॒मीय॑मा॒त्मना॒ परा॒ त्रीणि॑ च}]}%~(२)

%3.2.3.1
प॒रि॒भूर॒ग्निं प॑रि॒भूरिन्द्रं॑ परि॒भूर्विश्वा᳚न् दे॒वान्प॑रि॒भूर्माꣳ स॒ह ब्र॑ह्मवर्च॒सेन॒ स नः॑ पवस्व॒ शं गवे॒ शं जना॑य॒ शमर्व॑ते॒ शꣳ रा॑ज॒न्नोष॑धी॒भ्यो\-ऽच्छि॑न्नस्य ते रयिपते सु॒वीर्य॑स्य रा॒यस्पोष॑स्य ददि॒तारः॑ स्याम। तस्य॑ मे रास्व॒ तस्य॑ ते भक्षीय॒ तस्य॑ त इ॒दमुन्मृ॑जे। प्रा॒णाय॑ मे वर्चो॒दा वर्च॑से पवस्वापा॒नाय॑ व्या॒नाय॑ वा॒चे~(७)

%3.2.3.2
द॑क्षक्र॒तुभ्यां॒ चक्षु॑र्भ्यां मे वर्चो॒दौ वर्च॑से पवेथा॒ꣴ॒ श्रोत्रा॑या॒\-ऽऽ\-\-त्मने\-ऽङ्गे᳚भ्य॒ आयु॑षे वी॒र्या॑य॒ विष्णो॒रिन्द्र॑स्य॒ विश्वे॑षां दे॒वानां᳚ ज॒ठर॑मसि वर्चो॒दा मे॒ वर्च॑से पवस्व॒ को॑\-ऽसि॒ को नाम॒ कस्मै᳚ त्वा॒ काय॑ त्वा॒ यं त्वा॒ सोमे॒नाती॑तृपं॒ यं त्वा॒ सोमे॒नामी॑मदꣳ सुप्र॒जाः प्र॒जया॑ भूयासꣳ सु॒वीरो॑ वी॒रैः सु॒वर्चा॒ वर्च॑सा सु॒पोषः॒ पोषै॒र्विश्वे᳚भ्यो मे रू॒पेभ्यो॑ वर्चो॒दा~-~(८)

%3.2.3.3
वर्च॑से पवस्व॒ तस्य॑ मे रास्व॒ तस्य॑ ते भक्षीय॒ तस्य॑ त इ॒दमुन्मृ॑जे। बुभू॑ष॒न्नवे᳚क्षेतै॒ष वै पात्रि॑यः प्र॒जा\-प॑तिर्य॒ज्ञः प्र॒जा\-प॑ति॒स्तमे॒व त॑र्पयति॒ स ए॑नं तृ॒प्तो भूत्या॒\-ऽभि प॑वते ब्रह्मवर्च॒सका॒मो\-ऽवे᳚क्षेतै॒ष वै पात्रि॑यः प्र॒जा\-प॑तिर्य॒ज्ञः प्र॒जा\-प॑ति॒स्तमे॒व त॑र्पयति॒ स ए॑नं तृ॒प्तो ब्र॑ह्मवर्च॒सेना॒भि प॑वत आमया॒व्य-~(९)

%3.2.3.4
वे᳚क्षेतै॒ष वै पात्रि॑यः प्र॒जा\-प॑तिर्य॒ज्ञः प्र॒जा\-प॑ति॒स्तमे॒व त॑र्पयति॒ स ए॑नं तृ॒प्त आयु॑षा॒\-ऽभि प॑वते\-ऽभि॒चर॒न्नवे᳚क्षेतै॒ष वै पात्रि॑यः प्र॒जा\-प॑तिर्य॒ज्ञः प्र॒जा\-प॑ति॒स्तमे॒व त॑र्पयति॒ स ए॑नं तृ॒प्तः प्रा॑णापा॒ना\-भ्यां᳚ वा॒चो द॑क्षक्र॒तुभ्यां॒ चक्षु॑र्भ्या॒ꣴ॒ श्रोत्रा᳚भ्यामा॒त्मनो\-ऽङ्गे᳚भ्य॒ आयु॑षो॒\-ऽन्तरे॑ति ता॒जक्प्र ध॑न्वति॥~(१०)

{\anuvakamend[{वा॒चे रू॒पेभ्यो॑ वर्चो॒दा आ॑मया॒वी पञ्च॑चत्वारिꣳशच्च}]}%~(३)

%3.2.4.1
स्फ्यः स्व॒स्तिर्वि॑घ॒नः स्व॒स्तिः पर्\mbox{}शु॒र्वेदिः॑ पर॒शुर्नः॑ स्व॒स्तिः। य॒ज्ञिया॑ यज्ञ॒कृतः॑ स्थ॒ ते मा॒\-ऽस्मिन् य॒ज्ञ उप॑ ह्वयध्व॒मुप॑ मा॒ द्यावा॑पृथि॒वी ह्व॑येता॒मुपा᳚स्ता॒वः क॒लशः॒ सोमो॑ अ॒ग्निरुप॑ दे॒वा उप॑ य॒ज्ञ उप॑ मा॒ होत्रा॑ उपह॒वे ह्व॑यन्तां॒ नमो॒\-ऽग्नये॑ मख॒घ्ने म॒खस्य॑ मा॒ यशो᳚\-ऽर्या॒दित्या॑हव॒नीय॒मुप॑ तिष्ठते य॒ज्ञो वै म॒खो~(११)

%3.2.4.2
य॒ज्ञं वाव स तद॑ह॒न्तस्मा॑ ए॒व न॑म॒स्कृत्य॒ सदः॒ प्र स॑र्पत्या॒त्मनो\-ऽना᳚र्त्यै॒ नमो॑ रु॒द्राय॑ मख॒घ्ने नम॑स्कृत्या मा पा॒हीत्याग्नी᳚ध्रं॒ तस्मा॑ ए॒व न॑म॒स्कृत्य॒ सदः॒ प्र स॑र्पत्या॒त्मनो\-ऽना᳚र्त्यै॒ नम॒ इन्द्रा॑य मख॒घ्न इ॑न्द्रि॒यं मे॑ वी॒र्यं॑ मा निर्व॑धी॒रिति॑ हो॒त्रीय॑मा॒शिष॑मे॒वैतामा शा᳚स्त इन्द्रि॒यस्य॑ वी॒र्य॑स्यानि॑र्घाताय॒ या वै~(१२)

%3.2.4.3
दे॒वताः॒ सद॒स्यार्ति॑मा॒र्पय॑न्ति॒ यस्ता वि॒द्वान्प्र॒सर्प॑ति॒ न सद॒स्यार्ति॒मार्च्छ॑ति॒ नमो॒\-ऽग्नये॑ मख॒घ्न इत्या॑है॒ता वै दे॒वताः॒ सद॒स्यार्ति॒मार्प॑यन्ति॒ ता य ए॒वं वि॒द्वान्प्र॒सर्प॑ति॒ न सद॒स्यार्ति॒मार्च्छ॑ति दृ॒ढे स्थः॑ शिथि॒रे स॒मीची॒ माꣳह॑सस्पात॒ꣳ॒ सूर्यो॑ मा दे॒वो दि॒व्यादꣳह॑सस्पातु वा॒युर॒न्तरि॑क्षा-~(१३)

%3.2.4.4
द॒ग्निः पृ॑थि॒व्या य॒मः पि॒तृभ्यः॒ सर॑स्वती मनु॒ष्ये᳚भ्यो॒ देवी᳚ द्वारौ॒ मा मा॒ सन्ता᳚प्तं॒ नमः॒ सद॑से॒ नमः॒ सद॑स॒स्पत॑ये॒ नमः॒ सखी॑नां पुरो॒गाणां॒ चक्षु॑षे॒ नमो॑ दि॒वे नमः॑ पृथि॒व्या अहे॑ दैधिष॒व्योदत॑स्तिष्ठा॒न्यस्य॒ सद॑ने सीद॒ यो᳚\-ऽस्मत्पाक॑तर॒ उन्नि॒वत॒ उदु॒द्वत॑श्च गेषं पा॒तं मा᳚ द्यावा\-पृथिवी अ॒द्याह्नः॒ सदो॒ वै प्र॒सर्प॑न्तं~(१४)

%3.2.4.5
पि॒तरो\-ऽनु॒ प्र स॑र्पन्ति॒ त ए॑नमीश्व॒रा हिꣳसि॑तोः॒ सदः॑ प्र॒सृप्य॑ दक्षिणा॒र्धं परे᳚क्षे॒ताग॑न्त पितरः पितृ॒मान॒हं यु॒ष्माभि॑र्भूयासꣳ सुप्र॒जसो॒ मया॑ यू॒यं भू॑या॒स्तेति॒ तेभ्य॑ ए॒व न॑म॒स्कृत्य॒ सदः॒ प्र स॑र्पत्या॒त्मनो\-ऽना᳚र्त्यै॥~(१५)

{\anuvakamend[{म॒खो वा अ॒न्तरि॑क्षात्प्र॒सर्प॑न्त॒न्त्रय॑स्त्रिꣳशच्च}]}%~(४)

%3.2.5.1
भक्षेहि॒ मा वि॑श दीर्घायु॒त्वाय॑ शन्तनु॒त्वाय॑ रा॒यस्पोषा॑य॒ वर्च॑से सुप्रजा॒स्त्वायेहि॑ वसो पुरोवसो प्रि॒यो मे॑ हृ॒दो᳚\-ऽस्य॒श्विनो᳚स्त्वा बा॒हुभ्याꣳ॑ सघ्यासं नृ॒चक्ष॑सं त्वा देव सोम सु॒चक्षा॒ अव॑ ख्येषं म॒न्द्राभिभू॑तिः के॒तुर्य॒ज्ञानां॒ वाग्जु॑षा॒णा सोम॑स्य तृप्यतु म॒न्द्रा स्व॑र्वा॒च्यदि॑ति॒रना॑हतशीर्ष्णी॒ वाग्जु॑षा॒णा सोम॑स्य तृप्य॒त्वेहि॑ विश्वचर्\mbox{}षणे~(१६)

%3.2.5.2
श॒म्भूर्म॑यो॒भूः स्व॒स्ति मा॑ हरिवर्ण॒ प्र च॑र॒ क्रत्वे॒ दक्षा॑य रा॒यस्पोषा॑य सुवी॒रता॑यै॒ मा मा॑ राज॒न्वि बी॑भिषो॒ मा मे॒ हार्दि॑ त्वि॒षा व॑धीः। वृष॑णे॒ शुष्मा॒या\-ऽऽ\-यु॑षे॒ वर्च॑से॥ वसु॑मद्गणस्य सोम देव ते मति॒विदः॑ प्रातःसव॒नस्य॑ गाय॒त्रछ॑न्दस॒ इन्द्र॑पीतस्य॒ नरा॒शꣳस॑पीतस्य पि॒तृपी॑तस्य॒ मधु॑मत॒ उप॑हूत॒स्योप॑हूतो भक्षयामि रु॒द्रव॑द्गणस्य सोम देव ते मति॒विदो॒ माध्य॑न्दिनस्य॒ सव॑नस्य त्रि॒ष्टुप्छ॑न्दस॒ इन्द्र॑पीतस्य॒ नरा॒शꣳस॑पीतस्य~(१७)

%3.2.5.3
पि॒तृपी॑तस्य॒ मधु॑मत॒ उप॑हूत॒स्योप॑हूतो भक्षयाम्यादि॒त्यव॑द्गणस्य सोम देव ते मति॒विद॑स्तृ॒तीय॑स्य॒ सव॑नस्य॒ जग॑तीछन्दस॒ इन्द्र॑पीतस्य॒ नरा॒शꣳस॑पीतस्य पि॒तृपी॑तस्य॒ मधु॑मत॒ उप॑हूत॒स्योप॑हूतो भक्षयामि। आ प्या॑यस्व॒ समे॑तु ते वि॒श्वतः॑ सोम॒ वृष्णि॑यम्। भवा॒ वाज॑स्य सङ्ग॒थे। हिन्व॑ मे॒ गात्रा॑ हरिवो ग॒णान्मे॒ मा वि ती॑तृषः। शि॒वो मे॑ सप्त॒र्॒\mbox{}षीनुप॑ तिष्ठस्व॒ मा मे\-ऽवा॒ङ्नाभि॒मति॑~(१८)

%3.2.5.4
गाः। अपा॑म॒ सोम॑म॒मृता॑ अभू॒माद॑र्श्म॒ ज्योति॒रवि॑दाम दे॒वान्। किम॒स्मान्कृ॑णव॒दरा॑तिः॒ किमु॑ धू॒र्तिर॑मृत॒ मर्त्य॑स्य। यन्म॑ आ॒त्मनो॑ मि॒न्दाभू॑द॒ग्निस्तत्पुन॒राहा᳚र्जा॒तवे॑दा॒ विच॑र्\mbox{}षणिः। पुन॑र॒ग्निश्चक्षु॑रदा॒त्पुन॒रिन्द्रो॒ बृह॒स्पतिः॑। पुन॑र्मे अश्विना यु॒वं चक्षु॒रा ध॑त्तम॒क्ष्योः। इ॒ष्टय॑जुषस्ते देव सोम स्तु॒तस्तो॑मस्य~(१९)

%3.2.5.5
श॒स्तोक्थ॑स्य॒ हरि॑वत॒ इन्द्र॑पीतस्य॒ मधु॑मत॒ उप॑हूत॒स्योप॑हूतो भक्षयामि। आ॒पूर्याः॒ स्था मा॑ पूरयत प्र॒जया॑ च॒ धने॑न च। ए॒तत्ते॑ तत॒ ये च॒ त्वामन्वे॒तत्ते॑ पितामह प्रपितामह॒ ये च॒ त्वामन्वत्र॑ पितरो यथाभा॒गं म॑न्दध्वं॒ नमो॑ वः पितरो॒ रसा॑य॒ नमो॑ वः पितरः॒ शुष्मा॑य॒ नमो॑ वः पितरो जी॒वाय॒ नमो॑ वः पितरः~(२०)

%3.2.5.6
स्व॒धायै॒ नमो॑ वः पितरो म॒न्यवे॒ नमो॑ वः पितरो घो॒राय॒ पित॑रो॒ नमो॑ वो॒ य ए॒तस्मिँ॑ल्लो॒के स्थ यु॒ष्माꣴस्ते\-ऽनु॒ ये᳚\-ऽस्मिँल्लो॒के मां ते\-ऽनु॒ य ए॒तस्मिँ॑ल्लो॒के स्थ यू॒यं तेषां॒ वसि॑ष्ठा भूयास्त॒ ये᳚\-ऽस्मिँल्लो॒के॑\-ऽहं तेषां॒ वसि॑ष्ठो भूयासं॒ प्रजा॑पते॒ न त्वदे॒तान्य॒न्यो विश्वा॑ जा॒तानि॒ परि॒ ता ब॑भूव।~(२१)

%3.2.5.7
यत्का॑मास्ते जुहु॒मस्तन्नो॑ अस्तु व॒यꣴ स्या॑म॒ पत॑यो रयी॒णाम्। दे॒वकृ॑त॒स्यैन॑सो\-ऽव॒यज॑नमसि मनु॒ष्य॑कृत॒स्यैन॑सो\-ऽ\-व॒यज॑नमसि पि॒तृकृ॑त॒स्यैन॑सो\-ऽव॒यज॑नमस्य॒फ्सु धौ॒तस्य॑ सोम देव ते॒ नृभिः॑ सु॒तस्ये॒ष्टय॑जुषः स्तु॒तस्तो॑मस्य श॒स्तोक्थ॑स्य॒ यो भ॒क्षो अ॑श्व॒सनि॒र्यो गो॒सनि॒स्तस्य॑ ते पि॒तृभि॑र्भ॒क्षं कृ॑त॒स्योप॑हूत॒स्योप॑हूतो भक्षयामि॥~(२२)

{\anuvakamend[{वि॒श्व॒च॒र्॒\mbox{}ष॒णे॒ त्रि॒ष्टुफ्छ॑न्दस॒ इन्द्र॑पीतस्य॒ नरा॒शꣳस॑पीत॒स्याति॑ स्तु॒तस्तो॑मस्य जी॒वाय॒ नमो॑ वः पितरो बभूव॒ चतु॑श्चत्वारिꣳशच्च}]}%~(५)

%3.2.6.1
म॒ही॒नां पयो॑\-ऽसि॒ विश्वे॑षां दे॒वानां᳚ त॒नूर्\mbox{}ऋ॒ध्यास॑म॒द्य पृ॑षतीनां॒ ग्रहं॒ पृष॑तीनां॒ ग्रहो॑\-ऽसि॒ विष्णो॒र्॒\mbox{}हृद॑यम॒स्येक॑मिष॒ विष्णु॒स्त्वा\-ऽनु॒ वि च॑क्रमे भू॒तिर्द॒ध्ना घृ॒तेन॑ वर्धतां॒ तस्य॑ मे॒ष्टस्य॑ वी॒तस्य॒ द्रवि॑ण॒मा ग॑म्या॒ज्ज्योति॑रसि वैश्वान॒रं पृश्ञि॑यै दु॒ग्धं याव॑ती॒ द्यावा॑पृथि॒वी म॑हि॒त्वा याव॑च्च स॒प्त सिन्ध॑वो वित॒स्थुः। ताव॑न्तमिन्द्र ते॒~(२३)

%3.2.6.2
ग्रहꣳ॑ स॒होर्जा गृ॑ह्णा॒म्यस्तृ॑तम्। यत्कृ॑ष्णशकु॒नः पृ॑षदा॒ज्यम॑व\-मृ॒शेच्छू॒द्रा अ॑स्य प्र॒मायु॑काः स्यु॒र्यच्छ्वा\-ऽव॑मृ॒शेच्चतु॑ष्पादो\-ऽस्य प॒शवः॑ प्र॒मायु॑काः स्यु॒र्यथ्स्कन्दे॒द्यज॑मानः प्र॒मायु॑कः स्यात्प॒शवो॒ वै पृ॑षदा॒ज्यं प॒शवो॒ वा ए॒तस्य॑ स्कन्दन्ति॒ यस्य॑ पृषदा॒ज्यꣴ स्कन्द॑ति॒ यत्पृ॑षदा॒ज्यं पुन॑र्गृ॒ह्णाति॑ प॒शूने॒वास्मै॒ पुन॑र्गृह्णाति प्रा॒णो वै पृ॑षदा॒ज्यं प्रा॒णो वा~(२४)

%3.2.6.3
ए॒तस्य॑ स्कन्दति॒ यस्य॑ पृषदा॒ज्यꣴ स्कन्द॑ति॒ यत्पृ॑षदा॒ज्यं पुन॑र्गृ॒ह्णाति॑ प्रा॒णमे॒वास्मै॒ पुन॑र्गृह्णाति॒ हिर॑ण्यमव॒धाय॑ गृह्णात्य॒मृतं॒ वै हिर॑ण्यं प्रा॒णः पृ॑षदा॒ज्यम॒मृत॑मे॒वास्य॑ प्रा॒णे द॑धाति श॒तमा॑नं भवति श॒तायुः॒ पुरु॑षः श॒तेन्द्रि॑य॒ आयु॑ष्ये॒वेन्द्रि॒ये प्रति॑ तिष्ठ॒त्यश्व॒मव॑ घ्रापयति प्राजाप॒त्यो वा अश्वः॑ प्राजाप॒त्यः प्रा॒णः स्वादे॒वास्मै॒ योनेः᳚ प्रा॒णं निर्मि॑मीते॒ वि वा ए॒तस्य॑ य॒ज्ञश्छि॑द्यते॒ यस्य॑ पृषदा॒ज्यꣴ स्कन्द॑ति वैष्ण॒व्यर्चा पुन॑र्गृह्णाति य॒ज्ञो वै विष्णु॑र्य॒ज्ञेनै॒व य॒ज्ञꣳ सं त॑नोति॥~(२५)

{\anuvakamend[{ते॒ पृ॒ष॒दा॒ज्यं प्रा॒णो वै योनेः᳚ प्रा॒णं द्वाविꣳ॑शतिश्च}]}%~(६)

%3.2.7.1
देव॑ सवितरे॒तत्ते॒ प्रा\-ऽऽ\-ह॒ तत्प्र च॑ सु॒व प्र च॑ यज॒ बृह॒स्पति॑र्ब्र॒ह्मा\-ऽऽ\-यु॑ष्मत्या ऋ॒चो मा गा॑त तनू॒पाथ्साम्नः॑ स॒त्या व॑ आ॒शिषः॑ सन्तु स॒त्या आकू॑तय ऋ॒तं च॑ स॒त्यं च॑ वदत स्तु॒त दे॒वस्य॑ सवि॒तुः प्र॑स॒वे स्तु॒तस्य॑ स्तु॒तम॒स्यूर्जं॒ मह्यꣴ॑ स्तु॒तं दु॑हा॒मा मा᳚ स्तु॒तस्य॑ स्तु॒तं ग॑म्याच्छ॒स्त्रस्य॑ श॒स्त्र-~(२६)

%3.2.7.2
म॒स्यूर्जं॒ मह्यꣳ॑ श॒स्त्रं दु॑हा॒मा मा॑ श॒स्त्रस्य॑ श॒स्त्रं ग॑म्यादिन्द्रि॒याव॑न्तो वनामहे धुक्षी॒महि॑ प्र॒जामिषम्᳚। सा मे॑ स॒त्याशीर्दे॒वेषु॑ भूयाद् ब्रह्मवर्च॒सं मा ग॑म्यात्। य॒ज्ञो ब॑भूव॒ स आ ब॑भूव॒ स प्र ज॑ज्ञे॒ स वा॑वृधे। स दे॒वाना॒मधि॑पतिर्बभूव॒ सो अ॒स्माꣳ अधि॑पतीन्करोतु व॒यꣴ स्या॑म॒ पत॑यो रयी॒णाम्। य॒ज्ञो वा॒ वै~(२७)

%3.2.7.3
य॒ज्ञप॑तिं दु॒हे य॒ज्ञप॑तिर्वा य॒ज्ञं दु॑हे॒ स यः स्तु॑तश॒स्त्रयो॒र्दोह॒म\-वि॑द्वा॒न्॒ यज॑ते॒ तं य॒ज्ञो दु॑हे॒ स इ॒ष्ट्वा पापी॑यान्भवति॒ य ए॑नयो॒र्दोहं॑ वि॒द्वान् यज॑ते॒ स य॒ज्ञं दु॑हे॒ स इ॒ष्ट्वा वसी॑यान्भवति स्तु॒तस्य॑ स्तु॒तम॒स्यूर्जं॒ मह्यꣴ॑ स्तु॒तं दु॑हा॒मा मा᳚ स्तु॒तस्य॑ स्तु॒तं ग॑म्याच्छ॒स्त्रस्य॑ श॒स्त्रम॒स्यूर्जं॒ मह्यꣳ॑ श॒स्त्रं दु॑हा॒मा मा॑ श॒स्त्रस्य॑ श॒स्त्रं ग॑म्या॒दित्या॑है॒ष वै स्तु॑तश॒स्त्रयो॒र्दोह॒स्तं य ए॒वं वि॒द्वान् यज॑ते दु॒ह ए॒व य॒ज्ञमि॒ष्ट्वा वसी॑यान्भवति॥~(२८)

{\anuvakamend[{श॒स्त्रं वै श॒स्त्रन्दु॑हा॒न्द्वाविꣳ॑शतिश्च}]}%~(७)

%3.2.8.1
श्ये॒नाय॒ पत्व॑ने॒ स्वाहा॒ वट्थ्स्व॒य\-म॑भि\-गूर्ताय॒\\
नमो॑ विष्ट॒म्भाय॒ धर्म॑णे॒ स्वाहा॒ वट्थ्स्व॒य\-म॑भि\-गूर्ताय॒\\
नमः॑ परि॒धये॑ जन॒प्रथ॑नाय॒ स्वाहा॒ वट्थ्स्व॒य\-म॑भि\-गूर्ताय॒\\
नम॑ ऊ॒र्जे होत्रा॑णा॒ꣴ॒ स्वाहा॒ वट्थ्स्व॒य\-म॑भि\-गूर्ताय॒\\
नमः॒ पय॑से॒ होत्रा॑णा॒ꣴ॒ स्वाहा॒ वट्थ्स्व॒य\-म॑भि\-गूर्ताय॒\\
नमः॑ प्र॒जा\-प॑तये॒ मन॑वे॒ स्वाहा॒ वट्थ्स्व॒य\-म॑भि\-गूर्ताय॒\\
नम॑ ऋ॒तमृ॑तपाः सुवर्वा॒ट्थ्\-स्वाहा॒ वट्थ्स्व॒य\-म॑भि\-गूर्ताय॒\\
नम॑स्तृ॒म्पन्ता॒ꣳ॒ होत्रा॒ मधो᳚र्घृ॒तस्य॑ य॒ज्ञप॑ति॒मृष॑य॒ एन॑सा~(२९)

%3.2.8.2
ऽ\-ऽहुः। प्र॒जा निर्भ॑क्ता अनुत॒प्यमा॑ना मध॒व्यौ᳚ स्तो॒कावप॒ तौ र॑राध। सं न॒स्ताभ्याꣳ॑ सृजतु वि॒श्वक॑र्मा घो॒रा ऋष॑यो॒ नमो॑ अस्त्वेभ्यः। चक्षु॑ष एषां॒ मन॑सश्च स॒न्धौ बृह॒स्पत॑ये॒ महि॒ षद्द्यु॒मन्नमः॑। नमो॑ वि॒श्वक॑र्मणे॒ स उ॑ पात्व॒स्मान॑न॒न्यान्थ्सो॑म॒पान्मन्य॑मानः। प्रा॒णस्य॑ वि॒द्वान्थ्स॑म॒रे न धीर॒ एन॑श्चकृ॒वान्महि॑ ब॒द्ध ए॑षाम्। तं वि॑श्वकर्म॒न्~(३०)

%3.2.8.3
प्र मु॑ञ्चा स्व॒स्तये॒ ये भ॒क्षय॑न्तो॒ न वसू᳚न्यानृ॒हुः। यान॒ग्नयो॒\-ऽन्वत॑प्यन्त॒ धिष्णि॑या इ॒यं तेषा॑मव॒या दुरि॑ष्ट्यै॒ स्वि॑ष्टिं न॒स्तां कृ॑णोतु वि॒श्वक॑र्मा। नमः॑ पि॒तृभ्यो॑ अ॒भि ये नो॒ अख्य॑न् यज्ञ॒कृतो॑ य॒ज्ञका॑माः सुदे॒वा अ॑का॒मा वो॒ दक्षि॑णां॒ न नी॑निम॒ मा न॒स्तस्मा॒देन॑सः पापयिष्ट। याव॑न्तो॒ वै स॑द॒स्या᳚स्ते सर्वे॑ दक्षि॒ण्या᳚स्तेभ्यो॒ यो दक्षि॑णां॒ न~(३१)

%3.2.8.4
नये॒दैभ्यो॑ वृश्च्येत॒ यद्वै᳚श्वकर्म॒णानि॑ जु॒होति॑ सद॒स्या॑ने॒व तत्प्री॑णात्य॒स्मे दे॑वासो॒ वपु॑षे चिकिथ्सत॒ यमा॒शिरा॒ दम्प॑ती वा॒मम॑श्ञु॒तः। पुमा᳚न्पु॒त्रो जा॑यते वि॒न्दते॒ वस्वथ॒ विश्वे॑ अर॒पा ए॑धते गृ॒हः। आ॒शी॒र्दा॒या दम्प॑ती वा॒मम॑श्ञुता॒मरि॑ष्टो॒ रायः॑ सचता॒ꣳ॒ समो॑कसा। य आसि॑च॒थ्सन्दु॑ग्धं कु॒म्भ्या स॒हेष्टेन॒ याम॒न्नम॑तिं जहातु॒ सः। स॒र्पि॒र्ग्री॒वी~(३२)


%3.2.8.5
पीव॑र्यस्य जा॒या पीवा॑नः पु॒त्रा अकृ॑शासो अस्य। स॒हजा॑नि॒र्यः सु॑मख॒स्यमा॑न॒ इन्द्रा॑या॒शिरꣳ॑ स॒ह कु॒म्भ्या\-ऽदा᳚त्। आ॒शीर्म॒ ऊर्ज॑मु॒त सु॑प्रजा॒स्त्वमिषं॑ दधातु॒ द्रवि॑ण॒ꣳ॒ सव॑र्चसम्। स॒ञ्जय॒न्क्षेत्रा॑णि॒ सह॑सा॒\-ऽहमि॑न्द्र कृण्वा॒नो अ॒न्याꣳ अध॑रान्थ्स॒पत्नान्॑। भू॒तम॑सि भू॒ते मा॑ धा॒ मुख॑मसि॒ मुखं॑ भूयासं॒ द्यावा॑पृथि॒वी\-भ्यां᳚ त्वा॒ परि॑ गृह्णामि॒ विश्वे᳚ त्वा दे॒वा वै᳚श्वान॒राः~(३३)

%3.2.8.6
प्र च्या॑वयन्तु दि॒वि दे॒वां दृꣳ॑हा॒न्तरि॑क्षे॒ वयाꣳ॑सि पृथि॒व्यां पार्थि॑वान्ध्रु॒वं ध्रु॒वेण॑ ह॒विषा\-ऽव॒ सोमं॑ नयामसि। यथा॑ नः॒ सर्व॒मिज्जग॑दय॒क्ष्मꣳ सु॒मना॒ अस॑त्। यथा॑ न॒ इन्द्र॒ इद्विशः॒ केव॑लीः॒ सर्वाः॒ सम॑नसः॒ कर॑त्। यथा॑ नः॒ सर्वा॒ इद्दिशो॒\-ऽस्माकं॒ केव॑ली॒रसन्न्॑॥~(३४)

{\anuvakamend[{एन॑सा विश्वकर्म॒न्॒ यो दक्षि॑णां॒ न स॑र्पिर्ग्री॒वी वै᳚श्वान॒राश्च॑त्वारि॒ꣳ॒शच्च॑}]}%~(८)

%3.2.9.1
यद्वै होता᳚ध्व॒र्युम॑भ्या॒ह्वय॑ते॒ वज्र॑मेनम॒भि प्र व॑र्तय॒त्युक्थ॑शा॒ इत्या॑ह प्रातःसव॒नं प्र॑ति॒गीर्य॒ त्रीण्ये॒तान्य॒क्षरा॑णि त्रि॒पदा॑ गाय॒त्री गा॑य॒त्रं प्रा॑तःसव॒नं गा॑यत्रि॒यैव प्रा॑तःसव॒ने वज्र॑म॒न्तर्ध॑त्त उ॒क्थं वा॒चीत्या॑ह॒ माध्य॑न्दिन॒ꣳ॒ सव॑नं प्रति॒गीर्य॑ च॒त्वार्ये॒तान्य॒क्षरा॑णि॒ चतु॑ष्पदा त्रि॒ष्टुप्त्रैष्टु॑भं॒ माध्य॑न्दिन॒ꣳ॒ सव॑नं त्रि॒ष्टुभै॒व माध्य॑न्दिने॒ सव॑ने॒ वज्र॑म॒न्तर्ध॑त्त~-~(३५)

%3.2.9.2
उ॒क्थं वा॒चीन्द्रा॒येत्या॑ह तृतीयसव॒नं प्र॑ति॒गीर्य॑ स॒प्तैतान्य॒क्षरा॑णि स॒प्तप॑दा॒ शक्व॑री शाक्व॒रो वज्रो॒ वज्रे॑णै॒व तृ॑तीयसव॒ने वज्र॑म॒न्तर्ध॑त्ते ब्रह्मवा॒दिनो॑ वदन्ति॒ स त्वा अ॑ध्व॒र्युः स्या॒द्यो य॑थासव॒नं प्र॑तिग॒रे छन्दाꣳ॑सि सम्पा॒दये॒त्तेजः॑ प्रातःसव॒न आ॒त्मन्दधी॑तेन्द्रि॒यं माध्य॑न्दिने॒ सव॑ने प॒शूꣴस्तृ॑तीयसव॒न इत्युक्थ॑शा॒ इत्या॑ह प्रातःसव॒नं प्र॑ति॒गीर्य॒ त्रीण्ये॒तान्य॒क्षरा॑णि~(३६)

%3.2.9.3
त्रि॒पदा॑ गाय॒त्री गा॑य॒त्रं प्रा॑तःसव॒नं प्रा॑तःसव॒न ए॒व प्र॑तिग॒रे छन्दाꣳ॑सि॒ सम्पा॑दय॒त्यथो॒ तेजो॒ वै गा॑य॒त्री तेजः॑ प्रातःसव॒नं तेज॑ ए॒व प्रा॑तःसव॒न आ॒त्मन्ध॑त्त उ॒क्थं वा॒चीत्या॑ह॒ माध्य॑न्दिन॒ꣳ॒ सव॑नं प्रति॒गीर्य॑ च॒त्वार्ये॒तान्य॒क्षरा॑णि॒ चतु॑ष्पदा त्रि॒ष्टुप्त्रैष्टु॑भं॒ माध्य॑न्दिन॒ꣳ॒ सव॑नं॒ माध्य॑न्दिन ए॒व सव॑ने प्रतिग॒रे छन्दाꣳ॑सि॒ सम्पा॑दय॒त्यथो॑ इन्द्रि॒यं वै त्रि॒ष्टुगि॑न्द्रि॒यं माध्य॑न्दिन॒ꣳ॒ सव॑न-~(३७)

%3.2.9.4
मिन्द्रि॒यमे॒व माध्य॑न्दिने॒ सव॑न आ॒त्मन्ध॑त्त उ॒क्थं वा॒चीन्द्रा॒येत्या॑ह तृतीयसव॒नं प्र॑ति॒गीर्य॑ स॒प्तैतान्य॒क्षरा॑णि स॒प्तप॑दा॒ शक्व॑री शाक्व॒राः प॒शवो॒ जाग॑तं तृतीयसव॒नं तृ॑तीयसव॒न ए॒व प्र॑तिग॒रे छन्दाꣳ॑सि॒ सम्पा॑दय॒त्यथो॑ प॒शवो॒ वै जग॑ती प॒शव॑स्तृतीयसव॒नं प॒शूने॒व तृ॑तीयसव॒न आ॒त्मन्ध॑त्ते॒ यद्वै होता᳚ध्व॒र्युम॑भ्या॒ह्वय॑त आ॒व्य॑मस्मिन्दधाति॒ तद्यन्ना~(३८)

%3.2.9.5
ऽप॒हनी॑त पु॒रास्य॑ संवथ्स॒राद्गृ॒ह आ वे॑वीर॒ञ्छोꣳसा॒ मोद॑ इ॒वेति॑ प्र॒त्याह्व॑यते॒ तेनै॒व तदप॑ हते॒ यथा॒ वा आय॑तां प्र॒तीक्ष॑त ए॒वम॑ध्व॒र्युः प्र॑तिग॒रं प्रती᳚क्षते॒ यद॑भिप्रतिगृणी॒याद्यथाय॑तया समृ॒च्छते॑ ता॒दृगे॒व तद्यद॑र्ध॒र्चाल्लुप्ये॑त॒ यथा॒ धाव॑द्भ्यो॒ हीय॑ते ता॒दृगे॒व तत्प्र॒बाहु॒ग्वा ऋ॒त्विजा॑मुद्गी॒था उ॑द्गी॒थ ए॒वोद्गा॑तृ॒णा-~(३९)

%3.2.9.6
मृ॒चः प्र॑ण॒व उ॑क्थश॒ꣳ॒सिनां᳚ प्रतिग॒रो᳚\-ऽध्वर्यू॒णां य ए॒वं वि॒द्वान्प्र॑तिगृ॒णात्य॑न्ना॒द ए॒व भ॑व॒त्यास्य॑ प्र॒जायां᳚ वा॒जी जा॑यत इ॒यं वै होता॒साव॑ध्व॒र्युर्यदासी॑नः॒ शꣳस॑त्य॒स्या ए॒व तद्धोता॒ नैत्यास्त॑ इव॒ हीयमथो॑ इ॒मामे॒व तेन॒ यज॑मानो दुहे॒ यत्तिष्ठ॑न्प्रतिगृ॒णात्य॒मुष्या॑ ए॒व तद॑ध्व॒र्युर्नैति॒~(४०)

%3.2.9.7
तिष्ठ॑तीव॒ ह्य॑सावथो॑ अ॒मूमे॒व तेन॒ यज॑मानो दुहे॒ यदासी॑नः॒ शꣳस॑ति॒ तस्मा॑दि॒तःप्र॑दानं दे॒वा उप॑ जीवन्ति॒ यत्तिष्ठ॑न्प्रतिगृ॒णाति॒ तस्मा॑द॒मुतः॑प्रदानं मनु॒ष्या॑ उप॑ जीवन्ति॒ यत्प्राङासी॑नः॒ शꣳस॑ति प्र॒त्यङ्तिष्ठ॑न्प्रतिगृ॒णाति॒ तस्मा᳚त्प्रा॒चीन॒ꣳ॒ रेतो॑ धीयते प्र॒तीचीः᳚ प्र॒जा जा॑यन्ते॒ यद्वै होता᳚ध्व॒र्युम॑भ्या॒ह्वय॑ते॒ वज्र॑मेनम॒भि प्र व॑र्तयति॒ परा॒ङा व॑र्तते॒ वज्र॑मे॒व तन्नि क॑रोति॥~(४१)

{\anuvakamend[{सव॑ने॒ वज्र॑म॒न्तर्ध॑त्ते॒ त्रीण्ये॒तान्य॒क्षरा॑णीन्द्रि॒यं माध्य॑न्दिन॒ꣳ॒ सव॑न॒न्नोद्गा॑तृ॒णाम॑ध्व॒र्युर्नैति॑ वर्तयत्य॒ष्टौ च॑}]}%~(९)

%3.2.10.1
उ॒प॒या॒मगृ॑हीतो\-ऽसि वाक्ष॒सद॑सि वा॒क्पा\-भ्यां᳚ त्वा क्रतु॒पाभ्या॑म॒स्य य॒ज्ञस्य॑ ध्रु॒वस्याध्य॑क्षाभ्यां गृह्णाम्युपया॒मगृ॑हीतो\-ऽस्यृत॒सद॑सि चक्षु॒ष्पा\-भ्यां᳚ त्वा क्रतु॒पाभ्या॑म॒स्य य॒ज्ञस्य॑ ध्रु॒वस्याध्य॑क्षाभ्यां गृह्णाम्युपया॒मगृ॑हीतो\-ऽसि श्रुत॒सद॑सि श्रोत्र॒पा\-भ्यां᳚ त्वा क्रतु॒पाभ्या॑म॒स्य य॒ज्ञस्य॑ ध्रु॒वस्याध्य॑क्षाभ्यां गृह्णामि दे॒वेभ्य॑स्त्वा वि॒श्वदे॑वेभ्यस्त्वा॒ विश्वे᳚भ्यस्त्वा दे॒वेभ्यो॒ विष्ण॑वुरुक्रमै॒ष ते॒ सोम॒स्तꣳ र॑क्षस्व॒~(४२)

%3.2.10.2
तं ते॑ दु॒श्चक्षा॒ माव॑ ख्य॒न्मयि॒ वसुः॑ पुरो॒वसु॑र्वा॒क्पा वाचं॑ मे पाहि॒ मयि॒ वसु॑र्वि॒दद्व॑सुश्चक्षु॒ष्पाश्चक्षु॑र्मे पाहि॒ मयि॒ वसुः॑ सं॒यद्व॑सुः श्रोत्र॒पाः श्रोत्रं॑ मे पाहि॒ भूर॑सि॒ श्रेष्ठो॑ रश्मी॒नां प्रा॑ण॒पाः प्रा॒णं मे॑ पाहि॒ धूर॑सि॒ श्रेष्ठो॑ रश्मी॒नाम॑पान॒पा अ॑पा॒नं मे॑ पाहि॒ यो न॑ इन्द्रवायू मित्रावरुणावश्विनावभि॒दास॑ति॒ भ्रातृ॑व्य उ॒त्पिपी॑ते शुभस्पती इ॒दम॒हं तमध॑रं पादयामि॒ यथे᳚न्द्रा॒हमु॑त्त॒मश्चे॒तया॑नि॥~(४३)

{\anuvakamend[{र॒क्ष॒स्व॒ भ्रातृ॑व्य॒स्त्रयो॑दश च}]}%॥10॥

%3.2.11.1
प्र सो अ॑ग्ने॒ तवो॒तिभिः॑ सु॒वीरा॑भिस्तरति॒ वाज॑कर्मभिः। यस्य॒ त्वꣳ स॒ख्यमावि॑थ। प्र होत्रे॑ पू॒र्व्यं वचो॒\-ऽग्नये॑ भरता बृ॒हत्। वि॒पां ज्योतीꣳ॑षि॒ बिभ्र॑ते॒ न वे॒धसे᳚। अग्ने॒ त्री ते॒ वाजि॑ना॒ त्री ष॒धस्था॑ ति॒स्रस्ते॑ जि॒ह्वा ऋ॑तजात पू॒र्वीः। ति॒स्र उ॑ ते त॒नुवो॑ दे॒ववा॑ता॒स्ताभि॑र्नः पाहि॒ गिरो॒ अप्र॑युच्छन्न्। सं वां॒ कर्म॑णा॒ समि॒षा~(४४)

%3.2.11.2
हि॑नो॒मीन्द्रा॑विष्णू॒ अप॑सस्पा॒रे अ॒स्य। जु॒षेथां᳚ य॒ज्ञं द्रवि॑णं च धत्त॒मरि॑ष्टैर्नः प॒थिभिः॑ पा॒रय॑न्ता। उ॒भा जि॑ग्यथु॒र्न परा॑ जयेथे॒ न परा॑ जिग्ये कत॒रश्च॒नैनोः᳚। इन्द्र॑श्च विष्णो॒ यदप॑स्पृधेथां त्रे॒धा स॒हस्रं॒ वि तदै॑रयेथाम्। त्रीण्यायूꣳ॑षि॒ तव॑ जातवेदस्ति॒स्र आ॒जानी॑रु॒षस॑स्ते अग्ने। ताभि॑र्दे॒वाना॒मवो॑ यक्षि वि॒द्वानथा॑~(४५)

%3.2.11.3
भव॒ यज॑मानाय॒ शं योः। अ॒ग्निस्त्रीणि॑ त्रि॒धातू॒न्या क्षे॑ति वि॒दथा॑ क॒विः। स त्रीꣳरे॑काद॒शाꣳ इ॒ह। यक्ष॑च्च पि॒प्रय॑च्च नो॒ विप्रो॑ दू॒तः परि॑ष्कृतः। नभ॑न्तामन्य॒के स॑मे। इन्द्रा॑विष्णू दृꣳहि॒ताः शम्ब॑रस्य॒ नव॒ पुरो॑ नव॒तिं च॑ श्ञथिष्टम्। श॒तं व॒र्चिनः॑ स॒हस्रं॑ च सा॒कꣳ ह॒थो अ॑प्र॒त्यसु॑रस्य वी॒रान्। उ॒त मा॒ता म॑हि॒षमन्व॑वेनद॒मी त्वा॑ जहति पुत्र दे॒वाः। अथा᳚ब्रवीद्वृ॒त्रमिन्द्रो॑ हनि॒ष्यन्थ्सखे॑ विष्णो वित॒रं वि क्र॑मस्व॥~(४६)

{\anuvakamend[{इ॒षा\-ऽथ॑ त्वा॒ त्रयो॑दश च}]}%॥11॥

{\prashnaend[{यो वै पव॑मानाना॒न्त्रीणि॑ परि॒भूः स्फ्यः स्व॒स्तिर्भक्षेहि॑ मही॒नां पयो॑\-ऽसि॒ देव॑ सवितरे॒तत्ते᳚ श्ये॒नाय॒ यद्वै होतो॑पयाम॒गृ॑हीतो\-ऽसि वाक्ष॒सत्प्र सो अ॑ग्न॒ एका॑\-दश॥११॥ यो वै स्फ्यः स्व॒स्तिः स्व॒धायै॒ नमः॒ प्र मु़॑ञ्च॒ तिष्ठ॑तीव॒ षट्च॑त्वारिꣳशत्॥४६॥ यो वै पव॑मानानां॒ वि क्र॑मस्व॥}]}

%%% END PRASHNA
