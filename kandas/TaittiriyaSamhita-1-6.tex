\chapt{काण्डम् १}
\sect{षष्ठमः प्रश्नः}\setcounter{anuvakam}{0}
\dnsub{तैत्तिरीयसंहितायां प्रथमकाण्डे षष्ठमः प्रश्नः}
%1.6.1.1
सं त्वा॑ सिञ्चामि॒ यजु॑षा प्र॒जामायु॒र्धनं॑ च। बृह॒स्पति॑प्रसूतो॒ यज॑मान इ॒ह मा रि॑षत्॥ आज्य॑मसि स॒त्यम॑सि स॒त्यस्याध्य॑क्षमसि ह॒विर॑सि वैश्वान॒रं वै᳚श्वदे॒वमुत्पू॑तशुष्मꣳ स॒त्यौजाः॒ सहो॑\-ऽसि॒ सह॑मानमसि॒ सह॒स्वारा॑तीः॒ सह॑स्वारातीय॒तः सह॑स्व॒ पृत॑नाः॒ सह॑स्व पृतन्य॒तः। स॒हस्र॑वीर्यमसि॒ तन्मा॑ जि॒न्वा\-ऽऽ\-ज्य॒स्या\-ऽऽ\-ज्य॑मसि स॒त्यस्य॑ स॒त्यम॑सि स॒त्यायु॑-~(१)

%1.6.1.2
रसि स॒त्यशु॑ष्ममसि स॒त्येन॑ त्वा॒\-ऽभि घा॑रयामि॒ तस्य॑ ते भक्षीय\\
पञ्चा॒नां त्वा॒ वाता॑नां य॒न्त्राय॑ ध॒र्त्राय॑ गृह्णामि\\
पञ्चा॒नां त्व॑र्तू॒नां य॒न्त्राय॑ ध॒र्त्राय॑ गृह्णामि\\
पञ्चा॒नां त्वा॑ दि॒शां य॒न्त्राय॑ ध॒र्त्राय॑ गृह्णामि\\
पञ्चा॒नां त्वा॑ पञ्चज॒नानां᳚ य॒न्त्राय॑ ध॒र्त्राय॑ गृह्णामि\\
च॒रोस्त्वा॒ पञ्च॑बिलस्य य॒न्त्राय॑ ध॒र्त्राय॑ गृह्णामि॒\\
ब्रह्म॑णस्त्वा॒ तेज॑से य॒न्त्राय॑ ध॒र्त्राय॑ गृह्णामि\\
क्ष॒त्रस्य॒ त्वौज॑से य॒न्त्राय॑~(२)

%1.6.1.3
ध॒र्त्राय॑ गृह्णामि\\
वि॒शे त्वा॑ य॒न्त्राय॑ ध॒र्त्राय॑ गृह्णामि\\
सु॒वीर्या॑य त्वा गृह्णामि सुप्रजा॒स्त्वाय॑ त्वा गृह्णामि रा॒यस्पोषा॑य त्वा गृह्णामि ब्रह्मवर्च॒साय॑ त्वा गृह्णामि॒ भूर॒स्माकꣳ॑ ह॒विर्दे॒वाना॑मा॒शिषो॒ यज॑मानस्य दे॒वानां᳚ त्वा दे॒वता᳚भ्यो गृह्णामि॒ कामा॑य त्वा गृह्णामि॥~(३)

{\anuvakamend[{स॒त्यायु॒रोज॑से य॒न्त्राय॒ त्रय॑स्त्रिꣳशच्च}]}%~(१)

%1.6.2.1
ध्रु॒वो॑\-ऽसि ध्रु॒वो॑\-ऽहꣳ स॑जा॒तेषु॑ भूयासं॒\\
धीर॒श्चेत्ता॑ वसु॒विदु॒ग्रो᳚\-ऽस्यु॒ग्रो॑\-ऽहꣳ स॑जा॒तेषु॑ भूयास-\\
मु॒ग्रश्चेत्ता॑ वसु॒विद॑भि॒भूर॑स्यभि॒भूर॒हꣳ स॑जा॒तेषु॑ भूयास-\\
मभि॒भूश्चेत्ता॑ वसु॒विद्यु॒नज्मि॑ त्वा॒ ब्रह्म॑णा॒ दैव्ये॑न ह॒व्याया॒स्मै वोढ॒वे जा॑तवेदः। इन्धा॑नास्त्वा सुप्र॒जसः॑ सु॒वीरा॒ ज्योग्जी॑वेम बलि॒हृतो॑ व॒यं ते᳚॥ यन्मे॑ अग्ने अ॒स्य य॒ज्ञस्य॒ रिष्या॒-~(४)

%1.6.2.2
द्यद्वा॒ स्कन्दा॒दाज्य॑स्यो॒त वि॑ष्णो। तेन॑ हन्मि स॒पत्नं॑ दुर्मरा॒युमैनं॑ दधामि॒ निर्\mbox{}ऋ॑त्या उ॒पस्थे᳚। भूर्भुवः॒ सुव॒रुच्छु॑ष्मो अग्ने॒ यज॑मानायैधि॒ निशु॑ष्मो अभि॒दास॑ते। अग्ने॒ देवे᳚द्ध॒ मन्वि॑द्ध॒ मन्द्र॑जि॒ह्वाम॑र्त्यस्य ते होतर्मू॒र्धन्ना जि॑घर्मि रा॒यस्पोषा॑य सुप्रजा॒स्त्वाय॑ सु॒वीर्या॑य॒ मनो॑\-ऽसि प्राजाप॒त्यं मन॑सा मा भू॒तेना\-ऽऽ\-वि॑श॒ वाग॑स्यै॒न्द्री स॑पत्न॒क्षय॑णी~(५)

%1.6.2.3
वा॒चा मे᳚न्द्रि॒येणा\-ऽऽ\-वि॑श\\
वस॒न्तमृ॑तू॒नां प्री॑णामि॒ स मा᳚ प्री॒तः प्री॑णातु\\
ग्री॒ष्ममृ॑तू॒नां प्री॑णामि॒ स मा᳚ प्री॒तः प्री॑णातु\\
व॒र्॒\mbox{}षा ऋ॑तू॒नां प्री॑णामि॒ ता मा᳚ प्री॒ताः प्री॑णन्तु\\
श॒रद॑मृतू॒नां प्री॑णामि॒ सा मा᳚ प्री॒ता प्री॑णातु\\
हेमन्तशिशि॒रावृ॑तू॒नां प्री॑णामि॒ तौ मा᳚ प्री॒तौ प्री॑णीता-\\
म॒ग्नी\-षोम॑योर॒हं दे॑वय॒ज्यया॒ चक्षु॑ष्मान् भूयासम॒-\\
ग्नेर॒हं दे॑वय॒ज्यया᳚न्ना॒दो भू॑यासं॒~(६)

%1.6.2.4
दब्धि॑र॒स्यद॑ब्धो भूयास-\\
म॒मुं द॑भेयम॒ग्नी\-षोम॑योर॒हं दे॑वय॒ज्यया॑ वृत्र॒हा भू॑यास-\\
मिन्द्राग्नि॒योर॒हं दे॑वय॒ज्यये᳚न्द्रिया॒\-व्य॑न्ना॒दो भू॑यास॒-\\
मिन्द्र॑स्या॒हं दे॑वय॒ज्यये᳚न्द्रिया॒वी भू॑यासं\\
महे॒न्द्रस्या॒हं दे॑वय॒ज्यया॑ जे॒मानं॑ महि॒मानं॑ गमेयम॒ग्नेः स्वि॑ष्ट॒कृतो॒\-ऽहं दे॑वय॒ज्यया\-ऽऽ\-यु॑ष्मान् य॒ज्ञेन॑ प्रति॒ष्ठां ग॑मेयम्॥~(७)

{\anuvakamend[{रिष्या᳚थ्सपत्न॒क्षय॑ण्यन्ना॒दो भू॑यास॒ꣳ॒ षट्त्रिꣳ॑शच्च}]}%~(२)

%1.6.3.1
अ॒ग्निर्मा॒ दुरि॑ष्टात् पातु सवि॒ता\-ऽघशꣳ॑सा॒द्यो मे\-ऽन्ति॑ दू॒रे॑\-ऽराती॒यति॒ तमे॒तेन॑ जेष॒ꣳ॒ सुरू॑पवर्\mbox{}षवर्ण॒ एही॒मान् भ॒द्रान् दुर्याꣳ॑ अ॒भ्येहि॒ मामनु॑व्रता॒ न्यु॑ शी॒र्॒\mbox{}षाणि॑ मृढ्व॒मिड॒ एह्यदि॑त॒ एहि॒ सर॑स्व॒त्येहि॒ रन्ति॑रसि॒ रम॑तिरसि सू॒नर्य॑सि॒ जुष्टे॒ जुष्टिं॑ ते\-ऽशी॒योप॑हूत उपह॒वं~(८)

%1.6.3.2
ते॑\-ऽशीय॒ सा मे॑ स॒त्याशीर॒स्य य॒ज्ञस्य॑ भूया॒दरे॑डता॒ मन॑सा॒ तच्छ॑केयं य॒ज्ञो दिवꣳ॑ रोहतु य॒ज्ञो दिवं॑ गच्छतु॒ यो दे॑व॒यानः॒ पन्था॒स्तेन॑ य॒ज्ञो दे॒वाꣳ अप्ये᳚त्व॒स्मास्विन्द्र॑ इन्द्रि॒यं द॑धात्व॒स्मान्राय॑ उ॒त य॒ज्ञाः स॑चन्ताम॒स्मासु॑ सन्त्वा॒शिषः॒ सा नः॑ प्रि॒या सु॒प्रतू᳚र्तिर्म॒घोनी॒ जुष्टि॑रसि जु॒षस्व॑ नो॒ जुष्टा॑ नो-~(९)

%1.6.3.3
ऽसि॒ जुष्टिं॑ ते गमेयं॒ मनो॒ ज्योति॑र्जुषता॒माज्यं॒ विच्छि॑न्नं य॒ज्ञꣳ समि॒मं द॑धातु। बृह॒स्पति॑स्तनुतामि॒मं नो॒ विश्वे॑ दे॒वा इ॒ह मा॑दयन्ताम्॥ ब्रध्न॒ पिन्व॑स्व॒ दद॑तो मे॒ मा क्षा॑यि कुर्व॒तो मे॒ मोप॑दसत् प्र॒जा\-प॑तेर्भा॒गो᳚\-ऽस्यूर्ज॑स्वा॒न् पय॑स्वान् प्राणापा॒नौ मे॑ पाहि समानव्या॒नौ मे॑ पाह्युदानव्या॒नौ मे॑ पा॒ह्यक्षि॑तो॒\-ऽस्यक्षि॑त्यै त्वा॒ मा मे᳚ क्षेष्ठा अ॒मुत्रा॒मुष्मिँ॑ल्लो॒के॥~(१०)

{\anuvakamend[{उ॒प॒ह॒वं जुष्टा॑ नस्त्वा॒ षट् च॑}]}%~(३)

%1.6.4.1
ब॒र्॒\mbox{}हिषो॒\-ऽहं दे॑वय॒ज्यया᳚ प्र॒जावा᳚न् भूयासं॒ नरा॒शꣳस॑स्या॒हं दे॑वय॒ज्यया॑ पशु॒मान् भू॑यासम॒ग्नेः स्वि॑ष्ट॒कृतो॒\-ऽहं दे॑वय॒ज्यया\-ऽऽ\-यु॑ष्मान् य॒ज्ञेन॑ प्रति॒ष्ठां ग॑मेयम॒ग्ने\-र॒हमुज्जि॑ति॒\-मनूज्जे॑ष॒ꣳ॒ सोम॑\-स्या॒\-हमुज्जि॑ति॒\-मनूज्जे॑षम॒ग्नेर॒हमुज्जि॑ति॒\-मनूज्जे॑षम॒ग्नी\-षोम॑यो\-र॒ह\-मु\-ज्जि॑ति॒\-मनूज्जे॑ष\-मिन्द्राग्नि॒यो\-र॒हमुज्जि॑ति॒\-मनूज्जे॑ष॒\-मिन्द्र॑स्या॒हमु-~(११)

%1.6.4.2
ज्जि॑ति॒मनूज्जे॑षं महे॒न्द्रस्या॒हमुज्जि॑ति॒\-मनूज्जे॑षम॒ग्नेः स्वि॑ष्ट॒कृतो॒\-ऽहमुज्जि॑ति॒मनूज्जे॑षं॒ वाज॑स्य मा प्रस॒वेनो᳚द्ग्रा॒भेणोद॑\-ग्रभीत्। अथा॑ स॒पत्ना॒ꣳ॒ इन्द्रो॑ मे निग्रा॒भेणाध॑राꣳ अकः॥ उ॒द्ग्रा॒भं च॑ निग्रा॒भं च॒ ब्रह्म॑ दे॒वा अ॑वीवृधन्न्। अथा॑ स॒पत्ना॑निन्द्रा॒ग्नी मे॑ विषू॒चीना॒न्व्य॑स्यताम्॥ एमा अ॑ग्मन्ना॒शिषो॒ दोह॑कामा॒ इन्द्र॑वन्तो~(१२)

%1.6.4.3
वनामहे धुक्षी॒महि॑ प्र॒जामिषम्᳚॥ रोहि॑तेन त्वा॒\-ऽग्निर्दे॒वतां᳚ गमयतु॒ हरि॑भ्यां॒ त्वेन्द्रो॑ दे॒वतां᳚ गमय॒त्वेत॑शेन त्वा॒ सूर्यो॑ दे॒वतां᳚ गमयतु॒ वि ते॑ मुञ्चामि रश॒ना वि र॒श्मीन् वि योक्त्रा॒ यानि॑ परि॒चर्त॑नानि ध॒त्ताद॒स्मासु॒ द्रवि॑णं॒ यच्च॑ भ॒द्रं प्र णो᳚ ब्रूताद्भाग॒धान् दे॒वता॑सु॥ विष्णोः᳚ शं॒योर॒हं दे॑वय॒ज्यया॑ य॒ज्ञेन॑ प्रति॒ष्ठां ग॑मेय॒ꣳ॒ सोम॑स्या॒हं दे॑वय॒ज्यया॑~(१३)

%1.6.4.4
सु॒रेता॒ रेतो॑ धिषीय॒ त्वष्टु॑र॒हं दे॑वय॒ज्यया॑ पशू॒नाꣳ रू॒पं पु॑षेयं दे॒वानां॒ पत्नी॑र॒ग्निर्गृ॒ह\-प॑तिर्य॒ज्ञस्य॑ मिथु॒नं तयो॑र॒हं दे॑वय॒ज्यया॑ मिथु॒नेन॒ प्र भू॑यासं वे॒दो॑\-ऽसि॒ वित्ति॑रसि वि॒देय॒ कर्मा॑सि क॒रुण॑मसि क्रि॒यासꣳ॑ स॒निर॑सि सनि॒तासि॑ स॒नेयं॑ घृ॒तव॑न्तं कुला॒यिनꣳ॑ रा॒यस्पोषꣳ॑ सह॒स्रिणं॑ वे॒दो द॑दातु वा॒जिनम्᳚॥~(१४)

{\anuvakamend[{इन्द्र॑स्या॒हमिन्द्र॑वन्तः॒ सोम॑स्या॒हं दे॑वय॒ज्यया॒ चतु॑श्चत्वारिꣳशच्च}]}%~(४)

%1.6.5.1
आ प्या॑यतां ध्रु॒वा घृ॒तेन॑ य॒ज्ञं य॑ज्ञं॒ प्रति॑ देव॒यद्भ्यः॑। सू॒र्याया॒ ऊधो\-ऽदि॑त्या उ॒पस्थ॑ उ॒रुधा॑रा पृथि॒वी य॒ज्ञे अ॒स्मिन्॥ प्र॒जा\-प॑तेर्वि॒भान्नाम॑ लो॒कस्तस्मिꣴ॑स्त्वा दधामि स॒ह यज॑मानेन॒ सद॑सि॒ सन्मे॑ भूयाः॒ सर्व॑मसि॒ सर्वं॑ मे भूयाः पू॒र्णम॑सि पू॒र्णं मे॑ भूया॒ अक्षि॑तमसि॒ मा मे᳚ क्षेष्ठाः॒ प्राच्यां᳚ दि॒शि दे॒वा ऋ॒त्विजो॑ मार्जयन्तां॒ दक्षि॑णायां~(१५)

%1.6.5.2
दि॒शि मासाः᳚ पि॒तरो॑ मार्जयन्तां प्र॒तीच्यां᳚ दि॒शि गृ॒हाः प॒शवो॑ मार्जयन्ता॒मुदी᳚च्यां दि॒श्याप॒ ओष॑धयो॒ वन॒स्पत॑यो मार्जयन्तामू॒र्ध्वायां᳚ दि॒शि य॒ज्ञः सं॑वथ्स॒रो य॒ज्ञप॑तिर्मार्जयन्तां॒ विष्णोः॒ क्रमो᳚\-ऽस्यभिमाति॒हा गा॑य॒त्रेण॒ छन्द॑सा पृथि॒वीमनु॒ वि क्र॑मे॒ निर्भ॑क्तः॒ स यं द्वि॒ष्मो विष्णोः॒ क्रमो᳚\-ऽस्यभिशस्ति॒हा त्रैष्टु॑भेन॒ छन्द॑सा॒\-ऽन्तरि॑क्ष॒मनु॒ वि क्र॑मे॒ निर्भ॑क्तः॒ स यं द्वि॒ष्मो विष्णोः॒ क्रमो᳚\-ऽस्यरातीय॒तो ह॒न्ता जाग॑तेन॒ छन्द॑सा॒ दिव॒मनु॒ वि क्र॑मे॒ निर्भ॑क्तः॒ स यं द्वि॒ष्मो विष्णोः॒ क्रमो॑\-ऽसि शत्रूय॒तो ह॒न्ता\-ऽऽ\-नु॑ष्टुभेन॒ छन्द॑सा॒ दिशो\-ऽनु॒ वि क्र॑मे॒ निर्भ॑क्तः॒ स यं द्वि॒ष्मः॥~(१६)

{\anuvakamend[{दक्षि॑णायां द्वि॒ष्मो विष्णो॒रेका॒न्नत्रि॒ꣳ॒शच्च॑}]}%~(५)

%1.6.6.1
अग॑न्म॒ सुवः॒ सुव॑रगन्म स॒न्दृश॑स्ते॒ मा छि॑थ्सि॒ यत्ते॒ तप॒स्तस्मै॑ ते॒ मा\-ऽऽ\-वृ॑क्षि सु॒भूर॑सि॒ श्रेष्ठो॑ रश्मी॒नामा॑यु॒र्धा अ॒स्यायु॑र्मे धेहि वर्चो॒धा अ॑सि॒ वर्चो॒ मयि॑ धेही॒दम॒हम॒मुं भ्रातृ॑व्यमा॒भ्यो दि॒ग्भ्यो᳚\-ऽस्यै दि॒वो᳚\-ऽस्माद॒न्तरि॑क्षाद॒स्यै पृ॑थि॒व्या अ॒स्मा\-द॒न्नाद्या॒न्निर्भ॑जामि॒ निर्भ॑क्तः॒ स यं द्वि॒ष्मः।~(१७)

%1.6.6.2
सं ज्योति॑षा\-ऽभूवमै॒न्द्रीमा॒वृत॑\-म॒न्वाव॑र्ते॒ सम॒हं प्र॒जया॒ सं मया᳚ प्र॒जा सम॒हꣳ रा॒यस्पोषे॑ण॒ सं मया॑ रा॒यस्पोषः॒ समि॑द्धो अग्ने मे दीदिहि समे॒द्धा ते॑ अग्ने दीद्यासं॒ वसु॑मान् य॒ज्ञो वसी॑यान् भूयास॒मग्न॒ आयूꣳ॑षि पवस॒ आ सु॒वोर्ज॒मिषं॑ च नः। आ॒रे बा॑धस्व दु॒च्छुना᳚म्॥ अग्ने॒ पव॑स्व॒ स्वपा॑ अ॒स्मे वर्चः॑ सु॒वीर्यम्᳚।~(१८)

%1.6.6.3
दध॒त् पोषꣳ॑ र॒यिं मयि॑। अग्ने॑ गृहपते सुगृहप॒तिर॒हं त्वया॑ गृ॒हप॑तिना भूयासꣳ सुगृहप॒तिर्मया॒ त्वं गृ॒हप॑तिना भूयाः श॒तꣳ हिमा॒स्तामा॒शिष॒मा शा॑से॒ तन्त॑वे॒ ज्योति॑ष्मतीं॒ तामा॒शिष॒माशा॑से॒\-ऽमुष्मै॒ ज्योति॑ष्मतीं॒ कस्त्वा॑ युनक्ति॒ स त्वा॒ विमु॑ञ्च॒त्वग्ने᳚ व्रतपते व्र॒तम॑चारिषं॒ तद॑शकं॒ तन्मे॑\-ऽराधि य॒ज्ञो ब॑भूव॒ स आ~(१९)

%1.6.6.4
ब॑भूव॒ स प्र ज॑ज्ञे॒ स वा॑वृधे। स दे॒वाना॒मधि॑पतिर्बभूव॒ सो अ॒स्माꣳ अधि॑पतीन् करोतु व॒यꣴ स्या॑म॒ पत॑यो रयी॒णाम्॥ गोमाꣳ॑ अ॒ग्ने\-ऽवि॑माꣳ अ॒श्वी य॒ज्ञो नृ॒वथ्स॑खा॒ सद॒मिद॑प्रमृ॒ष्यः। इडा॑वाꣳ ए॒षो अ॑सुर प्र॒जावा᳚न् दी॒र्घो र॒यिः पृ॑थुबु॒ध्नः स॒भावान्॑॥~(२०)

{\anuvakamend[{द्वि॒ष्मः सु॒वीर्य॒ꣳ॒ स आ पञ्च॑त्रिꣳशच्च}]}%~(६)

%1.6.7.1
यथा॒ वै स॑मृतसो॒मा ए॒वं वा ए॒ते स॑मृतय॒ज्ञा यद्द॑र्\mbox{}श\-पूर्ण\-मा॒सौ कस्य॒ वाह॑ दे॒वा य॒ज्ञमा॒ गच्छ॑न्ति॒ कस्य॑ वा॒ न ब॑हू॒नां यज॑मानानां॒ यो वै दे॒वताः॒ पूर्वः॑ परिगृ॒ह्णाति॒ स ए॑नाः॒ श्वो भू॒ते य॑जत ए॒तद्वै दे॒वाना॑मा॒यत॑नं॒ यदा॑हव॒नीयो᳚\-ऽन्त॒राग्नी प॑शू॒नां गार्\mbox{}ह॑पत्यो मनु॒ष्या॑णामन्वाहार्य॒पच॑नः पितृ॒णाम॒ग्निं गृ॑ह्णाति॒ स्व ए॒वायत॑ने दे॒वताः॒ परि॑~(२१)

%1.6.7.2
गृह्णाति॒ ताः श्वो भू॒ते य॑जते व्र॒तेन॒ वै मेध्यो॒\-ऽग्निर्व्र॒तप॑तिर्ब्राह्म॒णो व्र॑त॒भृद् व्र॒तमु॑पै॒ष्यन् ब्रू॑या॒दग्ने᳚ व्रतपते व्र॒तं च॑रिष्या॒मीत्य॒ग्निर्वै दे॒वानां᳚ व्र॒तप॑ति॒स्तस्मा॑ ए॒व प्र॑ति॒प्रोच्य॑ व्र॒तमाल॑भते ब॒र्॒\mbox{}हिषा॑ पू॒र्णमा॑से व्र॒तमुपै॑ति व॒थ्सैर॑मा\-वा॒स्या॑यामे॒तद्ध्ये॑तयो॑\-रा॒यत॑न\-मुप॒स्तीर्यः॒ पूर्व॑श्चा॒ग्निरप॑र॒\-श्चेत्या॑हुर्मनु॒ष्या॑~(२२)

%1.6.7.3
इन्न्वा उप॑स्तीर्णमि॒च्छन्ति॒ किमु॑ दे॒वा येषां॒ नवा॑वसान॒\-मुपा᳚स्मि॒ञ्छ्वो य॒क्ष्यमा॑णे दे॒वता॑ वसन्ति॒ य ए॒वं वि॒द्वान॒ग्निमु॑पस्तृ॒णाति॒ यज॑मानेन ग्रा॒म्याश्च॑ प॒शवो॑\-ऽव॒रुध्या॑ आर॒ण्याश्चेत्या॑\-हु॒र्यद्ग्रा॒म्यानु॑प॒\-वस॑ति॒ तेन॑ ग्रा॒म्यानव॑ रुन्धे॒ यदा॑र॒ण्यस्या॒श्ञाति॒ तेना॑र॒ण्यान् यदना᳚श्वानुप॒वसे᳚त् पितृदेव॒त्यः॑ स्यादा\-र॒ण्यस्या᳚श्ञातीन्द्रि॒यं~(२३)

%1.6.7.4
वा आ॑र॒ण्यमि॑न्द्रि॒यमे॒वा\-ऽऽ\-त्मन् ध॑त्ते॒ यदना᳚श्वानुप॒वसे॒त् क्षोधु॑कः स्या॒द्यद॑श्ञी॒याद्रु॒द्रो᳚\-ऽस्य प॒शून॒भिम॑न्येता॒पो᳚\-ऽश्ञाति॒ तन्नेवा॑शि॒तं नेवान॑शितं॒ न क्षोधु॑को॒ भव॑ति॒ नास्य॑ रु॒द्रः प॒शून॒भि म॑न्यते॒ वज्रो॒ वै य॒ज्ञः क्षुत्खलु॒ वै म॑नु॒ष्य॑स्य॒ भ्रातृ॑व्यो॒ यदना᳚श्वानुप॒वस॑ति॒ वज्रे॑णै॒व सा॒क्षात्क्षुधं॒ भ्रातृ॑व्यꣳ हन्ति॥~(२४)

{\anuvakamend[{परि॑ मनु॒ष्या॑ इन्द्रि॒यꣳ सा॒क्षात् त्रीणि॑ च}]}%~(७)

%1.6.8.1
यो वै श्र॒द्धामना॑रभ्य य॒ज्ञेन॒ यज॑ते॒ नास्ये॒ष्टाय॒ श्रद्द॑धते॒\-ऽपः प्र ण॑यति श्र॒द्धा वा आपः॑ श्र॒द्धामे॒वा\-ऽऽ\-रभ्य॑ य॒ज्ञेन॑ यजत उ॒भये᳚\-ऽस्य देवमनु॒ष्या इ॒ष्टाय॒ श्रद्द॑धते॒ तदा॑हु॒रति॒ वा ए॒ता वर्त्र॑न्नेद॒न्त्यति॒ वाचं॒ मनो॒ वावैता नाति॑ नेद॒न्तीति॒ मन॑सा॒ प्र ण॑यती॒यं वै मनो॒-~(२५)

%1.6.8.2
ऽनयै॒वैनाः॒ प्र ण॑य॒त्यस्क॑न्नहविर्भवति॒ य ए॒वं वेद॑ यज्ञायु॒धानि॒ सम्भ॑रति य॒ज्ञो वै य॑ज्ञायु॒धानि॑ य॒ज्ञमे॒व तथ्सम्भ॑रति॒ यदेक॑मेकꣳ स॒म्भरे᳚त् पितृदेव॒त्या॑नि स्यु॒र्यथ्स॒ह सर्वा॑णि मानु॒षाणि॒ द्वेद्वे॒ सम्भ॑रति याज्यानुवा॒क्य॑योरे॒व रू॒पं क॑रो॒त्यथो॑ मिथु॒नमे॒व यो वै दश॑ यज्ञायु॒धानि॒ वेद॑ मुख॒तो᳚\-ऽस्य य॒ज्ञः क॑ल्पते॒ स्फ्य-~(२६)

%1.6.8.3
श्च॑ क॒पाला॑नि चाग्निहोत्र॒हव॑णी च॒ शूर्पं॑ च कृष्णाजि॒नं च॒ शम्या॑ चो॒लूख॑लं च॒ मुस॑लं च दृ॒षच्चोप॑ला चै॒तानि॒ वै दश॑ यज्ञायु॒धानि॒ य ए॒वं वेद॑ मुख॒तो᳚\-ऽस्य य॒ज्ञः क॑ल्पते॒ यो वै दे॒वेभ्यः॑ प्रति॒प्रोच्य॑ य॒ज्ञेन॒ यज॑ते जु॒षन्ते᳚\-ऽस्य दे॒वा ह॒व्यꣳ ह॒विर्नि॑रु॒प्यमा॑णम॒भि म॑न्त्रयेता॒ग्निꣳ होता॑रमि॒ह तꣳ हु॑व॒ इति॑~(२७)

%1.6.8.4
दे॒वेभ्य॑ ए॒व प्र॑ति॒प्रोच्य॑ य॒ज्ञेन॑ यजते जु॒षन्ते᳚\-ऽस्य दे॒वा ह॒व्यमे॒ष वै य॒ज्ञस्य॒ ग्रहो॑ गृही॒त्वैव य॒ज्ञेन॑ यजते॒ तदु॑दि॒त्वा वाचं॑ यच्छति य॒ज्ञस्य॒ धृत्या॒ अथो॒ मन॑सा॒ वै प्र॒जा\-प॑तिर्य॒ज्ञम॑तनुत॒ मन॑सै॒व तद्य॒ज्ञं त॑नुते॒ रक्ष॑सा॒मन॑न्ववचाराय॒ यो वै य॒ज्ञं योग॒ आग॑ते यु॒नक्ति॑ यु॒ङ्क्ते यु॑ञ्जा॒नेषु॒ कस्त्वा॑ युनक्ति॒ स त्वा॑ युन॒क्त्वित्या॑ह प्र॒जा\-प॑ति॒र्वै कः प्र॒जा\-प॑तिनै॒वैनं॑ युनक्ति यु॒ङ्क्ते यु॑ञ्जा॒नेषु॑॥~(२८)

{\anuvakamend[{वै मनः॒ स्फ्य इति॑ युन॒क्त्वेका॑\-दश च}]}%~(८)

%1.6.9.1
प्र॒जा\-प॑तिर्य॒ज्ञान॑सृजता\-ग्निहो॒त्रं चा᳚ग्निष्टो॒मं च॑ पौर्णमा॒सीं चो॒क्थ्यं॑ चामावा॒स्यां᳚ चातिरा॒त्रं च॒ तानुद॑मिमीत॒ याव॑दग्निहो॒त्रमासी॒त् तावा॑नग्निष्टो॒मो याव॑ती पौर्णमा॒सी तावा॑नु॒क्थ्यो॑ याव॑त्यमावा॒स्या॑ तावा॑नतिरा॒त्रो य ए॒वं वि॒द्वान॑ग्निहो॒त्रं जु॒होति॒ याव॑दग्निष्टो॒मेनो॑पा॒प्नोति॒ ताव॒दुपा᳚\-ऽऽ\-प्नोति॒ य ए॒वं वि॒द्वान् पौ᳚र्णमा॒सीं यज॑ते॒ याव॑दु॒क्थ्ये॑नो\-पा॒प्नोति॒~(२९)

%1.6.9.2
ताव॒दुपा᳚\-ऽऽ\-प्नोति॒ य ए॒वं वि॒द्वान॑मावा॒स्यां᳚ यज॑ते॒ याव॑दतिरा॒त्रेणो॑पा॒प्नोति॒ ताव॒दुपा᳚\-ऽऽ\-प्नोति परमे॒ष्ठिनो॒ वा ए॒ष य॒ज्ञो\-ऽग्र॑ आसी॒त् तेन॒ स प॑र॒मां काष्ठा॑मगच्छ॒त् तेन॑ प्र॒जा\-प॑तिं नि॒रवा॑सायय॒त् तेन॑ प्र॒जा\-प॑तिः पर॒मां काष्ठा॑मगच्छ॒त् तेनेन्द्रं॑ नि॒रवा॑सायय॒त् तेनेन्द्रः॑ पर॒मां काष्ठा॑मगच्छ॒त् तेना॒ग्नी\-षोमौ॑ नि॒रवा॑सायय॒त् तेना॒ग्नी\-षोमौ॑ पर॒मां काष्ठा॑मगच्छतां॒ य~(३०)

%1.6.9.3
ए॒वं वि॒द्वान् द॑र्\mbox{}शपूर्णमा॒सौ यज॑ते पर॒मामे॒व काष्ठां᳚ गच्छति॒ यो वै प्रजा॑तेन य॒ज्ञेन॒ यज॑ते॒ प्र प्र॒जया॑ प॒शुभि॑र्मिथु॒नैर्जा॑यते॒ द्वाद॑श॒ मासाः᳚ संवथ्स॒रो द्वाद॑श द्व॒न्द्वानि॑ दर्\mbox{}श\-पूर्ण\-मा॒सयो॒स्तानि॑ स॒म्पाद्या॒नीत्या॑हुर्व॒थ्सं चो॑पावसृ॒जत्यु॒खां चाधि॑ श्रय॒त्यव॑ च॒ हन्ति॑ दृ॒षदौ॑ च स॒माह॒न्त्यधि॑ च॒ वप॑ते क॒पाला॑नि॒ चोप॑ दधाति पुरो॒डाशं॑ चा-~(३१)

%1.6.9.4
धि॒श्रय॒त्याज्यं॑ च स्तम्बय॒जुश्च॒ हर॑त्य॒भि च॑ गृह्णाति॒ वेदिं॑ च परिगृ॒ह्णाति॒ पत्नीं᳚ च॒ सं न॑ह्यति॒ प्रोक्ष॑णीश्चा\-ऽऽ\-सा॒दय॒त्याज्यं॑ चै॒तानि॒ वै द्वाद॑श द्व॒न्द्वानि॑ दर्\mbox{}श\-पूर्ण\-मा॒सयो॒स्तानि॒ य ए॒वꣳ स॒म्पाद्य॒ यज॑ते॒ प्रजा॑तेनै॒व य॒ज्ञेन॑ यजते॒ प्र प्र॒जया॑ प॒शुभि॑र्मिथु॒नैर्जा॑यते॥~(३२)

{\anuvakamend[{उ॒क्थ्ये॑नोपा॒प्नोत्य॑गच्छतां॒ यः पु॑रो॒डाशं॑ च चत्वारि॒ꣳ॒शच्च॑}]}%~(९)

%1.6.10.1
ध्रु॒वो॑\-ऽसि ध्रु॒वो॑\-ऽहꣳ स॑जा॒तेषु॑ भूयास॒मित्या॑ह ध्रु॒वाने॒वैना᳚न् कुरुत उ॒ग्रो᳚\-ऽस्यु॒ग्रो॑\-ऽहꣳ स॑जा॒तेषु॑ भूयास॒मित्या॒हाप्र॑तिवादिन ए॒वैना᳚न्कुरुते\-ऽभि॒भूर॑स्यभि॒भूर॒हꣳ स॑जा॒तेषु॑ भूयास॒मित्या॑ह॒ य ए॒वैनं॑ प्रत्यु॒त्पिपी॑ते॒ तमुपा᳚स्यते यु॒नज्मि॑ त्वा॒ ब्रह्म॑णा॒ दैव्ये॒नेत्या॑है॒ष वा अ॒ग्नेर्योग॒स्तेनै॒-~(३३)

%1.6.10.2
वैनं॑ युनक्ति य॒ज्ञस्य॒ वै समृ॑द्धेन दे॒वाः सु॑व॒र्गं लो॒कमा॑यन् य॒ज्ञस्य॒ व्यृ॑द्धे॒नासु॑रा॒न् परा॑भावय॒न्॒ यन्मे॑ अग्ने अ॒स्य य॒ज्ञस्य॒ रिष्या॒दित्या॑ह य॒ज्ञस्यै॒व तथ्समृ॑द्धेन॒ यज॑मानः सुव॒र्गं लो॒कमे॑ति य॒ज्ञस्य॒ व्यृ॑द्धेन॒ भ्रातृ॑व्या॒न् परा॑ भावयत्यग्नि\-हो॒त्रमे॒ताभि॒र्व्याहृ॑तीभि॒\-रुप॑ सादयेद्यज्ञमु॒खं वा अ॑ग्निहो॒त्रं ब्रह्मै॒ता व्याहृ॑तयो यज्ञमु॒ख ए॒व ब्रह्म॑~(३४)

%1.6.10.3
कुरुते संवथ्स॒रे प॒र्याग॑त ए॒ताभि॑रे॒वोप॑सादये॒द् ब्रह्म॑णै॒वोभ॒यतः॑ संवथ्स॒रं परि॑गृह्णाति दर्\mbox{}श\-पूर्ण\-मा॒सौ चा॑तुर्मा॒स्यान्या॒लभ॑मान ए॒ताभि॒र्व्याहृ॑तीभिर्\mbox{}ह॒वीꣴष्यासा॑द\-येद्यज्ञमु॒खं वै द॑र्\mbox{}शपूर्णमा॒सौ चा॑तुर्मा॒स्यानि॒ ब्रह्मै॒ता व्याहृ॑तयो यज्ञमु॒ख ए॒व ब्रह्म॑ कुरुते संवथ्स॒रे प॒र्याग॑त ए॒ताभि॑रे॒वासा॑द\-ये॒द् ब्रह्म॑णै॒वोभ॒यतः॑ संवथ्स॒रं परि॑गृह्णाति॒ यद्वै य॒ज्ञस्य॒ साम्ना᳚ क्रि॒यते॑ रा॒ष्ट्रं~(३५)

%1.6.10.4
य॒ज्ञस्या॒\-ऽऽ\-शीर्ग॑च्छति॒ यदृ॒चा विशं॑ य॒ज्ञस्या॒\-ऽऽ\-शीर्ग॑च्छ॒त्यथ॑ ब्राह्म॒णो॑\-ऽना॒शीर्के॑ण य॒ज्ञेन॑ यजते सामिधे॒नीर॑नुव॒क्ष्यन्ने॒ता व्याहृ॑तीः पु॒रस्ता᳚द्दध्या॒द् ब्रह्मै॒व प्र॑ति॒पदं॑ कुरुते॒ तथा᳚ ब्राह्म॒णः साशी᳚र्केण य॒ज्ञेन॑ यजते॒ यं का॒मये॑त॒ यज॑मानं॒ भ्रातृ॑व्यमस्य य॒ज्ञस्या॒\-ऽऽ\-शीर्ग॑च्छे॒दिति॒ तस्यै॒ता व्याहृ॑तीः पुरो\-ऽनुवा॒क्या॑यां दध्याद् भ्रातृव्यदेव॒त्या॑ वै पु॑रो\-ऽनुवा॒क्या᳚ भ्रातृ॑व्यमे॒वास्य॑ य॒ज्ञस्या॒-~(३६)

%1.6.10.5
ऽ\-ऽशीर्ग॑च्छति॒ यान् का॒मये॑त॒ यज॑मानान्थ्स॒माव॑त्येनान् य॒ज्ञस्या॒\-ऽऽ\-शीर्ग॑च्छे॒दिति॒ तेषा॑मे॒ता व्याहृ॑तीः पुरो\-ऽनुवा॒क्या॑या अर्ध॒र्च एकां᳚ दध्याद्या॒ज्या॑यै पु॒रस्ता॒देकां᳚ या॒ज्या॑या अर्ध॒र्च एकां॒ तथै॑नान्थ्स॒माव॑ती य॒ज्ञस्या॒\-ऽऽ\-शीर्ग॑च्छति॒ यथा॒ वै प॒र्जन्यः॒ सुवृ॑ष्टं॒ वर्\mbox{}ष॑त्ये॒वं य॒ज्ञो यज॑मानाय वर्\mbox{}षति॒ स्थल॑योद॒कं प॑रिगृ॒ह्णन्त्या॒शिषा॑ य॒ज्ञं यज॑मानः॒ परि॑गृह्णाति॒ मनो॑\-ऽसि प्राजाप॒त्यं~(३७)

%1.6.10.6
मन॑सा मा भू॒तेना\-ऽऽ\-वि॒शेत्या॑ह॒ मनो॒ वै प्रा॑जाप॒त्यं प्रा॑जाप॒त्यो य॒ज्ञो मन॑ ए॒व य॒ज्ञमा॒त्मन् ध॑त्ते॒ वाग॑स्यै॒न्द्री स॑पत्न॒क्षय॑णी वा॒चा मे᳚न्द्रि॒येणा\-ऽऽ\-वि॒शेत्या॑है॒न्द्री वै वाग्वाच॑मे॒वैन्द्रीमा॒त्मन् ध॑त्ते॥~(३८)

{\anuvakamend[{तेनै॒व ब्रह्म॑ रा॒ष्ट्रमे॒वास्य॑ य॒ज्ञस्य॑ प्राजाप॒त्यꣳ षट्त्रिꣳ॑शच्च}]}%॥10॥

%1.6.11.1
यो वै स॑प्तद॒शं प्र॒जा\-प॑तिं य॒ज्ञम॒न्वाय॑त्तं॒ वेद॒ प्रति॑ य॒ज्ञेन॑ तिष्ठति॒ न य॒ज्ञाद् भ्रꣳ॑शत॒ आ श्रा॑व॒येति॒ चतु॑रक्षर॒मस्तु॒ श्रौष॒डिति॒ चतु॑रक्षरं॒ यजेति॒ द्व्य॑क्षरं॒ ये यजा॑मह॒ इति॒ पञ्चा᳚क्षरं द्व्यक्ष॒रो व॑षट्का॒र ए॒ष वै स॑प्तद॒शः प्र॒जा\-प॑तिर्य॒ज्ञम॒न्वाय॑त्तो॒ य ए॒वं वेद॒ प्रति॑ य॒ज्ञेन॑ तिष्ठति॒ न य॒ज्ञाद् भ्रꣳ॑शते॒ यो वै य॒ज्ञस्य॒ प्राय॑णं प्रति॒ष्ठा-~(३९)

%1.6.11.2
मु॒दय॑नं॒ वेद॒ प्रति॑ष्ठिते॒नारि॑ष्टेन य॒ज्ञेन॑ स॒ꣴ॒स्थां ग॑च्छ॒त्या श्रा॑व॒यास्तु॒ श्रौष॒ड्यज॒ ये यजा॑महे वषट्का॒र ए॒तद्वै य॒ज्ञस्य॒ प्राय॑णमे॒षा प्र॑ति॒ष्ठैतदु॒दय॑नं॒ य ए॒वं वेद॒ प्रति॑ष्ठिते॒नारि॑ष्टेन य॒ज्ञेन॑ स॒ꣴ॒स्थां ग॑च्छति॒ यो वै सू॒नृता॑यै॒ दोहं॒ वेद॑ दु॒ह ए॒वैनां᳚ य॒ज्ञो वै सू॒नृता\-ऽऽ\-श्रा॑व॒येत्यैवैना॑मह्व॒दस्तु॒~(४०)

%1.6.11.3
श्रौष॒डित्यु॒पावा᳚स्रा॒ग्यजेत्युद॑नैषी॒द्ये यजा॑मह॒ इत्युपा॑स\-दद्वषट्का॒रेण॑ दोग्ध्ये॒ष वै सू॒नृता॑यै॒ दोहो॒ य ए॒वं वेद॑ दु॒ह ए॒वैनां᳚ दे॒वा वै स॒त्रमा॑सत॒ तेषां॒ दिशो॑\-ऽदस्य॒न्त ए॒तामा॒र्द्रां प॒ङ्क्तिम॑पश्य॒न्ना श्रा॑व॒येति॑ पुरोवा॒तम॑जनय॒न्नस्तु॒ श्रौष॒डित्य॒ब्भ्रꣳ सम॑प्लावय॒न्॒ यजेति॑ वि॒द्युत॑-~(४१)

%1.6.11.4
मजनय॒न्॒ ये यजा॑मह॒ इति॒ प्राव॑र्\mbox{}षयन्न॒भ्य॑स्तनयन् वषट्का॒रेण॒ ततो॒ वै तेभ्यो॒ दिशः॒ प्राप्या॑यन्त॒ य ए॒वं वेद॒ प्रास्मै॒ दिशः॑ प्यायन्ते प्र॒जा\-प॑तिं त्वो॒वेद॑ प्र॒जा\-प॑तिस्त्वं वेद॒ यं प्र॒जा\-प॑ति॒र्वेद॒ स पुण्यो॑ भवत्ये॒ष वै छ॑न्द॒स्यः॑ प्र॒जा\-प॑ति॒रा श्रा॑व॒यास्तु॒ श्रौष॒ड्यज॒ ये यजा॑महे वषट्का॒रो य ए॒वं वेद॒ पुण्यो॑ भवति वस॒न्त-~(४२)

%1.6.11.5
मृ॑तू॒नां प्री॑णा॒मीत्या॑ह॒र्तवो॒ वै प्र॑या॒जा ऋ॒तूने॒व प्री॑णाति॒ ते᳚\-ऽस्मै प्री॒ता य॑थापू॒र्वं क॑ल्पन्ते॒ कल्प॑न्ते\-ऽस्मा ऋ॒तवो॒ य ए॒वं वेदा॒ग्नी\-षोम॑योर॒हं दे॑वय॒ज्यया॒ चक्षु॑ष्मान् भूयास॒मित्या॑हा॒ग्नी\-षोमा᳚भ्यां॒ वै य॒ज्ञश्चक्षु॑ष्मा॒न् ताभ्या॑मे॒व चक्षु॑रा॒त्मन् ध॑त्ते॒\-ऽग्नेर॒हं दे॑वय॒ज्यया᳚न्ना॒दो भू॑यास॒मित्या॑हा॒ग्निर्वै दे॒वाना॑मन्ना॒दस्तेनै॒वा-~(४३)

%1.6.11.6
ऽन्नाद्य॑मा॒त्मन् ध॑त्ते॒ दब्धि॑र॒स्यद॑ब्धो भूयासम॒मुं द॑भेय॒मित्या॑\-है॒तया॒ वै दब्ध्या॑ दे॒वा असु॑रानदभ्नुव॒न् तयै॒व भ्रातृ॑व्यं दभ्नोत्य॒ग्नी\-षोम॑यो\-र॒हं दे॑वय॒ज्यया॑ वृत्र॒हा भू॑यास॒मित्या॑हा॒ग्नी\-षोमा᳚भ्यां॒ वा इन्द्रो॑ वृ॒त्रम॑ह॒न् ताभ्या॑मे॒व भ्रातृ॑व्यꣴ स्तृणुत इन्द्राग्नि॒योर॒हं दे॑व\-य॒ज्यये᳚\-न्द्रिया॒व्य॑न्ना॒दो भू॑यास॒मित्या॑हेन्द्रिया॒व्ये॑वान्ना॒दो भ॑व॒तीन्द्र॑स्या॒-~(४४)

%1.6.11.7
ऽहं दे॑वय॒ज्यये᳚न्द्रिया॒वी भू॑यास॒मित्या॑हेन्द्रिया॒व्ये॑व भ॑वति महे॒न्द्रस्या॒हं दे॑वय॒ज्यया॑ जे॒मानं॑ महि॒मानं॑ गमेय॒मित्या॑ह जे॒मान॑मे॒व म॑हि॒मानं॑ गच्छत्य॒ग्नेः स्वि॑ष्ट॒कृतो॒\-ऽहं दे॑वय॒ज्यया\-ऽऽ\-यु॑ष्मान् य॒ज्ञेन॑ प्रति॒ष्ठां ग॑मेय॒मित्या॒हा\-ऽऽ\-यु॑रे॒वा\-ऽऽ\-त्मन् ध॑त्ते॒ प्रति॑ य॒ज्ञेन॑ तिष्ठति॥~(४५)

{\anuvakamend[{प्र॒ति॒ष्ठाम॑ह्व॒दस्तु॑ वि॒द्युतं॑ वस॒न्तमे॒वेन्द्र॑स्या॒\-ऽष्टात्रिꣳ॑शच्च}]}%॥11॥

%1.6.12.1
इन्द्रं॑ वो वि॒श्वत॒स्परि॒ हवा॑महे॒ जने᳚भ्यः। अ॒स्माक॑मस्तु॒ केव॑लः॥ इन्द्रं॒ नरो॑ ने॒मधि॑ता हवन्ते॒ यत्पार्या॑ यु॒नज॑ते॒ धिय॒स्ताः। शूरो॒ नृषा॑ता॒ शव॑सश्चका॒न आ गोम॑ति व्र॒जे भ॑जा॒ त्वं नः॑॥ इ॒न्द्रि॒याणि॑ शतक्रतो॒ या ते॒ जने॑षु प॒ञ्चसु॑। इन्द्र॒ तानि॑ त॒ आ वृ॑णे॥ अनु॑ ते दायि म॒ह इ॑न्द्रि॒याय॑ स॒त्रा ते॒ विश्व॒मनु॑ वृत्र॒हत्ये᳚। अनु॑~(४६)

%1.6.12.2
क्ष॒त्रमनु॒ सहो॑ यज॒त्रेन्द्र॑ दे॒वेभि॒रनु॑ ते नृ॒षह्ये᳚॥ आ यस्मि᳚न्थ्स॒प्त वा॑स॒वास्तिष्ठ॑न्ति स्वा॒रुहो॑ यथा। ऋषि॑र्\mbox{}ह दीर्घ॒श्रुत्त॑म॒ इन्द्र॑स्य घ॒र्मो अति॑थिः॥ आ॒मासु॑ प॒क्वमैर॑य॒ आ सूर्यꣳ॑ रोहयो दि॒वि। घ॒र्मं न साम॑न्तपता सुवृ॒क्तिभि॒र्जुष्टं॒ गिर्व॑णसे॒ गिरः॑॥ इन्द्र॒मिद्गा॒थिनो॑ बृ॒हदिन्द्र॑म॒र्केभि॑र॒र्किणः॑। इन्द्रं॒ वाणी॑रनूषत॥ गाय॑न्ति त्वा \mbox{गाय॒त्रिणो-~(४७)}

%1.6.12.3
ऽर्च॑न्त्य॒र्कम॒र्किणः॑। ब्र॒ह्माण॑स्त्वा शतक्रत॒वुद्व॒ꣳ॒शमि॑व येमिरे॥ अ॒ꣳ॒हो॒मुचे॒ प्र भ॑रेमा मनी॒षामो॑षिष्ठ॒दाव्न्ने॑ सुम॒तिं गृ॑णा॒नाः। इ॒दमि॑न्द्र॒ प्रति॑ ह॒व्यं गृ॑भाय स॒त्याः स॑न्तु॒ यज॑मानस्य॒ कामाः᳚॥ वि॒वेष॒ यन्मा॑ धि॒षणा॑ ज॒जान॒ स्तवै॑ पु॒रा पार्या॒दिन्द्र॒मह्नः॑। अꣳह॑सो॒ यत्र॑ पी॒पर॒द्यथा॑ नो ना॒वेव॒ यान्त॑मु॒भये॑ हवन्ते॥ प्र स॒म्राजं॑ प्रथ॒मम॑ध्व॒राणा॑-~(४८)

%1.6.12.4
मꣳहो॒मुचं॑ वृष॒भं य॒ज्ञिया॑नाम्। अ॒पां नपा॑तमश्विना॒ हय॑न्तम॒स्मिन्न॑र इन्द्रि॒यं ध॑त्त॒मोजः॑॥ वि न॑ इन्द्र॒ मृधो॑ जहि नी॒चा य॑च्छ पृतन्य॒तः। अ॒ध॒स्प॒दं तमीं᳚ कृधि॒ यो अ॒स्माꣳ अ॑भि॒दास॑ति॥ इन्द्र॑ क्ष॒त्रम॒भि वा॒ममोजो\-ऽजा॑यथा वृषभ चर्\mbox{}षणी॒नाम्। अपा॑नुदो॒ जन॑ममित्र॒यन्त॑मु॒रुं दे॒वेभ्यो॑ अकृणोरु लो॒कम्॥ मृ॒गो न भी॒मः कु॑च॒रो गि॑रि॒ष्ठाः प॑रा॒वत॒~-~(४९)

%1.6.12.5
आ ज॑गामा॒ पर॑स्याः। सृ॒कꣳ स॒ꣳ॒शाय॑ प॒विमि॑न्द्र ति॒ग्मं वि शत्रू᳚न् ताढि॒ वि मृधो॑ नुदस्व॥ वि शत्रू॒न्॒ वि मृधो॑ नुद॒ वि वृ॒त्रस्य॒ हनू॑ रुज। वि म॒न्युमि॑न्द्र भामि॒तो॑\-ऽमित्र॑स्याभि॒दास॑तः॥ त्रा॒तार॒मिन्द्र॑मवि॒तार॒मिन्द्र॒ꣳ॒ हवे॑हवे सु॒हव॒ꣳ॒ शूर॒मिन्द्रम्᳚। हु॒वे नु श॒क्रं पु॑रुहू॒तमिन्द्रꣴ॑ स्व॒स्ति नो॑ म॒घवा॑ धा॒त्विन्द्रः॑॥ मा ते॑ अ॒स्याꣳ~(५०)

%1.6.12.6
स॑हसाव॒न् परि॑ष्टाव॒घाय॑ भूम हरिवः परा॒दै। त्राय॑स्व नो\-ऽवृ॒केभि॒र्वरू॑थै॒स्तव॑ प्रि॒यासः॑ सू॒रिषु॑ स्याम॥ अन॑वस्ते॒ रथ॒मश्वा॑य तक्ष॒न् त्वष्टा॒ वज्रं॑ पुरुहूत द्यु॒मन्तम्᳚। ब्र॒ह्माण॒ इन्द्रं॑ म॒हय॑न्तो अ॒र्कैरव॑र्धय॒न्नह॑ये॒ हन्त॒वा उ॑॥ वृष्णे॒ यत् ते॒ वृष॑णो अ॒र्कमर्चा॒निन्द्र॒ ग्रावा॑णो॒ अदि॑तिः स॒जोषाः᳚। अ॒न॒श्वासो॒ ये प॒वयो॑\-ऽर॒था इन्द्रे॑षिता अ॒भ्यव॑र्तन्त॒ दस्यून्॑॥~(५१)

{\anuvakamend[{वृ॒त्र॒हत्ये\-ऽनु॑ गाय॒त्रिणो᳚\-ऽध्व॒राणां᳚ परा॒वतो॒\-ऽस्याम॒ष्टाच॑त्वारिꣳशच्च}]}%॥12॥

\prashnaend{सं त्वा॑ सिञ्चामि ध्रु॒वो᳚स्य॒ग्निर्मा॑ ब॒र्हिषो॒ऽहमाप्या॑यता॒मग॑न्म॒ यथा॒ वै यो वै श्र॒द्धां प्र॒जाप॑तिर्य॒ज्ञान् ध्रु॒वो॑सीत्या॑ह॒ यो वै स॑प्तद॒शमिन्द्रं॑ वो॒ द्वाद॑श॥१२॥}{सं त्वा॑ ब॒र्हिषो॒ऽहं यथा॒ वा ए॒वं वि॒द्वाञ्छ्रौष॑ट्थ् सहसाव॒न्नेक॑पञ्चा॒शत्॥५१॥}{सं त्वा॑ सिञ्चामि॒ दस्यून्॑॥}%%१-६
{हरिः॑ ॐ}{॥कृष्ण-यजुर्वेदीय-तैत्तिरीय-संहितायां प्रथमकाण्डे षष्ठः प्रश्नः समाप्तः॥१-६॥}
%%% END PRASHNA
