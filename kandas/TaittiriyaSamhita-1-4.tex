\chapt{काण्डम् १}
\sect{चतुर्थः प्रश्नः}\setcounter{anuvakam}{0}
\dnsub{तैत्तिरीयसंहितायां प्रथमकाण्डे चतुर्थः प्रश्नः}
%1.4.1.1
आ द॑दे॒ ग्रावा᳚स्यध्वर॒कृद् दे॒वेभ्यो॑ गम्भी॒रमि॒मम॑ध्व॒रं कृ॑ध्युत्त॒मेन॑ प॒विनेन्द्रा॑य॒ सोम॒ꣳ॒ सुषु॑तं॒ मधु॑मन्तं॒ पय॑स्वन्तं वृष्टि॒वनि॒मिन्द्रा॑य त्वा वृत्र॒घ्न इन्द्रा॑य त्वा वृत्र॒तुर॒ इन्द्रा॑य त्वा\-ऽभिमाति॒घ्न इन्द्रा॑य त्वा\-ऽऽ\-दि॒त्यव॑त॒ इन्द्रा॑य त्वा वि॒श्वदे᳚व्यावते श्वा॒त्राः स्थ॑ वृत्र॒तुरो॒ राधो॑गूर्ता अ॒मृत॑स्य॒ पत्नी॒स्ता दे॑वीर्देव॒त्रेमं य॒ज्ञं ध॒त्तोप॑हूताः॒ सोम॑स्य पिब॒तोप॑हूतो यु॒ष्माक॒ꣳ॒~(१)

%1.4.1.2
सोमः॑ पिबतु॒ यत्ते॑ सोम दि॒वि ज्योति॒र्यत् पृ॑थि॒व्यां यदु॒राव॒न्तरि॑क्षे॒ तेना॒स्मै यज॑मानायो॒रु रा॒या कृ॒ध्यधि॑ दा॒त्रे वो॑चो॒ धिष॑णे वी॒डू स॒ती वी॑डयेथा॒मूर्जं॑ दधाथा॒मूर्जं॑ मे धत्तं॒ मा वाꣳ॑ हिꣳसिषं॒ मा मा॑ हिꣳसिष्टं॒ प्रागपा॒गुद॑गध॒राक्तास्त्वा॒ दिश॒ आ धा॑व॒न्त्वम्ब॒ नि ष्व॑र। यत्ते॑ सो॒मादा᳚भ्यं॒ नाम॒ जागृ॑वि॒ तस्मै॑ ते सोम॒ सोमा॑य॒ स्वाहा᳚॥~(२)

{\anuvakamend[{यु॒ष्माकꣴ॑ स्वर॒ यत्ते॒ नव॑ च}]}%~(१)

%1.4.2.1
वा॒चस्पत॑ये पवस्व वाजि॒न् वृषा॒ वृष्णो॑ अ॒ꣳ॒शुभ्यां॒ गभ॑स्तिपूतो दे॒वो दे॒वानां᳚ प॒वित्र॑मसि॒ येषां᳚ भा॒गो\-ऽसि॒ तेभ्य॑स्त्वा॒ स्वां कृ॑तो\-ऽसि॒ मधु॑मतीर्न॒ इष॑स्कृधि॒ विश्वे᳚भ्यस्त्वेन्द्रि॒येभ्यो॑ दि॒व्येभ्यः॒ पार्थि॑वेभ्यो॒ मन॑स्त्वाष्टू॒र्व॑न्त\-रि॑क्ष॒मन्वि॑हि॒ स्वाहा᳚ त्वा सुभवः॒ सूर्या॑य दे॒वेभ्य॑स्त्वा मरीचि॒पेभ्य॑ ए॒ष ते॒ योनिः॑ प्रा॒णाय॑ त्वा॥~(३)

{\anuvakamend[{वा॒चः स॒प्तच॑त्वारिꣳशत्}]}%~(२)

%1.4.3.1
उ॒प॒या॒मगृ॑हीतो\-ऽस्य॒न्तर्य॑च्छ मघवन् पा॒हि सोम॑मुरु॒ष्य रायः॒ समिषो॑ यजस्वा॒न्तस्ते॑ दधामि॒ द्यावा॑पृथि॒वी अ॒न्तरु॒र्व॑न्तरि॑क्षꣳ स॒जोषा॑ दे॒वैरव॑रैः॒ परै᳚श्चान्तर्या॒मे म॑घवन् मादयस्व॒ स्वां कृ॑तो\-ऽसि॒ मधु॑मतीर्न॒ इष॑स्कृधि॒ विश्वे᳚भ्यस्त्वेन्द्रि॒येभ्यो॑ दि॒व्येभ्यः॒ पार्थि॑वेभ्यो॒ मन॑स्त्वाष्टू॒र्व॑न्त\-रि॑क्ष॒मन्वि॑हि॒ स्वाहा᳚ त्वा सुभवः॒ सूर्या॑य दे॒वेभ्य॑स्त्वा मरीचि॒पेभ्य॑ ए॒ष ते॒ योनि॑रपा॒नाय॑ त्वा॥~(४)

{\anuvakamend[{दे॒वेभ्यः॑ स॒प्त च॑}]}%~(३)

%1.4.4.1
आ वा॑यो भूष शुचिपा॒ उप॑ नः स॒हस्रं॑ ते नि॒युतो॑ विश्ववार। उपो॑ ते॒ अन्धो॒ मद्य॑मयामि॒ यस्य॑ देव दधि॒षे पू᳚र्व॒पेयम्᳚॥ उ॒प॒या॒मगृ॑हीतो\-ऽसि वा॒यवे॒ त्वेन्द्र॑वायू इ॒मे सु॒ताः। उप॒ प्रयो॑भि॒रा ग॑त॒मिन्द॑वो वामु॒शन्ति॒ हि॥ उ॒प॒या॒मगृ॑हीतो\-ऽसीन्द्रवा॒यु\-भ्यां᳚ त्वै॒ष ते॒ योनिः॑ स॒जोषा᳚भ्यां त्वा॥~(५)

{\anuvakamend[{आ वा॑यो॒ त्रिच॑त्वारिꣳशत्}]}%~(४)

%1.4.5.1
अ॒यं वां᳚ मित्रावरुणा सु॒तः सोम॑ ऋतावृधा। ममेदि॒ह श्रु॑त॒ꣳ॒ हवम्᳚। उ॒प॒या॒मगृ॑हीतो\-ऽसि मि॒त्रावरु॑णाभ्यां त्वै॒ष ते॒ योनि॑र् ऋता॒यु\-भ्यां᳚ त्वा॥~(६)

{\anuvakamend[{अ॒यं वां᳚ विꣳश॒तिः}]}%~(५)

%1.4.6.1
या वां॒ कशा॒ मधु॑म॒त्यश्वि॑ना सू॒नृता॑वती। तया॑ य॒ज्ञं मि॑मिक्षतम्। उ॒प॒या॒मगृ॑हीतो\-ऽस्य॒श्वि\-भ्यां᳚ त्वै॒ष ते॒ योनि॒र्माध्वी᳚भ्यां त्वा॥~(७)

{\anuvakamend[{या वा॑म॒ष्टाद॑श}]}%~(६)

%1.4.7.1
प्रा॒त॒र्युजौ॒ वि मु॑च्येथा॒मश्वि॑ना॒वेह ग॑च्छतम्। अ॒स्य सोम॑स्य पी॒तये᳚॥ उ॒प॒या॒मगृ॑हीतो\-ऽस्य॒श्वि\-भ्यां᳚ त्वै॒ष ते॒ योनि॑र॒श्वि\-भ्यां᳚ त्वा॥~(८)

{\anuvakamend[{प्रा॒त॒र्युजा॒वेका॒न्नविꣳ॑शतिः}]}%~(७)

%1.4.8.1
अ॒यं वे॒नश्चो॑दय॒त् पृश्ञि॑गर्भा॒ ज्योति॑र्जरायू॒ रज॑सो वि॒माने᳚। इ॒मम॒पाꣳ स॑ङ्ग॒मे सूर्य॑स्य॒ शिशुं॒ न विप्रा॑ म॒तिभी॑ रिहन्ति॥ उ॒प॒या॒मगृ॑हीतो\-ऽसि॒ शण्डा॑य त्वै॒ष ते॒ योनि॑र्वी॒रतां᳚ पाहि॥~(९)

{\anuvakamend[{अ॒यं वे॒नः पञ्च॑विꣳशतिः}]}%~(८)

%1.4.9.1
तं प्र॒त्नथा॑ पू॒र्वथा॑ वि॒श्वथे॒मथा᳚ ज्ये॒ष्ठता॑तिं बर्\mbox{}हि॒षदꣳ॑ सुव॒र्विदं॑ प्रतीची॒नं वृ॒जनं॑ दोहसे गि॒रा\-ऽऽ\-शुं जय॑न्त॒मनु॒ यासु॒ वर्ध॑से। उ॒प॒या॒मगृ॑हीतो\-ऽसि॒ मर्का॑य त्वै॒ष ते॒ योनिः॑ प्र॒जाः पा॑हि॥~(१०)

{\anuvakamend[{तꣳ षड्विꣳ॑शतिः}]}%~(९)

%1.4.10.1
ये दे॑वा दि॒व्येका॑\-दश॒ स्थ पृ॑थि॒व्यामध्येका॑\-दश॒ स्था\-ऽफ्सु॒षदो॑ महि॒नैका॑\-दश॒ स्थ ते दे॑वा य॒ज्ञमि॒मं जु॑षध्वमुपया॒मगृ॑हीतो\-ऽस्याग्रय॒णो॑\-ऽसि॒ स्वा᳚ग्रयणो॒ जिन्व॑ य॒ज्ञं जिन्व॑ य॒ज्ञप॑तिम॒भि सव॑ना पाहि॒ विष्णु॒स्त्वां पा॑तु॒ विशं॒ त्वं पा॑हीन्द्रि॒येणै॒ष ते॒ योनि॒र्विश्वे᳚भ्यस्त्वा दे॒वेभ्यः॑॥~(११)

{\anuvakamend[{ये दे॑वा॒स्त्रिच॑त्वारिꣳशत्}]}%॥10॥

%1.4.11.1
त्रि॒ꣳ॒शत्त्रय॑श्च ग॒णिनो॑ रु॒जन्तो॒ दिवꣳ॑ रु॒द्राः पृ॑थि॒वीं च॑ सचन्ते। ए॒का॒द॒शासो॑ अफ्सु॒षदः॑ सु॒तꣳ सोमं॑ जुषन्ता॒ꣳ॒ सव॑नाय॒ विश्वे᳚॥ उ॒प॒या॒मगृ॑हीतो\-ऽस्याग्रय॒णो॑\-ऽसि॒ स्वा᳚ग्रयणो॒ जिन्व॑ य॒ज्ञं जिन्व॑ य॒ज्ञप॑तिम॒भि सव॑ना पाहि॒ विष्णु॒स्त्वां पा॑तु॒ विशं॒ त्वं पा॑हीन्द्रि॒येणै॒ष ते॒ योनि॒र्विश्वे᳚भ्यस्त्वा दे॒वेभ्यः॑॥~(१२)

{\anuvakamend[{त्रि॒ꣳ॒शद् द्विच॑त्वारिꣳशत्}]}%॥11॥

%1.4.12.1
उ॒प॒या॒मगृ॑हीतो॒\-ऽसीन्द्रा॑य त्वा बृ॒हद्व॑ते॒ वय॑स्वत उक्था॒युवे॒ यत् त॑ इन्द्र बृ॒हद्वय॒स्तस्मै᳚ त्वा॒ विष्ण॑वे त्वै॒ष ते॒ योनि॒रिन्द्रा॑य त्वोक्था॒युवे᳚॥~(१३)

{\anuvakamend[{उ॒प॒या॒मगृ॑हीतो॒ द्वाविꣳ॑शतिः}]}%॥12॥

%1.4.13.1
मू॒र्धानं॑ दि॒वो अ॑र॒तिं पृ॑थि॒व्या वै᳚श्वान॒रमृ॒ताय॑ जा॒तम॒ग्निम्। क॒विꣳ स॒म्राज॒मति॑थिं॒ जना॑नामा॒सन्ना पात्रं॑ जनयन्त दे॒वाः॥ उ॒प॒या॒मगृ॑हीतो\-ऽस्य॒ग्नये᳚ त्वा वैश्वान॒राय॑ ध्रु॒वो॑\-ऽसि ध्रु॒वक्षि॑तिर्ध्रु॒वाणां᳚ ध्रु॒वत॒मो\-ऽच्यु॑तानामच्युत॒क्षित्त॑म ए॒ष ते॒ योनि॑र॒ग्नये᳚ त्वा वैश्वान॒राय॑॥~(१४)

{\anuvakamend[{मू॒र्धानं॒ पञ्च॑त्रिꣳशत्}]}%॥13॥

%1.4.14.1
मधु॑श्च॒ माध॑वश्च शु॒क्रश्च॒ शुचि॑श्च॒ नभ॑श्च नभ॒स्य॑श्चे॒षश्चो॒र्जश्च॒ सह॑श्च सह॒स्य॑श्च॒ तप॑श्च तप॒स्य॑श्चोपया॒मगृ॑हीतो\-ऽसि स॒ꣳ॒सर्पो᳚\-ऽस्यꣳहस्प॒त्याय॑ त्वा॥~(१५)

{\anuvakamend[{मधु॑स्त्रि॒ꣳ॒शत्}]}%॥14॥

%1.4.15.1
इन्द्रा᳚ग्नी॒ आ ग॑तꣳ सु॒तं गी॒र्भिर्नभो॒ वरे᳚ण्यम्। अ॒स्य पा॑तं धि॒येषि॒ता॥ उ॒प॒या॒मगृ॑हीतो\-ऽसीन्द्रा॒ग्नि\-भ्यां᳚ त्वै॒ष ते॒ योनि॑रिन्द्रा॒ग्नि\-भ्यां᳚ त्वा॥~(१६)

{\anuvakamend[{इन्द्रा᳚ग्नी विꣳश॒तिः}]}%॥15॥

%1.4.16.1
ओमा॑सश्चर्\mbox{}षणीधृतो॒ विश्वे॑ देवास॒ आ ग॑त। दा॒श्वाꣳसो॑ दा॒शुषः॑ सु॒तम्॥ उ॒प॒या॒मगृ॑हीतो\-ऽसि॒ विश्वे᳚भ्यस्त्वा दे॒वेभ्य॑ ए॒ष ते॒ योनि॒र्विश्वे᳚भ्यस्त्वा दे॒वेभ्यः॑॥~(१७)

{\anuvakamend[{इन्द्रा᳚ग्नी॒ ओमा॑सो विꣳश॒तिर्विꣳ॑शतिः}]}%॥16॥

%1.4.17.1
म॒रुत्व॑न्तं वृष॒भं वा॑वृधा॒नमक॑वारिं दि॒व्यꣳ शा॒समिन्द्रम्᳚। वि॒श्वा॒साह॒मव॑से॒ नूत॑नायो॒ग्रꣳ स॑हो॒दामि॒ह तꣳ हु॑वेम॥ उ॒प॒या॒मगृ॑हीतो॒\-ऽसीन्द्रा॑य त्वा म॒रुत्व॑त ए॒ष ते॒ योनि॒रिन्द्रा॑य त्वा म॒रुत्व॑ते॥~(१८)

{\anuvakamend[{म॒रुत्व॑न्त॒ꣳ॒ षड्विꣳ॑शतिः}]}%॥17॥

%1.4.18.1
इन्द्र॑ मरुत्व इ॒ह पा॑हि॒ सोमं॒ यथा॑ शार्या॒ते अपि॑बः सु॒तस्य॑। तव॒ प्रणी॑ती॒ तव॑ शूर॒ शर्म॒न्ना वि॑वासन्ति क॒वयः॑ सुय॒ज्ञाः॥ उ॒प॒या॒मगृ॑हीतो॒\-ऽसीन्द्रा॑य त्वा म॒रुत्व॑त ए॒ष ते॒ योनि॒रिन्द्रा॑य त्वा म॒रुत्व॑ते॥~(१९)

{\anuvakamend[{इन्द्रैका॒न्नत्रि॒ꣳ॒शत्}]}%॥18॥

%1.4.19.1
म॒रुत्वाꣳ॑ इन्द्र वृष॒भो रणा॑य॒ पिबा॒ सोम॑मनुष्व॒धं मदा॑य। आ सि॑ञ्चस्व ज॒ठरे॒ मध्व॑ ऊ॒र्मिं त्वꣳ राजा॑\-ऽसि प्र॒दिवः॑ सु॒ताना᳚म्॥ उ॒प॒या॒मगृ॑हीतो॒\-ऽसीन्द्रा॑य त्वा म॒रुत्व॑त ए॒ष ते॒ योनि॒रिन्द्रा॑य त्वा म॒रुत्व॑ते॥~(२०)

{\anuvakamend[{इन्द्र॑ मरुत्वो म॒रुत्वा॒नेका॒न्न त्रि॒ꣳ॒शदेका॒न्न त्रि॒ꣳ॒शत्}]}%॥19॥

%1.4.20.1
म॒हाꣳ इन्द्रो॒ य ओज॑सा प॒र्जन्यो॑ वृष्टि॒माꣳ इ॑व। स्तोमै᳚र्व॒थ्सस्य॑ वावृधे॥ उ॒प॒या॒मगृ॑हीतो\-ऽसि महे॒न्द्राय॑ त्वै॒ष ते॒ योनि॑र्महे॒न्द्राय॑ त्वा॥~(२१)

{\anuvakamend[{म॒हानेका॒न्नविꣳ॑शतिः}]}%॥20॥

%1.4.21.1
म॒हाꣳ इन्द्रो॑ नृ॒वदा च॑र्\mbox{}षणि॒प्रा उ॒त द्वि॒बर्\mbox{}हा॑ अमि॒नः सहो॑भिः। अ॒स्म॒द्रिय॑ग्वावृधे वी॒र्या॑यो॒रुः पृ॒थुः सुकृ॑तः क॒र्तृभि॑र्भूत्॥ उ॒प॒या॒मगृ॑हीतो\-ऽसि महे॒न्द्राय॑ त्वै॒ष ते॒ योनि॑र्महे॒न्द्राय॑ त्वा॥~(२२)

{\anuvakamend[{म॒हान्नृ॒वत्षड्विꣳ॑शतिः}]}%॥21॥

%1.4.22.1
क॒दा च॒न स्त॒रीर॑सि॒ नेन्द्र॑ सश्चसि दा॒शुषे᳚। उपो॒पेन्नु म॑घव॒न् भूय॒ इन्नु ते॒ दानं॑ दे॒वस्य॑ पृच्यते॥ उ॒प॒या॒मगृ॑हीतो\-ऽस्यादि॒त्येभ्य॑स्त्वा॥ क॒दा च॒न प्र यु॑च्छस्यु॒भे नि पा॑सि॒ जन्म॑नी। तुरी॑यादित्य॒ सव॑नं त इन्द्रि॒यमा त॑स्थाव॒मृतं॑ दि॒वि॥ य॒ज्ञो दे॒वानां॒ प्रत्ये॑ति सु॒म्नमादि॑त्यासो॒ भव॑ता मृड॒यन्तः॑। आ वो॒\-ऽर्वाची॑ सुम॒तिर्व॑वृत्याद॒ꣳ॒होश्चि॒द्या व॑रिवो॒वित्त॒रास॑त्॥ विव॑स्व आदित्यै॒ष ते॑ सोमपी॒थस्तेन॑ मन्दस्व॒ तेन॑ तृप्य तृ॒प्यास्म॑ ते व॒यं त॑र्पयि॒तारो॒ या दि॒व्या वृष्टि॒स्तया᳚ त्वा श्रीणामि॥~(२३)

{\anuvakamend[{वः॒ स॒प्तविꣳ॑शतिश्च}]}%॥2॥

%1.4.23.1
वा॒मम॒द्य स॑वितर्वा॒ममु॒ श्वो दि॒वेदि॑वे वा॒मम॒स्मभ्यꣳ॑ सावीः। वा॒मस्य॒ हि क्षय॑स्य देव॒ भूरे॑र॒या धि॒या वा॑म॒भाजः॑ स्याम॥ उ॒प॒या॒मगृ॑हीतो\-ऽसि दे॒वाय॑ त्वा सवि॒त्रे॥~(२४)

{\anuvakamend[{वा॒मं चतु॑र्विꣳशतिः}]}%॥23॥

%1.4.24.1
अद॑ब्धेभिः सवितः पा॒युभि॒ष्ट्वꣳ शि॒वेभि॑र॒द्य परि॑ पाहि नो॒ गयम्᳚। हिर॑ण्यजिह्वः सुवि॒ताय॒ नव्य॑से॒ रक्षा॒ माकि॑र्नो अ॒घशꣳ॑स ईशत॥ उ॒प॒या॒मगृ॑हीतो\-ऽसि दे॒वाय॑ त्वा सवि॒त्रे॥~(२५)

{\anuvakamend[{अद॑ब्धेभि॒स्त्रयो॑विꣳशतिः}]}%॥24॥

%1.4.25.1
हिर॑ण्यपाणिमू॒तये॑ सवि॒तार॒मुप॑ ह्वये। स चेत्ता॑ दे॒वता॑ प॒दम्॥ उ॒प॒या॒मगृ॑हीतो\-ऽसि दे॒वाय॑ त्वा सवि॒त्रे॥~(२६)

{\anuvakamend[{हिर॑ण्यपाणिं॒ चतु॑र्दश}]}%॥25॥

%1.4.26.1
सु॒शर्मा॑\-ऽसि सुप्रतिष्ठा॒नो बृ॒हदु॒क्षे नम॑ ए॒ष ते॒ योनि॒र्विश्वे᳚भ्यस्त्वा दे॒वेभ्यः॑॥~(२७)

{\anuvakamend[{सु॒शर्मा॒ द्वाद॑श}]}%॥26॥

%1.4.27.1
बृह॒स्पति॑सुतस्य त इन्दो इन्द्रि॒याव॑तः॒ पत्नी॑वन्तं॒ ग्रहं॑ गृह्णा॒म्यग्ना(३)इ पत्नी॒वा(३)ः स॒जूर्दे॒वेन॒ त्वष्ट्रा॒ सोमं॑ पिब॒ स्वाहा᳚॥~(२८)

{\anuvakamend[{बृह॒स्पति॑सुतस्य॒ पञ्च॑दश}]}%॥27॥

%1.4.28.1
हरि॑रसि हारियोज॒नो हर्योः᳚ स्था॒ता वज्र॑स्य भ॒र्ता पृश्ञेः᳚ प्रे॒ता तस्य॑ ते देव सोमे॒ष्टय॑जुषः स्तु॒तस्तो॑मस्य श॒स्तोक्थ॑स्य॒ हरि॑वन्तं॒ ग्रहं॑ गृह्णामि ह॒रीः स्थ॒ हर्यो᳚र्धा॒नाः स॒हसो॑मा॒ इन्द्रा॑य॒ स्वाहा᳚॥~(२९)

{\anuvakamend[{हरिः॒ षड्विꣳ॑शतिः}]}%॥28॥

%1.4.29.1
अग्न॒ आयूꣳ॑षि पवस॒ आ सु॒वोर्ज॒मिषं॑ च नः। आ॒रे बा॑धस्व दु॒च्छुना᳚म्॥ उ॒प॒या॒मगृ॑हीतो\-ऽस्य॒ग्नये᳚ त्वा॒ तेज॑स्वत ए॒ष ते॒ योनि॑र॒ग्नये᳚ त्वा॒ तेज॑स्वते॥~(३०)

{\anuvakamend[{अग्न॒ आयूꣳ॑षि॒ त्रयो॑वि ꣳशतिः}]}%॥29॥

%1.4.30.1
उ॒त्तिष्ठ॒न्नोज॑सा स॒ह पी॒त्वा शिप्रे॑ अवेपयः। सोम॑मिन्द्र च॒मू सु॒तम्॥ उ॒प॒या॒मगृ॑हीतो॒\-ऽसीन्द्रा॑य॒ त्वौज॑स्वत ए॒ष ते॒ योनि॒रिन्द्रा॑य॒ त्वौज॑स्वते॥~(३१)

{\anuvakamend[{उ॒त्तिष्ठ॒न्नेक॑विꣳशतिः}]}%॥30॥

%1.4.31.1
त॒रणि॑र्वि॒श्वद॑र्\mbox{}शतो ज्योति॒ष्कृद॑सि सूर्य। विश्व॒मा भा॑सि रोच॒नम्॥ उ॒प॒या॒मगृ॑हीतो\-ऽसि॒ सूर्या॑य त्वा॒ भ्राज॑स्वत ए॒ष ते॒ योनिः॒ सूर्या॑य त्वा॒ भ्राज॑स्वते॥~(३२)

{\anuvakamend[{त॒रणि॑र्विꣳश॒तिः}]}%॥31॥

%1.4.32.1
आ प्या॑यस्व मदिन्तम॒ सोम॒ विश्वा॑भिरू॒तिभिः॑। भवा॑ नः स॒प्रथ॑स्तमः॥~(३३)

{\anuvakamend[{आ प्या॑यस्व॒ नव॑}]}%॥32॥

%1.4.33.1
ई॒युष्टे ये पूर्व॑तरा॒मप॑श्यन् व्यु॒च्छन्ती॑मु॒षसं॒ मर्त्या॑सः। अ॒स्माभि॑रू॒ नु प्र॑ति॒चक्ष्या॑\-ऽभू॒दो ते य॑न्ति॒ ये अ॑प॒रीषु॒ पश्यान्॑॥~(३४)

{\anuvakamend[{ई॒युरेका॒न्नविꣳ॑शतिः}]}%॥33॥

%1.4.34.1
ज्योति॑ष्मतीं त्वा सादयामि ज्योति॒ष्कृतं॑ त्वा सादयामि ज्योति॒र्विदं॑ त्वा सादयामि॒ भास्व॑तीं त्वा सादयामि॒ ज्वल॑न्तीं त्वा सादयामि मल्मला॒भव॑न्तीं त्वा सादयामि॒ दीप्य॑मानां त्वा सादयामि॒ रोच॑मानां त्वा सादया॒म्यज॑स्रां त्वा सादयामि बृ॒हज्ज्यो॑तिषं त्वा सादयामि बो॒धय॑न्तीं त्वा सादयामि॒ जाग्र॑तीं त्वा सादयामि॥~(३५)

{\anuvakamend[{ज्योति॑ष्मती॒ꣳ॒ षट्त्रिꣳ॑शत्}]}%॥34॥

%1.4.35.1
प्र॒या॒साय॒ स्वाहा॑\-ऽऽ\-या॒साय॒ स्वाहा॑ विया॒साय॒ स्वाहा॑ संया॒साय॒ स्वाहो᳚द्या॒साय॒ स्वाहा॑\-ऽवया॒साय॒ स्वाहा॑ शु॒चे स्वाहा॒ शोका॑य॒ स्वाहा॑ तप्य॒त्वै स्वाहा॒ तप॑ते॒ स्वाहा᳚ ब्रह्मह॒त्यायै॒ स्वाहा॒ सर्व॑स्मै॒ स्वाहा᳚॥~(३६)

{\anuvakamend[{प्र॒या॒साय॒ चतु॑र्विꣳशतिः}]}%॥35॥

%1.4.36.1
चि॒त्तꣳ स॑न्ता॒नेन॑ भ॒वं य॒क्ना रु॒द्रं तनि॑म्ना पशु॒पतिꣴ॑ स्थूलहृद॒येना॒ग्निꣳ हृद॑येन रु॒द्रं लोहि॑तेन श॒र्वं मत॑स्नाभ्यां महादे॒व\-म॒न्तःपा᳚र्श्वेनौषिष्ठ॒\-हनꣳ॑ शिङ्गीनिको॒श्या᳚भ्याम्॥~(३७)

{\anuvakamend[{चि॒त्तम॒ष्टाद॑श}]}%॥36॥

%1.4.37.1
आ ति॑ष्ठ वृत्रह॒न् रथं॑ यु॒क्ता ते॒ ब्रह्म॑णा॒ हरी᳚। अ॒र्वा॒चीन॒ꣳ॒ सु ते॒ मनो॒ ग्रावा॑ कृणोतु व॒ग्नुना᳚॥ उ॒प॒या॒मगृ॑हीतो॒\-ऽसीन्द्रा॑य त्वा षोड॒शिन॑ ए॒ष ते॒ योनि॒रिन्द्रा॑य त्वा षोड॒शिने᳚॥~(३८)

{\anuvakamend[{आ ति॑ष्ठ॒ षड्विꣳ॑शतिः}]}%॥37॥

%1.4.38.1
इन्द्र॒मिद्धरी॑ वह॒तो\-ऽप्र॑तिधृष्टशवस॒मृषी॑णां च स्तु॒तीरुप॑ य॒ज्ञं च॒ मानु॑षाणाम्॥ उ॒प॒या॒मगृ॑हीतो॒\-ऽसीन्द्रा॑य त्वा षोड॒शिन॑ ए॒ष ते॒ योनि॒रिन्द्रा॑य त्वा षोड॒शिने᳚॥~(३९)

{\anuvakamend[{इन्द्र॒मित्त्रयो॑विꣳशतिः}]}%॥38॥

%1.4.39.1
असा॑वि॒ सोम॑ इन्द्र ते॒ शवि॑ष्ठ धृष्ण॒वा ग॑हि। आ त्वा॑ पृणक्त्विन्द्रि॒यꣳ रजः॒ सूर्यं॒ न र॒श्मिभिः॑॥ उ॒प॒या॒मगृ॑हीतो॒\-ऽसीन्द्रा॑य त्वा षोड॒शिन॑ ए॒ष ते॒ योनि॒रिन्द्रा॑य त्वा षोड॒शिने᳚॥~(४०)

{\anuvakamend[{असा॑वि स॒प्तविꣳ॑शतिः}]}%॥39॥

%1.4.40.1
सर्व॑स्य प्रति॒शीव॑री॒ भूमि॑स्त्वो॒पस्थ॒ आ\-ऽधि॑त। स्यो॒ना\-ऽस्मै॑ सु॒षदा॑ भव॒ यच्छा᳚स्मै॒ शर्म॑ स॒प्रथाः᳚॥ उ॒प॒या॒मगृ॑हीतो॒\-ऽसीन्द्रा॑य त्वा षोड॒शिन॑ ए॒ष ते॒ योनि॒रिन्द्रा॑य त्वा षोड॒शिने᳚॥~(४१)

{\anuvakamend[{सर्व॑स्य॒ षड्विꣳ॑शतिः}]}%॥40॥

%1.4.41.1
म॒हाꣳ इन्द्रो॒ वज्र॑बाहुः षोड॒शी शर्म॑ यच्छतु। स्व॒स्ति नो॑ म॒घवा॑ करोतु॒ हन्तु॑ पा॒प्मानं॒ यो᳚\-ऽस्मान् द्वेष्टि॑॥ उ॒प॒या॒मगृ॑हीतो॒\-ऽसीन्द्रा॑य त्वा षोड॒शिन॑ ए॒ष ते॒ योनि॒रिन्द्रा॑य त्वा षोड॒शिने᳚॥~(४२)

{\anuvakamend[{सर्व॑स्य म॒हान्थ्षड्विꣳ॑शतिः॒ षड्विꣳ॑शतिः}]}%॥41॥

%1.4.42.1
स॒जोषा॑ इन्द्र॒ सग॑णो म॒रुद्भिः॒ सोमं॑ पिब वृत्रहञ्छूर वि॒द्वान्। ज॒हि शत्रू॒ꣳ॒ रप॒ मृधो॑ नुद॒स्वा\-ऽथाभ॑यं कृणुहि वि॒श्वतो॑ नः॥ उ॒प॒या॒मगृ॑हीतो॒\-ऽसीन्द्रा॑य त्वा षोड॒शिन॑ ए॒ष ते॒ योनि॒रिन्द्रा॑य त्वा षोड॒शिने᳚॥~(४३)

{\anuvakamend[{स॒जोषा᳚स्त्रि॒ꣳ॒शत्}]}%॥42॥

%1.4.43.1
उदु॒ त्यं जा॒तवे॑दसं दे॒वं व॑हन्ति के॒तवः॑। दृ॒शे विश्वा॑य॒ सूर्यम्᳚॥ चि॒त्रं दे॒वाना॒मुद॑गा॒दनी॑कं॒ चक्षु॑र्मि॒त्रस्य॒ वरु॑णस्या॒ग्नेः। आ\-ऽप्रा॒ द्यावा॑पृथि॒वी अ॒न्तरि॑क्ष॒ꣳ॒ सूर्य॑ आ॒त्मा जग॑तस्त॒स्थुष॑श्च॥ अग्ने॒ नय॑ सु॒पथा॑ रा॒ये अ॒स्मान् विश्वा॑नि देव व॒युना॑नि वि॒द्वान्। यु॒यो॒ध्य॑स्मज्जु॑हुरा॒णमेनो॒ भूयि॑ष्ठां ते॒ नम॑ उक्तिं विधेम॥ दिवं॑ गच्छ॒ सुवः॑ पत रू॒पेण॑~(४४)

%1.4.43.2
वो रू॒पम॒भ्यैमि॒ वय॑सा॒ वयः॑। तु॒थो वो॑ वि॒श्ववे॑दा॒ वि भ॑जतु॒ वर्\mbox{}षि॑ष्ठे॒ अधि॒ नाके᳚॥ ए॒तत् ते॑ अग्ने॒ राध॒ ऐति॒ सोम॑च्युतं॒ तन्मि॒त्रस्य॑ प॒था न॑य॒र्तस्य॑ प॒था प्रेत॑ च॒न्द्रद॑क्षिणा य॒ज्ञस्य॑ प॒था सु॑वि॒ता नय॑न्तीर्ब्राह्म॒णम॒द्य रा᳚ध्यास॒मृषि॑मार्\mbox{}षे॒यं पि॑तृ॒मन्तं॑ पैतृम॒त्यꣳ सु॒धातु॑दक्षिणं॒ वि सुवः॒ पश्य॒ व्य॑न्तरि॑क्षं॒ यत॑स्व सद॒स्यै॑र॒स्मद्दा᳚त्रा देव॒त्रा ग॑च्छत॒ मधु॑मतीः प्रदा॒तार॒मा वि॑श॒तान॑वहाया॒स्मान् दे॑व॒याने॑न प॒थेत॑ सु॒कृतां᳚ लो॒के सी॑दत॒ तन्नः॑ सꣴस्कृ॒तम्॥~(४५)

{\anuvakamend[{रू॒पेण॑ सद॒स्यै॑र॒ष्टाद॑श च}]}%॥43~(३७)॥

%1.4.44.1
धा॒ता रा॒तिः स॑वि॒तेदं जु॑षन्तां प्र॒जा\-प॑तिर्निधि॒पति॑र्नो अ॒ग्निः। त्वष्टा॒ विष्णुः॑ प्र॒जया॑ सꣳररा॒णो यज॑मानाय॒ द्रवि॑णं दधातु॥ समि॑न्द्र णो॒ मन॑सा नेषि॒ गोभिः॒ सꣳ सू॒रिभि॑र्मघव॒न्थ्सꣴ स्व॒स्त्या। सं ब्रह्म॑णा दे॒वकृ॑तं॒ यदस्ति॒ सं दे॒वानाꣳ॑ सुम॒त्या य॒ज्ञिया॑नाम्॥ सं वर्च॑सा॒ पय॑सा॒ सं त॒नूभि॒रग॑न्महि॒ मन॑सा॒ सꣳ शि॒वेन॑। त्वष्टा॑ नो॒ अत्र॒ वरि॑वः कृणो॒-~(४६)

%1.4.44.2
त्वनु॑ मार्ष्टु त॒नुवो॒ यद्विलि॑ष्टम्॥ यद॒द्य त्वा᳚ प्रय॒ति य॒ज्ञे अ॒स्मिन्नग्ने॒ होता॑र॒मवृ॑णीमही॒ह। ऋध॑गया॒डृध॑गु॒ताश॑मिष्ठाः प्रजा॒नन् य॒ज्ञमुप॑याहि वि॒द्वान्॥ स्व॒गा वो॑ देवाः॒ सद॑नमकर्म॒ य आ॑ज॒ग्म सव॑ने॒दं जु॑षा॒णाः। ज॒क्षि॒वाꣳसः॑ पपि॒वाꣳस॑श्च॒ विश्वे॒\-ऽस्मे ध॑त्त वसवो॒ वसू॑नि॥ याना\-ऽव॑ह उश॒तो दे॑व दे॒वान्तान्~(४७)

%1.4.44.3
प्रेर॑य॒ स्वे अ॑ग्ने स॒धस्थे᳚। वह॑माना॒ भर॑माणा ह॒वीꣳषि॒ वसुं॑ घ॒र्मं दिव॒मा ति॑ष्ठ॒तानु॑॥ यज्ञ॑ य॒ज्ञं ग॑च्छ य॒ज्ञप॑तिं गच्छ॒ स्वां योनिं॑ गच्छ॒ स्वाहै॒ष ते॑ य॒ज्ञो य॑ज्ञपते स॒हसू᳚क्तवाकः सु॒वीरः॒ स्वाहा॒ देवा॑ गातुविदो गा॒तुं वि॒त्वा गा॒तुमि॑त॒ मन॑सस्पत इ॒मं नो॑ देव दे॒वेषु॑ य॒ज्ञꣴ स्वाहा॑ वा॒चि स्वाहा॒ वाते॑ धाः॥~(४८)

{\anuvakamend[{कृ॒णो॒तु॒ तान॒ष्टाच॑त्वारिꣳशच्च}]}%॥44~(३८)॥

%1.4.45.1
उ॒रुꣳ हि राजा॒ वरु॑णश्च॒कार॒ सूर्या॑य॒ पन्था॒मन्वे॑त॒वा उ॑। अ॒पदे॒ पादा॒ प्रति॑धातवे\-ऽकरु॒ताप॑व॒क्ता हृ॑दया॒विध॑श्चित्॥ श॒तं ते॑ राजन् भि॒षजः॑ स॒हस्र॑मु॒र्वी ग॑म्भी॒रा सु॑म॒तिष्टे॑ अस्तु। बाध॑स्व॒ द्वेषो॒ निर्\mbox{}ऋ॑तिं परा॒चैः कृ॒तं चि॒देनः॒ प्र मु॑मुग्ध्य॒स्मत्॥ अ॒भिष्ठि॑तो॒ वरु॑णस्य॒ पाशो॒\-ऽग्नेरनी॑कम॒प आ वि॑वेश। अपां᳚ नपात् प्रति॒रक्ष॑न्नसु॒र्यं॑ दमे॑दमे~(४९)

%1.4.45.2
स॒मिधं॑ यक्ष्यग्ने॥ प्रति॑ ते जि॒ह्वा घृ॒तमुच्च॑रण्येथ्समु॒द्रे ते॒ हृद॑यम॒फ्स्व॑न्तः। सं त्वा॑ विश॒न्त्वोष॑धीरु॒ता\-ऽऽ\-पो॑ य॒ज्ञस्य॑ त्वा यज्ञपते ह॒विर्भिः॑॥ सू॒क्त॒वा॒के न॑मोवा॒के वि॑धे॒माव॑भृथ निचङ्कुण निचे॒रुर॑सि निचङ्कु॒णाव॑ दे॒वैर्दे॒वकृ॑त॒मेनो॑\-ऽया॒डव॒ मर्त्यै॒र्मर्त्य॑कृतमु॒रोरा नो॑ देव रि॒षस्पा॑हि सुमि॒त्रा न॒ आप॒ ओष॑धयः~(५०)

%1.4.45.3
सन्तु दुर्मि॒त्रास्तस्मै॑ भूयासु॒र्यो᳚\-ऽस्मान् द्वेष्टि॒ यं च॑ व॒यं द्वि॒ष्मो देवी॑राप ए॒ष वो॒ गर्भ॒स्तं वः॒ सुप्री॑त॒ꣳ॒ सुभृ॑तमकर्म दे॒वेषु॑ नः सु॒कृतो᳚ ब्रूता॒त् प्रति॑युतो॒ वरु॑णस्य॒ पाशः॒ प्रत्य॑स्तो॒ वरु॑णस्य॒ पाश॒ एधो᳚\-ऽस्येधिषी॒महि॑ स॒मिद॑सि॒ तेजो॑\-ऽसि॒ तेजो॒ मयि॑ धेह्य॒पो अन्व॑चारिष॒ꣳ॒ रसे॑न॒ सम॑सृक्ष्महि। पय॑स्वाꣳ अग्न॒ आ\-ऽग॑मं॒ तं मा॒ सꣳ सृ॑ज॒ वर्च॑सा॥~(५१)

{\anuvakamend[{दमे॑दम॒ ओष॑धय॒ आ षट् च॑}]}%॥45~(३९)॥

%1.4.46.1
यस्त्वा॑ हृ॒दा की॒रिणा॒ मन्य॑मा॒नो\-ऽम॑र्त्यं॒ मर्त्यो॒ जोह॑वीमि। जात॑वेदो॒ यशो॑ अ॒स्मासु॑ धेहि प्र॒जाभि॑रग्ने अमृत॒त्वम॑श्याम्॥ यस्मै॒ त्वꣳ सु॒कृते॑ जातवेद॒ उ लो॒कम॑ग्ने कृ॒णवः॑ स्यो॒नम्। अ॒श्विन॒ꣳ॒ स पु॒त्रिणं॑ वी॒रव॑न्तं॒ गोम॑न्तꣳ र॒यिं न॑शते स्व॒स्ति॥ त्वे सु पु॑त्र शव॒सो\-ऽवृ॑त्र॒न् काम॑कातयः। न त्वामि॒न्द्राति॑ रिच्यते॥ उ॒क्थउ॑क्थे॒ सोम॒ इन्द्रं॑ ममाद नी॒थेनी॑थे म॒घवा॑नꣳ~(५२)

%1.4.46.2
सु॒तासः॑। यदीꣳ॑ स॒बाधः॑ पि॒तरं॒ न पु॒त्राः स॑मा॒नद॑क्षा॒ अव॑से॒ हव॑न्ते॥ अग्ने॒ रसे॑न॒ तेज॑सा॒ जात॑वेदो॒ वि रो॑चसे। र॒क्षो॒हा\-ऽमी॑व॒चात॑नः॥ अ॒पो अन्व॑चारिष॒ꣳ॒ रसे॑न॒ सम॑सृक्ष्महि। पय॑स्वाꣳ अग्न॒ आ\-ऽग॑मं॒ तं मा॒ सꣳ सृ॑ज॒ वर्च॑सा॥ वसु॒र्वसु॑पति॒र्॒\mbox{}हिक॒मस्य॑ग्ने वि॒भाव॑सुः। स्याम॑ ते सुम॒तावपि॑॥ त्वाम॑ग्ने॒ वसु॑पतिं॒ वसू॑नाम॒भि प्र म॑न्दे~(५३)

%1.4.46.3
अध्व॒रेषु॑ राजन्न्। त्वया॒ वाजं॑ वाज॒यन्तो॑ जयेमा॒भि ष्या॑म पृथ्सु॒तीर्मर्त्या॑नाम्। त्वाम॑ग्ने वाज॒सात॑मं॒ विप्रा॑ वर्धन्ति॒ सुष्टु॑तम्। स नो॑ रास्व सु॒वीर्यम्᳚॥ अ॒यं नो॑ अ॒ग्निर्वरि॑वः कृणोत्व॒यं मृधः॑ पु॒र ए॑तु प्रभि॒न्दन्न्। अ॒यꣳ शत्रू᳚ञ्जयतु॒ जर्\mbox{}हृ॑षाणो॒\-ऽयं वाजं॑ जयतु॒ वाज॑सातौ॥ अ॒ग्निना॒\-ऽग्निः समि॑ध्यते क॒विर्गृ॒हप॑ति॒र्युवा᳚। ह॒व्य॒वाड् जु॒ह्वा᳚स्यः॥ त्वꣴ ह्य॑ग्ने अ॒ग्निना॒ विप्रो॒ विप्रे॑ण॒ सन्थ्स॒ता। सखा॒ सख्या॑ समि॒ध्यसे᳚॥ उद॑ग्ने॒ शुच॑य॒स्तव॒ वि ज्योति॑षा॥~(५४)

{\anuvakamend[{म॒घवा॑नं मन्दे॒ ह्य॑ग्ने॒ चतु॑र्दश च}]}%॥46॥

{\prashnaend[{वा॒चः प्रा॒णाय॑ त्वा। उ॒प॒या॒म गृ॑हीतोस्यपा॒नाय॑ त्वा। आ वा॑यो वा॒यवे॑ स॒जोषा᳚भ्यां त्वा। अ॒यमृ॑ता॒युभ्यां᳚ त्वा। या वा॑म॒श्विभ्यां॒ माध्वी᳚भ्यां त्वा। प्रा॒त॒र्युजा॑व॒श्विभ्यां᳚ त्वा। अ॒यं वे॒नः शण्डा॑य त्वै॒ष ते॒ योनि॑र्वी॒रतां᳚ पाहि। तं मर्का॑य त्वै॒ष ते॒ योनिः॑ प्र॒जाः पा॑हि। ये दे॑वास्त्रि॒ꣳ॒शदा᳚ग्रय॒णो॑सि॒ विश्वे᳚भ्यस्त्वा दे॒वेभ्यः॑। उ॒प॒या॒म गृ॑हीतो॒सीन्द्रा॑य त्वोक्था॒युवे᳚। मू॒र्धान॑म॒ग्नये᳚ त्वा वैश्वान॒राय॑। मधु॑श्च स॒ꣳ॒ सर्पो॑सि। इन्द्रा᳚ग्नी इन्द्रा॒ग्निभ्यां᳚ त्वा। ओमा॑सो॒ विश्वे᳚भ्यस्त्वा दे॒वेभ्यः॑। म॒रुत्वं॑तं॒ त्रीणीन्द्रा॑य त्वा म॒रुत्व॑ते। म॒हान् द्वे महें॒द्राय॑ त्वा। क॒दाच॒नादि॒त्येभ्य॑स्त्वा। क॒दाच॒नस्त॒रीर्विव॑स्व आदित्य। इन्द्र॒ꣳ॒ शुचि॑र॒पः। वा॒मं त्रीं दे॒वाय॑ त्वा सवि॒त्रे। सु॒शर्मा॒ विश्वे᳚भ्यस्त्वा दे॒वेभ्यः॑। बृह॒स्पति॒स्त्वष्ट्रा॒ सोमं॑ पिब॒ स्वाहा᳚। हरि॑रसि स॒ह सो॑मा॒ इन्द्रा॑य॒ स्वाहा᳚। अग्न॒ आयूग्॑ष्य॒ग्नये᳚ त्वा॒ तेज॑स्वते। उ॒त्तिष्ठ॒न्निन्द्रा॑य॒ त्वौज॑स्वते। त॒रणिः॒ सूर्या॑य त्वा॒ भ्राज॑स्वते। आ ति॑ष्ठाद्याः॒ षडिन्द्रा॑य त्वा षोड॒शिने᳚। उदु॒ त्यं चि॒त्रम्। अग्ने॒नय॒ दिवं॑ गच्छ। उ॒रुमायु॑ष्टे॒ यद्दे॑वा मुमुग्धि। अग्ना॑विष्णू मुमुक्तम्। परा॒ वै पं॒क्त्यः॑। दे॒वा वै ये दे॒वाः पं॒क्त्यौ᳚। परा॒ वै सवाचम्᳚। दे॒वा॒सु॒राः का॒र्यम्᳚। भूमि॒र्व्य॑तृष्यन्। प्र॒जाप॑ति॒र्व्य॑क्षुध्यन्। भूमि॑रादि॒त्या वै। अ॒ग्नि॒ हो॒त्रमा॑दि॒त्यो वै। भूमि॒र्लेकः॒ सले॑कः सु॒लेकः॑। विष्णो॒रुदु॑त्त॒मम्। अन्न॑पते॒ पुन॑स्त्वाऽदि॒त्याः। उ॒रुꣳ सꣳ सृ॑ज॒ वर्च॑सा। यस्त्वा॒ सुष्टु॑तम्। त्वम॑ग्नेयु॒क्ष्वाहि सु॑ष्टु॒तिम्। त्वम॑ग्ने॒ विच॑र्षणे। यस्त्वा॒ विरो॑चसे॥\\आ द॑दे वा॒चस्पत॑य उपया॒मगृ॑हीतो॒स्या वा॑यो अ॒यं वां॒ या वां᳚ प्रात॒र्युजा॑व॒यं तं ये दे॑वास्त्रि॒ꣳ॒शदु॑पया॒म गृ॑हीतोसि मू॒र्धानं॒ मधु॒श्चेन्द्रा᳚ग्नी॒ ओमा॒सो म॒रुत्वं॑त॒मिन्द्र॑ मरुत्वो म॒रुत्वा᳚न्म॒हान्म॒हान्नृ॒वत्क॒दा वा॒ममद॑ब्धेभि॒र्॒ हिर॑ण्यपाणिꣳ सु॒शर्मा॒ बृह॒स्पति॑सुतस्य॒ हरि॑र॒स्यग्न॑ उ॒त्तिष्ठ॑न्त॒रणि॒रा प्या॑यस्वे॒युष्टे ये ज्योति॑ष्मतीं प्रया॒साय॑ चि॒त्तमा ति॒ष्ठेन्द्र॒मसा॑वि॒ सर्व॑स्य म॒हान्थ्स॒जोषा॒ उदु॒ त्यं धा॒तोरुꣳ हि यस्त्वा॒ षट्च॑त्वारिꣳशत्॥४६॥ आ द॑दे॒ ये दे॑वा म॒हानु॒त्तिष्ठ॒न्थ्सर्व॑स्य सन् दुर्मि॒त्राश्चतुः॑ पञ्चा॒शत्॥५०॥ आ द॑दे॒ तव॒ वि ज्योति॑षा॥}]}%%१-४

\centerline{॥हरिः॑ ॐ॥}

\centerline{॥कृष्ण-यजुर्वेदीय-तैत्तिरीय-संहितायां प्रथमकाण्डे चतुर्थः प्रश्नः समाप्तः॥१-४॥}
%%% END PRASHNA
