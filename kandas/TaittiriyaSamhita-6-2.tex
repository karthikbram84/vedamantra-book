\chapt{काण्डम् ६}
\sect{द्वितीयः प्रश्नः}\setcounter{anuvakam}{0}
\dnsub{तैत्तिरीयसंहितायां षष्ठमकाण्डे द्वितीयः प्रश्नः}
%6.2.1.1
यदु॒भौ वि॒मुच्या॑ति॒थ्यं गृ॑ह्णी॒याद्य॒ज्ञं विच्छि॑न्द्या॒द्यदु॒भाववि॑मुच्य॒ यथाना॑गतायाति॒थ्यं क्रि॒यते॑ ता॒दृगे॒व तद्विमु॑क्तो॒\-ऽन्यो॑\-ऽन॒ड्वान्भव॒त्यवि॑मुक्तो॒\-ऽन्यो\-ऽथा॑ति॒थ्यं गृ॑ह्णाति य॒ज्ञस्य॒ सन्त॑त्यै॒ पत्न्य॒न्वार॑भते॒ पत्नी॒ हि पारी॑णह्य॒स्येशे॒ पत्नि॑यै॒वानु॑मतं॒ निर्व॑पति॒ यद्वै पत्नी॑ य॒ज्ञस्य॑ क॒रोति॑ मिथु॒नं तदथो॒ पत्नि॑या ए॒व~(१)

%6.2.1.2
ए॒ष य॒ज्ञस्या᳚न्वार॒म्भो\-ऽन॑वच्छित्त्यै॒ याव॑द्भि॒र्वै राजा॑नुच॒रैरा॒गच्छ॑ति॒ सर्वे᳚भ्यो॒ वै तेभ्य॑ आति॒थ्यं क्रि॑यते॒ छन्दाꣳ॑सि॒ खलु॒ वै सोम॑स्य॒ राज्ञो॑\-ऽनुच॒राण्य॒ग्नेरा॑ति॒थ्यम॑सि॒ विष्ण॑वे॒ त्वेत्या॑ह गायत्रि॒या ए॒वैतेन॑ करोति॒ सोम॑स्याति॒थ्यम॑सि॒ विष्ण॑वे॒ त्वेत्या॑ह त्रि॒ष्टुभ॑ ए॒वैतेन॑ करो॒त्यति॑थेराति॒थ्यम॑सि॒ विष्ण॑वे॒ त्वेत्या॑ह॒ जग॑त्यै~(२)

%6.2.1.3
ए॒वैतेन॑ करोत्य॒ग्नये᳚ त्वा रायस्पोष॒दाव्ने॒ विष्ण॑वे॒ त्वेत्या॑हानु॒ष्टुभ॑ ए॒वैतेन॑ करोति श्ये॒नाय॑ त्वा सोम॒भृते॒ विष्ण॑वे॒ त्वेत्या॑ह गायत्रि॒या ए॒वैतेन॑ करोति॒ पञ्च॒ कृत्वो॑ गृह्णाति॒ पञ्चा᳚क्षरा प॒ङ्क्तिः पाङ्क्तो॑ य॒ज्ञो य॒ज्ञमे॒वाव॑ रुन्धे ब्रह्मवा॒दिनो॑ वदन्ति॒ कस्मा᳚थ्स॒त्याद्गा॑यत्रि॒या उ॑भ॒यत॑ आति॒थ्यस्य॑ क्रियत॒ इति॒ यदे॒वादः सोम॒मा~(३)

%6.2.1.4
अह॑र॒त्तस्मा᳚द्गायत्रि॒या उ॑भ॒यत॑ आति॒थ्यस्य॑ क्रियते पु॒रस्ता᳚च्चो॒परि॑ष्टाच्च॒ शिरो॒ वा ए॒तद्य॒ज्ञस्य॒ यदा॑ति॒थ्यं नव॑कपालः पुरो॒डाशो॑ भवति॒ तस्मा᳚न्नव॒धा शिरो॒ विष्यू॑त॒न्नव॑कपालः पुरो॒डाशो॑ भवति॒ ते त्रय॑स्त्रिकपा॒लास्त्रि॒वृता॒ स्तोमे॑न॒ सम्मि॑ता॒स्तेज॑स्त्रि॒वृत्तेज॑ ए॒व य॒ज्ञस्य॑ शी॒र्॒\mbox{}षं द॑धाति॒ नव॑कपालः पुरो॒डाशो॑ भवति॒ ते त्रय॑स्त्रिकपा॒लास्त्रि॒वृता᳚ प्रा॒णेन॒ सम्मि॑तास्त्रि॒वृद्वै~(४)

%6.2.1.5
प्रा॒णस्त्रि॒वृत॑मे॒व प्रा॒णम॑भिपू॒र्वं य॒ज्ञस्य॑ शी॒र्॒\mbox{}षं द॑धाति प्र॒जा\-प॑ते॒र्वा ए॒तानि॒ पक्ष्मा॑णि॒ यद॑श्ववा॒ला ऐ᳚क्ष॒वी ति॒रश्ची॒ यदाश्व॑वालः प्रस्त॒रो भव॑त्यैक्ष॒वी ति॒रश्ची᳚ प्र॒जा\-प॑तेरे॒व तच्चक्षुः॒ सम्भ॑रति दे॒वा वै या आहु॑ती॒रजु॑हवु॒स्ता असु॑रा नि॒ष्काव॑माद॒न्ते दे॒वाः का᳚र्\mbox{}ष्म॒र्य॑मपश्यन्कर्म॒ण्यो॑ वै कर्मै॑नेन कुर्वी॒तेति॒ ते का᳚र्ष्मर्य॒मया᳚न्परि॒धीन्~(५)

%6.2.1.6
अ॒कु॒र्व॒त॒ तैर्वै ते रक्षा॒ꣴ॒स्यपा᳚घ्नत॒ यत्का᳚र्ष्मर्य॒मयाः᳚ परि॒धयो॒ भव॑न्ति॒ रक्ष॑सा॒मप॑हत्यै॒ सꣴस्प॑र्\mbox{}शयति॒ रक्ष॑सा॒मन॑न्ववचाराय॒ न पु॒रस्ता॒त्परि॑ दधात्यादि॒त्यो ह्ये॑वोद्यन्पु॒रस्ता॒द्रक्षाꣴ॑स्यप॒हन्त्यू॒र्ध्वे स॒मिधा॒वा द॑धात्यु॒परि॑ष्टादे॒व रक्षा॒ꣴ॒स्यप॑हन्ति॒ यजु॑षा॒न्यां तू॒ष्णीम॒न्यां मि॑थुन॒त्वाय॒ द्वे आ द॑धाति द्वि॒पाद्यज॑मानः॒ प्रति॑ष्ठित्यै ब्रह्मवा॒दिनो॑ वदन्ति~(६)

%6.2.1.7
अ॒ग्निश्च॒ वा ए॒तौ सोम॑श्च क॒था सोमा॑याति॒थ्यं क्रि॒यते॒ नाग्नय॒ इति॒ यद॒ग्नाव॒ग्निं म॑थि॒त्वा प्र॒हर॑ति॒ तेनै॒वाग्नय॑ आति॒थ्यं क्रि॑य॒ते\-ऽथो॒ खल्वा॑हुर॒ग्निः सर्वा॑ दे॒वता॒ इति॒ यद्ध॒विरा॒साद्या॒ग्निं मन्थ॑ति ह॒व्यायै॒वास॑न्नाय॒ सर्वा॑ दे॒वता॑ जनयति॥~(७)

{\anuvakamend[{पत्नि॑या ए॒व जग॑त्या॒ आ त्रि॒वृद्वै प॑रि॒धीन् व॑द॒न्त्येक॑चत्वारिꣳशच्च}]}%~(१)

%6.2.2.1
दे॒वा॒सु॒राः संय॑त्ता आस॒न्ते दे॒वा मि॒थो विप्रि॑या आस॒न्ते\-ऽ  न्यो᳚न्यस्मै॒ ज्यैष्ठ्या॒याति॑ष्ठमानाः पञ्च॒धा व्य॑क्रामन्न॒ग्नि\-र्वसु॑भिः॒ सोमो॑ रु॒द्रैरिन्द्रो॑ म॒रुद्भि॒र्वरु॑ण आदि॒त्यैर्बृह॒स्पति॒र्विश्वै᳚र्दे॒वैस्ते॑\-ऽमन्य॒न्तासु॑रेभ्यो॒ वा इ॒दं भ्रातृ॑व्येभ्यो रध्यामो॒ यन्मि॒थो विप्रि॑याः॒ स्मो या न॑ इ॒माः प्रि॒यास्त॒नुव॒स्ताः स॒मव॑द्यामहै॒ ताभ्यः॒ स निर्\mbox{}ऋ॑च्छा॒द्यः~(८)

%6.2.2.2
नः॒ प्र॒थ॒मो\-ऽ  न्यो᳚न्यस्मै॒ द्रुह्या॒दिति॒ तस्मा॒द्यः सता॑नूनप्त्रिणां प्रथ॒मो द्रुह्य॑ति॒ स आर्ति॒मार्च्छ॑ति॒ यत्ता॑नून॒प्त्रꣳ स॑मव॒द्यति॒ भ्रातृ॑व्याभिभूत्यै॒ भव॑त्या॒त्मना॒ परा᳚स्य॒ भ्रातृ॑व्यो भवति॒ पञ्च॒ कृत्वो\-ऽव॑ द्यति पञ्च॒धा हि ते तथ्स॑म॒वाद्य॒न्ताथो॒ पञ्चा᳚क्षरा प॒ङ्क्तिः पाङ्क्तो॑ य॒ज्ञो य॒ज्ञमे॒वाव॑ रुन्ध॒ आप॑तये त्वा गृह्णा॒मीत्या॑ह प्रा॒णो वै~(९)

%6.2.2.3
आप॑तिः प्रा॒णमे॒व प्री॑णाति॒ परि॑पतय॒ इत्या॑ह॒ मनो॒ वै परि॑पति॒र्मन॑ ए॒व प्री॑णाति॒ तनू॒नप्त्र॒ इत्या॑ह त॒नुवो॒ हि ते ताः स॑म॒वाद्य॑न्त शाक्व॒रायेत्या॑ह॒ शक्त्यै॒ हि ते ताः स॑म॒वाद्य॑न्त॒ शक्म॒न्नोजि॑ष्ठा॒येत्या॒हौजि॑ष्ठ॒ꣳ॒ हि ते तदा॒त्मनः॑ सम॒वाद्य॒न्ताना॑धृष्टमस्यनाधृ॒ष्यमित्या॒हाना॑धृष्ट॒ꣴ॒ ह्ये॑तद॑नाधृ॒ष्यं दे॒वाना॒मोजः॑~(१०)

%6.2.2.4
इत्या॑ह दे॒वाना॒ꣴ॒ ह्ये॑तदोजो॑\-ऽभिशस्ति॒पा अ॑नभिशस्ते॒न्यमित्या॑हाभिशस्ति॒पा ह्ये॑तद॑नभिशस्ते॒न्यमनु॑ मे दी॒क्षां दी॒क्षाप॑तिर्मन्यता॒मित्या॑ह यथाय॒जुरे॒वैतद् घृ॒तं वै दे॒वा वज्रं॑ कृ॒त्वा सोम॑मघ्नन्नन्ति॒कमि॑व॒ खलु॒ वा अ॑स्यै॒तच्च॑रन्ति॒ यत्ता॑नून॒प्त्रेण॑ प्र॒चर॑न्त्य॒ꣳ॒शुरꣳ॑शुस्ते देव सो॒मा प्या॑यता॒मित्या॑ह॒ यत्~(११)

%6.2.2.5
ए॒वास्या॑पुवा॒यते॒ यन्मीय॑ते॒ तदे॒वास्यै॒तेना प्या॑यय॒त्या तुभ्य॒मिन्द्रः॑ प्यायता॒मा त्वमिन्द्रा॑य प्याय॒स्वेत्या॑हो॒भावे॒वेन्द्रं॑ च॒ सोमं॒ चा प्या॑यय॒त्या प्या॑यय॒ सखी᳚न्थ्स॒न्या मे॒धयेत्या॑ह॒र्त्विजो॒ वा अ॑स्य॒ सखा॑य॒स्ताने॒वा प्या॑ययति स्व॒स्ति ते॑ देव सोम सु॒त्याम॑शीय~(१२)

%6.2.2.6
इत्या॑हा॒\-ऽऽ\-शिष॑मे॒वैतामा शा᳚स्ते॒ प्र वा ए॒ते᳚\-ऽस्माल्लो॒काच्च्य॑वन्ते॒ ये सोम॑माप्या॒यय॑न्त्यन्तरिक्षदेव॒त्यो॑ हि सोम॒ आप्या॑यित॒ एष्टा॒ रायः॒ प्रेषे भगा॒येत्या॑ह॒ द्यावा॑पृथि॒वीभ्या॑मे॒व न॑म॒स्कृत्या॒स्मिँल्लो॒के प्रति॑ तिष्ठन्ति देवासु॒राः संय॑त्ता आस॒न्ते दे॒वा बिभ्य॑तो॒\-ऽग्निं प्रावि॑श॒न्तस्मा॑दाहुर॒ग्निः सर्वा॑ दे॒वता॒ इति॒ ते~(१३)

%6.2.2.7
अ॒ग्निमे॒व वरू॑थं कृ॒त्वासु॑रान॒भ्य॑भवन्न॒ग्निमि॑व॒ खलु॒ वा ए॒ष प्र वि॑शति॒ यो॑\-ऽवान्तरदी॒क्षामु॒पैति॒ भ्रातृ॑व्याभिभूत्यै॒ भव॑त्या॒त्मना॒ परा᳚स्य॒ भ्रातृ॑व्यो भवत्या॒त्मान॑मे॒व दी॒क्षया॑ पाति प्र॒जाम॑वान्तरदी॒क्षया॑ सन्त॒रां मेख॑लाꣳ स॒माय॑च्छते प्र॒जा ह्या᳚त्मनो\-ऽन्त॑रतरा त॒प्तव्र॑तो भवति॒ मद॑न्तीभिर्मार्जयते॒ निर्\mbox{}ह्य॑ग्निः शी॒तेन॒ वाय॑ति॒ समि॑द्ध्यै॒ या ते॑ अग्ने॒ रुद्रि॑या त॒नूरित्या॑ह॒ स्वयै॒वैन॑द्दे॒वत॑या व्रतयति सयोनि॒त्वाय॒ शान्त्यै᳚॥~(१४)

{\anuvakamend[{यो वा ओज॑ आह॒ यद॑शी॒येति॒ ते᳚\-ऽग्न॒ एका॑\-दश च}]}%~(२)

%6.2.3.1
तेषा॒मसु॑राणान्ति॒स्रः पुर॑ आसन्नय॒स्मय्य॑व॒मा\-ऽथ॑ रज॒ता\-ऽथ॒ हरि॑णी॒ ता दे॒वा जेतु॒न्नाश॑क्नुव॒न्ता उ॑प॒सदै॒वाजि॑गीष॒न्तस्मा॑दाहु॒र्यश्चै॒वं वेद॒ यश्च॒ नोप॒सदा॒ वै म॑हापु॒रं ज॑य॒न्तीति॒ त इषु॒ꣳ॒ सम॑स्कुर्वता॒ग्निमनी॑क॒ꣳ॒ सोमꣳ॑ श॒ल्यं विष्णु॒न्तेज॑न॒न्ते᳚\-ऽब्रुव॒न्क इ॒माम॑सिष्य॒तीति॑~(१५)

%6.2.3.2
रु॒द्र इत्य॑ब्रुवन्रु॒द्रो वै क्रू॒रः सो᳚\-ऽस्य॒त्विति॒ सो᳚\-ऽब्रवी॒द्वरं॑ वृणा अ॒हमे॒व प॑शू॒नामधि॑पतिरसा॒नीति॒ तस्मा᳚द्रु॒द्रः प॑शू॒नामधि॑पति॒स्ताꣳ रु॒द्रो\-ऽवा॑सृज॒थ्स ति॒स्रः पुरो॑ भि॒त्त्वैभ्यो लो॒केभ्यो\-ऽसु॑रा॒न्प्राणु॑दत॒ यदु॑प॒सद॑ उपस॒द्यन्ते॒ भ्रातृ॑व्यपराणुत्त्यै॒ नान्यामाहु॑तिं पु॒रस्ता᳚ज्जुहुया॒द्यद॒न्यामाहु॑तिं पु॒रस्ता᳚ज्जुहु॒यात्~(१६)

%6.2.3.3
अ॒न्यन्मुखं॑ कुर्याथ्स्रु॒वेणा॑घा॒रमा घा॑रयति य॒ज्ञस्य॒ प्रज्ञा᳚त्यै॒ परा॑ङति॒क्रम्य॑ जुहोति॒ परा॑च ए॒वैभ्यो लो॒केभ्यो॒ यज॑मानो॒ भ्रातृ॑व्या॒न्प्र णु॑दते॒ पुन॑रत्या॒क्रम्यो॑प॒सदं॑ जुहोति प्र॒णुद्यै॒वैभ्यो लो॒केभ्यो॒ भ्रातृ॑व्याञ्जि॒त्वा भ्रा॑तृव्यलो॒कम॒भ्यारो॑हति दे॒वा वै याः प्रा॒तरु॑प॒सद॑ उ॒पासी॑द॒न्नह्न॒स्ताभि॒रसु॑रा॒न्प्राणु॑दन्त॒ याः सा॒यꣳ रात्रि॑यै॒ ताभि॒र्यथ्सा॒यं प्रा॑तरुप॒सदः॑~(१७)

%6.2.3.4
उ॒प॒स॒द्यन्ते॑\-ऽहोरा॒त्राभ्या॑मे॒व तद्यज॑मानो॒ भ्रातृ॑व्या॒न्प्र णु॑दते॒ याः प्रा॒तर्या॒ज्याः᳚ स्युस्ताः सा॒यं पु॑रोनुवा॒क्याः᳚ कुर्या॒दया॑तयामत्वाय ति॒स्र उ॑प॒सद॒ उपै॑ति॒ त्रय॑ इ॒मे लो॒का इ॒माने॒व लो॒कान्प्री॑णाति॒ षट्थ्सं प॑द्यन्ते॒ षड्वा ऋ॒तव॑ ऋ॒तूने॒व प्री॑णाति॒ द्वाद॑शा॒हीने॒ सोम॒ उपै॑ति॒ द्वाद॑श॒ मासाः᳚ संवथ्स॒रः सं॑वथ्स॒रमे॒व प्री॑णाति॒ चतु॑र्विꣳशतिः॒ सम्~(१८)

%6.2.3.5
प॒द्य॒न्ते॒ चतु॑र्विꣳशतिरर्धमा॒सा अ॑र्धमा॒साने॒व प्री॑णा॒त्यारा᳚ग्रामवान्तरदी॒क्षामुपे॑या॒द्यः का॒मये॑ता॒स्मिन्मे॑ लो॒के\-ऽर्धु॑कꣴ स्या॒दित्येक॒मग्रे\-ऽथे॒ द्वावथ॒ त्रीनथ॑ च॒तुर॑ ए॒षा वा आरा᳚ग्रावान्तरदी॒क्षास्मिन्ने॒वास्मै॑ लो॒के\-ऽर्धु॑कं भवति प॒रोव॑रीयसीमवान्तरदी॒क्षामुपे॑या॒द्यः का॒मये॑ता॒मुष्मि॑न्मे लो॒के\-ऽर्धु॑कꣴ स्या॒दिति॑ च॒तुरो\-ऽग्रे\-ऽथ॒ त्रीनथ॒ द्वावथैक॑मे॒षा वै प॒रोव॑रीयस्यवान्तरदी॒क्षामुष्मि॑न्ने॒वास्मै॑ लो॒के\-ऽर्धु॑कं भवति॥~(१९)

{\anuvakamend[{अ॒सि॒ष्य॒तीति॑ जुहु॒याथ्सा॒यं प्रा॑तरुप॒सद॒श्चतु॑र्विꣳशतिः॒ सञ्च॒तुरो\-ऽग्रे॒ षोड॑श च}]}%~(३)

%6.2.4.1
सु॒व॒र्गं वा ए॒ते लो॒कं य॑न्ति॒ य उ॑प॒सद॑ उप॒यन्ति॒ तेषां॒ य उ॒न्नय॑ते॒ हीय॑त ए॒व स नोद॑ने॒षीति॒ सू᳚न्नीयमिव॒ यो वै स्वा॒र्थेतां᳚ य॒ताꣴ श्रा॒न्तो हीय॑त उ॒त स नि॒ष्ट्याय॑ स॒ह व॑सति॒ तस्मा᳚थ्स॒कृदु॒न्नीय॒ नाप॑र॒मुन्न॑येत द॒ध्नोन्न॑येतै॒तद्वै प॑शू॒नाꣳ रू॒पꣳ रू॒पेणै॒व प॒शूनव॑ रुन्धे~(२०)

%6.2.4.2
य॒ज्ञो दे॒वेभ्यो॒ निला॑यत॒ विष्णू॑ रू॒पं कृ॒त्वा स पृ॑थि॒वीं प्रावि॑श॒त्तं दे॒वा हस्ता᳚न्थ्स॒ꣳ॒रभ्यै᳚च्छ॒न्तमिन्द्र॑ उ॒पर्यु॑प॒र्यत्य॑क्राम॒थ्सो᳚\-ऽब्रवी॒त्को मा॒यमु॒पर्यु॑प॒र्यत्य॑क्रमी॒दित्य॒हं दु॒र्गे हन्तेत्यथ॒ कस्त्वमित्य॒हं दु॒र्गादाह॒र्तेति॒ सो᳚\-ऽब्रवीद्दु॒र्गे वै हन्ता॑वोचथा वरा॒हो॑\-ऽयं वा॑ममो॒षः~(२१)

%6.2.4.3
स॒प्ता॒नां गि॑री॒णां प॒रस्ता᳚द्वि॒त्तं वेद्य॒मसु॑राणां बिभर्ति॒ तं ज॑हि॒ यदि॑ दु॒र्गे हन्तासीति॒ स द॑र्भपुञ्जी॒लमु॒द्वृह्य॑ स॒प्त गि॒रीन्भि॒त्त्वा तम॑ह॒न्थ्सो᳚\-ऽब्रवीद्दु॒र्गाद्वा आह॑र्तावोचथा ए॒तमा ह॒रेति॒ तमे᳚भ्यो य॒ज्ञ ए॒व य॒ज्ञमाह॑र॒द्यत्तद्वि॒त्तं वेद्य॒मसु॑राणा॒मवि॑न्दन्त॒ तदेकं॒ वेद्यै॑ वेदि॒त्वमसु॑राणाम्~(२२)

%6.2.4.4
वा इ॒यमग्र॑ आसी॒द्याव॒दासी॑नः परा॒पश्य॑ति॒ ताव॑द्दे॒वाना॒न्ते दे॒वा अ॑ब्रुव॒न्नस्त्वे॒व नो॒\-ऽस्यामपीति॒ किय॑द्वो दास्याम॒ इति॒ याव॑दि॒यꣳ स॑लावृ॒की त्रिः प॑रि॒क्राम॑ति॒ ताव॑न्नो द॒त्तेति॒ स इन्द्रः॑ सलावृ॒की रू॒पं कृ॒त्वेमां त्रिः स॒र्वतः॒ पर्य॑क्राम॒त्तदि॒माम॑विन्दन्त॒ यदि॒मामवि॑न्दन्त॒ तद्वेद्यै॑ वेदि॒त्वम्~(२३)

%6.2.4.5
सा वा इ॒यꣳ सर्वै॒व वेदि॒रिय॑ति शक्ष्या॒मीति॒ त्वा अ॑व॒माय॑ यजन्ते त्रि॒ꣳ॒शत्प॒दानि॑ प॒श्चात्ति॒रश्ची॑ भवति॒ षट्त्रिꣳ॑श॒त्प्राची॒ चतु॑र्विꣳशतिः पु॒रस्ता᳚त्ति॒रश्ची॒ दश॑दश॒ सं प॑द्यन्ते॒ दशा᳚क्षरा वि॒राडन्नं॑ वि॒राड्वि॒राजै॒वान्नाद्य॒मव॑ रुन्ध॒ उद्ध॑न्ति॒ यदे॒वास्या॑ अमे॒ध्यं तदप॑ ह॒न्त्युद्ध॑न्ति॒ तस्मा॒दोष॑धयः॒ परा॑ भवन्ति ब॒र्॒\mbox{}हिः स्तृ॑णाति॒ तस्मा॒दोष॑धयः॒ पुन॒रा भ॑व॒न्त्युत्त॑रं ब॒र्॒\mbox{}हिष॑ उत्तरब॒र्॒\mbox{}हिः स्तृ॑णाति प्र॒जा वै ब॒र्॒\mbox{}हिर्यज॑मान उत्तरब॒र्॒\mbox{}हिर्यज॑मानमे॒वाय॑जमाना॒दुत्त॑रं करोति॒ तस्मा॒द्यज॑मा॒नो\-ऽय॑जमाना॒दुत्त॑रः॥~(२४)

{\anuvakamend[{रु॒न्धे॒ वा॒म॒मो॒षो वे॑दि॒त्वमसु॑राणां वेदि॒त्वं भ॑वन्ति॒ पञ्च॑विꣳशतिश्च}]}%~(४)

%6.2.5.1
यद्वा अनी॑शानो भा॒रमा॑द॒त्ते वि वै स लि॑शते॒ यद्द्वाद॑श सा॒ह्नस्यो॑प॒सदः॒ स्युस्ति॒स्रो॑\-ऽहीन॑स्य य॒ज्ञस्य॒ विलो॑म क्रियेत ति॒स्र ए॒व सा॒ह्नस्यो॑प॒सदो॒ द्वाद॑शा॒हीन॑स्य य॒ज्ञस्य॑ सवीर्य॒त्वायाथो॒ सलो॑म क्रियते व॒थ्सस्यैकः॒ स्तनो॑ भा॒गी हि सो\-ऽथैक॒ꣴ॒ स्तनं॑ व्र॒तमुपै॒त्यथ॒ द्वावथ॒ त्रीनथ॑ च॒तुर॑ ए॒तद्वै~(२५)

%6.2.5.2
क्षु॒रप॑वि॒ नाम॑ व्र॒तं येन॒ प्र जा॒तान्भ्रातृ॑व्यान्नु॒दते॒ प्रति॑ जनि॒ष्यमा॑णा॒नथो॒ कनी॑यसै॒व भूय॒ उपै॑ति च॒तुरो\-ऽग्रे॒ स्तना᳚न्व्र॒तमुपै॒त्यथ॒ त्रीनथ॒ द्वावथैक॑मे॒तद्वै सु॑जघ॒नं नाम॑ व्र॒तं त॑प॒स्यꣳ॑ सुव॒र्ग्य॑मथो॒ प्रैव जा॑यते प्र॒जया॑ प॒शुभि॑र्यवा॒गू रा॑ज॒न्य॑स्य व्र॒तं क्रू॒रेव॒ वै य॑वा॒गूः क्रू॒र इ॑व~(२६)

%6.2.5.3
रा॒ज॒न्यो॑ वज्र॑स्य रू॒पꣳ समृ॑द्ध्या आ॒मिक्षा॒ वैश्य॑स्य पाकय॒ज्ञस्य॑ रू॒पं पुष्ट्यै॒ पयो᳚ ब्राह्म॒णस्य॒ तेजो॒ वै ब्रा᳚ह्म॒णस्तेजः॒ पय॒स्तेज॑सै॒व तेजः॒ पय॑ आ॒त्मन्ध॒त्ते\-ऽथो॒ पय॑सा॒ वै गर्भा॑ वर्धन्ते॒ गर्भ॑ इव॒ खलु॒ वा ए॒ष यद्दी᳚क्षि॒तो यद॑स्य॒ पयो᳚ व्र॒तं भव॑त्या॒त्मान॑मे॒व तद्व॑र्धयति॒ त्रिव्र॑तो॒ वै मनु॑रासी॒द्द्विव्र॑ता॒ असु॑रा॒ एक॑व्रताः~(२७)

%6.2.5.4
दे॒वाः प्रा॒तर्म॒ध्यन्दि॑ने सा॒यं तन्मनो᳚र्व्र॒तमा॑सीत्पाकय॒ज्ञस्य॑ रू॒पं पुष्ट्यै᳚ प्रा॒तश्च॑ सा॒यं चासु॑राणां निर्म॒ध्यं क्षु॒धो रू॒पं तत॒स्ते परा॑भवन्म॒ध्यन्दि॑ने मध्यरा॒त्रे दे॒वानां॒ तत॒स्ते॑\-ऽभवन्थ्सुव॒र्गं लो॒कमा॑य॒न्॒ यद॑स्य म॒ध्यन्दि॑ने मध्यरा॒त्रे व्र॒तं भव॑ति मध्य॒तो वा अन्ने॑न भुञ्जते मध्य॒त ए॒व तदूर्जं॑ धत्ते॒ भ्रातृ॑व्याभिभूत्यै॒ भव॑त्या॒त्मना᳚~(२८)




%6.2.5.5
परा᳚\-ऽस्य॒ भ्रातृ॑व्यो भवति॒ गर्भो॒ वा ए॒ष यद्दी᳚क्षि॒तो योनि॑र्दीक्षितविमि॒तं यद्दी᳚क्षि॒तो दी᳚क्षितविमि॒तात्प्र॒वसे॒द्यथा॒ योने॒र्गर्भः॒ स्कन्द॑ति ता॒दृगे॒व तन्न प्र॑वस्त॒व्य॑मा॒त्मनो॑ गोपी॒थायै॒ष वै व्या॒घ्रः कु॑लगो॒पो यद॒ग्निस्तस्मा॒द्यद्दी᳚क्षि॒तः प्र॒वसे॒थ्स ए॑नमीश्व॒रो॑\-ऽनू॒त्थाय॒ हन्तो॒र्न प्र॑वस्त॒व्य॑मा॒त्मनो॒ गुप्त्यै॑ दक्षिण॒तः श॑य ए॒तद्वै यज॑मानस्या॒यत॑न॒ꣴ॒ स्व ए॒वायत॑ने शये॒\-ऽग्निम॑भ्या॒वृत्य॑ शये दे॒वता॑ ए॒व य॒ज्ञम॑भ्या॒वृत्य॑ शये॥~(२९)

{\anuvakamend[{ए॒तद्वै क्रू॒र इ॒वैक॑व्रता आ॒त्मना॒ यज॑मानस्य॒ त्रयो॑दश च}]}%~(५)

%6.2.6.1
पु॒रोह॑विषि देव॒यज॑ने याजये॒द्यं का॒मये॒तोपै॑न॒मुत्त॑रो य॒ज्ञो न॑मेद॒भि सु॑व॒र्गं लो॒कं ज॑ये॒दित्ये॒तद्वै पु॒रोह॑विर्देव॒यज॑नं॒ यस्य॒ होता᳚ प्रातरनुवा॒कम॑नुब्रु॒वन्न॒ग्निम॒प आ॑दि॒त्यम॒भि वि॒पश्य॒त्युपै॑न॒मुत्त॑रो य॒ज्ञो न॑मत्य॒भि सु॑व॒र्गं लो॒कं ज॑यत्या॒प्ते दे॑व॒यज॑ने याजये॒द्भ्रातृ॑व्यवन्तं॒ पन्थां᳚ वाधिस्प॒र्॒\mbox{}शये॑त्क॒र्तं वा॒ याव॒न्नान॑से॒ यात॒वै~(३०)

%6.2.6.2
न रथा॑यै॒तद्वा आ॒प्तं दे॑व॒यज॑नमा॒प्नोत्ये॒व भ्रातृ॑व्यं॒ नैन॒म्भ्रातृ॑व्य आप्नो॒त्येको᳚न्नते देव॒यज॑ने याजयेत्प॒शुका॑म॒मेको᳚न्नता॒द्वै दे॑व॒यज॑ना॒दङ्गि॑रसः प॒शून॑सृजन्तान्त॒रा स॑दोहविर्धा॒ने उ॑न्न॒तꣴ स्या॑दे॒तद्वा एको᳚न्नतं देव॒यज॑नं पशु॒माने॒व भ॑वति॒ त्र्यु॑न्नते देव॒यज॑ने याजयेथ्सुव॒र्गका॑म॒न्त्र्यु॑न्नता॒द्वै दे॑व॒यज॑ना॒दङ्गि॑रसः सुव॒र्गं लो॒कमा॑यन्नन्त॒राह॑व॒नीयं॑ च हवि॒र्धानं॑ च~(३१)

%6.2.6.3
उ॒न्न॒तꣴ स्या॑दन्त॒रा ह॑वि॒र्धानं॑ च॒ सद॑श्चान्त॒रा सद॑श्च॒ गार्\mbox{}ह॑पत्यं चै॒तद्वै त्र्यु॑न्नतं देव॒यज॑नꣳ सुव॒र्गमे॒व लो॒कमे॑ति॒ प्रति॑ष्ठिते देव॒यज॑ने याजयेत्प्रति॒ष्ठाका॑ममे॒तद्वै प्रति॑ष्ठितं देव॒यज॑नं॒ यथ्स॒र्वतः॑ स॒मं प्रत्ये॒व ति॑ष्ठति॒ यत्रा॒न्याअ॑न्या॒ ओष॑धयो॒ व्यति॑षक्ताः॒ स्युस्तद्या॑जयेत्प॒शुका॑ममे॒तद्वै प॑शू॒नाꣳ रू॒पꣳ रू॒पेणै॒वास्मै॑ प॒शून्~(३२)

%6.2.6.4
अव॑ रुन्धे पशु॒माने॒व भ॑वति॒ निर्\mbox{}ऋ॑तिगृहीते देव॒यज॑ने याजये॒द्यं का॒मये॑त॒ निर्\mbox{}ऋ॑त्यास्य य॒ज्ञं ग्रा॑हयेय॒मित्ये॒तद्वै निर्\mbox{}ऋ॑तिगृहीतं देव॒यज॑नं॒ यथ्स॒दृश्यै॑ स॒त्या॑ ऋ॒क्षन्निर्\mbox{}ऋ॑त्यै॒वास्य॑ य॒ज्ञं ग्रा॑हयति॒ व्यावृ॑त्ते देव॒यज॑ने याजयेद्व्या॒वृत्का॑मं॒ यं पात्रे॑ वा॒ तल्पे॑ वा॒ मीमाꣳ॑सेरन्प्रा॒चीन॑माहव॒नीया᳚त्प्रव॒णꣴ स्या᳚त्प्रती॒चीनं॒ गार्\mbox{}ह॑पत्यादे॒तद्वै व्यावृ॑त्तं देव॒यज॑नं॒ वि पा॒प्मना॒ भ्रातृ॑व्ये॒णा व॑र्तते॒ नैनं॒ पात्रे॒ न तल्पे॑ मीमाꣳसन्ते का॒र्ये॑ देव॒यज॑ने याजये॒द्भूति॑कामं का॒र्यो॑ वै पुरु॑षो॒ भव॑त्ये॒व॥~(३३)

{\anuvakamend[{यात॒वै ह॑वि॒र्धान॑ञ्च प॒शून्पा॒प्मना॒\-ऽष्टाद॑श च}]}%~(६)

%6.2.7.1
तेभ्य॑ उत्तरवे॒दिः सि॒ꣳ॒ही रू॒पं कृ॒त्वोभया॑नन्त॒राप॒क्रम्या॑तिष्ठ॒त्ते दे॒वा अ॑मन्यन्त यत॒रान् वा इ॒यमु॑पाव॒र्थ्स्यति॒ त इ॒दं भं॑विष्य॒न्तीति॒ तामुपा॑मन्त्रयन्त॒ साब्र॑वी॒द्वरं॑ वृणै॒ सर्वा॒न्मया॒ कामा॒न्व्य॑श्ञवथ॒ पूर्वां तु मा॒\-ऽग्नेराहु॑तिरश्ञवता॒ इति॒ तस्मा॑दुत्तरवे॒दिं पूर्वा॑म॒ग्नेर्व्याघा॑रयन्ति॒ वारे॑वृत॒ꣴ॒ ह्य॑स्यै॒ शम्य॑या॒ परि॑ मिमीते~(३४)

%6.2.7.2
मात्रै॒वास्यै॒ सा\-ऽथो॑ यु॒क्तेनै॒व यु॒क्तमव॑ रुन्धे वि॒त्ताय॑नी मे॒\-ऽसीत्या॑ह वि॒त्ता ह्ये॑ना॒नाव॑त्ति॒क्ताय॑नी मे॒\-ऽसीत्या॑ह ति॒क्तान् ह्ये॑ना॒नाव॒दव॑तान्मा नाथि॒तमित्या॑ह नाथि॒तान् ह्ये॑ना॒नाव॒दव॑तान्मा व्यथि॒तमित्या॑ह व्यथि॒तान् ह्ये॑ना॒नाव॑द्वि॒देर॒ग्निर्नभो॒ नाम॑~(३५)

%6.2.7.3
अग्ने॑ अङ्गिर॒ इति॒ त्रिर्\mbox{}ह॑रति॒ य ए॒वैषु लो॒केष्व॒ग्नय॒स्ताने॒वाव॑ रुन्धे तू॒ष्णीं च॑तु॒र्थꣳ ह॑र॒त्यनि॑रुक्तमे॒वाव॑ रुन्धे सि॒ꣳ॒हीर॑सि महि॒षीर॒सीत्या॑ह सि॒ꣳ॒हीर्\mbox{}ह्ये॑षा रू॒पं कृ॒त्वोभया॑नन्त॒राप॒क्रम्याति॑ष्ठदु॒रु प्र॑थस्वो॒रु ते॑ य॒ज्ञप॑तिः प्रथता॒मित्या॑ह॒ यज॑मानमे॒व प्र॒जया॑ प॒शुभिः॑ प्रथयति ध्रु॒वा~(३६)

%6.2.7.4
अ॒सीति॒ सꣳ ह॑न्ति॒ धृत्यै॑ दे॒वेभ्यः॑ शुन्धस्व दे॒वेभ्यः॑ शुम्भ॒स्वेत्यव॑ चो॒क्षति॒ प्र च॑ किरति॒ शुद्ध्या॑ इन्द्रघो॒षस्त्वा॒ वसु॑भिः पु॒रस्ता᳚त्पा॒त्वित्या॑ह दि॒ग्भ्य ए॒वैनां॒ प्रोक्ष॑ति दे॒वाꣴश्चेदु॑त्तरवे॒दिरु॒पाव॑वर्ती॒हैव वि ज॑यामहा॒ इत्यसु॑रा॒ वज्र॑मु॒द्यत्य॑ दे॒वान॒भ्या॑यन्त॒ तानि॑न्द्रघो॒षो वसु॑भिः पु॒रस्ता॒दप॑~(३७)

%6.2.7.5
अ॒नु॒द॒त॒ मनो॑जवाः पि॒तृभि॑र्दक्षिण॒तः प्रचे॑ता रु॒द्रैः प॒श्चाद्वि॒श्वक॑र्मादि॒त्यैरु॑त्तर॒तो यदे॒वमु॑त्तरवे॒दिं प्रो॒क्षति॑ दि॒ग्भ्य ए॒व तद्यज॑मानो॒ भ्रातृ॑व्या॒न्प्रणु॑दत॒ इन्द्रो॒ यती᳚न्थ्सालावृ॒केभ्यः॒ प्राय॑च्छ॒त्तान्द॑क्षिण॒त उ॑त्तरवे॒द्या आ॑द॒न्॒ यत्प्रोक्ष॑णीनामु॒च्छिष्ये॑त॒ तद्द॑क्षिण॒त उ॑त्तरवे॒द्यै नि न॑ये॒द्यदे॒व तत्र॑ क्रू॒रं तत्तेन॑ शमयति॒ यं द्वि॒ष्यात्तं ध्या॑येच्छु॒चैवैन॑मर्पयति॥~(३८)

{\anuvakamend[{मि॒मी॒ते॒ नाम॑ ध्रु॒वा\-ऽप॑ शु॒चा त्रीणि॑ च}]}%~(७)

%6.2.8.1
सोत्त॑रवे॒दिर॑ब्रवी॒थ्सर्वा॒न्मया॒ कामा॒न्व्य॑श्ञव॒थेति॒ ते दे॒वा अ॑कामय॒न्तासु॑रा॒न्भ्रातृ॑व्यान॒भि भ॑वे॒मेति॒ ते॑\-ऽजुहवुः सि॒ꣳ॒हीर॑सि सपत्नसा॒ही स्वाहेति॒ ते\-ऽसु॑रा॒न्भ्रातृ॑व्यान॒भ्य॑भव॒न्ते\-ऽसु॑रा॒न्भ्रातृ॑व्यानभि॒भूया॑कामयन्त प्र॒जां वि॑न्देम॒हीति॒ ते॑\-ऽजुहवुः सि॒ꣳ॒हीर॑सि सुप्रजा॒वनिः॒ स्वाहेति॒ ते प्र॒जाम॑विन्दन्त॒ ते प्र॒जां वि॒त्त्वा~(३९)

%6.2.8.2
अ॒का॒म॒य॒न्त॒ प॒शून् वि॑न्देम॒हीति॒ ते॑\-ऽजुहवुः सि॒ꣳ॒हीर॑सि रायस्पोष॒वनिः॒ स्वाहेति॒ ते प॒शून॑विन्दन्त॒ ते प॒शून् वि॒त्त्वा\-ऽ का॑मयन्त प्रति॒ष्ठां वि॑न्देम॒हीति॒ ते॑\-ऽजुहवुः सि॒ꣳ॒हीर॑स्यादित्य॒वनिः॒ स्वाहेति॒ त इ॒मां प्र॑ति॒ष्ठाम॑विन्दन्त॒ त इ॒मां प्र॑ति॒ष्ठां वि॒त्त्वाका॑मयन्त दे॒वता॑ आ॒शिष॒ उपे॑या॒मेति॒ ते॑\-ऽजुहवुः सि॒ꣳ॒हीर॒स्या व॑ह दे॒वान्दे॑वय॒ते~(४०)

%6.2.8.3
यज॑मानाय॒ स्वाहेति॒ ते दे॒वता॑ आ॒शिष॒ उपा॑य॒न्पञ्च॒ कृत्वो॒ व्याघा॑रयति॒ पञ्चा᳚क्षरा प॒ङ्क्तिः पाङ्क्तो॑ य॒ज्ञो य॒ज्ञमे॒वाव॑ रुन्धे\-ऽक्ष्ण॒या व्याघा॑रयति॒ तस्मा॑दक्ष्ण॒या प॒शवो\-ऽङ्गा॑नि॒ प्र ह॑रन्ति॒ प्रति॑ष्ठित्यै भू॒तेभ्य॒स्त्वेति॒ स्रुच॒मुद्गृ॑ह्णाति॒ य ए॒व दे॒वा भू॒तास्तेषा॒न्तद्भा॑ग॒धेय॒न्ताने॒व तेन॑ प्रीणाति॒ पौतु॑द्रवान्परि॒धीन्परि॑ दधात्ये॒षाम्~(४१)

%6.2.8.4
लो॒कानां॒ विधृ॑त्या अ॒ग्नेस्त्रयो॒ ज्यायाꣳ॑सो॒ भ्रात॑र आस॒न्ते दे॒वेभ्यो॑ ह॒व्यं वह॑न्तः॒ प्रामी॑यन्त॒ सो᳚\-ऽग्निर॑बिभेदि॒त्थं वाव स्य आर्ति॒मारि॑ष्य॒तीति॒ स निला॑यत॒ स यां वन॒स्पति॒ष्वव॑स॒त्तां पूतु॑द्रौ॒ यामोष॑धीषु॒ ताꣳ सु॑गन्धि॒तेज॑ने॒ यां प॒शुषु॒ तां पेत्व॑स्यान्त॒रा शृङ्गे॒ तं दे॒वताः॒ प्रैष॑मैच्छ॒न्तमन्व॑विन्द॒न्तम॑ब्रुवन्न्~(४२)

%6.2.8.5
उप॑ न॒ आ व॑र्तस्व ह॒व्यं नो॑ व॒हेति॒ सो᳚\-ऽब्रवी॒द्वरं॑ वृणै॒ यदे॒व गृ॑ही॒तस्याहु॑तस्य बहिःपरि॒धि स्कन्दा॒त्तन्मे॒ भ्रातृ॑णां भाग॒धेय॑मस॒दिति॒ तस्मा॒द्यद् गृ॑ही॒तस्याहु॑तस्य बहिःपरि॒धि स्कन्द॑ति॒ तेषा॒न्तद्भा॑ग॒धेयं॒ ताने॒व तेन॑ प्रीणाति॒ सो॑\-ऽमन्यतास्थ॒न्वन्तो॑ मे॒ पूर्वे॒ भ्रात॑रः॒ प्रामे॑षता॒स्थानि॑ शातया॒ इति॒ स यानि॑~(४३)

%6.2.8.6
अ॒स्थान्यशा॑तयत॒ तत्पूतु॑द्र्वभव॒द्यन्मा॒ꣳ॒समुप॑मृतं॒ तद्गुल्गु॑लु॒ यदे॒तान्थ्स॑म्भा॒रान्थ्स॒म्भर॑त्य॒ग्निमे॒व तथ्सम्भ॑रत्य॒ग्नेः पुरी॑षम॒सीत्या॑हा॒ग्नेर्\mbox{}ह्ये॑तत्पुरी॑षं॒ यथ्सं॑भा॒रा अथो॒ खल्वा॑हुरे॒ते वावैनं॒ ते भ्रात॑रः॒ परि॑ शेरे॒ यत्पौतु॑द्रवाः परि॒धय॒ इति॑॥~(४४)

{\anuvakamend[{वि॒त्त्वा दे॑वय॒त ए॒षाम॑ब्रुव॒न्॒ यानि॒ चतु॑श्चत्वारिꣳशच्च}]}%~(८)

%6.2.9.1
ब॒द्धमव॑ स्यति वरुणपा॒शादे॒वैने॑ मुञ्चति॒ प्र णे॑नेक्ति॒ मेध्ये॑ ए॒वैने॑ करोति सावित्रि॒यर्चा हु॒त्वा ह॑वि॒र्धाने॒ प्र व॑र्तयति सवि॒तृप्र॑सूत ए॒वैने॒ प्र व॑र्तयति॒ वरु॑णो॒ वा ए॒ष दु॒र्वागु॑भ॒यतो॑ ब॒द्धो यदक्षः॒ स यदु॒थ्सर्जे॒द्यज॑मानस्य गृ॒हान॒भ्युथ्स॑र्जेथ्सु॒वाग्दे॑व॒ दुर्या॒ꣳ॒ आ व॒देत्या॑ह गृ॒हा वै दुर्याः॒ शान्त्यै॒ पत्नी᳚~(४५)

%6.2.9.2
उपा॑नक्ति॒ पत्नी॒ हि सर्व॑स्य मि॒त्रं मि॑त्र॒त्वाय॒ यद्वै पत्नी॑ य॒ज्ञस्य॑ क॒रोति॑ मिथु॒नं तदथो॒ पत्नि॑या ए॒वैष य॒ज्ञस्या᳚न्वार॒म्भो\-ऽन॑वच्छित्त्यै॒ वर्त्म॑ना॒ वा अ॒न्वित्य॑ य॒ज्ञꣳ रक्षाꣳ॑सि जिघाꣳसन्ति वैष्ण॒वीभ्या॑मृ॒ग्भ्यां वर्त्म॑नोर्जुहोति य॒ज्ञो वै विष्णु॑र्य॒ज्ञादे॒व रक्षा॒ꣴ॒स्यप॑ हन्ति॒ यद॑ध्व॒र्युर॑न॒ग्नावाहु॑तिञ्जुहु॒याद॒न्धो᳚\-ऽध्व॒र्युः स्या॒द्रक्षाꣳ॑सि य॒ज्ञꣳ ह॑न्युः~(४६)

%6.2.9.3
हिर॑ण्यमु॒पास्य॑ जुहोत्यग्नि॒वत्ये॒व जु॑होति॒ नान्धो᳚\-ऽध्व॒र्युर्भव॑ति॒ न य॒ज्ञꣳ रक्षाꣳ॑सि घ्नन्ति॒ प्राची॒ प्रेत॑मध्व॒रं क॒ल्पय॑न्ती॒ इत्या॑ह सुव॒र्गमे॒वैने॑ लो॒कं ग॑मय॒त्यत्र॑ रमेथां॒ वर्ष्म॑न्पृथि॒व्या इत्या॑ह॒ वर्ष्म॒ ह्ये॑तत्पृ॑थि॒व्या यद्दे॑व॒यज॑न॒ꣳ॒ शिरो॒ वा ए॒तद्य॒ज्ञस्य॒ यद्ध॑वि॒र्धान॑न्दि॒वो वा॑ विष्णवु॒त वा॑ पृथि॒व्याः~(४७)

%6.2.9.4
इत्या॒शीर्प॑दय॒र्चा दक्षि॑णस्य हवि॒र्धान॑स्य मे॒थीं नि ह॑न्ति शीर्\mbox{}ष॒त ए॒व य॒ज्ञस्य॒ यज॑मान आ॒शिषो\-ऽव॑ रुन्धे द॒ण्डो वा औ॑प॒रस्तृ॒तीय॑स्य हवि॒र्धान॑स्य वषट्का॒रेणाक्ष॑मच्छिन॒द्यत्तृ॒तीयं॑ छ॒दिर्\mbox{}ह॑वि॒र्धान॑योरुदाह्रि॒यते॑ तृ॒तीय॑स्य हवि॒र्धान॒स्याव॑रुद्ध्यै॒ शिरो॒ वा ए॒तद्य॒ज्ञस्य॒ यद्ध॑वि॒र्धानं॒ विष्णो॑ र॒राट॑मसि॒ विष्णोः᳚ पृ॒ष्ठम॒सीत्या॑ह॒ तस्मा॑देताव॒द्धा शिरो॒ विष्यू॑तं॒ विष्णोः॒ स्यूर॑सि॒ विष्णो᳚र्ध्रु॒वम॒सीत्या॑ह वैष्ण॒वꣳ हि दे॒वत॑या हवि॒र्धानं॒ यं प्र॑थ॒मं ग्र॒न्थिं ग्र॑थ्नी॒याद्यत्तं न वि॑स्र॒ꣳ॒सये॒दमे॑हेनाध्व॒र्युः प्र मी॑येत॒ तस्मा॒थ्स वि॒स्रस्यः॑॥~(४८)

{\anuvakamend[{पत्नी॑ हन्युर्वा पृथि॒व्या विष्यू॑तं॒ विष्णोः॒ षड्विꣳ॑शतिश्च}]}%~(९)

%6.2.10.1
दे॒वस्य॑ त्वा सवि॒तुः प्र॑स॒व इत्यभ्रि॒मा द॑त्ते॒ प्रसू᳚त्या अ॒श्विनो᳚र्बा॒हुभ्या॒मित्या॑हा॒श्विनौ॒ हि दे॒वाना॑मध्व॒र्यू आस्तां᳚ पू॒ष्णो हस्ता᳚भ्या॒मित्या॑ह॒ यत्यै॒ वज्र॑ इव॒ वा ए॒षा यदभ्रि॒रभ्रि॑रसि॒ नारि॑र॒सीत्या॑ह॒ शान्त्यै॒ काण्डे॑काण्डे॒ वै क्रि॒यमा॑णे य॒ज्ञꣳ रक्षाꣳ॑सि जिघाꣳसन्ति॒ परि॑लिखित॒ꣳ॒ रक्षः॒ परि॑लिखिता॒ अरा॑तय॒ इत्या॑ह॒ रक्ष॑सा॒मप॑हत्यै~(४९)

%6.2.10.2
इ॒दम॒हꣳ रक्ष॑सो ग्री॒वा अपि॑ कृन्तामि॒ यो᳚\-ऽस्मान्द्वेष्टि॒ यं च॑ व॒यं द्वि॒ष्म इत्या॑ह॒ द्वौ वाव पुरु॑षौ॒ यं चै॒व द्वेष्टि॒ यश्चै॑नं॒ द्वेष्टि॒ तयो॑रे॒वान॑न्तरायं ग्री॒वाः कृ॑न्तति दि॒वे त्वा॒ऽन्तरि॑क्षाय त्वा पृथि॒व्यै त्वेत्या॑है॒भ्य ए॒वैनाँ᳚ल्लो॒केभ्यः॒ प्रोक्ष॑ति प॒रस्ता॑द॒र्वाचीं॒ प्रोक्ष॑ति॒ तस्मा᳚त्~(५०)

%6.2.10.3
प॒रस्ता॑द॒र्वाचीं᳚ मनु॒ष्या॑ ऊर्ज॒मुप॑ जीवन्ति क्रू॒रमि॑व॒ वा ए॒तत्क॑रोति॒ यत्खन॑त्य॒पो\-ऽव॑ नयति॒ शान्त्यै॒ यव॑मती॒रव॑ नय॒त्यूर्ग्वै यव॒ ऊर्गु॑दु॒म्बर॑ ऊ॒र्जैवोर्ज॒ꣳ॒ सम॑र्धयति॒ यज॑मानेन॒ सम्मि॒तौदु॑म्बरी भवति॒ यावा॑ने॒व यज॑मान॒स्ताव॑तीमे॒वास्मि॒न्नूर्जं॑ दधाति पितृ॒णाꣳ सद॑नम॒सीति॑ ब॒र्॒\mbox{}हिरव॑ स्तृणाति पितृदेव॒त्यम्᳚~(५१)

%6.2.10.4
ह्ये॑तद्यन्निखा॑तं॒ यद्ब॒र्॒\mbox{}हिरन॑वस्तीर्य मिनु॒यात्पि॑तृदेव॒त्या॑ निखा॑ता स्याद्ब॒र्॒\mbox{}हिर॑व॒स्तीर्य॑ मिनोत्य॒स्यामे॒वैनां᳚ मिनो॒त्यथो᳚ स्वा॒रुह॑मे॒वैना᳚ङ्करो॒त्युद्दिवꣴ॑ स्तभा॒नान्तरि॑क्षं पृ॒णेत्या॑है॒षां लो॒कानां॒ विधृ॑त्यै द्युता॒नस्त्वा॑ मारु॒तो मि॑नो॒त्वित्या॑ह द्युता॒नो ह॑ स्म॒ वै मा॑रु॒तो दे॒वाना॒मौदु॑म्बरीं मिनोति॒ तेनै॒व~(५२)

%6.2.10.5
ए॒नां॒ मि॒नो॒ति॒ ब्र॒ह्म॒वनिं᳚ त्वा क्षत्र॒वनि॒मित्या॑ह यथाय॒जुरे॒वैतद् घृ॒तेन॑ द्यावा\-पृथिवी॒ आ पृ॑णेथा॒मित्यौदु॑म्बर्यां जुहोति॒ द्यावा॑पृथि॒वी ए॒व रसे॑नानक्त्या॒न्तम॒न्वव॑स्रावयत्या॒न्तमे॒व यज॑मानं॒ तेज॑सा\-ऽनक्त्यै॒न्द्रम॒सीति॑ छ॒दिरधि॒ नि द॑धात्यै॒न्द्रꣳ हि दे॒वत॑या॒ सदो॑ विश्वज॒नस्य॑ छा॒येत्या॑ह विश्वज॒नस्य॒ ह्ये॑षा छा॒या यथ्सदो॒ नव॑छदि~(५३)

%6.2.10.6
तेज॑स्कामस्य मिनुयात्त्रि॒वृता॒ स्तोमे॑न॒ सम्मि॑त॒न्तेज॑स्त्रि॒वृत्ते॑ज॒स्व्ये॑व भ॑व॒त्येका॑\-दशछदीन्द्रि॒यका॑म॒स्यैका॑\-दशाक्षरा त्रि॒ष्टुगि॑न्द्रि॒यं त्रि॒ष्टुगि॑न्द्रिया॒व्ये॑व भ॑वति॒ पञ्च॑दशछदि॒ भ्रातृ॑व्यवतः पञ्चद॒शो वज्रो॒ भ्रातृ॑व्याभिभूत्यै स॒प्तद॑शछदि प्र॒जाका॑मस्य सप्तद॒शः प्र॒जा\-प॑तिः प्र॒जा\-प॑ते॒राप्त्या॒ एक॑विꣳशतिछदि प्रति॒ष्ठाका॑मस्यैकवि॒ꣳ॒शः स्तोमा॑नां प्रति॒ष्ठा प्रति॑ष्ठित्या उ॒दरं॒ वै सद॒ ऊर्गु॑दु॒म्बरो॑ मध्य॒त औदु॑म्बरीं मिनोति मध्य॒त ए॒व प्र॒जाना॒मूर्जं॑ दधाति॒ तस्मा᳚त्~(५४)

%6.2.10.7
म॒ध्य॒त ऊ॒र्जा भु॑ञ्जते यजमानलो॒के वै दक्षि॑णानि छ॒दीꣳषि॑ भ्रातृव्यलो॒क उत्त॑राणि॒ दक्षि॑णा॒न्युत्त॑राणि करोति॒ यज॑मानमे॒वाय॑जमाना॒दुत्त॑रं करोति॒ तस्मा॒द्यज॑मा॒नो\-ऽय॑जमाना॒दुत्त॑रो\-ऽन्तर्व॒र्तान्क॑रोति॒ व्यावृ॑त्त्यै॒ तस्मा॒दर॑ण्यं प्र॒जा उप॑ जीवन्ति॒ परि॑ त्वा गिर्वणो॒ गिर॒ इत्या॑ह यथाय॒जुरे॒वैतदिन्द्र॑स्य॒ स्यूर॒सीन्द्र॑स्य ध्रु॒वम॒सीत्या॑है॒न्द्रꣳ हि दे॒वत॑या॒ सदो॒ यं प्र॑थ॒मं ग्र॒न्थिं ग्र॑थ्नी॒याद्यत्तं न वि॑स्र॒ꣳ॒सये॒दमे॑हेनाध्व॒र्युः प्र मी॑येत॒ तस्मा॒थ्स वि॒स्रस्यः॑॥~(५)

{\anuvakamend[{अप॑हत्यै॒ तस्मा᳚त्पितृदेव॒त्य॑न्तेनै॒व नव॑छदि॒ तस्मा॒थ्सदः॒ पञ्च॑दश च}]}%॥10॥

%6.2.11.1
शिरो॒ वा ए॒तद्य॒ज्ञस्य॒ यद्ध॑वि॒र्धानं॑ प्रा॒णा उ॑पर॒वा ह॑वि॒र्धाने॑ खायन्ते॒ तस्मा᳚च्छी॒र्॒\mbox{}षन्प्रा॒णा अ॒धस्ता᳚त्खायन्ते॒ तस्मा॑द॒धस्ता᳚च्छी॒र्ष्णः प्रा॒णा र॑क्षो॒हणो॑ वलग॒हनो॑ वैष्ण॒वान्ख॑ना॒मीत्या॑ह वैष्ण॒वा हि दे॒वत॑योपर॒वा असु॑रा॒ वै नि॒र्यन्तो॑ दे॒वानां᳚ प्रा॒णेषु॑ वल॒गान्न्य॑खन॒न्तान्बा॑हुमा॒त्रे\-ऽन्व॑विन्द॒न्तस्मा᳚द्बाहुमा॒त्राः खा॑यन्त इ॒दम॒हं तं व॑ल॒गमुद्व॑पामि~(५६)

%6.2.11.2
यं नः॑ समा॒नो यमस॑मानो निच॒खानेत्या॑ह॒ द्वौ वाव पुरु॑षौ॒ यश्चै॒व स॑मा॒नो यश्चास॑मानो॒ यमे॒वास्मै॒ तौ व॑ल॒गं नि॒खन॑त॒स्तमे॒वोद्व॑पति॒ सं तृ॑णत्ति॒ तस्मा॒थ्सन्तृ॑ण्णा अन्तर॒तः प्रा॒णा न सम्भि॑नत्ति॒ तस्मा॒दस॑म्भिन्नाः प्रा॒णा अ॒पो\-ऽव॑ नयति॒ तस्मा॑दा॒र्द्रा अ॑न्तर॒तः प्रा॒णा यव॑मती॒रव॑ नयति~(५७)

%6.2.11.3
ऊर्ग्वै यवः॑ प्रा॒णा उ॑पर॒वाः प्रा॒णेष्वे॒वोर्जं॑ दधाति ब॒र्॒\mbox{}हिरव॑ स्तृणाति॒ तस्मा᳚ल्लोम॒शा अ॑न्तर॒तः प्रा॒णा आज्ये॑न॒ व्याघा॑रयति॒ तेजो॒ वा आज्यं॑ प्रा॒णा उ॑पर॒वाः प्रा॒णेष्वे॒व तेजो॑ दधाति॒ हनू॒ वा ए॒ते य॒ज्ञस्य॒ यद॑धि॒षव॑णे॒ न सं तृ॑ण॒त्त्यस॑न्तृण्णे॒ हि हनू॒ अथो॒ खलु॑ दीर्घसो॒मे स॒न्तृद्ये॒ धृत्यै॒ शिरो॒ वा ए॒तद्य॒ज्ञस्य॒ यद्ध॑वि॒र्धानम्᳚~(५८)

%6.2.11.4
प्रा॒णा उ॑पर॒वा हनू॑ अधि॒षव॑णे जि॒ह्वा चर्म॒ ग्रावा॑णो॒ दन्ता॒ मुख॑माहव॒नीयो॒ नासि॑कोत्तरवे॒दिरु॒दर॒ꣳ॒ सदो॑ य॒दा खलु॒ वै जि॒ह्वया॑ द॒थ्स्वधि॒ खाद॒त्यथ॒ मुखं॑ गच्छति य॒दा मुखं॒ गच्छ॒त्यथो॒दरं॑ गच्छति॒ तस्मा᳚द्धवि॒र्धाने॒ चर्म॒न्नधि॒ ग्राव॑भिरभि॒षुत्या॑हव॒नीये॑ हु॒त्वा प्र॒त्यञ्चः॑ प॒रेत्य॒ सद॑सि भक्षयन्ति॒ यो वै वि॒राजो॑ यज्ञमु॒खे दोहं॒ वेद॑ दु॒ह ए॒वैना॑मि॒यं वै वि॒राट्तस्यैँ त्वक्चर्मोधो॑\-ऽधि॒षव॑णे॒ स्तना॑ उपर॒वा ग्रावा॑णो व॒थ्सा ऋ॒त्विजो॑ दुहन्ति॒ सोमः॒ पयो॒ य ए॒वं वेद॑ दु॒ह ए॒वैना᳚म्॥~(५९)

{\anuvakamend[{व॒पा॒मि॒ यव॑मती॒रव॑ नयति हवि॒र्धान॑मे॒व त्रयो॑विꣳशतिश्च}]}%॥11॥

\prashnaend{यदु॒भौ दे॑वासु॒रा मि॒थस्तेषाꣳ॑ सुव॒र्गं यद्वा अनी॑शानः पु॒रोह॑विषि॒ तेभ्यः॒ सोत्त॑रवे॒दिर्ब॒द्धं दे॒वस्याभ्रि॒ꣳ॒ शिरो॒ वा एका॑\-दश॥११॥}{यदु॒भावित्या॑ह दे॒वानां᳚ य॒ज्ञो दे॒वेभ्यो॒ न रथा॑य॒ यज॑मानाय प॒रस्ता॑द॒र्वाची॒न्नव॑पञ्चा॒शत्॥५९॥}{यदु॒भौ दु॒ह ए॒वैना᳚म्॥}%%६-२
{॥हरिः॑ ॐ॥}{॥कृष्ण-यजुर्वेदीय-तैत्तिरीय-संहितायां षष्ठकाण्डे द्वितीयः प्रश्नः समाप्तः॥६-२॥}
%%% END PRASHNA
