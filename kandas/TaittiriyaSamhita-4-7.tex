\chapt{काण्डम् ४}
\sect{सप्तमः प्रश्नः}\setcounter{anuvakam}{0}
\dnsub{तैत्तिरीयसंहितायां चतुर्थकाण्डे सप्तमः प्रश्नः}
%4.7.1.1
अग्ना॑विष्णू स॒जोष॑से॒मा व॑र्धन्तु वां॒ गिरः॑। द्यु॒म्नैर्वाजे॑भि॒राग॑तम्। वाज॑श्च मे प्रस॒वश्च॑ मे॒ प्रय॑तिश्च मे॒ प्रसि॑तिश्च मे धी॒तिश्च॑ मे॒ क्रतु॑श्च मे॒ स्वर॑श्च मे॒ श्लोक॑श्च मे श्रा॒वश्च॑ मे॒ श्रुति॑श्च मे॒ ज्योति॑श्च मे॒ सुव॑श्च मे प्रा॒णश्च॑ मे\-ऽपा॒नः~(१)

%4.7.1.2
च॒ मे॒ व्या॒नश्च॒ मे\-ऽसु॑श्च मे चि॒त्तं च॑ म॒ आधी॑तं च मे॒ वाक्च॑ मे॒ मन॑श्च मे॒ चक्षु॑श्च मे॒ श्रोत्रं॑ च मे॒ दक्ष॑श्च मे॒ बलं॑ च म॒ ओज॑श्च मे॒ सह॑श्च म॒ आयु॑श्च मे ज॒रा च॑ म आ॒त्मा च॑ मे त॒नूश्च॑ मे॒ शर्म॑ च मे॒ वर्म॑ च॒ मे\-ऽङ्गा॑नि च मे॒\-ऽस्थानि॑ च मे॒ परूꣳ॑षि च मे॒ शरी॑राणि च मे॥~(२)

{\anuvakamend[{अ॒पा॒नस्त॒नूश्च॑ मे॒\-ऽष्टाद॑श च}]}%~(१)

%4.7.2.1
ज्यैष्ठ्यं॑ च म॒ आधि॑पत्यं च मे म॒न्युश्च॑ मे॒ भाम॑श्च॒ मे\-ऽम॑श्च॒ मे\-ऽम्भ॑श्च मे जे॒मा च॑ मे महि॒मा च॑ मे वरि॒मा च॑ मे प्रथि॒मा च॑ मे व॒र्ष्मा च॑ मे द्राघु॒या च॑ मे वृ॒द्धं च॑ मे॒ वृद्धि॑श्च मे स॒त्यं च॑ मे श्र॒द्धा च॑ मे॒ जग॑च्च~(३)

%4.7.2.2
मे॒ धनं॑ च मे॒ वश॑श्च मे॒ त्विषि॑श्च मे क्री॒डा च॑ मे॒ मोद॑श्च मे जा॒तं च॑ मे जनि॒ष्यमा॑णं च मे सू॒क्तं च॑ मे सुकृ॒तं च॑ मे वि॒त्तं च॑ मे॒ वेद्यं॑ च मे भू॒तं च॑ मे भवि॒ष्यच्च॑ मे सु॒गं च॑ मे सु॒पथं॑ च म ऋ॒द्धं च॑ म॒ ऋद्धि॑श्च मे कॢ॒प्तं च॑ मे॒ कॢप्ति॑श्च मे म॒तिश्च॑ मे सुम॒तिश्च॑ मे॥~(४)

{\anuvakamend[{जग॒च्चर्द्धि॒श्चतु॑र्दश च}]}%~(२)

%4.7.3.1
शं च॑ मे॒ मय॑श्च मे प्रि॒यं च॑ मे\-ऽनुका॒मश्च॑ मे॒ काम॑श्च मे सौमन॒सश्च॑ मे भ॒द्रं च॑ मे॒ श्रेय॑श्च मे॒ वस्य॑श्च मे॒ यश॑श्च मे॒ भग॑श्च मे॒ द्रवि॑णं च मे य॒न्ता च॑ मे ध॒र्ता च॑ मे॒ क्षेम॑श्च मे॒ धृति॑श्च मे॒ विश्वं॑ च~(५)

%4.7.3.2
मे॒ मह॑श्च मे सं॒विच्च॑ मे॒ ज्ञात्रं॑ च मे॒ सूश्च॑ मे प्र॒सूश्च॑ मे॒ सीरं॑ च मे ल॒यश्च॑ म ऋ॒तं च॑ मे॒\-ऽमृतं॑ च मे\-ऽय॒क्ष्मं च॒ मे\-ऽना॑मयच्च मे जी॒वातु॑श्च मे दीर्घायु॒त्वं च॑ मे\-ऽनमि॒त्रं च॒ मे\-ऽभ॑यं च मे सु॒गं च॑ मे॒ शय॑नं च मे सू॒षा च॑ मे सु॒दिनं॑ च मे॥~(६)

{\anuvakamend[{विश्वं॑ च॒ शय॑नम॒ष्टौ च॑}]}%~(३)

%4.7.4.1
ऊर्क्च॑ मे सू॒नृता॑ च मे॒ पय॑श्च मे॒ रस॑श्च मे घृ॒तं च॑ मे॒ मधु॑ च मे॒ सग्धि॑श्च मे॒ सपी॑तिश्च मे कृ॒षिश्च॑ मे॒ वृष्टि॑श्च मे॒ जैत्रं॑ च म॒ औद्भि॑द्यं च मे र॒यिश्च॑ मे॒ राय॑श्च मे पु॒ष्टं च॑ मे॒ पुष्टि॑श्च मे वि॒भु च॑~(७)

%4.7.4.2
मे॒ प्र॒भु च॑ मे ब॒हु च॑ मे॒ भूय॑श्च मे पू॒र्णं च॑ मे पू॒र्णत॑रं च॒ मे\-ऽक्षि॑तिश्च मे॒ कूय॑वाश्च॒ मे\-ऽन्नं॑ च॒ मे\-ऽक्षु॑च्च मे व्री॒हय॑श्च मे॒ यवा᳚श्च मे॒ माषा᳚श्च मे॒ तिला᳚श्च मे मु॒द्गाश्च॑ मे ख॒ल्वा᳚श्च मे गो॒धूमा᳚श्च मे म॒सुरा᳚श्च मे प्रि॒यङ्ग॑वश्च॒ मे\-ऽण॑वश्च मे श्या॒माका᳚श्च मे नी॒वारा᳚श्च मे॥~(८)

{\anuvakamend[{वि॒भु च॑ म॒सुरा॒श्चतु॑र्दश च}]}%~(४)

%4.7.5.1
अश्मा॑ च मे॒ मृत्ति॑का च मे गि॒रय॑श्च मे॒ पर्व॑ताश्च मे॒ सिक॑ताश्च मे॒ वन॒स्पत॑यश्च मे॒ हिर॑ण्यं च॒ मे\-ऽय॑श्च मे॒ सीसं॑ च मे॒ त्रपु॑श्च मे श्या॒मं च॑ मे लो॒हं च॑ मे॒\-ऽग्निश्च॑ म॒ आप॑श्च मे वी॒रुध॑श्च म॒ ओष॑धयश्च मे कृष्टप॒च्यं च॑~(९)

%4.7.5.2
मे॒\-ऽकृ॒ष्ट॒प॒च्यं च॑ मे ग्रा॒म्याश्च॑ मे प॒शव॑ आर॒ण्याश्च॑ य॒ज्ञेन॑ कल्पन्तां वि॒त्तं च॑ मे॒ वित्ति॑श्च मे भू॒तं च॑ मे॒ भूति॑श्च मे॒ वसु॑ च मे वस॒तिश्च॑ मे॒ कर्म॑ च मे॒ शक्ति॑श्च॒ मे\-ऽर्थ॑श्च म॒ एम॑श्च म॒ इति॑श्च मे॒ गति॑श्च मे॥~(१०)

{\anuvakamend[{कृ॒ष्ट॒प॒च्यञ्चा॒ष्टाच॑त्वारिꣳशच्च}]}%~(५)

%4.7.6.1
अ॒ग्निश्च॑ म॒ इन्द्र॑श्च मे॒ सोम॑श्च म॒ इन्द्र॑श्च मे सवि॒ता च॑ म॒ इन्द्र॑श्च मे॒ सर॑स्वती च म॒ इन्द्र॑श्च मे पू॒षा च॑ म॒ इन्द्र॑श्च मे॒ बृह॒स्पति॑श्च म॒ इन्द्र॑श्च मे मि॒त्रश्च॑ म॒ इन्द्र॑श्च मे॒ वरु॑णश्च म॒ इन्द्र॑श्च मे॒ त्वष्टा॑ च~(११)

%4.7.6.2
म॒ इन्द्र॑श्च मे धा॒ता च॑ म॒ इन्द्र॑श्च मे॒ विष्णु॑श्च म॒ इन्द्र॑श्च मे॒\-ऽश्विनौ॑ च म॒ इन्द्र॑श्च मे म॒रुत॑श्च म॒ इन्द्र॑श्च मे॒ विश्वे॑ च मे दे॒वा इन्द्र॑श्च मे पृथि॒वी च॑ म॒ इन्द्र॑श्च मे॒\-ऽन्तरि॑क्षञ्च म॒ इन्द्र॑श्च मे॒ द्यौश्च॑ म॒ इन्द्र॑श्च मे॒ दिश॑श्च म॒ इन्द्र॑श्च मे मू॒र्धा च॑ म॒ इन्द्र॑श्च मे प्र॒जा\-प॑तिश्च म॒ इन्द्र॑श्च मे॥~(१२)

{\anuvakamend[{त्वष्टा॑ च॒ द्यौश्च॑ म॒ एक॑विꣳशतिश्च}]}%~(६)

%4.7.7.1
अ॒ꣳ॒शुश्च॑ मे र॒श्मिश्च॒ मे\-ऽदा᳚भ्यश्च॒ मे\-ऽधि॑पतिश्च म उपा॒ꣳ॒शुश्च॑ मे\-ऽन्तर्या॒मश्च॑ म ऐन्द्रवाय॒वश्च॑ मे मैत्रावरु॒णश्च॑ म आश्वि॒नश्च॑ मे प्रतिप्र॒स्थान॑श्च मे शु॒क्रश्च॑ मे म॒न्थी च॑ म आग्रय॒णश्च॑ मे वैश्वदे॒वश्च॑ मे ध्रु॒वश्च॑ मे वैश्वान॒रश्च॑ म ऋतुग्र॒हाश्च॑~(१३)

%4.7.7.2
मे॒\-ऽति॒ग्रा॒ह्या᳚श्च म ऐन्द्रा॒ग्नश्च॑ मे वैश्वदे॒वश्च॑ मे मरुत्व॒तीया᳚श्च मे माहे॒न्द्रश्च॑ म आदि॒त्यश्च॑ मे सावि॒त्रश्च॑ मे सारस्व॒तश्च॑ मे पौ॒ष्णश्च॑ मे पात्नीव॒तश्च॑ मे हारियोज॒नश्च॑ मे॥~(१४)

{\anuvakamend[{ऋ॒तु॒ग्र॒हाश्च॒ चतु॑स्त्रिꣳशच्च}]}%~(७)

%4.7.8.1
इ॒ध्मश्च॑ मे ब॒र्॒\mbox{}हिश्च॑ मे॒ वेदि॑श्च मे॒ धिष्णि॑याश्च मे॒ स्रुच॑श्च मे चम॒साश्च॑ मे॒ ग्रावा॑णश्च मे॒ स्वर॑वश्च म उपर॒वाश्च॑ मे\-ऽधि॒षव॑णे च मे द्रोणकल॒शश्च॑ मे वाय॒व्या॑नि च मे पूत॒भृच्च॑ म आधव॒नीय॑श्च म॒ आग्नी᳚ध्रं च मे हवि॒र्धानं॑ च मे गृ॒हाश्च॑ मे॒ सद॑श्च मे पुरो॒डाशा᳚श्च मे पच॒ताश्च॑ मे\-ऽवभृ॒थश्च॑ मे स्वगाका॒रश्च॑ मे॥~(१५)

{\anuvakamend[{गृ॒हाश्च॒ षोड॑श च}]}%~(८)

%4.7.9.1
अ॒ग्निश्च॑ मे घ॒र्मश्च॑ मे॒\-ऽर्कश्च॑ मे॒ सूर्य॑श्च मे प्रा॒णश्च॑ मे\-ऽश्वमे॒धश्च॑ मे पृथि॒वी च॒ मे\-ऽदि॑तिश्च मे॒ दिति॑श्च मे॒ द्यौश्च॑ मे॒ शक्व॑रीर॒ङ्गुल॑यो॒ दिश॑श्च मे य॒ज्ञेन॑ कल्पन्ता॒मृक्च॑ मे॒ साम॑ च मे॒ स्तोम॑श्च मे॒ यजु॑श्च मे दी॒क्षा च॑ मे॒ तप॑श्च म ऋ॒तुश्च॑ मे व्र॒तं च॑ मे\-ऽहोरा॒त्रयो᳚र्वृ॒ष्ट्या बृ॑हद्रथन्त॒रे च॑ मे य॒ज्ञेन॑ कल्पेताम्॥~(१६)

{\anuvakamend[{दी॒क्षा\-ऽष्टाद॑श च}]}%~(९)

%4.7.10.1
गर्भा᳚श्च मे व॒थ्साश्च॑ मे॒ त्र्यवि॑श्च मे त्र्य॒वी च॑ मे दित्य॒वाट्च॑ मे दित्यौ॒ही च॑ मे॒ पञ्चा॑विश्च मे पञ्चा॒वी च॑ मे त्रिव॒थ्सश्च॑ मे त्रिव॒थ्सा च॑ मे तुर्य॒वाट्च॑ मे तुर्यौ॒ही च॑ मे पष्ठ॒वाच्च॑ मे पष्ठौ॒ही च॑ म उ॒क्षा च॑ मे व॒शा च॑ म ऋष॒भश्च॑~(१७)

%4.7.10.2
मे॒ वे॒हच्चमे\-ऽन॒ड्वां च॑ मे धे॒नुश्च॑ म॒ आयु॑र्य॒ज्ञेन॑ कल्पतां प्रा॒णो य॒ज्ञेन॑ कल्पतामपा॒नो य॒ज्ञेन॑ कल्पताव्व्याँ॒नो य॒ज्ञेन॑ कल्पतां॒ चक्षु॑र्य॒ज्ञेन॑ कल्पता॒ꣴ॒ श्रोत्रं॑ य॒ज्ञेन॑ कल्पतां॒ मनो॑ य॒ज्ञेन॑ कल्पतां॒ वाग्य॒ज्ञेन॑ कल्पतामा॒त्मा य॒ज्ञेन॑ कल्पतां य॒ज्ञो य॒ज्ञेन॑ कल्पताम्॥~(१८)

{\anuvakamend[{ऋ॒ष॒भश्च॑ चत्वारि॒ꣳ॒शच्च॑}]}%॥10॥

%4.7.11.1
एका॑ च मे ति॒स्रश्च॑ मे॒ पञ्च॑ च मे स॒प्त च॑ मे॒ नव॑ च म॒ एका॑\-दश च मे॒ त्रयो॑दश च मे॒ पञ्च॑दश च मे स॒प्तद॑श च मे॒ नव॑दश च म॒ एक॑विꣳशतिश्च मे॒ त्रयो॑विꣳशतिश्च मे॒ पञ्च॑विꣳशतिश्च मे स॒प्तविꣳ॑शतिश्च मे॒ नव॑विꣳशतिश्च म॒ एक॑त्रिꣳशच्च मे॒ त्रय॑स्त्रिꣳशच्च~(१९)

%4.7.11.2
मे॒ चत॑स्रश्च मे॒\-ऽष्टौ च॑ मे॒ द्वाद॑श च मे॒ षोड॑श च मे विꣳश॒तिश्च॑ मे॒ चतु॑र्विꣳशतिश्च मे॒\-ऽष्टाविꣳ॑शतिश्च मे॒ द्वात्रिꣳ॑शच्च मे॒ षट्त्रिꣳ॑शच्च मे चत्वारि॒ꣳ॒शच्च॑ मे॒ चतु॑श्चत्वारिꣳशच्च मे॒\-ऽष्टाच॑त्वारिꣳशच्च मे॒ वाज॑श्च प्रस॒वश्चा॑पि॒जश्च॒ क्रतु॑श्च॒ सुव॑श्च मू॒र्धा च॒ व्यश्ञि॑यश्चा\-ऽऽ\-न्त्याय॒नश्चान्त्य॑श्च भौव॒नश्च॒ भुव॑न॒श्चाधि॑पतिश्च॥~(२०)

{\anuvakamend[{त्रय॑स्त्रिꣳशच्च॒ व्यश्ञि॑य॒ एका॑\-दश च}]}%॥11॥

%4.7.12.1
वाजो॑ नः स॒प्त प्र॒दिश॒श्चत॑स्रो वा परा॒वतः॑। वाजो॑ नो॒ विश्वै᳚र्दे॒वैर्धन॑सातावि॒हाव॑तु। विश्वे॑ अ॒द्य म॒रुतो॒ विश्व॑ ऊ॒ती विश्वे॑ भवन्त्व॒ग्नयः॒ समि॑द्धाः। विश्वे॑ नो दे॒वा अव॒सा ग॑मन्तु॒ विश्व॑मस्तु॒ द्रवि॑णं॒ वाजो॑ अ॒स्मे। वाज॑स्य प्रस॒वं दे॑वा॒ रथै᳚र्याता हिर॒ण्ययैः᳚। अ॒ग्निरिन्द्रो॒ बृह॒स्पति॑र्म॒रुतः॒ सोम॑पीतये। वाजे॑वाजे\-ऽवत वाजिनो नो॒ धने॑षु~(२१)

%4.7.12.2
वि॒प्रा॒ अ॒मृ॒ता॒ ऋ॒त॒ज्ञाः॒। अ॒स्य मध्वः॑ पिबत मा॒दय॑ध्वं तृ॒प्ता या॑त प॒थिभि॑र्देव॒यानैः᳚। वाजः॑ पु॒रस्ता॑दु॒त म॑ध्य॒तो नो॒ वाजो॑ दे॒वाꣳ ऋ॒तुभिः॑ कल्पयाति। वाज॑स्य॒ हि प्र॑स॒वो नन्न॑मीति॒ विश्वा॒ आशा॒ वाज॑पतिर्भवेयम्। पयः॑ पृथि॒व्यां पय॒ ओष॑धीषु॒ पयो॑ दिव्य॒न्तरि॑क्षे॒ पयो॑ धाम्। पय॑स्वतीः प्र॒दिशः॑ सन्तु॒ मह्यम्᳚। सम्मा॑ सृजामि॒ पय॑सा घृ॒तेन॒ सम्मा॑ सृजाम्य॒पः~(२२)

%4.7.12.3
ओष॑धीभिः। सो॑\-ऽहं वाजꣳ॑ सनेयमग्ने। नक्तो॒षासा॒ सम॑नसा॒ विरू॑पे धा॒पये॑ते॒ शिशु॒मेकꣳ॑ समी॒ची। द्यावा॒ क्षामा॑ रु॒क्मो अ॒न्तर्वि भा॑ति दे॒वा अ॒ग्निं धा॑रयन्द्रविणो॒दाः। स॒मु॒द्रो॑\-ऽसि॒ नभ॑स्वाना॒र्द्रदा॑नुः श॒म्भूर्म॑यो॒भूर॒भि मा॑ वाहि॒ स्वाहा॑ मारु॒तो॑\-ऽसि म॒रुतां᳚ ग॒णः श॒म्भूर्म॑यो॒भूर॒भि मा॑ वाहि॒ स्वाहा॑व॒स्युर॑सि॒ दुव॑स्वाञ्छ॒म्भूर्म॑यो॒भूर॒भि मा॑ वाहि॒ स्वाहा᳚॥~(२३)

{\anuvakamend[{धने᳚ष्व॒पो दुव॑स्वाञ्छ॒म्भूर्म॑यो॒भूर॒भि मा॒ द्वे च॑}]}%॥12॥

%4.7.13.1
अ॒ग्निं यु॑नज्मि॒ शव॑सा घृ॒तेन॑ दि॒व्यꣳ सु॑प॒र्णं वय॑सा बृ॒हन्तम्᳚। तेन॑ व॒यं प॑तेम ब्र॒ध्नस्य॑ वि॒ष्टप॒ꣳ॒ सुवो॒ रुहा॑णा॒ अधि॒ नाक॑ उत्त॒मे। इ॒मौ ते॑ प॒क्षाव॒जरौ॑ पत॒त्रिणो॒ याभ्या॒ꣳ॒ रक्षाꣴ॑स्यप॒हꣴस्य॑ग्ने। ता\-भ्यां᳚ पतेम सु॒कृता॑मु लो॒कं यत्रर्\mbox{}ष॑यः प्रथम॒जा ये पु॑रा॒णाः। चिद॑सि समु॒द्रयो॑नि॒रिन्दु॒र्दक्षः॑ श्ये॒न ऋ॒तावा᳚। हिर॑ण्यपक्षः शकु॒नो भु॑र॒ण्युर्म॒हान्थ्स॒धस्थे᳚ ध्रु॒वः~(२४)

%4.7.13.2
आ निष॑त्तः। नम॑स्ते अस्तु॒ मा मा॑ हिꣳसी॒र्विश्व॑स्य मू॒र्धन्नधि॑ तिष्ठसि श्रि॒तः। स॒मु॒द्रे ते॒ हृद॑यम॒न्तरायु॒र्द्यावा॑पृथि॒वी भुव॑ने॒ष्वर्पि॑ते। उ॒द्नो द॑त्तोद॒धिं भि॑न्त दि॒वः प॒र्जन्या॑द॒न्तरि॑क्षात्पृथि॒व्यास्ततो॑ नो॒ वृष्ट्या॑वत। दि॒वो मू॒र्धासि॑ पृथि॒व्या नाभि॒रूर्ग॒पामोष॑धीनाम्। वि॒श्वायुः॒ शर्म॑ स॒प्रथा॒ नम॑स्प॒थे। येनर्\mbox{}ष॑य॒स्तप॑सा स॒त्रम्~(२५)

%4.7.13.3
आस॒तेन्धा॑ना अ॒ग्निꣳ सुव॑रा॒भर॑न्तः। तस्मि॑न्न॒हं नि द॑धे॒ नाके॑ अ॒ग्निमे॒तं यमा॒हुर्मन॑वः स्ती॒र्णब॑र्\mbox{}हिषम्। तं पत्नी॑भि॒रनु॑ गच्छेम देवाः पु॒त्रैर्भ्रातृ॑भिरु॒त वा॒ हिर॑ण्यैः। नाकं॑ गृह्णा॒नाः सु॑कृ॒तस्य॑ लो॒के तृ॒तीये॑ पृ॒ष्ठे अधि॑ रोच॒ने दि॒वः। आ वा॒चो मध्य॑मरुहद्भुर॒ण्युर॒यम॒ग्निः सत्प॑ति॒श्चेकि॑तानः। पृ॒ष्ठे पृ॑थि॒व्या निहि॑तो॒ दवि॑द्युतदधस्प॒दं कृ॑णुते~(२६)

%4.7.13.4
ये पृ॑त॒न्यवः॑। अ॒यम॒ग्निर्वी॒रत॑मो वयो॒धाः स॑ह॒स्रियो॑ दीप्यता॒मप्र॑युच्छन्न्। वि॒भ्राज॑मानः सरि॒रस्य॒ मध्य॒ उप॒ प्र या॑त दि॒व्यानि॒ धाम॑। सम्प्र च्य॑वध्व॒मनु॒ सम्प्र या॒ताग्ने॑ प॒थो दे॑व॒याना᳚न्कृणुध्वम्। अ॒स्मिन्थ्स॒धस्थे॒ अध्युत्त॑रस्मि॒न्विश्वे॑ देवा॒ यज॑मानश्च सीदत। येना॑ स॒हस्रं॒ वह॑सि॒ येना᳚ग्ने सर्ववेद॒सम्। तेने॒मं य॒ज्ञं नो॑ वह देव॒यानो॒ यः~(२७)

%4.7.13.5
उ॒त्त॒मः। उद्बु॑ध्यस्वाग्ने॒ प्रति॑ जागृह्येनमिष्टापू॒र्ते सꣳसृ॑जेथाम॒यं च॑। पुनः॑ कृ॒ण्वꣴस्त्वा॑ पि॒तरं॒ युवा॑नम॒न्वाताꣳ॑सी॒त् त्वयि॒ तन्तु॑मे॒तम्। अ॒यं ते॒ योनि॑र्\mbox{}ऋ॒त्वियो॒ यतो॑ जा॒तो अरो॑चथाः। तं जा॒नन्न॑ग्न॒ आ रो॒हाथा॑ नो वर्धया र॒यिम्॥~(२८)

{\anuvakamend[{ध्रु॒वः स॒त्रं कृ॑णुते॒ यः स॒प्तत्रिꣳ॑शच्च}]}%॥13॥

%4.7.14.1
ममा᳚ग्ने॒ वर्चो॑ विह॒वेष्व॑स्तु व॒यं त्वेन्धा॑नास्त॒नुवं॑ पुषेम। मह्यं॑ नमन्तां प्र॒दिश॒श्चत॑स्र॒स्त्वयाध्य॑क्षेण॒ पृत॑ना जयेम। मम॑ दे॒वा वि॑ह॒वे स॑न्तु॒ सर्व॒ इन्द्रा॑वन्तो म॒रुतो॒ विष्णु॑र॒ग्निः। ममा॒न्तरि॑क्षमु॒रु गो॒पम॑स्तु॒ मह्यं॒ वातः॑ पवतां॒ कामे॑ अ॒स्मिन्न्। मयि॑ दे॒वा द्रवि॑ण॒मा य॑जन्तां॒ मय्या॒शीर॑स्तु॒ मयि॑ दे॒वहू॑तिः। दैव्या॒ होता॑रा वनिषन्त~(२९)

%4.7.14.2
पूर्वे\-ऽरि॑ष्टाः स्याम त॒नुवा॑ सु॒वीराः᳚। मह्यं॑ यजन्तु॒ मम॒ यानि॑ ह॒व्याकू॑तिः स॒त्या मन॑सो मे अस्तु। एनो॒ मा नि गां᳚ कत॒मच्च॒नाहं विश्वे॑ देवासो॒ अधि॑ वोचता मे। देवीः᳚ षडुर्वीरु॒रु णः॑ कृणोत॒ विश्वे॑ देवास इ॒ह वी॑रयध्वम्। मा हा᳚स्महि प्र॒जया॒ मा त॒नूभि॒र्मा र॑धाम द्विष॒ते सो॑म राजन्न्। अ॒ग्निर्म॒न्युं प्र॑तिनु॒दन्पु॒रस्ता᳚त्~(३०)

%4.7.14.3
अद॑ब्धो गो॒पाः परि॑ पाहि न॒स्त्वम्। प्र॒त्यञ्चो॑ यन्तु नि॒गुतः॒ पुन॒स्ते॑\-ऽमैषां᳚ चि॒त्तं प्र॒बुधा॒ वि ने॑शत्। धा॒ता धा॑तृ॒णां भुव॑नस्य॒ यस्पति॑र्दे॒वꣳ स॑वि॒तार॑मभिमाति॒षाहम्᳚। इ॒मं य॒ज्ञम॒श्विनो॒भा बृह॒स्पति॑र्दे॒वाः पा᳚न्तु॒ यज॑मानं न्य॒र्थात्। उ॒रु॒व्यचा॑ नो महि॒षः शर्म॑ यꣳसद॒स्मिन् हवे॑ पुरुहू॒तः पु॑रु॒क्षु। स नः॑ प्र॒जायै॑ हर्यश्व मृड॒येन्द्र॒ मा~(३१)

%4.7.14.4
नो॒ री॒रि॒षो॒ मा परा॑ दाः। ये नः॑ स॒पत्ना॒ अप॒ ते भ॑वन्त्विन्द्रा॒ग्निभ्या॒मव॑ बाधामहे॒ तान्। वस॑वो रु॒द्रा आ॑दि॒त्या उ॑परि॒स्पृशं॑ मो॒ग्रं चेत्ता॑रमधिरा॒जम॑क्रन्न्। अ॒र्वाञ्च॒मिन्द्र॑म॒मुतो॑ हवामहे॒ यो गो॒जिद्ध॑न॒जिद॑श्व॒जिद्यः। इ॒मं नो॑ य॒ज्ञं वि॑ह॒वे जु॑षस्वा॒स्य कु॑र्मो हरिवो मे॒दिनं॑ त्वा॥~(३२)

{\anuvakamend[{व॒नि॒ष॒न्त॒ पु॒रस्ता॒न्मा त्रिच॑त्वारिꣳशच्च}]}%॥14॥

%4.7.15.1
अ॒ग्नेर्म॑न्वे प्रथ॒मस्य॒ प्रचे॑तसो॒ यं पाञ्च॑जन्यं ब॒हवः॑ समि॒न्धते᳚। विश्व॑स्यां वि॒शि प्र॑विविशि॒वाꣳस॑मीमहे॒ स नो॑ मुञ्च॒त्वꣳह॑सः। यस्ये॒दं प्रा॒णन्नि॑मि॒षद्यदेज॑ति॒ यस्य॑ जा॒तं जन॑मानं च॒ केव॑लम्। स्तौम्य॒ग्निं ना॑थि॒तो जो॑हवीमि॒ स नो॑ मुञ्च॒त्वꣳह॑सः। इन्द्र॑स्य मन्ये प्रथ॒मस्य॒ प्रचे॑तसो वृत्र॒घ्नः स्तोमा॒ उप॒ मामु॒पागुः॑। यो दा॒शुषः॑ सु॒कृतो॒ हव॒मुप॒ गन्ता᳚~(३३)

%4.7.15.2
स नो॑ मुञ्च॒त्वꣳह॑सः। यः स॑ङ्ग्रमं नय॑ति॒ सं व॒शी यु॒धे यः पु॒ष्टानि॑ सꣳसृ॒जति॑ त्र॒याणि॑। स्तौमीन्द्रं॑ नाथि॒तो जो॑हवीमि॒ स नो॑ मुञ्च॒त्वꣳह॑सः। म॒न्वे वां᳚ मित्रावरुणा॒ तस्य॑ वित्त॒ꣳ॒ सत्यौ॑जसा दृꣳहणा॒ यं नु॒देथे᳚। या राजा॑नꣳ स॒रथं॑ या॒थ उ॑ग्रा॒ ता नो॑ मुञ्चत॒माग॑सः। यो वा॒ꣳ॒ रथ॑ ऋ॒जुर॑श्मिः स॒त्यध॑र्मा॒ मिथु॒श्चर॑न्तमुप॒याति॑ दू॒षयन्न्॑। स्तौमि॑~(३४)

%4.7.15.3
मि॒त्रावरु॑णा नाथि॒तो जो॑हवीमि॒ तौ नो॑ मुञ्चत॒माग॑सः। वा॒योः स॑वि॒तुर्वि॒दथा॑नि मन्महे॒ यावा᳚त्म॒न्वद्बि॑भृ॒तो यौ च॒ रक्ष॑तः। यौ विश्व॑स्य परि॒भू ब॑भू॒वतु॒स्तौ नो॑ मुञ्चत॒माग॑सः। उप॒ श्रेष्ठा॑ न आ॒शिषो॑ दे॒वयो॒र्धर्मे॑ अस्थिरन्न्। स्तौमि॑ वा॒युꣳ स॑वि॒तारं॑ नाथि॒तो जो॑हवीमि॒ तौ नो॑ मुञ्चत॒माग॑सः। र॒थीत॑मौ रथी॒नाम॑ह्व ऊ॒तये॒ शुभं॒ गमि॑ष्ठौ सु॒यमे॑भि॒रश्वैः᳚। ययोः᳚~(३५)

%4.7.15.4
वां॒ दे॒वौ॒ दे॒वेष्वनि॑शित॒मोज॒स्तौ नो॑ मुञ्चत॒माग॑सः। यदया॑तं वह॒तुꣳ सू॒र्याया᳚स्त्रिच॒क्रेण॑ स॒ꣳ॒सद॑मि॒च्छमा॑नौ। स्तौमि॑ दे॒वाव॒श्विनौ॑ नाथि॒तो जो॑हवीमि॒ तौ नो॑ मुञ्चत॒माग॑सः। म॒रुतां᳚ मन्वे॒ अधि॑ नो ब्रुवन्तु॒ प्रेमां वाचं॒ विश्वा॑मवन्तु॒ विश्वे᳚। आ॒शून् हु॑वे सु॒यमा॑नू॒तये॒ ते नो॑ मुञ्च॒न्त्वेन॑सः। ति॒ग्ममायु॑धं वीडि॒तꣳ सह॑स्वद्दि॒व्यꣳ शर्धः॑~(३६)

%4.7.15.5
पृत॑नासु जि॒ष्णु। स्तौमि॑ दे॒वान्म॒रुतो॑ नाथि॒तो जो॑हवीमि॒ ते नो॑ मुञ्च॒न्त्वेन॑सः। दे॒वानां᳚ मन्वे॒ अधि॑ नो ब्रुवन्तु॒ प्रेमां वाचं॒ विश्वा॑मवन्तु॒ विश्वे᳚। आ॒शून् हु॑वे सु॒यमा॑नू॒तये॒ ते नो॑ मुञ्च॒न्त्वेन॑सः। यदि॒दं मा॑भि॒शोच॑ति॒ पौरु॑षेयेण॒ दैव्ये॑न। स्तौमि॒ विश्वा᳚न् दे॒वान्ना॑थि॒तो जो॑हवीमि॒ ते नो॑ मुञ्च॒न्त्वेन॑सः। अनु॑ नो॒\-ऽद्यानु॑मति॒रनु॑~(३७)

%4.7.15.6
इद॑नुमते॒ त्वं वै᳚श्वान॒रो न॑ ऊ॒त्या पृ॒ष्टो दि॒वि। ये अप्र॑थेता॒ममि॑तेभि॒रोजो॑भि॒र्ये प्र॑ति॒ष्ठे अभ॑वतां॒ वसू॑नाम्। स्तौमि॒ द्यावा॑पृथि॒वी ना॑थि॒तो जो॑हवीमि॒ ते नो॑ मुञ्चत॒मꣳह॑सः। उर्वी॑ रोदसी॒ वरि॑वः कृणोतं॒ क्षेत्र॑स्य पत्नी॒ अधि॑ नो ब्रूयातम्। स्तौमि॒ द्यावा॑पृथि॒वी ना॑थि॒तो जो॑हवीमि॒ ते नो॑ मुञ्चत॒मꣳह॑सः। यत्ते॑ व॒यं पु॑रुष॒त्रा य॑वि॒ष्ठावि॑द्वाꣳसश्चकृ॒मा कच्च॒न~(३८)

%4.7.15.7
आगः॑। कृ॒धी स्व॑स्माꣳ अदि॑ते॒रना॑गा॒ व्येनाꣳ॑सि शिश्रथो॒ विष्व॑गग्ने। यथा॑ ह॒ तद्व॑सवो गौ॒र्यं॑ चित्प॒दि षि॒ताममु॑ञ्चता यजत्राः। ए॒वा त्वम॒स्मत्प्र मु॑ञ्चा॒ व्यꣳहः॒ प्राता᳚र्यग्ने प्रत॒रां न॒ आयुः॑~(३९)


{\anuvakamend[{गन्ता॑ दू॒षय॒न्थ्स्तौमि॒ ययोः॒ शर्धो\-ऽनु॑मति॒रनु॑ च॒न चतु॑स्त्रिꣳशच्च}]}%॥15॥

{\anuvakamend[{अ॒ग्निष्ट्वा॑ वा॒मश्वो॒ द्विच॑त्वारिꣳशच्च}]}%॥11॥
%%% END PRASHNA
