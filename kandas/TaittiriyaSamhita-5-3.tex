\chapt{काण्डम् ५}
\sect{तृतीयः प्रश्नः}\setcounter{anuvakam}{0}
\dnsub{तैत्तिरीयसंहितायां पञ्चमकाण्डे तृतीयः प्रश्नः}
%5.3.1.1
उ॒थ्स॒न्न॒य॒ज्ञो वा ए॒ष यद॒ग्निः किं वाहै॒तस्य॑ क्रि॒यते॒ किं वा॒ न यद्वै य॒ज्ञस्य॑ क्रि॒यमा॑णस्यान्त॒र्यन्ति॒ पूय॑ति॒ वा अ॑स्य॒ तदा᳚श्वि॒नीरुप॑ दधात्य॒श्विनौ॒ वै दे॒वानां᳚ भि॒षजौ॒ ताभ्या॑मे॒वास्मै॑ भेष॒जं क॑रोति॒ पञ्चोप॑ दधाति॒ पाङ्क्तो॑ य॒ज्ञो यावा॑ने॒व य॒ज्ञस्तस्मै॑ भेष॒जं क॑रोत्यृत॒व्या॑ उप॑ दधात्यृतू॒नां कॢप्त्यै᳚~(१)

%5.3.1.2
पञ्चोप॑ दधाति पञ्च॒ वा ऋ॒तवो॒ याव॑न्त ए॒वर्तव॒स्तान्क॑ल्पयति समा॒नप्र॑भृतयो भवन्ति समा॒नोद॑र्का॒स्तस्मा᳚थ्समा॒ना ऋ॒तव॒ एके॑न प॒देन॒ व्याव॑र्तन्ते॒ तस्मा॑दृ॒तवो॒ व्याव॑र्तन्ते प्राण॒भृत॒ उप॑ दधात्यृ॒तुष्वे॒व प्रा॒णान्द॑धाति॒ तस्मा᳚थ्समा॒नाः सन्त॑ ऋ॒तवो॒ न जी᳚र्य॒न्त्यथो॒ प्र ज॑नयत्ये॒वैना॑ने॒ष वै वा॒युर्यत्प्रा॒णो यदृ॑त॒व्या॑ उप॒धाय॑ प्राण॒भृतः॑~(२)

%5.3.1.3
उ॒प॒दधा॑ति॒ तस्मा॒थ्सर्वा॑नृ॒तूननु॑ वा॒युरा व॑रीवर्त्ति वृष्टि॒सनी॒रुप॑ दधाति॒ वृष्टि॑मे॒वाव॑ रुन्धे॒ यदे॑क॒धोप॑द॒ध्यादेक॑मृ॒तुं व॑र्\mbox{}षेदनुपरि॒हारꣳ॑ सादयति॒ तस्मा॒थ्सर्वा॑नृ॒तून् व॑र्\mbox{}षति॒ यत्प्रा॑ण॒भृत॑ उप॒धाय॑ वृष्टि॒सनी॑रुप॒दधा॑ति॒ तस्मा᳚द्वा॒युप्र॑च्युता दि॒वो वृ॑ष्टिरीर्ते प॒शवो॒ वै व॑य॒स्या॑ नाना॑मनसः॒ खलु॒ वै प॒शवो॒ नाना᳚व्रता॒स्ते॑\-ऽप ए॒वाभि सम॑नसः~(३)

%5.3.1.4
यं का॒मये॑ताप॒शुः स्या॒दिति॑ वय॒स्या᳚स्तस्यो॑प॒धाया॑प॒स्या॑ उप॑ दध्या॒दसं᳚ज्ञानमे॒वास्मै॑ प॒शुभिः॑ करोत्यप॒शुरे॒व भ॑वति॒ यं का॒मये॑त पशु॒मान्थ्स्या॒दित्य॑प॒स्या᳚स्तस्यो॑प॒धाय॑ वय॒स्या॑ उप॑ दध्याथ्सं॒ज्ञान॑मे॒वास्मै॑ प॒शुभिः॑ करोति पशु॒माने॒व भ॑वति॒ चत॑स्रः पु॒रस्ता॒दुप॑ दधाति॒ तस्मा᳚च्च॒त्वारि॒ चक्षु॑षो रू॒पाणि॒ द्वे शु॒क्ले द्वे कृ॒ष्णे~(४)

%5.3.1.5
मू॒र्ध॒न्वती᳚र्भवन्ति॒ तस्मा᳚त्पु॒रस्ता᳚न्मू॒र्धा पञ्च॒ दक्षि॑णाया॒ꣴ॒ श्रोण्या॒मुप॑ दधाति॒ पञ्चोत्त॑रस्यां॒ तस्मा᳚त्प॒श्चाद्वर्\mbox{}षी॑यान् पु॒रस्ता᳚त्प्रवणः प॒शुर्ब॒स्तो वय॒ इति॒ दक्षि॒णे\-ऽꣳस॒ उप॑ दधाति वृ॒ष्णिर्वय॒ इत्युत्त॒रे\-ऽꣳसा॑वे॒व प्रति॑ दधाति व्या॒घ्रो वय॒ इति॒ दक्षि॑णे प॒क्ष उप॑ दधाति सि॒ꣳ॒हो वय॒ इत्युत्त॑रे प॒क्षयो॑रे॒व वी॒र्यं॑ दधाति॒ पुरु॑षो॒ वय॒ इति॒ मध्ये॒ तस्मा॒त्पुरु॑षः पशू॒नामधि॑पतिः॥~(५)

{\anuvakamend[{कॢप्त्या॑ उप॒धाय॑ प्राण॒भृतः॒ सम॑नसः कृ॒ष्णे पुरु॑षो॒ वय॒ इति॒ पञ्च॑ च}]}%~(१)

%5.3.2.1
इन्द्रा᳚ग्नी॒ अव्य॑थमाना॒मिति॑ स्वयमातृ॒ण्णामुप॑ दधातीन्द्रा॒ग्निभ्यां॒ वा इ॒मौ लो॒कौ विधृ॑ताव॒नयो᳚र्लो॒कयो॒र्विधृ॑त्या॒ अधृ॑तेव॒ वा ए॒षा यन्म॑ध्य॒मा चिति॑र॒न्तरि॑क्षमिव॒ वा ए॒षेन्द्रा᳚ग्नी॒ इत्या॑हेन्द्रा॒ग्नी वै दे॒वाना॑मोजो॒भृता॒वोज॑सै॒वैना॑\-म॒न्तरि॑क्षे चिनुते॒ धृत्यै᳚ स्वयमातृ॒ण्णामुप॑ दधात्य॒न्तरि॑क्षं॒ वै स्व॑यमातृ॒ण्णान्तरि॑क्षमे॒वोप॑ ध॒त्ते\-ऽश्व॒मुप॑~(६)

%5.3.2.2
घ्रा॒प॒य॒ति॒ प्रा॒णमे॒वास्यां᳚ दधा॒त्यथो᳚ प्राजाप॒त्यो वा अश्वः॑ प्र॒जा\-प॑तिनै॒वाग्निं चि॑नुते स्वयमातृ॒ण्णा भ॑वति प्रा॒णाना॒मुथ्सृ॑ष्ट्या॒ अथो॑ सुव॒र्गस्य॑ लो॒कस्यानु॑ख्यात्यै दे॒वानां॒ वै सु॑व॒र्गं लो॒कं य॒तां दिशः॒ सम॑व्लीयन्त॒ त ए॒ता दिश्या॑ अपश्य॒न्ता उपा॑दधत॒ ताभि॒र्वै ते दिशो॑\-ऽदृꣳह॒न्॒यद्दिश्या॑ उप॒दधा॑ति दि॒शां विधृ॑त्यै॒ दश॑ प्राण॒भृतः॑ पु॒रस्ता॒दुप॑~(७)

%5.3.2.3
द॒धा॒ति॒ नव॒ वै पुरु॑षे प्रा॒णा नाभि॑र्दश॒मी प्रा॒णाने॒व पु॒रस्ता᳚द्धत्ते॒ तस्मा᳚त्पु॒रस्ता᳚त्प्रा॒णा ज्योति॑ष्मतीमुत्त॒मामुप॑ दधाति॒ तस्मा᳚त्प्रा॒णानां॒ वाग्ज्योति॑रुत्त॒मा दशोप॑ दधाति॒ दशा᳚क्षरा वि॒राड्वि॒राट्छन्द॑सां॒ ज्योति॒र्ज्योति॑रे॒व पु॒रस्ता᳚द्धत्ते॒ तस्मा᳚त्पु॒रस्ता॒ज्ज्योति॒रुपा᳚स्महे॒ छन्दाꣳ॑सि प॒शुष्वा॒जिम॑यु॒स्तान्बृ॑ह॒त्युद॑जय॒त्तस्मा॒द्बार्\mbox{}ह॑ताः~(८)

%5.3.2.4
प॒शव॑ उच्यन्ते॒ मा छन्द॒ इति॑ दक्षिण॒त उप॑ दधाति॒ तस्मा᳚द्दक्षि॒णावृ॑तो॒ मासाः᳚ पृथि॒वी छन्द॒ इति॑ प॒श्चात्प्रति॑ष्ठित्या अ॒ग्निर्दे॒वतेत्यु॑त्तर॒त ओजो॒ वा अ॒ग्निरोज॑ ए॒वोत्त॑र॒तो ध॑त्ते॒ तस्मा॑दुत्तरतोभिप्रया॒यी ज॑यति॒ षट्त्रिꣳ॑श॒थ्सं प॑द्यन्ते॒ षट्त्रिꣳ॑शदक्षरा बृह॒ती बार्\mbox{}ह॑ताः प॒शवो॑ बृह॒त्यैवास्मै॑ प॒शूनव॑ रुन्धे बृह॒ती छन्द॑सा॒ꣴ॒ स्वारा᳚ज्यं॒ परी॑याय॒ यस्यै॒ताः~(९)

%5.3.2.5
उ॒प॒धी॒यन्ते॒ गच्छ॑ति॒ स्वारा᳚ज्यꣳ स॒प्त वाल॑खिल्याः पु॒रस्ता॒दुप॑ दधाति स॒प्त प॒श्चाथ्स॒प्त वै शी॑र्\mbox{}ष॒ण्याः᳚ प्रा॒णा द्वाववा᳚ञ्चौ प्रा॒णानाꣳ॑ सवीर्य॒त्वाय॑ मू॒र्धासि॒ राडिति॑ पु॒रस्ता॒दुप॑ दधाति॒ यन्त्री॒ राडिति॑ प॒श्चात्प्रा॒णाने॒वास्मै॑ स॒मीचो॑ दधाति॥~(१०)

{\anuvakamend[{अश्व॒मुप॑ पु॒रस्ता॒दुप॒ बार्\mbox{}ह॑ता ए॒ताश्चतु॑स्त्रिꣳशच्च}]}%~(२)

%5.3.3.1
दे॒वा वै यद्य॒ज्ञे\-ऽकु॑र्वत॒ तदसु॑रा अकुर्वत॒ ते दे॒वा ए॒ता अ॑क्ष्णयास्तो॒मीया॑ अपश्य॒न्ता अ॒न्यथा॒नूच्या॒न्यथोपा॑दधत॒ तदसु॑रा॒ नान्ववा॑य॒न्ततो॑ दे॒वा अभ॑व॒न्परासु॑रा॒ यद॑क्ष्णयास्तो॒मीया॑ अ॒न्यथा॒नूच्या॒न्यथो॑प॒दधा॑ति॒ भ्रातृ॑व्याभिभूत्यै॒ भव॑त्या॒त्मना॒ परा᳚स्य॒ भ्रातृ॑व्यो भवत्या॒शुस्त्रि॒वृदिति॑ पु॒रस्ता॒दुप॑ दधाति य॒ज्ञमु॒खं वै त्रि॒वृत्~(११)

%5.3.3.2
य॒ज्ञ॒मु॒खमे॒व पु॒रस्ता॒द्वि या॑तयति॒ व्यो॑म सप्तद॒श इति॑ दक्षिण॒तो\-ऽन्नं॒ वै व्यो॑मान्नꣳ॑ सप्तद॒शो\-ऽन्न॑मे॒व द॑क्षिण॒तो ध॑त्ते॒ तस्मा॒द्दक्षि॑णे॒नान्न॑मद्यते ध॒रुण॑ एकवि॒ꣳ॒श इति॑ प॒श्चात्प्र॑ति॒ष्ठा वा ए॑कवि॒ꣳ॒शः प्रति॑ष्ठित्यै भा॒न्तः प॑ञ्चद॒श इत्यु॑त्तर॒त ओजो॒ वै भा॒न्त ओजः॑ पञ्चद॒श ओज॑ ए॒वोत्त॑र॒तो ध॑त्ते॒ तस्मा॑दुत्तरतोभिप्रया॒यी ज॑यति॒ प्रतू᳚र्तिरष्टाद॒श इति॑ पु॒रस्ता᳚त्~(१२)

%5.3.3.3
उप॑ दधाति॒ द्वौ त्रि॒वृता॑वभिपू॒र्वं य॑ज्ञमु॒खे वि या॑तयत्यभिव॒र्तः स॑वि॒ꣳ॒श इति॑ दक्षिण॒तो\-ऽन्नं॒ वा अ॑भिव॒र्तो\-ऽन्नꣳ॑ सवि॒ꣳ॒शो\-ऽन्न॑मे॒व द॑क्षिण॒तो ध॑त्ते॒ तस्मा॒द्दक्षि॑णे॒नान्न॑मद्यते॒ वर्चो᳚ द्वावि॒ꣳ॒श इति॑ प॒श्चाद्यद्विꣳ॑श॒तिर्द्वे तेन॑ वि॒राजौ॒ यद्द्वे प्र॑ति॒ष्ठा तेन॑ वि॒राजो॑रे॒वाभि॑पू॒र्वम॒न्नाद्ये॒ प्रति॑ तिष्ठति॒ तपो॑ नवद॒श इत्यु॑त्तर॒तस्तस्मा᳚थ्स॒व्यः~(१३)

%5.3.3.4
हस्त॑योस्तप॒स्वित॑रो॒ योनि॑श्चतुर्वि॒ꣳ॒श इति॑ पु॒रस्ता॒दुप॑ दधाति॒ चतु॑र्विꣳशत्यक्षरा गाय॒त्री गा॑य॒त्री य॑ज्ञमु॒खं य॑ज्ञमु॒खमे॒व पु॒रस्ता॒द्वि या॑तयति॒ गर्भाः᳚ पञ्चवि॒ꣳ॒श इति॑ दक्षिण॒तो\-ऽन्नं॒ वै गर्भा॒ अन्नं॑ पञ्चवि॒ꣳ॒शो\-ऽन्न॑मे॒व द॑क्षिण॒तो ध॑त्ते॒ तस्मा॒द्दक्षि॑णे॒नान्न॑मद्यत॒ ओज॑स्त्रिण॒व इति॑ प॒श्चादि॒मे वै लो॒कास्त्रि॑ण॒व ए॒ष्वे॑व लो॒केषु॒ प्रति॑ तिष्ठति स॒म्भर॑णस्त्रयोवि॒ꣳ॒श इति॑~(१४)

%5.3.3.5
उ॒त्त॒र॒तस्तस्मा᳚थ्स॒व्यो हस्त॑योः सम्भा॒र्य॑तरः॒ क्रतु॑रेकत्रि॒ꣳ॒श इति॑ पु॒रस्ता॒दुप॑ दधाति॒ वाग्वै क्रतु॑र्यज्ञमु॒खं वाग्य॑ज्ञमु॒खमे॒व पु॒रस्ता॒द्वि या॑तयति ब्र॒ध्नस्य॑ वि॒ष्टपं॑ चतुस्त्रि॒ꣳ॒श इति॑ दक्षिण॒तो॑\-ऽसौ वा आ॑दि॒त्यो ब्र॒ध्नस्य॑ वि॒ष्टपं॑ ब्रह्मवर्च॒समे॒व द॑क्षिण॒तो ध॑त्ते॒ तस्मा॒द्दक्षि॒णो\-ऽर्धो᳚ ब्रह्मवर्च॒सित॑रः प्रति॒ष्ठा त्र॑यस्त्रि॒ꣳ॒श इति॑ प॒श्चात्प्रति॑ष्ठित्यै॒ नाकः॑ षट्त्रि॒ꣳ॒श इत्यु॑त्तर॒तः सु॑व॒र्गो वै लो॒को नाकः॑ सुव॒र्गस्य॑ लो॒कस्य॒ सम॑ष्ट्यै॥~(१५)

{\anuvakamend[{वै त्रि॒वृदिति॑ पु॒रस्ता᳚थ्स॒व्यस्त्र॑योवि॒ꣳ॒श इति॑ सुव॒र्गो वै पञ्च॑ च}]}%~(३) आ॒शुर्व्यो॑म ध॒रुणो॑ भा॒न्तः प्रतू᳚र्तिरभिव॒र्तो वर्च॒स्तपो॒ योनि॒र्गर्भा॒ ओजः॑ स॒म्भर॑णः॒ क्रतु॑र्ब्र॒ध्नस्य॑ प्रति॒ष्ठा नाकः॒ षोड॑श॥

%5.3.4.1
अ॒ग्नेर्भा॒गो॑\-ऽसीति॑ पु॒रस्ता॒दुप॑ दधाति यज्ञमु॒खं वा अ॒ग्निर्य॑ज्ञमु॒खं दी॒क्षा य॑ज्ञमु॒खं ब्रह्म॑ यज्ञमु॒खं त्रि॒वृद्य॑ज्ञमु॒खमे॒व पु॒रस्ता॒द्वि या॑तयति नृ॒चक्ष॑सां भा॒गो॑\-ऽसीति॑ दक्षिण॒तः शु॑श्रु॒वाꣳसो॒ वै नृ॒चक्ष॒सो\-ऽन्नं॑ धा॒ता जा॒तायै॒वास्मा॒ अन्न॒मपि॑ दधाति॒ तस्मा᳚ज्जा॒तो\-ऽन्न॑मत्ति ज॒नित्रꣴ॑ स्पृ॒तꣳ स॑प्तद॒शः स्तोम॒ इत्या॒हान्नं॒ वै ज॒नित्रम्᳚~(१६)

%5.3.4.2
अन्नꣳ॑ सप्तद॒शो\-ऽन्न॑मे॒व द॑क्षिण॒तो ध॑त्ते॒ तस्मा॒द्दक्षि॑णे॒नान्न॑मद्यते मि॒त्रस्य॑ भा॒गो॑\-ऽसीति॑ प॒श्चात्प्रा॒णो वै मि॒त्रो॑\-ऽपा॒नो वरु॑णः प्राणापा॒नावे॒वास्मि॑न्दधाति दि॒वो वृ॒ष्टिर्वाताः᳚ स्पृ॒ता ए॑कवि॒ꣳ॒शः स्तोम॒ इत्या॑ह प्रति॒ष्ठा वा ए॑कवि॒ꣳ॒शः प्रति॑ष्ठित्या॒ इन्द्र॑स्य भा॒गो॑\-ऽसीत्यु॑त्तर॒त ओजो॒ वा इन्द्र॒ ओजो॒ विष्णु॒रोजः॑ क्ष॒त्रमोजः॑ पञ्चद॒शः~(१७)

%5.3.4.3
ओज॑ ए॒वोत्त॑र॒तो ध॑त्ते॒ तस्मा॑दुत्तरतोभिप्रया॒यी ज॑यति॒ वसू॑नां भा॒गो॑\-ऽसीति॑ पु॒रस्ता॒दुप॑ दधाति यज्ञमु॒खं वै वस॑वो ॑यज्ञमु॒खꣳ रु॒द्रा य॑ज्ञमु॒खं च॑तुर्वि॒ꣳ॒शो य॑ज्ञमु॒खमे॒व पु॒रस्ता॒द्वि या॑तयत्यादि॒त्यानां᳚ भा॒गो॑\-ऽसीति॑ दक्षिण॒तो\-ऽन्नं॒ वा आ॑दि॒त्या अन्नं॑ म॒रुतो\-ऽन्नं॒ गर्भा॒ अन्नं॑ पञ्चवि॒ꣳ॒शो\-ऽन्न॑मे॒व द॑क्षिण॒तो ध॑त्ते॒ तस्मा॒द्दक्षि॑णे॒नान्न॑मद्य॒ते\-ऽदि॑त्यै भा॒गः~(१८)

%5.3.4.4
अ॒सीति॑ प॒श्चात्प्र॑ति॒ष्ठा वा अदि॑तिः प्रति॒ष्ठा पू॒षा प्र॑ति॒ष्ठा त्रि॑ण॒वः प्रति॑ष्ठित्यै दे॒वस्य॑ सवि॒तुर्भा॒गो॑\-ऽसीत्यु॑त्तर॒तो ब्रह्म॒ वै दे॒वः स॑वि॒ता ब्रह्म॒ बृह॒स्पति॒र्ब्रह्म॑ चतुष्टो॒मो ब्र॑ह्मवर्च॒समे॒वोत्त॑र॒तो ध॑त्ते॒ तस्मा॒दुत्त॒रो\-ऽर्धो᳚ ब्रह्मवर्च॒सित॑रः सावि॒त्रव॑ती भवति॒ प्रसू᳚त्यै॒ तस्मा᳚द्ब्राह्म॒णाना॒मुदी॑ची स॒निः प्रसू॑ता ध॒र्त्रश्च॑तुष्टो॒म इति॑ पु॒रस्ता॒दुप॑ दधाति यज्ञमु॒खं वै ध॒र्त्रः~(१९)

%5.3.4.5
य॒ज्ञ॒मु॒खं च॑तुष्टो॒मो य॑ज्ञमु॒खमे॒व पु॒रस्ता॒द्वि या॑तयति॒ यावा॑नां भा॒गो॑\-ऽसीति॑ दक्षिण॒तो मासा॒ वै यावा॑ अर्धमा॒सा अया॑वा॒स्तस्मा᳚द्दक्षि॒णावृ॑तो॒ मासा॒ अन्नं॒ वै यावा॒ अन्नं॑ प्र॒जा अन्न॑मे॒व द॑क्षिण॒तो ध॑त्ते॒ तस्मा॒द्दक्षि॑णे॒नान्न॑मद्यत ऋभू॒णां भा॒गो॑\-ऽसीति॑ प॒श्चात् प्रति॑ष्ठित्यै विव॒र्तो᳚\-ऽष्टाचत्वारि॒ꣳ॒श इत्यु॑त्तर॒तो॑\-ऽनयो᳚र्लो॒कयोः᳚ सवीर्य॒त्वाय॒ तस्मा॑दि॒मौ लो॒कौ स॒माव॑द्वीर्यौ~(२०)

%5.3.4.6
यस्य॒ मुख्य॑वतीः पु॒रस्ता॑दुपधी॒यन्ते॒ मुख्य॑ ए॒व भ॑व॒त्यास्य॒ मुख्यो॑ जायते॒ यस्यान्न॑वतीर्दक्षिण॒तो\-ऽत्त्यन्न॒मास्या᳚न्ना॒दो जा॑यते॒ यस्य॑ प्रति॒ष्ठाव॑तीः प॒श्चात्प्रत्ये॒व ति॑ष्ठति॒ यस्यौज॑स्वतीरुत्तर॒त ओ॑ज॒स्व्ये॑व भ॑व॒त्यास्यौ॑ज॒स्वी जा॑यते॒\-ऽर्को वा ए॒ष यद॒ग्निस्तस्यै॒तदे॒व स्तो॒त्रमे॒तच्छ॒स्त्रं यदे॒षा वि॒धा~(२१)

%5.3.4.7
वि॒धी॒यते॒\-ऽर्क ए॒व तद॒र्क्य॑मनु॒ वि धी॑य॒ते\-ऽत्त्यन्न॒मास्या᳚न्ना॒दो जा॑यते॒ यस्यै॒षा वि॒धा वि॑धी॒यते॒ य उ॑ चैनामे॒वं वेद॒ सृष्टी॒रुप॑ दधाति यथासृ॒ष्टमे॒वाव॑ रुन्धे॒ न वा इ॒दं दिवा॒ न नक्त॑मासी॒दव्या॑वृत्त॒न्ते दे॒वा ए॒ता व्यु॑ष्टीरपश्य॒न्ता उपा॑दधत॒ ततो॒ वा इ॒दं व्यौ᳚च्छ॒द्यस्यै॒ता उ॑पधी॒यन्ते॒ व्ये॑वास्मा॑ उच्छ॒त्यथो॒ तम॑ ए॒वाप॑ हते॥~(२२)

{\anuvakamend[{वै ज॒नित्रं॑ पञ्चद॒शो\-ऽदि॑त्यै भा॒गो वै ध॒र्त्रः स॒माव॑द्वीर्यौ वि॒धा ततो॒ वा इ॒दं चतु॑र्दश च}]}%~(४) अ॒ग्नेर्नृ॒चक्ष॑साञ्ज॒नित्रं॑ मि॒त्रस्येन्द्र॑स्य॒ वसू॑नामादि॒त्याना॒मदि॑त्यै दे॒वस्य॑ सवि॒तुः सा॑वि॒त्रव॑ती ध॒र्त्रो यावा॑नामृभू॒णां वि॑व॒र्तश्चतु॑र्दश॥

%5.3.5.1
अग्ने॑ जा॒तान्प्र णु॑दा नः स॒पत्ना॒निति॑ पु॒रस्ता॒दुप॑ दधाति जा॒ताने॒व भ्रातृ॑व्या॒न्प्र णु॑दते॒ सह॑सा जा॒तानिति॑ प॒श्चाज्ज॑नि॒ष्यमा॑णाने॒व प्रति॑ नुदते चतुश्चत्वारि॒ꣳ॒शः स्तोम॒ इति॑ दक्षिण॒तो ब्र॑ह्मवर्च॒सं वै च॑तुश्चत्वारि॒ꣳ॒शो ब्र॑ह्मवर्च॒समे॒व द॑क्षिण॒तो ध॑त्ते॒ तस्मा॒द्दक्षि॒णो\-ऽर्धो᳚ ब्रह्मवर्च॒सित॑रः षोड॒शः स्तोम॒ इत्यु॑त्तर॒त ओजो॒ वै षो॑ड॒श ओज॑ ए॒वोत्त॑र॒तो ध॑त्ते॒ तस्मा᳚त्~(२३)

%5.3.5.2
उ॒त्त॒र॒तो॒भि॒प्र॒या॒यी ज॑यति॒ वज्रो॒ वै च॑तुश्चत्वारि॒ꣳ॒शो वज्रः॑ षोड॒शो यदे॒ते इष्ट॑के उप॒दधा॑ति जा॒ताꣴश्चै॒व ज॑नि॒ष्यमा॑णाꣴश्च॒ भ्रातृ॑व्यान्प्र॒णुद्य॒ वज्र॒मनु॒ प्र ह॑रति॒ स्तृत्यै॒ पुरी॑षवतीं॒ मध्य॒ उप॑ दधाति॒ पुरी॑षं॒ वै मध्य॑मा॒त्मनः॒ सात्मा॑नमे॒वाग्निं चि॑नुते॒ सात्मा॒मुष्मिँ॑ल्लो॒के भ॑वति॒ य ए॒वं वेदै॒ता वा अ॑सप॒त्ना नामेष्ट॑का॒ यस्यै॒ता उ॑पधी॒यन्ते᳚~(२४)

%5.3.5.3
नास्य॑ स॒पत्नो॑ भवति प॒शुर्वा ए॒ष यद॒ग्निर्वि॒राज॑ उत्त॒मायां॒ चित्या॒मुप॑ दधाति वि॒राज॑मे॒वोत्त॒मां प॒शुषु॑ दधाति॒ तस्मा᳚त्पशु॒मानु॑त्त॒मां वाचं॑ वदति॒ दश॑द॒शोप॑ दधाति सवीर्य॒\-त्वा\-या᳚क्ष्ण॒\-योप॑ दधाति॒ तस्मा॑दक्ष्ण॒या प॒शवो\-ऽङ्गा॑नि॒ प्र ह॑रन्ति॒ प्रति॑ष्ठित्यै॒ यानि॒ वै छन्दाꣳ॑सि सुव॒र्ग्या᳚ण्यास॒न्तैर्दे॒वाः सु॑व॒र्गं लो॒कमा॑य॒न्तेनर्\mbox{}ष॑यः~(२५)

%5.3.5.4
अ॒श्रा॒म्य॒न्ते तपो॑\-ऽतप्यन्त॒ तानि॒ तप॑सापश्य॒न्तेभ्य॑ ए॒ता इष्ट॑का॒ निर॑मिम॒तेव॒श्छन्दो॒ वरि॑व॒श्छन्द॒ इति॒ ता उपा॑दधत॒ ताभि॒र्वै ते सु॑व॒र्गं लो॒कमा॑य॒न्॒ यदे॒ता इष्ट॑का उप॒दधा॑ति॒ यान्ये॒व छन्दाꣳ॑सि सुव॒र्ग्या॑णि॒ तैरे॒व यज॑मानः सुव॒र्गं लो॒कमे॑ति य॒ज्ञेन॒ वै प्र॒जा\-प॑तिः प्र॒जा अ॑सृजत॒ ताः स्तोम॑भागैरे॒वासृ॑जत॒ यत्~(२६)

%5.3.5.5
स्तोम॑भागा उप॒दधा॑ति प्र॒जा ए॒व तद्यज॑मानः सृजते॒ बृह॒स्पति॒र्वा ए॒तद्य॒ज्ञस्य॒ तेजः॒ सम॑भर॒द्यथ्स्तोम॑भागा॒ यथ्स्तोम॑भागा उप॒दधा॑ति॒ सते॑जसमे॒वाग्निं चि॑नुते॒ बृह॒स्पति॒र्वा ए॒तां य॒ज्ञस्य॑ प्रति॒ष्ठाम॑पश्य॒द्यथ्स्तोम॑भागा॒ यथ्स्तोम॑भागा उप॒दधा॑ति य॒ज्ञस्य॒ प्रति॑ष्ठित्यै स॒प्तस॒प्तोप॑ दधाति सवीर्य॒त्वाय॑ ति॒स्रो मध्ये॒ प्रति॑ष्ठित्यै॥~(२७)

{\anuvakamend[{उ॒त्त॒र॒तो ध॑त्ते॒ तस्मा॑दुपधी॒यन्त॒ ऋष॑यो\-ऽसृजत॒ यत्त्रिच॑त्वारिꣳशच्च}]}%~(५)

%5.3.6.1
र॒श्मिरित्ये॒वा\-ऽऽ\-दि॒त्यम॑सृजत॒ प्रेति॒रिति॒ धर्म॒मन्वि॑ति॒रिति॒ दिवꣳ॑ सं॒धिरित्य॒न्तरि॑क्षं प्रति॒धिरिति॑ पृथि॒वीं वि॑ष्ट॒म्भ इति॒ वृष्टिं॑ प्र॒वेत्यह॑रनु॒वेति॒ रात्रि॑मु॒शिगिति॒ वसू᳚न्प्रके॒त इति॑ रु॒द्रान्थ्सु॑दी॒तिरित्या॑दि॒त्यानोज॒ इति॑ पि॒तॄꣴस्तन्तु॒रिति॑ प्र॒जाः पृ॑तना॒षाडिति॑ प॒शून्रे॒वदित्योष॑धीरभि॒जिद॑सि यु॒क्तग्रा॑वा~(२८)

%5.3.6.2
इन्द्रा॑य॒ त्वेन्द्रं॑ जि॒न्वेत्ये॒व द॑क्षिण॒तो वज्रं॒ पर्यौ॑हद॒भिजि॑त्यै॒ ताः प्र॒जा अप॑प्राणा असृजत॒ तास्वधि॑पतिर॒सीत्ये॒व प्रा॒णम॑दधाद्य॒न्तेत्य॑पा॒नꣳ स॒ꣳ॒सर्प॒ इति॒ चक्षु॑र्वयो॒धा इति॒ श्रोत्र॒न्ताः प्र॒जाः प्रा॑ण॒तीर॑पान॒तीः पश्य॑न्तीः शृण्व॒तीर्न मि॑थु॒नी अ॑भव॒न्तासु॑ त्रि॒वृद॒सीत्ये॒व मि॑थु॒नम॑दधा॒त्ताः प्र॒जा मि॑थु॒नी~(२९)

%5.3.6.3
भव॑न्ती॒र्न प्राजा॑यन्त॒ ताः सꣳ॑रो॒हो॑\-ऽसि नीरो॒हो॑\-ऽसीत्ये॒व प्राज॑नय॒त्ताः प्र॒जाः प्रजा॑ता॒ न प्रत्य॑तिष्ठ॒न्ता व॑सु॒को॑\-ऽसि॒ वेष॑श्रिरसि॒ वस्य॑ष्टिर॒सीत्ये॒वैषु लो॒केषु॒ प्रत्य॑स्थापय॒द्यदाह॑ वसु॒को॑\-ऽसि॒ वेष॑श्रिरसि॒ वस्य॑ष्टिर॒सीति॑ प्र॒जा ए॒व प्रजा॑ता ए॒षु लो॒केषु॒ प्रति॑\-ष्ठापयति॒ सात्मा॒न्तरि॑क्षꣳ रोहति॒ सप्रा॑णो॒\-ऽमुष्मिँ॑ल्लो॒के प्रति॑ तिष्ठ॒त्यव्य॑र्धुकः प्राणापा॒ना\-भ्यां᳚ भवति॒ य ए॒वं वेद॑॥~(३०)

{\anuvakamend[{यु॒क्तग्रा॑वा प्र॒जा मि॑थु॒न्य॑न्तरि॑क्षं॒ द्वाद॑श च}]}%~(६)

%5.3.7.1
ना॒क॒सद्भि॒र्वै दे॒वाः सु॑व॒र्गं लो॒कमा॑य॒न्तन्ना॑क॒सदां᳚ नाकस॒त्त्वं यन्ना॑क॒सद॑ उप॒दधा॑ति नाक॒सद्भि॑रे॒व तद्यज॑मानः सुव॒र्गं लो॒कमे॑ति सुव॒र्गो वै लो॒को नाको॒ यस्यै॒ता उ॑पधी॒यन्ते॒ नास्मा॒ अकं॑ भवति यजमानायत॒नं वै ना॑क॒सदो॒ यन्ना॑क॒सद॑ उप॒दधा᳚त्या॒यत॑नमे॒व तद्यज॑मानः कुरुते पृ॒ष्ठानां॒ वा ए॒तत्तेजः॒ सम्भृ॑तं॒ यन्ना॑क॒सदो॒ यन्ना॑क॒सदः॑~(३१)

%5.3.7.2
उ॒प॒दधा॑ति पृ॒ष्ठाना॑मे॒व तेजो\-ऽव॑ रुन्धे पञ्च॒चोडा॒ उप॑ दधात्यफ्स॒रस॑ ए॒वैन॑मे॒ता भू॒ता अ॒मुष्मिँ॑ल्लो॒क उप॑ शे॒रे\-ऽथो॑ तनू॒पानी॑रे॒वैता यज॑मानस्य॒ यं द्वि॒ष्यात्तमु॑प॒दध॑द्ध्यायेदे॒ताभ्य॑ ए॒वैनं॑ दे॒वता᳚भ्य॒ आ वृ॑श्चति ता॒जगार्ति॒मार्च्छ॒त्युत्त॑रा नाक॒सद्भ्य॒ उप॑ दधाति॒ यथा॑ जा॒यामा॒नीय॑ गृ॒हेषु॑ निषा॒दय॑ति ता॒दृगे॒व तत्~(३२)

%5.3.7.3
प॒श्चात्प्राची॑मुत्त॒मामुप॑ दधाति॒ तस्मा᳚त्प॒श्चात्प्राची॒ पत्न्यन्वा᳚स्ते स्वयमातृ॒ण्णां च॑ विक॒र्णीं चो᳚त्त॒मे उप॑ दधाति प्रा॒णो वै स्व॑यमातृ॒ण्णायु॑र्विक॒र्णी प्रा॒णं चै॒वायु॑श्च प्रा॒णाना॑मुत्त॒मौ ध॑त्ते॒ तस्मा᳚त्प्रा॒णश्चायु॑श्च प्रा॒णाना॑मुत्त॒मौ नान्यामुत्त॑रा॒मिष्ट॑का॒मुप॑ दध्या॒द्यद॒न्यामुत्त॑रा॒मिष्ट॑कामुपद॒ध्यात्प॑शू॒नाम्~(३३)

%5.3.7.4
च॒ यज॑मानस्य च प्रा॒णं चायु॒श्चापि॑ दध्या॒त्तस्मा॒न्नान्योत्त॒रेष्ट॑कोप॒धेया᳚ स्वयमातृ॒ण्णामुप॑ दधात्य॒सौ वै स्व॑यमातृ॒ण्णामूमे॒वोप॑ ध॒त्ते\-ऽश्व॒मुप॑ घ्रापयति प्रा॒णमे॒वास्यां᳚ दधा॒त्यथो᳚ प्राजाप॒त्यो वा अश्वः॑ प्र॒जा\-प॑तिनै॒वाग्निं चि॑नुते स्वयमातृ॒ण्णा भ॑वति प्रा॒णाना॒मुथ्सृ॑ष्ट्या॒ अथो॑ सुव॒र्गस्य॑ लो॒कस्यानु॑ख्यात्या ए॒षा वै दे॒वानां॒ विक्रा᳚न्ति॒र्यद्वि॑क॒र्णी यद्वि॑क॒र्णीमु॑प॒दधा॑ति दे॒वाना॑मे॒व विक्रा᳚न्ति॒मनु॒ वि क्र॑मत उत्तर॒त उप॑ दधाति॒ तस्मा॑दुत्तर॒तउ॑पचारो॒\-ऽग्निर्वा॑यु॒मती॑ भवति॒ समि॑द्ध्यै॥~(३४)

{\anuvakamend[{सम्भृ॑तं॒ यन्ना॑क॒सदो॒ यन्ना॑क॒सद॒स्तत्प॑शू॒नामे॒षां वै द्वाविꣳ॑शतिश्च}]}%~(७)

%5.3.8.1
छन्दा॒ꣴ॒स्युप॑ दधाति प॒शवो॒ वै छन्दाꣳ॑सि प॒शूने॒वाव॑ रुन्धे॒ छन्दाꣳ॑सि॒ वै दे॒वानां᳚ वा॒मं प॒शवो॑ वा॒ममे॒व प॒शूनव॑ रुन्ध ए॒ताꣳ ह॒ वै य॒ज्ञसे॑नश्चैत्रियाय॒णश्चितिं॑ वि॒दां च॑कार॒ तया॒ वै स प॒शूनवा॑रुन्ध॒ यदे॒तामु॑प॒दधा॑ति प॒शूने॒वाव॑ रुन्धे गाय॒त्रीः पु॒रस्ता॒दुप॑ दधाति॒ तेजो॒ वै गा॑य॒त्री तेज॑ ए॒व~(३५)

%5.3.8.2
मु॒ख॒तो ध॑त्ते मूर्ध॒न्वती᳚र्भवन्ति मू॒र्धान॑मे॒वैनꣳ॑ समा॒नानां᳚ करोति त्रि॒ष्टुभ॒ उप॑ दधातीन्द्रि॒यं वै त्रि॒ष्टुगि॑न्द्रि॒यमे॒व म॑ध्य॒तो ध॑त्ते॒ जग॑ती॒रुप॑ दधाति॒ जाग॑ता॒ वै प॒शवः॑ प॒शूने॒वाव॑ रुन्धे\-ऽनु॒ष्टुभ॒ उप॑ दधाति प्रा॒णा वा अ॑नु॒ष्टुप्प्रा॒णाना॒मुथ्सृ॑ष्ट्यै बृह॒तीरु॒ष्णिहाः᳚ प॒ङ्क्तीर॒क्षर॑पङ्क्ती॒रिति॒ विषु॑रूपाणि॒ छन्दा॒ꣴ॒स्युप॑ दधाति॒ विषु॑रूपा॒ वै प॒शवः॑ प॒शवः॑~(३६)

%5.3.8.3
छन्दाꣳ॑सि॒ विषु॑रूपाने॒व प॒शूनव॑ रुन्धे॒ विषु॑रूपमस्य गृ॒हे दृ॑श्यते॒ यस्यै॒ता उ॑पधी॒यन्ते॒ य उ॑ चैना ए॒वं वेदाति॑च्छन्दस॒मुप॑ दधा॒त्यति॑च्छन्दा॒ वै सर्वा॑णि॒ छन्दाꣳ॑सि॒ सर्वे॑भिरे॒वैनं॒ छन्दो॑भिश्चिनुते॒ वर्ष्म॒ वा ए॒षा छन्द॑सां॒ यदति॑च्छन्दा॒ यदति॑च्छन्दसमुप॒दधा॑ति॒ वर्ष्मै॒वैनꣳ॑ समा॒नानां᳚ करोति द्वि॒पदा॒ उप॑ दधाति द्वि॒पाद्यज॑मानः॒ प्रति॑ष्ठित्यै॥~(३७)

{\anuvakamend[{तेज॑ ए॒व प॒शवः॑ प॒शवो॒ यज॑मान॒ एक॑ञ्च}]}%~(८)

%5.3.9.1
सर्वा᳚भ्यो॒ वै दे॒वता᳚भ्यो॒\-ऽग्निश्ची॑यते॒ यथ्स॒युजो॒ नोप॑द॒ध्याद्दे॒वता॑ अस्या॒ग्निं वृ॑ञ्जीर॒न्॒ यथ्स॒युज॑ उप॒दधा᳚त्या॒त्मनै॒वैनꣳ॑ स॒युजं॑ चिनुते॒ नाग्निना॒ व्यृ॑ध्य॒ते\-ऽथो॒ यथा॒ पुरु॑षः॒ स्नाव॑भिः॒ सन्त॑त ए॒वमे॒वैताभि॑र॒ग्निः सन्त॑तो॒\-ऽग्निना॒ वै दे॒वाः सु॑व॒र्गं लो॒कमा॑य॒न्ता अ॒मूः कृ॑त्तिका अभव॒न्॒ यस्यै॒ता उ॑पधी॒यन्ते॑ सुव॒र्गमे॒व~(३८)

%5.3.9.2
लो॒कमे॑ति॒ गच्छ॑ति प्रका॒शं चि॒त्रमे॒व भ॑वति मण्डलेष्ट॒का उप॑ दधाती॒मे वै लो॒का म॑ण्डलेष्ट॒का इ॒मे खलु॒ वै लो॒का दे॑वपु॒रा दे॑वपु॒रा ए॒व प्र वि॑शति॒ नार्ति॒मार्च्छ॑त्य॒ग्निं चि॑क्या॒नो वि॒श्वज्यो॑तिष॒ उप॑ दधाती॒माने॒वैताभि॑र्लो॒कां ज्योति॑ष्मतः कुरु॒ते\-ऽथो᳚ प्रा॒णाने॒वैता यज॑मानस्य दाध्रत्ये॒ता वै दे॒वताः᳚ सुव॒र्ग्या᳚स्ता ए॒वान्वा॒रभ्य॑ सुव॒र्गं लो॒कमे॑ति॥~(३९)

{\anuvakamend[{सु॒व॒र्गमे॒व ता ए॒व च॒त्वारि॑ च}]}%~(९)

%5.3.10.1
वृ॒ष्टि॒सनी॒रुप॑ दधाति॒ वृष्टि॑मे॒वाव॑ रुन्धे॒ यदे॑क॒धोप॑द॒ध्यादेक॑मृ॒तुं व॑र्\mbox{}षेदनुपरि॒हारꣳ॑ सादयति॒ तस्मा॒थ्सर्वा॑नृ॒तून् व॑र्\mbox{}षति पुरोवात॒सनि॑र॒सीत्या॑है॒तद्वै वृष्ट्यै॑ रू॒पꣳ रू॒पेणै॒व वृष्टि॒मव॑ रुन्धे सं॒यानी॑भि॒र्वै दे॒वा इ॒माँल्लो॒कान्थ्सम॑यु॒स्तथ्सं॒यानी॑नाꣳ संयानि॒त्वं यथ्सं॒यानी॑रुप॒दधा॑ति॒ यथा॒फ्सु ना॒वा सं॒यात्ये॒वम्~(४०)

%5.3.10.2
ए॒वैताभि॒र्यज॑मान इ॒माँल्लो॒कान्थ्सं या॑ति प्ल॒वो वा ए॒षो᳚\-ऽग्नेर्यथ्सं॒यानी॒र्यथ्सं॒यानी॑रुप॒दधा॑ति प्ल॒वमे॒वैतम॒ग्नय॒ उप॑ दधात्यु॒त यस्यै॒तासूप॑हिता॒स्वापो॒\-ऽग्निꣳ हर॒न्त्यहृ॑त ए॒वास्या॒ग्निरा॑दित्येष्ट॒का उप॑ दधात्यादि॒त्या वा ए॒तं भूत्यै॒ प्रति॑ नुदन्ते॒ यो\-ऽलं॒ भूत्यै॒ सन्भूतिं॒ न प्रा॒प्नोत्या॑दि॒त्याः~(४१)

%5.3.10.3
ए॒वैनं॒ भूतिं॑ गमयन्त्य॒सौ वा ए॒तस्या॑दि॒त्यो रुच॒मा द॑त्ते॒ यो᳚\-ऽग्निं चि॒त्वा न रोच॑ते॒ यदा॑दित्येष्ट॒का उ॑प॒दधा᳚त्य॒सावे॒वास्मि॑न्नादि॒त्यो रुचं॑ दधाति॒ यथा॒सौ दे॒वाना॒ꣳ॒ रोच॑त ए॒वमे॒वैष म॑नु॒ष्या॑णाꣳ रोचते घृतेष्ट॒का उप॑ दधात्ये॒तद्वा अ॒ग्नेः प्रि॒यं धाम॒ यद् घृ॒तं प्रि॒येणै॒वैनं॒ धाम्ना॒ सम॑र्धयति~(४२)

%5.3.10.4
अथो॒ तेज॑सानुपरि॒हारꣳ॑ सादय॒त्यप॑रिवर्गमे॒वास्मि॒न्तेजो॑ दधाति प्र॒जा\-प॑तिर॒ग्निम॑चिनुत॒ स यश॑सा॒ व्या᳚र्ध्यत॒ स ए॒ता य॑शो॒दा अ॑पश्य॒त्ता उपा॑धत्त॒ ताभि॒र्वै स यश॑ आ॒त्मन्न॑धत्त॒ यद्य॑शो॒दा उ॑प॒दधा॑ति॒ यश॑ ए॒व ताभि॒र्यज॑मान आ॒त्मन्ध॑त्ते॒ पञ्चोप॑ दधाति॒ पाङ्क्तः॒ पुरु॑षो॒ यावा॑ने॒व पुरु॑ष॒स्तस्मि॒न्॒ यशो॑ दधाति॥~(४३)

{\anuvakamend[{ए॒वं प्रा॒प्नोत्या॑दि॒त्या अ॑र्धय॒त्येका॒न्नप॑ञ्चा॒शच्च॑}]}%॥10॥

%5.3.11.1
दे॒वा॒सु॒राः संय॑त्ता आस॒न्कनी॑याꣳसो दे॒वा आस॒न्भूया॒ꣳ॒सो\-ऽसु॑रा॒स्ते दे॒वा ए॒ता इष्ट॑का अपश्य॒न्ता उपा॑दधत भूय॒स्कृद॒सीत्ये॒व भूयाꣳ॑सो\-ऽभव॒न्वन॒स्पति॑भि॒रोष॑धीभिर्वरिव॒स्कृद॒सीती॒माम॑जय॒न्प्राच्य॒सीति॒ प्राचीं॒ दिश॑मजयन्नू॒र्ध्वासीत्य॒मूम॑जयन्नन्तरिक्ष॒सद॑स्य॒न्तरि॑क्षे सी॒देत्य॒न्तरि॑क्षमजय॒न्ततो॑ दे॒वा अभ॑वन्न्~(४४)

%5.3.11.2
परासु॑रा॒ यस्यै॒ता उ॑पधी॒यन्ते॒ भूया॑ने॒व भ॑वत्य॒भीमाँल्लो॒काञ्ज॑यति॒ भव॑त्या॒त्मना॒ परा᳚स्य॒ भ्रातृ॑व्यो भवत्यफ्सु॒षद॑सि श्येन॒सद॒सीत्या॑है॒तद्वा अ॒ग्ने रू॒पꣳ रू॒पेणै॒वाग्निमव॑ रुन्धे पृथि॒व्यास्त्वा॒ द्रवि॑णे सादया॒मीत्या॑हे॒माने॒वैताभि॑र्लो॒कान् द्रवि॑णावतः कुरुत आयु॒ष्या॑ उप॑ दधा॒त्यायु॑रे॒व~(४५)

%5.3.11.3
अ॒स्मि॒न्द॒धा॒त्यग्ने॒ यत्ते॒ पर॒ꣳ॒ हृन्नामेत्या॑है॒तद्वा अ॒ग्नेः प्रि॒यं धाम॑ प्रि॒यमे॒वास्य॒ धामोपा᳚प्नोति॒ तावेहि॒ सꣳ र॑भावहा॒ इत्या॑ह॒ व्ये॑वैने॑न॒ परि॑ धत्ते॒ पाञ्च॑जन्ये॒ष्वप्ये᳚ध्यग्न॒ इत्या॑है॒ष वा अ॒ग्निः पाञ्च॑जन्यो॒ यः पञ्च॑चितीक॒स्तस्मा॑दे॒वमा॑हर्त॒व्या॑ उप॑ दधात्ये॒तद्वा ऋ॑तू॒नां प्रि॒यं धाम॒ यदृ॑त॒व्या॑ ऋतू॒नामे॒व प्रि॒यं धामाव॑ रुन्धे सु॒मेक॒ इत्या॑ह संवथ्स॒रो वै सु॒मेकः॑ संवथ्स॒रस्यै॒व प्रि॒यं धामोपा᳚प्नोति॥~(४६)

{\anuvakamend[{अभ॑व॒न्नायु॑रे॒वर्त॒व्या॑ उप॒ षड्विꣳ॑शतिश्च}]}%॥11॥

%5.3.12.1
प्र॒जा\-प॑ते॒रक्ष्य॑श्वय॒त्तत्परा॑पत॒त्तदश्वो॑\-ऽभव॒द्यदश्व॑य॒त्तदश्व॑स्याश्व॒त्वन्तद्दे॒वा अ॑श्वमे॒धेनै॒व प्रत्य॑दधुरे॒ष वै प्र॒जा\-प॑ति॒ꣳ॒ सर्वं॑ करोति॒ यो᳚\-ऽश्वमे॒धेन॒ यज॑ते॒ सर्व॑ ए॒व भ॑वति॒ सर्व॑स्य॒ वा ए॒षा प्राय॑श्चित्तिः॒ सर्व॑स्य भेष॒जꣳ सर्वं॒ वा ए॒तेन॑ पा॒प्मानं॑ दे॒वा अ॑तर॒न्नपि॒ वा ए॒तेन॑ ब्रह्मह॒त्याम॑तर॒न्थ्सर्वं॑ पा॒प्मानम्᳚~(४७)

%5.3.12.2
त॒र॒ति॒ तर॑ति ब्रह्मह॒त्यां यो᳚\-ऽश्वमे॒धेन॒ यज॑ते॒ य उ॑ चैनमे॒वं वेदोत्त॑रं॒ वै तत्प्र॒जा\-प॑ते॒रक्ष्य॑श्वय॒त्तस्मा॒दश्व॑स्योत्तर॒तो\-ऽव॑ द्यन्ति दक्षिण॒तो᳚\-ऽन्येषां᳚ पशू॒नाम्वै॑त॒सः कटो॑ भवत्य॒फ्सुयो॑नि॒र्वा अश्वो᳚\-ऽफ्सु॒जो वे॑त॒सः स्व ए॒वैनं॒ योनौ॒ प्रति॑\-ष्ठापयति चतुष्टो॒मः स्तोमो॑ भवति स॒रड्ढ॒ वा अश्व॑स्य॒ सक्थ्यावृ॑ह॒त्तद्दे॒वाश्च॑तुष्टो॒मेनै॒व प्रत्य॑दधु॒र्यच्च॑तुष्टो॒मः स्तोमो॒ भव॒त्यश्व॑स्य सर्व॒त्वाय॑॥~(४८)

{\anuvakamend[{सर्वं॑ पा॒प्मान॑मवृह॒द्द्वाद॑श च}]}%॥12॥

{\prashnaend[{उ॒थ्स॒न्न॒य॒ज्ञ इन्द्रा᳚ग्नी दे॒वा वा अ॑क्षणयास्तो॒मीया॑ अ॒ग्नेर्भा॒गो᳚\-ऽस्यग्ने॑ जा॒तान्र॒श्मिरिति॑ नाक॒सद्भि॒श्छन्दाꣳ॑सि॒ सर्वा᳚भ्यो वृष्टि॒सनी᳚र्देवासु॒राः कनी॑याꣳसः प्र॒जा\-प॑ते॒रक्षि॒ द्वाद॑श॥१२॥ उ॒थ्स॒न्न॒य॒ज्ञो दे॒वा वै यस्य॒ मुख्य॑वतीर्नाक॒सद्भि॑रे॒वैताभि॑र॒ष्टाच॑त्वारिꣳशत्॥४८॥ उ॒थ्स॒न्न॒य॒ज्ञः स॑र्व॒त्वाय॑॥}]}%%५-३

\centerline{॥हरिः॑ ॐ॥}

\centerline{॥कृष्ण-यजुर्वेदीय-तैत्तिरीय-संहितायां पञ्चम्काण्डे तृतीयः प्रश्नः समाप्तः॥५-३॥}
%%% END PRASHNA
