\chapt{काण्डम् ५}
\sect{द्वितीयः प्रश्नः}\setcounter{anuvakam}{0}
\dnsub{तैत्तिरीयसंहितायां पञ्चमकाण्डे द्वितीयः प्रश्नः}
%5.2.1.0
%5.2.1.1
विष्णु॑मुखा॒ वै दे॒वाश्छन्दो॑भिरि॒माँल्लो॒कान॑नपज॒य्यम॒भ्य॑जय॒न्॒ यद्वि॑ष्णुक्र॒मान्क्रम॑ते॒ विष्णु॑रे॒व भू॒त्वा यज॑मान॒श्छन्दो॑भिरि॒माँल्लो॒कान॑नपज॒य्यम॒भि ज॑यति॒ विष्णोः॒ क्रमो᳚\-ऽस्यभिमाति॒हेत्या॑ह गाय॒त्री वै पृ॑थि॒वी त्रैष्ठु॑भम॒न्तरि॑क्षं॒ जाग॑ती॒ द्यौरानु॑ष्टुभी॒र्दिश॒श्छन्दो॑भिरे॒वेमाँल्लो॒कान् य॑थापू॒र्वम॒भि ज॑यति प्र॒जा\-प॑तिर॒ग्निम॑सृजत॒ सो᳚\-ऽस्माथ्सृ॒ष्टः~(१)

%5.2.1.2
परा॑ङै॒त्तमे॒तयान्वै॒दक्र॑न्द॒दिति॒ तया॒ वै सो᳚\-ऽग्नेः प्रि॒यं धामावा॑रुन्ध॒ यदे॒ताम॒न्वाहा॒ग्नेरे॒वैतया᳚ प्रि॒यं धामाव॑ रुन्ध ईश्व॒रो वा ए॒ष परा᳚ङ्प्र॒दघो॒ यो वि॑ष्णुक्र॒मान्क्रम॑ते चत॒सृभि॒रा व॑र्तते च॒त्वारि॒ छन्दाꣳ॑सि॒ छन्दाꣳ॑सि॒ खलु॒ वा अ॒ग्नेः प्रि॒या त॒नूः प्रि॒यामे॒वास्य॑ त॒नुव॑म॒भि~(२)

%5.2.1.3
प॒र्याव॑र्तते दक्षि॒णा प॒र्याव॑र्तते॒ स्वमे॒व वी॒र्य॑मनु॑ प॒र्याव॑र्तते॒ तस्मा॒द्दक्षि॒णो\-ऽर्ध॑ आ॒त्मनो॑ वी॒र्या॑वत्त॒रो\-ऽथो॑ आदि॒त्यस्यै॒वावृत॒मनु॑ प॒र्याव॑र्तते॒ शुनः॒शेप॒माजी॑गर्तिं॒ वरु॑णो\-ऽगृह्णा॒थ्स ए॒तां वा॑रु॒णीम॑पश्य॒त्तया॒ वै स आ॒त्मानं॑ वरुणपा॒शाद॑मुञ्च॒द्वरु॑णो॒ वा ए॒तं गृ॑ह्णाति॒ य उ॒खां प्र॑तिमु॒ञ्चत॒ उदु॑त्त॒मं व॑रुण॒ पाश॑म॒स्मदित्या॑हा॒ऽ॒ऽ॒त्मान॑मे॒वैतया᳚~(३)

%5.2.1.4
व॒रु॒ण॒पा॒शान्मु॑ञ्च॒त्या त्वा॑हार्\mbox{}ष॒मित्या॒हा ह्ये॑न॒ꣳ॒ हर॑ति ध्रु॒वस्ति॒ष्ठावि॑चाचलि॒रित्या॑ह॒ प्रति॑ष्ठित्यै॒ विश॑स्त्वा॒ सर्वा॑ वाञ्छ॒न्त्वित्या॑ह वि॒शैवैन॒ꣳ॒ सम॑र्धयत्य॒स्मिन्रा॒ष्ट्रमधि॑ श्र॒येत्या॑ह रा॒ष्ट्रमे॒वास्मि॑न्ध्रु॒वम॑क॒र्यं का॒मये॑त रा॒ष्ट्रꣴ स्या॒दिति॒ तं मन॑सा ध्यायेद्रा॒ष्ट्रमे॒व भ॑वति~(४)

%5.2.1.5
अग्रे॑ बृ॒हन्नु॒षसा॑मू॒र्ध्वो अ॑स्था॒दित्या॒हाग्र॑मे॒वैनꣳ॑ समा॒नानां᳚ करोति निर्जग्मि॒वान्तम॑स॒ इत्या॑ह॒ तम॑ ए॒वास्मा॒दप॑ हन्ति॒ ज्योति॒षागा॒दित्या॑ह॒ ज्योति॑रे॒वास्मि॑न्दधाति चत॒सृभिः॑ सादयति च॒त्वारि॒ छन्दाꣳ॑सि॒ छन्दो॑भिरे॒वाति॑छन्दसोत्त॒मया॒ वर्ष्म॒ वा ए॒षा छन्द॑सां॒ यदति॑च्छन्दा॒ वर्ष्मै॒वैनꣳ॑ समा॒नानां᳚ करोति॒ सद्व॑ती~(५)

%5.2.1.6
भ॒व॒ति॒ स॒त्त्वमे॒वैनं॑ गमयति वाथ्स॒प्रेणोप॑ तिष्ठत ए॒तेन॒ वै व॑थ्स॒प्रीर्भा॑लन्द॒नो᳚\-ऽग्नेः प्रि॒यं धामावा॑रुन्धा॒ग्नेरे॒वैतेन॑ प्रि॒यं धामाव॑ रुन्ध एकाद॒शं भ॑वत्येक॒धैव यज॑माने वी॒र्यं॑ दधाति॒ स्तोमे॑न॒ वै दे॒वा अ॒स्मिँल्लो॒क आ᳚र्ध्नुव॒ञ्छन्दो॑भिर॒मुष्मि॒न्थ्स्तोम॑स्येव॒ खलु॒ वा ए॒तद्रू॒पं यद्वा᳚थ्स॒प्रम्यद्वा᳚थ्स॒प्रेणो॑प॒तिष्ठ॑ते~(६)

%5.2.1.7
इ॒ममे॒व तेन॑ लो॒कम॒भि ज॑य॒ति यद्वि॑ष्णुक्र॒मान्क्रम॑ते॒\-ऽमुमे॒व तैर्लो॒कम॒भि ज॑यति पूर्वे॒द्युः प्र क्रा॑मत्युत्तरे॒द्युरुप॑ तिष्ठते॒ तस्मा॒द्योगे॒\-ऽन्यासां᳚ प्र॒जानां॒ मनः॒ क्षेमे॒\-ऽन्यासा॒न्तस्मा᳚द्यायाव॒रः क्षे॒म्यस्ये॑शे॒ तस्मा᳚द्यायाव॒रः क्षे॒म्यम॒ध्यव॑स्यति मु॒ष्टी क॑रोति॒ वाचं॑ यच्छति य॒ज्ञस्य॒ धृत्यै᳚॥~(७)

%5.2.2.0
{\anuvakamend[{सृ॒ष्टो\-ऽभ्ये॑तया॑ भवति॒ सद्व॑त्युप॒तिष्ठ॑ते॒ द्विच॑त्वारिꣳशच्च}]}%~(१)

%5.2.2.1
अन्न॑प॒ते\-ऽन्न॑स्य नो दे॒हीत्या॑हा॒ग्निर्वा अन्न॑पतिः॒ स ए॒वास्मा॒ अन्नं॒ प्र य॑च्छत्यनमी॒वस्य॑ शु॒ष्मिण॒ इत्या॑हाय॒क्ष्मस्येति॒ वावैतदा॑ह॒ प्र प्र॑दा॒तारं॑ तारिष॒ ऊर्जं॑ नो धेहि द्वि॒पदे॒ चतु॑ष्पद॒ इत्या॑हा॒ऽऽशिष॑मे॒वैतामा शा᳚स्त॒ उदु॑ त्वा॒ विश्वे॑ दे॒वा इत्या॑ह प्रा॒णा वै विश्वे॑ दे॒वाः~(८)

%5.2.2.2
प्रा॒णैरे॒वैन॒मुद्य॑च्छ॒ते\-ऽग्ने॒ भर॑न्तु॒ चित्ति॑भि॒रित्या॑ह॒ यस्मा॑ ए॒वैनं॑ चि॒त्तायो॒द्यच्छ॑ते॒ तेनै॒वैन॒ꣳ॒ सम॑र्धयति चत॒सृभि॒रा सा॑दयति च॒त्वारि॒ छन्दाꣳ॑सि॒ छन्दो॑भिरे॒वाति॑च्छन्दसोत्त॒मया॒ वर्ष्म॒ वा ए॒षा छन्द॑सां॒ यदति॑च्छन्दा॒ वर्ष्मै॒वैनꣳ॑ समा॒नानां᳚ करोति॒ सद्व॑ती भवति स॒त्त्वमे॒वैनं॑ गमयति॒ प्रेद॑ग्ने॒ ज्योति॑ष्मान्~(९)

%5.2.2.3
या॒हीत्या॑ह॒ ज्योति॑रे॒वास्मि॑न्दधाति त॒नुवा॒ वा ए॒ष हि॑नस्ति॒ यꣳ हि॒नस्ति॒ मा हिꣳ॑सीस्त॒नुवा᳚ प्र॒जा इत्या॑ह प्र॒जाभ्य॑ ए॒वैनꣳ॑ शमयति॒ रक्षाꣳ॑सि॒ वा ए॒तद्य॒ज्ञꣳ स॑चन्ते॒ यदन॑ उ॒थ्सर्ज॒त्यक्र॑न्द॒दित्यन्वा॑ह॒ रक्ष॑सा॒मप॑हत्या॒ अन॑सा वह॒न्त्यप॑चितिमे॒वास्मि॑न्दधाति॒ तस्मा॑दन॒स्वी च॑ र॒थी चाति॑थीना॒मप॑चिततमौ~(१०)

%5.2.2.4
अप॑चितिमान्भवति॒ य ए॒वं वेद॑ स॒मिधा॒\-ऽग्निं दु॑वस्य॒तेति॑ घृतानुषि॒क्तामव॑सिते स॒मिध॒मा द॑धाति॒ यथाति॑थय॒ आग॑ताय स॒र्पिष्व॑दाति॒थ्यं क्रि॒यते॑ ता॒दृगे॒व तद्गा॑यत्रि॒या ब्रा᳚ह्म॒णस्य॑ गाय॒त्रो हि ब्रा᳚ह्म॒णस्त्रि॒ष्टुभा॑ राज॒न्य॑स्य॒ त्रैष्टु॑भो॒ हि रा॑ज॒न्यो᳚\-ऽफ्सु भस्म॒ प्र वे॑शयत्य॒फ्सुयो॑नि॒र्वा अ॒ग्निः स्वामे॒वैनं॒ योनिं॑ गमयति ति॒सृभिः॒ प्र वे॑शयति त्रि॒वृद्वै~(११)

%5.2.2.5
अ॒ग्निर्यावा॑ने॒वाग्निस्तं प्र॑ति॒ष्ठां ग॑मयति॒ परा॒ वा ए॒षो᳚\-ऽग्निं व॑पति॒ यो᳚\-ऽफ्सु भस्म॑ प्रवे॒शय॑ति॒ ज्योति॑ष्मतीभ्या॒मव॑ दधाति॒ ज्योति॑रे॒वास्मि॑न्दधाति॒ द्वाभ्यां॒ प्रति॑ष्ठित्यै॒ परा॒ वा ए॒ष प्र॒जां प॒शून् व॑पति॒ यो᳚\-ऽफ्सु भस्म॑ प्रवे॒शय॑ति॒ पुन॑रू॒र्जा स॒ह र॒य्येति॒ पुन॑रु॒दैति॑ प्र॒जामे॒व प॒शूना॒त्मन्ध॑त्ते॒ पुन॑स्त्वादि॒त्याः~(१२)

%5.2.2.6
रु॒द्रा वस॑वः॒ समि॑न्धता॒मित्या॑है॒ता वा ए॒तं दे॒वता॒ अग्रे॒ समै᳚न्धत॒ ताभि॑रे॒वैन॒ꣳ॒ समि॑न्द्धे॒ बोधा॒ स बो॒धीत्युप॑ तिष्ठते बो॒धय॑त्ये॒वैन॒न्तस्मा᳚थ्सु॒प्त्वा प्र॒जाः प्र बु॑ध्यन्ते यथास्था॒नमुप॑ तिष्ठते॒ तस्मा᳚द्यथास्था॒नं प॒शवः॒ पुन॒रेत्योप॑ तिष्ठन्ते॥~(१३)

%5.2.3.0
{\anuvakamend[{वै विश्वे॑ दे॒वा ज्योति॑ष्मा॒नप॑चिततमौ त्रि॒वृद्वा आ॑दि॒त्या द्विच॑त्वारिꣳशच्च}]}%~(२)

%5.2.3.1
याव॑ती॒ वै पृ॑थि॒वी तस्यै॑ य॒म आधि॑पत्यं॒ परी॑याय॒ यो वै य॒मं दे॑व॒यज॑नम॒स्या अनि॑र्याच्या॒ग्निं चि॑नु॒ते य॒मायै॑न॒ꣳ॒ स चि॑नु॒ते\-ऽपे॒तेत्य॒ध्यव॑साययति य॒ममे॒व दे॑व॒यज॑नम॒स्यै नि॒र्याच्या॒ऽऽत्मने॒\-ऽग्निं चि॑नुत इष्व॒ग्रेण॒ वा अ॒स्या अना॑मृतमि॒च्छन्तो॒ नावि॑न्द॒न्ते दे॒वा ए॒तद्यजु॑रपश्य॒न्नपे॒तेति॒ यदे॒तेना᳚ध्यवसा॒यय॑ति~(१४)

%5.2.3.2
अना॑मृत ए॒वाग्निं चि॑नुत॒ उद्ध॑न्ति॒ यदे॒वास्या॑ अमे॒ध्यं तदप॑ हन्त्य॒पो\-ऽवो᳚क्षति॒ शान्त्यै॒ सिक॑ता॒ नि व॑पत्ये॒तद्वा अ॒ग्नेर्वै᳚श्वान॒रस्य॑ रू॒पꣳ रू॒पेणै॒व वै᳚श्वान॒रमव॑ रुन्ध॒ ऊषा॒न्नि व॑पति॒ पुष्टि॒र्वा ए॒षा प्र॒जन॑नं॒ यदूषाः॒ पुष्ट्या॑मे॒व प्र॒जन॑ने॒\-ऽग्निं चि॑नु॒ते\-ऽथो॑ सं॒ज्ञान॑ ए॒व सं॒ज्ञान॒ꣴ॒ ह्ये॑तत्~(१५)

%5.2.3.3
प॒शू॒नां यदूषा॒ द्यावा॑पृथि॒वी स॒हास्ता॒न्ते वि॑य॒ती अ॑ब्रूता॒मस्त्वे॒व नौ॑ स॒ह य॒ज्ञिय॒मिति॒ यद॒मुष्या॑ य॒ज्ञिय॒मासी॒त्तद॒स्याम॑दधा॒त्त ऊषा॑ अभव॒न्॒ यद॒स्या य॒ज्ञिय॒मासी॒त्तद॒मुष्या॑मदधा॒त्तद॒दश्च॒न्द्रम॑सि कृ॒ष्णमूषा᳚न्नि॒वप॑न्न॒दो ध्या॑ये॒द्द्यावा॑पृथि॒व्योरे॒व य॒ज्ञिये॒\-ऽग्निं चि॑नुते॒\-ऽयꣳ सो अ॒ग्निरिति॑ वि॒श्वामि॑त्रस्य~(१६)

%5.2.3.4
सू॒क्तं भ॑वत्ये॒तेन॒ वै वि॒श्वामि॑त्रो॒\-ऽग्नेः प्रि॒यं धामावा॑रुन्धा॒ग्नेरे॒वैतेन॑ प्रि॒यं धामाव॑ रुन्धे॒ छन्दो॑भि॒र्वै दे॒वाः सु॑व॒र्गं लो॒कमा॑य॒ञ्चत॑स्रः॒ प्राची॒रुप॑ दधाति च॒त्वारि॒ छन्दाꣳ॑सि॒ छन्दो॑भिरे॒व तद्यज॑मानः सुव॒र्गं लो॒कमे॑ति॒ तेषाꣳ॑ सुव॒र्गं लो॒कं य॒तां दिशः॒ सम॑व्लीयन्त॒ ते द्वे पु॒रस्ता᳚थ्स॒मीची॒ उपा॑दधत॒ द्वे~(१७)

%5.2.3.5
प॒श्चाथ्स॒मीची॒ ताभि॒र्वै ते दिशो॑\-ऽदृꣳह॒न्॒ यद्द्वे पु॒रस्ता᳚थ्स॒मीची॑ उप॒दधा॑ति॒ द्वे प॒श्चाथ्स॒मीची॑ दि॒शां विधृ॑त्या॒ अथो॑ प॒शवो॒ वै छन्दाꣳ॑सि पशूने॒वास्मै॑ स॒मीचो॑ दधात्य॒ष्टावुप॑ दधात्य॒ष्टाक्ष॑रा गाय॒त्री गा॑य॒त्रो᳚\-ऽग्निर्यावा॑ने॒वाग्निस्तं चि॑नुते॒\-ऽष्टावुप॑ दधात्य॒ष्टाक्ष॑रा गाय॒त्री गा॑य॒त्री सु॑व॒र्गं लो॒कमञ्ज॑सा वेद सुव॒र्गस्य॑ लो॒कस्य॑~(१८)

%5.2.3.6
प्रज्ञा᳚त्यै॒ त्रयो॑दश लोकं पृ॒णा उप॑ दधा॒त्येक॑विꣳशतिः॒ सम्प॑द्यन्ते प्रति॒ष्ठा वा ए॑कवि॒ꣳ॒शः प्र॑ति॒ष्ठा गार्\mbox{}ह॑पत्य एकवि॒ꣳ॒शस्यै॒व प्र॑ति॒ष्ठां गार्\mbox{}ह॑पत्य॒मनु॒ प्रति॑ तिष्ठति॒ प्रत्य॒ग्निं चि॑क्या॒नस्ति॑ष्ठति॒ य ए॒वं वेद॒ पञ्च॑चितीकं चिन्वीत प्रथ॒मं चि॑न्वा॒नः पाङ्क्तो॑ य॒ज्ञः पाङ्क्ताः᳚ प॒शवो॑ य॒ज्ञमे॒व प॒शूनव॑ रुन्धे॒ त्रिचि॑तीकं चिन्वीत द्वि॒तीयं॑ चिन्वा॒नस्त्रय॑ इ॒मे लो॒का ए॒ष्वे॑व लो॒केषु॑~(१९)

%5.2.3.7
प्रति॑ तिष्ठ॒त्येक॑चितीकं चिन्वीत तृ॒तीयं॑ चिन्वा॒न ए॑क॒धा वै सु॑व॒र्गो लो॒क ए॑क॒वृतै॒व सु॑व॒र्गं लो॒कमे॑ति॒ पुरी॑षेणा॒भ्यू॑हति॒ तस्मा᳚न्मा॒ꣳ॒सेनास्थि॑ छ॒न्नन्न दु॒श्चर्मा॑ भवति॒ य ए॒वं वेद॒ पञ्च॒ चित॑यो भवन्ति प॒ञ्चभिः॒ पुरी॑षैर॒भ्यू॑हति॒ दश॒ सम्प॑द्यन्ते॒ दशा᳚क्षरा वि॒राडन्नं॑ वि॒राड्वि॒राज्ये॒वान्नाद्ये॒ प्रति॑ तिष्ठति॥~(२०)

%5.2.4.0
{\anuvakamend[{अ॒द्ध्य॒व॒सा॒यय॑ति॒ ह्ये॑तद्वि॒श्वामि॑त्रस्यादधत॒ द्वे लो॒कस्य॑ लो॒केषु॑ स॒प्तच॑त्वारिꣳशच्च}]}%~(३)

%5.2.4.1
वि वा ए॒तौ द्वि॑षाते॒ यश्च॑ पु॒राग्निर्यश्चो॒खाया॒ꣳ॒ समि॑त॒मिति॑ चत॒सृभिः॒ सं नि व॑पति च॒त्वारि॒ छन्दाꣳ॑सि॒ छन्दाꣳ॑सि॒ खलु॒ वा अ॒ग्नेः प्रि॒या त॒नूः प्रि॒ययै॒वैनौ॑ त॒नुवा॒ सꣳ शा᳚स्ति॒ समि॑त॒मित्या॑ह॒ तस्मा॒द्ब्रह्म॑णा क्ष॒त्रꣳ समे॑ति॒ यथ्सं॒न्युप्य॑ वि॒हर॑ति॒ तस्मा॒द्ब्रह्म॑णा क्ष॒त्रं व्ये᳚त्यृ॒तुभिः॑~(२१)

%5.2.4.2
वा ए॒तं दी᳚क्षयन्ति॒ स ऋ॒तुभि॑रे॒व वि॒मुच्यो॑ मा॒तेव॑ पु॒त्रं पृ॑थि॒वी पु॑री॒ष्य॑मित्या॑ह॒र्तुभि॑रे॒वैनं॑ दीक्षयि॒त्वर्तुभि॒र्वि मु॑ञ्चति वैश्वान॒र्या शि॒क्य॑मा द॑त्ते स्व॒दय॑त्ये॒वैन॑न्नैर्\mbox{}ऋ॒तीः कृ॒ष्णास्ति॒स्रस्तुष॑पक्वा भवन्ति॒ निर्\mbox{}ऋ॑त्यै॒ वा ए॒तद्भा॑ग॒धेयं॒ यत्तुषा॒ निर्\mbox{}ऋ॑त्यै रू॒पं कृ॒ष्णꣳ रू॒पेणै॒व निर्\mbox{}ऋ॑तिं नि॒रव॑दयत इ॒मां दिशं॑ यन्त्ये॒षा~(२२)

%5.2.4.3
वै निर्\mbox{}ऋ॑त्यै॒ दिख्स्वाया॑मे॒व दि॒शि निर्\mbox{}ऋ॑तिं नि॒रव॑दयते॒ स्वकृ॑त॒ इरि॑ण॒ उप॑ दधाति प्रद॒रे वै॒तद्वै निर्\mbox{}ऋ॑त्या आ॒यत॑न॒ꣴ॒ स्व ए॒वायत॑ने॒ निर्\mbox{}ऋ॑तिं नि॒रव॑दयते शि॒क्य॑म॒भ्युप॑ दधाति नैर्\mbox{}ऋ॒तो वै पाशः॑ सा॒क्षादे॒वैनं॑ निर्\mbox{}ऋतिपा॒शान्मु॑ञ्चति ति॒स्र उप॑ दधाति त्रेधाविहि॒तो वै पुरु॑षो॒ यावा॑ने॒व पुरु॑ष॒स्तस्मा॒न्निर्\mbox{}ऋ॑ति॒मव॑ यजते॒ परा॑ची॒रुप॑~(२३)

%5.2.4.4
द॒धा॒ति॒ परा॑चीमे॒वास्मा॒न्निर्\mbox{}ऋ॑तिं॒ प्र णु॑द॒ते\-ऽप्र॑तीक्ष॒मा य॑न्ति॒ निर्\mbox{}ऋ॑त्या अ॒न्तर्\mbox{}हि॑त्यै मार्जयि॒त्वोप॑ तिष्ठन्ते मेध्य॒त्वाय॒ गार्\mbox{}ह॑पत्य॒मुप॑ तिष्ठन्ते निर्\mbox{}ऋतिलो॒क ए॒व च॑रि॒त्वा पू॒ता दे॑वलो॒कमु॒पाव॑र्तन्त॒ एक॒योप॑ तिष्ठन्त एक॒धैव यज॑माने वी॒र्यं॑ दधति नि॒वेश॑नः सं॒गम॑नो॒ वसू॑ना॒मित्या॑ह प्र॒जा वै प॒शवो॒ वसु॑ प्र॒जयै॒वैनं॑ प॒शुभिः॒ सम॑र्धयन्ति॥~(२४)

%5.2.5.0
{\anuvakamend[{ऋ॒तुभि॑रे॒षा परा॑ची॒रुपा॒ष्टाच॑त्वारिꣳशच्च}]}%~(४)

%5.2.5.1
पु॒रु॒ष॒मा॒त्रेण॒ वि मि॑मीते य॒ज्ञेन॒ वै पुरु॑षः॒ सम्मि॑तो यज्ञप॒रुषै॒वैनं॒ वि मि॑मीते॒ यावा॒न्पुरु॑ष ऊ॒र्ध्वबा॑हु॒स्तावा᳚न्भव\-त्ये॒ताव॒द्वै पुरु॑षे वी॒र्यं॑ वी॒र्ये॑णै॒वैनं॒ वि मि॑मीते प॒क्षी भ॑वति॒ न ह्य॑प॒क्षः पति॑तु॒मर्\mbox{}ह॑त्यर॒त्निना॑ प॒क्षौ द्राघी॑याꣳसौ भवत॒स्तस्मा᳚त्प॒क्षप्र॑वयाꣳसि॒ वयाꣳ॑सि व्याममा॒त्रौ प॒क्षौ च॒ पुच्छं॑ च भवत्ये॒ताव॒द्वै पुरु॑षे वी॒र्यम्᳚~(२५)

%5.2.5.2
वी॒र्य॑सम्मितो॒ वेणु॑ना॒ वि मि॑मीत आग्ने॒यो वै वेणुः॑ सयोनि॒त्वाय॒ यजु॑षा युनक्ति॒ यजु॑षा कृषति॒ व्यावृ॑त्त्यै षड्ग॒वेन॑ कृषति॒ षड्वा ऋ॒तव॑ ऋ॒तुभि॑रे॒वैनं॑ कृषति॒ यद्द्वा॑दशग॒वेन॑ संवथ्स॒रेणै॒वेयं वा अ॒ग्नेर॑तिदा॒हाद॑बिभे॒थ्सैतद्द्वि॑गु॒णम॑पश्यत्कृ॒ष्टं चाकृ॑ष्टं च॒ ततो॒ वा इ॒मां नात्य॑दह॒द्यत्कृ॒ष्टं चाकृ॑ष्टं च~(२६)

%5.2.5.3
भव॑त्य॒स्या अन॑तिदाहाय द्विगु॒णं त्वा अ॒ग्निमुद्य॑न्तुमर्\mbox{}ह॒तीत्या॑हु॒र्यत्कृ॒ष्टं चाकृ॑ष्टं च॒ भव॑त्य॒ग्नेरुद्य॑त्या ए॒ताव॑न्तो॒ वै प॒शवो᳚ द्वि॒पाद॑श्च॒ चतु॑ष्पादश्च॒ तान् यत्प्राच॑ उथ्सृ॒जेद्रु॒द्रायापि॑ दध्या॒द्यद्द॑क्षि॒णा पि॒तृभ्यो॒ नि धु॑वे॒द्यत्प्र॒तीचो॒ रक्षाꣳ॑सि हन्यु॒रुदी॑च॒ उथ्सृ॑जत्ये॒षा वै दे॑वमनु॒ष्याणाꣳ॑ शा॒न्ता दिक्~(२७)

%5.2.5.4
तामे॒वैना॒ननूथ्सृ॑ज॒त्यथो॒ खल्वि॒मां दिश॒मुथ्सृ॑जत्य॒सौ वा आ॑दि॒त्यः प्रा॒णः प्रा॒णमे॒वैना॒ननूथ्सृ॑जति दक्षि॒णा प॒र्याव॑र्तन्ते॒ स्वमे॒व वी॒र्य॑मनु॑ प॒र्याव॑र्तन्ते॒ तस्मा॒द्दक्षि॒णो\-ऽर्ध॑ आ॒त्मनो॑ वी॒र्या॑वत्त॒रो\-ऽथो॑ आदि॒त्यस्यै॒वावृत॒मनु॑ प॒र्याव॑र्तन्ते॒ तस्मा॒त्परा᳚ञ्चः प॒शवो॒ वि ति॑ष्ठन्ते प्र॒त्यं च॒ आ व॑र्तन्ते ति॒स्रस्ति॑स्रः॒ सीताः᳚~(२८)

%5.2.5.5
कृ॒ष॒ति॒ त्रि॒वृत॑मे॒व य॑ज्ञमु॒खे वि या॑तय॒त्योष॑धीर्वपति॒ ब्रह्म॒णान्न॒मव॑ रुन्धे॒\-ऽर्के᳚\-ऽर्कश्ची॑यते चतुर्द॒शभि॑र्वपति स॒प्त ग्रा॒म्या ओष॑धयः स॒प्तार॒ण्या उ॒भयी॑षा॒मव॑रुद्ध्या॒ अन्न॑स्यान्नस्य वप॒त्यन्न॑स्यान्न॒स्याव॑रुद्ध्यै कृ॒ष्टे व॑पति कृ॒ष्टे ह्योष॑धयः प्रति॒तिष्ठ॑न्त्यनुसी॒तं व॑पति॒ प्रजा᳚त्यै द्वाद॒शसु॒ सीता॑सु वपति॒ द्वाद॑श॒ मासाः᳚ संवथ्स॒रः सं॑वथ्स॒रेणै॒वास्मा॒ अन्नं॑ पचति॒ यद॑ग्नि॒चित्~(२९)

%5.2.5.6
अन॑वरुद्धस्याश्ञी॒यादव॑रुद्धेन॒ व्यृ॑द्ध्येत॒ ये वन॒स्पती॑नाम्फल॒ग्रह॑य॒स्तानि॒ध्मे\-ऽपि॒ प्रोक्षे॒दन॑वरुद्ध॒स्याव॑रुद्ध्यै दि॒ग्भ्यो लो॒ष्टान्थ्सम॑स्यति दि॒शामे॒व वी॒र्य॑मव॒रुध्य॑ दि॒शां वी॒र्ये᳚\-ऽग्निं चि॑नुते॒ यं द्वि॒ष्याद्यत्र॒ स स्यात्तस्यै॑ दि॒शो लो॒ष्टमा ह॑रे॒दिष॒मूर्ज॑म॒हमि॒त आ द॑द॒ इतीष॑मे॒वोर्जं॒ तस्यै॑ दि॒शो\-ऽव॑ रुन्धे॒ क्षोधु॑को भवति॒ यस्तस्यां᳚ दि॒शि भव॑त्युत्तरवे॒दिमुप॑ वपत्युत्तरवे॒द्याꣳ ह्य॑ग्निश्ची॒यते\-ऽथो॑ प॒शवो॒ वा उ॑त्तरवे॒दिः प॒शूने॒वाव॑ रु॒न्धे\-ऽथो॑ यज्ञप॒रुषो\-ऽन॑न्तरित्यै॥~(३०)

%5.2.6.0
{\anuvakamend[{च॒ भ॒व॒त्ये॒ताव॒द्वै पुरु॑षे वी॒र्यं॑ यत्कृ॒ष्टञ्चाकृ॑ष्टं च॒ दिख्सीता॑ अग्नि॒चिदव॒ पञ्च॑विꣳशतिश्च}]}%~(५)

%5.2.6.1
अग्ने॒ तव॒ श्रवो॒ वय॒ इति॒ सिक॑ता॒ नि व॑पत्ये॒तद्वा अ॒ग्नेर्वै᳚श्वान॒रस्य॑ सू॒क्तꣳ सू॒क्तेनै॒व वै᳚श्वान॒रमव॑ रुन्धे ष॒ड्भिर्नि व॑पति॒ षड्वा ऋ॒तवः॑ संवथ्स॒रः सं॑वथ्स॒रो᳚\-ऽग्निर्वै᳚श्वान॒रः सा॒क्षादे॒व वै᳚श्वान॒रमव॑ रुन्धे समु॒द्रं वै नामै॒तच्छन्दः॑ समु॒द्रमनु॑ प्र॒जाः प्र जा॑यन्ते॒ यदे॒तेन॒ सिक॑ता नि॒वप॑ति प्र॒जानां᳚ प्र॒जन॑ना॒येन्द्रः॑~(३१)

%5.2.6.2
वृ॒त्राय॒ वज्रं॒ प्राह॑र॒थ्स त्रे॒धा व्य॑भव॒थ्स्फ्यस्तृती॑य॒ꣳ॒ रथ॒स्तृती॑यं॒ यूप॒स्तृती॑यं॒ ये᳚\-ऽन्तःश॒रा अशी᳚र्यन्त॒ ताः शर्क॑रा अभव॒न्तच्छर्क॑राणाꣳ शर्कर॒त्वं वज्रो॒ वै शर्क॑राः प॒शुर॒ग्निर्यच्छर्क॑राभिर॒ग्निं प॑रिमि॒नोति॒ वज्रे॑णै॒वास्मै॑ प॒शून्परि॑ गृह्णाति॒ तस्मा॒द्वज्रे॑ण प॒शवः॒ परि॑गृहीता॒स्तस्मा॒थ्स्थेया॒नस्थे॑यसो॒ नोप॑ हरते त्रिस॒प्ताभिः॑~(३२)

%5.2.6.3
प॒शुका॑मस्य॒ परि॑ मिनुयाथ्स॒प्त वै शी॑र्\mbox{}ष॒ण्याः᳚ प्रा॒णाः प्रा॒णाः प॒शवः॑ प्रा॒णैरे॒वास्मै॑ प॒शूनव॑ रुन्धे त्रिण॒वाभि॒\-र्भ्रातृ॑व्यवतस्त्रि॒वृत॑मे॒व वज्रꣳ॑ स॒म्भृत्य॒ भ्रातृ॑व्याय॒ प्र ह॑रति॒ स्तृत्या॒ अप॑रिमिताभिः॒ परि॑ मिनुया॒दप॑रिमित॒स्याव॑रुद्ध्यै॒ यं का॒मये॑ताप॒शुः स्या॒दित्यप॑रिमित्य॒ तस्य॒ शर्क॑राः॒ सिक॑ता॒ व्यू॑हे॒दप॑रिगृहीत ए॒वास्य॑ विषू॒चीन॒ꣳ॒ रेतः॒ परा॒ सिञ्चत्यप॒शुरे॒व भ॑वति~(३३)

%5.2.6.4
यं का॒मये॑त पशु॒मान्थ्स्या॒दिति॑ परि॒मित्य॒ तस्य॒ शर्क॑राः॒ सिक॑ता॒ व्यू॑हे॒त्परि॑गृहीत ए॒वास्मै॑ समी॒चीन॒ꣳ॒ रेतः॑ सिञ्चति पशु॒माने॒व भ॑वति सौ॒म्या व्यू॑हति॒ सोमो॒ वै रे॑तो॒धा रेत॑ ए॒व तद्द॑धाति गायत्रि॒या ब्रा᳚ह्म॒णस्य॑ गाय॒त्रो हि ब्रा᳚ह्म॒णस्त्रि॒ष्टुभा॑ राज॒न्य॑स्य॒ त्रैष्टु॑भो॒ हि रा॑ज॒न्यः॑ शं॒युं बा॑र्\mbox{}हस्प॒त्यं मेधो॒ नोपा॑नम॒थ्सो᳚\-ऽग्निं प्रावि॑शत्~(३४)

%5.2.6.5
सो᳚\-ऽग्नेः कृष्णो॑ रू॒पं कृ॒त्वोदा॑यत॒ सो\-ऽश्वं॒ प्रावि॑श॒थ्सो\-ऽश्व॑स्यावान्तरश॒फो॑\-ऽभव॒द्यदश्व॑माक्र॒मय॑ति॒ य ए॒व मेधो\-ऽश्वं॒ प्रावि॑श॒त्तमे॒वाव॑ रुन्धे प्र॒जा\-प॑तिना॒ग्निश्चे॑त॒व्य॑ इत्या॑हुः प्राजाप॒त्यो\-ऽश्वो॒ यदश्व॑माक्र॒मय॑ति प्र॒जा\-प॑तिनै॒वाग्निं चि॑नुते पुष्करप॒र्णमुप॑ दधाति॒ योनि॒र्वा अ॒ग्नेः पु॑ष्करप॒र्णꣳ सयो॑निमे॒वाग्निं चि॑नुते॒\-ऽपां पृ॒ष्ठम॒सीत्युप॑ दधात्य॒पां वा ए॒तत्पृ॒ष्ठं यत्पु॑ष्करप॒र्णꣳ रू॒पेणै॒वैन॒दुप॑ दधाति॥~(३५)

%5.2.7.0
{\anuvakamend[{इन्द्रः॑ प॒शुका॑मस्य भवत्यविश॒थ्सयो॑निं विꣳश॒तिश्च॑}]}%~(६)

%5.2.7.1
ब्रह्म॑ जज्ञा॒नमिति॑ रु॒क्ममुप॑ दधाति॒ ब्रह्म॑मुखा॒ वै प्र॒जा\-प॑तिः प्र॒जा अ॑सृजत॒ ब्रह्म॑मुखा ए॒व तत्प्र॒जा यज॑मानः सृजते॒ ब्रह्म॑ जज्ञा॒नमित्या॑ह॒ तस्मा᳚द्ब्राह्म॒णो मुख्यो॒ मुख्यो॑ भवति॒ य ए॒वं वेद॑ ब्रह्मवा॒दिनो॑ वदन्ति॒ न पृ॑थि॒व्यां नान्तरि॑क्षे॒ न दि॒व्य॑ग्निश्चे॑त॒व्य॑ इति॒ यत्पृ॑थि॒व्यां चि॑न्वी॒त पृ॑थि॒वीꣳ शु॒चार्प॑ये॒न्नौष॑धयो॒ न वन॒स्पत॑यः~(३६)

%5.2.7.2
प्र जा॑येर॒न्॒ यद॒न्तरि॑क्षे चिन्वी॒तान्तरि॑क्षꣳ शु॒चार्प॑ये॒न्न वयाꣳ॑सि॒ प्र जा॑येर॒न्॒ यद्दि॒वि चि॑न्वी॒त दिवꣳ॑ शु॒चार्प॑ये॒न्न प॒र्जन्यो॑ वर्\mbox{}षेद्रु॒क्ममुप॑ दधात्य॒मृतं॒ वै हिर॑ण्यम॒मृत॑ ए॒वाग्निं चि॑नुते॒ प्रजा᳚त्यै हिर॒ण्मयं॒ पुरु॑ष॒मुप॑ दधाति यजमानलो॒कस्य॒ विधृ॑त्यै॒ यदिष्ट॑काया॒ आतृ॑ण्णमनूपद॒ध्यात्प॑शू॒नां च॒ यज॑मानस्य च प्रा॒णमपि॑ दध्याद्दक्षिण॒तः~(३७)

%5.2.7.3
प्राञ्च॒मुप॑ दधाति दा॒धार॑ यजमानलो॒कन्न प॑शू॒नां च॒ यज॑मानस्य च प्रा॒णमपि॑ दधा॒त्यथो॒ खल्विष्ट॑काया॒ आतृ॑ण्ण॒मनूप॑ दधाति प्रा॒णाना॒मुथ्सृ॑ष्ट्यै द्र॒फ्सश्च॑स्क॒न्देत्य॒भि मृ॑शति॒ होत्रा᳚स्वे॒वैनं॒ प्रति॑\-ष्ठापयति॒ स्रुचा॒वुप॑ दधा॒त्याज्य॑स्य पू॒र्णां का᳚र्ष्मर्य॒मयीं᳚ द॒ध्नः पू॒र्णामौदु॑म्बरीमि॒यं वै का᳚र्ष्मर्य॒मय्य॒सावौदु॑म्बरी॒मे ए॒वोप॑ धत्ते~(३८)

%5.2.7.4
तू॒ष्णीमुप॑ दधाति॒ न हीमे यजु॒षाप्तु॒मर्\mbox{}ह॑ति॒ दक्षि॑णां कार्ष्मर्य॒मयी॒मुत्त॑रा॒मौदु॑म्बरी॒न्तस्मा॑द॒स्या अ॒सावुत्त॒राज्य॑स्य पू॒र्णां का᳚र्ष्मर्य॒मयीं॒ वज्रो॒ वा आज्यं॒ वज्रः॑ कार्ष्म॒र्यो॑ वज्रे॑णै॒व य॒ज्ञस्य॑ दक्षिण॒तो रक्षा॒ꣴ॒स्यप॑ हन्ति द॒ध्नः पू॒र्णामौदु॑म्बरीं प॒शवो॒ वै दध्यूर्गु॑दु॒म्बरः॑ प॒शुष्वे॒वोर्जं॑ दधाति पू॒र्णे उप॑ दधाति पू॒र्णे ए॒वैनम्᳚~(३९)

%5.2.7.5
अ॒मुष्मिँ॑ल्लो॒क उप॑ तिष्ठेते वि॒राज्य॒ग्निश्चे॑त॒व्य॑ इत्या॒॑हुः स्रुग्वै वि॒राड्यथ्स्रुचा॑वुप॒दधा॑ति वि॒राज्ये॒वाग्निं चि॑नुते यज्ञमु॒खेय॑ज्ञमुखे॒ वै क्रि॒यमा॑णे य॒ज्ञꣳ रक्षाꣳ॑सि जिघाꣳसन्ति यज्ञमु॒खꣳ रु॒क्मो यद्रु॒क्मं व्या॑घा॒रय॑ति यज्ञमु॒खादे॒व रक्षा॒ꣴ॒स्यप॑ हन्ति प॒ञ्चभि॒र्व्याघा॑रयति॒ पाङ्क्तो॑ य॒ज्ञो यावा॑ने॒व य॒ज्ञस्तस्मा॒द्रक्षा॒ꣴ॒स्यप॑ हन्त्यक्ष्ण॒या व्याघा॑रयति॒ तस्मा॑दक्ष्ण॒या प॒शवो\-ऽङ्गा॑नि॒ प्र ह॑रन्ति॒ प्रति॑ष्ठित्यै॥~(४०)

%5.2.8.0
{\anuvakamend[{वन॒स्पत॑यो दक्षिण॒तो ध॑त्त एन॒न्तस्मा॑दक्ष्ण॒या पञ्च॑ च}]}%~(७)

%5.2.8.1
स्व॒य॒मा॒तृ॒ण्णामुप॑ दधाती॒यं वै स्व॑यमातृ॒ण्णेमामे॒वोप॑ ध॒त्ते\-ऽश्व॒मुप॑ घ्रापयति प्रा॒णमे॒वास्यां᳚ दधा॒त्यथो᳚ प्राजाप॒त्यो वा अश्वः॑ प्र॒जा\-प॑तिनै॒वाग्निं चि॑नुते प्रथ॒मेष्ट॑कोपधी॒यमा॑ना पशू॒नां च॒ यज॑मानस्य च प्रा॒णमपि॑ दधाति स्वयमातृ॒ण्णा भ॑वति प्रा॒णाना॒मुथ्सृ॑ष्ट्या॒ अथो॑ सुव॒र्गस्य॑ लो॒कस्यानु॑ख्यात्या अ॒ग्नाव॒ग्निश्चे॑त॒व्य॑ इत्या॑हुरे॒ष वै~(४१)

%5.2.8.2
अ॒ग्निर्वै᳚श्वान॒रो यद्ब्रा᳚ह्म॒णस्तस्मै᳚ प्रथ॒मामिष्ट॑कां॒ यजु॑ष्कृतां॒ प्र य॑च्छे॒त्तां ब्रा᳚ह्म॒णश्चोप॑ दध्याताम॒ग्नावे॒व तद॒ग्निं चि॑नुत ईश्व॒रो वा ए॒ष आर्ति॒मार्तो॒र्यो\-ऽवि॑द्वा॒निष्ट॑कामुप॒दधा॑ति॒ त्रीन् वरा᳚न्दद्या॒त्त्रयो॒ वै प्रा॒णाः प्रा॒णाना॒ꣴ॒ स्पृत्यै॒ द्वावे॒व देयौ॒ द्वौ हि प्रा॒णावेक॑ ए॒व देय॒ एको॒ हि प्रा॒णः प॒शुः~(४२)

%5.2.8.3
वा ए॒ष यद॒ग्निर्न खलु॒ वै प॒शव॒ आय॑वसे रमन्ते दूर्वेष्ट॒कामुप॑ दधाति पशू॒नां धृत्यै॒ द्वाभ्यां॒ प्रति॑ष्ठित्यै॒ काण्डा᳚त्काण्डात्प्र॒रोह॒न्तीत्या॑ह॒ काण्डे॑नकाण्डेन॒ ह्ये॑षा प्र॑ति॒तिष्ठ॑त्ये॒वा नो॑ दूर्वे॒ प्र त॑नु स॒हस्रे॑ण श॒तेन॒ चेत्या॑ह साह॒स्रः प्र॒जा\-प॑तिः प्र॒जा\-प॑ते॒राप्त्यै॑ देवल॒क्ष्मं वै त्र्या॑लिखि॒ता तामुत्त॑रलक्ष्माणं दे॒वा उपा॑दध॒ताध॑रलक्ष्माण॒मसु॑रा॒ यम्~(४३)

%5.2.8.4
का॒मये॑त॒ वसी॑यान्थ्स्या॒दित्युत्त॑रलक्ष्माणं॒ तस्योप॑ दध्या॒द्वसी॑याने॒व भ॑वति॒ यं का॒मये॑त॒ पापी॑यान्थ्स्या॒दित्यध॑र\-लक्ष्माणं॒ तस्योप॑ दध्यादसुरयो॒निमे॒वैन॒मनु॒ परा॑ भावयति॒ पापी॑यान्भवति त्र्यालिखि॒ता भ॑वती॒मे वै लो॒का\-स्त्र्या॑लिखि॒तैभ्य ए॒व लो॒केभ्यो॒ भ्रातृ॑व्यम॒न्तरे॒त्यङ्गि॑रसः सुव॒र्गं लो॒कं य॒तः पु॑रो॒डाशः॑ कू॒र्मो भू॒त्वानु॒ प्रास॑र्पत्~(४४)

%5.2.8.5
यत्कू॒र्ममु॑प॒दधा॑ति॒ यथा᳚ क्षेत्र॒विदञ्ज॑सा॒ नय॑त्ये॒वमे॒वैनं॑ कू॒र्मः सु॑व॒र्गं लो॒कमञ्ज॑सा नयति॒ मेधो॒ वा ए॒ष प॑शू॒नां यत्कू॒र्मो यत्कू॒र्ममु॑प॒दधा॑ति॒ स्वमे॒व मेधं॒ पश्य॑न्तः प॒शव॒ उप॑ तिष्ठन्ते श्मशा॒नं वा ए॒तत्क्रि॑यते॒ यन्मृ॒तानां᳚ पशू॒नाꣳ शी॒र्॒\mbox{}षाण्यु॑पधी॒यन्ते॒ यज्जीव॑न्तं कू॒र्ममु॑प॒दधा॑ति॒ तेनाश्म॑शानचिद्वास्त॒व्यो॑ वा ए॒ष यत्~(४५)

%5.2.8.6
कू॒र्मो मधु॒ वाता॑ ऋताय॒त इति॑ द॒ध्ना म॑धुमि॒श्रेणा॒भ्य॑नक्ति स्व॒दय॑त्ये॒वैनं॑ ग्रा॒म्यं वा ए॒तदन्नं॒ यद्दध्या॑र॒ण्यं मधु॒ यद्द॒ध्ना म॑धुमि॒श्रेणा᳚भ्य॒नक्त्यु॒भय॒स्याव॑रुद्ध्यै म॒ही द्यौः पृ॑थि॒वी च॑ न॒ इत्या॑हा॒भ्यामे॒वैन॑मुभ॒यतः॒ परि॑ गृह्णाति॒ प्राञ्च॒मुप॑ दधाति॒ सुव॒र्गस्य॑ लो॒कस्य॒ सम॑ष्ट्यै पु॒रस्ता᳚त्प्र॒त्यञ्च॒मुप॑ दधाति॒ तस्मा᳚त्~(४६)

%5.2.8.7
पु॒रस्ता᳚त्प्र॒त्यञ्चः॑ प॒शवो॒ मेध॒मुप॑ तिष्ठन्ते॒ यो वा अप॑नाभिम॒ग्निं चि॑नु॒ते यज॑मानस्य॒ नाभि॒मनु॒ प्र वि॑शति॒ स ए॑नमीश्व॒रो हिꣳसि॑तोरु॒लूख॑ल॒मुप॑ दधात्ये॒षा वा अ॒ग्नेर्नाभिः॒ सना॑भिमे॒वाग्निं चि॑नु॒ते\-ऽहिꣳ॑साया॒ औ॑दुम्बरं भव॒त्यूर्ग्वा उ॑दु॒म्बर॒ ऊर्ज॑मे॒वाव॑ रुन्धे मध्य॒त उप॑ दधाति मध्य॒त ए॒वास्मा॒ ऊर्जं॑ दधाति॒ तस्मा᳚न्मध्य॒त ऊ॒र्जा भु॑ञ्जत॒ इय॑द्भवति प्र॒जा\-प॑तिना यज्ञमु॒खेन॒ सम्मि॑त॒मव॑ ह॒न्त्यन्न॑मे॒वाक॑र्वैष्ण॒व्यर्चोप॑ दधाति॒ विष्णु॒र्वै य॒ज्ञो वै᳚ष्ण॒वा वन॒स्पत॑यो य॒ज्ञ ए॒व य॒ज्ञं प्रति॑\-ष्ठापयति॥~(४७)

%5.2.9.0
{\anuvakamend[{ए॒ष वै प॒शुर्यम॑सर्पदे॒ष यत्तस्मा॒त्तस्मा᳚थ्स॒प्तविꣳ॑शतिश्च}]}%~(८)

%5.2.9.1
ए॒षां वा ए॒तल्लो॒कानां॒ ज्योतिः॒ सम्भृ॑तं॒ यदु॒खा यदु॒खामु॑प॒दधा᳚त्ये॒भ्य ए॒व लो॒केभ्यो॒ ज्योति॒रव॑ रुन्धे मध्य॒त उप॑ दधाति मध्य॒त ए॒वास्मै॒ ज्योति॑र्दधाति॒ तस्मा᳚न्मध्य॒तो ज्योति॒रुपा᳚स्महे॒ सिक॑ताभिः पूरयत्ये॒तद्वा अ॒ग्नेर्वै᳚श्वान॒रस्य॑ रू॒पꣳ रू॒पेणै॒व वै᳚श्वान॒रमव॑ रुन्धे॒ यं का॒मये॑त॒ क्षोधु॑कः स्या॒दित्यू॒नां तस्योप॑~(४८)

%5.2.9.2
द॒ध्या॒त्क्षोधु॑क ए॒व भ॑वति॒ यं का॒मये॒तानु॑पदस्य॒दन्न॑मद्या॒दिति॑ पू॒र्णां तस्योप॑ दध्या॒दनु॑पदस्यदे॒वान्न॑मत्ति स॒हस्रं॒ वै प्रति॒ पुरु॑षः पशू॒नां य॑च्छति स॒हस्र॑म॒न्ये प॒शवो॒ मध्ये॑ पुरुषशी॒र्॒\mbox{}षमुप॑ दधाति सवीर्य॒त्वायो॒खाया॒मपि॑ दधाति प्रति॒ष्ठामे॒वैन॑द्गमयति॒ व्यृ॑द्धं॒ वा ए॒तत्प्रा॒णैर॑मे॒ध्यं यत्पु॑रुषशी॒र्॒\mbox{}षम॒मृतं॒ खलु॒ वै प्रा॒णाः~(४९)

%5.2.9.3
अ॒मृत॒ꣳ॒ हिर॑ण्यं प्रा॒णेषु॑ हिरण्यश॒ल्कान्प्रत्य॑स्यति प्रति॒ष्ठामे॒वैन॑द्गमयि॒त्वा प्रा॒णैः सम॑र्धयति द॒ध्ना म॑धुमि॒श्रेण॑ पूरयति मध॒व्यो॑\-ऽसा॒नीति॑ शृतात॒ङ्क्ये॑न मेध्य॒त्वाय॑ ग्रा॒म्यं वा ए॒तदन्नं॒ यद्दध्या॑र॒ण्यं मधु॒ यद्द॒ध्ना म॑धुमि॒श्रेण॑ पू॒रय॑त्यु॒भय॒स्याव॑रुद्ध्यै पशुशी॒र्॒\mbox{}षाण्युप॑ दधाति प॒शवो॒ वै प॑शुशी॒र्॒\mbox{}षाणि॑ प॒शूने॒वाव॑ रुन्धे॒ यं का॒मये॑ताप॒शुः स्या॒दिति॑~(५०)

%5.2.9.4
वि॒षू॒चीना॑नि॒ तस्योप॑ दध्या॒द्विषू॑च ए॒वास्मा᳚त्प॒शून्द॑धात्यप॒शुरे॒व भ॑वति॒ यं का॒मये॑त पशु॒मान्थ्स्या॒दिति॑ समी॒चीना॑नि॒ तस्योप॑ दध्याथ्स॒मीच॑ ए॒वास्मै॑ प॒शून्द॑धाति पशु॒माने॒व भ॑वति पु॒रस्ता᳚त्प्रती॒चीन॒मश्व॒स्योप॑ दधाति प॒श्चात्प्रा॒चीन॑मृष॒भस्याप॑शवो॒ वा अ॒न्ये गो॑अ॒श्वेभ्यः॑ प॒शवो॑ गोअ॒श्वाने॒वास्मै॑ स॒मीचो॑ दधात्ये॒ताव॑न्तो॒ वै प॒शवः॑~(५१)

%5.2.9.5
द्वि॒पाद॑श्च॒ चतु॑ष्पादश्च॒ तान् वा ए॒तद॒ग्नौ प्र द॑धाति॒ यत्प॑शुशी॒र्॒\mbox{}षाण्यु॑प॒दधा᳚त्य॒मुमा॑र॒ण्यमनु॑ ते दिशा॒मीत्या॑ह ग्रा॒म्येभ्य॑ ए॒व प॒शुभ्य॑ आर॒ण्यान्प॒शूञ्छुच॒मनूथ्सृ॑जति॒ तस्मा᳚थ्स॒माव॑त्पशू॒नां प्र॒जाय॑मानानामार॒ण्याः प॒शवः॒ कनी॑याꣳसः शु॒चा ह्यृ॑ताः स॑र्पशी॒र्॒\mbox{}षमुप॑ दधाति॒ यैव स॒र्पे त्विषि॒स्तामे॒वाव॑ रुन्धे~(५२)

%5.2.9.6
यथ्स॑मी॒चीनं॑ पशुशी॒र्॒\mbox{}षैरु॑पद॒ध्याद्ग्रा॒म्यान्प॒शून्दꣳशु॑काः स्यु॒र्यद्वि॑षू॒चीन॑मार॒ण्यान् यजु॑रे॒व व॑दे॒दव॒ तां त्विषिꣳ॑ रुन्धे॒ या स॒र्पे न ग्रा॒म्यान्प॒शून् हि॒नस्ति॒ नार॒ण्यानथो॒ खलू॑प॒धेय॑मे॒व यदु॑प॒दधा॑ति॒ तेन॒ तां त्विषि॒मव॑ रुन्धे॒ या स॒र्पे यद्यजु॒र्वद॑ति॒ तेन॑ शा॒न्तम्॥~(५३)

%5.2.10.0
{\anuvakamend[{ऊ॒नान्तस्योप॑ प्रा॒णाः स्या॒दिति॒ वै प॒शवो॑ रुन्धे॒ चतु॑श्चत्वारिꣳशच्च}]}%~(९)

%5.2.10.1
प॒शुर्वा ए॒ष यद॒ग्निर्योनिः॒ खलु॒ वा ए॒षा प॒शोर्वि क्रि॑यते॒ यत्प्रा॒चीन॑मैष्ट॒काद्यजुः॑ क्रि॒यते॒ रेतो॑\-ऽप॒स्या॑ अप॒स्या॑ उप॑ दधाति॒ योना॑वे॒व रेतो॑ दधाति॒ पञ्चोप॑ दधाति॒ पाङ्क्ताः᳚ प॒शवः॑ प॒शूने॒वास्मै॒ प्र ज॑नयति॒ पञ्च॑ दक्षिण॒तो वज्रो॒ वा अ॑प॒स्या॑ वज्रे॑णै॒व य॒ज्ञस्य॑ दक्षिण॒तो रक्षा॒ꣴ॒स्यप॑ हन्ति॒ पञ्च॑ प॒श्चात्~(५४)

%5.2.10.2
प्राची॒रुप॑ दधाति प॒श्चाद्वै प्रा॒चीन॒ꣳ॒ रेतो॑ धीयते प॒श्चादे॒वास्मै᳚ प्रा॒चीन॒ꣳ॒ रेतो॑ दधाति॒ पञ्च॑ पु॒रस्ता᳚त्प्र॒तीची॒रुप॑ दधाति॒ पञ्च॑ प॒श्चात्प्राची॒स्तस्मा᳚त्प्रा॒चीन॒ꣳ॒ रेतो॑ धीयते प्र॒तीचीः᳚ प्र॒जा जा॑यन्ते॒ पञ्चो᳚त्तर॒तश्छ॑न्द॒स्याः᳚ प॒शवो॒ वै छ॑न्द॒स्याः᳚ प॒शूने॒व प्रजा॑ता॒न्थ्स्वमा॒यत॑नम॒भि पर्यू॑हत इ॒यं वा अ॒ग्नेर॑तिदा॒हाद॑बिभे॒थ्सैताः~(५)

%5.2.10.3
अ॒प॒स्या॑ अपश्य॒त्ता उपा॑धत्त॒ ततो॒ वा इ॒मां नात्य॑दह॒द्यद॑प॒स्या॑ उप॒दधा᳚त्य॒स्या अन॑तिदाहायो॒वाच॑ हे॒यमद॒दिथ्स ब्रह्म॒णान्नं॒ यस्यै॒ता उ॑पधी॒यान्तै॒ य उ॑ चैना ए॒वं वेद॒दिति॑ प्राण॒भृत॒ उप॑ दधाति॒ रेत॑स्ये॒व प्रा॒णान्द॑धाति॒ तस्मा॒द्वद॑न्प्रा॒णन्पश्य॑ञ्छृ॒ण्वन्प॒शुर्जा॑यते॒\-ऽयं पु॒रः~(५६)

%5.2.10.4
भुव॒ इति॑ पु॒रस्ता॒दुप॑ दधाति प्रा॒णमे॒वैताभि॑र्दाधारा॒यं द॑क्षि॒णा वि॒श्वक॒र्मेति॑ दक्षिण॒तो मन॑ ए॒वैताभि॑र्दाधारा॒यं प॒श्चाद्वि॒श्वव्य॑चा॒ इति॑ प॒श्चाच्चक्षु॑रे॒वैताभि॑र्दाधारे॒दमु॑त्त॒राथ्सुव॒रित्यु॑त्तर॒तः श्रोत्र॑मे॒वैताभि॑र्दाधारे॒यमु॒परि॑ म॒तिरित्यु॒परि॑ष्टा॒द्वाच॑मे॒वैताभि॑र्दाधार॒ दश॑द॒शोप॑ दधाति सवीर्य॒त्वाया᳚क्ष्ण॒या~(५७)

%5.2.10.5
उप॑ दधाति॒ तस्मा॑दक्ष्ण॒या प॒शवो\-ऽङ्गा॑नि॒ प्र ह॑रन्ति॒ प्रति॑ष्ठित्यै॒ याः प्राची॒स्ताभि॒र्वसि॑ष्ठ आर्ध्नो॒द्या द॑क्षि॒णा ताभि॑र्भ॒रद्वा॑जो॒ याः प्र॒तीची॒स्ताभि॑र्वि॒श्वामि॑त्रो॒ या उदी॑ची॒स्ताभि॑र्ज॒मद॑ग्नि॒र्या ऊ॒र्ध्वास्ताभि॑र्वि॒श्वक॑र्मा॒ य ए॒वमे॒तासा॒मृद्धिं॒ वेद॒र्ध्नोत्ये॒व य आ॑सामे॒वं ब॒न्धुतां॒ वेद॒ बन्धु॑मान्भवति॒ य आ॑सामे॒वं कॢप्तिं॒ वेद॒ कल्प॑ते~(५८)

%5.2.10.6
अ॒स्मै॒ य आ॑सामे॒वमा॒यत॑नं॒ वेदा॒यत॑नवान्भवति॒ य आ॑सामे॒वं प्र॑ति॒ष्ठां वेद॒ प्रत्ये॒व ति॑ष्ठति प्राण॒भृत॑ उप॒धाय॑ सं॒यत॒ उप॑ दधाति प्रा॒णाने॒वास्मि॑न्धि॒त्वा सं॒यद्भिः॒ सं य॑च्छति॒ तथ्सं॒यताꣳ॑ संय॒त्त्वमथो᳚ प्रा॒ण ए॒वापा॒नं द॑धाति॒ तस्मा᳚त्प्राणापा॒नौ सं च॑रतो॒ विषू॑ची॒रुप॑ दधाति॒ तस्मा॒द्विष्व॑ञ्चौ प्राणापा॒नौ यद्वा अ॒ग्नेरसं॑ यतम्~(५९)

%5.2.10.7
असु॑वर्ग्यमस्य॒ तथ्सु॑व॒र्ग्यो᳚\-ऽग्निर्यथ्सं॒ यत॑ उप॒दधा॑ति॒ समे॒वैनं॑ यच्छति सुव॒र्ग्य॑मे॒वाक॒स्त्र्यवि॒र्वयः॑ कृ॒तमया॑ना॒मित्या॑ह॒ वयो॑भिरे॒वाया॒नव॑ रु॒न्धे\-ऽयै॒र्वयाꣳ॑सि स॒र्वतो॑ वायु॒मती᳚र्भवन्ति॒ तस्मा॑द॒यꣳ स॒र्वतः॑ पवते॥~(६०)

%5.2.11.0
{\anuvakamend[{प॒श्चादे॒ताः पु॒रो᳚\-ऽक्ष्ण॒या कल्प॒ते\-ऽसं॑ यतं॒ पञ्च॑त्रिꣳशच्च}]}%॥10॥

%5.2.11.1
गा॒य॒त्री त्रि॒ष्टुब्जग॑त्यनु॒ष्टुक्प॒ङ्क्त्या॑ स॒ह। बृ॒ह॒त्यु॑ष्णिहा॑ क॒कुथ्सू॒चीभिः॑ शिम्यन्तु त्वा। द्वि॒पदा॒ या चतु॑ष्पदा त्रि॒पदा॒ या च॒ षट्प॑दा। सछ॑न्दा॒ या च॒ विच्छ॑न्दाः सू॒चीभिः॑ शिम्यन्तु त्वा। म॒हाना᳚म्नी रे॒वत॑यो॒ विश्वा॒ आशाः᳚ प्र॒सूव॑रीः। मेघ्या॑ वि॒द्युतो॒ वाचः॑ सू॒चीभिः॑ शिम्यन्तु त्वा। र॒ज॒ता हरि॑णीः॒ सीसा॒ युजो॑ युज्यन्ते॒ कर्म॑भिः। अश्व॑स्य वा॒जिन॑स्त्व॒चि सू॒चीभिः॑ शिम्यन्तु त्वा। नारीः᳚~(६१)

%5.2.11.2
ते॒ पत्न॑यो॒ लोम॒ वि चि॑न्वन्तु मनी॒षया᳚। दे॒वानां॒ पत्नी॒र्दिशः॑ सू॒चीभिः॑ शिम्यन्तु त्वा। कु॒विद॒ङ्ग यव॑मन्तो॒ यवं॑ चि॒द्यथा॒ दान्त्य॑नुपू॒र्वं वि॒यूय॑। इ॒हेहै॑षां कृणुत॒ भोज॑नानि॒ ये ब॒र्॒\mbox{}हिषो॒ नमो॑वृक्तिं॒ न ज॒ग्मुः॥~(६२)

%5.2.12.0
{\anuvakamend[{नारी᳚स्त्रि॒ꣳ॒शच्च॑}]}%॥11॥

%5.2.12.1
कस्त्वा᳚ छ्यति॒ कस्त्वा॒ वि शा᳚स्ति॒ कस्ते॒ गात्रा॑णि शिम्यति। क उ॑ ते शमि॒ता क॒विः। ऋ॒तव॑स्त ऋतु॒धा परुः॑ शमि॒तारो॒ वि शा॑सतु। सं॒व॒थ्स॒रस्य॒ धाय॑सा॒ शिमी॑भिः शिम्यन्तु त्वा। दैव्या॑ अध्व॒र्यव॑स्त्वा॒ छ्यन्तु॒ वि च॑ शासतु। गात्रा॑णि पर्व॒शस्ते॒ शिमाः᳚ कृण्वन्तु॒ शिम्य॑न्तः। अ॒र्ध॒मा॒साः परूꣳ॑षि ते॒ मासा᳚श्छ्यन्तु॒ शिम्य॑न्तः। अ॒हो॒रा॒त्राणि॑ म॒रुतो॒ विलि॑ष्टं~(६३)

%5.2.12.2
सू॒द॒य॒न्तु॒ ते॒। पृ॒थि॒वी ते॒\-ऽन्तरि॑क्षेण वा॒युश्छि॒द्रं भि॑षज्यतु। द्यौस्ते॒ नक्ष॑त्रैः स॒ह रू॒पं कृ॑णोतु साधु॒या। शं ते॒ परे᳚भ्यो॒ गात्रे᳚भ्यः॒ शम॒स्त्वव॑रेभ्यः। शम॒स्थभ्यो॑ म॒ज्जभ्यः॒ शमु॑ ते त॒नुवे॑ भुवत्~(६४)

%5.3.0.0

%5.3.0.0
{\anuvakamend[{विलि॑ष्टन्त्रि॒ꣳ॒शच्च॑}]}%॥12॥

{\anuvakamend[{उ॒थ्स॒न्न॒य॒ज्ञ इन्द्रा᳚ग्नी दे॒वा वा अ॑क्षणयास्तो॒मीया॑ अ॒ग्नेर्भा॒गो᳚\-ऽस्यग्ने॑ जा॒तान्र॒श्मिरिति॑ नाक॒सद्भि॒श्छन्दाꣳ॑सि॒ सर्वा᳚भ्यो वृष्टि॒सनी᳚र्देवासु॒राः कनी॑याꣳसः प्र॒जा\-प॑ते॒रक्षि॒ द्वाद॑श}]}%॥12॥ 
{\prashnaend{उ॒थ्स॒न्न॒य॒ज्ञो दे॒वा वै यस्य॒ मुख्य॑वतीर्नाक॒सद्भि॑रे॒वैताभि॑र॒ष्टाच॑त्वारिꣳशत्॥48॥ उ॒थ्स॒न्न॒य॒ज्ञः स॑र्व॒त्वाय॑॥}}
%%% END PRASHNA
