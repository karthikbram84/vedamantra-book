\chapt{काण्डम् १}
\sect{प्रथमः प्रश्नः}\setcounter{anuvakam}{0}
\dnsub{तैत्तिरीयसंहितायां प्रथमकाण्डे प्रथमः प्रश्नः}
%1.1.1.1
इ॒षे त्वो॒र्जे त्वा॑ वा॒यवः॑ स्थोपा॒यवः॑ स्थ दे॒वो वः॑ सवि॒ता प्रार्प॑यतु॒ श्रेष्ठ॑तमाय॒ कर्म॑ण॒ आ प्या॑यध्वमघ्निया देवभा॒गमूर्ज॑स्वती॒ पय॑स्वतीः प्र॒जाव॑तीरनमी॒वा अ॑य॒क्ष्मा मा वः॑ स्ते॒न ई॑शत॒ मा\-ऽघशꣳ॑सो रु॒द्रस्य॑ हे॒तिः परि॑ वो वृणक्तु ध्रु॒वा अ॒स्मिन्गोप॑तौ स्यात ब॒ह्वीर्यज॑मानस्य प॒शून्पा॑हि॥~(१)

%1.1.2.0
{\anuvakamend[{इ॒षे त्रिच॑त्वारिꣳशत्}]}

%1.1.2.1
य॒ज्ञस्य॑ घो॒षद॑सि॒ प्रत्यु॑ष्ट॒ꣳ॒ रक्ष॒ प्रत्यु॑ष्टा॒ अरा॑तय॒ प्रेयम॑गाद्धि॒षणा॑ ब॒र्॒हिरच्छ॒ मनु॑ना कृ॒ता स्व॒धया॒ वित॑ष्टा॒ त आव॑हन्ति क॒वय॑ पु॒रस्ता᳚द्दे॒वेभ्यो॒ जुष्ट॑मि॒ह ब॒र्॒हिरा॒सदे॑ दे॒वानां᳚ परिषू॒तम॑सि व॒र्॒षवृ॑द्धमसि॒ देव॑बर्\mbox{}हि॒र्मा त्वा॒\-ऽन्वङ्मा ति॒र्यक्पर्व॑ ते राध्यासमाच्छे॒त्ता ते॒ मा रि॑षं॒ देव॑बर्\mbox{}हिः श॒तव॑ल्\mbox{}शं॒ वि रो॑ह स॒हस्र॑वल्\mbox{}शा॒~(२)

%1.1.2.2
वि व॒यꣳ रु॑हेम पृथि॒व्याः स॒म्पृच॑ पाहि सुस॒म्भृता᳚ त्वा॒ सम्भ॑रा॒म्यदि॑त्यै॒ रास्ना॑\-ऽसीन्द्रा॒ण्यै स॒न्नह॑नं पू॒षा ते᳚ ग्र॒न्थिं ग्र॑थ्नातु॒ स ते॒ मा\-ऽ\-ऽस्था॒दिन्द्र॑स्य त्वा बा॒हुभ्या॒मुद्य॑च्छे॒ बृह॒स्पते᳚र्मू॒र्ध्ना ह॑राम्यु॒र्व॑न्तरि॑क्ष॒मन्वि॑हि देवङ्ग॒मम॑सि॥~(३)

%1.1.3.0
{\anuvakamend[{स॒हस्र॑वल्\mbox{}शा अ॒ष्टात्रिꣳ॑शच्च}]}

%1.1.3.1
शुन्ध॑ध्वं॒ दैव्या॑य॒ कर्म॑णे देवय॒ज्यायै॑ मात॒रिश्व॑नो घ॒र्मो॑\-ऽसि॒ द्यौर॑सि पृथि॒व्य॑सि वि॒श्वधा॑या असि पर॒मेण॒ धाम्ना॒ दृꣳह॑स्व॒ मा ह्वा॒र्वसू॑नां प॒वित्र॑मसि श॒तधा॑रं॒ वसू॑नां प॒वित्र॑मसि स॒हस्र॑धारꣳ हु॒तः स्तो॒को हु॒तो द्र॒फ्सो᳚\-ऽग्नये॑ बृह॒ते नाका॑य॒ स्वाहा॒ द्यावा॑पृथि॒वीभ्या॒ꣳ॒ सा वि॒श्वायुः॒ सा वि॒श्वव्य॑चाः॒ सा वि॒श्वक॑र्मा॒ सम्पृ॑च्यध्वमृतावरीरू॒र्मिणी॒र्मधु॑मत्तमा म॒न्द्रा धन॑स्य सा॒तये॒ सोमे॑न॒ त्वा\-ऽ\-ऽत॑न॒च्मीन्द्रा॑य॒ दधि॒ विष्णो॑ ह॒व्यꣳ र॑क्षस्व॥~(४)

%1.1.4.0
{\anuvakamend[{सोमे॑ना॒ष्टौ च॑}]}

%1.1.4.1
कर्म॑णे वां दे॒वेभ्यः॑ शकेयं॒ वेषा॑य त्वा॒ प्रत्यु॑ष्ट॒ꣳ॒ रक्ष॒ प्रत्यु॑ष्टा॒ अरा॑तयो॒ धूर॑सि॒ धूर्व॒ धूर्व॑न्तं॒ धूर्व॒ तं यो᳚\-ऽस्मान्धूर्व॑ति॒ तं धू᳚र्व॒ यं व॒यं धूर्वा॑म॒स्त्वं दे॒वाना॑मसि॒ सस्नि॑तमं॒ पप्रि॑तमं॒ जुष्ट॑तमं॒ वह्नि॑तमं देव॒हूत॑म॒मह्रु॑तमसि हवि॒र्धानं॒ दृꣳह॑स्व॒ मा ह्वा᳚र्मि॒त्रस्य॑ त्वा॒ चक्षु॑षा॒ प्रेक्षे॒ मा भेर्मा सं वि॑क्था॒ मा त्वा॑~(५)

%1.1.4.2
हिꣳसिषमु॒रु वाता॑य दे॒वस्य॑ त्वा सवि॒तुः प्र॑स॒वे᳚\-ऽश्विनो᳚र्बा॒हु\-भ्यां᳚ पू॒ष्णो हस्ता᳚भ्याम॒ग्नये॒ जुष्टं॒ निर्व॑पाम्य॒ग्नी\-षोमा᳚भ्यामि॒दं दे॒वाना॑मि॒दमु॑ नः स॒ह स्फा॒त्यै त्वा॒ नारा᳚त्यै॒ सुव॑र॒भि वि ख्ये॑षं वैश्वान॒रं ज्योति॒र्दृꣳह॑न्ता॒न्दुर्या॒ द्यावा॑\-पृथि॒व्योरु॒र्व॑न्तरि॑क्ष॒\-मन्वि॒ह्यदि॑त्यास्त्वो॒\-पस्थे॑ सादया॒म्यग्ने॑ ह॒व्यꣳ र॑क्षस्व॥~(६)

%1.1.5.0
{\anuvakamend[{मा त्वा॒ षट्च॑त्वारिꣳशच्च}]}

%1.1.5.1
दे॒वो वः॑ सवि॒तोत्पु॑ना॒त्वच्छि॑द्रेण प॒वित्रे॑ण॒ वसोः॒ सूर्य॑स्य र॒श्मिभि॒रापो॑ देवीरग्रेपुवो अग्रेगु॒वो\-ऽग्र॑ इ॒मं य॒ज्ञं न॑य॒ताग्रे॑ य॒ज्ञप॑तिं धत्त यु॒ष्मानिन्द्रो॑\-ऽवृणीत वृत्र॒तूर्ये॑ यू॒यमिन्द्र॑मवृणीध्वं वृत्र॒तूर्ये॒ प्रोक्षि॑ताः स्था॒ग्नये॑ वो॒ जुष्टं॒ प्रोक्षा᳚म्य॒ग्नी\-षोमा᳚भ्या॒ꣳ॒ शुन्ध॑ध्वं॒ दैव्या॑य॒ कर्म॑णे देवय॒ज्याया॒ अव॑धूत॒ꣳ॒ रक्षो\-ऽव॑धूता॒ अरा॑त॒यो\-ऽदि॑त्या॒स्त्वग॑सि॒ प्रति॑ त्वा~(७)

%1.1.5.2
पृथि॒वी वे᳚त्त्वधि॒षव॑णमसि वानस्प॒त्यं प्रति॒ त्वा\-ऽदि॑त्या॒स्त्वग्वे᳚त्त्व॒ग्नेस्त॒नूर॑सि वा॒चो वि॒सर्ज॑नं दे॒ववी॑तये त्वा गृह्णा॒म्यद्रि॑रसि वानस्प॒त्यः स इ॒दं दे॒वेभ्यो॑ ह॒व्यꣳ सु॒शमि॑ शमि॒ष्वेष॒मा व॒दोर्ज॒मा व॑द द्यु॒मद्व॑दत व॒यꣳ स॑ङ्घा॒तं जे᳚ष्म व॒र्॒षवृ॑द्धमसि॒ प्रति॑ त्वा व॒र्॒षवृ॑द्धं वेत्तु॒ परा॑पूत॒ꣳ॒ रक्ष॒ परा॑पूता॒ अरा॑तयो॒ रक्ष॑सां भा॒गो॑\-ऽसि वा॒युर्वो॒ विवि॑नक्तु दे॒वो वः॑ सवि॒ता हिर॑ण्यपाणि॒ प्रति॑ गृह्णातु॥~(८)

%1.1.6.0
{\anuvakamend[{त्वा॒ भा॒ग एका॑\-दश च}]}

%1.1.6.1
अव॑धूत॒ꣳ॒ रक्षो\-ऽव॑धूता॒ अरा॑त॒यो\-ऽदि॑त्या॒स्त्वग॑सि॒ प्रति॑ त्वा पृथि॒वी वे᳚त्तु दि॒वः स्क॑म्भ॒निर॑सि॒ प्रति॒ त्वा\-ऽदि॑त्या॒स्त्वग्वे᳚त्तु धि॒षणा॑\-ऽसि पर्व॒त्या प्रति॑ त्वा दि॒वः स्क॑म्भ॒निर्वे᳚त्तु धि॒षणा॑\-ऽसि पार्वते॒यी प्रति॑ त्वा पर्व॒तिर्वे᳚त्तु दे॒वस्य॑ त्वा सवि॒तुः प्र॑स॒वे᳚\-ऽश्विनो᳚र्बा॒हु\-भ्यां᳚ पू॒ष्णो हस्ता᳚भ्या॒मधि॑वपामि धा॒न्य॑मसि धिनु॒हि दे॒वान्प्रा॒णाय॑ त्वा\-ऽपा॒नाय॑ त्वा व्या॒नाय॑ त्वा दी॒र्घामनु॒ प्रसि॑ति॒मायु॑षे धां दे॒वो वः॑ सवि॒ता हिर॑ण्यपाणि॒ प्रति॑ गृह्णातु॥~(९)

%1.1.7.0
{\anuvakamend[{प्रा॒णाय॑ त्वा॒ पञ्च॑दश च}]}

%1.1.7.1
धृष्टि॑रसि॒ ब्रह्म॑ य॒च्छापा᳚\-ऽग्ने॒\-ऽग्निमा॒मादं॑ जहि॒ निष्क्र॒व्यादꣳ॑ से॒धा दे॑व॒यजं॑ वह॒ निर्द॑ग्ध॒ꣳ॒ रक्षो॒ निर्द॑ग्धा॒ अरा॑तयो ध्रु॒वम॑सि पृथि॒वीं दृ॒ꣳ॒हाऽऽयु॑र्दृꣳह प्र॒जां दृꣳ॑ह सजा॒तान॒स्मै यज॑मानाय॒ पर्यू॑ह ध॒र्त्रम॑स्य॒न्तरि॑क्षं दृꣳह प्रा॒णं दृꣳ॑हापा॒नं दृꣳ॑ह सजा॒ता\-न॒स्मै यज॑मानाय॒ पर्यू॑ह ध॒रुण॑मसि॒ दिवं॑ दृꣳह॒ चक्षु॑र्~(१०)

%1.1.7.2
दृꣳह॒ श्रोत्रं॑ दृꣳह सजा॒तान॒स्मै यज॑मानाय॒ पर्यू॑ह॒ धर्मा॑\-ऽसि॒ दिशो॑ दृꣳह॒ योनिं॑ दृꣳह प्र॒जां दृꣳ॑ह सजा॒तान॒स्मै यज॑मानाय॒ पर्यू॑ह॒ चितः॑ स्थ प्र॒जाम॒स्मै र॒यिम॒स्मै स॑जा॒तान॒स्मै यज॑मानाय॒ पर्यू॑ह॒ भृगू॑णा॒मङ्गि॑रसां॒ तप॑सा तप्यध्वं॒ यानि॑ घ॒र्मे क॒पाला᳚न्युपचि॒न्वन्ति॑ वे॒धसः॑। पू॒ष्णस्तान्यपि॑ व्र॒त इ॑न्द्रवा॒यू वि मु॑ञ्चताम्॥~(११)

%1.1.8.0
{\anuvakamend[{चक्षु॑र॒ष्टाच॑त्वारिꣳशच्च}]}

%1.1.8.1
सं व॑पामि॒ समापो॑ अ॒द्भिर॑ग्मत॒ समोष॑धयो॒ रसे॑न॒ सꣳ रे॒वती॒र्जग॑तीभि॒र्मधु॑मती॒र्मधु॑मतीभिः सृज्यध्वम॒द्भ्यः परि॒ प्रजा॑ताः स्थ॒ सम॒द्भिः पृ॑च्यध्वं॒ जन॑यत्यै त्वा॒ सं यौ᳚म्य॒ग्नये᳚ त्वा॒\-ऽग्नी\-षोमा᳚भ्यां म॒खस्य॒ शिरो॑\-ऽसि घ॒र्मो॑\-ऽसि वि॒श्वायु॑रु॒रु प्र॑थस्वो॒रु ते॑ य॒ज्ञप॑तिः प्रथतां॒ त्वचं॑ गृह्णीष्वा॒\-ऽन्तरि॑त॒ꣳ॒ रक्षो॒\-ऽन्तरि॑ता॒ अरा॑तयो दे॒वस्त्वा॑ सवि॒ता श्र॑पयतु॒ वर्\mbox{}षि॑ष्ठे॒ अधि॒ नाके॒\-ऽग्निस्ते॑ त॒नुवं॒ मा\-ऽति॑ धा॒गग्ने॑ ह॒व्यꣳ र॑क्षस्व॒ सं ब्रह्म॑णा पृच्यस्वैक॒ताय॒ स्वाहा᳚ द्वि॒ताय॒ स्वाहा᳚ त्रि॒ताय॒ स्वाहा᳚॥~(१२)

%1.1.9.0
{\anuvakamend[{स॒वि॒ता द्वाविꣳ॑शतिश्च}]}

%1.1.9.1
आद॑द॒ इन्द्र॑स्य बा॒हुर॑सि॒ दक्षि॑णः स॒हस्र॑भृष्टिः श॒तते॑जा वा॒युर॑सि ति॒ग्मते॑जा॒ पृथि॑वि देवयज॒न्योष॑ध्यास्ते॒ मूलं॒ मा हिꣳ॑सिष॒मप॑हतो॒\-ऽररु॑ पृथि॒व्यै व्र॒जं ग॑च्छ गो॒स्थानं॒ वर्\mbox{}ष॑तु ते॒ द्यौर्ब॑धा॒न दे॑व सवितः पर॒मस्यां᳚ परा॒वति॑ श॒तेन॒ पाशै॒र्यो᳚\-ऽस्मान्द्वेष्टि॒ यं च॑ व॒यं द्वि॒ष्मस्तमतो॒ मा मौ॒गप॑हतो॒\-ऽररु॑ पृथि॒व्यै दे॑व॒यज॑न्यै व्र॒जं~(१३)

%1.1.9.2
ग॑च्छ गो॒स्थानं॒ वर्\mbox{}ष॑तु ते॒ द्यौर्ब॑धा॒न दे॑व सवितः पर॒मस्यां᳚ परा॒वति॑ श॒तेन॒ पाशै॒र्यो᳚\-ऽस्मान्द्वेष्टि॒ यं च॑ व॒यं द्वि॒ष्मस्तमतो॒ मा मौ॒गप॑हतो॒\-ऽररु॑ पृथि॒व्या अदे॑वयजनो व्र॒जं ग॑च्छ गो॒स्थानं॒ वर्\mbox{}ष॑तु ते॒ द्यौर्ब॑धा॒न दे॑व सवितः पर॒मस्यां᳚ परा॒वति॑ श॒तेन॒ पाशै॒र्यो᳚\-ऽस्मान्द्वेष्टि॒ यं च॑ व॒यं द्वि॒ष्मस्तमतो॒ मा~(१४)

%1.1.9.3
मौ॑ग॒ररु॑स्ते॒ दिवं॒ मा स्का॒न्॒ वस॑वस्त्वा॒ परि॑गृह्णन्तु गाय॒त्रेण॒ छन्द॑सा रु॒द्रास्त्वा॒ परि॑गृह्णन्तु॒ त्रैष्टु॑भेन॒ छन्द॑सा\-ऽ\-ऽदि॒त्यास्त्वा॒ परि॑गृह्णन्तु॒ जाग॑तेन॒ छन्द॑सा दे॒वस्य॑ सवि॒तुः स॒वे कर्म॑ कृण्वन्ति वे॒धस॑ ऋ॒तम॑स्यृत॒सद॑नमस्यृत॒श्रीर॑सि॒ धा अ॑सि स्व॒धा अ॑स्यु॒र्वी चासि॒ वस्वी॑ चासि पु॒रा क्रू॒रस्य॑ वि॒सृपो॑ विरफ्शिन्नुदा॒दाय॑ पृथि॒वीं जी॒रदा॑नु॒र्यामैर॑यं च॒न्द्रम॑सि स्व॒धाभि॒स्तान्धीरा॑सो अनु॒दृश्य॑ यजन्ते॥~(१५)

%1.1.10.0
{\anuvakamend[{दे॒व॒यज॑न्यै व्र॒जन्तमतो॒ मा वि॑रफ्शि॒न्नेका॑\-दश च}]}

%1.1.10.1
प्रत्यु॑ष्ट॒ꣳ॒ रक्ष॒ प्रत्यु॑ष्टा॒ अरा॑तयो॒\-ऽग्नेर्व॒स्तेजि॑ष्ठेन॒ तेज॑सा॒ निष्ट॑पामि गो॒ष्ठं मा निर्मृ॑क्षं वा॒जिनं॑ त्वा सपत्नसा॒हꣳ सम्मा᳚र्ज्मि॒ वाचं॑ प्रा॒णं चक्षुः॒ श्रोत्रं॑ प्र॒जां योनिं॒ मा निर्मृ॑क्षं वा॒जिनीं᳚ त्वा सपत्नसा॒हीꣳ सम्मा᳚र्ज्म्या॒शासा॑ना सौमन॒सं प्र॒जाꣳ सौभा᳚ग्यं त॒नूम्। अ॒ग्नेरनु॑व्रता भू॒त्वा सन्न॑ह्ये सुकृ॒ताय॒ कम्। सु॒प्र॒जस॑स्त्वा व॒यꣳ सु॒पत्नी॒रुप॑~(१६)

%1.1.10.2
सेदिम। अग्ने॑ सपत्न॒दम्भ॑न॒मद॑ब्धासो॒ अदा᳚भ्यम्। इ॒मं विष्या॑मि॒ वरु॑णस्य॒ पाशं॒ यमब॑ध्नीत सवि॒ता सु॒शेवः॑। धा॒तुश्च॒ योनौ॑ सुकृ॒तस्य॑ लो॒के स्यो॒नं मे॑ स॒ह पत्या॑ करोमि। समायु॑षा॒ सम्प्र॒जया॒ सम॑ग्ने॒ वर्च॑सा॒ पुनः॑। सम्पत्नी॒ पत्या॒\-ऽहं ग॑च्छे॒ समा॒त्मा त॒नुवा॒ मम॑। म॒ही॒नां पयो॒\-ऽस्योष॑धीना॒ꣳ॒ रस॒स्तस्य॒ ते\-ऽक्षी॑यमाणस्य॒ निर्-~(१७)

%1.1.10.3
व॑पामि मही॒नां पयो॒\-ऽस्योष॑धीना॒ꣳ॒ रसो\-ऽद॑ब्धेन त्वा॒ चक्षु॒षा\-ऽवे᳚क्षे सुप्रजा॒स्त्वाय॒ तेजो॑\-ऽसि॒ तेजो\-ऽनु॒ प्रेह्य॒ग्निस्ते॒ तेजो॒ मा वि नै॑द॒ग्नेर्जि॒ह्वा\-ऽसि॑ सु॒भूर्दे॒वानां॒ धाम्ने॑धाम्ने दे॒वेभ्यो॒ यजु॑षेयजुषे भव शु॒क्रम॑सि॒ ज्योति॑रसि॒ तेजो॑\-ऽसि दे॒वो वः॑ सवि॒तोत्पु॑ना॒त्वच्छि॑द्रेण प॒वित्रे॑ण॒ वसोः॒ सूर्य॑स्य र॒श्मिभिः॑ शु॒क्रं त्वा॑ शु॒क्रायां॒ धाम्ने॑धाम्ने दे॒वेभ्यो॒ यजु॑षेयजुषे गृह्णामि॒ ज्योति॑स्त्वा॒ ज्योति॑ष्य॒र्चिस्त्वा॒\-ऽर्चिषि॒ धाम्ने॑धाम्ने दे॒वेभ्यो॒ यजु॑षेयजुषे गृह्णामि॥~(१८)

%1.1.11.0
{\anuvakamend[{उप॒ नी र॒श्मिभिः॑ शु॒क्रꣳ षोड॑श च}]}

%1.1.11.1
कृष्णो᳚\-ऽस्याखरे॒ष्ठो᳚\-ऽग्नये᳚ त्वा॒ स्वाहा॒ वेदि॑रसि ब॒र्॒हिषे᳚ त्वा॒ स्वाहा॑ ब॒र्॒हिर॑सि स्रु॒ग्भ्यस्त्वा॒ स्वाहा॑ दि॒वे त्वा॒\-ऽन्तरि॑क्षाय त्वा पृथि॒व्यै त्वा᳚ स्व॒धा पि॒तृभ्य॒ ऊर्ग्भ॑व बर्\mbox{}हि॒षद्भ्य॑ ऊ॒र्जा पृ॑थि॒वीं ग॑च्छत॒ विष्णोः॒ स्तूपो॒\-ऽस्यूर्णा᳚म्रदसं त्वा स्तृणामि स्वास॒स्थं दे॒वेभ्यो॑ गन्ध॒र्वो॑\-ऽसि वि॒श्वाव॑सु॒र्विश्व॑स्मा॒दीष॑तो॒ यज॑मानस्य परि॒धिरि॒ड ई॑डि॒त इन्द्र॑स्य बा॒हुर॑सि॒~(१९)

%1.1.11.2
दक्षि॑णो॒ यज॑मानस्य परि॒धिरि॒ड ई॑डि॒तो मि॒त्रावरु॑णौ त्वोत्तर॒तः परि॑धत्तां ध्रु॒वेण॒ धर्म॑णा॒ यज॑मानस्य परि॒धिरि॒ड ई॑डि॒तः सूर्य॑स्त्वा पु॒रस्ता᳚त्पातु॒ कस्या᳚श्चिद॒भिश॑स्त्या वी॒तिहो᳚त्रं त्वा कवे द्यु॒मन्त॒ꣳ॒ समि॑धीम॒ह्यग्ने॑ बृ॒हन्त॑मध्व॒रे वि॒शो य॒न्त्रे स्थो॒ वसू॑नाꣳ रु॒द्राणा॑मादि॒त्याना॒ꣳ॒ सद॑सि सीद जु॒हूरु॑प॒भृद्ध्रु॒वा\-ऽसि॑ घृ॒ताची॒ नाम्ना᳚ प्रि॒येण॒ नाम्ना᳚ प्रि॒ये सद॑सि सीदै॒ता अ॑सदन्थ्सुकृ॒तस्य॑ लो॒के ता वि॑ष्णो पाहि पा॒हि य॒ज्ञं पा॒हि य॒ज्ञप॑तिं पा॒हि मां य॑ज्ञ॒नियम्᳚॥~(२०)

%1.1.12.0
{\anuvakamend[{बा॒हुर॑सि प्रि॒ये सद॑सि पञ्च॑दश च}]}

%1.1.12.1
भुव॑नमसि॒ वि प्र॑थ॒स्वाग्ने॒ यष्ट॑रि॒दं नमः॑। जुह्वेह्य॒ग्निस्त्वा᳚ ह्वयति देवय॒ज्याया॒ उप॑भृ॒देहि॑ दे॒वस्त्वा॑ सवि॒ता ह्व॑यति देवय॒ज्याया॒ अग्ना॑विष्णू॒ मा वा॒मव॑ क्रमिषं॒ वि जि॑हाथां॒ मा मा॒ सन्ता᳚प्तं लो॒कं मे॑ लोककृतौ कृणुतं॒ विष्णोः॒ स्थान॑मसी॒त इन्द्रो॑ अकृणोद्वी॒र्या॑णि समा॒रभ्यो॒र्ध्वो अ॑ध्व॒रो दि॑वि॒स्पृश॒मह्रु॑तो य॒ज्ञो य॒ज्ञप॑ते॒\-रिन्द्रा॑\-वा॒न्थ्स्वाहा॑ बृ॒हद्भाः पा॒हि मा᳚ऽग्ने॒ दुश्च॑रिता॒दा मा॒ सुच॑रिते भज म॒खस्य॒ शिरो॑\-ऽसि॒ सं ज्योति॑षा॒ ज्योति॑रङ्क्ताम्॥~(२१)

%1.1.13.0
{\anuvakamend[{अह्रु॑त॒ एक॑विꣳशतिश्च}]}

%1.1.13.1
वाज॑स्य मा प्रस॒वेनो᳚द्ग्रा॒भेणोद॑ग्रभीत्। अथा॑ स॒पत्ना॒ꣳ॒ इन्द्रो॑ मे निग्रा॒भेणाध॑राꣳ अकः। उ॒द्ग्रा॒भं च॑ निग्रा॒भं च॒ ब्रह्म॑ दे॒वा अ॑वीवृधन्न्। अथा॑ स॒पत्ना॑निन्द्रा॒ग्नी मे॑ विषू॒चीना॒न्व्य॑स्यताम्। वसु॑भ्यस्त्वा रु॒द्रेभ्य॑स्त्वा\-ऽ\-ऽदि॒त्येभ्य॑स्त्वा॒ऽक्तꣳ रिहा॑णा वि॒यन्तु॒ वयः॑। प्र॒जां योनिं॒ मा निर्मृ॑क्ष॒मा प्या॑यन्ता॒माप॒ ओष॑धयो म॒रुतां॒ पृष॑तयः स्थ॒ दिवं॑~(२२)

%1.1.13.2
गच्छ॒ ततो॑ नो॒ वृष्टि॒मेर॑य। आ॒यु॒ष्पा अ॑ग्ने॒\-ऽस्यायु॑र्मे पाहि चक्षु॒ष्पा अ॑ग्ने\-ऽसि॒ चक्षु॑र्मे पाहि ध्रु॒वा\-ऽसि॒ यं प॑रि॒धिं प॒र्यध॑त्था॒ अग्ने॑ देव प॒णिभि॑र्वी॒यमा॑णः। तन्त॑ ए॒तमनु॒ जोषं॑ भरामि॒ नेदे॒ष त्वद॑पचे॒तया॑तै य॒ज्ञस्य॒ पाथ॒ उप॒ समि॑तꣳ सꣴस्रा॒वभा॑गाः स्थे॒षा बृ॒हन्त॑ प्रस्तरे॒ष्ठा ब॑र्\mbox{}हि॒षद॑श्च~(२३)

%1.1.13.3
दे॒वा इ॒मां वाच॑म॒भि विश्वे॑ गृ॒णन्त॑ आ॒सद्या॒स्मिन्ब॒र्॒हिषि॑ मादयध्वम॒ग्नेर्वा॒मप॑न्नगृहस्य॒ सद॑सि सादयामि सु॒म्नाय॑ सुम्निनी सु॒म्ने मा॑ धत्तं धु॒रि धु॒र्यौ॑ पात॒मग्ने॑\-ऽदब्धायो\-ऽशीततनो पा॒हि मा॒\-ऽद्य दि॒वः पा॒हि प्रसि॑त्यै पा॒हि दुरि॑ष्ट्यै पा॒हि दु॑रद्म॒न्यै पा॒हि दुश्च॑रिता॒दवि॑षन्नः पि॒तुं कृ॑णु सु॒षदा॒ योनि॒ꣴ॒ स्वाहा॒ देवा॑ गातुविदो गा॒तुं वि॒त्वा गा॒तुमि॑त॒ मन॑सस्पत इ॒मं नो॑ देव दे॒वेषु॑ य॒ज्ञꣴ स्वाहा॑ वा॒चि स्वाहा॒ वाते॑ धाः॥~(२४)

%1.1.14.0
{\anuvakamend[{दिव॑ञ्च वि॒त्वा गा॒तुन्त्रयो॑दश च}]}

%1.1.14.1
उ॒भा वा॑मिन्द्राग्नी आहु॒वध्या॑ उ॒भा राध॑सः स॒ह मा॑द॒यध्यै᳚। उ॒भा दा॒तारा॑वि॒षाꣳ र॑यी॒णामु॒भा वाज॑स्य सा॒तये॑ हुवे वाम्। अश्र॑व॒ꣳ॒ हि भू॑रि॒दाव॑त्तरा वां॒ वि जा॑मातुरु॒त वा॑ घा स्या॒लात्। अथा॒ सोम॑स्य॒ प्रय॑ती यु॒वभ्या॒मिन्द्रा᳚ग्नी॒ स्तोमं॑ जनयामि॒ नव्यम्᳚। इन्द्रा᳚ग्नी नव॒तिं पुरो॑ दा॒सप॑त्नीरधूनुतम्। सा॒कमेके॑न॒ कर्म॑णा। शुचिं॒ नु स्तोमं॒ नव॑जातम॒द्येन्द्रा᳚ग्नी वृत्रहणा जु॒षेथा᳚म्॥~(२५)

%1.1.14.2
उ॒भा हि वाꣳ॑ सु॒हवा॒ जोह॑वीमि॒ ता वाजꣳ॑ स॒द्य उ॑श॒ते धेष्ठा᳚। व॒यमु॑ त्वा पथस्पते॒ रथं॒ न वाज॑सातये। धि॒ये पू॑षन्नयुज्महि। प॒थस्प॑थः॒ परि॑पतिं वच॒स्या कामे॑न कृ॒तो अ॒भ्या॑नड॒र्कम्। स नो॑ रासच्छु॒रुध॑श्च॒न्द्राग्रा॒ धियं॑ धियꣳ सीषधाति॒ प्र पू॒षा। क्षेत्र॑स्य॒ पति॑ना व॒यꣳ हि॒तेने॑व जयामसि। गामश्वं॑ पोषयि॒त्न्वा स नो॑~(२६)

%1.1.14.3
मृडाती॒दृशे᳚। क्षेत्र॑स्य पते॒ मधु॑मन्तमू॒र्मिं धे॒नुरि॑व॒ पयो॑ अ॒स्मासु॑ धुक्ष्व। म॒धु॒श्चुतं॑ घृ॒तमि॑व॒ सुपू॑तमृ॒तस्य॑ न॒ पत॑यो मृडयन्तु। अग्ने॒ नय॑ सु॒पथा॑ रा॒ये अ॒स्मान् विश्वा॑नि देव व॒युना॑नि वि॒द्वान्। यु॒यो॒ध्य॑स्मज्जु॑हुरा॒णमेनो॒ भूयि॑ष्ठान्ते॒ नम॑ उक्तिं विधेम। आ दे॒वाना॒मपि॒ पन्था॑मगन्म॒ यच्छ॒क्नवा॑म॒ तदनु॒ प्रवो॑ढुम्। अ॒ग्निर्वि॒द्वान्थ्स य॑जा॒थ्~(२७)

%1.1.14.4
सेदु॒ होता॒ सो अ॑ध्व॒रान्थ्स ऋ॒तून्क॑ल्पयाति। यद्वा\-हि॑ष्ठं॒ तद॒ग्नये॑ बृ॒हद॑र्च विभावसो। महि॑षीव॒ त्वद्र॒यिस्त्वद्वाजा॒ उदी॑रते। अग्ने॒ त्वं पा॑रया॒ नव्यो॑ अ॒स्मान्थ्स्व॒स्तिभि॒रति॑ दु॒र्गाणि॒ विश्वा᳚। पूश्च॑ पृ॒थ्वी ब॑हु॒ला न॑ उ॒र्वी भवा॑ तो॒काय॒ तन॑याय॒ शं योः। त्वम॑ग्ने व्रत॒पा अ॑सि दे॒व आ मर्त्ये॒ष्वा। त्वं य॒ज्ञेष्वीड्यः॑। यद्वो॑ व॒यं प्र॑मि॒नाम॑ व्र॒तानि॑ वि॒दुषां᳚ देवा॒ अवि॑दुष्टरासः। अ॒ग्निष्टद्विश्व॒मा पृ॑णाति वि॒द्वान् येभि॑र्दे॒वाꣳ ऋ॒तुभिः॑ क॒ल्पया॑ति॥~(२८)

{\anuvakamend[{जु॒षेथा॒मा स नो॑ यजा॒दा त्रयो॑विꣳशतिश्च}]}
%1.1.1.0

{\prashnaend[{इ॒षे त्वा॑ य॒ज्ञस्य॒ शुन्ध॑ध्वं॒ कर्म॑णे दे॒वो\-ऽव॑धूत॒न्धृष्टिः॒ सं व॑पा॒म्या द॑दे॒
प्रत्यु॑ष्टं॒ कृष्णो॑\-ऽसि॒ भुव॑नमसि॒ वाज॑स्यो॒भा वां॒ चतु॑र्दश॥14॥ इ॒षे दृꣳ॑ह॒ भुव॑नम॒ष्टाविꣳ॑शतिः॥28॥ इ॒षे त्वा॑ क॒ल्पया॑ति॥}]}

%%% END PRASHNA

\sect{द्वितीयः प्रश्नः}\setcounter{anuvakam}{0}
\dnsub{तैत्तिरीयसंहितायां प्रथमकाण्डे द्वितीयः प्रश्नः}
%1.2.1.0
%1.2.1.1
आप॑ उन्दन्तु जी॒वसे॑ दीर्घायु॒त्वाय॒ वर्च॑स॒ ओष॑धे॒ त्राय॑स्वैन॒ꣴ॒ स्वधि॑ते॒ मैनꣳ॑ हिꣳसीर्देव॒श्रूरे॒तानि॒ प्र व॑पे स्व॒स्त्युत्त॑राण्यशी॒या\-ऽ\-ऽपो॑ अ॒स्मान्मा॒तरः॑ शुन्धन्तु घृ॒तेन॑ नो घृत॒पुवः॑ पुनन्तु॒ विश्व॑म॒स्मत्प्र व॑हन्तु रि॒प्रमुदा᳚भ्यः॒ शुचि॒रा पू॒त ए॑मि॒ सोम॑स्य त॒नूर॑सि त॒नुवं॑ मे पाहि मही॒नां पयो॑\-ऽसि वर्चो॒धा अ॑सि॒ वर्चो॒~(१)

%1.2.1.2
मयि॑ धेहि वृ॒त्रस्य॑ क॒नीनि॑का\-ऽसि चक्षु॒ष्पा अ॑सि॒ चक्षु॑र्मे पाहि चि॒त्पति॑स्त्वा पुनातु वा॒क्पति॑स्त्वा पुनातु दे॒वस्त्वा॑ सवि॒ता पु॑ना॒त्वच्छि॑द्रेण प॒वित्रे॑ण॒ वसोः॒ सूर्य॑स्य र॒श्मिभि॒स्तस्य॑ ते पवित्रपते प॒वित्रे॑ण॒ यस्मै॒ कं पु॒ने तच्छ॑केय॒मा वो॑ देवास ईमहे॒ सत्य॑धर्माणो अध्व॒रे यद्वो॑ देवास आगु॒रे यज्ञि॑यासो॒ हवा॑मह॒ इन्द्रा᳚ग्नी॒ द्यावा॑पृथिवी॒ आप॑ ओषधी॒स्त्वं दी॒क्षाणा॒मधि॑पतिरसी॒ह मा॒ सन्तं॑ पाहि॥~(२)

%1.2.2.0
{\anuvakamend[{वर्च॑ ओषधीर॒ष्टौ च॑}]}%~(१)

%1.2.2.1
आकू᳚त्यै प्र॒युजे॒\-ऽग्नये॒ स्वाहा॑ मे॒धायै॒ मन॑से॒\-ऽग्नये॒ स्वाहा॑ दी॒क्षायै॒ तप॑से॒\-ऽग्नये॒ स्वाहा॒ सर॑स्वत्यै पू॒ष्णे᳚\-ऽग्नये॒ स्वाहा\-ऽ\-ऽपो॑ देवीर्बृहतीर्विश्वशम्भुवो॒ द्यावा॑पृथि॒वी उ॒र्व॑न्तरि॑क्षं॒ बृह॒स्पति॑र्नो ह॒विषा॑ वृधातु॒ स्वाहा॒ विश्वे॑ दे॒वस्य॑ ने॒तुर्मर्तो॑\-ऽवृणीत स॒ख्यं विश्वे॑ रा॒य इ॑षुध्यसि द्यु॒म्नं वृ॑णीत पु॒ष्यसे॒ स्वाह॑र्ख्सा॒मयोः॒ शिल्पे᳚ स्थ॒स्ते वा॒मा र॑भे॒ ते मा॑~(३)

%1.2.2.2
पात॒मा\-ऽस्य य॒ज्ञस्यो॒दृच॑ इ॒मां धिय॒ꣳ॒ शिक्ष॑माणस्य देव॒ क्रतुं॒ दक्षं॑ वरुण॒ सꣳशि॑शाधि॒ यया\-ऽति॒ विश्वा॑ दुरि॒ता तरे॑म सु॒तर्मा॑ण॒मधि॒ नावꣳ॑ रुहे॒मोर्ग॑स्याङ्गिर॒स्यूर्ण॑म्रदा॒ ऊर्जं॑ मे यच्छ पा॒हि मा॒ मा मा॑ हिꣳसी॒र्विष्णोः॒ शर्मा॑सि॒ शर्म॒ यज॑मानस्य॒ शर्म॑ मे यच्छ॒ नक्ष॑त्राणां मा\-ऽतीका॒शात् पा॒हीन्द्र॑स्य॒ योनि॑रसि॒~(४)

%1.2.2.3
मा मा॑ हिꣳसीः कृ॒ष्यै त्वा॑ सुस॒स्यायै॑ सुपिप्प॒लाभ्य॒स्त्वौष॑\-धीभ्यः सूप॒स्था दे॒वो वन॒स्पति॑रू॒र्ध्वो मा॑ पा॒ह्योदृचः॒ स्वाहा॑ य॒ज्ञं मन॑सा॒ स्वाहा॒ द्यावा॑पृथि॒वीभ्या॒ꣴ॒ स्वाहो॒रोर॒न्तरि॑क्षा॒थ्\-स्वाहा॑ य॒ज्ञं वाता॒दा र॑भे॥~(५)

%1.2.3.0
{\anuvakamend[{मा॒ योनि॑रसि त्रि॒ꣳ॒शच्च॑}]}%~(२)

%1.2.3.1
दैवीं॒ धियं॑ मनामहे सुमृडी॒काम॒भिष्ट॑ये वर्चो॒धां य॒ज्ञवा॑हसꣳ सुपा॒रा नो॑ अस॒द्वशे᳚। ये दे॒वा मनो॑जाता मनो॒युजः॑ सु॒दक्षा॒ दक्ष॑पितार॒स्ते नः॑ पान्तु॒ ते नो॑\-ऽवन्तु॒ तेभ्यो॒ नम॒स्तेभ्यः॒ स्वाहा\-ऽग्ने॒ त्वꣳ सु जा॑गृहि व॒यꣳ सु म॑न्दिषीमहि गोपा॒य नः॑ स्व॒स्तये᳚ प्र॒बुधे॑ नः॒ पुन॑र्ददः। त्वम॑ग्ने व्रत॒पा अ॑सि दे॒व आ मर्त्ये॒ष्वा। त्वं~(६)

%1.2.3.2
य॒ज्ञेष्वीड्यः॑॥ विश्वे॑ दे॒वा अ॒भि मा मा\-ऽव॑वृत्रन् पू॒षा स॒न्या सोमो॒ राध॑सा दे॒वः स॑वि॒ता वसो᳚र्वसु॒दावा॒ रास्वेय॑थ्सो॒मा\-ऽ\-ऽभूयो॑ भर॒ मा पृ॒णन्पू॒र्त्या वि रा॑धि॒ मा\-ऽहमायु॑षा च॒न्द्रम॑सि॒ मम॒ भोगा॑य भव॒ वस्त्र॑मसि॒ मम॒ भोगा॑य भवो॒स्रा\-ऽसि॒ मम॒ भोगा॑य भव॒ हयो॑\-ऽसि॒ मम॒ भोगा॑य भव॒~(७)

%1.2.3.3
छागो॑\-ऽसि॒ मम॒ भोगा॑य भव मे॒षो॑\-ऽसि॒ मम॒ भोगा॑य भव वा॒यवे᳚ त्वा॒ वरु॑णाय त्वा॒ निर्\mbox{}ऋ॑त्यै त्वा रु॒द्राय॑ त्वा॒ देवी॑रापो अपां नपा॒द्य ऊ॒र्मिर्\mbox{}ह॑वि॒ष्य॑ इन्द्रि॒यावा᳚न्म॒दिन्त॑म॒स्तं वो॒ मा\-ऽव॑क्रमिष॒मच्छि॑न्नं॒ तन्तुं॑ पृथि॒व्या अनु॑ गेषं भ॒द्राद॒भि श्रेयः॒ प्रेहि॒ बृह॒स्पतिः॑ पुरए॒ता ते॑ अ॒स्त्वथे॒मव॑ स्य॒ वर॒ आ पृ॑थि॒व्या आ॒रे शत्रू᳚न् कृणुहि॒ सर्व॑वीर॒ एदम॑गन्म देव॒यज॑नं पृथि॒व्या विश्वे॑ दे॒वा यदजु॑षन्त॒ पूर्व॑ ऋख्सा॒माभ्यां॒ यजु॑षा स॒न्तर॑न्तो रा॒यस्पोषे॑ण॒ समि॒षा म॑देम॥~(८)

%1.2.4.0
{\anuvakamend[{आ त्वꣳ हयो॑\-ऽसि॒ मम॒ भोगा॑य भव स्य॒ पञ्च॑विꣳशतिश्च}]}%~(३)

%1.2.4.1
इ॒यं ते॑ शुक्र त॒नूरि॒दं वर्च॒स्तया॒ सं भ॑व॒ भ्राजं॑ गच्छ॒ जूर॑सि धृ॒ता मन॑सा॒ जुष्टा॒ विष्ण॑वे॒ तस्या᳚स्ते स॒त्यस॑वसः प्रस॒वे वा॒चो य॒न्त्रम॑शीय॒ स्वाहा॑ शु॒क्रम॑स्य॒मृत॑मसि वैश्वदे॒वꣳ ह॒विः सूर्य॑स्य॒ चक्षु॒रा\-ऽरु॑हम॒ग्नेर॒क्ष्णः क॒नीनि॑कां॒ यदेत॑शेभि॒रीय॑से॒ भ्राज॑मानो विप॒श्चिता॒ चिद॑सि म॒ना\-ऽसि॒ धीर॑सि॒ दक्षि॑णा-~(९)

%1.2.4.2
ऽसि य॒ज्ञिया॑\-ऽसि क्ष॒त्रिया॒\-ऽस्यदि॑तिरस्युभ॒यतः॑शीर्ष्णी॒ सा नः॒ सुप्रा॑ची॒ सुप्र॑तीची॒ सं भ॑व मि॒त्रस्त्वा॑ प॒दि ब॑ध्नातु पू॒षा\-ऽध्व॑नः पा॒त्विन्द्रा॒याध्य॑क्षा॒यानु॑ त्वा मा॒ता म॑न्यता॒मनु॑ पि॒ता\-ऽनु॒ भ्राता॒ सग॒र्भ्यो\-ऽनु॒ सखा॒ सयू᳚थ्यः॒ सा दे॑वि दे॒वमच्छे॒हीन्द्रा॑य॒ सोमꣳ॑ रु॒द्रस्त्वा\-ऽ\-ऽव॑र्तयतु मि॒त्रस्य॑ प॒था स्व॒स्ति सोम॑सखा॒ पुन॒रेहि॑ स॒ह र॒य्या॥~(१०)

%1.2.5.0
{\anuvakamend[{दक्षि॑णा॒ सोम॑सखा॒ पञ्च॑ च}]}%~(४)

%1.2.5.1
वस्व्य॑सि रु॒द्रा\-ऽस्यदि॑तिरस्यादि॒त्या\-ऽसि॑ शु॒क्रा\-ऽसि॑ च॒न्द्रा\-ऽसि॒ बृह॒स्पति॑स्त्वा सु॒म्ने र॑ण्वतु रु॒द्रो वसु॑भि॒रा चि॑केतु पृथि॒व्यास्त्वा॑ मू॒र्धन्ना जि॑घर्मि देव॒यज॑न॒ इडा॑याः प॒दे घृ॒तव॑ति॒ स्वाहा॒ परि॑लिखित॒ꣳ॒ रक्षः॒ परि॑लिखिता॒ अरा॑तय इ॒दम॒हꣳ रक्ष॑सो ग्री॒वा अपि॑ कृन्तामि॒ यो᳚\-ऽस्मान् द्वेष्टि॒ यं च॑ व॒यं द्वि॒ष्म इ॒दम॑स्य ग्री॒वा~(११)

%1.2.5.2
अपि॑ कृन्ताम्य॒स्मे राय॒स्त्वे राय॒स्तोते॒ रायः॒ सं दे॑वि दे॒व्योर्वश्या॑ पश्यस्व॒ त्वष्टी॑मती ते सपेय सु॒रेता॒ रेतो॒ दधा॑ना वी॒रं वि॑देय॒ तव॑ स॒न्दृशि॒ मा\-ऽहꣳ रा॒यस्पोषे॑ण॒ वि यो॑षम्॥~(१२)

%1.2.6.0
{\anuvakamend[{अ॒स्य॒ ग्री॒वा एका॒न्नत्रि॒ꣳ॒शच्च॑}]}%~(५)

%1.2.6.1
अ॒ꣳ॒शुना॑ ते अ॒ꣳ॒शुः पृ॑च्यतां॒ परु॑षा॒ परु॑र्ग॒न्धस्ते॒ काम॑मवतु॒ मदा॑य॒ रसो॒ अच्यु॑तो॒\-ऽमात्यो॑\-ऽसि शु॒क्रस्ते॒ ग्रहो॒\-ऽभि त्यं दे॒वꣳ स॑वि॒तार॑मू॒ण्योः᳚ क॒विक्र॑तु॒मर्चा॑मि स॒त्यस॑वसꣳ रत्न॒धाम॒भि प्रि॒यं म॒तिमू॒र्ध्वा यस्या॒मति॒र्भा अदि॑द्युत॒थ्सवी॑मनि॒ हिर॑ण्यपाणिरमिमीत सु॒क्रतुः॑ कृ॒पा सुवः॑। प्र॒जाभ्य॑स्त्वा प्रा॒णाय॑ त्वा व्या॒नाय॑ त्वा प्र॒जास्त्वमनु॒ प्राणि॑हि प्र॒जास्त्वामनु॒ प्राण॑न्तु॥~(१३)

%1.2.7.0
{\anuvakamend[{अनु॑ स॒प्त च॑}]}%~(६)

%1.2.7.1
सोमं॑ ते क्रीणा॒म्यूर्ज॑स्वन्तं॒ पय॑स्वन्तं वी॒र्या॑वन्तमभिमाति॒\-षाहꣳ॑ शु॒क्रं ते॑ शु॒क्रेण॑ क्रीणामि च॒न्द्रं च॒न्द्रेणा॒मृत॑म॒मृते॑न स॒म्यत्ते॒ गोर॒स्मे च॒न्द्राणि॒ तप॑सस्त॒नूर॑सि प्र॒जा\-प॑ते॒र्वर्ण॒स्तस्या᳚स्ते सहस्रपो॒षं पुष्य॑न्त्याश्चर॒मेण॑ प॒शुना᳚ क्रीणाम्य॒स्मे ते॒ बन्धु॒र्मयि॑ ते॒ रायः॑ श्रयन्ताम॒स्मे ज्योतिः॑ सोमविक्र॒यिणि॒ तमो॑ मि॒त्रो न॒ एहि॒ सुमि॑त्रधा॒ इन्द्र॑स्यो॒रु मा वि॑श॒ दक्षि॑णमु॒शन्नु॒शन्तꣴ॑ स्यो॒नः स्यो॒नꣴ स्वान॒ भ्राजाङ्घा॑रे॒ बम्भा॑रे॒ हस्त॒ सुह॑स्त॒ कृशा॑नवे॒ते वः॑ सोम॒क्रय॑णा॒स्तान्र॑क्षध्वं॒ मा वो॑ दभन्न्॥~(१४)

%1.2.8.0
{\anuvakamend[{ऊ॒रुं द्वाविꣳ॑शतिश्व}]}%~(७)

%1.2.8.1
उदायु॑षा स्वा॒युषोदोष॑धीना॒ꣳ॒ रसे॒नोत्प॒र्जन्य॑स्य॒ शुष्मे॒णोद॑स्था\-म॒मृता॒ꣳ॒ अनु॑। उ॒र्व॑न्तरि॑क्ष॒मन्वि॒ह्यदि॑त्याः॒ सदो॒\-ऽस्यदि॑त्याः॒ सद॒ आसी॒\-दास्त॑भ्ना॒द्द्यामृ॑ष॒भो अ॒न्तरि॑क्ष॒ममि॑मीत वरि॒माणं॑ पृथि॒व्या आसी॑द॒द्विश्वा॒ भुव॑नानि स॒म्राड्विश्वेत्तानि॒ वरु॑णस्य व्र॒तानि॒ वने॑षु॒ व्य॑न्तरि॑क्षं ततान॒ वाज॒मर्व॑थ्सु॒ पयो॑ अघ्नि॒यासु॑ हृ॒थ्सु~(१५)

%1.2.8.2
क्रतुं॒ वरु॑णो वि॒क्ष्व॑ग्निं दि॒वि सूर्य॑मदधा॒थ्सोम॒मद्रा॒वुदु॒त्यं जा॒तवे॑दसं दे॒वं व॑हन्ति के॒तवः॑। दृ॒शे विश्वा॑य॒ सूर्यम्᳚॥ उस्रा॒वेतं॑ धूर्\mbox{}षाहावन॒श्रू अवी॑रहणौ ब्रह्म॒चोद॑नौ॒ वरु॑णस्य॒ स्कम्भ॑नमसि॒ वरु॑णस्य स्कम्भ॒सर्ज॑नमसि॒ प्रत्य॑स्तो॒ वरु॑णस्य॒ पाशः॑॥~(१६)

%1.2.9.0
{\anuvakamend[{हृ॒थ्सु पञ्च॑त्रिꣳशच्च}]}%~(८)

%1.2.9.1
प्रच्य॑वस्व भुवस्पते॒ विश्वा᳚न्य॒भि धामा॑नि॒ मा त्वा॑ परिप॒री वि॑द॒न्मा त्वा॑ परिप॒न्थिनो॑ विद॒न्मा त्वा॒ वृका॑ अघा॒यवो॒ मा ग॑न्ध॒र्वो वि॒श्वाव॑सु॒रा द॑घच्छ्ये॒नो भू॒त्वा परा॑ पत॒ यज॑मानस्य नो गृ॒हे दे॒वैः सꣴ॑स्कृ॒तं यज॑मानस्य स्व॒स्त्यय॑न्य॒स्यपि॒ पन्था॑मगस्महि स्वस्ति॒गाम॑ने॒हसं॒ येन॒ विश्वाः॒ परि॒ द्विषो॑ वृ॒णक्ति॑ वि॒न्दते॒ वसु॒ नमो॑ मि॒त्रस्य॒ वरु॑णस्य॒ चक्ष॑से म॒हो दे॒वाय॒ तदृ॒तꣳ स॑पर्यत दूरे॒\-दृशे॑ दे॒व\-जा॑ताय के॒तवे॑ दि॒वस्पु॒त्राय॒ सूर्या॑य शꣳसत॒ वरु॑णस्य॒ स्कम्भ॑\-नमसि॒ वरु॑णस्य स्कम्भ॒\-सर्ज॑नम॒स्युन्मु॑क्तो॒ वरु॑णस्य॒ पाशः॑॥~(१७)

%1.2.10.0
{\anuvakamend[{मि॒त्रस्य॒ त्रयो॑विꣳशतिश्च}]}%~(९)

%1.2.10.1
अ॒ग्नेरा॑ति॒थ्यम॑सि॒ विष्ण॑वे त्वा॒ सोम॑स्या\-ऽ\-ऽति॒थ्यम॑सि॒ विष्ण॑वे॒ त्वा\-ऽति॑थेराति॒थ्यम॑सि॒ विष्ण॑वे त्वा॒\-ऽग्नये᳚ त्वा रायस्पोष॒दाव्न्ने॒ विष्ण॑वे त्वा श्ये॒नाय॑ त्वा सोम॒भृते॒ विष्ण॑वे त्वा॒ या ते॒ धामा॑नि ह॒विषा॒ यज॑न्ति॒ ता ते॒ विश्वा॑ परि॒भूर॑स्तु य॒ज्ञं ग॑य॒स्फानः॑ प्र॒तर॑णः सु॒वीरो\-ऽवी॑रहा॒ प्र च॑रा सोम॒ दुर्या॒नदि॑त्याः॒ सदो॒\-ऽस्यदि॑त्याः॒ सद॒ आ~(१८)

%1.2.10.2
सी॑द॒ वरु॑णो\-ऽसि धृ॒तव्र॑तो वारु॒णम॑सि शं॒योर्दे॒वानाꣳ॑ स॒ख्यान्मा दे॒वाना॑म॒पस॑श्छिथ्स्म॒ह्याप॑तये त्वा गृह्णामि॒ परि॑पतये त्वा गृह्णामि॒ तनू॒नप्त्रे᳚ त्वा गृह्णामि शाक्व॒राय॑ त्वा गृह्णामि॒ शक्म॒न्नोजि॑ष्ठाय त्वा गृह्णा॒म्यना॑धृष्टम\-स्यनाधृ॒ष्यं दे॒वाना॒मोजो॑\-ऽभिशस्ति॒पा अ॑नभिशस्ते॒\-ऽन्यमनु॑ मे दी॒क्षां दी॒क्षाप॑तिर्मन्यता॒मनु॒ तप॒स्तप॑स्पति॒रञ्ज॑सा स॒त्यमुप॑ गेषꣳ सुवि॒ते मा॑ धाः॥~(१९)

%1.2.11.0
{\anuvakamend[{आ मैकं॑ च}]}%॥10॥

%1.2.11.1
अ॒ꣳ॒शुरꣳ॑शुस्ते देव सो॒मा\-ऽ\-ऽप्या॑यता॒मिन्द्रा॑यैकधन॒विद॒ आ तुभ्य॒मिन्द्रः॑ प्यायता॒मा त्वमिन्द्रा॑य प्याय॒स्वा\-ऽ\-ऽ प्या॑यय॒ सखी᳚न्थ्स॒न्या मे॒धया᳚ स्व॒स्ति ते॑ देव सोम सु॒त्याम॑शी॒येष्टा॒ रायः॒ प्रेषे भगा॑य॒र्तमृ॑तवा॒दिभ्यो॒ नमो॑ दि॒वे नमः॑ पृथि॒व्या अग्ने᳚ व्रतपते॒ त्वं व्र॒तानां᳚ व्र॒तप॑तिरसि॒ या मम॑ त॒नूरे॒षा सा त्वयि॒~(२०)

%1.2.11.2
या तव॑ त॒नूरि॒यꣳ सा मयि॑ स॒ह नौ᳚ व्रतपते व्र॒तिनो᳚र्व्र॒तानि॒ या ते॑ अग्ने॒ रुद्रि॑या त॒नूस्तया॑ नः पाहि॒ तस्या᳚स्ते॒ स्वाहा॒ या ते॑ अग्ने\-ऽयाश॒या र॑जाश॒या ह॑राश॒या त॒नूर्वर्\mbox{}षि॑ष्ठा गह्वरे॒ष्ठोग्रं वचो॒ अपा॑वधीं त्वे॒षं वचो॒ अपा॑वधी॒ꣴ॒ स्वाहा᳚॥~(२१)

%1.2.12.0
{\anuvakamend[{त्वयि॑ चत्वारि॒ꣳ॒शच्च॑}]}%॥11॥

%1.2.12.1
वि॒त्ताय॑नी मे\-ऽसि ति॒क्ताय॑नी मे॒\-ऽस्यव॑तान्मा नाथि॒तमव॑तान्मा व्यथि॒तं वि॒देर॒ग्निर्नभो॒ नामाग्ने॑ अङ्गिरो॒ यो᳚\-ऽस्यां पृ॑थि॒व्यामस्याऽऽयु॑षा॒ नाम्नेहि॒ यत्ते\-ऽना॑धृष्टं॒ नाम॑ य॒ज्ञियं॒ तेन॒ त्वा\-ऽ\-ऽद॒धे\-ऽग्ने॑ अङ्गिरो॒ यो द्वि॒तीय॑स्यां तृ॒तीय॑स्यां पृथि॒व्यामस्याऽऽयु॑षा॒ नाम्नेहि॒ यत्ते\-ऽना॑धृष्टं॒ नाम॑~(२२)

%1.2.12.2
य॒ज्ञियं॒ तेन॒ त्वा\-ऽ\-ऽद॑धे सि॒ꣳ॒हीर॑सि महि॒षीर॑स्यु॒रु प्र॑थस्वो॒रु ते॑ य॒ज्ञप॑तिः प्रथतां ध्रु॒वा\-ऽसि॑ दे॒वेभ्यः॑ शुन्धस्व दे॒वेभ्यः॑ शुम्भस्वेन्द्रघो॒षस्त्वा॒ वसु॑भिः पु॒रस्ता᳚त्पातु॒ मनो॑जवास्त्वा पि॒तृभि॑र्दक्षिण॒तः पा॑तु॒ प्रचे॑तास्त्वा रु॒द्रैः प॒श्चात्पा॑तु वि॒श्वक॑र्मा त्वा\-ऽ\-ऽदि॒त्यैरु॑त्तर॒तः पा॑तु सि॒ꣳ॒हीर॑सि सपत्नसा॒ही स्वाहा॑ सि॒ꣳ॒हीर॑सि सुप्रजा॒वनिः॒ स्वाहा॑ सि॒ꣳ॒ही-~(२३)

%1.2.12.3
-र॑सि रायस्पोष॒वनिः॒ स्वाहा॑ सि॒ꣳ॒हीर॑स्यादित्य॒वनिः॒ स्वाहा॑ सि॒ꣳ॒हीर॒स्या व॑ह दे॒वान्दे॑वय॒ते यज॑मानाय॒ स्वाहा॑ भू॒तेभ्य॑स्त्वा वि॒श्वायु॑रसि पृथि॒वीं दृꣳ॑ह ध्रुव॒क्षिद॑स्य॒न्तरि॑क्षं दृꣳहाच्युत॒क्षिद॑सि॒ दिवं॑ दृꣳहा॒ग्नेर्भस्मा᳚स्य॒ग्नेः पुरी॑षमसि॥~(२४)

%1.2.13.0
{\anuvakamend[{नाम॑ सुप्रजा॒वनिः॒ स्वाहा॑ सि॒ꣳ॒हीः पञ्च॑त्रिꣳशच्च}]}%॥12॥

%1.2.13.1
यु॒ञ्जते॒ मन॑ उ॒त यु॑ञ्जते॒ धियो॒ विप्रा॒ विप्र॑स्य बृह॒तो वि॑प॒श्चितः॑। वि होत्रा॑ दधे वयुना॒विदेक॒ इन्म॒ही दे॒वस्य॑ सवि॒तुः परि॑ष्टुतिः॥ सु॒वाग्दे॑व॒ दुर्या॒ꣳ॒ आ व॑द देव॒श्रुतौ॑ दे॒वेष्वा घो॑षेथा॒मा नो॑ वी॒रो जा॑यतां कर्म॒ण्यो॑ यꣳ सर्वे॑\-ऽनु॒जीवा॑म॒ यो ब॑हू॒नामस॑द्व॒शी। इ॒दं विष्णु॒र्वि च॑क्रमे त्रे॒धा नि द॑धे प॒दम्। समू॑ढमस्य~(२५)

%1.2.13.2
पाꣳसु॒र इरा॑वती धेनु॒मती॒ हि भू॒तꣳ सू॑यव॒सिनी॒ मन॑वे यश॒स्ये᳚। व्य॑स्कभ्ना॒द्रोद॑सी॒ विष्णु॑रे॒ते दा॒धार॑ पृथि॒वीम॒भितो॑ म॒यूखैः᳚॥ प्राची॒ प्रेत॑मध्व॒रं क॒ल्पय॑न्ती ऊ॒र्ध्वं य॒ज्ञं न॑यतं॒ मा जी᳚ह्वरत॒मत्र॑ रमेथां॒ वर्ष्म॑न्पृथि॒व्या दि॒वो वा॑ विष्णवु॒त वा॑ पृथि॒व्या म॒हो वा॑ विष्णवु॒त वा॒\-ऽन्तरि॑क्षा॒द्धस्तौ॑ पृणस्व ब॒हुभि॑र्वस॒व्यै॑रा प्र य॑च्छ॒~(२६)

%1.2.13.3
दक्षि॑णा॒दोत स॒व्यात्। विष्णो॒र्नुकं॑ वी॒र्या॑णि॒ प्र वो॑चं॒ यः पार्थि॑वानि विम॒मे रजाꣳ॑सि॒ यो अस्क॑भाय॒दुत्त॑रꣳ स॒धस्थं॑ विचक्रमा॒णस्त्रे॒धोरु॑गा॒यो विष्णो॑ र॒राट॑मसि॒ विष्णोः᳚ पृ॒ष्ठम॑सि॒ विष्णोः॒ श्ञप्त्रे᳚ स्थो॒ विष्णोः॒ स्यूर॑सि॒ विष्णो᳚र्ध्रु॒वम॑सि वैष्ण॒वम॑सि॒ विष्ण॑वे त्वा॥~(२७)

%1.2.14.0
{\anuvakamend[{अ॒स्य॒ य॒च्छैका॒न्नच॑त्वारि॒ꣳ॒शच्च॑}]}%॥13॥

%1.2.14.1
कृ॒णु॒ष्व पाजः॒ प्रसि॑तिं॒ न पृ॒थ्वीं या॒हि राजे॒वाम॑वा॒ꣳ॒ इभे॑न। तृ॒ष्वीमनु॒ प्रसि॑तिं द्रूणा॒नो\-ऽस्ता॑सि॒ विध्य॑ र॒क्षस॒स्तपि॑ष्ठैः॥ तव॑ भ्र॒मास॑ आशु॒या प॑त॒न्त्यनु॑ स्पृश धृष॒ता शोशु॑चानः। तपूꣴ॑ष्यग्ने जु॒ह्वा॑ पत॒ङ्गानस॑न्दितो॒ वि सृ॑ज॒ विष्व॑गु॒ल्काः॥ प्रति॒ स्पशो॒ वि सृ॑ज॒ तूर्णि॑तमो॒ भवा॑ पा॒युर्वि॒शो अ॒स्या अद॑ब्धः। यो नो॑ दू॒रे अ॒घशꣳ॑सो॒~(२८)

%1.2.14.2
यो अन्त्यग्ने॒ माकि॑ष्टे॒ व्यथि॒रा द॑धर्षीत्। उद॑ग्ने तिष्ठ॒ प्रत्या\-ऽ\-ऽत॑नुष्व॒ न्य॑मित्राꣳ॑ ओषतात्तिग्महेते। यो नो॒ अरा॑तिꣳ समिधान च॒क्रे नी॒चा तं ध॑क्ष्यत॒सं न शुष्कम्᳚॥ ऊ॒र्ध्वो भ॑व॒ प्रति॑ वि॒ध्याध्य॒स्मदा॒विष्कृ॑णुष्व॒ दैव्या᳚न्यग्ने। अव॑ स्थि॒रा त॑नुहि यातु॒जूनां᳚ जा॒मिमजा॑मिं॒ प्र मृ॑णीहि॒ शत्रून्॑॥ स ते॑~(२९)

%1.2.14.3
जानाति सुम॒तिं य॑विष्ठ॒ य ईव॑ते॒ ब्रह्म॑णे गा॒तुमैर॑त्। विश्वा᳚न्यस्मै सु॒दिना॑नि रा॒यो द्यु॒म्नान्य॒र्यो वि दुरो॑ अ॒भि द्यौ᳚त्॥ सेद॑ग्ने अस्तु सु॒भगः॑ सु॒दानु॒र्यस्त्वा॒ नित्ये॑न ह॒विषा॒ य उ॒क्थैः। पिप्री॑षति॒ स्व आयु॑षि दुरो॒णे विश्वेद॑स्मै सु॒दिना॒ सा\-ऽस॑दि॒ष्टिः॥ अर्चा॑मि ते सुम॒तिं घोष्य॒र्वाख्सं ते॑ वा॒वाता॑ जरता-~(३०)

%1.2.14.4
मि॒यङ्गीः। स्वश्वा᳚स्त्वा सु॒रथा॑ मर्जयेमा॒स्मे क्ष॒त्राणि॑ धारये॒रनु॒ द्यून्॥ इ॒ह त्वा॒ भूर्या च॑रे॒दुप॒ त्मन्दोषा॑वस्तर्दीदि॒वाꣳ\-स॒मनु॒ द्यून्। कीड॑न्तस्त्वा सु॒मन॑सः सपेमा॒भि द्यु॒म्ना त॑स्थि॒वाꣳसो॒ जना॑नाम्॥ यस्त्वा॒ स्वश्वः॑ सुहिर॒ण्यो अ॑ग्न उप॒याति॒ वसु॑मता॒ रथे॑न। तस्य॑ त्रा॒ता भ॑वसि॒ तस्य॒ सखा॒ यस्त॑ आति॒थ्यमा॑नु॒षग्जुजो॑षत्॥ म॒हो रु॑जामि~(३१)

%1.2.14.5
ब॒न्धुता॒ वचो॑भि॒स्तन्मा॑ पि॒तुर्गोत॑मा॒दन्वि॑याय॥ त्वं नो॑ अ॒स्य वच॑सश्चिकिद्धि॒ होत॑र्यविष्ठ सुक्रतो॒ दमू॑नाः॥ अस्व॑प्नजस्त॒रण॑यः सु॒शेवा॒ अत॑न्द्रासो\-ऽवृ॒का अश्र॑मिष्ठाः। ते पा॒यवः॑ स॒ध्रिय॑ञ्चो नि॒षद्या\-ऽग्ने॒ तव॑ नः पान्त्वमूर॥ ये पा॒यवो॑ मामते॒यं ते॑ अग्ने॒ पश्य॑न्तो अ॒न्धं दु॑रि॒तादर॑क्षन्। र॒रक्ष॒ तान्थ्सु॒कृतो॑ वि॒श्ववे॑दा॒ दिफ्स॑न्त॒ इद्रि॒पवो॒ ना ह॑~(३२)

%1.2.14.6
देभुः॥ त्वया॑ व॒यꣳ स॑ध॒न्य॑स्त्वोता॒स्तव॒ प्रणी᳚त्यश्याम॒ वाजान्॑। उ॒भा शꣳसा॑ सूदय सत्यताते\-ऽनुष्ठु॒या कृ॑णुह्यह्रयाण॥ अ॒या ते॑ अग्ने स॒मिधा॑ विधेम॒ प्रति॒ स्तोमꣳ॑ श॒स्यमा॑नं गृभाय। दहा॒शसो॑ र॒क्षसः॑ पा॒ह्य॑स्मान्द्रु॒हो नि॒दोऽमि॑त्रमहो अव॒द्यात्॥ र॒क्षो॒हणं॑ वा॒जिन॒मा\-ऽ\-ऽजि॑घर्मि मि॒त्रं प्रथि॑ष्ठ॒मुप॑ यामि॒ शर्म॑। शिशा॑नो अ॒ग्निः क्रतु॑भिः॒ समि॑द्धः॒ स नो॒ दिवा॒~(३३)

%1.2.14.7
स रि॒षः पा॑तु॒ नक्तम्᳚॥ वि ज्योति॑षा बृह॒ता भा᳚त्य॒ग्निरा॒विर्विश्वा॑नि कृणुते महि॒त्वा। प्रादे॑वीर्मा॒याः स॑हते दु॒रेवाः॒ शिशी॑ते॒ शृङ्गे॒ रक्ष॑से वि॒निक्षे᳚॥ उ॒त स्वा॒नासो॑ दि॒विष॑न्त्व॒ग्नेस्ति॒ग्मायु॑धा॒ रक्ष॑से॒ हन्त॒वा उ॑। मदे॑ चिदस्य॒ प्ररु॑जन्ति॒ भामा॒ न व॑रन्ते परि॒बाधो॒ अदे॑वीः॥~(३४)

%1.3.0.0
{\anuvakamend[{अ॒घशꣳ॑सः॒ स ते॑ जरताꣳ रुजामि ह॒ दिवैक॑चत्वारिꣳशच्च}]}

%%% END PRASHNA

\sect{तृतीयः प्रश्नः}\setcounter{anuvakam}{0}
\dnsub{तैत्तिरीयसंहितायां प्रथमकाण्डे तृतीयः प्रश्नः}
%1.3.1.0
%1.3.1.1
दे॒वस्य॑ त्वा सवि॒तुः प्र॑स॒वे᳚\-ऽश्विनो᳚र्बा॒हु\-भ्यां᳚ पू॒ष्णो हस्ता᳚भ्या॒माद॒दे\-ऽभ्रि॑रसि॒ नारि॑रसि॒ परि॑लिखित॒ꣳ॒ रक्षः॒ परि॑लिखिता॒ अरा॑तय इ॒दम॒हꣳ रक्ष॑सो ग्री॒वा अपि॑ कृन्तामि॒ यो᳚\-ऽस्मान् द्वेष्टि॒ यं च॑ व॒यं द्वि॒ष्म इ॒दम॑स्य ग्री॒वा अपि॑ कृन्तामि दि॒वे त्वा॒\-ऽन्तरि॑क्षाय त्वा पृथि॒व्यै त्वा॒ शुन्ध॑तां लो॒कः पि॑तृ॒षद॑नो॒ यवो॑\-ऽसि य॒वया॒स्मद्द्वेषो॑~(१)

%1.3.1.2
य॒वयारा॑तीः पितृ॒णाꣳ सद॑नम॒स्युद्दिवꣴ॑ स्तभा॒ना\-ऽन्तरि॑क्षं पृण पृथि॒वीं दृꣳ॑ह द्युता॒नस्त्वा॑ मारु॒तो मि॑नोतु मि॒त्रावरु॑णयोर्ध्रु॒वेण॒ धर्म॑णा ब्रह्म॒वनिं॑ त्वा क्षत्र॒वनिꣳ॑ सुप्रजा॒वनिꣳ॑ रायस्पोष॒वनिं॒ पर्यू॑हामि॒ ब्रह्म॑ दृꣳह क्ष॒त्रं दृꣳ॑ह प्र॒जां दृꣳ॑ह रा॒यस्पोषं॑ दृꣳह घृ॒तेन॑ द्यावा\-पृथिवी॒ आ पृ॑णेथा॒मिन्द्र॑स्य॒ सदो॑\-ऽसि विश्वज॒नस्य॑ छा॒या परि॑ त्वा गिर्वणो॒ गिर॑ इ॒मा भ॑वन्तु वि॒श्वतो॑ वृ॒द्धायु॒मनु॒ वृद्ध॑यो॒ जुष्टा॑ भवन्तु॒ जुष्ट॑य॒ इन्द्र॑स्य॒ स्यूर॒सीन्द्र॑स्य ध्रु॒वम॑स्यै॒न्द्रम॒सीन्द्रा॑य त्वा॥~(२)

%1.3.2.0
{\anuvakamend[{द्वेष॑ इ॒मा अ॒ष्टाद॑श च}]}%~(१)

%1.3.2.1
र॒क्षो॒हणो॑ वलग॒हनो॑ वैष्ण॒वान्ख॑नामी॒दम॒हं तं व॑ल॒गमुद्व॑पामि॒ यं नः॑ समा॒नो यमस॑मानो निच॒खाने॒दमे॑न॒मध॑रं करोमि॒ यो नः॑ समा॒नो यो\-ऽस॑मानो\-ऽराती॒यति॑ गाय॒त्रेण॒ छन्द॒सा\-ऽव॑बाढो वल॒गः किमत्र॑ भ॒द्रं तन्नौ॑ स॒ह वि॒राड॑सि सपत्न॒हा स॒म्राड॑सि भ्रातृव्य॒हा स्व॒राड॑स्यभिमाति॒हा वि॑श्वा॒राड॑सि॒ विश्वा॑सां ना॒ष्ट्राणाꣳ॑ ह॒न्ता~(३)

%1.3.2.2
र॑क्षो॒हणो॑ वलग॒हनः॒ प्रोक्षा॑मि वैष्ण॒वान् र॑क्षो॒हणो॑ वलग॒हनो\-ऽव॑ नयामि वैष्ण॒वान् यवो॑\-ऽसि य॒वया॒स्मद्द्वेषो॑ य॒वयारा॑ती रक्षो॒हणो॑ वलग॒हनो\-ऽव॑ स्तृणामि वैष्ण॒वान् र॑क्षो॒हणो॑ वलग॒हनो॒\-ऽभि जु॑होमि वैष्ण॒वान् र॑क्षो॒हणौ॑ वलग॒हना॒वुप॑ दधामि वैष्ण॒वी र॑क्षो॒हणौ॑ वलग॒हनौ॒ पर्यू॑हामि वैष्ण॒वी र॑क्षो॒हणौ॑ वलग॒हनौ॒ परि॑ स्तृणामि वैष्ण॒वी र॑क्षो॒हणौ॑ वलग॒हनौ॑ वैष्ण॒वी बृ॒हन्न॑सि बृ॒हद्ग्रा॑वा बृह॒तीमिन्द्रा॑य॒ वाचं॑ वद॥~(४)

%1.3.3.0
{\anuvakamend[{ह॒न्तेन्द्रा॑य॒ द्वे च॑}]}%~(२)

%1.3.3.1
वि॒भूर॑सि प्र॒वाह॑णो॒ वह्नि॑रसि हव्य॒वाह॑नः श्वा॒त्रो॑\-ऽसि॒ प्रचे॑तास्तु॒थो॑\-ऽसि वि॒श्ववे॑दा उ॒शिग॑सि क॒विरङ्घा॑रिरसि॒ बम्भा॑रिरव॒स्युर॑सि॒ दुव॑स्वाञ्छु॒न्ध्यूर॑सि मार्जा॒लीयः॑ स॒म्राड॑सि कृ॒शानुः॑ परि॒षद्यो॑\-ऽसि॒ पव॑मानः प्र॒तक्वा॑\-ऽसि॒ नभ॑स्वा॒नस॑म्मृष्टो\-ऽसि हव्य॒सूद॑ ऋ॒तधा॑मा\-ऽसि॒ सुव॑र्ज्योति॒र्ब्रह्म॑ज्योतिरसि॒ सुव॑र्धामा॒\-ऽजो᳚\-ऽस्येक॑पा॒दहि॑रसि बु॒ध्नियो॒ रौद्रे॒णानी॑केन पा॒हि मा᳚\-ऽग्ने पिपृ॒हि मा॒ मा मा॑ हिꣳसीः॥~(५)

%1.3.4.0
{\anuvakamend[{अनी॑केना॒ष्टौ च॑}]}%~(३)

%1.3.4.1
त्वꣳ सो॑म तनू॒कृद्भ्यो॒ द्वेषो᳚भ्यो॒\-ऽन्यकृ॑तेभ्य उ॒रु य॒न्तासि॒ वरू॑थ॒ꣴ॒ स्वाहा॑ जुषा॒णो अ॒प्तुराज्य॑स्य वेतु॒ स्वाहा॒\-ऽयं नो॑ अ॒ग्निर्वरि॑वः कृणोत्व॒यं मृधः॑ पु॒र ए॑तु प्रभि॒न्दन्न्। अ॒यꣳ शत्रू᳚ञ्जयतु॒ जर्\mbox{}हृ॑षाणो॒\-ऽयं वाजं॑ जयतु॒ वाज॑सातौ॥ उ॒रु वि॑ष्णो॒ वि क्र॑मस्वो॒रु क्षया॑य नः कृधि। घृ॒तं घृ॑तयोने पिब॒ प्रप्र॑ य॒ज्ञप॑तिं तिर॥ सोमो॑ जिगाति गातु॒विद्~(६)

%1.3.4.2
दे॒वाना॑मेति निष्कृ॒तमृ॒तस्य॒ योनि॑मा॒सद॒मदि॑त्याः॒ सदो॒\-ऽस्यदि॑त्याः॒ सद॒ आ सी॑दै॒ष वो॑ देव सवितः॒ सोम॒स्तꣳ र॑क्षध्वं॒ मा वो॑ दभदे॒तत् त्वꣳ सो॑म दे॒वो दे॒वानुपा॑गा इ॒दम॒हं म॑नु॒ष्यो॑ मनु॒ष्या᳚न्थ्स॒ह प्र॒जया॑ स॒ह रा॒यस्पोषे॑ण॒ नमो॑ दे॒वेभ्यः॑ स्व॒धा पि॒तृभ्य॑ इ॒दम॒हं निर्वरु॑णस्य॒ पाशा॒थ्सुव॑र॒भि~(७)

%1.3.4.3
वि ख्ये॑षं वैश्वान॒रं ज्योति॒रग्ने᳚ व्रतपते॒ त्वं व्र॒तानां᳚ व्र॒तप॑तिरसि॒ या मम॑ त॒नूस्त्वय्यभू॑दि॒यꣳ सा मयि॒ या तव॑ त॒नूर्मय्यभू॑दे॒षा सा त्वयि॑ यथाय॒थं नौ᳚ व्रतपते व्र॒तिनो᳚र्व्र॒तानि॑॥~(८)

%1.3.5.0
{\anuvakamend[{गा॒तु॒विद॒भ्येक॑त्रिꣳशच्च}]}%~(४)

%1.3.5.1
अत्य॒न्यानगां॒ नान्यानुपा॑गाम॒र्वाक्त्वा॒ परै॑रविदं प॒रो\-ऽव॑रै॒स्तं त्वा॑ जुषे वैष्ण॒वं दे॑वय॒ज्यायै॑ दे॒वस्त्वा॑ सवि॒ता मध्वा॑\-ऽन॒क्त्वोष॑धे॒ त्राय॑स्वैन॒ꣴ॒ स्वधि॑ते॒ मैनꣳ॑ हिꣳसी॒र्दिव॒मग्रे॑ण॒ मा ले॑खीर॒न्तरि॑क्षं॒ मध्ये॑न॒ मा हिꣳ॑सीः पृथि॒व्या सं भ॑व॒ वन॑स्पते श॒तव॑ल्\mbox{}शो॒ वि रो॑ह स॒हस्र॑वल्\mbox{}शा॒ वि व॒यꣳ रु॑हेम॒ यं त्वा॒\-ऽयꣴ स्वधि॑ति॒स्तेति॑जानः प्रणि॒नाय॑ मह॒ते सौभ॑गा॒या\-ऽच्छि॑न्नो॒ रायः॑ सु॒वीरः॑॥~(९)

%1.3.6.0
{\anuvakamend[{यं दश॑ च}]}%~(५)

%1.3.6.1
पृ॒थि॒व्यै त्वा॒\-ऽन्तरि॑क्षाय त्वा दि॒वे त्वा॒ शुन्ध॑तां लो॒कः पि॑तृ॒षद॑नो॒ यवो॑\-ऽसि य॒वया॒स्मद् द्वेषो॑ य॒वयारा॑तीः पितृ॒णाꣳ सद॑नमसि स्वावे॒शो᳚\-ऽस्यग्रे॒गा ने॑तृ॒णां वन॒स्पति॒रधि॑ त्वा स्थास्यति॒ तस्य॑ वित्ताद्दे॒वस्त्वा॑ सवि॒ता मध्वा॑\-ऽनक्तु सुपिप्प॒लाभ्य॒स्त्वौष॑धीभ्य॒ उद्दिवꣴ॑ स्तभा॒नान्तरि॑क्षं पृण पृथि॒वीमुप॑रेण दृꣳह॒ ते ते॒ धामा᳚न्युश्मसी~(१०)

%1.3.6.2
ग॒मध्ये॒ गावो॒ यत्र॒ भूरि॑शृङ्गा अ॒यासः॑। अत्राह॒ तदु॑रुगा॒यस्य॒ विष्णोः᳚ पर॒मं प॒दमव॑ भाति॒ भूरेः᳚॥ विष्णोः॒ कर्मा॑णि पश्यत॒ यतो᳚ व्र॒तानि॑ पस्प॒शे। इन्द्र॑स्य॒ युज्यः॒ सखा᳚॥ तद्विष्णोः᳚ पर॒मं प॒दꣳ सदा॑ पश्यन्ति सू॒रयः॑। दि॒वीव॒ चक्षु॒रात॑तम्॥ ब्र॒ह्म॒वनिं॑ त्वा क्षत्र॒वनिꣳ॑ सुप्रजा॒वनिꣳ॑ रायस्पोष॒वनिं॒ पर्यू॑हामि॒ ब्रह्म॑ दृꣳह क्ष॒त्रं दृꣳ॑ह प्र॒जां दृꣳ॑ह रा॒यस्पोषं॑ दृꣳह परि॒वीर॑सि॒ परि॑ त्वा॒ दैवी॒र्विशो᳚ व्ययन्तां॒ परी॒मꣳ रा॒यस्पोषो॒ यज॑मानं मनु॒ष्या॑ अ॒न्तरि॑क्षस्य त्वा॒ साना॒वव॑ गूहामि॥~(११)

%1.3.7.0
{\anuvakamend[{उ॒श्म॒सी॒ पोष॒मेका॒न्नविꣳ॑श॒तिश्च॑}]}%~(६)

%1.3.7.1
इ॒षे त्वो॑प॒वीर॒स्युपो॑ दे॒वान्दैवी॒र्विशः॒ प्रागु॒र्वह्नी॑रु॒शिजो॒ बृह॑स्पते धा॒रया॒ वसू॑नि ह॒व्या ते᳚ स्वदन्तां॒ देव॑ त्वष्ट॒र्वसु॑ रण्व॒ रेव॑ती॒ रम॑ध्वम॒ग्नेर्ज॒नित्र॑मसि॒ वृष॑णौ स्थ उ॒र्वश्य॑स्या॒युर॑सि पुरू॒रवा॑ घृ॒तेना॒क्ते वृष॑णं दधाथां गाय॒त्रं छन्दो\-ऽनु॒ प्र जा॑यस्व॒ त्रैष्टु॑भं॒ जाग॑तं॒ छन्दो\-ऽनु॒ प्रजा॑यस्व॒ भव॑तं~(१२)

%1.3.7.2
नः॒ सम॑नसौ॒ समो॑कसावरे॒पसौ᳚। मा य॒ज्ञꣳ हिꣳ॑सिष्टं॒ मा य॒ज्ञप॑तिं जातवेदसौ शि॒वौ भ॑वतम॒द्य नः॑॥ अ॒ग्नाव॒ग्निश्च॑रति॒ प्रवि॑ष्ट॒ ऋषी॑णां पु॒त्रो अ॑धिरा॒ज ए॒षः। स्वा॒हा॒कृत्य॒ ब्रह्म॑णा ते जुहोमि॒ मा दे॒वानां᳚ मिथु॒याक॑र्भाग॒धेयम्᳚॥~(१३)

%1.3.8.0
{\anuvakamend[{भव॑त॒मेक॑त्रिꣳशच्च}]}%~(७)

%1.3.8.1
आ द॑द ऋ॒तस्य॑ त्वा देवहविः॒ पाशे॒ना\-ऽ\-ऽर॑भे॒ धर्\mbox{}षा॒ मानु॑षान॒द्भ्यस्त्वौष॑धीभ्यः॒ प्रोक्षा᳚म्य॒पां पे॒रुर॑सि स्वा॒त्तं चि॒थ्सदे॑वꣳ ह॒व्यमापो॑ देवीः॒ स्वद॑तैन॒ꣳ॒ सं ते᳚ प्रा॒णो वा॒युना॑ गच्छता॒ꣳ॒ सं यज॑त्रै॒रङ्गा॑नि॒ सं य॒ज्ञप॑तिरा॒शिषा॑ घृ॒तेना॒क्तौ प॒शुं त्रा॑येथा॒ꣳ॒ रेव॑तीर्य॒ज्ञप॑तिं प्रिय॒धा\-ऽ\-ऽवि॑श॒तोरो॑ अन्तरिक्ष स॒जूर्दे॒वेन॒~(१४)

%1.3.8.2
वाते॑ना॒\-ऽस्य ह॒विष॒स्त्मना॑ यज॒ सम॑स्य त॒नुवा॑ भव॒ वर्\mbox{}षी॑यो॒ वर्\mbox{}षी॑यसि य॒ज्ञे य॒ज्ञप॑तिं धाः पृथि॒व्याः स॒म्पृचः॑ पाहि॒ नम॑स्त आताना\-ऽन॒र्वा प्रेहि॑ घृ॒तस्य॑ कु॒ल्यामनु॑ स॒ह प्र॒जया॑ स॒ह रा॒यस्पोषे॒णा\-ऽ\-ऽपो॑ देवीः शुद्धायुवः शु॒द्धा यू॒यं दे॒वाꣳ ऊ᳚ड्ढ्वꣳ शु॒द्धा व॒यं परि॑विष्टाः परिवे॒ष्टारो॑ वो भूयास्म॥~(१५)

%1.3.9.0
{\anuvakamend[{दे॒वेन॒ चतु॑श्चत्वारिꣳशच्च}]}%~(८)

%1.3.9.1
वाक्त॒ आ प्या॑यतां प्रा॒णस्त॒ आ प्या॑यतां॒ चक्षु॑स्त॒ आ प्या॑यता॒ꣴ॒ श्रोत्रं॑ त॒ आ प्या॑यतां॒ या ते᳚ प्रा॒णाञ्छुग्ज॒गाम॒ या चक्षु॒र्या श्रोत्रं॒ यत् ते᳚ क्रू॒रं यदास्थि॑तं॒ तत् त॒ आ प्या॑यतां॒ तत् त॑ ए॒तेन॑ शुन्धतां॒ नाभि॑स्त॒ आ प्या॑यतां पा॒युस्त॒ आ प्या॑यताꣳ शु॒द्धाश्च॒रित्राः॒ शम॒द्भ्यः~(१६)

%1.3.9.2
शमोष॑धीभ्यः॒ शं पृ॑थि॒व्यै शमहो᳚भ्या॒मोष॑धे॒ त्राय॑स्वैन॒ꣴ॒ स्वधि॑ते॒ मैनꣳ॑ हिꣳसी॒ रक्ष॑सां भा॒गो॑\-ऽसी॒दम॒हꣳ रक्षो॑\-ऽध॒मं तमो॑ नयामि॒ यो᳚\-ऽस्मान् द्वेष्टि॒ यं च॑ व॒यं द्वि॒ष्म इ॒दमे॑नमध॒मं तमो॑ नयामी॒षे त्वा॑ घृ॒तेन॑ द्यावा\-पृथिवी॒ प्रोर्ण्वा॑था॒मच्छि॑न्नो॒ रायः॑ सु॒वीर॑ उ॒र्व॑न्तरि॑क्ष॒मन्वि॑हि॒ वायो॒ वीहि॑ स्तो॒काना॒ꣴ॒ स्वाहो॒र्ध्वन॑भसं मारु॒तं ग॑च्छतम्॥~(१७)

%1.3.10.0
{\anuvakamend[{अ॒द्भ्यो वीहि॒ पञ्च॑ च}]}%~(९)

%1.3.10.1
सं ते॒ मन॑सा॒ मनः॒ सं प्रा॒णेन॑ प्रा॒णो जुष्टं॑ दे॒वेभ्यो॑ ह॒व्यं घृ॒तव॒थ्\-स्वाहै॒न्द्रः प्रा॒णो अङ्गे॑अङ्गे॒ नि दे᳚ध्यदै॒न्द्रो॑\-ऽपा॒नो अङ्गे॑अङ्गे॒ वि बो॑भुव॒द्देव॑ त्वष्ट॒र्भूरि॑ ते॒ सꣳस॑मेतु॒ विषु॑रूपा॒ यथ्सल॑क्ष्माणो॒ भव॑थ देव॒त्रा यन्त॒मव॑से॒ सखा॒यो\-ऽनु॑ त्वा मा॒ता पि॒तरो॑ मदन्तु॒ श्रीर॑स्य॒ग्निस्त्वा᳚ श्रीणा॒त्वापः॒ सम॑रिण॒न्वात॑स्य~(१८)

%1.3.10.2
त्वा॒ ध्रज्यै॑ पू॒ष्णो रꣴह्या॑ अ॒पामोष॑धीना॒ꣳ॒ रोहि॑ष्यै घृ॒तं घृ॑तपावानः पिबत॒ वसां᳚ वसापावानः पिबता॒न्तरि॑क्षस्य ह॒विर॑सि॒ स्वाहा᳚ त्वा॒\-ऽन्तरि॑क्षाय॒ दिशः॑ प्र॒दिश॑ आ॒दिशो॑ वि॒दिश॑ उ॒द्दिशः॒ स्वाहा॑ दि॒ग्भ्यो नमो॑ दि॒ग्भ्यः॥~(१९)

%1.3.11.0
{\anuvakamend[{वात॑स्या॒ष्टाविꣳ॑शतिश्च}]}%॥10॥

%1.3.11.1
स॒मु॒द्रं ग॑च्छ॒ स्वाहा॒\-ऽन्तरि॑क्षं गच्छ॒ स्वाहा॑ दे॒वꣳ स॑वि॒तारं॑ गच्छ॒ स्वाहा॑\-ऽहोरा॒त्रे ग॑च्छ॒ स्वाहा॑ मि॒त्रावरु॑णौ गच्छ॒ स्वाहा॒ सोमं॑ गच्छ॒ स्वाहा॑ य॒ज्ञं ग॑च्छ॒ स्वाहा॒ छन्दाꣳ॑सि गच्छ॒ स्वाहा॒ द्यावा॑पृथि॒वी ग॑च्छ॒ स्वाहा॒ नभो॑ दि॒व्यं ग॑च्छ॒ स्वाहा॒\-ऽग्निं वै᳚श्वान॒रं ग॑च्छ॒ स्वाहा॒\-ऽद्भ्यस्त्वौष॑धीभ्यो॒ मनो॑ मे॒ हार्दि॑ यच्छ त॒नूं त्वचं॑ पु॒त्रं नप्ता॑रमशीय॒ शुग॑सि॒ तम॒भि शो॑च॒ यो᳚\-ऽस्मान् द्वेष्टि॒ यं च॑ व॒यं द्वि॒ष्मो धाम्नो॑धाम्नो राजन्नि॒तो व॑रुण नो मुञ्च॒ यदापो॒ अघ्नि॑या॒ वरु॒णेति॒ शपा॑महे॒ ततो॑ वरुण नो मुञ्च॥~(२०)

%1.3.12.0
{\anuvakamend[{अ॒सि॒ षड्विꣳ॑शतिश्च}]}%॥11॥

%1.3.12.1
ह॒विष्म॑तीरि॒मा आपो॑ ह॒विष्मा᳚न् दे॒वो अ॑ध्व॒रो ह॒विष्मा॒ꣳ॒ आ वि॑वासति ह॒विष्माꣳ॑ अस्तु॒ सूर्यः॑॥ अ॒ग्नेर्वो\-ऽप॑न्नगृहस्य॒ सद॑सि सादयामि सु॒म्नाय॑ सुम्निनीः सु॒म्ने मा॑ धत्तेन्द्राग्नि॒योर्भा॑ग॒धेयीः᳚ स्थ मि॒त्रावरु॑णयोर्भाग॒धेयीः᳚ स्थ॒ विश्वे॑षां दे॒वानां᳚ भाग॒धेयीः᳚ स्थ य॒ज्ञे जा॑गृत॥~(२१)

%1.3.13.0
{\anuvakamend[{ह॒विष्म॑ती॒श्चतु॑स्रिꣳशत्}]}%॥12॥

%1.3.13.1
हृ॒दे त्वा॒ मन॑से त्वा दि॒वे त्वा॒ सूर्या॑य त्वो॒र्ध्वमि॒मम॑ध्व॒रं कृ॑धि दि॒वि दे॒वेषु॒ होत्रा॑ यच्छ॒ सोम॑ राज॒न्नेह्यव॑ रोह॒ मा भेर्मा सं वि॑क्था॒ मा त्वा॑ हिꣳसिषं प्र॒जास्त्वमु॒पाव॑रोह प्र॒जास्त्वामु॒पाव॑ रोहन्तु शृ॒णोत्व॒ग्निः स॒मिधा॒ हवं॑ मे शृ॒ण्वन्त्वापो॑ धि॒षणा᳚श्च दे॒वीः। शृ॒णोत॑ ग्रावाणो वि॒दुषो॒ नु~(२२)

%1.3.13.2
य॒ज्ञꣳ शृ॒णोतु॑ दे॒वः स॑वि॒ता हवं॑ मे। देवी॑रापो अपां नपा॒द्य ऊ॒र्मिर्\mbox{}ह॑वि॒ष्य॑ इन्द्रि॒यावा᳚न्म॒दिन्त॑म॒स्तं दे॒वेभ्यो॑ देव॒त्रा ध॑त्त शु॒क्रꣳ शु॑क्र॒पेभ्यो॒ येषां᳚ भा॒गः स्थ स्वाहा॒ कार्\mbox{}षि॑र॒स्यपा॒पां मृ॒ध्रꣳ स॑मु॒द्रस्य॒ वोक्षि॑त्या॒ उन्न॑ये। यम॑ग्ने पृ॒थ्सु मर्त्य॒मावो॒ वाजे॑षु॒ यं जु॒नाः। स यन्ता॒ शश्व॑ती॒रिषः॑॥~(२३)

%1.3.14.0
{\anuvakamend[{नु स॒प्तच॑त्वारिꣳशच्च}]}%॥13॥

%1.3.14.1
त्वम॑ग्ने रु॒द्रो असु॑रो म॒हो दि॒वस्त्वꣳ शर्धो॒ मारु॑तं पृ॒क्ष ई॑शिषे। त्वं वातै॑ररु॒णैर्या॑सि शङ्ग॒यस्त्वं पू॒षा वि॑ध॒तः पा॑सि॒ नु त्मना᳚॥ आ वो॒ राजा॑नमध्व॒रस्य॑ रु॒द्रꣳ होता॑रꣳ सत्य॒यज॒ꣳ॒ रोद॑स्योः। अ॒ग्निं पु॒रा त॑नयि॒त्नोर॒चित्ता॒द्धिर॑ण्यरूप॒मव॑से कृणुध्वम्॥ अ॒ग्निर्\mbox{}होता॒ निष॑सादा॒ यजी॑यानु॒पस्थे॑ मा॒तुः सु॑र॒भावु॑ लो॒के। युवा॑ क॒विः पु॑रुनि॒ष्ठ~-~(२४)

%1.3.14.2
ऋ॒तावा॑ ध॒र्ता कृ॑ष्टी॒नामु॒त मध्य॑ इ॒द्धः॥ सा॒ध्वीम॑कर्दे॒ववी॑तिं नो अ॒द्य य॒ज्ञस्य॑ जि॒ह्वाम॑विदाम॒ गुह्या᳚म्। स आयु॒रागा᳚थ्सुर॒\-भिर्वसा॑नो भ॒द्राम॑कर्दे॒वहू॑तिं नो अ॒द्य॥ अक्र॑न्दद॒ग्निः स्त॒नय॑न्निव॒ द्यौः क्षामा॒ रेरि॑हद्वी॒रुधः॑ सम॒ञ्जन्न्। स॒द्यो ज॑ज्ञा॒नो विहीमि॒द्धो अख्य॒दा रोद॑सी भा॒नुना॑ भात्य॒न्तः॥ त्वे वसू॑नि पुर्वणीक~(२५)

%1.3.14.3
होतर्दो॒षा वस्तो॒रेरि॑रे य॒ज्ञिया॑सः। क्षामे॑व॒ विश्वा॒ भुव॑नानि॒ यस्मि॒न्थ्सꣳ सौभ॑गानि दधि॒रे पा॑व॒के॥ तुभ्यं॒ ता अ॑ङ्गिरस्तम॒ विश्वाः᳚ सुक्षि॒तयः॒ पृथ॑क्। अग्ने॒ कामा॑य येमिरे॥ अ॒श्याम॒ तं काम॑मग्ने॒ तवो॒त्य॑श्याम॑ र॒यिꣳ र॑यिवः सु॒वीरम्᳚। अ॒श्याम॒ वाज॑म॒भि वा॒जय॑न्तो॒\-ऽश्याम॑ द्यु॒म्नम॑जरा॒जरं॑ ते॥ श्रेष्ठं॑ यविष्ठ भार॒ताग्ने᳚ द्यु॒मन्त॒माभ॑र।~(२६)

%1.3.14.4
वसो॑ पुरु॒स्पृहꣳ॑ र॒यिम्॥ स श्वि॑ता॒नस्त॑न्य॒तू रो॑चन॒स्था अ॒जरे॑भि॒र्नान॑दद्भि॒र्यवि॑ष्ठः। यः पा॑व॒कः पु॑रु॒तमः॑ पु॒रूणि॑ पृ॒थून्य॒ग्निर॑नु॒याति॒ भर्वन्न्॑॥ आयु॑ष्टे वि॒श्वतो॑ दधद॒यम॒ग्निर्वरे᳚ण्यः। पुन॑स्ते प्रा॒ण आय॑ति॒ परा॒ यक्ष्मꣳ॑ सुवामि ते॥ आ॒यु॒र्दा अ॑ग्ने ह॒विषो॑ जुषा॒णो घृ॒तप्र॑तीको घृ॒तयो॑निरेधि। घृ॒तं पी॒त्वा मधु॒ चारु॒ गव्यं॑ पि॒तेव॑ पु॒त्रम॒भि~(२७)

%1.3.14.5
र॑क्षतादि॒मम्॥ तस्मै॑ ते प्रति॒हर्य॑ते॒ जात॑वेदो॒ विच॑र्\mbox{}षणे। अग्ने॒ जना॑मि सुष्टु॒तिम्॥ दि॒वस्परि॑ प्रथ॒मं ज॑ज्ञे अ॒ग्निर॒स्मद् द्वि॒तीयं॒ परि॑ जा॒तवे॑दाः। तृ॒तीय॑म॒फ्सु नृ॒मणा॒ अज॑स्र॒मिन्धा॑न एनं जरते स्वा॒धीः॥ शुचिः॑ पावक॒ वन्द्यो\-ऽग्ने॑ बृ॒हद्वि रो॑चसे। त्वं घृ॒तेभि॒राहु॑तः॥ दृ॒शा॒नो रु॒क्म उ॒र्व्या व्य॑द्यौद् दु॒र्मर्\mbox{}ष॒मायुः॑ श्रि॒ये रु॑चा॒नः। अ॒ग्निर॒मृतो॑ अभव॒द्वयो॑भि॒र्-~(२८)

%1.3.14.6
यदे॑नं॒ द्यौरज॑नयथ्सु॒रेताः᳚॥ आ यदि॒षे नृ॒पतिं॒ तेज॒ आन॒ट्छुचि॒ रेतो॒ निषि॑क्तं॒ द्यौर॒भीके᳚। अ॒ग्निः शर्ध॑मनव॒द्यं युवा॑नꣴ स्वा॒धियं॑ जनयथ्सू॒दय॑च्च॥ स तेजी॑यसा॒ मन॑सा॒ त्वोत॑ उ॒त शि॑क्ष स्वप॒त्यस्य॑ शि॒क्षोः। अग्ने॑ रा॒यो नृत॑मस्य॒ प्रभू॑तौ भू॒याम॑ ते सुष्टु॒तय॑श्च॒ वस्वः॑॥ अग्ने॒ सह॑न्त॒मा भ॑र द्यु॒म्नस्य॑ प्रा॒सहा॑ र॒यिम्। विश्वा॒ यश्-~(२९)

%1.3.14.7
च॑र्\mbox{}ष॒णीर॒भ्या॑सा वाजे॑षु सा॒सह॑त्॥ तम॑ग्ने पृतना॒सहꣳ॑ र॒यिꣳ स॑हस्व॒ आ भ॑र। त्वꣳ हि स॒त्यो अद्भु॑तो दा॒ता वाज॑स्य॒ गोम॑तः॥ उ॒क्षान्ना॑य व॒शान्ना॑य॒ सोम॑पृष्ठाय वे॒धसे᳚। स्तोमै᳚र्विधेमा॒ग्नये᳚॥ व॒द्मा हि सू॑नो॒ अस्य॑द्म॒सद्वा॑ च॒क्रे अ॒ग्निर्ज॒नुषाज्मान्नम्᳚। स त्वं न॑ ऊर्जसन॒ ऊर्जं॑ धा॒ राजे॑व जेरवृ॒के क्षे᳚ष्य॒न्तः॥ अग्न॒ आयूꣳ॑षि~(३०)

%1.3.14.8
पवस॒ आ सु॒वोर्ज॒मिषं॑ च नः। आ॒रे बा॑धस्व दु॒च्छुना᳚म्॥ अग्ने॒ पव॑स्व॒ स्वपा॑ अ॒स्मे वर्चः॑ सु॒वीर्यम्᳚। दध॒त्पोषꣳ॑ र॒यिं मयि॑॥ अग्ने॑ पावक रो॒चिषा॑ म॒न्द्रया॑ देव जि॒ह्वया᳚। आ दे॒वान् व॑क्षि॒ यक्षि॑ च॥ स नः॑ पावक दीदि॒वो\-ऽग्ने॑ दे॒वाꣳ इ॒हा व॑ह। उप॑ य॒ज्ञꣳ ह॒विश्च॑ नः॥ अ॒ग्निः शुचि॑व्रततमः॒ शुचि॒र्विप्रः॒ शुचिः॑ क॒विः। शुची॑ रोचत॒ आहु॑तः॥ उद॑ग्ने॒ शुच॑य॒स्तव॑ शु॒क्रा भ्राज॑न्त ईरते। तव॒ ज्योतीꣴ॑ष्य॒र्चयः॑॥~(३१)

{\anuvakamend[{पु॒रु॒नि॒ष्ठः पु॑र्वणीक भरा॒\-ऽभि वयो॑भि॒र्य आयूꣳ॑षि॒ विप्रः॒ शुचि॒श्चतु॑र्दश च}]}%॥14॥

{\prashnaend[{दे॒वस्य॑ रक्षो॒हणो॑ वि॒भूस्त्वꣳ सो॒मात्य॒न्यानगां᳚ पृथि॒व्या इ॒षे त्वा\-ऽ\-ऽद॑दे॒ वाक्ते॒ सं ते॑ समु॒द्रꣳ ह॒विष्म॑तीर्\mbox{}हृ॒दे त्वम॑ग्ने रु॒द्रश्चतु॑र्दश॥ दे॒वस्य॑ ग॒मध्ये॑ ह॒विष्म॑तीः पवस॒ एक॑त्रिꣳशत्॥ दे॒वस्या॒र्चयः॑॥}]}

%%% END PRASHNA

\sect{चतुर्थः प्रश्नः}\setcounter{anuvakam}{0}
\dnsub{तैत्तिरीयसंहितायां प्रथमकाण्डे चतुर्थः प्रश्नः}
%1.4.1.0
%1.4.1.1
आ द॑दे॒ ग्रावा᳚स्यध्वर॒कृद् दे॒वेभ्यो॑ गम्भी॒रमि॒मम॑ध्व॒रं कृ॑ध्युत्त॒मेन॑ प॒विनेन्द्रा॑य॒ सोम॒ꣳ॒ सुषु॑तं॒ मधु॑मन्तं॒ पय॑स्वन्तं वृष्टि॒वनि॒मिन्द्रा॑य त्वा वृत्र॒घ्न इन्द्रा॑य त्वा वृत्र॒तुर॒ इन्द्रा॑य त्वा\-ऽभिमाति॒घ्न इन्द्रा॑य त्वा\-ऽ\-ऽदि॒त्यव॑त॒ इन्द्रा॑य त्वा वि॒श्वदे᳚व्यावते श्वा॒त्राः स्थ॑ वृत्र॒तुरो॒ राधो॑गूर्ता अ॒मृत॑स्य॒ पत्नी॒स्ता दे॑वीर्देव॒त्रेमं य॒ज्ञं ध॒त्तोप॑हूताः॒ सोम॑स्य पिब॒तोप॑हूतो यु॒ष्माक॒ꣳ॒~(१)

%1.4.1.2
सोमः॑ पिबतु॒ यत्ते॑ सोम दि॒वि ज्योति॒र्यत् पृ॑थि॒व्यां यदु॒राव॒न्तरि॑क्षे॒ तेना॒स्मै यज॑मानायो॒रु रा॒या कृ॒ध्यधि॑ दा॒त्रे वो॑चो॒ धिष॑णे वी॒डू स॒ती वी॑डयेथा॒मूर्जं॑ दधाथा॒मूर्जं॑ मे धत्तं॒ मा वाꣳ॑ हिꣳसिषं॒ मा मा॑ हिꣳसिष्टं॒ प्रागपा॒गुद॑गध॒राक्तास्त्वा॒ दिश॒ आ धा॑व॒न्त्वम्ब॒ नि ष्व॑र। यत्ते॑ सो॒मादा᳚भ्यं॒ नाम॒ जागृ॑वि॒ तस्मै॑ ते सोम॒ सोमा॑य॒ स्वाहा᳚॥~(२)

%1.4.2.0
{\anuvakamend[{यु॒ष्माकꣴ॑ स्वर॒ यत्ते॒ नव॑ च}]}%~(१)

%1.4.2.1
वा॒चस्पत॑ये पवस्व वाजि॒न् वृषा॒ वृष्णो॑ अ॒ꣳ॒शुभ्यां॒ गभ॑स्तिपूतो दे॒वो दे॒वानां᳚ प॒वित्र॑मसि॒ येषां᳚ भा॒गो\-ऽसि॒ तेभ्य॑स्त्वा॒ स्वां कृ॑तो\-ऽसि॒ मधु॑मतीर्न॒ इष॑स्कृधि॒ विश्वे᳚भ्यस्त्वेन्द्रि॒येभ्यो॑ दि॒व्येभ्यः॒ पार्थि॑वेभ्यो॒ मन॑स्त्वाष्टू॒र्व॑न्त\-रि॑क्ष॒मन्वि॑हि॒ स्वाहा᳚ त्वा सुभवः॒ सूर्या॑य दे॒वेभ्य॑स्त्वा मरीचि॒पेभ्य॑ ए॒ष ते॒ योनिः॑ प्रा॒णाय॑ त्वा॥~(३)

%1.4.3.0
{\anuvakamend[{वा॒चः स॒प्तच॑त्वारिꣳशत्}]}%~(२)

%1.4.3.1
उ॒प॒या॒मगृ॑हीतो\-ऽस्य॒न्तर्य॑च्छ मघवन् पा॒हि सोम॑मुरु॒ष्य रायः॒ समिषो॑ यजस्वा॒न्तस्ते॑ दधामि॒ द्यावा॑पृथि॒वी अ॒न्तरु॒र्व॑न्तरि॑क्षꣳ स॒जोषा॑ दे॒वैरव॑रैः॒ परै᳚श्चान्तर्या॒मे म॑घवन् मादयस्व॒ स्वां कृ॑तो\-ऽसि॒ मधु॑मतीर्न॒ इष॑स्कृधि॒ विश्वे᳚भ्यस्त्वेन्द्रि॒येभ्यो॑ दि॒व्येभ्यः॒ पार्थि॑वेभ्यो॒ मन॑स्त्वाष्टू॒र्व॑न्त\-रि॑क्ष॒मन्वि॑हि॒ स्वाहा᳚ त्वा सुभवः॒ सूर्या॑य दे॒वेभ्य॑स्त्वा मरीचि॒पेभ्य॑ ए॒ष ते॒ योनि॑रपा॒नाय॑ त्वा॥~(४)

%1.4.4.0
{\anuvakamend[{दे॒वेभ्यः॑ स॒प्त च॑}]}%~(३)

%1.4.4.1
आ वा॑यो भूष शुचिपा॒ उप॑ नः स॒हस्रं॑ ते नि॒युतो॑ विश्ववार। उपो॑ ते॒ अन्धो॒ मद्य॑मयामि॒ यस्य॑ देव दधि॒षे पू᳚र्व॒पेयम्᳚॥ उ॒प॒या॒मगृ॑हीतो\-ऽसि वा॒यवे॒ त्वेन्द्र॑वायू इ॒मे सु॒ताः। उप॒ प्रयो॑भि॒रा ग॑त॒मिन्द॑वो वामु॒शन्ति॒ हि॥ उ॒प॒या॒मगृ॑हीतो\-ऽसीन्द्रवा॒यु\-भ्यां᳚ त्वै॒ष ते॒ योनिः॑ स॒जोषा᳚भ्यां त्वा॥~(५)

%1.4.5.0
{\anuvakamend[{आ वा॑यो॒ त्रिच॑त्वारिꣳशत्}]}%~(४)

%1.4.5.1
अ॒यं वां᳚ मित्रावरुणा सु॒तः सोम॑ ऋतावृधा। ममेदि॒ह श्रु॑त॒ꣳ॒ हवम्᳚। उ॒प॒या॒मगृ॑हीतो\-ऽसि मि॒त्रावरु॑णाभ्यां त्वै॒ष ते॒ योनि॑र् ऋता॒यु\-भ्यां᳚ त्वा॥~(६)

%1.4.6.0
{\anuvakamend[{अ॒यं वां᳚ विꣳश॒तिः}]}%~(५)

%1.4.6.1
या वां॒ कशा॒ मधु॑म॒त्यश्वि॑ना सू॒नृता॑वती। तया॑ य॒ज्ञं मि॑मिक्षतम्। उ॒प॒या॒मगृ॑हीतो\-ऽस्य॒श्वि\-भ्यां᳚ त्वै॒ष ते॒ योनि॒र्माध्वी᳚भ्यां त्वा॥~(७)

%1.4.7.0
{\anuvakamend[{या वा॑म॒ष्टाद॑श}]}%~(६)

%1.4.7.1
प्रा॒त॒र्युजौ॒ वि मु॑च्येथा॒मश्वि॑ना॒वेह ग॑च्छतम्। अ॒स्य सोम॑स्य पी॒तये᳚॥ उ॒प॒या॒मगृ॑हीतो\-ऽस्य॒श्वि\-भ्यां᳚ त्वै॒ष ते॒ योनि॑र॒श्वि\-भ्यां᳚ त्वा॥~(८)

%1.4.8.0
{\anuvakamend[{प्रा॒त॒र्युजा॒वेका॒न्नविꣳ॑शतिः}]}%~(७)

%1.4.8.1
अ॒यं वे॒नश्चो॑दय॒त् पृश्ञि॑गर्भा॒ ज्योति॑र्जरायू॒ रज॑सो वि॒माने᳚। इ॒मम॒पाꣳ स॑ङ्ग॒मे सूर्य॑स्य॒ शिशुं॒ न विप्रा॑ म॒तिभी॑ रिहन्ति॥ उ॒प॒या॒मगृ॑हीतो\-ऽसि॒ शण्डा॑य त्वै॒ष ते॒ योनि॑र्वी॒रतां᳚ पाहि॥~(९)

%1.4.9.0
{\anuvakamend[{अ॒यं वे॒नः पञ्च॑विꣳशतिः}]}%~(८)

%1.4.9.1
तं प्र॒त्नथा॑ पू॒र्वथा॑ वि॒श्वथे॒मथा᳚ ज्ये॒ष्ठता॑तिं बर्\mbox{}हि॒षदꣳ॑ सुव॒र्विदं॑ प्रतीची॒नं वृ॒जनं॑ दोहसे गि॒रा\-ऽ\-ऽशुं जय॑न्त॒मनु॒ यासु॒ वर्ध॑से। उ॒प॒या॒मगृ॑हीतो\-ऽसि॒ मर्का॑य त्वै॒ष ते॒ योनिः॑ प्र॒जाः पा॑हि॥~(१०)

%1.4.10.0
{\anuvakamend[{तꣳ षड्विꣳ॑शतिः}]}%~(९)

%1.4.10.1
ये दे॑वा दि॒व्येका॑\-दश॒ स्थ पृ॑थि॒व्यामध्येका॑\-दश॒ स्था\-ऽफ्सु॒षदो॑ महि॒नैका॑\-दश॒ स्थ ते दे॑वा य॒ज्ञमि॒मं जु॑षध्वमुपया॒मगृ॑हीतो\-ऽस्याग्रय॒णो॑\-ऽसि॒ स्वा᳚ग्रयणो॒ जिन्व॑ य॒ज्ञं जिन्व॑ य॒ज्ञप॑तिम॒भि सव॑ना पाहि॒ विष्णु॒स्त्वां पा॑तु॒ विशं॒ त्वं पा॑हीन्द्रि॒येणै॒ष ते॒ योनि॒र्विश्वे᳚भ्यस्त्वा दे॒वेभ्यः॑॥~(११)

%1.4.12.0
{\anuvakamend[{ये दे॑वा॒स्त्रिच॑त्वारिꣳशत्}]}%॥10॥

%1.4.11.1
त्रि॒ꣳ॒शत्त्रय॑श्च ग॒णिनो॑ रु॒जन्तो॒ दिवꣳ॑ रु॒द्राः पृ॑थि॒वीं च॑ सचन्ते। ए॒का॒द॒शासो॑ अफ्सु॒षदः॑ सु॒तꣳ सोमं॑ जुषन्ता॒ꣳ॒ सव॑नाय॒ विश्वे᳚॥ उ॒प॒या॒मगृ॑हीतो\-ऽस्याग्रय॒णो॑\-ऽसि॒ स्वा᳚ग्रयणो॒ जिन्व॑ य॒ज्ञं जिन्व॑ य॒ज्ञप॑तिम॒भि सव॑ना पाहि॒ विष्णु॒स्त्वां पा॑तु॒ विशं॒ त्वं पा॑हीन्द्रि॒येणै॒ष ते॒ योनि॒र्विश्वे᳚भ्यस्त्वा दे॒वेभ्यः॑॥~(१२)

%1.4.10.0
{\anuvakamend[{त्रि॒ꣳ॒शद् द्विच॑त्वारिꣳशत्}]}%॥11॥

%1.4.12.1
उ॒प॒या॒मगृ॑हीतो॒\-ऽसीन्द्रा॑य त्वा बृ॒हद्व॑ते॒ वय॑स्वत उक्था॒युवे॒ यत् त॑ इन्द्र बृ॒हद्वय॒स्तस्मै᳚ त्वा॒ विष्ण॑वे त्वै॒ष ते॒ योनि॒रिन्द्रा॑य त्वोक्था॒युवे᳚॥~(१३)

%1.4.13.0
{\anuvakamend[{उ॒प॒या॒मगृ॑हीतो॒ द्वाविꣳ॑शतिः}]}%॥12॥

%1.4.13.1
मू॒र्धानं॑ दि॒वो अ॑र॒तिं पृ॑थि॒व्या वै᳚श्वान॒रमृ॒ताय॑ जा॒तम॒ग्निम्। क॒विꣳ स॒म्राज॒मति॑थिं॒ जना॑नामा॒सन्ना पात्रं॑ जनयन्त दे॒वाः॥ उ॒प॒या॒मगृ॑हीतो\-ऽस्य॒ग्नये᳚ त्वा वैश्वान॒राय॑ ध्रु॒वो॑\-ऽसि ध्रु॒वक्षि॑तिर्ध्रु॒वाणां᳚ ध्रु॒वत॒मो\-ऽच्यु॑तानामच्युत॒क्षित्त॑म ए॒ष ते॒ योनि॑र॒ग्नये᳚ त्वा वैश्वान॒राय॑॥~(१४)

%1.4.14.0
{\anuvakamend[{मू॒र्धानं॒ पञ्च॑त्रिꣳशत्}]}%॥13॥

%1.4.14.1
मधु॑श्च॒ माध॑वश्च शु॒क्रश्च॒ शुचि॑श्च॒ नभ॑श्च नभ॒स्य॑श्चे॒षश्चो॒र्जश्च॒ सह॑श्च सह॒स्य॑श्च॒ तप॑श्च तप॒स्य॑श्चोपया॒मगृ॑हीतो\-ऽसि स॒ꣳ॒सर्पो᳚\-ऽस्यꣳहस्प॒त्याय॑ त्वा॥~(१५)

%1.4.15.0
{\anuvakamend[{मधु॑स्त्रि॒ꣳ॒शत्}]}%॥14॥

%1.4.15.1
इन्द्रा᳚ग्नी॒ आ ग॑तꣳ सु॒तं गी॒र्भिर्नभो॒ वरे᳚ण्यम्। अ॒स्य पा॑तं धि॒येषि॒ता॥ उ॒प॒या॒मगृ॑हीतो\-ऽसीन्द्रा॒ग्नि\-भ्यां᳚ त्वै॒ष ते॒ योनि॑रिन्द्रा॒ग्नि\-भ्यां᳚ त्वा॥~(१६)

%1.4.16.0
{\anuvakamend[{इन्द्रा᳚ग्नी विꣳश॒तिः}]}%॥15॥

%1.4.16.1
ओमा॑सश्चर्\mbox{}षणीधृतो॒ विश्वे॑ देवास॒ आ ग॑त। दा॒श्वाꣳसो॑ दा॒शुषः॑ सु॒तम्॥ उ॒प॒या॒मगृ॑हीतो\-ऽसि॒ विश्वे᳚भ्यस्त्वा दे॒वेभ्य॑ ए॒ष ते॒ योनि॒र्विश्वे᳚भ्यस्त्वा दे॒वेभ्यः॑॥~(१७)

%1.4.17.0
{\anuvakamend[{इन्द्रा᳚ग्नी॒ ओमा॑सो विꣳश॒तिर्विꣳ॑शतिः}]}%॥16॥

%1.4.17.1
म॒रुत्व॑न्तं वृष॒भं वा॑वृधा॒नमक॑वारिं दि॒व्यꣳ शा॒समिन्द्रम्᳚। वि॒श्वा॒साह॒मव॑से॒ नूत॑नायो॒ग्रꣳ स॑हो॒दामि॒ह तꣳ हु॑वेम॥ उ॒प॒या॒मगृ॑हीतो॒\-ऽसीन्द्रा॑य त्वा म॒रुत्व॑त ए॒ष ते॒ योनि॒रिन्द्रा॑य त्वा म॒रुत्व॑ते॥~(१८)

%1.4.18.0
{\anuvakamend[{म॒रुत्व॑न्त॒ꣳ॒ षड्विꣳ॑शतिः}]}%॥17॥

%1.4.18.1
इन्द्र॑ मरुत्व इ॒ह पा॑हि॒ सोमं॒ यथा॑ शार्या॒ते अपि॑बः सु॒तस्य॑। तव॒ प्रणी॑ती॒ तव॑ शूर॒ शर्म॒न्ना वि॑वासन्ति क॒वयः॑ सुय॒ज्ञाः॥ उ॒प॒या॒मगृ॑हीतो॒\-ऽसीन्द्रा॑य त्वा म॒रुत्व॑त ए॒ष ते॒ योनि॒रिन्द्रा॑य त्वा म॒रुत्व॑ते॥~(१९)

%1.4.19.0
{\anuvakamend[{इन्द्रैका॒न्नत्रि॒ꣳ॒शत्}]}%॥18॥

%1.4.19.1
म॒रुत्वाꣳ॑ इन्द्र वृष॒भो रणा॑य॒ पिबा॒ सोम॑मनुष्व॒धं मदा॑य। आ सि॑ञ्चस्व ज॒ठरे॒ मध्व॑ ऊ॒र्मिं त्वꣳ राजा॑ऽसि प्र॒दिवः॑ सु॒ताना᳚म्॥ उ॒प॒या॒मगृ॑हीतो॒\-ऽसीन्द्रा॑य त्वा म॒रुत्व॑त ए॒ष ते॒ योनि॒रिन्द्रा॑य त्वा म॒रुत्व॑ते॥~(२०)

%1.4.20.0
{\anuvakamend[{इन्द्र॑ मरुत्वो म॒रुत्वा॒नेका॒न्न त्रि॒ꣳ॒शदेका॒न्न त्रि॒ꣳ॒शत्}]}%॥19॥

%1.4.20.1
म॒हाꣳ इन्द्रो॒ य ओज॑सा प॒र्जन्यो॑ वृष्टि॒माꣳ इ॑व। स्तोमै᳚र्व॒थ्सस्य॑ वावृधे॥ उ॒प॒या॒मगृ॑हीतो\-ऽसि महे॒न्द्राय॑ त्वै॒ष ते॒ योनि॑र्महे॒न्द्राय॑ त्वा॥~(२१)

%1.4.21.0
{\anuvakamend[{म॒हानेका॒न्नविꣳ॑शतिः}]}%॥20॥

%1.4.21.1
म॒हाꣳ इन्द्रो॑ नृ॒वदा च॑र्\mbox{}षणि॒प्रा उ॒त द्वि॒बर्\mbox{}हा॑ अमि॒नः सहो॑भिः। अ॒स्म॒द्रिय॑ग्वावृधे वी॒र्या॑यो॒रुः पृ॒थुः सुकृ॑तः क॒र्तृभि॑र्भूत्॥ उ॒प॒या॒मगृ॑हीतो\-ऽसि महे॒न्द्राय॑ त्वै॒ष ते॒ योनि॑र्महे॒न्द्राय॑ त्वा॥~(२२)

%1.4.22.0
{\anuvakamend[{म॒हान्नृ॒वत्षड्विꣳ॑शतिः}]}%॥21॥

%1.4.22.1
क॒दा च॒न स्त॒रीर॑सि॒ नेन्द्र॑ सश्चसि दा॒शुषे᳚। उपो॒पेन्नु म॑घव॒न् भूय॒ इन्नु ते॒ दानं॑ दे॒वस्य॑ पृच्यते॥ उ॒प॒या॒मगृ॑हीतो\-ऽस्यादि॒त्येभ्य॑स्त्वा॥ क॒दा च॒न प्र यु॑च्छस्यु॒भे नि पा॑सि॒ जन्म॑नी। तुरी॑यादित्य॒ सव॑नं त इन्द्रि॒यमा त॑स्थाव॒मृतं॑ दि॒वि॥ य॒ज्ञो दे॒वानां॒ प्रत्ये॑ति सु॒म्नमादि॑त्यासो॒ भव॑ता मृड॒यन्तः॑। आ वो॒ऽर्वाची॑ सुम॒तिर्व॑वृत्याद॒ꣳ॒होश्चि॒द्या व॑रिवो॒वित्त॒रास॑त्॥ विव॑स्व आदित्यै॒ष ते॑ सोमपी॒थस्तेन॑ मन्दस्व॒ तेन॑ तृप्य तृ॒प्यास्म॑ ते व॒यं त॑र्पयि॒तारो॒ या दि॒व्या वृष्टि॒स्तया᳚ त्वा श्रीणामि॥~(२३)

%1.4.23.0
{\anuvakamend[{वः॒ स॒प्तविꣳ॑शतिश्च}]}%॥2॥

%1.4.23.1
वा॒मम॒द्य स॑वितर्वा॒ममु॒ श्वो दि॒वेदि॑वे वा॒मम॒स्मभ्यꣳ॑ सावीः। वा॒मस्य॒ हि क्षय॑स्य देव॒ भूरे॑र॒या धि॒या वा॑म॒भाजः॑ स्याम॥ उ॒प॒या॒मगृ॑हीतो\-ऽसि दे॒वाय॑ त्वा सवि॒त्रे॥~(२४)

%1.4.24.0
{\anuvakamend[{वा॒मं चतु॑र्विꣳशतिः}]}%॥23॥

%1.4.24.1
अद॑ब्धेभिः सवितः पा॒युभि॒ष्ट्वꣳ शि॒वेभि॑र॒द्य परि॑ पाहि नो॒ गयम्᳚। हिर॑ण्यजिह्वः सुवि॒ताय॒ नव्य॑से॒ रक्षा॒ माकि॑र्नो अ॒घशꣳ॑स ईशत॥ उ॒प॒या॒मगृ॑हीतो\-ऽसि दे॒वाय॑ त्वा सवि॒त्रे॥~(२५)

%1.4.25.0
{\anuvakamend[{अद॑ब्धेभि॒स्त्रयो॑विꣳशतिः}]}%॥24॥

%1.4.25.1
हिर॑ण्यपाणिमू॒तये॑ सवि॒तार॒मुप॑ ह्वये। स चेत्ता॑ दे॒वता॑ प॒दम्॥ उ॒प॒या॒मगृ॑हीतो\-ऽसि दे॒वाय॑ त्वा सवि॒त्रे॥~(२६)

%1.4.26.0
{\anuvakamend[{हिर॑ण्यपाणिं॒ चतु॑र्दश}]}%॥25॥

%1.4.26.1
सु॒शर्मा॑\-ऽसि सुप्रतिष्ठा॒नो बृ॒हदु॒क्षे नम॑ ए॒ष ते॒ योनि॒र्विश्वे᳚भ्यस्त्वा दे॒वेभ्यः॑॥~(२७)

%1.4.27.0
{\anuvakamend[{सु॒शर्मा॒ द्वाद॑श}]}%॥26॥

%1.4.27.1
बृह॒स्पति॑सुतस्य त इन्दो इन्द्रि॒याव॑तः॒ पत्नी॑वन्तं॒ ग्रहं॑ गृह्णा॒म्यग्ना(३)इ पत्नी॒वा(३)ः स॒जूर्दे॒वेन॒ त्वष्ट्रा॒ सोमं॑ पिब॒ स्वाहा᳚॥~(२८)

%1.4.28.0
{\anuvakamend[{बृह॒स्पति॑सुतस्य॒ पञ्च॑दश}]}%॥27॥

%1.4.28.1
हरि॑रसि हारियोज॒नो हर्योः᳚ स्था॒ता वज्र॑स्य भ॒र्ता पृश्ञेः᳚ प्रे॒ता तस्य॑ ते देव सोमे॒ष्टय॑जुषः स्तु॒तस्तो॑मस्य श॒स्तोक्थ॑स्य॒ हरि॑वन्तं॒ ग्रहं॑ गृह्णामि ह॒रीः स्थ॒ हर्यो᳚र्धा॒नाः स॒हसो॑मा॒ इन्द्रा॑य॒ स्वाहा᳚॥~(२९)

%1.4.29.0
{\anuvakamend[{हरिः॒ षड्विꣳ॑शतिः}]}%॥28॥

%1.4.29.1
अग्न॒ आयूꣳ॑षि पवस॒ आ सु॒वोर्ज॒मिषं॑ च नः। आ॒रे बा॑धस्व दु॒च्छुना᳚म्॥ उ॒प॒या॒मगृ॑हीतो\-ऽस्य॒ग्नये᳚ त्वा॒ तेज॑स्वत ए॒ष ते॒ योनि॑र॒ग्नये᳚ त्वा॒ तेज॑स्वते॥~(३०)

%1.4.30.0
{\anuvakamend[{अग्न॒ आयूꣳ॑षि॒ त्रयो॑वि ꣳशतिः}]}%॥29॥

%1.4.30.1
उ॒त्तिष्ठ॒न्नोज॑सा स॒ह पी॒त्वा शिप्रे॑ अवेपयः। सोम॑मिन्द्र च॒मू सु॒तम्॥ उ॒प॒या॒मगृ॑हीतो॒\-ऽसीन्द्रा॑य॒ त्वौज॑स्वत ए॒ष ते॒ योनि॒रिन्द्रा॑य॒ त्वौज॑स्वते॥~(३१)

%1.4.31.0
{\anuvakamend[{उ॒त्तिष्ठ॒न्नेक॑विꣳशतिः}]}%॥30॥

%1.4.31.1
त॒रणि॑र्वि॒श्वद॑र्\mbox{}शतो ज्योति॒ष्कृद॑सि सूर्य। विश्व॒मा भा॑सि रोच॒नम्॥ उ॒प॒या॒मगृ॑हीतो\-ऽसि॒ सूर्या॑य त्वा॒ भ्राज॑स्वत ए॒ष ते॒ योनिः॒ सूर्या॑य त्वा॒ भ्राज॑स्वते॥~(३२)

%1.4.32.0
{\anuvakamend[{त॒रणि॑र्विꣳश॒तिः}]}%॥31॥

%1.4.32.1
आ प्या॑यस्व मदिन्तम॒ सोम॒ विश्वा॑भिरू॒तिभिः॑। भवा॑ नः स॒प्रथ॑स्तमः॥~(३३)

%1.4.33.0
{\anuvakamend[{आ प्या॑यस्व॒ नव॑}]}%॥32॥

%1.4.33.1
ई॒युष्टे ये पूर्व॑तरा॒मप॑श्यन् व्यु॒च्छन्ती॑मु॒षसं॒ मर्त्या॑सः। अ॒स्माभि॑रू॒ नु प्र॑ति॒चक्ष्या॑\-ऽभू॒दो ते य॑न्ति॒ ये अ॑प॒रीषु॒ पश्यान्॑॥~(३४)

%1.4.34.0
{\anuvakamend[{ई॒युरेका॒न्नविꣳ॑शतिः}]}%॥33॥

%1.4.34.1
ज्योति॑ष्मतीं त्वा सादयामि ज्योति॒ष्कृतं॑ त्वा सादयामि ज्योति॒र्विदं॑ त्वा सादयामि॒ भास्व॑तीं त्वा सादयामि॒ ज्वल॑न्तीं त्वा सादयामि मल्मला॒भव॑न्तीं त्वा सादयामि॒ दीप्य॑मानां त्वा सादयामि॒ रोच॑मानां त्वा सादया॒म्यज॑स्रां त्वा सादयामि बृ॒हज्ज्यो॑तिषं त्वा सादयामि बो॒धय॑न्तीं त्वा सादयामि॒ जाग्र॑तीं त्वा सादयामि॥~(३५)

%1.4.35.0
{\anuvakamend[{ज्योति॑ष्मती॒ꣳ॒ षट्त्रिꣳ॑शत्}]}%॥34॥

%1.4.35.1
प्र॒या॒साय॒ स्वाहा॑\-ऽ\-ऽया॒साय॒ स्वाहा॑ विया॒साय॒ स्वाहा॑ संया॒साय॒ स्वाहो᳚द्या॒साय॒ स्वाहा॑\-ऽवया॒साय॒ स्वाहा॑ शु॒चे स्वाहा॒ शोका॑य॒ स्वाहा॑ तप्य॒त्वै स्वाहा॒ तप॑ते॒ स्वाहा᳚ ब्रह्मह॒त्यायै॒ स्वाहा॒ सर्व॑स्मै॒ स्वाहा᳚॥~(३६)

%1.4.36.0
{\anuvakamend[{प्र॒या॒साय॒ चतु॑र्विꣳशतिः}]}%॥35॥

%1.4.36.1
चि॒त्तꣳ स॑न्ता॒नेन॑ भ॒वं य॒क्ना रु॒द्रं तनि॑म्ना पशु॒पतिꣴ॑ स्थूलहृद॒येना॒ग्निꣳ हृद॑येन रु॒द्रं लोहि॑तेन श॒र्वं मत॑स्नाभ्यां महादे॒व\-म॒न्तःपा᳚र्श्वेनौषिष्ठ॒\-हनꣳ॑ शिङ्गीनिको॒श्या᳚भ्याम्॥~(३७)

%1.4.37.0
{\anuvakamend[{चि॒त्तम॒ष्टाद॑श}]}%॥36॥

%1.4.37.1
आ ति॑ष्ठ वृत्रह॒न् रथं॑ यु॒क्ता ते॒ ब्रह्म॑णा॒ हरी᳚। अ॒र्वा॒चीन॒ꣳ॒ सु ते॒ मनो॒ ग्रावा॑ कृणोतु व॒ग्नुना᳚॥ उ॒प॒या॒मगृ॑हीतो॒\-ऽसीन्द्रा॑य त्वा षोड॒शिन॑ ए॒ष ते॒ योनि॒रिन्द्रा॑य त्वा षोड॒शिने᳚॥~(३८)

%1.4.38.0
{\anuvakamend[{आ ति॑ष्ठ॒ षड्विꣳ॑शतिः}]}%॥37॥

%1.4.38.1
इन्द्र॒मिद्धरी॑ वह॒तो\-ऽप्र॑तिधृष्टशवस॒मृषी॑णां च स्तु॒तीरुप॑ य॒ज्ञं च॒ मानु॑षाणाम्॥ उ॒प॒या॒मगृ॑हीतो॒\-ऽसीन्द्रा॑य त्वा षोड॒शिन॑ ए॒ष ते॒ योनि॒रिन्द्रा॑य त्वा षोड॒शिने᳚॥~(३९)

%1.4.39.0
{\anuvakamend[{इन्द्र॒मित्त्रयो॑विꣳशतिः}]}%॥38॥

%1.4.39.1
असा॑वि॒ सोम॑ इन्द्र ते॒ शवि॑ष्ठ धृष्ण॒वा ग॑हि। आ त्वा॑ पृणक्त्विन्द्रि॒यꣳ रजः॒ सूर्यं॒ न र॒श्मिभिः॑॥ उ॒प॒या॒मगृ॑हीतो॒\-ऽसीन्द्रा॑य त्वा षोड॒शिन॑ ए॒ष ते॒ योनि॒रिन्द्रा॑य त्वा षोड॒शिने᳚॥~(४०)

%1.4.40.0
{\anuvakamend[{असा॑वि स॒प्तविꣳ॑शतिः}]}%॥39॥

%1.4.40.1
सर्व॑स्य प्रति॒शीव॑री॒ भूमि॑स्त्वो॒पस्थ॒ आ\-ऽधि॑त। स्यो॒नाऽस्मै॑ सु॒षदा॑ भव॒ यच्छा᳚स्मै॒ शर्म॑ स॒प्रथाः᳚॥ उ॒प॒या॒मगृ॑हीतो॒\-ऽसीन्द्रा॑य त्वा षोड॒शिन॑ ए॒ष ते॒ योनि॒रिन्द्रा॑य त्वा षोड॒शिने᳚॥~(४१)

%1.4.41.0
{\anuvakamend[{सर्व॑स्य॒ षड्विꣳ॑शतिः}]}%॥40॥

%1.4.41.1
म॒हाꣳ इन्द्रो॒ वज्र॑बाहुः षोड॒शी शर्म॑ यच्छतु। स्व॒स्ति नो॑ म॒घवा॑ करोतु॒ हन्तु॑ पा॒प्मानं॒ यो᳚\-ऽस्मान् द्वेष्टि॑॥ उ॒प॒या॒मगृ॑हीतो॒\-ऽसीन्द्रा॑य त्वा षोड॒शिन॑ ए॒ष ते॒ योनि॒रिन्द्रा॑य त्वा षोड॒शिने᳚॥~(४२)

%1.4.42.0
{\anuvakamend[{सर्व॑स्य म॒हान्थ्षड्विꣳ॑शतिः॒ षड्विꣳ॑शतिः}]}%॥41॥

%1.4.42.1
स॒जोषा॑ इन्द्र॒ सग॑णो म॒रुद्भिः॒ सोमं॑ पिब वृत्रहञ्छूर वि॒द्वान्। ज॒हि शत्रू॒ꣳ॒ रप॒ मृधो॑ नुद॒स्वा\-ऽथाभ॑यं कृणुहि वि॒श्वतो॑ नः॥ उ॒प॒या॒मगृ॑हीतो॒\-ऽसीन्द्रा॑य त्वा षोड॒शिन॑ ए॒ष ते॒ योनि॒रिन्द्रा॑य त्वा षोड॒शिने᳚॥~(४३)

%1.4.43.0
{\anuvakamend[{स॒जोषा᳚स्त्रि॒ꣳ॒शत्}]}%॥42॥

%1.4.43.1
उदु॒ त्यं जा॒तवे॑दसं दे॒वं व॑हन्ति के॒तवः॑। दृ॒शे विश्वा॑य॒ सूर्यम्᳚॥ चि॒त्रं दे॒वाना॒मुद॑गा॒दनी॑कं॒ चक्षु॑र्मि॒त्रस्य॒ वरु॑णस्या॒ग्नेः। आ\-ऽप्रा॒ द्यावा॑पृथि॒वी अ॒न्तरि॑क्ष॒ꣳ॒ सूर्य॑ आ॒त्मा जग॑तस्त॒स्थुष॑श्च॥ अग्ने॒ नय॑ सु॒पथा॑ रा॒ये अ॒स्मान् विश्वा॑नि देव व॒युना॑नि वि॒द्वान्। यु॒यो॒ध्य॑स्मज्जु॑हुरा॒णमेनो॒ भूयि॑ष्ठां ते॒ नम॑ उक्तिं विधेम॥ दिवं॑ गच्छ॒ सुवः॑ पत रू॒पेण॑~(४४)

%1.4.43.2
वो रू॒पम॒भ्यैमि॒ वय॑सा॒ वयः॑। तु॒थो वो॑ वि॒श्ववे॑दा॒ वि भ॑जतु॒ वर्\mbox{}षि॑ष्ठे॒ अधि॒ नाके᳚॥ ए॒तत् ते॑ अग्ने॒ राध॒ ऐति॒ सोम॑च्युतं॒ तन्मि॒त्रस्य॑ प॒था न॑य॒र्तस्य॑ प॒था प्रेत॑ च॒न्द्रद॑क्षिणा य॒ज्ञस्य॑ प॒था सु॑वि॒ता नय॑न्तीर्ब्राह्म॒णम॒द्य रा᳚ध्यास॒मृषि॑मार्\mbox{}षे॒यं पि॑तृ॒मन्तं॑ पैतृम॒त्यꣳ सु॒धातु॑दक्षिणं॒ वि सुवः॒ पश्य॒ व्य॑न्तरि॑क्षं॒ यत॑स्व सद॒स्यै॑र॒स्मद्दा᳚त्रा देव॒त्रा ग॑च्छत॒ मधु॑मतीः प्रदा॒तार॒मा वि॑श॒तान॑वहाया॒स्मान् दे॑व॒याने॑न प॒थेत॑ सु॒कृतां᳚ लो॒के सी॑दत॒ तन्नः॑ सꣴस्कृ॒तम्॥~(४५)

%1.4.44.0
{\anuvakamend[{रू॒पेण॑ सद॒स्यै॑र॒ष्टाद॑श च}]}%॥43~(३७)॥

%1.4.44.1
धा॒ता रा॒तिः स॑वि॒तेदं जु॑षन्तां प्र॒जा\-प॑तिर्निधि॒पति॑र्नो अ॒ग्निः। त्वष्टा॒ विष्णुः॑ प्र॒जया॑ सꣳररा॒णो यज॑मानाय॒ द्रवि॑णं दधातु॥ समि॑न्द्र णो॒ मन॑सा नेषि॒ गोभिः॒ सꣳ सू॒रिभि॑र्मघव॒न्थ्सꣴ स्व॒स्त्या। सं ब्रह्म॑णा दे॒वकृ॑तं॒ यदस्ति॒ सं दे॒वानाꣳ॑ सुम॒त्या य॒ज्ञिया॑नाम्॥ सं वर्च॑सा॒ पय॑सा॒ सं त॒नूभि॒रग॑न्महि॒ मन॑सा॒ सꣳ शि॒वेन॑। त्वष्टा॑ नो॒ अत्र॒ वरि॑वः कृणो॒-~(४६)

%1.4.44.2
त्वनु॑ मार्ष्टु त॒नुवो॒ यद्विलि॑ष्टम्॥ यद॒द्य त्वा᳚ प्रय॒ति य॒ज्ञे अ॒स्मिन्नग्ने॒ होता॑र॒मवृ॑णीमही॒ह। ऋध॑गया॒डृध॑गु॒ताश॑मिष्ठाः प्रजा॒नन् य॒ज्ञमुप॑याहि वि॒द्वान्॥ स्व॒गा वो॑ देवाः॒ सद॑नमकर्म॒ य आ॑ज॒ग्म सव॑ने॒दं जु॑षा॒णाः। ज॒क्षि॒वाꣳसः॑ पपि॒वाꣳस॑श्च॒ विश्वे॒\-ऽस्मे ध॑त्त वसवो॒ वसू॑नि॥ याना\-ऽव॑ह उश॒तो दे॑व दे॒वान्तान्~(४७)

%1.4.44.3
प्रेर॑य॒ स्वे अ॑ग्ने स॒धस्थे᳚। वह॑माना॒ भर॑माणा ह॒वीꣳषि॒ वसुं॑ घ॒र्मं दिव॒मा ति॑ष्ठ॒तानु॑॥ यज्ञ॑ य॒ज्ञं ग॑च्छ य॒ज्ञप॑तिं गच्छ॒ स्वां योनिं॑ गच्छ॒ स्वाहै॒ष ते॑ य॒ज्ञो य॑ज्ञपते स॒हसू᳚क्तवाकः सु॒वीरः॒ स्वाहा॒ देवा॑ गातुविदो गा॒तुं वि॒त्वा गा॒तुमि॑त॒ मन॑सस्पत इ॒मं नो॑ देव दे॒वेषु॑ य॒ज्ञꣴ स्वाहा॑ वा॒चि स्वाहा॒ वाते॑ धाः॥~(४८)

%1.4.45.0
{\anuvakamend[{कृ॒णो॒तु॒ तान॒ष्टाच॑त्वारिꣳशच्च}]}%॥44~(३८)॥

%1.4.45.1
उ॒रुꣳ हि राजा॒ वरु॑णश्च॒कार॒ सूर्या॑य॒ पन्था॒मन्वे॑त॒वा उ॑। अ॒पदे॒ पादा॒ प्रति॑धातवे\-ऽकरु॒ताप॑व॒क्ता हृ॑दया॒विध॑श्चित्॥ श॒तं ते॑ राजन् भि॒षजः॑ स॒हस्र॑मु॒र्वी ग॑म्भी॒रा सु॑म॒तिष्टे॑ अस्तु। बाध॑स्व॒ द्वेषो॒ निर्\mbox{}ऋ॑तिं परा॒चैः कृ॒तं चि॒देनः॒ प्र मु॑मुग्ध्य॒स्मत्॥ अ॒भिष्ठि॑तो॒ वरु॑णस्य॒ पाशो॒\-ऽग्नेरनी॑कम॒प आ वि॑वेश। अपां᳚ नपात् प्रति॒रक्ष॑न्नसु॒र्यं॑ दमे॑दमे~(४९)

%1.4.45.2
स॒मिधं॑ यक्ष्यग्ने॥ प्रति॑ ते जि॒ह्वा घृ॒तमुच्च॑रण्येथ्समु॒द्रे ते॒ हृद॑यम॒फ्स्व॑न्तः। सं त्वा॑ विश॒न्त्वोष॑धीरु॒ता\-ऽ\-ऽपो॑ य॒ज्ञस्य॑ त्वा यज्ञपते ह॒विर्भिः॑॥ सू॒क्त॒वा॒के न॑मोवा॒के वि॑धे॒माव॑भृथ निचङ्कुण निचे॒रुर॑सि निचङ्कु॒णाव॑ दे॒वैर्दे॒वकृ॑त॒मेनो॑\-ऽया॒डव॒ मर्त्यै॒र्मर्त्य॑कृतमु॒रोरा नो॑ देव रि॒षस्पा॑हि सुमि॒त्रा न॒ आप॒ ओष॑धयः~(५०)

%1.4.45.3
सन्तु दुर्मि॒त्रास्तस्मै॑ भूयासु॒र्यो᳚\-ऽस्मान् द्वेष्टि॒ यं च॑ व॒यं द्वि॒ष्मो देवी॑राप ए॒ष वो॒ गर्भ॒स्तं वः॒ सुप्री॑त॒ꣳ॒ सुभृ॑तमकर्म दे॒वेषु॑ नः सु॒कृतो᳚ ब्रूता॒त् प्रति॑युतो॒ वरु॑णस्य॒ पाशः॒ प्रत्य॑स्तो॒ वरु॑णस्य॒ पाश॒ एधो᳚\-ऽस्येधिषी॒महि॑ स॒मिद॑सि॒ तेजो॑\-ऽसि॒ तेजो॒ मयि॑ धेह्य॒पो अन्व॑चारिष॒ꣳ॒ रसे॑न॒ सम॑सृक्ष्महि। पय॑स्वाꣳ अग्न॒ आ\-ऽग॑मं॒ तं मा॒ सꣳ सृ॑ज॒ वर्च॑सा॥~(५१)

%1.4.46.0
{\anuvakamend[{दमे॑दम॒ ओष॑धय॒ आ षट् च॑}]}%॥45~(३९)॥

%1.4.46.1
यस्त्वा॑ हृ॒दा की॒रिणा॒ मन्य॑मा॒नो\-ऽम॑र्त्यं॒ मर्त्यो॒ जोह॑वीमि। जात॑वेदो॒ यशो॑ अ॒स्मासु॑ धेहि प्र॒जाभि॑रग्ने अमृत॒त्वम॑श्याम्॥ यस्मै॒ त्वꣳ सु॒कृते॑ जातवेद॒ उ लो॒कम॑ग्ने कृ॒णवः॑ स्यो॒नम्। अ॒श्विन॒ꣳ॒ स पु॒त्रिणं॑ वी॒रव॑न्तं॒ गोम॑न्तꣳ र॒यिं न॑शते स्व॒स्ति॥ त्वे सु पु॑त्र शव॒सो\-ऽवृ॑त्र॒न् काम॑कातयः। न त्वामि॒न्द्राति॑ रिच्यते॥ उ॒क्थउ॑क्थे॒ सोम॒ इन्द्रं॑ ममाद नी॒थेनी॑थे म॒घवा॑नꣳ~(५२)

%1.4.46.2
सु॒तासः॑। यदीꣳ॑ स॒बाधः॑ पि॒तरं॒ न पु॒त्राः स॑मा॒नद॑क्षा॒ अव॑से॒ हव॑न्ते॥ अग्ने॒ रसे॑न॒ तेज॑सा॒ जात॑वेदो॒ वि रो॑चसे। र॒क्षो॒हा\-ऽमी॑व॒चात॑नः॥ अ॒पो अन्व॑चारिष॒ꣳ॒ रसे॑न॒ सम॑सृक्ष्महि। पय॑स्वाꣳ अग्न॒ आ\-ऽग॑मं॒ तं मा॒ सꣳ सृ॑ज॒ वर्च॑सा॥ वसु॒र्वसु॑पति॒र्॒\mbox{}हिक॒मस्य॑ग्ने वि॒भाव॑सुः। स्याम॑ ते सुम॒तावपि॑॥ त्वाम॑ग्ने॒ वसु॑पतिं॒ वसू॑नाम॒भि प्र म॑न्दे~(५३)

%1.4.46.3
अध्व॒रेषु॑ राजन्न्। त्वया॒ वाजं॑ वाज॒यन्तो॑ जयेमा॒भि ष्या॑म पृथ्सु॒तीर्मर्त्या॑नाम्। त्वाम॑ग्ने वाज॒सात॑मं॒ विप्रा॑ वर्धन्ति॒ सुष्टु॑तम्। स नो॑ रास्व सु॒वीर्यम्᳚॥ अ॒यं नो॑ अ॒ग्निर्वरि॑वः कृणोत्व॒यं मृधः॑ पु॒र ए॑तु प्रभि॒न्दन्न्। अ॒यꣳ शत्रू᳚ञ्जयतु॒ जर्\mbox{}हृ॑षाणो॒\-ऽयं वाजं॑ जयतु॒ वाज॑सातौ॥ अ॒ग्निना॒ऽग्निः समि॑ध्यते क॒विर्गृ॒हप॑ति॒र्युवा᳚। ह॒व्य॒वाड् जु॒ह्वा᳚स्यः॥ त्वꣴ ह्य॑ग्ने अ॒ग्निना॒ विप्रो॒ विप्रे॑ण॒ सन्थ्स॒ता। सखा॒ सख्या॑ समि॒ध्यसे᳚॥ उद॑ग्ने॒ शुच॑य॒स्तव॒ वि ज्योति॑षा॥~(५४)

{\anuvakamend[{म॒घवा॑नं मन्दे॒ ह्य॑ग्ने॒ चतु॑र्दश च}]}%॥46॥
%%% END PRASHNA

\sect{पञ्चमः प्रश्नः}\setcounter{anuvakam}{0}
\dnsub{तैत्तिरीयसंहितायां प्रथमकाण्डे पञ्चमः प्रश्नः}
%1.5.1.0
%1.5.1.1
दे॒वा॒सु॒राः संय॑त्ता आस॒न् ते दे॒वा वि॑ज॒यमु॑प॒यन्तो॒\-ऽग्नौ वा॒मं वसु॒ सं न्य॑दधते॒दमु॑ नो भविष्यति॒ यदि॑ नो जे॒ष्यन्तीति॒ तद॒ग्निर्न्य॑कामयत॒ तेनापा᳚क्राम॒त् तद्दे॒वा वि॒जित्या॑व॒रुरु॑थ्समाना॒ अन्वा॑य॒न् तद॑स्य॒ सह॒सा\-ऽदि॑थ्सन्त॒ सो॑\-ऽरोदी॒द्यदरो॑दी॒त् तद्रु॒द्रस्य॑ रुद्र॒त्वं यदश्र्वशी॑यत॒ तद्~(१)

%1.5.1.2
र॑ज॒तꣳ हिर॑ण्यमभव॒त् तस्मा᳚द्रज॒तꣳ हिर॑ण्यमदक्षि॒ण्य\-म॑श्रु॒जꣳ हि यो ब॒र्॒\mbox{}हिषि॒ ददा॑ति पु॒रा\-ऽस्य॑ संवथ्स॒राद्गृ॒हे रु॑दन्ति॒ तस्मा᳚द्ब॒र्॒\mbox{}हिषि॒ न देय॒ꣳ॒ सो᳚\-ऽग्निर॑ब्रवीद्भा॒ग्य॑सा॒न्यथ॑ व इ॒दमिति॑ पुनरा॒धेयं॑ ते॒ केव॑ल॒मित्य॑ब्रुवन्नृ॒ध्नव॒त् खलु॒ स इत्य॑ब्रवी॒द्यो म॑द्देव॒त्य॑म॒ग्निमा॒दधा॑ता॒ इति॒ तं पू॒षा\-ऽ\-ऽध॑त्त॒ तेन॑~(२)

%1.5.1.3
पू॒षा\-ऽ\-ऽर्ध्नो॒त् तस्मा᳚त् पौ॒ष्णाः प॒शव॑ उच्यन्ते॒ तं त्वष्टा\-ऽ\-ऽध॑त्त॒ तेन॒ त्वष्टा᳚\-ऽ\-ऽर्ध्नो॒त् तस्मा᳚त् त्वा॒ष्ट्राः प॒शव॑ उच्यन्ते॒ तं मनु॒रा\-ऽध॑त्त॒ तेन॒ मनु॑रार्ध्नो॒त् तस्मा᳚न्मान॒व्यः॑ प्र॒जा उ॑च्यन्ते॒ तं धा॒ता\-ऽ\-ऽध॑त्त॒ तेन॑ धा॒ता\-ऽ\-ऽर्ध्नो᳚थ्संवथ्स॒रो वै धा॒ता तस्मा᳚थ्संवथ्स॒रं प्र॒जाः प॒शवो\-ऽनु॒ प्र जा॑यन्ते॒ य ए॒वं पु॑नरा॒धेय॒स्यर्द्धिं॒ वेद॒-~(३)

%1.5.1.4
र्ध्नोत्ये॒व यो᳚\-ऽस्यै॒वं ब॒न्धुतां॒ वेद॒ बन्धु॑मान् भवति भाग॒धेयं॒ वा अ॒ग्निराहि॑त इ॒च्छमा॑नः प्र॒जां प॒शून् यज॑मान॒स्योप॑ दोद्रावो॒द्वास्य॒ पुन॒रा द॑धीत भाग॒धेये॑नै॒वैन॒ꣳ॒ सम॑र्धय॒त्यथो॒ शान्ति॑रे॒वास्यै॒षा पुन॑र्वस्वो॒रा द॑धीतै॒तद्वै पु॑नरा॒धेय॑स्य॒ नक्ष॑त्रं॒ यत्पुन॑र्वसू॒ स्वाया॑मे॒वैनं॑ दे॒वता॑यामा॒धाय॑ ब्रह्मवर्च॒सी भ॑वति द॒र्भैरा द॑धा॒त्यया॑तयामत्वाय द॒र्भैरा द॑धात्य॒द्भ्य ए॒वैन॒मोष॑धीभ्यो\-ऽव॒रुध्या\-ऽ\-ऽध॑त्ते॒ पञ्च॑कपालः पुरो॒डाशो॑ भवति॒ पञ्च॒ वा ऋ॒तव॑ ऋ॒तुभ्य॑ ए॒वैन॑मव॒रुध्या\-ऽ\-ऽध॑त्ते॥~(४)

%1.5.2.0
{\anuvakamend[{अशी॑यत॒ तत् तेन॒ वेद॑ द॒र्भैः पञ्च॑विꣳशतिश्च}]}%~(१)

%1.5.2.1
परा॒ वा ए॒ष य॒ज्ञं प॒शून् व॑पति॒ यो᳚\-ऽग्निमु॑द्वा॒सय॑ते॒ पञ्च॑कपालः पुरो॒डाशो॑ भवति॒ पाङ्क्तो॑ य॒ज्ञः पाङ्क्ताः᳚ प॒शवो॑ य॒ज्ञमे॒व प॒शूनव॑ रुन्धे वीर॒हा वा ए॒ष दे॒वानां॒ यो᳚\-ऽग्निमु॑द्वा॒सय॑ते॒ न वा ए॒तस्य॑ ब्राह्म॒णा ऋ॑ता॒यवः॑ पु॒रा\-ऽन्न॑मक्षन् प॒ङ्क्त्यो॑ याज्यानुवा॒क्या॑ भवन्ति॒ पाङ्क्तो॑ य॒ज्ञः पाङ्क्तः॒ पुरु॑षो दे॒वाने॒व वी॒रं नि॑रव॒दाया॒ग्निं पुन॒रा~(५)

%1.5.2.2
ध॑त्ते श॒ताक्ष॑रा भवन्ति श॒तायुः॒ पुरु॑षः श॒तेन्द्रि॑य॒ आयु॑ष्ये॒वेन्द्रि॒ये प्रति॑ तिष्ठति॒ यद्वा अ॒ग्निराहि॑तो॒ नर्ध्यते॒ ज्यायो॑ भाग॒धेयं॑ निका॒मय॑मानो॒ यदा᳚ग्ने॒यꣳ सर्वं॒ भव॑ति॒ सैवास्यर्धिः॒ सं वा ए॒तस्य॑ गृ॒हे वाक् सृ॑ज्यते॒ यो᳚\-ऽग्निमु॑द्वा॒सय॑ते॒ स वाच॒ꣳ॒ सꣳसृ॑ष्टां॒ यज॑मान ईश्व॒रो\-ऽनु॒ परा॑भवितो॒र्विभ॑क्तयो भवन्ति वा॒चो विधृ॑त्यै॒ यज॑मान॒स्याप॑राभावाय॒~(६)

%1.5.2.3
विभ॑क्तिं करोति॒ ब्रह्मै॒व तद॑करुपा॒ꣳ॒शु य॑जति॒ यथा॑ वा॒मं वसु॑ विविदा॒नो गूह॑ति ता॒दृगे॒व तद॒ग्निं प्रति॑ स्विष्ट॒कृतं॒ निरा॑ह॒ यथा॑ वा॒मं वसु॑ विविदा॒नः प्र॑का॒शं जिग॑मिषति ता॒दृगे॒व तद्विभ॑क्तिमु॒क्त्वा प्र॑या॒जेन॒ वष॑ट्करोत्या॒यत॑नादे॒व नैति॒ यज॑मानो॒ वै पु॑रो॒डाशः॑ प॒शव॑ ए॒ते आहु॑ती॒ यद॒भितः॑ पुरो॒डाश॑मे॒ते आहु॑ती~(७)

%1.5.2.4
जु॒होति॒ यज॑मानमे॒वोभ॒यतः॑ प॒शुभिः॒ परि॑ गृह्णाति कृ॒तय॑जुः॒ सम्भृ॑तसम्भार॒ इत्या॑हु॒र्न स॒म्भृत्याः᳚ सम्भा॒रा न यजुः॑ कर्त॒व्य॑मित्यथो॒ खलु॑ स॒म्भृत्या॑ ए॒व स॑म्भा॒राः क॑र्त॒व्यं॑ यजु॑र्य॒ज्ञस्य॒ समृ॑द्ध्यै पुनर्निष्कृ॒तो रथो॒ दक्षि॑णा पुनरुथ्स्यू॒तं वासः॑ पुनरुथ्सृ॒ष्टो॑\-ऽन॒ड्वान् पु॑नरा॒धेय॑स्य॒ समृ॑द्ध्यै स॒प्त ते॑ अग्ने स॒मिधः॑ स॒प्त जि॒ह्वा इत्य॑ग्निहो॒त्रं जु॑होति॒ यत्र॑यत्रै॒वास्य॒ न्य॑क्तं॒ तत॑~(८)

%1.5.2.5
ए॒वैन॒मव॑ रुन्धे वीर॒हा वा ए॒ष दे॒वानां॒ यो᳚\-ऽग्निमु॑द्वा॒सय॑ते॒ तस्य॒ वरु॑ण ए॒वर्ण॒यादा᳚ग्निवारु॒णमेका॑\-दश\-कपाल॒मनु॒ निर्व॑पे॒द्यं चै॒व हन्ति॒ यश्चा᳚स्यर्ण॒यात्तौ भा॑ग॒धेये॑न प्रीणाति॒ ना\-ऽ\-ऽर्ति॒मार्च्छ॑ति॒ यज॑मानः॥~(९)

%1.5.3.0
{\anuvakamend[{आ\-ऽप॑राभावाय पुरो॒डाश॑मे॒ते आहु॑ती॒ ततः॒ षट्त्रिꣳ॑शच्च}]}%~(२)

%1.5.3.1
भूमि॑र्भू॒म्ना द्यौर्व॑रि॒णा\-ऽन्तरि॑क्षं महि॒त्वा। उ॒पस्थे॑ ते देव्यदिते॒\-ऽग्निम॑न्ना॒दम॒न्नाद्या॒या\-ऽ\-ऽद॑धे॥ आ\-ऽयं गौः पृश्ञि॑रक्रमी॒दस॑नन्मा॒तरं॒ पुनः॑। पि॒तरं॑ च प्र॒यन्थ्सुवः॑॥ त्रि॒ꣳ॒शद्धाम॒ वि रा॑जति॒ वाक्प॑त॒ङ्गाय॑ शिश्रिये। प्रत्य॑स्य वह॒ द्युभिः॑॥ अ॒स्य प्रा॒णाद॑पान॒त्य॑न्तश्च॑रति रोच॒ना। व्य॑ख्यन्महि॒षः सुवः॑॥ यत् त्वा᳚~(१०)

%1.5.3.2
क्रु॒द्धः प॑रो॒वप॑ म॒न्युना॒ यदव॑र्त्या। सु॒कल्प॑मग्ने॒ तत् तव॒ पुन॒स्त्वोद्दी॑पयामसि॥ यत् ते॑ म॒न्युप॑रोप्तस्य पृथि॒वीमनु॑ दध्व॒से। आ॒दि॒त्या विश्वे॒ तद्दे॒वा वस॑वश्च स॒माभ॑रन्न्॥ मनो॒ ज्योति॑र्जुषता॒माज्यं॒ विच्छि॑न्नं य॒ज्ञꣳ समि॒मं द॑धातु। बृह॒स्पति॑स्तनुतामि॒मं नो॒ विश्वे॑ दे॒वा इ॒ह मा॑दयन्ताम्॥ स॒प्त ते॑ अग्ने स॒मिधः॑ स॒प्त जि॒ह्वाः स॒प्त~(११)

%1.5.3.3
ऋष॑यः स॒प्त धाम॑ प्रि॒याणि॑। स॒प्त होत्राः᳚ सप्त॒धा त्वा॑ यजन्ति स॒प्त योनी॒रा पृ॑णस्वा घृ॒तेन॑॥ पुन॑रू॒र्जा नि व॑र्तस्व॒ पुन॑रग्न इ॒षा\-ऽ\-ऽयु॑षा। पुन॑र्नः पाहि वि॒श्वतः॑॥ स॒ह र॒य्या नि व॑र्त॒स्वाग्ने॒ पिन्व॑स्व॒ धार॑या। वि॒श्वफ्स्नि॑या वि॒श्वत॒स्परि॑॥ लेकः॒ सले॑कः सु॒लेक॒स्ते न॑ आदि॒त्या आज्यं॑ जुषा॒णा वि॑यन्तु॒ केतः॒ सके॑तः सु॒केत॒स्ते न॑ आदि॒त्या आज्यं॑ जुषा॒णा वि॑यन्तु॒ विव॑स्वा॒ꣳ॒ अदि॑ति॒र्देव॑जूति॒स्ते न॑ आदि॒त्या आज्यं॑ जुषा॒णा वि॑यन्तु॥~(१२)

%1.5.4.0
{\anuvakamend[{त्वा॒ जि॒ह्वाः स॒प्त सु॒केत॒स्ते न॒स्त्रयो॑दश च}]}%~(३)

%1.5.4.1
भूमि॑र्भू॒म्ना द्यौर्व॑रि॒णेत्या॑हा॒\-ऽ\-ऽशिषै॒वैन॒मा ध॑त्ते स॒र्पा वै जीर्य॑न्तो\-ऽमन्यन्त॒ स ए॒तं क॑स॒र्णीरः॑ काद्रवे॒यो मन्त्र॑मपश्य॒त् ततो॒ वै ते जी॒र्णास्त॒नूरपा᳚घ्नत सर्परा॒ज्ञिया॑ ऋ॒ग्भिर्गार्\mbox{}ह॑पत्य॒मा द॑धाति पुनर्न॒वमे॒वैन॑म॒जरं॑ कृ॒त्वा\-ऽ\-ऽध॒त्ते\-ऽथो॑ पू॒तमे॒व पृ॑थि॒वीम॒न्नाद्यं॒ नोपा॑नम॒थ्सैतं~(१३)

%1.5.4.2
मन्त्र॑मपश्य॒त् ततो॒ वै ताम॒न्नाद्य॒मुपा॑\-नम॒द्यथ्स॑र्परा॒ज्ञिया॑ ऋ॒ग्भिर्गार्\mbox{}ह॑पत्यमा॒दधा᳚त्य॒न्नाद्य॒स्या\-व॑रुद्ध्या॒ अथो॑ अ॒स्यामे॒वैनं॒ प्रति॑ष्ठित॒मा ध॑त्ते॒ यत्त्वा᳚ क्रु॒द्धः प॑रो॒वपेत्या॒हाप॑ह्नुत ए॒वास्मै॒ तत् पुन॒स्त्वोद्दी॑पयाम॒सीत्या॑ह॒ समि॑न्ध ए॒वैनं॒ यत्ते॑ म॒न्युप॑रोप्त॒स्येत्या॑ह दे॒वता॑भिरे॒-~(१४)

%1.5.4.3
वैन॒ꣳ॒ सं भ॑रति॒ वि वा ए॒तस्य॑ य॒ज्ञश्छि॑द्यते॒ यो᳚\-ऽग्निमु॑द्वा॒सय॑ते॒ बृह॒स्पति॑वत्य॒र्चोप॑ तिष्ठते॒ ब्रह्म॒ वै दे॒वानां॒ बृह॒स्पति॒र्ब्रह्म॑णै॒व य॒ज्ञꣳ सं द॑धाति॒ विच्छि॑न्नं य॒ज्ञꣳ समि॒मं द॑धा॒त्वित्या॑ह॒ सन्त॑त्यै॒ विश्वे॑ दे॒वा इ॒ह मा॑दयन्ता॒मित्या॑ह स॒न्तत्यै॒व य॒ज्ञं दे॒वेभ्यो\-ऽनु॑ दिशति स॒प्त ते॑ अग्ने स॒मिधः॑ स॒प्त जि॒ह्वा~-~(१५)

%1.5.4.4
इत्या॑ह स॒प्तस॑प्त॒ वै स॑प्त॒धा\-ऽग्नेः प्रि॒यास्त॒नुव॒स्ता ए॒वाव॑ रुन्धे॒ पुन॑रू॒र्जा स॒ह र॒य्येत्य॒भितः॑ पुरो॒डाश॒माहु॑ती जुहोति॒ यज॑मानमे॒वोर्जा च॑ र॒य्या चो॑भ॒यतः॒ परि॑ गृह्णात्यादि॒त्या वा अ॒स्माल्लो॒काद॒मुं लो॒कमा॑य॒न्ते॑\-ऽमुष्मिँ॑ल्लो॒के व्य॑तृष्य॒न्त इ॒मं लो॒कं पुन॑रभ्य॒\-वेत्या॒ग्नि\-मा॒धायै॒तान् होमा॑नजुहवु॒स्त आ᳚र्ध्नुव॒न् ते सु॑व॒र्गं लो॒कमा॑य॒न्॒ यः प॑रा॒चीनं॑ पुनरा॒धेया॑द॒ग्निमा॒दधी॑त॒ स ए॒तान् होमा᳚ञ्जुहुया॒द्यामे॒वा\-ऽ\-ऽदि॒त्या ऋद्धि॒मार्ध्नु॑व॒न् तामे॒वर्ध्नो॑ति॥~(१६)

%1.5.5.0
{\anuvakamend[{ए॒तमे॒व जि॒ह्वा ए॒तान् पञ्च॑विꣳशतिश्च}]}%~(४)

%1.5.5.1
उ॒प॒प्र॒यन्तो॑ अध्व॒रं मन्त्रं॑ वोचेमा॒ग्नये᳚। आ॒रे अ॒स्मे च॑ शृण्व॒ते॥ अ॒स्य प्र॒त्नामनु॒ द्युतꣳ॑ शु॒क्रं दु॑दुह्रे॒ अह्र॑यः। पयः॑ सहस्र॒सामृषिम्᳚॥ अ॒ग्निर्मू॒र्धा दि॒वः क॒कुत् पतिः॑ पृथि॒व्या अ॒यम्। अ॒पाꣳ रेताꣳ॑सि जिन्वति॥ अ॒यमि॒ह प्र॑थ॒मो धा॑यि धा॒तृभि॒र्॒\mbox{}होता॒ यजि॑ष्ठो अध्व॒रेष्वीड्यः॑। यमप्न॑वानो॒ भृग॑वो विरुरु॒चुर्वने॑षु चि॒त्रं वि॒भुवं॑ वि॒शेवि॑शे॥ उ॒भा वा॑मिन्द्राग्नी आहु॒वध्या॑~(१७)

%1.5.5.2
उ॒भा राध॑सः स॒ह मा॑द॒यध्यै᳚। उ॒भा दा॒तारा॑वि॒षाꣳ र॑यी॒णामु॒भा वाज॑स्य सा॒तये॑ हुवे वाम्॥ अ॒यं ते॒ योनि॑र्\mbox{}ऋ॒त्वियो॒ यतो॑ जा॒तो अरो॑चथाः। तं जा॒नन्न॑ग्न॒ आ रो॒हाथा॑ नो वर्धया र॒यिम्॥ अग्न॒ आयूꣳ॑षि पवस॒ आ सु॒वोर्ज॒मिषं॑ च नः। आ॒रे बा॑धस्व दु॒च्छुना᳚म्॥ अग्ने॒ पव॑स्व॒ स्वपा॑ अ॒स्मे वर्चः॑ सु॒वीर्यम्᳚। दध॒त्पोषꣳ॑ र॒यिं~(१८)

%1.5.5.3
मयि॑॥ अग्ने॑ पावक रो॒चिषा॑ म॒न्द्रया॑ देव जि॒ह्वया᳚। आ दे॒वान् व॑क्षि॒ यक्षि॑ च॥ स नः॑ पावक दीदि॒वो\-ऽग्ने॑ दे॒वाꣳ इ॒हा\-ऽ\-ऽव॑ह। उप॑ य॒ज्ञꣳ ह॒विश्च॑ नः॥ अ॒ग्निः शुचि॑व्रततमः॒ शुचि॒र्विप्रः॒ शुचिः॑ क॒विः। शुची॑ रोचत॒ आहु॑तः॥ उद॑ग्ने॒ शुच॑य॒स्तव॑ शु॒क्रा भ्राज॑न्त ईरते। तव॒ ज्योतीꣴ॑ष्य॒र्चयः॑॥ आ॒यु॒र्दा अ॑ग्ने॒\-ऽस्यायु॑र्मे~(१९)

%1.5.5.4
देहि वर्चो॒दा अ॑ग्ने\-ऽसि॒ वर्चो॑ मे देहि तनू॒पा अ॑ग्ने\-ऽसि त॒नुवं॑ मे पा॒ह्यग्ने॒ यन्मे॑ त॒नुवा॑ ऊ॒नं तन्म॒ आ पृ॑ण॒ चित्रा॑वसो स्व॒स्ति ते॑ पा॒रम॑शी॒येन्धा॑नास्त्वा श॒तꣳ हिमा᳚ द्यु॒मन्तः॒ समि॑धीमहि॒ वय॑स्वन्तो वय॒स्कृतं॒ यश॑स्वन्तो यश॒स्कृतꣳ॑ सु॒वीरा॑सो॒ अदा᳚भ्यम्। अग्ने॑ सपत्न॒दम्भ॑नं॒ वर्\mbox{}षि॑ष्ठे॒ अधि॒ नाके᳚॥ सं त्वम॑ग्ने॒ सूर्य॑स्य॒ वर्च॑सा\-ऽगथाः॒ समृषी॑णाꣴ स्तु॒तेन॒ सं प्रि॒येण॒ धाम्ना᳚। त्वम॑ग्ने॒ सूर्य॑वर्चा असि॒ सं मामायु॑षा॒ वर्च॑सा प्र॒जया॑ सृज॥~(२०)

%1.5.6.0
{\anuvakamend[{आ॒हु॒वध्यै॑ र॒यिं मे॒ वर्च॑सा स॒प्तद॑श च}]}%~(५)

%1.5.6.1
सं प॑श्यामि प्र॒जा अ॒हमिड॑प्रजसो मान॒वीः। सर्वा॑ भवन्तु नो गृ॒हे॥ अम्भः॒ स्थाम्भो॑ वो भक्षीय॒ महः॑ स्थ॒ महो॑ वो भक्षीय॒ सहः॑ स्थ॒ सहो॑ वो भक्षी॒योर्जः॒ स्थोर्जं॑ वो भक्षीय॒ रेव॑ती॒ रम॑ध्वम॒स्मिँल्लो॒के᳚\-ऽस्मिन् गो॒ष्ठे᳚\-ऽस्मिन् क्षये॒\-ऽस्मिन् योना॑वि॒हैव स्ते॒तो मा\-ऽप॑ गात ब॒ह्वीर्मे॑ भूयास्त~(२१)

%1.5.6.2
सꣳहि॒तासि॑ विश्वरू॒पीरा मो॒र्जा वि॒शा\-ऽ\-ऽगौ॑प॒त्येना\-ऽ\-ऽरा॒यस्पोषे॑ण सहस्रपो॒षं वः॑ पुष्यासं॒ मयि॑ वो॒ रायः॑ श्रयन्ताम्॥ उप॑ त्वा\-ऽग्ने दि॒वेदि॑वे॒ दोषा॑वस्तर्धि॒या व॒यम्। नमो॒ भर॑न्त॒ एम॑सि। राज॑न्तमध्व॒राणां᳚ गो॒पामृ॒तस्य॒ दीदि॑विम्। वर्ध॑मान॒ꣴ॒ स्वे दमे᳚॥ स नः॑ पि॒तेव॑ सू॒नवे\-ऽग्ने॑ सूपाय॒नो भ॑व। सच॑स्वा नः स्व॒स्तये᳚॥ अग्ने॒~(२२)

%1.5.6.3
त्वं नो॒ अन्त॑मः। उ॒त त्रा॒ता शि॒वो भ॑व वरू॒थ्यः॑॥ तं त्वा॑ शोचिष्ठ दीदिवः। सु॒म्नाय॑ नू॒नमी॑महे॒ सखि॑भ्यः॥ वसु॑र॒ग्निर्वसु॑श्रवाः। अच्छा॑ नक्षि द्यु॒मत्त॑मो र॒यिं दाः᳚॥ ऊ॒र्जा वः॑ पश्याम्यू॒र्जा मा॑ पश्यत रा॒यस्पोषे॑ण वः पश्यामि रा॒यस्पोषे॑ण मा पश्य॒तेडाः᳚ स्थ मधु॒कृतः॑ स्यो॒ना मा\-ऽ\-ऽवि॑श॒तेरा॒ मदः॑। स॒ह॒स्र॒पो॒षं वः॑ पुष्यासं॒~(२३)

%1.5.6.4
मयि॑ वो॒ रायः॑ श्रयन्ताम्॥ तथ्स॑वि॒तुर्वरे᳚ण्यं॒ भर्गो॑ दे॒वस्य॑ धीमहि। धियो॒ यो नः॑ प्रचो॒दया᳚त्॥ सो॒मान॒ꣴ॒ स्वर॑णं कृणु॒हि ब्र॑ह्मणस्पते। क॒क्षीव॑न्तं॒ य औ॑शि॒जम्॥ क॒दा च॒न स्त॒रीर॑सि॒ नेन्द्र॑ सश्चसि दा॒शुषे᳚। उपो॒पेन्नु म॑घव॒न् भूय॒ इन्नु ते॒ दानं॑ दे॒वस्य॑ पृच्यते॥ परि॑ त्वाऽग्ने॒ पुरं॑ व॒यं विप्रꣳ॑ सहस्य धीमहि। धृ॒षद्व॑र्णं दि॒वेदि॑वे भे॒त्तारं॑ भङ्गु॒राव॑तः॥ अग्ने॑ गृहपते सुगृहप॒तिर॒हं त्वया॑ गृ॒हप॑तिना भूयासꣳ सुगृहप॒तिर्मया॒ त्वं गृ॒हप॑तिना भूयाः श॒तꣳ हिमा॒स्तामा॒शिष॒मा शा॑से॒ तन्त॑वे॒ ज्योति॑ष्मतीं॒ तामा॒शिष॒मा शा॑से॒\-ऽमुष्मै॒ ज्योति॑ष्मतीम्॥~(२४)

%1.5.7.0
{\anuvakamend[{भू॒या॒स्त॒ स्व॒स्तये\-ऽग्ने॑ पुष्यासं धृ॒षद्व॑र्ण॒मेका॒न्न\-त्रि॒ꣳ॒शच्च॑}]}%~(६)

%1.5.7.1
अय॑ज्ञो॒ वा ए॒ष यो॑\-ऽसा॒मोप॑प्र॒यन्तो॑ अध्व॒रमित्या॑ह॒ स्तोम॑मे॒वास्मै॑ युन॒क्त्युपेत्या॑ह प्र॒जा वै प॒शव॒ उपे॒मं लो॒कं प्र॒जामे॒व प॒शूनि॒मं लो॒कमुपै᳚त्य॒स्य प्र॒त्नामनु॒द्युत॒मित्या॑ह सुव॒र्गो वै लो॒कः प्र॒त्नः सु॑व॒र्गमे॒व लो॒कꣳ स॒मारो॑हत्य॒ग्निर्मू॒र्धा दि॒वः क॒कुदित्या॑ह मू॒र्धान॑-~(२५)

%1.5.7.2
मे॒वैनꣳ॑ समा॒नानां᳚ करो॒त्यथो॑ देवलो॒कादे॒व म॑नुष्यलो॒के प्रति॑तिष्ठत्य॒यमि॒ह प्र॑थ॒मो धा॑यि धा॒तृभि॒रित्या॑ह॒ मुख्य॑मे॒वैनं॑ करोत्यु॒भा वा॑मिन्द्राग्नी आहु॒वध्या॒ इत्या॒हौजो॒ बल॑मे॒वाव॑ रुन्धे॒\-ऽयं ते॒ योनि॑र्\mbox{}ऋ॒त्विय॒ इत्या॑ह प॒शवो॒ वै र॒यिः प॒शूने॒वाव॑ रुन्धे ष॒ड्भिरुप॑ तिष्ठते॒ षड्वा~(२६)

%1.5.7.3
ऋ॒तव॑ ऋ॒तुष्वे॒व प्रति॑ तिष्ठति ष॒ड्भिरुत्त॑राभि॒रुप॑ तिष्ठते॒ द्वाद॑श॒ सं प॑द्यन्ते॒ द्वाद॑श॒ मासाः᳚ संवथ्स॒रः सं॑वथ्स॒र ए॒व प्रति॑ तिष्ठति॒ यथा॒ वै पुरु॒षो\-ऽश्वो॒ गौर्जीर्य॑त्ये॒वम॒ग्निराहि॑तो जीर्यति संवथ्स॒रस्य॑ प॒रस्ता॑दाग्निपावमा॒नीभि॒रुप॑ तिष्ठते पुनर्न॒वमे॒वैन॑म॒जरं॑ करो॒त्यथो॑ पु॒नात्ये॒वोप॑ तिष्ठते॒ योग॑ ए॒वास्यै॒ष उप॑ तिष्ठते॒~(२७)

%1.5.7.4
दम॑ ए॒वास्यै॒ष उप॑ तिष्ठते या॒च्ञैवास्यै॒षोप॑ तिष्ठते॒ यथा॒ पापी॑या॒ञ्छ्रेय॑स आ॒हृत्य॑ नम॒स्यति॑ ता॒दृगे॒व तदा॑यु॒र्दा अ॑ग्ने॒\-ऽस्यायु॑र्मे दे॒हीत्या॑हा\-ऽ\-ऽयु॒र्दा ह्ये॑ष व॑र्चो॒दा अ॑ग्ने\-ऽसि॒ वर्चो॑ मे दे॒हीत्या॑ह वर्चो॒दा ह्ये॑ष त॑नू॒पा अ॑ग्ने\-ऽसि त॒नुवं॑ मे पा॒हीत्या॑ह~(२८)

%1.5.7.5
तनू॒पा ह्ये॑षो\-ऽग्ने॒ यन्मे॑ त॒नुवा॑ ऊ॒नं तन्म॒ आ पृ॒णेत्या॑ह॒ यन्मे᳚ प्र॒जायै॑ पशू॒नामू॒नं तन्म॒ आ पू॑र॒येति॒ वावैतदा॑ह॒ चित्रा॑वसो स्व॒स्ति ते॑ पा॒रम॑शी॒येत्या॑ह॒ रात्रि॒र्वै चि॒त्राव॑सु॒रव्यु॑ष्ट्यै॒ वा ए॒तस्यै॑ पु॒रा ब्रा᳚ह्म॒णा अ॑भैषु॒र्व्यु॑ष्टिमे॒वाव॑ रुन्ध॒ इन्धा॑नास्त्वा श॒तꣳ~(२९)

%1.5.7.6
हिमा॒ इत्या॑ह श॒तायुः॒ पुरु॑षः श॒तेन्द्रि॑य॒ आयु॑ष्ये॒वेन्द्रि॒ये प्रति॑ तिष्ठत्ये॒षा वै सू॒र्मी कर्ण॑कावत्ये॒तया॑ ह स्म॒ वै दे॒वा असु॑राणाꣳ शतत॒र्॒\mbox{}हाꣴ स्तृꣳ॑हन्ति॒ यदे॒तया॑ स॒मिध॑मा॒दधा॑ति॒ वज्र॑मे॒वैतच्छ॑त॒घ्नीं यज॑मानो॒ भ्रातृ॑व्याय॒ प्रह॑रति॒ स्तृत्या॒ अछ॑म्बट्कार॒ꣳ॒ सं त्वम॑ग्ने॒ सूर्य॑स्य॒ वर्च॑सा गथा॒ इत्या॑है॒तत्त्वमसी॒दम॒हं भू॑यास॒मिति॒ वावैतदा॑ह॒ त्वम॑ग्ने॒ सूर्य॑वर्चा अ॒सीत्या॑हा॒\-ऽ\-ऽशिष॑मे॒वैतामा शा᳚स्ते॥~(३०)

%1.5.8.0
{\anuvakamend[{मू॒र्धानं॒ वै तिष्ठ॑त आह श॒तम॒हꣳ षोड॑श च}]}%~(७)

%1.5.8.1
सं प॑श्यामि प्र॒जा अ॒हमित्या॑ह॒ याव॑न्त ए॒व ग्रा॒म्याः प॒शव॒स्ताने॒वाव॑ रु॒न्धे\-ऽम्भः॒ स्थाम्भो॑ वो भक्षी॒येत्या॒हाम्भो॒ ह्ये॑ता महः॑ स्थ॒ महो॑ वो भक्षी॒येत्या॑ह॒ महो॒ ह्ये॑ताः सहः॑ स्थ॒ सहो॑ वो भक्षी॒येत्या॑ह॒ सहो॒ ह्ये॑ता ऊर्ज॒स्थोर्जं॑ वो भक्षी॒ये-~(३१)

%1.5.8.2
त्या॒होर्जो॒ ह्ये॑ता रेव॑ती॒ रम॑ध्व॒मित्या॑ह प॒शवो॒ वै रे॒वतीः᳚ प॒शूने॒वात्मन् र॑मयत इ॒हैव स्ते॒तो मा\-ऽप॑ गा॒तेत्या॑ह ध्रु॒वा ए॒वैना॒ अन॑पगाः कुरुत इष्टक॒चिद्वा अ॒न्यो᳚\-ऽग्निः प॑शु॒चिद॒न्यः सꣳ॑हि॒तासि॑ विश्वरू॒पीरिति॑ व॒थ्सम॒भि मृ॑श॒त्युपै॒वैनं॑ धत्ते पशु॒चित॑मेनं कुरुते॒ प्र~(३२)

%1.5.8.3
वा ए॒षो᳚\-ऽस्माल्लो॒काच्च्य॑वते॒ य आ॑हव॒नीय॑मुप॒तिष्ठ॑ते॒ गार्\mbox{}ह॑पत्य॒मुप॑ तिष्ठते॒\-ऽस्मिन्ने॒व लो॒के प्रति॑ तिष्ठ॒त्यथो॒ गार्\mbox{}ह॑पत्यायै॒व नि ह्नु॑ते गाय॒त्रीभि॒रुप॑ तिष्ठते॒ तेजो॒ वै गा॑य॒त्री तेज॑ ए॒वात्मन् ध॒त्ते\-ऽथो॒ यदे॒तं तृ॒चम॒न्वाह॒ सन्त॑त्यै॒ गार्\mbox{}ह॑पत्यं॒ वा अनु॑ द्वि॒पादो॑ वी॒राः प्रजा॑यन्ते॒ य ए॒वं वि॒द्वान् द्वि॒पदा॑भि॒र्गार्\mbox{}ह॑पत्यमुप॒तिष्ठ॑त॒~-~(३३)

%1.5.8.4
आ\-ऽस्य॑ वी॒रो जा॑यत ऊ॒र्जा वः॑ पश्याम्यू॒र्जा मा॑ पश्य॒तेत्या॑हा॒\-ऽ\-ऽशिष॑मे॒वैतामा शा᳚स्ते॒ तथ्स॑वि॒तुर्वरे᳚ण्य॒मित्या॑ह॒ प्रसू᳚त्यै सो॒मान॒ꣴ॒ स्वर॑ण॒मित्या॑ह सोमपी॒थमे॒वाव॑ रुन्धे कृणु॒हि ब्र॑ह्मणस्पत॒ इत्या॑ह ब्रह्मवर्च॒समे॒वाव॑ रुन्धे क॒दा च॒न स्त॒रीर॒सीत्या॑ह॒ न स्त॒रीꣳ रात्रिं॑ वसति॒~(३४)

%1.5.8.5
य ए॒वं वि॒द्वान॒ग्निमु॑प॒तिष्ठ॑ते॒ परि॑ त्वाऽग्ने॒ पुरं॑ व॒यमित्या॑ह परि॒धिमे॒वैतं परि॑ दधा॒त्यस्क॑न्दा॒याग्ने॑ गृहपत॒ इत्या॑ह यथाय॒जुरे॒वैतच्छ॒तꣳ हिमा॒ इत्या॑ह श॒तं त्वा॑ हेम॒न्तानि॑न्धिषी॒\-येति॒ वावैतदा॑ह पु॒त्रस्य॒ नाम॑ गृह्णात्यन्ना॒दमे॒वैनं॑ करोति॒ तामा॒\-शिष॒मा शा॑से॒ तन्त॑वे॒ ज्योति॑ष्मती॒मिति॑ ब्रूया॒द्यस्य॑ पु॒त्रो\-ऽजा॑तः॒ स्यात् ते॑ज॒स्व्ये॑वास्य॑ ब्रह्मवर्च॒सी पु॒त्रो जा॑यते॒ तामा॒शिष॒मा शा॑से॒\-ऽमुष्मै॒ ज्योति॑ष्मती॒मिति॑ ब्रूया॒द्यस्य॑ पु॒त्रो जा॒तः स्यात् तेज॑ ए॒वास्मि॑न् ब्रह्मवर्च॒सं द॑धाति॥~(३५)

%1.5.9.0
{\anuvakamend[{ऊर्जं॑ वो भक्षी॒येति॒ प्र गार्\mbox{}ह॑पत्यमुप॒तिष्ठ॑ते वसति॒ ज्योति॑ष्मती॒मेका॒न्न\-त्रि॒ꣳ॒शच्च॑}]}%~(८)

%1.5.9.1
अ॒ग्नि॒हो॒त्रं जु॑होति॒ यदे॒व किं च॒ यज॑मानस्य॒ स्वं तस्यै॒व तद्रेतः॑ सिञ्चति प्र॒जन॑ने प्र॒जन॑न॒ꣳ॒ हि वा अ॒ग्निरथौष॑धी॒रन्त॑गता दहति॒ तास्ततो॒ भूय॑सीः॒ प्रजा॑यन्ते॒ यथ्सा॒यं जु॒होति॒ रेत॑ ए॒व तथ्सि॑ञ्चति॒ प्रैव प्रा॑त॒स्तने॑न जनयति॒ तद्रेतः॑ सि॒क्तं न त्वष्ट्रा\-ऽवि॑कृतं॒ प्रजा॑यते याव॒च्छो वै रेत॑सः सि॒क्तस्य॒~(३६)

%1.5.9.2
त्वष्टा॑ रू॒पाणि॑ विक॒रोति॑ ताव॒च्छो वै तत्प्रजा॑यत ए॒ष वै दैव्य॒स्त्वष्टा॒ यो यज॑ते ब॒ह्वीभि॒रुप॑ तिष्ठते॒ रेत॑स ए॒व सि॒क्तस्य॑ बहु॒शो रू॒पाणि॒ वि क॑रोति॒ स प्रैव जा॑यते॒ श्वःश्वो॒ भूया᳚न् भवति॒ य ए॒वं वि॒द्वान॒ग्निमु॑प॒तिष्ठ॒ते\-ऽह॑र्दे॒वाना॒\-मासी॒द्रात्रि॒रसु॑राणां॒ ते\-ऽसु॑रा॒ यद्दे॒वानां᳚ वि॒त्तं वेद्य॒मासी॒त्तेन॑ स॒ह~(३७)

%1.5.9.3
रात्रिं॒ प्रावि॑श॒न् ते दे॒वा ही॒ना अ॑मन्यन्त॒ ते॑\-ऽपश्यन्नाग्ने॒यी रात्रि॑राग्ने॒याः प॒शव॑ इ॒ममे॒वाग्निꣴ स्त॑वाम॒ स नः॑ स्तु॒तः प॒शून् पुन॑र्दास्य॒तीति॒ ते᳚\-ऽग्निम॑स्तुव॒न्थ्स ए᳚भ्यः स्तु॒तो रात्रि॑या॒ अध्यह॑र॒भि प॒शून्निरा᳚र्ज॒त् ते दे॒वाः प॒शून् वि॒त्वा कामाꣳ॑ अकुर्वत॒ य ए॒वं वि॒द्वान॒ग्निमु॑प॒तिष्ठ॑ते पशु॒मान् भ॑वत्या-~(३८)

%1.5.9.4
दि॒त्यो वा अ॒स्माल्लो॒काद॒मुं लो॒कमै॒थ्सो॑\-ऽमुं लो॒कं ग॒त्वा पुन॑रि॒मं लो॒कम॒भ्य॑ध्याय॒थ्स इ॒मं लो॒कमा॒गत्य॑ मृ॒त्योर॑बिभेन्मृ॒त्युसं॑युत इव॒ ह्य॑यं लो॒कः सो॑\-ऽमन्यते॒ममे॒वाग्निꣴ स्त॑वानि॒ स मा᳚ स्तु॒तः सु॑व॒र्गं लो॒कं ग॑मयिष्य॒तीति॒ सो᳚\-ऽग्निम॑स्तौ॒थ्स ए॑नꣴ स्तु॒तः सु॑व॒र्गं लो॒कम॑गमय॒द्य~-~(३९)

%1.5.9.5
ए॒वं वि॒द्वान॒ग्निमु॑प॒तिष्ठ॑ते सुव॒र्गमे॒व लो॒कमे॑ति॒ सर्व॒मायु॑रेत्य॒भि वा ए॒षो᳚\-ऽग्नी आ रो॑हति॒ य ए॑नावुप॒तिष्ठ॑ते॒ यथा॒ खलु॒ वै श्रेया॑न॒भ्यारू॑ढः का॒मय॑ते॒ तथा॑ करोति॒ नक्त॒मुप॑ तिष्ठते॒ न प्रा॒तः सꣳ हि नक्तं॑ व्र॒तानि॑ सृ॒ज्यन्ते॑ स॒ह श्रेयाꣴ॑श्च॒ पापी॑याꣴश्चासाते॒ ज्योति॒र्वा अ॒ग्निस्तमो॒ रात्रि॒र्यन्-~(४०)

%1.5.9.6
नक्त॑मुप॒तिष्ठ॑ते॒ ज्योति॑षै॒व तम॑स्तरत्युप॒स्थेयो॒\-ऽग्नी(३)र्नोप॒\-स्थेया(३) इत्या॑हुर्मनु॒ष्या॑येन्न्वै यो\-ऽह॑रहरा॒हृत्या\-थै॑नं॒ याच॑ति॒ स इन्न्वै तमुपा᳚र्च्छ॒त्यथ॒ को दे॒वानह॑रहर्याचिष्य॒तीति॒ तस्मा॒न्नोप॒स्थेयो\-ऽथो॒ खल्वा॑हुरा॒शिषे॒ वै कं यज॑मानो यजत॒ इत्ये॒षा खलु॒ वा~(४१)

%1.5.9.7
आहि॑ताग्नेरा॒शीर्यद॒ग्नि\-मु॑प॒तिष्ठ॑ते॒ तस्मा॑दुप॒स्थेयः॑ प्र॒जा\-प॑तिः प॒शून॑सृजत॒ ते सृ॒ष्टा अ॑होरा॒त्रे प्रावि॑श॒न् ताञ्छन्दो॑\-भि॒रन्व॑\-विन्द॒द्यच्छन्दो॑भि\-रुप॒तिष्ठ॑ते॒ स्वमे॒व तदन्वि॑च्छति॒ न तत्र॑ जा॒म्य॑स्तीत्या॑हु॒र्यो\-ऽह॑रहरुप॒तिष्ठ॑त॒ इति॒ यो वा अ॒ग्निं प्र॒त्यङ्ङु॑प॒तिष्ठ॑ते॒ प्रत्ये॑नमोषति॒ यः परा॒ङ्॒ विष्व॑ङ् प्र॒जया॑ प॒शुभि॑रेति॒ कवा॑तिर्यङ्ङि॒वोप॑ तिष्ठेत॒ नैनं॑ प्र॒त्योष॑ति॒ न विष्व॑ङ् प्र॒जया॑ प॒शुभि॑रेति॥~(४२)

%1.5.10.0
{\anuvakamend[{सि॒क्तस्य॑ स॒ह भ॑वति॒ यो यत्खलु॒ वै प॒शुभि॒स्त्रयो॑दश च}]}%~(९)

%1.5.10.1
मम॒ नाम॑ प्रथ॒मं जा॑तवेदः पि॒ता मा॒ता च॑ दधतु॒र्यदग्रे᳚। तत्त्वं बि॑भृहि॒ पुन॒रा मदैतो॒स्तवा॒हं नाम॑ बिभराण्यग्ने॥ मम॒ नाम॒ तव॑ च जातवेदो॒ वास॑सी इव वि॒वसा॑नौ॒ ये चरा॑वः। आयु॑षे॒ त्वं जी॒वसे॑ व॒यं य॑थाय॒थं वि परि॑ दधावहै॒ पुन॒स्ते॥ नमो॒\-ऽग्नये\-ऽप्र॑तिविद्धाय॒ नमो\-ऽना॑धृष्टाय॒ नमः॑ स॒म्राजे᳚। अषा॑ढो~(४३)

%1.5.10.2
अ॒ग्निर्बृ॒हद्व॑या विश्व॒जिथ्सह॑न्त्यः॒ श्रेष्ठो॑ गन्ध॒र्वः॥ त्वत्पि॑तारो अग्ने दे॒वास्त्वामा॑\-हुतय॒स्त्वद्वि॑वाचनाः। सं मामायु॑षा॒ सं गौ॑प॒त्येन॒ सुहि॑ते मा धाः॥ अ॒यम॒ग्निः श्रेष्ठ॑तमो॒\-ऽयं भग॑वत्तमो॒\-ऽयꣳ स॑हस्र॒सात॑मः। अ॒स्मा अ॑स्तु सु॒वीर्यम्᳚॥ मनो॒ ज्योति॑र्जुषता॒माज्यं॒ विच्छि॑न्नं य॒ज्ञꣳ समि॒मं द॑धातु। या इ॒ष्टा उ॒षसो॑ नि॒म्रुच॑श्च॒ ताः सं द॑धामि ह॒विषा॑ घृ॒तेन॑॥ पय॑स्वती॒रोष॑धयः॒~(४४)

%1.5.10.3
पय॑स्वद्वी॒रुधां॒ पयः॑। अ॒पां पय॑सो॒ यत्पय॒स्तेन॒ मामि॑न्द्र॒ सꣳ सृ॑ज॥ अग्ने᳚ व्रतपते व्र॒तं च॑रिष्यामि॒ तच्छ॑केयं॒ तन्मे॑ राध्यताम्॥ अ॒ग्निꣳ होता॑रमि॒ह तꣳ हु॑वे दे॒वान् य॒ज्ञिया॑नि॒ह यान् हवा॑महे॥ आ य॑न्तु दे॒वाः सु॑मन॒स्यमा॑ना वि॒यन्तु॑ दे॒वा ह॒विषो॑ मे अ॒स्य॥ कस्त्वा॑ युनक्ति॒ स त्वा॑ युनक्तु॒ यानि॑ घ॒र्मे क॒पाला᳚न्युपचि॒न्वन्ति॑~(४५)

%1.5.10.4
वे॒धसः॑। पू॒ष्णस्तान्यपि॑ व्र॒त इ॑न्द्रवा॒यू विमु॑ञ्चताम्॥ अभि॑न्नो घ॒र्मो जी॒रदा॑नु॒र्यत॒ आत्त॒स्तद॑ग॒न् पुनः॑। इ॒ध्मो वेदिः॑ परि॒धय॑श्च॒ सर्वे॑ य॒ज्ञस्या\-ऽ\-ऽयु॒रनु॒ सं च॑रन्ति॥ त्रय॑स्त्रिꣳश॒त्तन्त॑वो॒ ये वि॑तत्नि॒रे य इ॒मं य॒ज्ञꣴ स्व॒धया॒ दद॑न्ते॒ तेषां᳚ छि॒न्नं प्रत्ये॒तद्द॑धामि॒ स्वाहा॑ घ॒र्मो दे॒वाꣳ अप्ये॑तु॥~(४६)

%1.5.11.0
{\anuvakamend[{अषा॑ढ॒ ओष॑धय उपचि॒न्वन्ति॒ पञ्च॑चत्वारिꣳशच्च}]}%॥10॥

%1.5.11.1
वै॒श्वा॒न॒रो न॑ ऊ॒त्या\-ऽ\-ऽप्र या॑तु परा॒वतः॑। अ॒ग्निरु॒क्थेन॒ वाह॑सा॥ ऋ॒तावा॑नं वैश्वान॒रमृ॒तस्य॒ ज्योति॑ष॒स्पतिम्᳚। अज॑स्रं घ॒र्ममी॑महे॥ वै॒श्वा॒न॒रस्य॑ द॒ꣳ॒सना᳚भ्यो बृ॒हदरि॑णा॒देकः॑ स्वप॒स्य॑या क॒विः। उ॒भा पि॒तरा॑ म॒हय॑न्नजायता॒ग्निर्द्यावा॑\-पृथि॒वी भूरि॑रेतसा॥ पृ॒ष्टो दि॒वि पृ॒ष्टो अ॒ग्निः पृ॑थि॒व्यां पृ॒ष्टो विश्वा॒ ओष॑धी॒रा वि॑वेश। वै॒श्वा॒न॒रः सह॑सा पृ॒ष्टो अ॒ग्निः स नो॒ दिवा॒ स~(४७)

%1.5.11.2
रि॒षः पा॑तु॒ नक्तम्᳚॥ जा॒तो यद॑ग्ने॒ भुव॑ना॒ व्यख्यः॑ प॒शुं न गो॒पा इर्यः॒ परि॑ज्मा। वैश्वा॑नर॒ ब्रह्म॑णे विन्द गा॒तुं यू॒यं पा॑त स्व॒स्तिभिः॒ सदा॑ नः॥ त्वम॑ग्ने शो॒चिषा॒ शोशु॑चान॒ आ रोद॑सी अपृणा॒ जाय॑मानः। त्वं दे॒वाꣳ अ॒भिश॑स्तेरमुञ्चो॒ वैश्वा॑नर जातवेदो महि॒त्वा॥ अ॒स्माक॑मग्ने म॒घव॑थ्सु धार॒याना॑मि क्ष॒त्रम॒जरꣳ॑ सु॒वीर्यम्᳚। व॒यं ज॑येम श॒तिनꣳ॑ सह॒स्रिणं॒ वैश्वा॑नर॒~(४८)

%1.5.11.3
वाज॑मग्ने॒ तवो॒तिभिः॑॥ वै॒श्वा॒न॒रस्य॑ सुम॒तौ स्या॑म॒ राजा॒ हिकं॒ भुव॑नानामभि॒श्रीः। इ॒तो जा॒तो विश्व॑मि॒दं वि च॑ष्टे वैश्वान॒रो य॑तते॒ सूर्ये॑ण॥ अव॑ ते॒ हेडो॑ वरुण॒ नमो॑\-भि॒रव॑ य॒ज्ञेभि॑रीमहे ह॒विर्भिः॑। क्षय॑न्न॒स्मभ्य॑मसुर प्रचेतो॒ राज॒न्नेनाꣳ॑सि शिश्रथः कृ॒तानि॑॥ उदु॑त्त॒मं व॑रुण॒ पाश॑\-म॒स्मद\-वा॑ध॒मं वि म॑ध्य॒मꣴ श्र॑थाय। अथा॑ व॒यमा॑दित्य~(४९)

%1.5.11.4
व्र॒ते तवाना॑गसो॒ अदि॑तये स्याम॥ द॒धि॒क्राव्ण्णो॑ अकारिषं जि॒ष्णोरश्व॑स्य वा॒जिनः॑॥ सु॒र॒भि नो॒ मुखा॑ कर॒त् प्र ण॒ आयूꣳ॑षि तारिषत्॥ आ द॑धि॒क्राः शव॑सा॒ पञ्च॑ कृ॒ष्टीः सूर्य॑ इव॒ ज्योति॑षा॒\-ऽपस्त॑तान। स॒ह॒स्र॒साः श॑त॒सा वा॒ज्यर्वा॑ पृ॒णक्तु॒ मध्वा॒ समि॒मा वचाꣳ॑सि॥ अ॒ग्निर्मू॒र्धा भुवः॑। मरु॑तो॒ यद्ध॑ वो दि॒वः सु॑म्ना॒यन्तो॒ हवा॑महे। आ तू न॒~(५०)

%1.5.11.5
उप॑ गन्तन॥ या वः॒ शर्म॑ शशमा॒नाय॒ सन्ति॑ त्रि॒धातू॑नि दा॒शुषे॑ यच्छ॒ताधि॑। अ॒स्मभ्यं॒ तानि॑ मरुतो॒ वि य॑न्त र॒यिं नो॑ धत्त वृषणः सु॒वीरम्᳚॥ अदि॑तिर्न उरुष्य॒त्वदि॑तिः॒ शर्म॑ यच्छतु। अदि॑तिः पा॒त्वꣳह॑सः॥ म॒हीमू॒ षु मा॒तरꣳ॑ सुव्र॒ता\-ना॑\-मृ॒तस्य॒ पत्नी॒मव॑से हुवेम। तु॒वि॒क्ष॒त्रा\-म॒जर॑न्ती\-मुरू॒चीꣳ सु॒शर्मा॑ण॒मदि॑तिꣳ सु॒प्रणी॑तिम्॥ सु॒त्रामा॑णं पृथि॒वीं द्याम॑ने॒हसꣳ॑ सु॒शर्मा॑ण॒मदि॑तिꣳ सु॒प्रणी॑तिम्। दैवीं॒ नावꣴ॑ स्वरि॒त्रा\-मना॑\-गस॒मस्र॑वन्ती॒मा रु॑हेमा स्व॒स्तये᳚॥ इ॒माꣳ सु नाव॒मा\-ऽरु॑हꣳ श॒तारि॑त्राꣳ श॒तस्फ्या᳚म्। अच्छि॑द्रां पारयि॒ष्णुम्॥~(५१)

{\anuvakamend[{दिवा॒ स स॑ह॒स्रिणं॒ वैश्वा॑नरा\-ऽ\-ऽदित्य॒ तू नो॑\-ऽने॒हसꣳ॑ सु॒शर्मा॑ण॒मेका॒न्न\-विꣳ॑श॒तिश्च॑}]}%॥11॥
%%% END PRASHNA

\sect{षष्ठमः प्रश्नः}\setcounter{anuvakam}{0}
\dnsub{तैत्तिरीयसंहितायां प्रथमकाण्डे षष्ठमः प्रश्नः}
%1.6.1.0
%1.6.1.1
सं त्वा॑ सिञ्चामि॒ यजु॑षा प्र॒जामायु॒र्धनं॑ च। बृह॒स्पति॑प्रसूतो॒ यज॑मान इ॒ह मा रि॑षत्॥ आज्य॑मसि स॒त्यम॑सि स॒त्यस्याध्य॑क्षमसि ह॒विर॑सि वैश्वान॒रं वै᳚श्वदे॒वमुत्पू॑तशुष्मꣳ स॒त्यौजाः॒ सहो॑\-ऽसि॒ सह॑मानमसि॒ सह॒स्वारा॑तीः॒ सह॑स्वारातीय॒तः सह॑स्व॒ पृत॑नाः॒ सह॑स्व पृतन्य॒तः। स॒हस्र॑वीर्यमसि॒ तन्मा॑ जि॒न्वा\-ऽ\-ऽज्य॒स्या\-ऽ\-ऽज्य॑मसि स॒त्यस्य॑ स॒त्यम॑सि स॒त्यायु॑-~(१)

%1.6.1.2
रसि स॒त्यशु॑ष्ममसि स॒त्येन॑ त्वा॒\-ऽभि घा॑रयामि॒ तस्य॑ ते भक्षीय\\
पञ्चा॒नां त्वा॒ वाता॑नां य॒न्त्राय॑ ध॒र्त्राय॑ गृह्णामि\\
पञ्चा॒नां त्व॑र्तू॒नां य॒न्त्राय॑ ध॒र्त्राय॑ गृह्णामि\\
पञ्चा॒नां त्वा॑ दि॒शां य॒न्त्राय॑ ध॒र्त्राय॑ गृह्णामि\\
पञ्चा॒नां त्वा॑ पञ्चज॒नानां᳚ य॒न्त्राय॑ ध॒र्त्राय॑ गृह्णामि\\
च॒रोस्त्वा॒ पञ्च॑बिलस्य य॒न्त्राय॑ ध॒र्त्राय॑ गृह्णामि॒\\
ब्रह्म॑णस्त्वा॒ तेज॑से य॒न्त्राय॑ ध॒र्त्राय॑ गृह्णामि\\
क्ष॒त्रस्य॒ त्वौज॑से य॒न्त्राय॑~(२)

%1.6.1.3
ध॒र्त्राय॑ गृह्णामि\\
वि॒शे त्वा॑ य॒न्त्राय॑ ध॒र्त्राय॑ गृह्णामि\\
सु॒वीर्या॑य त्वा गृह्णामि सुप्रजा॒स्त्वाय॑ त्वा गृह्णामि रा॒यस्पोषा॑य त्वा गृह्णामि ब्रह्मवर्च॒साय॑ त्वा गृह्णामि॒ भूर॒स्माकꣳ॑ ह॒विर्दे॒वाना॑मा॒शिषो॒ यज॑मानस्य दे॒वानां᳚ त्वा दे॒वता᳚भ्यो गृह्णामि॒ कामा॑य त्वा गृह्णामि॥~(३)

%1.6.2.0
{\anuvakamend[{स॒त्यायु॒रोज॑से य॒न्त्राय॒ त्रय॑स्त्रिꣳशच्च}]}%~(१)

%1.6.2.1
ध्रु॒वो॑\-ऽसि ध्रु॒वो॑\-ऽहꣳ स॑जा॒तेषु॑ भूयासं॒\\
धीर॒श्चेत्ता॑ वसु॒विदु॒ग्रो᳚\-ऽस्यु॒ग्रो॑\-ऽहꣳ स॑जा॒तेषु॑ भूयास-\\
मु॒ग्रश्चेत्ता॑ वसु॒विद॑भि॒भूर॑स्यभि॒भूर॒हꣳ स॑जा॒तेषु॑ भूयास-\\
मभि॒भूश्चेत्ता॑ वसु॒विद्यु॒नज्मि॑ त्वा॒ ब्रह्म॑णा॒ दैव्ये॑न ह॒व्याया॒स्मै वोढ॒वे जा॑तवेदः। इन्धा॑नास्त्वा सुप्र॒जसः॑ सु॒वीरा॒ ज्योग्जी॑वेम बलि॒हृतो॑ व॒यं ते᳚॥ यन्मे॑ अग्ने अ॒स्य य॒ज्ञस्य॒ रिष्या॒-~(४)

%1.6.2.2
द्यद्वा॒ स्कन्दा॒दाज्य॑स्यो॒त वि॑ष्णो। तेन॑ हन्मि स॒पत्नं॑ दुर्मरा॒युमैनं॑ दधामि॒ निर्\mbox{}ऋ॑त्या उ॒पस्थे᳚। भूर्भुवः॒ सुव॒रुच्छु॑ष्मो अग्ने॒ यज॑मानायैधि॒ निशु॑ष्मो अभि॒दास॑ते। अग्ने॒ देवे᳚द्ध॒ मन्वि॑द्ध॒ मन्द्र॑जि॒ह्वाम॑र्त्यस्य ते होतर्मू॒र्धन्ना जि॑घर्मि रा॒यस्पोषा॑य सुप्रजा॒स्त्वाय॑ सु॒वीर्या॑य॒ मनो॑\-ऽसि प्राजाप॒त्यं मन॑सा मा भू॒तेना\-ऽ\-ऽवि॑श॒ वाग॑स्यै॒न्द्री स॑पत्न॒क्षय॑णी~(५)

%1.6.2.3
वा॒चा मे᳚न्द्रि॒येणा\-ऽ\-ऽवि॑श\\
वस॒न्तमृ॑तू॒नां प्री॑णामि॒ स मा᳚ प्री॒तः प्री॑णातु\\
ग्री॒ष्ममृ॑तू॒नां प्री॑णामि॒ स मा᳚ प्री॒तः प्री॑णातु\\
व॒र्॒\mbox{}षा ऋ॑तू॒नां प्री॑णामि॒ ता मा᳚ प्री॒ताः प्री॑णन्तु\\
श॒रद॑मृतू॒नां प्री॑णामि॒ सा मा᳚ प्री॒ता प्री॑णातु\\
हेमन्तशिशि॒रावृ॑तू॒नां प्री॑णामि॒ तौ मा᳚ प्री॒तौ प्री॑णीता-\\
म॒ग्नी\-षोम॑योर॒हं दे॑वय॒ज्यया॒ चक्षु॑ष्मान् भूयासम॒-\\
ग्नेर॒हं दे॑वय॒ज्यया᳚न्ना॒दो भू॑यासं॒~(६)

%1.6.2.4
दब्धि॑र॒स्यद॑ब्धो भूयास-\\
म॒मुं द॑भेयम॒ग्नी\-षोम॑योर॒हं दे॑वय॒ज्यया॑ वृत्र॒हा भू॑यास-\\
मिन्द्राग्नि॒योर॒हं दे॑वय॒ज्यये᳚न्द्रिया॒\-व्य॑न्ना॒दो भू॑यास॒-\\
मिन्द्र॑स्या॒हं दे॑वय॒ज्यये᳚न्द्रिया॒वी भू॑यासं\\
महे॒न्द्रस्या॒हं दे॑वय॒ज्यया॑ जे॒मानं॑ महि॒मानं॑ गमेयम॒ग्नेः स्वि॑ष्ट॒कृतो॒\-ऽहं दे॑वय॒ज्यया\-ऽ\-ऽयु॑ष्मान् य॒ज्ञेन॑ प्रति॒ष्ठां ग॑मेयम्॥~(७)

%1.6.3.0
{\anuvakamend[{रिष्या᳚थ्सपत्न॒क्षय॑ण्यन्ना॒दो भू॑यास॒ꣳ॒ षट्त्रिꣳ॑शच्च}]}%~(२)

%1.6.3.1
अ॒ग्निर्मा॒ दुरि॑ष्टात् पातु सवि॒ता\-ऽघशꣳ॑सा॒द्यो मे\-ऽन्ति॑ दू॒रे॑\-ऽराती॒यति॒ तमे॒तेन॑ जेष॒ꣳ॒ सुरू॑पवर्\mbox{}षवर्ण॒ एही॒मान् भ॒द्रान् दुर्याꣳ॑ अ॒भ्येहि॒ मामनु॑व्रता॒ न्यु॑ शी॒र्॒\mbox{}षाणि॑ मृढ्व॒मिड॒ एह्यदि॑त॒ एहि॒ सर॑स्व॒त्येहि॒ रन्ति॑रसि॒ रम॑तिरसि सू॒नर्य॑सि॒ जुष्टे॒ जुष्टिं॑ ते\-ऽशी॒योप॑हूत उपह॒वं~(८)

%1.6.3.2
ते॑\-ऽशीय॒ सा मे॑ स॒त्याशीर॒स्य य॒ज्ञस्य॑ भूया॒दरे॑डता॒ मन॑सा॒ तच्छ॑केयं य॒ज्ञो दिवꣳ॑ रोहतु य॒ज्ञो दिवं॑ गच्छतु॒ यो दे॑व॒यानः॒ पन्था॒स्तेन॑ य॒ज्ञो दे॒वाꣳ अप्ये᳚त्व॒स्मास्विन्द्र॑ इन्द्रि॒यं द॑धात्व॒स्मान्राय॑ उ॒त य॒ज्ञाः स॑चन्ताम॒स्मासु॑ सन्त्वा॒शिषः॒ सा नः॑ प्रि॒या सु॒प्रतू᳚र्तिर्म॒घोनी॒ जुष्टि॑रसि जु॒षस्व॑ नो॒ जुष्टा॑ नो-~(९)

%1.6.3.3
ऽसि॒ जुष्टिं॑ ते गमेयं॒ मनो॒ ज्योति॑र्जुषता॒माज्यं॒ विच्छि॑न्नं य॒ज्ञꣳ समि॒मं द॑धातु। बृह॒स्पति॑स्तनुतामि॒मं नो॒ विश्वे॑ दे॒वा इ॒ह मा॑दयन्ताम्॥ ब्रध्न॒ पिन्व॑स्व॒ दद॑तो मे॒ मा क्षा॑यि कुर्व॒तो मे॒ मोप॑दसत् प्र॒जा\-प॑तेर्भा॒गो᳚\-ऽस्यूर्ज॑स्वा॒न् पय॑स्वान् प्राणापा॒नौ मे॑ पाहि समानव्या॒नौ मे॑ पाह्युदानव्या॒नौ मे॑ पा॒ह्यक्षि॑तो॒\-ऽस्यक्षि॑त्यै त्वा॒ मा मे᳚ क्षेष्ठा अ॒मुत्रा॒मुष्मिँ॑ल्लो॒के॥~(१०)

%1.6.4.0
{\anuvakamend[{उ॒प॒ह॒वं जुष्टा॑ नस्त्वा॒ षट् च॑}]}%~(३)

%1.6.4.1
ब॒र्॒\mbox{}हिषो॒\-ऽहं दे॑वय॒ज्यया᳚ प्र॒जावा᳚न् भूयासं॒ नरा॒शꣳस॑स्या॒हं दे॑वय॒ज्यया॑ पशु॒मान् भू॑यासम॒ग्नेः स्वि॑ष्ट॒कृतो॒\-ऽहं दे॑वय॒ज्यया\-ऽ\-ऽयु॑ष्मान् य॒ज्ञेन॑ प्रति॒ष्ठां ग॑मेयम॒ग्ने\-र॒हमुज्जि॑ति॒\-मनूज्जे॑ष॒ꣳ॒ सोम॑\-स्या॒\-हमुज्जि॑ति॒\-मनूज्जे॑षम॒ग्नेर॒हमुज्जि॑ति॒\-मनूज्जे॑षम॒ग्नी\-षोम॑यो\-र॒ह\-मु\-ज्जि॑ति॒\-मनूज्जे॑ष\-मिन्द्राग्नि॒यो\-र॒हमुज्जि॑ति॒\-मनूज्जे॑ष॒\-मिन्द्र॑स्या॒हमु-~(११)

%1.6.4.2
ज्जि॑ति॒मनूज्जे॑षं महे॒न्द्रस्या॒हमुज्जि॑ति॒\-मनूज्जे॑षम॒ग्नेः स्वि॑ष्ट॒कृतो॒\-ऽहमुज्जि॑ति॒मनूज्जे॑षं॒ वाज॑स्य मा प्रस॒वेनो᳚द्ग्रा॒भेणोद॑\-ग्रभीत्। अथा॑ स॒पत्ना॒ꣳ॒ इन्द्रो॑ मे निग्रा॒भेणाध॑राꣳ अकः॥ उ॒द्ग्रा॒भं च॑ निग्रा॒भं च॒ ब्रह्म॑ दे॒वा अ॑वीवृधन्न्। अथा॑ स॒पत्ना॑निन्द्रा॒ग्नी मे॑ विषू॒चीना॒न्व्य॑स्यताम्॥ एमा अ॑ग्मन्ना॒शिषो॒ दोह॑कामा॒ इन्द्र॑वन्तो~(१२)

%1.6.4.3
वनामहे धुक्षी॒महि॑ प्र॒जामिषम्᳚॥ रोहि॑तेन त्वा॒\-ऽग्निर्दे॒वतां᳚ गमयतु॒ हरि॑भ्यां॒ त्वेन्द्रो॑ दे॒वतां᳚ गमय॒त्वेत॑शेन त्वा॒ सूर्यो॑ दे॒वतां᳚ गमयतु॒ वि ते॑ मुञ्चामि रश॒ना वि र॒श्मीन् वि योक्त्रा॒ यानि॑ परि॒चर्त॑नानि ध॒त्ताद॒स्मासु॒ द्रवि॑णं॒ यच्च॑ भ॒द्रं प्र णो᳚ ब्रूताद्भाग॒धान् दे॒वता॑सु॥ विष्णोः᳚ शं॒योर॒हं दे॑वय॒ज्यया॑ य॒ज्ञेन॑ प्रति॒ष्ठां ग॑मेय॒ꣳ॒ सोम॑स्या॒हं दे॑वय॒ज्यया॑~(१३)

%1.6.4.4
सु॒रेता॒ रेतो॑ धिषीय॒ त्वष्टु॑र॒हं दे॑वय॒ज्यया॑ पशू॒नाꣳ रू॒पं पु॑षेयं दे॒वानां॒ पत्नी॑र॒ग्निर्गृ॒ह\-प॑तिर्य॒ज्ञस्य॑ मिथु॒नं तयो॑र॒हं दे॑वय॒ज्यया॑ मिथु॒नेन॒ प्र भू॑यासं वे॒दो॑\-ऽसि॒ वित्ति॑रसि वि॒देय॒ कर्मा॑सि क॒रुण॑मसि क्रि॒यासꣳ॑ स॒निर॑सि सनि॒तासि॑ स॒नेयं॑ घृ॒तव॑न्तं कुला॒यिनꣳ॑ रा॒यस्पोषꣳ॑ सह॒स्रिणं॑ वे॒दो द॑दातु वा॒जिनम्᳚॥~(१४)

%1.6.5.0
{\anuvakamend[{इन्द्र॑स्या॒हमिन्द्र॑वन्तः॒ सोम॑स्या॒हं दे॑वय॒ज्यया॒ चतु॑श्चत्वारिꣳशच्च}]}%~(४)

%1.6.5.1
आ प्या॑यतां ध्रु॒वा घृ॒तेन॑ य॒ज्ञं य॑ज्ञं॒ प्रति॑ देव॒यद्भ्यः॑। सू॒र्याया॒ ऊधो\-ऽदि॑त्या उ॒पस्थ॑ उ॒रुधा॑रा पृथि॒वी य॒ज्ञे अ॒स्मिन्॥ प्र॒जा\-प॑तेर्वि॒भान्नाम॑ लो॒कस्तस्मिꣴ॑स्त्वा दधामि स॒ह यज॑मानेन॒ सद॑सि॒ सन्मे॑ भूयाः॒ सर्व॑मसि॒ सर्वं॑ मे भूयाः पू॒र्णम॑सि पू॒र्णं मे॑ भूया॒ अक्षि॑तमसि॒ मा मे᳚ क्षेष्ठाः॒ प्राच्यां᳚ दि॒शि दे॒वा ऋ॒त्विजो॑ मार्जयन्तां॒ दक्षि॑णायां~(१५)

%1.6.5.2
दि॒शि मासाः᳚ पि॒तरो॑ मार्जयन्तां प्र॒तीच्यां᳚ दि॒शि गृ॒हाः प॒शवो॑ मार्जयन्ता॒मुदी᳚च्यां दि॒श्याप॒ ओष॑धयो॒ वन॒स्पत॑यो मार्जयन्तामू॒र्ध्वायां᳚ दि॒शि य॒ज्ञः सं॑वथ्स॒रो य॒ज्ञप॑तिर्मार्जयन्तां॒ विष्णोः॒ क्रमो᳚\-ऽस्यभिमाति॒हा गा॑य॒त्रेण॒ छन्द॑सा पृथि॒वीमनु॒ वि क्र॑मे॒ निर्भ॑क्तः॒ स यं द्वि॒ष्मो विष्णोः॒ क्रमो᳚\-ऽस्यभिशस्ति॒हा त्रैष्टु॑भेन॒ छन्द॑सा॒\-ऽन्तरि॑क्ष॒मनु॒ वि क्र॑मे॒ निर्भ॑क्तः॒ स यं द्वि॒ष्मो विष्णोः॒ क्रमो᳚\-ऽस्यरातीय॒तो ह॒न्ता जाग॑तेन॒ छन्द॑सा॒ दिव॒मनु॒ वि क्र॑मे॒ निर्भ॑क्तः॒ स यं द्वि॒ष्मो विष्णोः॒ क्रमो॑\-ऽसि शत्रूय॒तो ह॒न्ता\-ऽ\-ऽनु॑ष्टुभेन॒ छन्द॑सा॒ दिशो\-ऽनु॒ वि क्र॑मे॒ निर्भ॑क्तः॒ स यं द्वि॒ष्मः॥~(१६)

%1.6.6.0
{\anuvakamend[{दक्षि॑णायां द्वि॒ष्मो विष्णो॒रेका॒न्नत्रि॒ꣳ॒शच्च॑}]}%~(५)

%1.6.6.1
अग॑न्म॒ सुवः॒ सुव॑रगन्म स॒न्दृश॑स्ते॒ मा छि॑थ्सि॒ यत्ते॒ तप॒स्तस्मै॑ ते॒ मा\-ऽ\-ऽवृ॑क्षि सु॒भूर॑सि॒ श्रेष्ठो॑ रश्मी॒नामा॑यु॒र्धा अ॒स्यायु॑र्मे धेहि वर्चो॒धा अ॑सि॒ वर्चो॒ मयि॑ धेही॒दम॒हम॒मुं भ्रातृ॑व्यमा॒भ्यो दि॒ग्भ्यो᳚\-ऽस्यै दि॒वो᳚\-ऽस्माद॒न्तरि॑क्षाद॒स्यै पृ॑थि॒व्या अ॒स्मा\-द॒न्नाद्या॒न्निर्भ॑जामि॒ निर्भ॑क्तः॒ स यं द्वि॒ष्मः।~(१७)

%1.6.6.2
सं ज्योति॑षा\-ऽभूवमै॒न्द्रीमा॒वृत॑\-म॒न्वाव॑र्ते॒ सम॒हं प्र॒जया॒ सं मया᳚ प्र॒जा सम॒हꣳ रा॒यस्पोषे॑ण॒ सं मया॑ रा॒यस्पोषः॒ समि॑द्धो अग्ने मे दीदिहि समे॒द्धा ते॑ अग्ने दीद्यासं॒ वसु॑मान् य॒ज्ञो वसी॑यान् भूयास॒मग्न॒ आयूꣳ॑षि पवस॒ आ सु॒वोर्ज॒मिषं॑ च नः। आ॒रे बा॑धस्व दु॒च्छुना᳚म्॥ अग्ने॒ पव॑स्व॒ स्वपा॑ अ॒स्मे वर्चः॑ सु॒वीर्यम्᳚।~(१८)

%1.6.6.3
दध॒त् पोषꣳ॑ र॒यिं मयि॑। अग्ने॑ गृहपते सुगृहप॒तिर॒हं त्वया॑ गृ॒हप॑तिना भूयासꣳ सुगृहप॒तिर्मया॒ त्वं गृ॒हप॑तिना भूयाः श॒तꣳ हिमा॒स्तामा॒शिष॒मा शा॑से॒ तन्त॑वे॒ ज्योति॑ष्मतीं॒ तामा॒शिष॒माशा॑से॒\-ऽमुष्मै॒ ज्योति॑ष्मतीं॒ कस्त्वा॑ युनक्ति॒ स त्वा॒ विमु॑ञ्च॒त्वग्ने᳚ व्रतपते व्र॒तम॑चारिषं॒ तद॑शकं॒ तन्मे॑\-ऽराधि य॒ज्ञो ब॑भूव॒ स आ~(१९)

%1.6.6.4
ब॑भूव॒ स प्र ज॑ज्ञे॒ स वा॑वृधे। स दे॒वाना॒मधि॑पतिर्बभूव॒ सो अ॒स्माꣳ अधि॑पतीन् करोतु व॒यꣴ स्या॑म॒ पत॑यो रयी॒णाम्॥ गोमाꣳ॑ अ॒ग्ने\-ऽवि॑माꣳ अ॒श्वी य॒ज्ञो नृ॒वथ्स॑खा॒ सद॒मिद॑प्रमृ॒ष्यः। इडा॑वाꣳ ए॒षो अ॑सुर प्र॒जावा᳚न् दी॒र्घो र॒यिः पृ॑थुबु॒ध्नः स॒भावान्॑॥~(२०)

%1.6.7.0
{\anuvakamend[{द्वि॒ष्मः सु॒वीर्य॒ꣳ॒ स आ पञ्च॑त्रिꣳशच्च}]}%~(६)

%1.6.7.1
यथा॒ वै स॑मृतसो॒मा ए॒वं वा ए॒ते स॑मृतय॒ज्ञा यद्द॑र्\mbox{}श\-पूर्ण\-मा॒सौ कस्य॒ वाह॑ दे॒वा य॒ज्ञमा॒ गच्छ॑न्ति॒ कस्य॑ वा॒ न ब॑हू॒नां यज॑मानानां॒ यो वै दे॒वताः॒ पूर्वः॑ परिगृ॒ह्णाति॒ स ए॑नाः॒ श्वो भू॒ते य॑जत ए॒तद्वै दे॒वाना॑मा॒यत॑नं॒ यदा॑हव॒नीयो᳚\-ऽन्त॒राग्नी प॑शू॒नां गार्\mbox{}ह॑पत्यो मनु॒ष्या॑णामन्वाहार्य॒पच॑नः पितृ॒णाम॒ग्निं गृ॑ह्णाति॒ स्व ए॒वायत॑ने दे॒वताः॒ परि॑~(२१)

%1.6.7.2
गृह्णाति॒ ताः श्वो भू॒ते य॑जते व्र॒तेन॒ वै मेध्यो॒\-ऽग्निर्व्र॒तप॑तिर्ब्राह्म॒णो व्र॑त॒भृद् व्र॒तमु॑पै॒ष्यन् ब्रू॑या॒दग्ने᳚ व्रतपते व्र॒तं च॑रिष्या॒मीत्य॒ग्निर्वै दे॒वानां᳚ व्र॒तप॑ति॒स्तस्मा॑ ए॒व प्र॑ति॒प्रोच्य॑ व्र॒तमाल॑भते ब॒र्॒\mbox{}हिषा॑ पू॒र्णमा॑से व्र॒तमुपै॑ति व॒थ्सैर॑मा\-वा॒स्या॑यामे॒तद्ध्ये॑तयो॑\-रा॒यत॑न\-मुप॒स्तीर्यः॒ पूर्व॑श्चा॒ग्निरप॑र॒\-श्चेत्या॑हुर्मनु॒ष्या॑~(२२)

%1.6.7.3
इन्न्वा उप॑स्तीर्णमि॒च्छन्ति॒ किमु॑ दे॒वा येषां॒ नवा॑वसान॒\-मुपा᳚स्मि॒ञ्छ्वो य॒क्ष्यमा॑णे दे॒वता॑ वसन्ति॒ य ए॒वं वि॒द्वान॒ग्निमु॑पस्तृ॒णाति॒ यज॑मानेन ग्रा॒म्याश्च॑ प॒शवो॑\-ऽव॒रुध्या॑ आर॒ण्याश्चेत्या॑\-हु॒र्यद्ग्रा॒म्यानु॑प॒\-वस॑ति॒ तेन॑ ग्रा॒म्यानव॑ रुन्धे॒ यदा॑र॒ण्यस्या॒श्ञाति॒ तेना॑र॒ण्यान् यदना᳚श्वानुप॒वसे᳚त् पितृदेव॒त्यः॑ स्यादा\-र॒ण्यस्या᳚श्ञातीन्द्रि॒यं~(२३)

%1.6.7.4
वा आ॑र॒ण्यमि॑न्द्रि॒यमे॒वा\-ऽ\-ऽत्मन् ध॑त्ते॒ यदना᳚श्वानुप॒वसे॒त् क्षोधु॑कः स्या॒द्यद॑श्ञी॒याद्रु॒द्रो᳚\-ऽस्य प॒शून॒भिम॑न्येता॒पो᳚\-ऽश्ञाति॒ तन्नेवा॑शि॒तं नेवान॑शितं॒ न क्षोधु॑को॒ भव॑ति॒ नास्य॑ रु॒द्रः प॒शून॒भि म॑न्यते॒ वज्रो॒ वै य॒ज्ञः क्षुत्खलु॒ वै म॑नु॒ष्य॑स्य॒ भ्रातृ॑व्यो॒ यदना᳚श्वानुप॒वस॑ति॒ वज्रे॑णै॒व सा॒क्षात्क्षुधं॒ भ्रातृ॑व्यꣳ हन्ति॥~(२४)

%1.6.8.0
{\anuvakamend[{परि॑ मनु॒ष्या॑ इन्द्रि॒यꣳ सा॒क्षात् त्रीणि॑ च}]}%~(७)

%1.6.8.1
यो वै श्र॒द्धामना॑रभ्य य॒ज्ञेन॒ यज॑ते॒ नास्ये॒ष्टाय॒ श्रद्द॑धते॒\-ऽपः प्र ण॑यति श्र॒द्धा वा आपः॑ श्र॒द्धामे॒वा\-ऽ\-ऽरभ्य॑ य॒ज्ञेन॑ यजत उ॒भये᳚\-ऽस्य देवमनु॒ष्या इ॒ष्टाय॒ श्रद्द॑धते॒ तदा॑हु॒रति॒ वा ए॒ता वर्त्र॑न्नेद॒न्त्यति॒ वाचं॒ मनो॒ वावैता नाति॑ नेद॒न्तीति॒ मन॑सा॒ प्र ण॑यती॒यं वै मनो॒-~(२५)

%1.6.8.2
ऽनयै॒वैनाः॒ प्र ण॑य॒त्यस्क॑न्नहविर्भवति॒ य ए॒वं वेद॑ यज्ञायु॒धानि॒ सम्भ॑रति य॒ज्ञो वै य॑ज्ञायु॒धानि॑ य॒ज्ञमे॒व तथ्सम्भ॑रति॒ यदेक॑मेकꣳ स॒म्भरे᳚त् पितृदेव॒त्या॑नि स्यु॒र्यथ्स॒ह सर्वा॑णि मानु॒षाणि॒ द्वेद्वे॒ सम्भ॑रति याज्यानुवा॒क्य॑योरे॒व रू॒पं क॑रो॒त्यथो॑ मिथु॒नमे॒व यो वै दश॑ यज्ञायु॒धानि॒ वेद॑ मुख॒तो᳚\-ऽस्य य॒ज्ञः क॑ल्पते॒ स्फ्य-~(२६)

%1.6.8.3
श्च॑ क॒पाला॑नि चाग्निहोत्र॒हव॑णी च॒ शूर्पं॑ च कृष्णाजि॒नं च॒ शम्या॑ चो॒लूख॑लं च॒ मुस॑लं च दृ॒षच्चोप॑ला चै॒तानि॒ वै दश॑ यज्ञायु॒धानि॒ य ए॒वं वेद॑ मुख॒तो᳚\-ऽस्य य॒ज्ञः क॑ल्पते॒ यो वै दे॒वेभ्यः॑ प्रति॒प्रोच्य॑ य॒ज्ञेन॒ यज॑ते जु॒षन्ते᳚\-ऽस्य दे॒वा ह॒व्यꣳ ह॒विर्नि॑रु॒प्यमा॑णम॒भि म॑न्त्रयेता॒ग्निꣳ होता॑रमि॒ह तꣳ हु॑व॒ इति॑~(२७)

%1.6.8.4
दे॒वेभ्य॑ ए॒व प्र॑ति॒प्रोच्य॑ य॒ज्ञेन॑ यजते जु॒षन्ते᳚\-ऽस्य दे॒वा ह॒व्यमे॒ष वै य॒ज्ञस्य॒ ग्रहो॑ गृही॒त्वैव य॒ज्ञेन॑ यजते॒ तदु॑दि॒त्वा वाचं॑ यच्छति य॒ज्ञस्य॒ धृत्या॒ अथो॒ मन॑सा॒ वै प्र॒जा\-प॑तिर्य॒ज्ञम॑तनुत॒ मन॑सै॒व तद्य॒ज्ञं त॑नुते॒ रक्ष॑सा॒मन॑न्ववचाराय॒ यो वै य॒ज्ञं योग॒ आग॑ते यु॒नक्ति॑ यु॒ङ्क्ते यु॑ञ्जा॒नेषु॒ कस्त्वा॑ युनक्ति॒ स त्वा॑ युन॒क्त्वित्या॑ह प्र॒जा\-प॑ति॒र्वै कः प्र॒जा\-प॑तिनै॒वैनं॑ युनक्ति यु॒ङ्क्ते यु॑ञ्जा॒नेषु॑॥~(२८)

%1.6.9.0
{\anuvakamend[{वै मनः॒ स्फ्य इति॑ युन॒क्त्वेका॑\-दश च}]}%~(८)

%1.6.9.1
प्र॒जा\-प॑तिर्य॒ज्ञान॑सृजता\-ग्निहो॒त्रं चा᳚ग्निष्टो॒मं च॑ पौर्णमा॒सीं चो॒क्थ्यं॑ चामावा॒स्यां᳚ चातिरा॒त्रं च॒ तानुद॑मिमीत॒ याव॑दग्निहो॒त्रमासी॒त् तावा॑नग्निष्टो॒मो याव॑ती पौर्णमा॒सी तावा॑नु॒क्थ्यो॑ याव॑त्यमावा॒स्या॑ तावा॑नतिरा॒त्रो य ए॒वं वि॒द्वान॑ग्निहो॒त्रं जु॒होति॒ याव॑दग्निष्टो॒मेनो॑पा॒प्नोति॒ ताव॒दुपा᳚\-ऽ\-ऽप्नोति॒ य ए॒वं वि॒द्वान् पौ᳚र्णमा॒सीं यज॑ते॒ याव॑दु॒क्थ्ये॑नो\-पा॒प्नोति॒~(२९)

%1.6.9.2
ताव॒दुपा᳚\-ऽ\-ऽप्नोति॒ य ए॒वं वि॒द्वान॑मावा॒स्यां᳚ यज॑ते॒ याव॑दतिरा॒त्रेणो॑पा॒प्नोति॒ ताव॒दुपा᳚\-ऽ\-ऽप्नोति परमे॒ष्ठिनो॒ वा ए॒ष य॒ज्ञो\-ऽग्र॑ आसी॒त् तेन॒ स प॑र॒मां काष्ठा॑मगच्छ॒त् तेन॑ प्र॒जा\-प॑तिं नि॒रवा॑सायय॒त् तेन॑ प्र॒जा\-प॑तिः पर॒मां काष्ठा॑मगच्छ॒त् तेनेन्द्रं॑ नि॒रवा॑सायय॒त् तेनेन्द्रः॑ पर॒मां काष्ठा॑मगच्छ॒त् तेना॒ग्नी\-षोमौ॑ नि॒रवा॑सायय॒त् तेना॒ग्नी\-षोमौ॑ पर॒मां काष्ठा॑मगच्छतां॒ य~(३०)

%1.6.9.3
ए॒वं वि॒द्वान् द॑र्\mbox{}शपूर्णमा॒सौ यज॑ते पर॒मामे॒व काष्ठां᳚ गच्छति॒ यो वै प्रजा॑तेन य॒ज्ञेन॒ यज॑ते॒ प्र प्र॒जया॑ प॒शुभि॑र्मिथु॒नैर्जा॑यते॒ द्वाद॑श॒ मासाः᳚ संवथ्स॒रो द्वाद॑श द्व॒न्द्वानि॑ दर्\mbox{}श\-पूर्ण\-मा॒सयो॒स्तानि॑ स॒म्पाद्या॒नीत्या॑हुर्व॒थ्सं चो॑पावसृ॒जत्यु॒खां चाधि॑ श्रय॒त्यव॑ च॒ हन्ति॑ दृ॒षदौ॑ च स॒माह॒न्त्यधि॑ च॒ वप॑ते क॒पाला॑नि॒ चोप॑ दधाति पुरो॒डाशं॑ चा-~(३१)

%1.6.9.4
धि॒श्रय॒त्याज्यं॑ च स्तम्बय॒जुश्च॒ हर॑त्य॒भि च॑ गृह्णाति॒ वेदिं॑ च परिगृ॒ह्णाति॒ पत्नीं᳚ च॒ सं न॑ह्यति॒ प्रोक्ष॑णीश्चा\-ऽ\-ऽसा॒दय॒त्याज्यं॑ चै॒तानि॒ वै द्वाद॑श द्व॒न्द्वानि॑ दर्\mbox{}श\-पूर्ण\-मा॒सयो॒स्तानि॒ य ए॒वꣳ स॒म्पाद्य॒ यज॑ते॒ प्रजा॑तेनै॒व य॒ज्ञेन॑ यजते॒ प्र प्र॒जया॑ प॒शुभि॑र्मिथु॒नैर्जा॑यते॥~(३२)

%1.6.10.0
{\anuvakamend[{उ॒क्थ्ये॑नोपा॒प्नोत्य॑गच्छतां॒ यः पु॑रो॒डाशं॑ च चत्वारि॒ꣳ॒शच्च॑}]}%~(९)

%1.6.10.1
ध्रु॒वो॑\-ऽसि ध्रु॒वो॑\-ऽहꣳ स॑जा॒तेषु॑ भूयास॒मित्या॑ह ध्रु॒वाने॒वैना᳚न् कुरुत उ॒ग्रो᳚\-ऽस्यु॒ग्रो॑\-ऽहꣳ स॑जा॒तेषु॑ भूयास॒मित्या॒हाप्र॑तिवादिन ए॒वैना᳚न्कुरुते\-ऽभि॒भूर॑स्यभि॒भूर॒हꣳ स॑जा॒तेषु॑ भूयास॒मित्या॑ह॒ य ए॒वैनं॑ प्रत्यु॒त्पिपी॑ते॒ तमुपा᳚स्यते यु॒नज्मि॑ त्वा॒ ब्रह्म॑णा॒ दैव्ये॒नेत्या॑है॒ष वा अ॒ग्नेर्योग॒स्तेनै॒-~(३३)

%1.6.10.2
वैनं॑ युनक्ति य॒ज्ञस्य॒ वै समृ॑द्धेन दे॒वाः सु॑व॒र्गं लो॒कमा॑यन् य॒ज्ञस्य॒ व्यृ॑द्धे॒नासु॑रा॒न् परा॑भावय॒न्॒ यन्मे॑ अग्ने अ॒स्य य॒ज्ञस्य॒ रिष्या॒दित्या॑ह य॒ज्ञस्यै॒व तथ्समृ॑द्धेन॒ यज॑मानः सुव॒र्गं लो॒कमे॑ति य॒ज्ञस्य॒ व्यृ॑द्धेन॒ भ्रातृ॑व्या॒न् परा॑ भावयत्यग्नि\-हो॒त्रमे॒ताभि॒र्व्याहृ॑तीभि॒\-रुप॑ सादयेद्यज्ञमु॒खं वा अ॑ग्निहो॒त्रं ब्रह्मै॒ता व्याहृ॑तयो यज्ञमु॒ख ए॒व ब्रह्म॑~(३४)

%1.6.10.3
कुरुते संवथ्स॒रे प॒र्याग॑त ए॒ताभि॑रे॒वोप॑सादये॒द् ब्रह्म॑णै॒वोभ॒यतः॑ संवथ्स॒रं परि॑गृह्णाति दर्\mbox{}श\-पूर्ण\-मा॒सौ चा॑तुर्मा॒स्यान्या॒लभ॑मान ए॒ताभि॒र्व्याहृ॑तीभिर्\mbox{}ह॒वीꣴष्यासा॑द\-येद्यज्ञमु॒खं वै द॑र्\mbox{}शपूर्णमा॒सौ चा॑तुर्मा॒स्यानि॒ ब्रह्मै॒ता व्याहृ॑तयो यज्ञमु॒ख ए॒व ब्रह्म॑ कुरुते संवथ्स॒रे प॒र्याग॑त ए॒ताभि॑रे॒वासा॑द\-ये॒द् ब्रह्म॑णै॒वोभ॒यतः॑ संवथ्स॒रं परि॑गृह्णाति॒ यद्वै य॒ज्ञस्य॒ साम्ना᳚ क्रि॒यते॑ रा॒ष्ट्रं~(३५)

%1.6.10.4
य॒ज्ञस्या॒\-ऽ\-ऽशीर्ग॑च्छति॒ यदृ॒चा विशं॑ य॒ज्ञस्या॒\-ऽ\-ऽशीर्ग॑च्छ॒त्यथ॑ ब्राह्म॒णो॑\-ऽना॒शीर्के॑ण य॒ज्ञेन॑ यजते सामिधे॒नीर॑नुव॒क्ष्यन्ने॒ता व्याहृ॑तीः पु॒रस्ता᳚द्दध्या॒द् ब्रह्मै॒व प्र॑ति॒पदं॑ कुरुते॒ तथा᳚ ब्राह्म॒णः साशी᳚र्केण य॒ज्ञेन॑ यजते॒ यं का॒मये॑त॒ यज॑मानं॒ भ्रातृ॑व्यमस्य य॒ज्ञस्या॒\-ऽ\-ऽशीर्ग॑च्छे॒दिति॒ तस्यै॒ता व्याहृ॑तीः पुरो\-ऽनुवा॒क्या॑यां दध्याद् भ्रातृव्यदेव॒त्या॑ वै पु॑रो\-ऽनुवा॒क्या᳚ भ्रातृ॑व्यमे॒वास्य॑ य॒ज्ञस्या॒-~(३६)

%1.6.10.5
ऽऽशीर्ग॑च्छति॒ यान् का॒मये॑त॒ यज॑मानान्थ्स॒माव॑त्येनान् य॒ज्ञस्या॒\-ऽ\-ऽशीर्ग॑च्छे॒दिति॒ तेषा॑मे॒ता व्याहृ॑तीः पुरो\-ऽनुवा॒क्या॑या अर्ध॒र्च एकां᳚ दध्याद्या॒ज्या॑यै पु॒रस्ता॒देकां᳚ या॒ज्या॑या अर्ध॒र्च एकां॒ तथै॑नान्थ्स॒माव॑ती य॒ज्ञस्या॒\-ऽ\-ऽशीर्ग॑च्छति॒ यथा॒ वै प॒र्जन्यः॒ सुवृ॑ष्टं॒ वर्\mbox{}ष॑त्ये॒वं य॒ज्ञो यज॑मानाय वर्\mbox{}षति॒ स्थल॑योद॒कं प॑रिगृ॒ह्णन्त्या॒शिषा॑ य॒ज्ञं यज॑मानः॒ परि॑गृह्णाति॒ मनो॑\-ऽसि प्राजाप॒त्यं~(३७)

%1.6.10.6
मन॑सा मा भू॒तेना\-ऽ\-ऽवि॒शेत्या॑ह॒ मनो॒ वै प्रा॑जाप॒त्यं प्रा॑जाप॒त्यो य॒ज्ञो मन॑ ए॒व य॒ज्ञमा॒त्मन् ध॑त्ते॒ वाग॑स्यै॒न्द्री स॑पत्न॒क्षय॑णी वा॒चा मे᳚न्द्रि॒येणा\-ऽ\-ऽवि॒शेत्या॑है॒न्द्री वै वाग्वाच॑मे॒वैन्द्रीमा॒त्मन् ध॑त्ते॥~(३८)

%1.6.11.0
{\anuvakamend[{तेनै॒व ब्रह्म॑ रा॒ष्ट्रमे॒वास्य॑ य॒ज्ञस्य॑ प्राजाप॒त्यꣳ षट्त्रिꣳ॑शच्च}]}%॥10॥

%1.6.11.1
यो वै स॑प्तद॒शं प्र॒जा\-प॑तिं य॒ज्ञम॒न्वाय॑त्तं॒ वेद॒ प्रति॑ य॒ज्ञेन॑ तिष्ठति॒ न य॒ज्ञाद् भ्रꣳ॑शत॒ आ श्रा॑व॒येति॒ चतु॑रक्षर॒मस्तु॒ श्रौष॒डिति॒ चतु॑रक्षरं॒ यजेति॒ द्व्य॑क्षरं॒ ये यजा॑मह॒ इति॒ पञ्चा᳚क्षरं द्व्यक्ष॒रो व॑षट्का॒र ए॒ष वै स॑प्तद॒शः प्र॒जा\-प॑तिर्य॒ज्ञम॒न्वाय॑त्तो॒ य ए॒वं वेद॒ प्रति॑ य॒ज्ञेन॑ तिष्ठति॒ न य॒ज्ञाद् भ्रꣳ॑शते॒ यो वै य॒ज्ञस्य॒ प्राय॑णं प्रति॒ष्ठा-~(३९)

%1.6.11.2
मु॒दय॑नं॒ वेद॒ प्रति॑ष्ठिते॒नारि॑ष्टेन य॒ज्ञेन॑ स॒ꣴ॒स्थां ग॑च्छ॒त्या श्रा॑व॒यास्तु॒ श्रौष॒ड्यज॒ ये यजा॑महे वषट्का॒र ए॒तद्वै य॒ज्ञस्य॒ प्राय॑णमे॒षा प्र॑ति॒ष्ठैतदु॒दय॑नं॒ य ए॒वं वेद॒ प्रति॑ष्ठिते॒नारि॑ष्टेन य॒ज्ञेन॑ स॒ꣴ॒स्थां ग॑च्छति॒ यो वै सू॒नृता॑यै॒ दोहं॒ वेद॑ दु॒ह ए॒वैनां᳚ य॒ज्ञो वै सू॒नृता\-ऽ\-ऽश्रा॑व॒येत्यैवैना॑मह्व॒दस्तु॒~(४०)

%1.6.11.3
श्रौष॒डित्यु॒पावा᳚स्रा॒ग्यजेत्युद॑नैषी॒द्ये यजा॑मह॒ इत्युपा॑स\-दद्वषट्का॒रेण॑ दोग्ध्ये॒ष वै सू॒नृता॑यै॒ दोहो॒ य ए॒वं वेद॑ दु॒ह ए॒वैनां᳚ दे॒वा वै स॒त्रमा॑सत॒ तेषां॒ दिशो॑\-ऽदस्य॒न्त ए॒तामा॒र्द्रां प॒ङ्क्तिम॑पश्य॒न्ना श्रा॑व॒येति॑ पुरोवा॒तम॑जनय॒न्नस्तु॒ श्रौष॒डित्य॒ब्भ्रꣳ सम॑प्लावय॒न्॒ यजेति॑ वि॒द्युत॑-~(४१)

%1.6.11.4
मजनय॒न्॒ ये यजा॑मह॒ इति॒ प्राव॑र्\mbox{}षयन्न॒भ्य॑स्तनयन् वषट्का॒रेण॒ ततो॒ वै तेभ्यो॒ दिशः॒ प्राप्या॑यन्त॒ य ए॒वं वेद॒ प्रास्मै॒ दिशः॑ प्यायन्ते प्र॒जा\-प॑तिं त्वो॒वेद॑ प्र॒जा\-प॑तिस्त्वं वेद॒ यं प्र॒जा\-प॑ति॒र्वेद॒ स पुण्यो॑ भवत्ये॒ष वै छ॑न्द॒स्यः॑ प्र॒जा\-प॑ति॒रा श्रा॑व॒यास्तु॒ श्रौष॒ड्यज॒ ये यजा॑महे वषट्का॒रो य ए॒वं वेद॒ पुण्यो॑ भवति वस॒न्त-~(४२)

%1.6.11.5
मृ॑तू॒नां प्री॑णा॒मीत्या॑ह॒र्तवो॒ वै प्र॑या॒जा ऋ॒तूने॒व प्री॑णाति॒ ते᳚\-ऽस्मै प्री॒ता य॑थापू॒र्वं क॑ल्पन्ते॒ कल्प॑न्ते\-ऽस्मा ऋ॒तवो॒ य ए॒वं वेदा॒ग्नी\-षोम॑योर॒हं दे॑वय॒ज्यया॒ चक्षु॑ष्मान् भूयास॒मित्या॑हा॒ग्नी\-षोमा᳚भ्यां॒ वै य॒ज्ञश्चक्षु॑ष्मा॒न् ताभ्या॑मे॒व चक्षु॑रा॒त्मन् ध॑त्ते॒\-ऽग्नेर॒हं दे॑वय॒ज्यया᳚न्ना॒दो भू॑यास॒मित्या॑हा॒ग्निर्वै दे॒वाना॑मन्ना॒दस्तेनै॒वा-~(४३)

%1.6.11.6
ऽन्नाद्य॑मा॒त्मन् ध॑त्ते॒ दब्धि॑र॒स्यद॑ब्धो भूयासम॒मुं द॑भेय॒मित्या॑\-है॒तया॒ वै दब्ध्या॑ दे॒वा असु॑रानदभ्नुव॒न् तयै॒व भ्रातृ॑व्यं दभ्नोत्य॒ग्नी\-षोम॑यो\-र॒हं दे॑वय॒ज्यया॑ वृत्र॒हा भू॑यास॒मित्या॑हा॒ग्नी\-षोमा᳚भ्यां॒ वा इन्द्रो॑ वृ॒त्रम॑ह॒न् ताभ्या॑मे॒व भ्रातृ॑व्यꣴ स्तृणुत इन्द्राग्नि॒योर॒हं दे॑व\-य॒ज्यये᳚\-न्द्रिया॒व्य॑न्ना॒दो भू॑यास॒मित्या॑हेन्द्रिया॒व्ये॑वान्ना॒दो भ॑व॒तीन्द्र॑स्या॒-~(४४)

%1.6.11.7
ऽहं दे॑वय॒ज्यये᳚न्द्रिया॒वी भू॑यास॒मित्या॑हेन्द्रिया॒व्ये॑व भ॑वति महे॒न्द्रस्या॒हं दे॑वय॒ज्यया॑ जे॒मानं॑ महि॒मानं॑ गमेय॒मित्या॑ह जे॒मान॑मे॒व म॑हि॒मानं॑ गच्छत्य॒ग्नेः स्वि॑ष्ट॒कृतो॒\-ऽहं दे॑वय॒ज्यया\-ऽ\-ऽयु॑ष्मान् य॒ज्ञेन॑ प्रति॒ष्ठां ग॑मेय॒मित्या॒हाऽऽयु॑रे॒वात्मन् ध॑त्ते॒ प्रति॑ य॒ज्ञेन॑ तिष्ठति॥~(४५)

%1.6.12.0
{\anuvakamend[{प्र॒ति॒ष्ठाम॑ह्व॒दस्तु॑ वि॒द्युतं॑ वस॒न्तमे॒वेन्द्र॑स्या॒\-ऽष्टात्रिꣳ॑शच्च}]}%॥11॥

%1.6.12.1
इन्द्रं॑ वो वि॒श्वत॒स्परि॒ हवा॑महे॒ जने᳚भ्यः। अ॒स्माक॑मस्तु॒ केव॑लः॥ इन्द्रं॒ नरो॑ ने॒मधि॑ता हवन्ते॒ यत्पार्या॑ यु॒नज॑ते॒ धिय॒स्ताः। शूरो॒ नृषा॑ता॒ शव॑सश्चका॒न आ गोम॑ति व्र॒जे भ॑जा॒ त्वं नः॑॥ इ॒न्द्रि॒याणि॑ शतक्रतो॒ या ते॒ जने॑षु प॒ञ्चसु॑। इन्द्र॒ तानि॑ त॒ आ वृ॑णे॥ अनु॑ ते दायि म॒ह इ॑न्द्रि॒याय॑ स॒त्रा ते॒ विश्व॒मनु॑ वृत्र॒हत्ये᳚। अनु॑~(४६)

%1.6.12.2
क्ष॒त्रमनु॒ सहो॑ यज॒त्रेन्द्र॑ दे॒वेभि॒रनु॑ ते नृ॒षह्ये᳚॥ आ यस्मि᳚न्थ्स॒प्त वा॑स॒वास्तिष्ठ॑न्ति स्वा॒रुहो॑ यथा। ऋषि॑र्\mbox{}ह दीर्घ॒श्रुत्त॑म॒ इन्द्र॑स्य घ॒र्मो अति॑थिः॥ आ॒मासु॑ प॒क्वमैर॑य॒ आ सूर्यꣳ॑ रोहयो दि॒वि। घ॒र्मं न साम॑न्तपता सुवृ॒क्तिभि॒र्जुष्टं॒ गिर्व॑णसे॒ गिरः॑॥ इन्द्र॒मिद्गा॒थिनो॑ बृ॒हदिन्द्र॑म॒र्केभि॑र॒र्किणः॑। इन्द्रं॒ वाणी॑रनूषत॥ गाय॑न्ति त्वा \mbox{गाय॒त्रिणो-~(४७)}

%1.6.12.3
ऽर्च॑न्त्य॒र्कम॒र्किणः॑। ब्र॒ह्माण॑स्त्वा शतक्रत॒वुद्व॒ꣳ॒शमि॑व येमिरे॥ अ॒ꣳ॒हो॒मुचे॒ प्र भ॑रेमा मनी॒षामो॑षिष्ठ॒दाव्न्ने॑ सुम॒तिं गृ॑णा॒नाः। इ॒दमि॑न्द्र॒ प्रति॑ ह॒व्यं गृ॑भाय स॒त्याः स॑न्तु॒ यज॑मानस्य॒ कामाः᳚॥ वि॒वेष॒ यन्मा॑ धि॒षणा॑ ज॒जान॒ स्तवै॑ पु॒रा पार्या॒दिन्द्र॒मह्नः॑। अꣳह॑सो॒ यत्र॑ पी॒पर॒द्यथा॑ नो ना॒वेव॒ यान्त॑मु॒भये॑ हवन्ते॥ प्र स॒म्राजं॑ प्रथ॒मम॑ध्व॒राणा॑-~(४८)

%1.6.12.4
मꣳहो॒मुचं॑ वृष॒भं य॒ज्ञिया॑नाम्। अ॒पां नपा॑तमश्विना॒ हय॑न्तम॒स्मिन्न॑र इन्द्रि॒यं ध॑त्त॒मोजः॑॥ वि न॑ इन्द्र॒ मृधो॑ जहि नी॒चा य॑च्छ पृतन्य॒तः। अ॒ध॒स्प॒दं तमीं᳚ कृधि॒ यो अ॒स्माꣳ अ॑भि॒दास॑ति॥ इन्द्र॑ क्ष॒त्रम॒भि वा॒ममोजो\-ऽजा॑यथा वृषभ चर्\mbox{}षणी॒नाम्। अपा॑नुदो॒ जन॑ममित्र॒यन्त॑मु॒रुं दे॒वेभ्यो॑ अकृणोरु लो॒कम्॥ मृ॒गो न भी॒मः कु॑च॒रो गि॑रि॒ष्ठाः प॑रा॒वत॒~-~(४९)

%1.6.12.5
आ ज॑गामा॒ पर॑स्याः। सृ॒कꣳ स॒ꣳ॒शाय॑ प॒विमि॑न्द्र ति॒ग्मं वि शत्रू᳚न् ताढि॒ वि मृधो॑ नुदस्व॥ वि शत्रू॒न्॒ वि मृधो॑ नुद॒ वि वृ॒त्रस्य॒ हनू॑ रुज। वि म॒न्युमि॑न्द्र भामि॒तो॑\-ऽमित्र॑स्याभि॒दास॑तः॥ त्रा॒तार॒मिन्द्र॑मवि॒तार॒मिन्द्र॒ꣳ॒ हवे॑हवे सु॒हव॒ꣳ॒ शूर॒मिन्द्रम्᳚। हु॒वे नु श॒क्रं पु॑रुहू॒तमिन्द्रꣴ॑ स्व॒स्ति नो॑ म॒घवा॑ धा॒त्विन्द्रः॑॥ मा ते॑ अ॒स्याꣳ~(५०)

%1.6.12.6
स॑हसाव॒न् परि॑ष्टाव॒घाय॑ भूम हरिवः परा॒दै। त्राय॑स्व नो\-ऽवृ॒केभि॒र्वरू॑थै॒स्तव॑ प्रि॒यासः॑ सू॒रिषु॑ स्याम॥ अन॑वस्ते॒ रथ॒मश्वा॑य तक्ष॒न् त्वष्टा॒ वज्रं॑ पुरुहूत द्यु॒मन्तम्᳚। ब्र॒ह्माण॒ इन्द्रं॑ म॒हय॑न्तो अ॒र्कैरव॑र्धय॒न्नह॑ये॒ हन्त॒वा उ॑॥ वृष्णे॒ यत् ते॒ वृष॑णो अ॒र्कमर्चा॒निन्द्र॒ ग्रावा॑णो॒ अदि॑तिः स॒जोषाः᳚। अ॒न॒श्वासो॒ ये प॒वयो॑\-ऽर॒था इन्द्रे॑षिता अ॒भ्यव॑र्तन्त॒ दस्यून्॑॥~(५१)

{\anuvakamend[{वृ॒त्र॒हत्ये\-ऽनु॑ गाय॒त्रिणो᳚\-ऽध्व॒राणां᳚ परा॒वतो॒\-ऽस्याम॒ष्टाच॑त्वारिꣳशच्च}]}%॥12॥
%%% END PRASHNA

\sect{सप्तमः प्रश्नः}\setcounter{anuvakam}{0}
\dnsub{तैत्तिरीयसंहितायां प्रथमकाण्डे सप्तमः प्रश्नः}
%1.7.1.0
%1.7.1.1
पा॒क॒य॒ज्ञं वा अन्वाहि॑ताग्नेः प॒शव॒ उप॑ तिष्ठन्त॒ इडा॒ खलु॒ वै पा॑कय॒ज्ञः सैषाऽन्त॒रा प्र॑याजानूया॒जान् यज॑मानस्य लो॒के\-ऽव॑हिता॒ तामा᳚ह्रि॒यमा॑णाम॒भि म॑न्त्रयेत॒ सुरू॑पवर्\mbox{}षवर्ण॒ एहीति॑ प॒शवो॒ वा इडा॑ प॒शूने॒वोप॑ ह्वयते य॒ज्ञं वै दे॒वा अदु॑ह्रन् य॒ज्ञो\-ऽसु॑राꣳ अदुह॒त् ते\-ऽसु॑रा य॒ज्ञदु॑ग्धाः॒ परा॑\-ऽभव॒न्॒ यो वै य॒ज्ञस्य॒ दोहं॑ वि॒द्वान्~(१)

%1.7.1.2
यज॒ते\-ऽप्य॒न्यं यज॑मानं दुहे॒ सा मे॑ स॒त्या\-ऽ\-ऽशीर॒स्य य॒ज्ञस्य॑ भूया॒दित्या॑है॒ष वै य॒ज्ञस्य॒ दोह॒स्तेनै॒वैनं॑ दुहे॒ प्रत्ता॒ वै गौर्दु॑हे॒ प्रत्तेडा॒ यज॑मानाय दुह ए॒ते वा इडा॑यै॒ स्तना॒ इडोप॑हू॒तेति॑ वा॒युर्व॒थ्सो यर्\mbox{}हि॒ होतेडा॑मुप॒ह्वये॑त॒ तर्\mbox{}हि॒ यज॑मानो॒ होता॑र॒मीक्ष॑माणो वा॒युं मन॑सा ध्यायेन्~(२)

%1.7.1.3
मा॒त्रे व॒थ्समु॒पाव॑सृजति॒ सर्वे॑ण॒ वै य॒ज्ञेन॑ दे॒वाः सु॑व॒र्गं लो॒कमा॑यन् पाकय॒ज्ञेन॒ मनु॑रश्राम्य॒थ्सेडा॒ मनु॑मु॒पाव॑र्तत॒ तान्दे॑वासु॒रा व्य॑ह्वयन्त प्र॒तीचीं᳚ दे॒वाः परा॑ची॒मसु॑राः॒ सा दे॒वानु॒पाव॑र्तत प॒शवो॒ वै तद्दे॒वान॑वृणत प॒शवो\-ऽसु॑रानजहु॒र्यं का॒मये॑ताप॒शुः स्या॒दिति॒ परा॑चीं॒ तस्येडा॒मुप॑ह्वयेताप॒शुरे॒व भ॑वति॒ यं~(३)

%1.7.1.4
का॒मये॑त पशु॒मान्थ्स्या॒दिति॑ प्र॒तीचीं॒ तस्येडा॒मुप॑ह्वयेत पशु॒माने॒व भ॑वति ब्रह्मवा॒दिनो॑ वदन्ति॒ स त्वा इडा॒मुप॑ह्वयेत॒ य इडा॑मुप॒हूया॒\-ऽ\-ऽत्मान॒मिडा॑यामुप॒ह्वये॒तेति॒ सा नः॑ प्रि॒या सु॒प्रतू᳚र्तिर्म॒घोनीत्या॒हेडा॑मे॒वोप॒हूया॒\-ऽ\-ऽत्मान॒मिडा॑या॒मुप॑ ह्वयते॒ व्य॑स्तमिव॒ वा ए॒तद्य॒ज्ञस्य॒ यदिडा॑ सा॒मि प्रा॒श्ञन्ति॑~(४)

%1.7.1.5
सा॒मि मा᳚र्जयन्त ए॒तत् प्रति॒ वा असु॑राणां य॒ज्ञो व्य॑च्छिद्यत॒ ब्रह्म॑णा दे॒वाः सम॑दधु॒र्बृह॒स्पति॑स्तनुतामि॒मं न॒ इत्या॑ह॒ ब्रह्म॒ वै दे॒वानां॒ बृह॒स्पति॒र्ब्रह्म॑णै॒व य॒ज्ञꣳ सन्द॑धाति॒ विच्छि॑न्नं य॒ज्ञꣳ समि॒मं द॑धा॒त्वित्या॑ह॒ सन्त॑त्यै॒ विश्वे॑ दे॒वा इ॒ह मा॑दयन्ता॒मित्या॑ह स॒न्तत्यै॒व य॒ज्ञं दे॒वेभ्यो\-ऽनु॑ दिशति॒ यां वै~(५)


%1.7.1.6
य॒ज्ञे दक्षि॑णां॒ ददा॑ति॒ ताम॑स्य प॒शवो\-ऽनु॒ सङ्क्रा॑मन्ति॒ स ए॒ष ई॑जा॒नो॑\-ऽप॒शुर्भावु॑को॒ यज॑मानेन॒ खलु॒ वै तत्का॒र्य॑मित्या॑हु॒र्यथा॑ देव॒त्रा द॒त्तं कु॑र्वी॒ताऽऽत्मन् प॒शून् र॒मये॒तेति॒ ब्रध्न॒ पिन्व॒स्वेत्या॑ह य॒ज्ञो वै ब्र॒ध्नो य॒ज्ञमे॒व तन्म॑हय॒त्यथो॑ देव॒त्रैव द॒त्तं कु॑रुत आ॒त्मन् प॒शून् र॑मयते॒ दद॑तो मे॒ मा क्षा॒यीत्या॒हाक्षि॑तिमे॒वोपै॑ति कुर्व॒तो मे॒ मोप॑ दस॒दित्या॑ह भू॒मान॑मे॒वोपै॑ति॥~(६)

%1.7.2.0
{\anuvakamend[{वि॒द्वान्ध्या॑ये॒ द्यं प्रा॒श्ञन्ति॒ यां वै म॒ एका॒न्नविꣳ॑श॒तिश्च॑}]}%~(१)

%1.7.2.1
सꣴश्र॑वा ह सौवर्चन॒सस्तुमि॑ञ्ज॒मौपो॑दितिमुवाच॒ यथ्स॒त्रिणा॒ꣳ॒ होता\-ऽभूः॒ कामिडा॒मुपा᳚ह्वथा॒ इति॒ तामुपा᳚ह्व॒ इति॑ होवाच॒ या प्रा॒णेन॑ दे॒वान् दा॒धार॑ व्या॒नेन॑ मनु॒ष्या॑नपा॒नेन॑ पि॒तॄनिति॑ छि॒नत्ति॒ सा न छि॑न॒त्ती~(३) इति॑ छि॒नत्तीति॑ होवाच॒ शरी॑रं॒ वा अ॑स्यै॒ तदुपा᳚ह्वथा॒ इति॑ होवाच॒ गौर्वा~(७)

%1.7.2.2
अ॑स्यै॒ शरी॑रं॒ गां वाव तौ तत्पर्य॑वदतां॒ या य॒ज्ञे दी॒यते॒ सा प्रा॒णेन॑ दे॒वान् दा॑धार॒ यया॑ मनु॒ष्या॑ जीव॑न्ति॒ सा व्या॒नेन॑ मनु॒ष्यान्॑ यां पि॒तृभ्यो॒ घ्नन्ति॒ सा\-ऽपा॒नेन॑ पि॒तॄन् य ए॒वं वेद॑ पशु॒मान् भ॑व॒त्यथ॒ वै तामुपा᳚ह्व॒ इति॑ होवाच॒ या प्र॒जाः प्र॒भव॑न्तीः॒ प्रत्या॒भव॒तीत्यन्नं॒ वा अ॑स्यै॒ त-~(८)

%1.7.2.3
दुपा᳚ह्वथा॒ इति॑ होवा॒चौष॑धयो॒ वा अ॑स्या॒ अन्न॒मोष॑धयो॒ वै प्र॒जाः प्र॒भव॑न्तीः॒ प्रत्या भ॑वन्ति॒ य ए॒वं वेदा᳚न्ना॒दो भ॑व॒त्यथ॒ वै तामुपा᳚ह्व॒ इति॑ होवाच॒ या प्र॒जाः प॑रा॒भव॑न्तीरनुगृ॒ह्णाति॒ प्रत्या॒भव॑न्तीर्गृ॒ह्णातीति॑ प्रति॒ष्ठां वा अ॑स्यै॒ तदुपा᳚ह्वथा॒ इति॑ होवाचे॒यं वा अ॑स्यै प्रति॒ष्ठे-~(९)

%1.7.2.4
यं वै प्र॒जाः प॑रा॒भव॑न्ती॒रनु॑गृह्णाति॒ प्रत्या॒भव॑न्तीर्गृह्णाति॒ य ए॒वं वेद॒ प्रत्ये॒व ति॑ष्ठ॒त्यथ॒ वै तामुपा᳚ह्व॒ इति॑ होवाच॒ यस्यै॑ नि॒क्रम॑णे घृ॒तं प्र॒जाः स॒ञ्जीव॑न्तीः॒ पिब॒न्तीति॑ छि॒नत्ति॒ सा न छि॑न॒त्ती~(३) इति॒ न छि॑न॒त्तीति॑ होवाच॒ प्र तु ज॑नय॒तीत्ये॒ष वा इडा॒मुपा᳚ह्वथा॒ इति॑ होवाच॒ वृष्टि॒र्वा इडा॒ वृष्ट्यै॒ वै नि॒क्रम॑णे घृ॒तं प्र॒जाः स॒ञ्जीव॑न्तीः पिबन्ति॒ य ए॒वं वेद॒ प्रैव जा॑यते\-ऽन्ना॒दो भ॑वति॥~(१०)

%1.7.3.0
{\anuvakamend[{गौर्वा अ॑स्यै॒ तत् प्र॑ति॒ष्ठा\-ऽह्व॑था॒ इति॑ विꣳश॒तिश्च॑}]}%~(२)

%1.7.3.1
प॒रोक्षं॒ वा अ॒न्ये दे॒वा इ॒ज्यन्ते᳚ प्र॒त्यक्ष॑म॒न्ये यद्यज॑ते॒ य ए॒व दे॒वाः प॒रोक्ष॑मि॒ज्यन्ते॒ ताने॒व तद्य॑जति॒ यद॑न्वाहा॒र्य॑मा॒हर॑त्ये॒ते वै दे॒वाः प्र॒त्यक्षं॒ यद् ब्रा᳚ह्म॒णास्ताने॒व तेन॑ प्रीणा॒त्यथो॒ दक्षि॑णै॒वास्यै॒षा\-ऽथो॑ य॒ज्ञस्यै॒व छि॒द्रमपि॑ दधाति॒ यद्वै य॒ज्ञस्य॑ क्रू॒रं यद्विलि॑ष्टं॒ तद॑न्वाहा॒र्ये॑णा॒-~(११)

%1.7.3.2
न्वाह॑रति॒ तद॑न्वाहा॒र्य॑स्यान्वाहार्य॒त्वं दे॑वदू॒ता वा ए॒ते यदृ॒त्विजो॒ यद॑न्वाहा॒र्य॑मा॒हर॑ति देवदू॒ताने॒व प्री॑णाति प्र॒जा\-प॑तिर्दे॒वेभ्यो॑ य॒ज्ञान् व्यादि॑श॒थ्स रि॑रिचा॒नो॑\-ऽमन्यत॒ स ए॒तम॑न्वाहा॒र्य॑मभ॑क्तमपश्य॒त् तमा॒त्मन्न॑धत्त॒ स वा ए॒ष प्रा॑जाप॒त्यो यद॑न्वाहा॒र्यो॑ यस्यै॒वं वि॒दुषो᳚\-ऽन्वाहा॒र्य॑ आह्रि॒यते॑ सा॒क्षादे॒व प्र॒जा\-प॑तिमृध्नो॒त्यप॑रिमितो नि॒रुप्यो\-ऽप॑रिमितः प्र॒जा\-प॑तिः प्र॒जा\-प॑ते॒-~(१२)

%1.7.3.3
राप्त्यै॑ दे॒वा वै यद्य॒ज्ञे\-ऽकु॑र्वत॒ तदसु॑रा अकुर्वत॒ ते दे॒वा ए॒तं प्रा॑जाप॒त्यम॑न्वाहा॒र्य॑मपश्य॒न् तम॒न्वाह॑रन्त॒ ततो॑ दे॒वा अभ॑व॒न् परासु॑रा॒ यस्यै॒वं वि॒दुषो᳚\-ऽन्वाहा॒र्य॑ आह्रि॒यते॒ भव॑त्या॒त्मना॒ परा᳚स्य॒ भ्रातृ॑व्यो भवति य॒ज्ञेन॒ वा इ॒ष्टी प॒क्वेन॑ पू॒र्ती यस्यै॒वं वि॒दुषो᳚\-ऽन्वाहा॒र्य॑ आह्रि॒यते॒ स त्वे॑वेष्टा॑पू॒र्ती प्र॒जा\-प॑तेर्भा॒गो॑\-\mbox{ऽसी-~(१३)}

%1.7.3.4
त्या॑ह प्र॒जा\-प॑तिमे॒व भा॑ग॒धेये॑न॒ सम॑र्धय॒त्यूर्ज॑स्वा॒न् पय॑स्वा॒नित्या॒होर्ज॑मे॒वास्मि॒न् पयो॑ दधाति प्राणापा॒नौ मे॑ पाहि समानव्या॒नौ मे॑ पा॒हीत्या॑हा॒\-ऽ\-ऽशिष॑मे॒वैतामा शा॒स्ते\-ऽक्षि॑तो॒\-ऽस्यक्षि॑त्यै त्वा॒ मा मे᳚ क्षेष्ठा अ॒मुत्रा॒मुष्मिँ॑ल्लो॒क इत्या॑ह॒ क्षीय॑ते॒ वा अ॒मुष्मिँ॑ल्लो॒के\-ऽन्न॑मि॒तः प्र॑दान॒ꣴ॒ ह्य॑मुष्मिँ॑ल्लो॒के प्र॒जा उ॑प॒जीव॑न्ति॒ यदे॒वम॑भिमृ॒शत्यक्षि॑तिमे॒वैन॑द्गमयति॒ नास्या॒मुष्मिँ॑ल्लो॒के\-ऽन्नं॑ क्षीयते॥~(१४)

%1.7.4.0
{\anuvakamend[{अ॒न्वा॒हा॒र्ये॑ण प्र॒जा\-प॑तेरसि॒ ह्य॑मुष्मिँ॑ल्लो॒के पञ्च॑दश च}]}%~(३)

%1.7.4.1
ब॒र्॒\mbox{}हिषो॒\-ऽहं दे॑वय॒ज्यया᳚ प्र॒जावा᳚न् भूयास॒मित्या॑ह ब॒र्॒\mbox{}हिषा॒ वै प्र॒जा\-प॑तिः प्र॒जा अ॑सृजत॒ तेनै॒व प्र॒जाः सृ॑जते॒ नरा॒शꣳस॑स्या॒हं दे॑वय॒ज्यया॑ पशु॒मान् भू॑यास॒मित्या॑ह॒ नरा॒शꣳसे॑न॒ वै प्र॒जा\-प॑तिः प॒शून॑सृजत॒ तेनै॒व प॒शून्थ्सृ॑जते॒\-ऽग्नेः स्वि॑ष्ट॒कृतो॒\-ऽहं दे॑वय॒ज्यया\-ऽ\-ऽयु॑ष्मान् य॒ज्ञेन॑ प्रति॒ष्ठां ग॑मेय॒मित्या॒हा\-ऽ\-ऽयु॑रे॒वात्मन् ध॑त्ते॒ प्रति॑ य॒ज्ञेन॑ तिष्ठति दर्\mbox{}श\-पूर्ण\-मा॒सयो॒र्-~(१५)

%1.7.4.2
र्वै दे॒वा उज्जि॑ति॒मनूद॑जयन् दर्\mbox{}श\-पूर्ण\-मा॒साभ्या॒मसु॑रा॒नपा॑\-नुदन्ता॒ग्नेर॒ह\-मुज्जि॑ति॒\-मनूज्जे॑ष॒मित्या॑ह दर्\mbox{}श\-पूर्ण\-मा॒सयो॑रे॒व दे॒वता॑नां॒ यज॑मान॒ उज्जि॑ति॒मनूज्ज॑यति दर्\mbox{}श\-पूर्ण\-मा॒साभ्यां॒ भ्रातृ॑व्या॒नप॑ नुदते॒ वाज॑वतीभ्यां॒ व्यू॑ह॒त्यन्नं॒ वै वाजो\-ऽन्न॑मे॒वाव॑\-रुन्धे॒ द्वाभ्यां॒ प्रति॑ष्ठित्यै॒ यो वै य॒ज्ञस्य॒ द्वौ दोहौ॑ वि॒द्वान् यज॑त उभ॒यत॑~-~(१६)

%1.7.4.3
ए॒व य॒ज्ञं दु॑हे पु॒रस्ता᳚च्चो॒परि॑ष्टाच्चै॒ष वा अ॒न्यो य॒ज्ञस्य॒ दोह॒ इडा॑याम॒न्यो यर्\mbox{}हि॒ होता॒ यज॑मानस्य॒ नाम॑ गृह्णी॒यात् तर्\mbox{}हि॑ ब्रूया॒देमा अ॑ग्मन्ना॒शिषो॒ दोह॑कामा॒ इति॒ सꣴस्तु॑ता ए॒व दे॒वता॑ दु॒हे\-ऽथो॑ उभ॒यत॑ ए॒व य॒ज्ञं दु॑हे पु॒रस्ता᳚च्चो॒परि॑ष्टाच्च॒ रोहि॑तेन त्वा॒\-ऽग्निर्दे॒वतां᳚ गमय॒त्वित्या॑है॒ते वै दे॑वा॒श्वा~-~(१७)

%1.7.4.4
यज॑मानः प्रस्त॒रो यदे॒तैः प्र॑स्त॒रं प्र॒हर॑ति देवा॒श्वैरे॒व यज॑मानꣳ सुव॒र्गं लो॒कं ग॑मयति॒ वि ते॑ मुञ्चामि रश॒ना वि र॒श्मीनित्या॑है॒ष वा अ॒ग्नेर्वि॑मो॒कस्तेनै॒वैनं॒ वि मु॑ञ्चति॒ विष्णोः᳚ शं॒योर॒हं दे॑वय॒ज्यया॑ य॒ज्ञेन॑ प्रति॒ष्ठां ग॑मेय॒मित्या॑ह य॒ज्ञो वै विष्णु॑र्य॒ज्ञ ए॒वान्त॒तः प्रति॑ तिष्ठति॒ सोम॑स्या॒हं दे॑वय॒ज्यया॑ सु॒रेता॒~-~(१८)

%1.7.4.5
रेतो॑ धिषी॒येत्या॑ह॒ सोमो॒ वै रे॑तो॒धास्तेनै॒व रेत॑ आ॒त्मन् ध॑त्ते॒ त्वष्टु॑र॒हं दे॑वय॒ज्यया॑ पशू॒नाꣳ रू॒पं पु॑षेय॒मित्या॑ह॒ त्वष्टा॒ वै प॑शू॒नां मि॑थु॒नानाꣳ॑ रूप॒कृत्तेनै॒व प॑शू॒नाꣳ रू॒पमा॒त्मन् ध॑त्ते दे॒वानां॒ पत्नी॑र॒ग्निर्गृ॒हप॑तिर्य॒ज्ञस्य॑ मिथु॒नं तयो॑र॒हं दे॑वय॒ज्यया॑ मिथु॒नेन॒ प्र भू॑यास॒मित्या॑है॒तस्मा॒द्वै मि॑थु॒नात्प्र॒जा\-प॑तिर्मिथु॒नेन॒~(१९)

%1.7.4.6
प्राजा॑यत॒ तस्मा॑दे॒व यज॑मानो मिथु॒नेन॒ प्र जा॑यते वे॒दो॑\-ऽसि॒ वित्ति॑रसि वि॒देयेत्या॑ह वे॒देन॒ वै दे॒वा असु॑राणां वि॒त्तं वेद्य॑मविन्दन्त॒ तद्वे॒दस्य॑ वेद॒त्वं यद्य॒द् भ्रातृ॑व्यस्याभि॒ध्याये॒त् तस्य॒ नाम॑ गृह्णीया॒त् तदे॒वास्य॒ सर्वं॑ वृङ्क्ते घृ॒तव॑न्तं कुला॒यिनꣳ॑ रा॒यस्पोषꣳ॑ सह॒स्रिणं॑ वे॒दो द॑दातु वा॒जिन॒मित्या॑ह॒ प्र स॒हस्रं॑ प॒शूना᳚प्नो॒त्यास्य॑ प्र॒जायां᳚ वा॒जी जा॑यते॒ य ए॒वं वेद॑॥~(२०)

%1.7.5.0
{\anuvakamend[{द॒र्॒\mbox{}श॒पू॒र्ण॒मा॒सयो॑रुभ॒यतो॑ देवा॒श्वाः सु॒रेताः᳚ प्र॒जा\-प॑तिर्मिथु॒नेना᳚\-ऽ\-ऽप्नोत्य॒ष्टौ च॑}]}%~(४)

%1.7.5.1
ध्रु॒वां वै रिच्य॑मानां य॒ज्ञो\-ऽनु॑ रिच्यते य॒ज्ञं यज॑मानो॒ यज॑मानं प्र॒जा ध्रु॒वामा॒प्याय॑मानां य॒ज्ञो\-ऽन्वा प्या॑यते य॒ज्ञं यज॑मानो॒ यज॑मानं प्र॒जा आ प्या॑यतां ध्रु॒वा घृ॒तेनेत्या॑ह ध्रु॒वामे॒वा\-ऽ\-ऽप्या॑ययति॒ तामा॒प्याय॑मानां य॒ज्ञो\-ऽन्वा प्या॑यते य॒ज्ञं यज॑मानो॒ यज॑मानं प्र॒जाः प्र॒जा\-प॑तेर्वि॒भान्नाम॑ लो॒कस्तस्मिꣴ॑ स्त्वा दधामि स॒ह यज॑माने॒ने-~(२१)

%1.7.5.2
त्या॑हा॒यं वै प्र॒जा\-प॑तेर्वि॒भान्नाम॑ लो॒कस्तस्मि॑न्ने॒वैनं॑ दधाति स॒ह यज॑मानेन॒ रिच्य॑त इव॒ वा ए॒तद्यद्यज॑ते॒ यद्य॑जमानभा॒गं प्रा॒श्ञात्या॒त्मान॑मे॒व प्री॑णात्ये॒तावा॒न्॒ वै य॒ज्ञो यावान्॑ यजमानभा॒गो य॒ज्ञो यज॑मानो॒ यद्य॑जमानभा॒गं प्रा॒श्ञाति॑ य॒ज्ञ ए॒व य॒ज्ञं प्रति॑ष्ठापयत्ये॒तद्वै सू॒यव॑स॒ꣳ॒ सोद॑कं॒ यद्ब॒र्॒\mbox{}हिश्चा\-ऽ\-ऽप॑श्चै॒तद्~(२२)

%1.7.5.3
यज॑मानस्या॒\-ऽ\-ऽयत॑नं॒ यद्वेदि॒र्यत् पू᳚र्णपा॒त्रम॑न्तर्वे॒दि नि॒नय॑ति॒ स्व ए॒वा\-ऽ\-ऽयत॑ने सू॒यव॑स॒ꣳ॒ सोद॑कं कुरुते॒ सद॑सि॒ सन्मे॑ भूया॒ इत्या॒हा\-ऽ\-ऽपो॒ वै य॒ज्ञ आपो॒\-ऽमृतं॑ य॒ज्ञमे॒वामृत॑मा॒त्मन्ध॑त्ते॒ सर्वा॑णि॒ वै भू॒तानि॑ व्र॒तमु॑प॒यन्त॒मनूप॑ यन्ति॒ प्राच्यां᳚ दि॒शि दे॒वा ऋ॒त्विजो॑ मार्जयन्ता॒मित्या॑है॒ष वै द॑र्\mbox{}शपूर्णमा॒सयो॑रवभृ॒थो~(२३)

%1.7.5.4
यान्ये॒वैनं॑ भू॒तानि॑ व्र॒तमु॑प॒यन्त॑मनूप॒यन्ति॒ तैरे॒व स॒हाव॑भृ॒थमवै॑ति॒ विष्णु॑मुखा॒ वै दे॒वाश्छन्दो॑भिरि॒माँल्लो॒कान॑नप\-ज॒य्यम॒भ्य॑जय॒न्॒ यद्वि॑ष्णुक्र॒मान् क्रम॑ते॒ विष्णु॑रे॒व भू॒त्वा यज॑मान॒श्छन्दो॑भि\-रि॒माँल्लो॒का\-न॑नप\-ज॒य्य\-म॒भि ज॑यति॒ विष्णोः॒ क्रमो᳚\-ऽस्यभिमाति॒हेत्या॑ह गाय॒त्री वै पृ॑थि॒वी त्रैष्टु॑भम॒न्तरि॑क्षं॒ जाग॑ती॒ द्यौरानु॑ष्टुभी॒र्दिश॒श्छन्दो॑\-भिरे॒वेमाँल्लो॒कान् य॑थापू॒र्वम॒भि ज॑यति॥~(२४)

%1.7.6.0
{\anuvakamend[{इत्ये॒तद॑वभृ॒थो दिशः॑ स॒प्त च॑}]}%~(५)

%1.7.6.1
अग॑न्म॒ सुवः॒ सुव॑रग॒न्मेत्या॑ह सुव॒र्गमे॒व लो॒कमे॑ति स॒न्दृश॑स्ते॒ मा छि॑थ्सि॒ यत्ते॒ तप॒स्तस्मै॑ ते॒ मा वृ॒क्षीत्या॑ह यथाय॒जुरे॒वैतथ्सु॒भूर॑सि॒ श्रेष्ठो॑ रश्मी॒नामा॑यु॒र्धा अ॒स्यायु॑र्मे धे॒हीत्या॑हा॒\-ऽ\-ऽशिष॑मे॒वैतामा शा᳚स्ते॒ प्र वा ए॒षो᳚\-ऽस्माल्लो॒काच्च्य॑वते॒ यो~(२५)

%1.7.6.2
वि॑ष्णुक्र॒मान् क्रम॑ते सुव॒र्गाय॒ हि लो॒काय॑ विष्णुक्र॒माः क्र॒म्यन्ते᳚ ब्रह्मवा॒दिनो॑ वदन्ति॒ स त्वै वि॑ष्णुक्र॒मान् क्र॑मेत॒ य इ॒माँल्लो॒कान् भ्रातृ॑व्यस्य सं॒विद्य॒ पुन॑रि॒मं लो॒कं प्र॑त्यव॒रोहे॒दित्ये॒ष वा अ॒स्य लो॒कस्य॑ प्रत्यवरो॒हो यदाहे॒दम॒हम॒मुं भ्रातृ॑व्यमा॒भ्यो दि॒ग्भ्यो᳚\-ऽस्यै दि॒व इती॒माने॒व लो॒कान्भ्रातृ॑व्यस्य सं॒विद्य॒ पुन॑रि॒मं लो॒कं प्र॒त्यव॑रोहति॒ सं~(२६)

%1.7.6.3
ज्योति॑षा\-ऽभूव॒मित्या॑हा॒स्मिन्ने॒व लो॒के प्रति॑ तिष्ठत्यै॒न्द्रीमा॒\-वृत॑म॒न्वाव॑र्त॒ इत्या॑हा॒सौ वा आ॑दि॒त्य इन्द्र॒स्तस्यै॒वा\-ऽ\-ऽ\-वृत॒मनु॑ प॒र्याव॑र्तते दक्षि॒णा प॒र्याव॑र्तते॒ स्वमे॒व वी॒र्य॑मनु॑ प॒र्याव॑र्तते॒ तस्मा॒द्दक्षि॒णो\-ऽर्ध॑ आ॒त्मनो॑ वी॒र्या॑वत्त॒रो\-ऽथो॑ आदि॒त्यस्यै॒वा\-ऽ\-ऽवृत॒मनु॑ प॒र्याव॑र्तते॒ सम॒हं प्र॒जया॒ सं मया᳚ प्र॒जेत्या॑हा॒\-ऽ\-ऽशिष॑-~(२७)

%1.7.6.4
मे॒वैतामा शा᳚स्ते॒ समि॑द्धो अग्ने मे दीदिहि समे॒द्धा ते॑ अग्ने दीद्यास॒मित्या॑ह यथाय॒जुरे॒वैतद्वसु॑मान् य॒ज्ञो वसी॑यान् भूयास॒मित्या॑हा॒\-ऽ\-ऽशिष॑मे॒वैतामा शा᳚स्ते ब॒हु वै गार्\mbox{}ह॑पत्य॒स्यान्ते॑ मि॒श्रमि॑व चर्यत आग्निपावमा॒नीभ्यां॒ गार्\mbox{}ह॑पत्य॒मुप॑ तिष्ठते पु॒नात्ये॒वाग्निं पु॑नी॒त आ॒त्मानं॒ द्वाभ्यां॒ प्रति॑ष्ठित्या॒ अग्ने॑ गृहपत॒ इत्या॑ह~(२८)

%1.7.6.5
यथाय॒जुरे॒वैतच्छ॒तꣳ हिमा॒ इत्या॑ह श॒तं त्वा॑ हेम॒न्तानि॑न्धिषी॒येति॒ वावैतदा॑ह पु॒त्रस्य॒ नाम॑ गृह्णात्यन्ना॒दमे॒वैनं॑ करोति॒ तामा॒शिष॒मा शा॑से॒ तन्त॑वे॒ ज्योति॑ष्मती॒मिति॑ ब्रूया॒द्यस्य॑ पु॒त्रो\-ऽजा॑तः॒ स्यात् ते॑ज॒स्व्ये॑वास्य॑ ब्रह्मवर्च॒सी पु॒त्रो जा॑यते॒ तामा॒शिष॒मा शा॑से॒\-ऽमुष्मै॒ ज्योति॑ष्मती॒मिति॑ ब्रूया॒द्यस्य॑ पु॒त्रो~(२९)

%1.7.6.6
जा॒तः स्यात् तेज॑ ए॒वास्मि॑न् ब्रह्मवर्च॒सं द॑धाति॒ यो वै य॒ज्ञं प्र॒युज्य॒ न वि॑मु॒ञ्चत्य॑प्रतिष्ठा॒नो वै स भ॑वति॒ कस्त्वा॑ युनक्ति॒ स त्वा॒ वि मु॑ञ्च॒त्वित्या॑ह प्र॒जा\-प॑ति॒र्वै कः प्र॒जा\-प॑तिनै॒वैनं॑ यु॒नक्ति॑ प्र॒जा\-प॑तिना॒ वि मु॑ञ्चति॒ प्रति॑ष्ठित्या ईश्व॒रं वै व्र॒तमवि॑सृष्टं प्र॒दहो\-ऽग्ने᳚ व्रतपते व्र॒तम॑चारिष॒मित्या॑ह व्र॒तमे॒व~(३०)

%1.7.6.7
वि सृ॑जते॒ शान्त्या॒ अप्र॑दाहाय॒ परा॒ङ्॒ वाव य॒ज्ञ ए॑ति॒ न नि व॑र्तते॒ पुन॒र्यो वै य॒ज्ञस्य॑ पुनराल॒म्भं वि॒द्वान् यज॑ते॒ तम॒भि नि व॑र्तते य॒ज्ञो ब॑भूव॒ स आ ब॑भू॒वेत्या॑है॒ष वै य॒ज्ञस्य॑ पुनराल॒म्भस्तेनै॒वैनं॒ पुन॒राल॑भ॒ते\-ऽन॑वरुद्धा॒ वा ए॒तस्य॑ वि॒राड्य आहि॑ताग्निः॒ सन्न॑स॒भः प॒शवः॒ खलु॒ वै ब्रा᳚ह्म॒णस्य॑ स॒भेष्ट्वा प्राङु॒त्क्रम्य॑ ब्रूया॒द्गोमाꣳ॑ अ॒ग्ने\-ऽवि॑माꣳ अ॒श्वी य॒ज्ञ इत्यव॑ स॒भाꣳ रु॒न्धे प्र स॒हस्रं॑ प॒शूना᳚प्नो॒त्यास्य॑ प्र॒जायां᳚ वा॒जी जा॑यते॥~(३१)

%1.7.7.0
{\anuvakamend[{यः स मा॒शिषं॑ गृहपत॒ इत्या॑ह॒ यस्य॑ पु॒त्रो व्र॒तमे॒व खलु॒ वै चतु॑र्विꣳशतिश्च}]}%~(६)

%1.7.7.1
देव॑ सवितः॒ प्रसु॑व य॒ज्ञं प्रसु॑व य॒ज्ञप॑तिं॒ भगा॑य दि॒व्यो ग॑न्ध॒र्वः। के॒त॒पूः केतं॑ नः पुनातु वा॒चस्पति॒र्वाच॑म॒द्य स्व॑दाति नः॥ इन्द्र॑स्य॒ वज्रो॑\-ऽसि॒ वार्त्र॑घ्न॒स्त्वया॒\-ऽयं वृ॒त्रं व॑ध्यात्॥ वाज॑स्य॒ नु प्र॑स॒वे मा॒तरं॑ म॒हीमदि॑तिं॒ नाम॒ वच॑सा करामहे। यस्या॑मि॒दं विश्वं॒ भुव॑नमावि॒वेश॒ तस्यां᳚ नो दे॒वः स॑वि॒ता धर्म॑ साविषत्॥ अ॒-~(३२)

%1.7.7.2
फ्स्व॑न्तर॒मृत॑म॒फ्सु भे॑ष॒जम॒पामु॒त प्रश॑स्ति॒ष्वश्वा॑ भवथ वाजिनः॥ वा॒युर्वा᳚ त्वा॒ मनु॑र्वा त्वा गन्ध॒र्वाः स॒प्तविꣳ॑शतिः। ते अग्रे॒ अश्व॑मायुञ्ज॒न्ते अ॑स्मिञ्ज॒वमाद॑धुः॥ अपां᳚ नपादाशुहेम॒न्॒ य ऊ॒र्मिः क॒कुद्मा॒न् प्रतू᳚र्तिर्वाज॒सात॑म॒स्तेना॒यं वाजꣳ॑ सेत्॥ विष्णोः॒ क्रमो॑\-ऽसि॒ विष्णोः᳚ क्रा॒न्तम॑सि॒ विष्णो॒र्विक्रा᳚न्तमस्य॒ङ्कौ न्य॒ङ्काव॒भितो॒ रथं॒ यौ ध्वा॒न्तं वा॑ता॒ग्रमनु॑ स॒ञ्चर॑न्तौ दू॒रेहे॑तिरिन्द्रि॒यावा᳚न्पत॒त्री ते नो॒\-ऽग्नयः॒ पप्र॑यः पारयन्तु॥~(३३)

%1.7.8.0
{\anuvakamend[{अ॒फ्सु न्य॒ङ्कौ पञ्च॑दश च}]}%~(७)

%1.7.8.1
दे॒वस्या॒हꣳ स॑वि॒तुः प्र॑स॒वे बृह॒स्पति॑ना वाज॒जिता॒ वाजं॑ जेषं दे॒वस्या॒हꣳ स॑वि॒तुः प्र॑स॒वे बृह॒स्पति॑ना वाज॒जिता॒ वर्\mbox{}षि॑ष्ठं॒ नाकꣳ॑ रुहेय॒मिन्द्रा॑य॒ वाचं॑ वद॒तेन्द्रं॒ वाजं॑ जापय॒तेन्द्रो॒ वाज॑मजयित्। अश्वा॑जनि वाजिनि॒ वाजे॑षु वाजिनीव॒त्यश्वा᳚न्थ्स॒मथ्सु॑ वाजय॥ अर्वा॑ऽसि॒ सप्ति॑रसि वा॒ज्य॑सि॒ वाजि॑नो॒ वाजं॑ धावत म॒रुतां᳚ प्रस॒वे ज॑यत॒ वि योज॑ना मिमीध्व॒मध्व॑नः स्कभ्नीत॒~(३४)

%1.7.8.2
काष्ठां᳚ गच्छत॒ वाजे॑वाजे\-ऽवत वाजिनो नो॒ धने॑षु विप्रा अमृता ऋतज्ञाः॥ अ॒स्य मध्वः॑ पिबत मा॒दय॑ध्वं तृ॒प्ता या॑त प॒थिभि॑र्देव॒यानैः᳚॥ ते नो॒ अर्व॑न्तो हवन॒श्रुतो॒ हवं॒ विश्वे॑ शृण्वन्तु वा॒जिनः॑॥ मि॒तद्र॑वः सहस्र॒सा मे॒धसा॑ता सनि॒ष्यवः॑। म॒हो ये रत्नꣳ॑ समि॒थेषु॑ जभ्रि॒रे शं नो॑ भवन्तु वा॒जिनो॒ हवे॑षु॥ दे॒वता॑ता मि॒तद्र॑वः स्व॒र्काः। ज॒म्भय॒न्तो\-ऽहिं॒ वृक॒ꣳ॒ रक्षाꣳ॑सि॒ सने᳚म्य॒स्मद्यु॑यव॒न्न-~(३५)

%1.7.8.3
मी॑वाः॥ ए॒ष स्य वा॒जी क्षि॑प॒णिं तु॑रण्यति ग्री॒वायां᳚ ब॒द्धो अ॑पिक॒क्ष आ॒सनि॑। क्रतुं॑ दधि॒क्रा अनु॑ स॒न्तवी᳚त्वत् प॒थामङ्का॒ꣴ॒स्यन्वा॒पनी॑फणत्॥ उ॒त स्मा᳚स्य॒ द्रव॑तस्तुरण्य॒तः प॒र्णं न वेरनु॑ वाति प्रग॒र्धिनः॑। श्ये॒नस्ये॑व॒ ध्रज॑तो अङ्क॒सं परि॑ दधि॒क्राव्ण्णः॑ स॒होर्जा तरि॑त्रतः॥ आ मा॒ वाज॑स्य प्रस॒वो ज॑गम्या॒दा द्यावा॑पृथि॒वी वि॒श्वश॑म्भू। आ मा॑ गन्तां पि॒तरा॑~(३६)

%1.7.8.4
मा॒तरा॒ चा\-ऽ\-ऽमा॒ सोमो॑ अमृत॒त्वाय॑ गम्यात्॥ वाजि॑नो वाजजितो॒ वाजꣳ॑ सरि॒ष्यन्तो॒ वाजं॑ जे॒ष्यन्तो॒ बृह॒स्पते᳚र्भा॒गमव॑ जिघ्रत॒ वाजि॑नो वाजजितो॒ वाजꣳ॑ ससृ॒वाꣳसो॒ वाजं॑ जिगि॒वाꣳसो॒ बृह॒स्पते᳚र्भा॒गे नि मृ॑ढ्वमि॒यं वः॒ सा स॒त्या स॒न्धा\-ऽभू॒द्यामिन्द्रे॑ण स॒मध॑ध्व॒मजी॑जिपत वनस्पतय॒ इन्द्रं॒ वाजं॒ विमु॑च्यध्वम्॥~(३७)

%1.7.9.0
{\anuvakamend[{स्क॒भ्नी॒त॒ यु॒य॒व॒न्पि॒तरा॒ द्विच॑त्वारिꣳशच्च}]}%~(८)

%1.7.9.1
क्ष॒त्रस्योल्ब॑मसि क्ष॒त्रस्य॒ योनि॑रसि॒ जाय॒ एहि॒ सुवो॒ रोहा॑व॒ रोहा॑व॒ हि सुव॑र॒हं ना॑वु॒भयोः॒ सुवो॑ रोक्ष्यामि॒ वाज॑श्च प्रस॒वश्चा॑पि॒जश्च॒ क्रतु॑श्च॒ सुव॑श्च मू॒र्धा च॒ व्यश्ञि॑यश्चा\-ऽ\-ऽन्त्याय॒नश्चान्त्य॑श्च भौव॒नश्च॒ भुव॑न॒श्चाधि॑पतिश्च। आयु॑र्य॒ज्ञेन॑ कल्पतां प्रा॒णो य॒ज्ञेन॑ कल्पतामपा॒नो~-~(३८)

%1.7.9.2
य॒ज्ञेन॑ कल्पतां व्या॒नो य॒ज्ञेन॑ कल्पतां॒ चक्षु॑र्य॒ज्ञेन॑ कल्पता॒ꣴ॒ श्रोत्रं॑ य॒ज्ञेन॑ कल्पतां॒ मनो॑ य॒ज्ञेन॑ कल्पतां॒ वाग्य॒ज्ञेन॑ कल्पतामा॒त्मा य॒ज्ञेन॑ कल्पतां य॒ज्ञो य॒ज्ञेन॑ कल्पता॒ꣳ॒ सुव॑र्दे॒वाꣳ अ॑गन्मा॒मृता॑ अभूम प्र॒जा\-प॑तेः प्र॒जा अ॑भूम॒ सम॒हं प्र॒जया॒ सं मया᳚ प्र॒जा सम॒हꣳ रा॒यस्पोषे॑ण॒ सं मया॑ रा॒यस्पोषो\-ऽन्ना॑य त्वा॒\-ऽन्नाद्या॑य त्वा॒ वाजा॑य त्वा वाजजि॒त्यायै᳚ त्वा॒\-ऽमृत॑मसि॒ पुष्टि॑रसि प्र॒जन॑नमसि॥~(३९)

%1.7.10.0
{\anuvakamend[{अ॒पा॒नो वाजा॑य॒ नव॑ च}]}%~(९)

%1.7.10.1
वाज॑स्ये॒मं प्र॑स॒वः सु॑षुवे॒ अग्रे॒ सोम॒ꣳ॒ राजा॑न॒मोष॑धीष्व॒फ्सु। ता अ॒स्मभ्यं॒ मधु॑मतीर्भवन्तु व॒यꣳ रा॒ष्ट्रे जा᳚ग्रियाम पु॒रोहि॑ताः॥ वाज॑स्ये॒दं प्र॑स॒व आ ब॑भूवे॒मा च॒ विश्वा॒ भुव॑नानि स॒र्वतः॑। स वि॒राजं॒ पर्ये॑ति प्रजा॒नन् प्र॒जां पुष्टिं॑ व॒र्धय॑मानो अ॒स्मे॥ वाज॑स्ये॒मां प्र॑स॒वः शि॑श्रिये॒ दिव॑मि॒मा च॒ विश्वा॒ भुव॑नानि स॒म्राट्। अदि॑थ्सन्तं दापयतु प्रजा॒नन् र॒यिं~(४०)

%1.7.10.2
च॑ नः॒ सर्व॑वीरां॒ नि य॑च्छतु॥ अग्ने॒ अच्छा॑ वदे॒ह नः॒ प्रति॑ नः सु॒मना॑ भव। प्र णो॑ यच्छ भुवस्पते धन॒दा अ॑सि न॒स्त्वम्॥ प्र णो॑ यच्छत्वर्य॒मा प्र भगः॒ प्र बृह॒स्पतिः॑। प्र दे॒वाः प्रोत सू॒नृता॒ प्र वाग्दे॒वी द॑दातु नः॥ अ॒र्य॒मणं॒ बृह॒स्पति॒मिन्द्रं॒ दाना॑य चोदय। वाचं॒ विष्णु॒ꣳ॒ सर॑स्वतीꣳ सवि॒तारं॑~(४१)

%1.7.10.3
च वा॒जिनम्᳚॥ सोम॒ꣳ॒ राजा॑नं॒ वरु॑णम॒ग्निम॒न्वार॑भामहे। आ॒दि॒त्यान् विष्णु॒ꣳ॒ सूर्यं॑ ब्र॒ह्माणं॑ च॒ बृह॒स्पतिम्᳚॥ दे॒वस्य॑ त्वा सवि॒तुः प्र॑स॒वे᳚\-ऽश्विनो᳚र्बा॒हु\-भ्यां᳚ पू॒ष्णो हस्ता᳚भ्या॒ꣳ॒ सर॑स्वत्यै वा॒चो य॒न्तुर्य॒न्त्रेणा॒ग्नेस्त्वा॒ साम्रा᳚ज्येना॒भिषि॑ञ्चा॒मीन्द्र॑स्य॒ बृह॒स्पते᳚स्त्वा॒ साम्रा᳚ज्येना॒भिषि॑ञ्चामि॥~(४२)

%1.7.11.0
{\anuvakamend[{र॒यिꣳ स॑वि॒तार॒ꣳ॒ षट्त्रिꣳ॑शच्च}]}%॥10॥

%1.7.11.1
अ॒ग्निरेका᳚क्षरेण॒ वाच॒मुद॑जयद॒श्विनौ॒ द्व्य॑क्षरेण प्राणा\-पा॒ना\-वुद॑\-जयतां॒ विष्णु॒स्त्र्य॑क्षरेण॒ त्रीँल्लो॒कानुद॑जय॒थ्सोम॒श्चतु॑रक्षरेण॒ चतु॑ष्पदः प॒शूनुद॑जयत् पू॒षा पञ्चा᳚क्षरेण प॒ङ्क्तिमुद॑जयद्धा॒ता षड॑क्षरेण॒ षडृ॒तूनुद॑जयन्म॒रुतः॑ स॒प्ताक्ष॑रेण स॒प्तप॑दा॒ꣳ॒ शक्व॑री॒मुद॑जय॒न् बृह॒स्पति॑र॒ष्टाक्ष॑रेण गाय॒त्रीमुद॑जयन्मि॒त्रो नवा᳚क्षरेण त्रि॒वृत॒ꣴ॒ स्तोम॒मुद॑जय॒-~(४३)

%1.7.11.2
द्वरु॑णो॒ दशा᳚क्षरेण वि॒राज॒मुद॑जय॒दिन्द्र॒ एका॑\-दशा\-क्षरेण त्रि॒ष्टुभ॒मुद॑जय॒द् विश्वे॑ दे॒वा द्वाद॑शाक्षरेण॒ जग॑ती॒मुद॑जय॒न् वस॑व॒स्त्रयो॑\-दशा\-क्षरेण त्रयोद॒शꣴस्तोम॒मुद॑जयन् रु॒द्राश्चतु॑र्दशा\-क्षरेण चतुर्द॒शꣴ स्तोम॒मुद॑जयन्नादि॒त्याः पञ्च॑\-दशा\-क्षरेण पञ्चद॒शꣴ स्तोम॒मुद॑जय॒न्नदि॑तिः॒ षोड॑शाक्षरेण षोड॒शꣴ स्तोम॒मुद॑जयत् प्र॒जा\-प॑तिः स॒प्तद॑शाक्षरेण सप्तद॒शꣴ स्तोम॒मुद॑जयत्॥~(४४)

%1.7.12.0
{\anuvakamend[{अ॒ज॒य॒त् षट्च॑त्वारिꣳशच्च}]}%॥11॥

%1.7.12.1
उ॒प॒या॒मगृ॑हीतो\-ऽसि नृ॒षदं॑ त्वा द्रु॒षदं॑ भुवन॒सद॒मिन्द्रा॑य॒ जुष्टं॑ गृह्णाम्ये॒ष ते॒ योनि॒रिन्द्रा॑य त्वोपया॒मगृ॑हीतो\-ऽस्यफ्सु॒षदं॑ त्वा घृत॒सदं॑ व्योम॒सद॒मिन्द्रा॑य॒ जुष्टं॑ गृह्णाम्ये॒ष ते॒ योनि॒रिन्द्रा॑य त्वोपया॒मगृ॑हीतो\-ऽसि पृथिवि॒षदं॑ त्वा\-ऽन्तरिक्ष॒सदं॑ नाक॒सद॒मिन्द्रा॑य॒ जुष्टं॑ गृह्णाम्ये॒ष ते॒ योनि॒रिन्द्रा॑य त्वा॥ ये ग्रहाः᳚ पञ्चज॒नीना॒ येषां᳚ ति॒स्रः प॑रम॒जाः। दैव्यः॒ कोशः॒~(४५)

%1.7.12.2
समु॑ब्जितः। तेषां॒ विशि॑प्रियाणा॒मिष॒मूर्ज॒ꣳ॒ सम॑ग्रभीमे॒ष ते॒ योनि॒रिन्द्रा॑य त्वा॥ अ॒पाꣳ रस॒मुद्व॑यस॒ꣳ॒ सूर्य॑रश्मिꣳ स॒माभृ॑तम्। अ॒पाꣳ रस॑स्य॒ यो रस॒स्तं वो॑ गृह्णाम्युत्त॒ममे॒ष ते॒ योनि॒रिन्द्रा॑य त्वा॥ अ॒या वि॒ष्ठा ज॒नय॒न्कर्व॑राणि॒ स हि घृणि॑रु॒रुर्वरा॑य गा॒तुः। स प्रत्युदै᳚द्ध॒रुणो॒ मध्वो॒ अग्र॒ꣴ॒ स्वायां॒ यत्त॒नुवां᳚ त॒नूमैर॑यत। उ॒प॒या॒मगृ॑हीतो\-ऽसि प्र॒जा\-प॑तये त्वा॒ जुष्टं॑ गृह्णाम्ये॒ष ते॒ योनिः॑ प्र॒जा\-प॑तये त्वा॥~(४६)

%1.7.13.0
{\anuvakamend[{कोश॑स्त॒नुवां॒ त्रयो॑दश च}]}%॥12॥

%1.7.13.1
अन्वह॒ मासा॒ अन्विद्वना॒न्यन्वोष॑धी॒रनु॒ पर्व॑तासः। अन्विन्द्र॒ꣳ॒ रोद॑सी वावशा॒ने अन्वापो॑ अजिहत॒ जाय॑मानम्॥ अनु॑ ते दायि म॒ह इ॑न्द्रि॒याय॑ स॒त्रा ते॒ विश्व॒मनु॑ वृत्र॒हत्ये᳚। अनु॑ क्ष॒त्रमनु॒ सहो॑ यज॒त्रेन्द्र॑ दे॒वेभि॒रनु॑ ते नृ॒षह्ये᳚॥ इ॒न्द्रा॒णीमा॒सु नारि॑षु सु॒पत्नी॑म॒हम॑श्रवम्। न ह्य॑स्या अप॒रं च॒न ज॒रसा॒~(४७)

%1.7.13.2
मर॑ते॒ पतिः॑॥ नाहमि॑न्द्राणि रारण॒ सख्यु॑र्वृ॒षाक॑पेर्‌ऋ॒ते। यस्ये॒दमप्यꣳ॑ ह॒विः प्रि॒यं दे॒वेषु॒ गच्छ॑ति॥ यो जा॒त ए॒व प्र॑थ॒मो मन॑स्वान् दे॒वो दे॒वान् क्रतु॑ना प॒र्यभू॑षत्। यस्य॒ शुष्मा॒द्रोद॑सी॒ अभ्य॑सेतां नृ॒म्णस्य॑ म॒ह्ना स ज॑नास॒ इन्द्रः॑॥ आ ते॑ म॒ह इ॑न्द्रो॒त्यु॑ग्र॒ सम॑न्यवो॒ यथ्स॒मर॑न्त॒ सेनाः᳚। पता॑ति दि॒द्युन्नर्य॑स्य बाहु॒वोर्मा ते॒~(४८)

%1.7.13.3
मनो॑ विष्व॒द्रिय॒ग्विचा॑रीत्॥ मा नो॑ मर्धी॒रा भ॑रा द॒द्धि तन्नः॒ प्र दा॒शुषे॒ दात॑वे॒ भूरि॒ यत् ते᳚। नव्ये॑ दे॒ष्णे श॒स्ते अ॒स्मिन् त॑ उ॒क्थे प्र ब्र॑वाम व॒यमि॑न्द्र स्तु॒वन्तः॑॥ आ तू भ॑र॒ माकि॑रे॒तत् परि॑ष्ठाद्वि॒द्मा हि त्वा॒ वसु॑पतिं॒ वसू॑नाम्। इन्द्र॒ यत् ते॒ माहि॑नं॒ दत्र॒मस्त्य॒स्मभ्यं॒ तद्ध॑र्यश्व॒~(४९)

%1.7.13.4
प्र य॑न्धि॥ प्र॒दा॒तारꣳ॑ हवामह॒ इन्द्र॒मा ह॒विषा॑ व॒यम्। उ॒भा हि हस्ता॒ वसु॑ना पृ॒णस्वा\-ऽ\-ऽप्र य॑च्छ॒ दक्षि॑णा॒दोत स॒व्यात्॥ प्र॒दा॒ता व॒ज्री वृ॑ष॒भस्तु॑रा॒षाट्छु॒ष्मी राजा॑ वृत्र॒हा सो॑म॒पावा᳚। अ॒स्मिन् य॒ज्ञे ब॒र्॒\mbox{}हिष्या नि॒षद्याथा॑ भव॒ यज॑मानाय॒ शं योः॥ इन्द्रः॑ सु॒त्रामा॒ स्ववा॒ꣳ॒ अवो॑भिः सुमृडी॒को भ॑वतु वि॒श्ववे॑दाः। बाध॑तां॒ द्वेषो॒ अभ॑यं कृणोतु सु॒वीर्य॑स्य॒~(५०)

%1.7.13.5
पत॑यः स्याम॥ तस्य॑ व॒यꣳ सु॑म॒तौ य॒ज्ञिय॒स्यापि॑ भ॒द्रे सौ॑मन॒से स्या॑म। स सु॒त्रामा॒ स्ववा॒ꣳ॒ इन्द्रो॑ अ॒स्मे आ॒राच्चि॒द्द्वेषः॑ सनु॒तर्यु॑योतु॥ रे॒वती᳚र्नः सध॒माद॒ इन्द्रे॑ सन्तु तु॒विवा॑जाः। क्षु॒मन्तो॒ याभि॒र्मदे॑म॥ प्रो ष्व॑स्मै पुरोर॒थमिन्द्रा॑य शू॒षम॑र्चत। अ॒भीके॑ चिदु लोक॒कृथ्स॒ङ्गे स॒मथ्सु॑ वृत्र॒हा। अ॒स्माकं॑ बोधि चोदि॒ता नभ॑न्तामन्य॒केषा᳚म्। ज्या॒का अधि॒ धन्व॑सु॥~(५१)

{\anuvakamend[{ज॒रसा॒ मा ते॑ हर्यश्व सु॒वीर्य॒स्याध्येकं॑ च}]}%॥13॥
%%% END PRASHNA

\sect{अष्टमः प्रश्नः}\setcounter{anuvakam}{0}
\dnsub{तैत्तिरीयसंहितायां प्रथमकाण्डे अष्टमः प्रश्नः}
%1.8.1.0
%1.8.1.1
अनु॑मत्यै पुरो॒डाश॑\-म॒ष्टा\-क॑पालं॒ निर्व॑पति धे॒नुर्दक्षि॑णा॒ ये प्र॒त्यञ्चः॒ शम्या॑या अव॒शीय॑न्ते॒ तन्नैर्॑ऋ॒तमेक॑कपालं कृ॒ष्णं वासः॑ कृ॒ष्णतू॑षं॒ दक्षि॑णा॒ वीहि॒ स्वाहा\-ऽ\-ऽहु॑तिं जुषा॒ण ए॒ष ते॑ निर्\mbox{}ऋते भा॒गो भूते॑ ह॒विष्म॑त्यसि मु॒ञ्चेममꣳह॑सः॒ स्वाहा॒ नमो॒ य इ॒दं च॒कारा॑\-ऽ\-ऽदि॒त्यं च॒रुं निर्व॑पति॒ वरो॒ दक्षि॑णा\-ऽ\-ऽग्नावैष्ण॒वमेका॑\-दश\-कपालं वाम॒नो व॒ही दक्षि॑णा\-ऽग्नीषो॒मीय॒-~(१)

%1.8.1.2
मेका॑\-दश\-कपाल॒ꣳ॒ हिर॑ण्यं॒ दक्षि॑णै॒न्द्रमेका॑\-दश\-कपालमृष॒भो व॒ही दक्षि॑णा\-ऽ\-ऽग्ने॒यम॒ष्टाक॑पालमै॒न्द्रं दध्यृ॑ष॒भो व॒ही दक्षि॑णैन्द्रा॒ग्नं द्वाद॑श\-कपालं वैश्वदे॒वं च॒रुं प्र॑थम॒जो व॒थ्सो दक्षि॑णा सौ॒म्यꣴ श्या॑मा॒कं च॒रुं वासो॒ दक्षि॑णा॒ सर॑स्वत्यै च॒रुꣳ सर॑स्वते च॒रुं मि॑थु॒नौ गावौ॒ दक्षि॑णा॥~(२)

%1.8.2.0
{\anuvakamend[{अ॒ग्नी॒षो॒मीयं॒ चतु॑स्त्रिꣳशच्च}]}%~(१)

%1.8.2.1
आ॒ग्ने॒यम॒ष्टा\-क॑पालं॒ निर्व॑पति सौ॒म्यं च॒रुꣳ सा॑वि॒त्रं द्वाद॑श\-कपालꣳ सारस्व॒तं च॒रुं पौ॒ष्णं च॒रुं मा॑रु॒तꣳ स॒प्तक॑पालं वैश्वदे॒वीमा॒मिक्षां᳚ द्यावा\-पृथि॒व्य॑मेक॑कपालम्॥~(३)

%1.8.3.0
{\anuvakamend[{आ॒ग्ने॒यम॒ष्टाद॑श}]}%~(२)

%1.8.3.1
ऐ॒न्द्रा॒ग्नमेका॑\-दश\-कपालं मारु॒तीमा॒मिक्षां᳚ वारु॒णीमा॒मिक्षां᳚ का॒यमेक॑कपालं प्रघा॒स्यान्॑ हवामहे म॒रुतो॑ य॒ज्ञवा॑हसः कर॒म्भेण॑ स॒जोष॑सः॥ मो षू ण॑ इन्द्र पृ॒थ्सु दे॒वास्तु॑ स्म ते शुष्मिन्नव॒या। म॒ही ह्य॑स्य मी॒ढुषो॑ य॒व्या। ह॒विष्म॑तो म॒रुतो॒ वन्द॑ते॒ गीः॥ यद् ग्रामे॒ यदर॑ण्ये॒ यथ्स॒भायां॒ यदि॑न्द्रि॒ये। यच्छू॒द्रे यद॒र्य॑ एन॑श्चकृ॒मा व॒यम्। यदेक॒स्याधि॒ धर्म॑णि॒ तस्या॑व॒यज॑नमसि॒ स्वाहा᳚॥ अक्र॒न् कर्म॑ कर्म॒कृतः॑ स॒ह वा॒चा म॑योभु॒वा॥ दे॒वेभ्यः॒ कर्म॑ कृ॒त्वा\-ऽस्तं॒ प्रेत॑ सुदानवः॥~(४)

%1.8.4.0
{\anuvakamend[{व॒यं यद् विꣳ॑श॒तिश्च॑}]}%~(३)

%1.8.4.1
अ॒ग्नये\-ऽनी॑कवते पुरो॒डाश॑\-म॒ष्टा\-क॑पालं॒ निर्व॑पति सा॒कꣳ सूर्ये॑णोद्य॒ता म॒रुद्भ्यः॑ सान्तप॒नेभ्यो॑ म॒ध्यन्दि॑ने च॒रुं म॒रुद्भ्यो॑ गृहमे॒धिभ्यः॒ सर्वा॑सां दु॒ग्धे सा॒यं च॒रुं पू॒र्णा द॑र्वि॒ परा॑ पत॒ सुपू᳚र्णा॒ पुन॒राप॑त। व॒स्नेव॒ वि क्री॑णावहा॒ इष॒मूर्जꣳ॑ शतक्रतो॥ दे॒हि मे॒ ददा॑मि ते॒ नि मे॑ धेहि॒ नि ते॑ दधे। नि॒हार॒मिन्नि मे॑ हरा नि॒हारं॒~(५)


%1.8.4.2
नि ह॑रामि ते॥ म॒रुद्भ्यः॑ क्री॒डिभ्यः॑ पुरो॒डाशꣳ॑ स॒प्त\-क॑पालं॒ निर्व॑पति सा॒कꣳ सूर्ये॑णोद्य॒ताग्ने॒यम॒ष्टा\-क॑पालं॒ निर्व॑पति सौ॒म्यं च॒रुꣳ सा॑वि॒त्रं द्वाद॑श\-कपालꣳ सारस्व॒तं च॒रुं पौ॒ष्णं च॒रुमै᳚न्द्रा॒ग्नमेका॑\-दश\-कपालमै॒न्द्रं च॒रुं वै᳚श्वकर्म॒णमेक॑कपालम्॥~(६)

%1.8.5.0
{\anuvakamend[{ह॒रा॒ नि॒हारं॑ त्रि॒ꣳ॒शच्च॑}]}%~(४)

%1.8.5.1
सोमा॑य पितृ॒मते॑ पुरो॒डाश॒ꣳ॒ षट्\-क॑पालं॒ निर्व॑पति पि॒तृभ्यो॑ बर्\mbox{}हि॒षद्भ्यो॑ धा॒नाः पि॒तृभ्यो᳚\-ऽग्निष्वा॒त्तेभ्यो॑\-ऽभिवा॒न्या॑यै दु॒ग्धे म॒न्थमे॒तत् ते॑ तत॒ ये च॒ त्वामन्वे॒तत् ते॑ पितामह प्रपितामह॒ ये च॒ त्वामन्वत्र॑ पितरो यथाभा॒गं म॑न्दध्वꣳ सुस॒न्दृशं॑ त्वा व॒यं मघ॑वन् मन्दिषी॒महि॑॥ प्र नू॒नं पू॒र्णव॑न्धुरः स्तु॒तो या॑सि॒ वशा॒ꣳ॒ अनु॑॥ योजा॒ न्वि॑न्द्र ते॒ हरी᳚॥~(७)

%1.8.5.2
अक्ष॒न्नमी॑मदन्त॒ ह्यव॑ प्रि॒या अ॑धूषत॥ अस्तो॑षत॒ स्वभा॑नवो॒ विप्रा॒ नवि॑ष्ठया म॒ती॥ योजा॒ न्वि॑न्द्र ते॒ हरी᳚॥ अक्ष॑न् पि॒तरो\-ऽमी॑मदन्त पि॒तरो\-ऽती॑तृपन्त पि॒तरो\-ऽमी॑मृजन्त पि॒तरः॑॥ परे॑त पितरः सोम्या गम्भी॒रैः प॒थिभिः॑ पू॒र्व्यैः॥ अथा॑ पि॒तॄन्थ्सु॑वि॒दत्रा॒ꣳ॒ अपी॑त य॒मेन॒ ये स॑ध॒मादं॒ मद॑न्ति॥ मनो॒ न्वा हु॑वामहे नाराश॒ꣳ॒सेन॒ स्तोमे॑न पितृ॒णां च॒ मन्म॑भिः॥ आ~(८)

%1.8.5.3
न॑ एतु॒ मनः॒ पुनः॒ क्रत्वे॒ दक्षा॑य जी॒वसे᳚॥ ज्योक् च॒ सूर्यं॑ दृ॒शे॥ पुन॑र्नः पि॒तरो॒ मनो॒ ददा॑तु॒ दैव्यो॒ जनः॑॥ जी॒वं व्रातꣳ॑ सचेमहि॥ यद॒न्तरि॑क्षं पृथि॒वीमु॒त द्यां यन्मा॒तरं॑ पि॒तरं॑ वा जिहिꣳसि॒म॥ अ॒ग्निर्मा॒ तस्मा॒देन॑सो॒ गार्\mbox{}ह॑पत्यः॒ प्र मु॑ञ्चतु दुरि॒ता यानि॑ चकृ॒म क॒रोतु॒ माम॑ने॒नसम्᳚॥~(९)

%1.8.6.0
{\anuvakamend[{हरी॒ मन्म॑भि॒रा चतु॑श्चत्वारिꣳशच्च}]}%~(५)

%1.8.6.1
प्र॒ति॒पू॒रु॒षमेक॑कपाला॒न्निर्व॑प॒त्येक॒\-मति॑रिक्तं॒ याव॑न्तो गृ॒ह्याः᳚ स्मस्तेभ्यः॒ कम॑करं पशू॒नाꣳ शर्मा॑सि॒ शर्म॒ यज॑मानस्य॒ शर्म॑ मे य॒च्छैक॑ ए॒व रु॒द्रो न द्वि॒तीया॑य तस्थ आ॒खुस्ते॑ रुद्र प॒शुस्तं जु॑षस्वै॒ष ते॑ रुद्र भा॒गः स॒ह स्वस्रा\-ऽम्बि॑कया॒ तं जु॑षस्व भेष॒जं गवे\-ऽश्वा॑य॒ पुरु॑षाय भेष॒जमथो॑ अ॒स्मभ्यं॑ भेष॒जꣳ सुभे॑षजं॒~(१०)

%1.8.6.2
यथा\-ऽस॑ति॥ सु॒गं मे॒षाय॑ मे॒ष्या॑ अवा᳚म्ब रु॒द्रम॑दिम॒ह्यव॑ दे॒वं त्र्य॑म्बकम्॥ यथा॑ नः॒ श्रेय॑सः॒ कर॒द्यथा॑ नो॒ वस्य॑सः॒ कर॒द्यथा॑ नः पशु॒मतः॒ कर॒द्यथा॑ नो व्यवसा॒यया᳚त्॥ त्र्य॑म्बकं यजामहे सुग॒न्धिं पु॑ष्टि॒वर्ध॑नम्॥ उ॒र्वा॒रु॒कमि॑व॒ बन्ध॑नान्मृ॒त्योर्मु॑क्षीय॒ मा\-ऽमृता᳚त्॥ ए॒ष ते॑ रुद्र भा॒गस्तं जु॑षस्व॒ तेना॑व॒सेन॑ प॒रो मूज॑व॒तो\-ऽती॒ह्यव॑ततधन्वा॒ पिना॑कहस्तः॒ कृत्ति॑वासाः॥~(११)

%1.8.7.0
{\anuvakamend[{सुभे॑षजमिहि॒ त्रीणि॑ च}]}%~(६)

%1.8.7.1
ऐ॒न्द्रा॒ग्नं द्वाद॑श\-कपालं वैश्वदे॒वं च॒रुमिन्द्रा॑य॒ शुना॒सीरा॑य पुरो॒डाशं॒ द्वाद॑श\-कपालं वाय॒व्यं॑ पयः॑ सौ॒र्यमेक॑कपालं द्वादशग॒वꣳ सीरं॒ दक्षि॑णा\-ऽ\-ऽग्ने॒यम॒ष्टा\-क॑पालं॒ निर्व॑पति रौ॒द्रं गा॑वीधु॒कं च॒रुमै॒न्द्रं दधि॑ वारु॒णं य॑व॒मयं॑ च॒रुं व॒हिनी॑ धे॒नुर्दक्षि॑णा॒ ये दे॒वाः पु॑रः॒सदो॒\-ऽग्निने᳚त्रा दक्षिण॒सदो॑ य॒मने᳚त्राः पश्चा॒थ्सदः॑ सवि॒तृने᳚त्रा उत्तर॒सदो॒ वरु॑णनेत्रा उपरि॒षदो॒ बृह॒स्पति॑नेत्रा रक्षो॒हण॒स्ते नः॑ पान्तु॒ ते नो॑\-ऽवन्तु॒ तेभ्यो॒~(१२)

%1.8.7.2
नम॒स्तेभ्यः॒ स्वाहा॒ समू॑ढ॒ꣳ॒ रक्षः॒ सन्द॑ग्ध॒ꣳ॒ रक्ष॑ इ॒दम॒हꣳ रक्षो॒\-ऽभि सं द॑हाम्य॒ग्नये॑ रक्षो॒घ्ने स्वाहा॑ य॒माय॑ सवि॒त्रे वरु॑णाय॒ बृह॒स्पत॑ये॒ दुव॑स्वते रक्षो॒घ्ने स्वाहा᳚ प्रष्टिवा॒ही रथो॒ दक्षि॑णा दे॒वस्य॑ त्वा सवि॒तुः प्र॑स॒वे᳚\-ऽश्विनो᳚र्बा॒हु\-भ्यां᳚ पू॒ष्णो हस्ता᳚भ्या॒ꣳ॒ रक्ष॑सो व॒धं जु॑होमि ह॒तꣳ रक्षो\-ऽव॑धिष्म॒ रक्षो॒ यद्वस्ते॒ तद्दक्षि॑णा॥~(१३)


%1.8.8.0
{\anuvakamend[{तेभ्यः॒ पञ्च॑चत्वारिꣳशच्च}]}%~(७)

%1.8.8.1
धा॒त्रे पु॑रो॒डाशं॒ द्वाद॑श\-कपालं॒ निर्व॑प॒त्यनु॑मत्यै च॒रुꣳ रा॒कायै॑ च॒रुꣳ सि॑नीवा॒ल्यै च॒रुं कु॒ह्वै॑ च॒रुं मि॑थु॒नौ गावौ॒ दक्षि॑णा\-ऽ\-ऽग्नावैष्णव॒मेका॑\-दश\-कपालं॒ निर्व॑पत्यैन्द्रावैष्ण॒वमेका॑\-दश\-कपालं वैष्ण॒वं त्रि॑कपा॒लं वा॑म॒नो व॒ही दक्षि॑णा\-ऽग्नीषो॒मीय॒मेका॑\-दश\-कपालं॒ निर्व॑पतीन्द्रा\-सो॒मीय॒\-मेका॑\-दश\-कपालꣳ सौ॒म्यं च॒रुं ब॒भ्रुर्दक्षि॑णा सोमापौ॒ष्णं च॒रुं निर्व॑पत्यैन्द्रापौ॒ष्णं च॒रुं पौ॒ष्णं च॒रुꣴ श्या॒मो दक्षि॑णा वैश्वान॒रं द्वाद॑श\-कपालं॒ निर्व॑पति॒ हिर॑ण्यं॒ दक्षि॑णा वारु॒णं य॑व॒मयं॑ च॒रुमश्वो॒ दक्षि॑णा॥~(१४)

%1.8.9.0
{\anuvakamend[{निर॒ष्टौ च}]}%~(८)

%1.8.9.1
बा॒र्॒\mbox{}ह॒स्प॒त्यं च॒रुं निर्व॑पति ब्र॒ह्मणो॑ गृ॒हे शि॑तिपृ॒ष्ठो दक्षि॑णै॒न्द्रमेका॑\-दश\-कपालꣳ राज॒न्य॑स्य गृ॒ह ऋ॑ष॒भो दक्षि॑णा\-ऽ\-ऽदि॒त्यं च॒रुं महि॑ष्यै गृ॒हे धे॒नुर्दक्षि॑णा नैर्\mbox{}ऋ॒तं च॒रुं प॑रिवृ॒क्त्यै॑ गृ॒हे कृ॒ष्णानां᳚ व्रीही॒णां न॒खनि॑र्भिन्नं कृ॒ष्णा कू॒टा दक्षि॑णा\-ऽ\-ऽग्ने॒यम॒ष्टाक॑पालꣳ सेना॒न्यो॑ गृ॒हे हिर॑ण्यं॒ दक्षि॑णा वारु॒णं दश॑\-कपालꣳ सू॒तस्य॑ गृ॒हे म॒हानि॑रष्टो॒ दक्षि॑णा मारु॒तꣳ स॒प्तक॑पालं ग्राम॒ण्यो॑ गृ॒हे पृश्ञि॒र्दक्षि॑णा सावि॒त्रं द्वाद॑श\-कपालं~(१५)

%1.8.9.2
क्ष॒त्तुर्गृ॒ह उ॑पध्व॒स्तो दक्षि॑णा\-ऽ\-ऽश्वि॒नं द्वि॑कपा॒लꣳ स॑ङ्ग्रही॒तुर्गृ॒हे स॑वा॒त्यौ॑ दक्षि॑णा पौ॒ष्णं च॒रुं भा॑गदु॒घस्य॑ गृ॒हे श्या॒मो दक्षि॑णा रौ॒द्रं गा॑वीधु॒कं च॒रुम॑क्षावा॒पस्य॑ गृ॒हे श॒बल॒ उद्वा॑रो॒ दक्षि॒णेन्द्रा॑य सु॒त्राम्णे॑ पुरो॒डाश॒मेका॑\-दश\-कपालं॒ प्रति॒ निर्व॑प॒तीन्द्रा॑याꣳहो॒मुचे॒\-ऽयं नो॒ राजा॑ वृत्र॒हा राजा॑ भू॒त्वा वृ॒त्रं व॑ध्यान्मैत्राबार्\mbox{}हस्प॒त्यं भ॑वति श्वे॒तायै᳚ श्वे॒तव॑थ्सायै दु॒ग्धे स्व॑यं मू॒र्ते स्व॑यं मथि॒त आज्य॒ आश्व॑त्थे॒~(१६)

%1.8.9.3
पात्रे॒ चतुः॑स्रक्तौ स्वयमवप॒न्नायै॒ शाखा॑यै क॒र्णाꣴश्चा\-क॑र्णाꣴश्च तण्डु॒लान् वि चि॑नुया॒द्ये क॒र्णाः स पय॑सि बार्\mbox{}हस्प॒त्यो ये\-ऽक॑र्णाः॒ स आज्ये॑ मै॒त्रः स्व॑यं कृ॒ता वेदि॑र्भवति स्वयं दि॒नं ब॒र्॒\mbox{}हिः स्व॑यं कृ॒त इ॒ध्मः सैव श्वे॒ता श्वे॒तव॑थ्सा॒ दक्षि॑णा॥~(१७)


%1.8.10.0
{\anuvakamend[{सा॒वि॒त्रं द्वाद॑श\-कपाल॒माश्व॑त्थे॒ त्रय॑स्त्रिꣳशच्च}]}%~(९)

%1.8.10.1
अ॒ग्नये॑ गृ॒हप॑तये पुरो॒डाश॑\-म॒ष्टा\-क॑पालं॒ निर्व॑पति कृ॒ष्णानां᳚ व्रीही॒णाꣳ सोमा॑य॒ वन॒स्पत॑ये श्यामा॒कं च॒रुꣳ स॑वि॒त्रे स॒त्यप्र॑सवाय पुरो॒डाशं॒ द्वाद॑श\-कपालमाशू॒नां व्री॑ही॒णाꣳ रु॒द्राय॑ पशु॒पत॑ये गावीधु॒कं च॒रुं बृह॒स्पत॑ये वा॒चस्पत॑ये नैवा॒रं च॒रुमिन्द्रा॑य ज्ये॒ष्ठाय॑ पुरो॒डाश॒मेका॑\-दश\-कपालं म॒हाव्री॑हीणां मि॒त्राय॑ स॒त्याया॒\-ऽम्बानां᳚ च॒रुं वरु॑णाय॒ धर्म॑पतये यव॒मयं॑ च॒रुꣳ स॑वि॒ता त्वा᳚ प्रस॒वानाꣳ॑ सुवताम॒ग्निर्गृ॒हप॑तीना॒ꣳ॒ सोमो॒ वन॒स्पती॑नाꣳ रु॒द्रः प॑शू॒नां~(१८)

%1.8.10.2
बृह॒स्पति॑र्वा॒चामिन्द्रो᳚ ज्ये॒ष्ठानां᳚ मि॒त्रः स॒त्यानां॒ वरु॑णो॒ धर्म॑पतीनां॒ ये दे॑वा देव॒सुवः॒ स्थ त इ॒ममा॑मुष्याय॒णम॑\-नमि॒त्राय॑ सुवध्वं मह॒ते क्ष॒त्राय॑ मह॒त आधि॑पत्याय मह॒ते जान॑राज्यायै॒ष वो॑ भरता॒ राजा॒ सोमो॒\-ऽस्माकं॑ ब्राह्म॒णाना॒ꣳ॒ राजा॒ प्रति॒ त्यन्नाम॑ रा॒ज्यम॑धायि॒ स्वां त॒नुवं॒ वरु॑णो अशिश्रे॒च्छुचे᳚र्मि॒त्रस्य॒ व्रत्या॑ अभू॒माम॑न्महि मह॒त ऋ॒तस्य॒ नाम॒ सर्वे॒ व्राता॒ वरु॑णस्याभूव॒न्वि मि॒त्र एवै॒ररा॑तिमतारी॒दसू॑षुदन्त य॒ज्ञिया॑ ऋ॒तेन॒ व्यु॑ त्रि॒तो ज॑रि॒माणं॑ न आन॒ड् विष्णोः॒ क्रमो॑\-ऽसि॒ विष्णोः᳚ क्रा॒न्तम॑सि॒ विष्णो॒र्विक्रा᳚न्तमसि॥~(१९)

%1.8.11.0
{\anuvakamend[{प॒शू॒नां व्राताः॒ पञ्च॑विꣳशतिश्च}]}%॥10॥

%1.8.11.1
अ॒र्थेतः॑ स्था॒\-ऽपां पति॑रसि॒ वृषा᳚स्यू॒र्मिर्वृ॑षसे॒नो॑\-ऽसि व्रज॒क्षितः॑ स्थ म॒रुता॒मोजः॑ स्थ॒ सूर्य॑वर्चसः स्थ॒ सूर्य॑त्वचसः स्थ॒ मान्दाः᳚ स्थ॒ वाशाः᳚ स्थ॒ शक्व॑रीः स्थ विश्व॒भृतः॑ स्थ जन॒भृतः॑ स्था॒\-ऽग्नेस्ते॑ज॒स्याः᳚ स्था॒\-ऽपामोष॑धीना॒ꣳ॒ रसः॑ स्था॒\-ऽपो दे॒वीर्मधु॑मतीरगृह्ण॒न्नूर्ज॑स्वती राज॒सूया॑य॒ चिता॑नाः। याभि॑र्मि॒त्रावरु॑णाव॒भ्यषि॑ञ्च॒न्॒ याभि॒रिन्द्र॒मन॑य॒न्नत्यरा॑तीः॥ रा॒ष्ट्र॒दाः स्थ॑ रा॒ष्ट्रं द॑त्त॒ स्वाहा॑ राष्ट्र॒दाः स्थ॑ रा॒ष्ट्रम॒मुष्मै॑ दत्त॥~(२०)

%1.8.12.0
{\anuvakamend[{अत्येका॑\-दश च}]}%॥11॥

%1.8.12.1
देवी॑रापः॒ सं मधु॑मती॒र्मधु॑मतीभिः सृज्यध्वं॒ महि॒ वर्चः॑ क्ष॒त्रिया॑य वन्वा॒ना अना॑धृष्टाः सीद॒तोर्ज॑स्वती॒र्महि॒ वर्चः॑ क्ष॒त्रिया॑य॒ दध॑ती॒रनि॑भृष्टमसि वा॒चो बन्धु॑स्तपो॒जाः सोम॑स्य दा॒त्रम॑सि शु॒क्रा वः॑ शु॒क्रेणोत्पु॑नामि च॒न्द्राश्च॒न्द्रेणा॒मृता॑ अ॒मृते॑न॒ स्वाहा॑ राज॒सूया॑य॒ चिता॑नाः॥ स॒ध॒मादो᳚ द्यु॒म्निनी॒रूर्ज॑ ए॒ता अनि॑भृष्टा अप॒स्युवो॒ वसा॑नः। प॒स्त्या॑सु चक्रे॒ वरु॑णः स॒धस्थ॑म॒पाꣳ शिशु॑र्-~(२१)

%1.8.12.2
मा॒तृत॑मास्व॒न्तः॥ क्ष॒त्रस्योल्ब॑मसि क्ष॒त्रस्य॒ योनि॑र॒स्यावि॑न्नो अ॒ग्निर्गृ॒हप॑ति॒रावि॑न्न॒ इन्द्रो॑ वृ॒द्धश्र॑वा॒ आवि॑न्नः पू॒षा वि॒श्ववे॑दा॒ आवि॑न्नौ मि॒त्रावरु॑णावृता॒वृधा॒वावि॑न्ने॒ द्यावा॑पृथि॒वी धृ॒तव्र॑ते॒ आवि॑न्ना दे॒व्यदि॑तिर्विश्वरू॒प्यावि॑न्नो॒\-ऽयम॒सावा॑मुष्याय॒णो᳚\-ऽस्यां वि॒श्य॑स्मिन् रा॒ष्ट्रे म॑ह॒ते क्ष॒त्राय॑ मह॒त आधि॑पत्याय मह॒ते जान॑राज्यायै॒ष वो॑ भरता॒ राजा॒ सोमो॒\-ऽस्माकं॑ ब्राह्म॒णाना॒ꣳ॒ राजेन्द्र॑स्य॒~(२२)

%1.8.12.3
वज्रो॑\-ऽसि॒ वार्त्र॑घ्न॒स्त्वया॒ऽयं वृ॒त्रं व॑ध्याच्छत्रु॒बाध॑नाः स्थ पा॒त मा᳚ प्र॒त्यञ्चं॑ पा॒त मा॑ ति॒र्यञ्च॑म॒न्वञ्चं॑ मा पात दि॒ग्भ्यो मा॑ पात॒ विश्वा᳚भ्यो मा ना॒ष्ट्राभ्यः॑ पात॒ हिर॑ण्यवर्णावु॒षसां᳚ विरो॒के\-ऽयः॑ स्थूणा॒वुदि॑तौ॒ सूर्य॒स्या\-ऽ\-ऽरो॑हतं वरुण मित्र॒ गर्तं॒ तत॑श्चक्षाथा॒मदि॑तिं॒ दितिं॑ च॥~(२३)

%1.8.13.0
{\anuvakamend[{शिशु॒रिन्द्र॒स्यैक॑चत्वारिꣳशच्च}]}%॥12॥

%1.8.13.1
स॒मिध॒मा ति॑ष्ठ गाय॒त्री त्वा॒ छन्द॑सामवतु त्रि॒वृथ्स्तोमो॑ रथन्त॒रꣳ सामा॒ग्निर्दे॒वता॒ ब्रह्म॒ द्रवि॑णमु॒ग्रामा ति॑ष्ठ त्रि॒ष्टुप् त्वा॒ छन्द॑सामवतु पञ्चद॒शः स्तोमो॑ बृ॒हथ्सामेन्द्रो॑ दे॒वता᳚ क्ष॒त्रं द्रवि॑णं वि॒राज॒मा ति॑ष्ठ॒ जग॑ती त्वा॒ छन्द॑सामवतु सप्तद॒शः स्तोमो॑ वैरू॒पꣳ साम॑ म॒रुतो॑ दे॒वता॒ विड्द्रवि॑ण॒मुदी॑ची॒मा ति॑ष्ठानु॒ष्टुप् त्वा॒~(२४)

%1.8.13.2
छन्द॑सामवत्वेकवि॒ꣳ॒शः स्तोमो॑ वैरा॒जꣳ साम॑ मि॒त्रावरु॑णौ दे॒वता॒ बलं॒ द्रवि॑णमू॒र्ध्वामा ति॑ष्ठ प॒ङ्क्तिस्त्वा॒ छन्द॑सामवतु त्रिणवत्रयस्त्रि॒ꣳ॒शौ स्तोमौ॑ शाक्वररैव॒ते साम॑नी॒ बृह॒स्पति॑र्दे॒वता॒ वर्चो॒ द्रवि॑णमी॒दृङ् चा᳚न्या॒दृङ् चै॑ता॒दृङ् च॑ प्रति॒दृङ् च॑ मि॒तश्च॒ सम्मि॑तश्च॒ सभ॑राः। शु॒क्रज्यो॑तिश्च चि॒त्रज्यो॑तिश्च स॒त्यज्यो॑तिश्च॒ ज्योति॑ष्माꣴश्च स॒त्यश्च॑र्त॒पाश्चा-~(२५)

%1.8.13.3
त्यꣳ॑हाः। अ॒ग्नये॒ स्वाहा॒ सोमा॑य॒ स्वाहा॑ सवि॒त्रे स्वाहा॒ सर॑स्वत्यै॒ स्वाहा॑ पू॒ष्णे स्वाहा॒ बृह॒स्पत॑ये॒ स्वाहेन्द्रा॑य॒ स्वाहा॒ घोषा॑य॒ स्वाहा॒ श्लोका॑य॒ स्वाहा\-ऽꣳशा॑य॒ स्वाहा॒ भगा॑य॒ स्वाहा॒ क्षेत्र॑स्य॒ पत॑ये॒ स्वाहा॑ पृथि॒व्यै स्वाहा॒\-ऽन्तरि॑क्षाय॒ स्वाहा॑ दि॒वे स्वाहा॒ सूर्या॑य॒ स्वाहा॑ च॒न्द्रम॑से॒ स्वाहा॒ नक्ष॑त्रेभ्यः॒ स्वाहा॒\-ऽद्भ्यः स्वाहौष॑धीभ्यः॒ स्वाहा॒ वन॒स्पति॑भ्यः॒ स्वाहा॑ चराच॒रेभ्यः॒ स्वाहा॑ परिप्ल॒वेभ्यः॒ स्वाहा॑ सरीसृ॒पेभ्यः॒ स्वाहा᳚॥~(२६)

%1.8.14.0
{\anuvakamend[{अ॒नु॒ष्टुप्त्व॑र्त॒पाश्च॑ सरीसृ॒पेभ्यः॒ स्वाहा᳚}]}%॥13॥

%1.8.14.1
सोम॑स्य॒ त्विषि॑रसि॒ तवे॑व मे॒ त्विषि॑र्भूयाद॒मृत॑मसि मृ॒त्योर्मा॑ पाहि दि॒द्योन्मा॑ पा॒ह्यवे᳚ष्टा दन्द॒शूका॒ निर॑स्तं॒ नमु॑चेः॒ शिरः॑॥ सोमो॒ राजा॒ वरु॑णो दे॒वा ध॑र्म॒सुव॑श्च॒ ये। ते ते॒ वाचꣳ॑ सुवन्तां॒ ते ते᳚ प्रा॒णꣳ सु॑वन्तां॒ ते ते॒ चक्षुः॑ सुवन्तां॒ ते ते॒ श्रोत्रꣳ॑ सुवन्ता॒ꣳ॒ सोम॑स्य त्वा द्यु॒म्नेना॒भिषि॑ञ्चाम्य॒ग्ने-~(२७)

%1.8.14.2
स्तेज॑सा॒ सूर्य॑स्य॒ वर्च॒सेन्द्र॑स्येन्द्रि॒येण॑ मि॒त्रावरु॑णयोर्वी॒र्ये॑ण म॒रुता॒\-मोज॑सा क्ष॒त्राणां᳚ क्ष॒त्रप॑तिर॒स्यति॑ दि॒वस्पा॑हि स॒माव॑वृत्रन्नध॒रा\-गुदी॑ची॒\-रहिं॑ बु॒ध्निय॒मनु॑ स॒ञ्चर॑न्ती॒स्ताः पर्व॑तस्य वृष॒भस्य॑ पृ॒ष्ठे नाव॑श्चरन्ति स्व॒सिच॑ इया॒नाः॥ रुद्र॒ यत्ते॒ क्रयी॒ परं॒ नाम॒ तस्मै॑ हु॒तम॑सि य॒मेष्ट॑मसि। प्रजा॑पते॒ न त्वदे॒तान्य॒न्यो विश्वा॑ जा॒तानि॒ परि॒ ता ब॑भूव। यत्का॑मास्ते जुहु॒मस्तन्नो॑ अस्तु व॒यꣴ स्या॑म॒ पत॑यो रयी॒णाम्॥~(२८)

%1.8.15.0
{\anuvakamend[{अ॒ग्नेस्तैका॑\-दश च}]}%॥14॥

%1.8.15.1
इन्द्र॑स्य॒ वज्रो॑\-ऽसि॒ वार्त्र॑घ्न॒स्त्वया॒\-ऽयं वृ॒त्रं व॑ध्यान्मि॒त्रावरु॑ण\-योस्त्वा प्रशा॒स्त्रोः प्र॒शिषा॑ युनज्मि य॒ज्ञस्य॒ योगे॑न॒ विष्णोः॒ क्रमो॑\-ऽसि॒ विष्णोः᳚ क्रा॒न्तम॑सि॒ विष्णो॒र्विक्रा᳚न्तमसि म॒रुतां᳚ प्रस॒वे जे॑षमा॒प्तं मनः॒ सम॒हमि॑न्द्रि॒येण॑ वी॒र्ये॑ण पशू॒नां म॒न्युर॑सि॒ तवे॑व मे म॒न्युर्भू॑या॒न्नमो॑ मा॒त्रे पृ॑थि॒व्यै मा\-ऽहं मा॒तरं॑ पृथि॒वीꣳ हिꣳ॑सिषं॒ मा~(२९)

%1.8.15.2
मां मा॒ता पृ॑थि॒वी हिꣳ॑सी॒दिय॑द॒स्यायु॑र॒स्यायु॑र्मे धे॒ह्यूर्ग॒स्यूर्जं॑ मे धेहि॒ युङ्ङ॑सि॒ वर्चो॑\-ऽसि॒ वर्चो॒ मयि॑ धेह्य॒ग्नये॑ गृ॒हप॑तये॒ स्वाहा॒ सोमा॑य॒ वन॒स्पत॑ये॒ स्वाहेन्द्र॑स्य॒ बला॑य॒ स्वाहा॑ म॒रुता॒मोज॑से॒ स्वाहा॑ ह॒ꣳ॒सः शु॑चि॒षद्वसु॑रन्तरिक्ष॒\-सद्धोता॑ वेदि॒षदति॑थिर्दुरोण॒सत्। नृ॒षद्व॑र॒सदृ॑त॒सद्व्यो॑म॒सद॒ब्जा गो॒जा ऋ॑त॒जा अ॑द्रि॒जा ऋ॒तं बृ॒हत्॥~(३०)

%1.8.16.0
{\anuvakamend[{हि॒ꣳ॒सि॒षं॒ मर्त॒जास्त्रीणि॑ च}]}%॥15॥

%1.8.16.1
मि॒त्रो॑\-ऽसि॒ वरु॑णो\-ऽसि॒ सम॒हं विश्वै᳚र्दे॒वैः क्ष॒त्रस्य॒ नाभि॑रसि क्ष॒त्रस्य॒ योनि॑रसि स्यो॒नामा सी॑द सु॒षदा॒मा सी॑द॒ मा त्वा॑ हिꣳसी॒न्मा मा॑ हिꣳसी॒न्निष॑साद धृ॒तव्र॑तो॒ वरु॑णः प॒स्त्या᳚स्वा साम्रा᳚ज्याय सु॒क्रतु॒र्ब्रह्मा(३)न् त्वꣳ रा॑जन् ब्र॒ह्मा\-ऽसि॑ सवि॒ता\-ऽसि॑ स॒त्यस॑वो॒ ब्रह्मा(३)न् त्वꣳ रा॑जन् ब्र॒ह्मा\-ऽसीन्द्रो॑\-ऽसि स॒त्यौजा॒~(३१)

%1.8.16.2
ब्रह्मा(३)न् त्वꣳ रा॑जन् ब्र॒ह्मा\-ऽसि॑ मि॒त्रो॑\-ऽसि सु॒शेवो॒ ब्रह्मा(३)न् त्वꣳ रा॑जन् ब्र॒ह्मा\-ऽसि॒ वरु॑णो\-ऽसि स॒त्यध॒र्मेन्द्र॑स्य॒ वज्रो॑\-ऽसि॒ वार्त्र॑घ्न॒स्तेन॑ मे रध्य॒ दिशो॒\-ऽभ्य॑यꣳ राजा॑\-ऽभू॒थ्सुश्लो॒काँ(४) सुम॑ङ्ग॒लाँ(४) सत्य॑रा॒जा(३)न्। अ॒पां नप्त्रे॒ स्वाहो॒र्जो नप्त्रे॒ स्वाहा॒\-ऽग्नये॑ गृ॒हप॑तये॒ स्वाहा᳚॥~(३२)

%1.8.17.0
{\anuvakamend[{स॒त्यौजा᳚श्चत्वारि॒ꣳ॒शच्च॑}]}%॥16॥

%1.8.17.1
आ॒ग्ने॒यम॒ष्टा\-क॑पालं॒ निर्व॑पति॒ हिर॑ण्यं॒ दक्षि॑णा सारस्व॒तं च॒रुं व॑थ्सत॒री दक्षि॑णा सावि॒त्रं द्वाद॑श\-कपालमुपध्व॒स्तो दक्षि॑णा पौ॒ष्णं च॒रुꣴ श्या॒मो दक्षि॑णा बार्\mbox{}हस्प॒त्यं च॒रुꣳ शि॑तिपृ॒ष्ठो दक्षि॑णै॒न्द्रमेका॑\-दश\-कपालमृष॒भो दक्षि॑णा वारु॒णं दश॑\-कपालं म॒हानि॑रष्टो॒ दक्षि॑णा सौ॒म्यं च॒रुं ब॒भ्रुर्दक्षि॑णा त्वा॒ष्ट्रम॒ष्टाक॑पालꣳ शु॒ण्ठो दक्षि॑णा वैष्ण॒वं त्रि॑कपा॒लं वा॑म॒नो दक्षि॑णा॥~(३३)

%1.8.18.0
{\anuvakamend[{आ॒ग्ने॒यं द्विच॑त्वारिꣳशत्}]}%॥17॥

%1.8.18.1
स॒द्यो दी᳚क्षयन्ति स॒द्यः सोमं॑ क्रीणन्ति पुण्डरिस्र॒जां प्र य॑च्छति द॒शभि॑र्वथ्सत॒रैः सोमं॑ क्रीणाति दश॒पेयो॑ भवति श॒तं ब्रा᳚ह्म॒णाः पि॑बन्ति सप्तद॒शꣴ स्तो॒त्रं भ॑वति प्राका॒शाव॑ध्व॒र्यवे॑ ददाति॒ स्रज॑मुद्गा॒त्रे रु॒क्मꣳ होत्रे\-ऽश्वं॑ प्रस्तोतृप्रतिह॒र्तृभ्यां॒ द्वाद॑श पष्ठौ॒हीर्ब्र॒ह्मणे॑ व॒शां मै᳚त्रावरु॒णाय॑र्\mbox{}ष॒भं ब्रा᳚ह्मणाच्छ॒ꣳ॒सिने॒ वास॑सी नेष्टापो॒तृभ्या॒ꣴ॒ स्थूरि॑ यवाचि॒तम॑च्छावा॒काया॑न॒ड्वाह॑म॒ग्नीधे॑ भार्ग॒वो होता॑ भवति श्राय॒न्तीयं॑ ब्रह्मसा॒मं भ॑वति वारव॒न्तीय॑मग्निष्टोमसा॒मꣳ सा॑रस्व॒तीर॒पो गृ॑ह्णाति॥~(३४)

%1.8.19.0
{\anuvakamend[{वा॒र॒व॒न्तीयं॑ च॒त्वारि॑ च}]}%॥18॥

%1.8.19.1
आ॒ग्ने॒यम॒ष्टा\-क॑पालं॒ निर्व॑पति॒ हिर॑ण्यं॒ दक्षि॑णै॒न्द्रमेका॑\-दश\-कपालमृष॒भो दक्षि॑णा वैश्वदे॒वं च॒रुं पि॒शङ्गी॑ पष्ठौ॒ही दक्षि॑णा मैत्रावरु॒णीमा॒मिक्षां᳚ व॒शा दक्षि॑णा बार्\mbox{}हस्प॒त्यं च॒रुꣳ शि॑तिपृ॒ष्ठो दक्षि॑णा\-ऽ\-ऽदि॒त्यां म॒ल्॒\mbox{}हां ग॒र्भिणी॒मा ल॑भते मारु॒तीं पृश्ञिं॑ पष्ठौ॒हीम॒श्वि\-भ्यां᳚ पू॒ष्णे पु॑रो॒डाशं॒ द्वाद॑श\-कपालं॒ निर्व॑पति॒ सर॑स्वते सत्य॒वाचे॑ च॒रुꣳ स॑वि॒त्रे स॒त्यप्र॑सवाय पुरो॒डाशं॒ द्वाद॑श\-कपालं तिसृध॒न्वꣳ शु॑ष्कदृ॒तिर्दक्षि॑णा॥~(३५)

%1.8.20.0
{\anuvakamend[{आ॒ग्ने॒यꣳ स॒प्तच॑त्वारिꣳशत्}]}%॥19॥

%1.8.20.1
आ॒ग्ने॒यम॒ष्टा\-क॑पालं॒ निर्व॑पति सौ॒म्यं च॒रुꣳ सा॑वि॒त्रं द्वाद॑श\-कपालं बार्\mbox{}हस्प॒त्यं च॒रुं त्वा॒ष्ट्रम॒ष्टाक॑पालं वैश्वान॒रं द्वाद॑श\-कपालं॒ दक्षि॑णो रथवाहनवा॒हो दक्षि॑णा सारस्व॒तं च॒रुं निर्व॑पति पौ॒ष्णं च॒रुं मै॒त्रं च॒रुं वा॑रु॒णं च॒रुं क्षै᳚त्रप॒त्यं च॒रुमा॑दि॒त्यं च॒रुमुत्त॑रो रथवाहनवा॒हो दक्षि॑णा॥~(३६)

%1.8.21.0
{\anuvakamend[{आ॒ग्ने॒यं चतु॑स्त्रिꣳशत्}]}%॥20॥

%1.8.21.1
स्वा॒द्वीं त्वा᳚ स्वा॒दुना॑ ती॒व्रां ती॒व्रेणा॒मृता॑म॒मृते॑न सृ॒जामि॒ सꣳसोमे॑न॒ सोमो᳚\-ऽस्य॒श्वि\-भ्यां᳚ पच्यस्व॒ सर॑स्वत्यै पच्य॒स्वेन्द्रा॑य सु॒त्राम्णे॑ पच्यस्व पु॒नातु॑ ते परि॒स्रुत॒ꣳ॒ सोम॒ꣳ॒ सूर्य॑स्य दुहि॒ता। वारे॑ण॒ शश्व॑ता॒ तना᳚॥ वा॒युः पू॒तः प॒वित्रे॑ण प्र॒त्यङ्ख्सोमो॒ अति॑द्रुतः। इन्द्र॑स्य॒ युज्यः॒ सखा᳚॥ कु॒विद॒ङ्ग यव॑मन्तो॒ यवं॑ चि॒द्यथा॒ दान्त्य॑नुपू॒र्वं वि॒यूय॑। इ॒हेहै॑षां कृणुत॒ भोज॑नानि॒ ये ब॒र्॒\mbox{}हिषो॒ नमो॑वृक्तिं॒ न ज॒ग्मुः॥ आ॒श्वि॒नं धू॒म्रमा ल॑भते सारस्व॒तं मे॒षमै॒न्द्रमृ॑ष॒भमै॒न्द्रमेका॑\-दश\-कपालं॒ निर्व॑पति सावि॒त्रं द्वाद॑श\-कपालं वारु॒णं दश॑\-कपाल॒ꣳ॒ सोम॑प्रतीकाः पितरस्तृप्णुत॒ वड॑बा॒ दक्षि॑णा॥~(३७)

%1.8.22.0
{\anuvakamend[{भोज॑नानि॒ षड्विꣳ॑शतिश्च}]}%॥21॥

%1.8.22.1
अग्ना॑विष्णू॒ महि॒ तद्वां᳚ महि॒त्वं वी॒तं घृ॒तस्य॒ गुह्या॑नि॒ नाम॑। दमे॑दमे स॒प्त रत्ना॒ दधा॑ना॒ प्रति॑ वां जि॒ह्वा घृ॒तमा च॑रण्येत्॥ अग्ना॑विष्णू॒ महि॒ धाम॑ प्रि॒यं वां᳚ वी॒थो घृ॒तस्य॒ गुह्या॑ जुषा॒णा। दमे॑दमे सुष्टु॒तीर्वा॑वृधा॒ना प्रति॑ वां जि॒ह्वा घृ॒तमुच्च॑रण्येत्॥ प्र णो॑ दे॒वी सर॑स्वती॒ वाजे॑भिर्वा॒जिनी॑वती। धी॒नाम॑वि॒त्र्य॑वतु। आ नो॑ दि॒वो बृ॑ह॒तः~(३८)

%1.8.22.2
पर्व॑ता॒दा सर॑स्वती यज॒ता ग॑न्तु य॒ज्ञम्। हवं॑ दे॒वी जु॑जुषा॒णा घृ॒ताची॑ श॒ग्मां नो॒ वाच॑मुश॒ती शृ॑णोतु॥ बृह॑स्पते जु॒षस्व॑ नो ह॒व्यानि॑ विश्वदेव्य। रास्व॒ रत्ना॑नि दा॒शुषे᳚॥ ए॒वा पि॒त्रे वि॒श्वदे॑वाय॒ वृष्णे॑ य॒ज्ञैर्वि॑धेम॒ नम॑सा ह॒विर्भिः॑। बृह॑स्पते सुप्र॒जा वी॒रव॑न्तो व॒यꣴ स्या॑म॒ पत॑यो रयी॒णाम्॥ बृह॑स्पते॒ अति॒ यद॒र्यो अर्\mbox{}हा᳚द्द्यु॒मद्वि॒भाति॒ क्रतु॑म॒ज्जने॑षु। यद्दी॒दय॒च्छव॑स-~(३९)

%1.8.22.3
-र्तप्रजात॒ तद॒स्मासु॒ द्रवि॑णं धेहि चि॒त्रम्॥ आ नो॑ मित्रावरुणा घृ॒तैर्गव्यू॑तिमुक्षतम्। मध्वा॒ रजाꣳ॑सि सुक्रतू॥ प्र बा॒हवा॑ सिसृतं जी॒वसे॑ न॒ आ नो॒ गव्यू॑तिमुक्षतं घृ॒तेन॑। आ नो॒ जने᳚ श्रवयतं युवाना श्रु॒तं मे॑ मित्रावरुणा॒ हवे॒मा॥ अ॒ग्निं वः॑ पू॒र्व्यं गि॒रा दे॒वमी॑डे॒ वसू॑नाम्। स॒प॒र्यन्तः॑ पुरुप्रि॒यं मि॒त्रं न क्षे᳚त्र॒साध॑सम्॥ म॒क्षू दे॒वव॑तो॒ रथः॒~(४०)

%1.8.22.4
शूरो॑ वा पृ॒थ्सु कासु॑ चित्। दे॒वानां॒ य इन्मनो॒ यज॑मान॒ इय॑क्षत्य॒भीदय॑ज्वनो भुवत्॥ न य॑जमान रिष्यसि॒ न सु॑न्वान॒ न दे॑वयो॥ अस॒दत्र॑ सु॒वीर्य॑मु॒त त्यदा॒श्वश्वि॑यम्॥ नकि॒ष्टं कर्म॑णा नश॒न्न प्र यो॑ष॒न्न यो॑षति॥ उप॑ क्षरन्ति॒ सिन्ध॑वो मयो॒भुव॑ ईजा॒नं च॑ य॒क्ष्यमा॑णं च धे॒नवः॑। पृ॒णन्तं॑ च॒ पपु॑रिं च~(४१)

%1.8.22.5
श्रव॒स्यवो॑ घृ॒तस्य॒ धारा॒ उप॑ यन्ति वि॒श्वतः॑॥ सोमा॑रुद्रा॒ वि वृ॑हतं॒ विषू॑ची॒ममी॑वा॒ या नो॒ गय॑मावि॒वेश॑। आ॒रे बा॑धेथां॒ निर्\mbox{}ऋ॑तिं परा॒चैः कृ॒तं चि॒देनः॒ प्रमु॑मुक्तम॒स्मत्॥ सोमा॑रुद्रा यु॒वमे॒तान्य॒स्मे विश्वा॑ त॒नूषु॑ भेष॒जानि॑ धत्तम्। अव॑ स्यतं मु॒ञ्चतं॒ यन्नो॒ अस्ति॑ त॒नूषु॑ ब॒द्धं कृ॒तमेनो॑ अ॒स्मत्॥ सोमा॑पूषणा॒ जन॑ना रयी॒णां जन॑ना दि॒वो जन॑ना पृथि॒व्याः। जा॒तौ विश्व॑स्य॒ भुव॑नस्य गो॒पौ दे॒वा अ॑कृण्वन्न॒मृत॑स्य॒ नाभिम्᳚॥ इ॒मौ दे॒वौ जाय॑मानौ जुषन्ते॒मौ तमाꣳ॑सि गूहता॒मजु॑ष्टा। आ॒भ्यामिन्द्रः॑ प॒क्वमा॒मास्व॒न्तः सो॑मापू॒ष\-भ्यां᳚ जनदु॒स्रिया॑सु॥~(४२)

{\anuvakamend[{बृ॒ह॒तः शव॑सा॒ रथः॒ पपु॑रिं च दि॒वो जन॑ना॒ पञ्च॑विꣳशतिश्च}]}%॥22॥
%%% END PRASHNA
%%% END KANDAM
