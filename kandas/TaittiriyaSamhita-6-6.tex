\chapt{काण्डम् ६}
\sect{षष्ठमः प्रश्नः}\setcounter{anuvakam}{0}
\dnsub{तैत्तिरीयसंहितायां षष्ठमकाण्डे षष्ठमः प्रश्नः}
%6.6.1.1
सु॒व॒र्गाय॒ वा ए॒तानि॑ लो॒काय॑ हूयन्ते॒ यद्दा᳚क्षि॒णानि॒ द्वाभ्यां॒ गार्\mbox{}ह॑पत्ये जुहोति द्वि॒पाद्यज॑मानः॒ प्रति॑ष्ठित्या॒ आग्नी᳚ध्रे जुहोत्य॒न्तरि॑क्ष ए॒वा क्र॑मते॒ सदो॒\-ऽभ्यैति॑ सुव॒र्गमे॒वैनं॑ लो॒कं ग॑मयति सौ॒रीभ्या॑मृ॒ग्भ्यां गार्\mbox{}ह॑पत्ये जुहोत्य॒मुमे॒वैनं॑ लो॒कꣳ स॒मारो॑हयति॒ नय॑वत्य॒र्चाग्नी᳚ध्रे जुहोति सुव॒र्गस्य॑ लो॒कस्या॒भिनी᳚त्यै॒ दिवं॑ गच्छ॒ सुवः॑ प॒तेति॒ हिर॑ण्यम्~(१)

%6.6.1.2
हु॒त्वोद्गृ॑ह्णाति सुव॒र्गमे॒वैनं॑ लो॒कं ग॑मयति रू॒पेण॑ वो रू॒पम॒भ्यैमीत्या॑ह रू॒पेण॒ ह्या॑साꣳ रू॒पम॒भ्यैति॒ यद्धिर॑ण्येन तु॒थो वो॑ वि॒श्ववे॑दा॒ वि भ॑ज॒त्वित्या॑ह तु॒थो ह॑ स्म॒ वै वि॒श्ववे॑दा दे॒वानां॒ दक्षि॑णा॒ वि भ॑जति॒ तेनै॒वैना॒ वि भ॑जत्ये॒तत्ते॑ अग्ने॒ राधः॑~(२)

%6.6.1.3
ऐति॒ सोम॑च्युत॒मित्या॑ह॒ सोम॑च्युत॒ꣴ॒ ह्य॑स्य॒ राध॒ ऐति॒ तन्मि॒त्रस्य॑ प॒था न॒येत्या॑ह॒ शान्त्या॑ ऋ॒तस्य॑ प॒था प्रेत॑ च॒न्द्रद॑क्षिणा॒ इत्या॑ह स॒त्यं वा ऋ॒तꣳ स॒त्येनै॒वैना॑ ऋ॒तेन॒ वि भ॑जति य॒ज्ञस्य॑ प॒था सु॑वि॒ता नय॑न्ती॒रित्या॑ह य॒ज्ञस्य॒ ह्ये॑ताः प॒था यन्ति॒ यद्दक्षि॑णा ब्राह्म॒णम॒द्य रा᳚ध्यासम्~(३)

%6.6.1.4
ऋषि॑मार्\mbox{}षे॒यमित्या॑है॒ष वै ब्रा᳚ह्म॒ण ऋषि॑रार्\mbox{}षे॒यो यः शु॑श्रु॒वान्तस्मा॑दे॒वमा॑ह॒ वि सुवः॒ पश्य॒ व्य॑न्तरि॑क्ष॒मित्या॑ह सुव॒र्गमे॒वैनं॑ लो॒कं ग॑मयति॒ यत॑स्व सद॒स्यै॑रित्या॑ह मित्र॒त्वाया॒स्मद्दा᳚त्रा देव॒त्रा ग॑च्छत॒ मधु॑मतीः प्र दा॒तार॒मा वि॑श॒तेत्या॑ह व॒यमि॒ह प्र॑दा॒तारः॒ स्मो᳚\-ऽस्मान॒मुत्र॒ मधु॑मती॒रा वि॑श॒तेति॑~(४)

%6.6.1.5
वावैतदा॑ह॒ हिर॑ण्यं ददाति॒ ज्योति॒र्वै हिर॑ण्यं॒ ज्योति॑रे॒व पु॒रस्ता᳚द्धत्ते सुव॒र्गस्य॑ लो॒कस्यानु॑ख्यात्या अ॒ग्नीधे॑ ददात्य॒ग्निमु॑खाने॒वर्तून्प्री॑णाति ब्र॒ह्मणे॑ ददाति॒ प्रसू᳚त्यै॒ होत्रे॑ ददात्या॒त्मा वा ए॒ष य॒ज्ञस्य॒ यद्धोता॒त्मान॑मे॒व य॒ज्ञस्य॒ दक्षि॑णाभिः॒ सम॑र्धयति॥~(५)

{\anuvakamend[{हिर॑ण्य॒ꣳ॒ राधो॑ राध्यासम॒मुत्र॒ मधु॑मती॒रा वि॑श॒तेत्य॒ष्टात्रिꣳ॑शच्च}]}%~(१)

%6.6.2.1
स॒मि॒ष्ट॒य॒जूꣳषि॑ जुहोति य॒ज्ञस्य॒ समि॑ष्ट्यै॒ यद्वै य॒ज्ञस्य॑ क्रू॒रं यद्विलि॑ष्टं॒ यद॒त्येति॒ यन्नात्येति॒ यद॑तिक॒रोति॒ यन्नापि॑ क॒रोति॒ तदे॒व तैः प्री॑णाति॒ नव॑ जुहोति॒ नव॒ वै पुरु॑षे प्रा॒णाः पुरु॑षेण य॒ज्ञः सम्मि॑तो॒ यावा॑ने॒व य॒ज्ञस्तं प्री॑णाति॒ षडृग्मि॑याणि जुहोति॒ षड्वा ऋ॒तव॑ ऋ॒तूने॒व प्री॑णाति॒ त्रीणि॒ यजूꣳ॑षि~(६)

%6.6.2.2
त्रय॑ इ॒मे लो॒का इ॒माने॒व लो॒कान्प्री॑णाति॒ यज्ञ॑ य॒ज्ञं ग॑च्छ य॒ज्ञप॑तिं ग॒च्छेत्या॑ह य॒ज्ञप॑तिमे॒वैनं॑ गमयति॒ स्वां योनिं॑ ग॒च्छेत्या॑ह॒ स्वामे॒वैनं॒ योनिं॑ गमयत्ये॒ष ते॑ य॒ज्ञो य॑ज्ञपते स॒हसू᳚क्तवाकः सु॒वीर॒ इत्या॑ह॒ यज॑मान ए॒व वी॒र्यं॑ दधाति वासि॒ष्ठो ह॑ सात्यह॒व्यो दे॑वभा॒गं प॑प्रच्छ॒ यथ्सृञ्ज॑यान्बहुया॒जिनो\-ऽयी॑यजो य॒ज्ञे~(७)

%6.6.2.3
य॒ज्ञं प्रत्य॑तिष्ठि॒पा(३)य॒ज्ञप॒ता(३)विति॒ स हो॑वाच य॒ज्ञप॑ता॒विति॑ स॒त्याद्वै सृञ्ज॑याः॒ परा॑ बभूवु॒रिति॑ होवाच य॒ज्ञे वाव य॒ज्ञः प्र॑ति॒ष्ठाप्य॑ आसी॒द्यज॑मान॒स्याप॑राभावा॒येति॒ देवा॑ गातुविदो गा॒तुं वि॒त्त्वा गा॒तुमि॒तेत्या॑ह य॒ज्ञ ए॒व य॒ज्ञं प्रति॑\-ष्ठापयति॒ यज॑मान॒स्याप॑राभावाय॥~(८)

{\anuvakamend[{यजूꣳ॑षि य॒ज्ञ एक॑चत्वारिꣳशच्च}]}%~(२)

%6.6.3.1
अ॒व॒भृ॒थ॒य॒जूꣳषि॑ जुहोति॒ यदे॒वार्वा॒चीन॒मेक॑हायना॒देनः॑ क॒रोति॒ तदे॒व तैरव॑ यजते॒\-ऽपो॑\-ऽवभृ॒थमवै᳚त्य॒फ्सु वै वरु॑णः सा॒क्षादे॒व वरु॑ण॒मव॑ यजते॒ वर्त्म॑ना॒ वा अ॒न्वित्य॑ य॒ज्ञꣳ रक्षाꣳ॑सि जिघाꣳसन्ति॒ साम्ना᳚ प्रस्तो॒तान्ववै॑ति॒ साम॒ वै र॑क्षो॒हा रक्ष॑सा॒मप॑हत्यै॒ त्रिर्नि॒धन॒मुपै॑ति॒ त्रय॑ इ॒मे लो॒का ए॒भ्य ए॒व लो॒केभ्यो॒ रक्षाꣳ॑सि~(९)

%6.6.3.2
अप॑ हन्ति॒ पुरु॑षःपुरुषो नि॒धन॒मुपै॑ति॒ पुरु॑षःपुरुषो॒ हि र॑क्ष॒स्वी रक्ष॑सा॒मप॑हत्या उ॒रुꣳ हि राजा॒ वरु॑णश्च॒कारेत्या॑ह॒ प्रति॑ष्ठित्यै श॒तं ते॑ राजन्भि॒षजः॑ स॒हस्र॒मित्या॑ह भेष॒जमे॒वास्मै॑ करोत्य॒भिष्ठि॑तो॒ वरु॑णस्य॒ पाश॒ इत्या॑ह वरुणपा॒शमे॒वाभि ति॑ष्ठति ब॒र्॒\mbox{}हिर॒भि जु॑हो॒त्याहु॑तीनां॒ प्रति॑ष्ठित्या॒ अथो॑ अग्नि॒वत्ये॒व जु॑हो॒त्यप॑बर्\mbox{}हिषः प्रया॒जान्~(१०)

%6.6.3.3
य॒ज॒ति॒ प्र॒जा वै ब॒र्॒\mbox{}हिः प्र॒जा ए॒व व॑रुणपा॒शान्मु॑ञ्च॒त्याज्य॑भागौ यजति य॒ज्ञस्यै॒व चक्षु॑षी॒ नान्तरे॑ति॒ वरु॑णं यजति वरुणपा॒शादे॒वैनं॑ मुञ्चत्य॒ग्नीवरु॑णौ यजति सा॒क्षादे॒वैनं॑ वरुणपा॒शान्मु॑ञ्च॒त्यप॑बर्\mbox{}हिषावनूया॒जौ य॑जति प्र॒जा वै ब॒र्॒\mbox{}हिः प्र॒जा ए॒व व॑रुणपा॒शान्मु॑ञ्चति च॒तुरः॑ प्रया॒जान् य॑जति॒ द्वाव॑नूया॒जौ षट्थ्सं प॑द्यन्ते॒ षड्वा ऋ॒तवः॑~(११)

%6.6.3.4
ऋ॒तुष्वे॒व प्रति॑ तिष्ठ॒त्यव॑भृथ निचङ्कु॒णेत्या॑ह यथोदि॒तमे॒व वरु॑ण॒मव॑ यजते समु॒द्रे ते॒ हृद॑यम॒फ्स्व॑न्तरित्या॑ह समु॒द्रे ह्य॑न्तर्वरु॑णः॒ सं त्वा॑ विश॒न्त्वोष॑धीरु॒ताप॒ इत्या॑हा॒द्भिरे॒वैन॒मोष॑धीभिः स॒म्यञ्चं॑ दधाति॒ देवी॑राप ए॒ष वो॒ गर्भ॒ इत्या॑ह यथाय॒जुरे॒वैतत्प॒शवो॒ वै~(१२)

%6.6.3.5
सोमो॒ यद्भि॑न्दू॒नां भ॒क्षये᳚त्पशु॒मान्थ्स्या॒द्वरु॑ण॒स्त्वे॑नं गृह्णीया॒द्यन्न भ॒क्षये॑दप॒शुः स्या॒न्नैनं॒ वरु॑णो गृह्णीयादुप॒स्पृश्य॑मे॒व प॑शु॒मान्भ॑वति॒ नैनं॒ वरु॑णो गृह्णाति॒ प्रति॑युतो॒ वरु॑णस्य॒ पाश॒ इत्या॑ह वरुणपा॒शादे॒व निर्मु॑च्य॒ते\-ऽप्र॑तीक्ष॒मा य॑न्ति॒ वरु॑णस्या॒न्तर्\mbox{}हि॑त्या॒ एधो᳚\-ऽस्येधिषीम॒हीत्या॑ह स॒मिधै॒वाग्निं न॑म॒स्यन्त॑ उ॒पाय॑न्ति॒ तेजो॑\-ऽसि॒ तेजो॒ मयि॑ धे॒हीत्या॑ह॒ तेज॑ ए॒वा\-ऽऽ\-त्मन्ध॑त्ते॥~(१३)

{\anuvakamend[{रक्षाꣳ॑सि प्रया॒जानृ॒तवो॒ वै न॑म॒स्यन्तो॒ द्वाद॑श च}]}%~(३)

%6.6.4.1
स्फ्येन॒ वेदि॒मुद्ध॑न्ति रथा॒क्षेण॒ वि मि॑मीते॒ यूपं॑ मिनोति त्रि॒वृत॑मे॒व वज्रꣳ॑ स॒म्भृत्य॒ भ्रातृ॑व्याय॒ प्र ह॑रति॒ स्तृत्यै॒ यद॑न्तर्वे॒दि मि॑नु॒याद्दे॑वलो॒कम॒भि ज॑ये॒द्यद्ब॑हिर्वे॒दि म॑नुष्यलो॒कं वे᳚द्य॒न्तस्य॑ सं॒धौ मि॑नोत्यु॒भयो᳚र्लो॒कयो॑र॒भिजि॑त्या॒ उप॑रसम्मितां मिनुयात्पितृलो॒कका॑मस्य रश॒नस॑म्मितां मनुष्यलो॒कका॑मस्य च॒षाल॑सम्मितामिन्द्रि॒यका॑मस्य॒ सर्वा᳚न्थ्स॒मान्प्र॑ति॒ष्ठाका॑मस्य॒ ये त्रयो॑ मध्य॒मास्तान्थ्स॒मान्प॒शुका॑मस्यै॒तान् वै~(१४)

%6.6.4.2
अनु॑ प॒शव॒ उप॑ तिष्ठन्ते पशु॒माने॒व भ॑वति॒ व्यति॑षजे॒दित॑रान्प्र॒जयै॒वैनं॑ प॒शुभि॒र्व्यति॑षजति॒ यं का॒मये॑त प्र॒मायु॑कः स्या॒दिति॑ गर्त॒मितं॒ तस्य॑ मिनुयादुत्तरा॒र्ध्यं॑ वर्\mbox{}षि॑ष्ठ॒मथ॒ ह्रसी॑याꣳसमे॒षा वै ग॑र्त॒मिद्यस्यै॒वं मि॒नोति॑ ता॒जक्प्र मी॑यते दक्षिणा॒र्ध्यं॑ वर्\mbox{}षि॑ष्ठं मिनुयाथ्सुव॒र्गका॑म॒स्याथ॒ ह्रसी॑याꣳसमा॒क्रम॑णमे॒व तथ्सेतुं॒ यज॑मानः कुरुते सुव॒र्गस्य॑ लो॒कस्य॒ सम॑ष्ट्यै~(१५)

%6.6.4.3
यदेक॑स्मि॒न्॒ यूपे॒ द्वे र॑श॒ने प॑रि॒व्यय॑ति॒ तस्मा॒देको॒ द्वे जा॒ये वि॑न्दते॒ यन्नैकाꣳ॑ रश॒नां द्वयो॒र्यूप॑योः परि॒व्यय॑ति॒ तस्मा॒न्नैका॒ द्वौ पती॑ विन्दते॒ यं का॒मये॑त॒ स्त्र्य॑स्य जाये॒तेत्यु॑पा॒न्ते तस्य॒ व्यति॑षजे॒थ्स्त्र्ये॑वास्य॑ जायते॒ यं का॒मये॑त॒ पुमा॑नस्य जाये॒तेत्या॒न्तं तस्य॒ प्र वे᳚ष्टये॒त्पुमा॑ने॒वास्य॑~(१६)

%6.6.4.4
जा॒य॒ते\-ऽसु॑रा॒ वै दे॒वान्द॑क्षिण॒त उपा॑नय॒न्तां दे॒वा उ॑पश॒येनै॒वापा॑नुदन्त॒ तदु॑पश॒यस्यो॑पशय॒त्वं यद्द॑क्षिण॒त उ॑पश॒य उ॑प॒शये॒ भ्रातृ॑व्यापनुत्त्यै॒ सर्वे॒ वा अ॒न्ये यूपाः᳚ पशु॒मन्तो\-ऽथो॑पश॒य ए॒वाप॒शुस्तस्य॒ यज॑मानः प॒शुर्यन्न नि॑र्दि॒शेदार्ति॒\-मार्च्छे॒द्यज॑मानो॒\-ऽसौ ते॑ प॒शुरिति॒ निर्दि॑शे॒द्यं द्वि॒ष्याद्यमे॒व~(१७)

%6.6.4.5
द्वेष्टि॒ तम॑स्मै प॒शुं निर्दि॑शति॒ यदि॒ न द्वि॒ष्यादा॒खुस्ते॑ प॒शुरिति॑ ब्रूया॒न्न ग्रा॒म्यान्प॒शून् हि॒नस्ति॒ नार॒ण्यान्प्र॒जा\-प॑तिः प्र॒जा अ॑सृजत॒ सो᳚\-ऽन्नाद्ये॑न॒ व्या᳚र्ध्यत॒ स ए॒तामे॑काद॒शिनी॑मपश्य॒त्तया॒ वै सो᳚\-ऽन्नाद्य॒मवा॑रुन्ध॒ यद्दश॒ यूपा॒ भव॑न्ति॒ दशा᳚क्षरा वि॒राडन्नं॑ वि॒राड्वि॒राजै॒वान्नाद्य॒मव॑ रुन्धे~(१८)

%6.6.4.6
य ए॑काद॒शः स्तन॑ ए॒वास्यै॒ स दु॒ह ए॒वैनां॒ तेन॒ वज्रो॒ वा ए॒षा सम्मी॑यते॒ यदे॑काद॒शिनी॒ सेश्व॒रा पु॒रस्ता᳚त्प्र॒त्यञ्चं॑ य॒ज्ञꣳ सम्म॑र्दितो॒र्यत्पा᳚त्नीव॒तं मि॒नोति॑ य॒ज्ञस्य॒ प्रत्युत्त॑ब्ध्यै सय॒त्वाय॑॥~(१९)

{\anuvakamend[{वै सम॑ष्ट्यै॒ पुमा॑ने॒वास्य॒ यमे॒व रु॑न्धे त्रि॒ꣳ॒शच्च॑}]}%~(४)

%6.6.5.1
प्र॒जा\-प॑तिः प्र॒जा अ॑सृजत॒ स रि॑रिचा॒नो॑\-ऽमन्यत॒ स ए॒तामे॑काद॒शिनी॑मपश्य॒त्तया॒ वै स आयु॑रिन्द्रि॒यं वी॒र्य॑मा॒त्मन्न॑धत्त प्र॒जा इ॑व॒ खलु॒ वा ए॒ष सृ॑जते॒ यो यज॑ते॒ स ए॒तर्\mbox{}हि॑ रिरिचा॒न इ॑व॒ यदे॒षैका॑द॒शिनी॒ भव॒त्यायु॑रे॒व तये᳚न्द्रि॒यं वी॒र्यं॑ यज॑मान आ॒त्मन्ध॑त्ते॒ प्रैवा\-ऽऽ\-ग्ने॒येन॑ वापयति मिथु॒नꣳ सा॑रस्व॒त्या क॑रोति॒ रेतः॑~(२०)

%6.6.5.2
सौ॒म्येन॑ दधाति॒ प्र ज॑नयति पौ॒ष्णेन॑ बार्\mbox{}हस्प॒त्यो भ॑वति॒ ब्रह्म॒ वै दे॒वानां॒ बृह॒स्पति॒र्ब्रह्म॑णै॒वास्मै᳚ प्र॒जाः प्र ज॑नयति वैश्वदे॒वो भ॑वति वैश्वदे॒व्यो॑ वै प्र॒जाः प्र॒जा ए॒वास्मै॒ प्र ज॑नयतीन्द्रि॒यमे॒वैन्द्रेणाव॑रुन्धे॒ विशं॑ मारु॒तेनौजो॒ बल॑मैन्द्रा॒ग्नेन॑ प्रस॒वाय॑ सावि॒त्रो नि॑र्वरुण॒त्वाय॑ वारु॒णो म॑ध्य॒त ऐ॒न्द्रमा ल॑भते मध्य॒त ए॒वेन्द्रि॒यं यज॑माने दधाति~(२१)

%6.6.5.3
पु॒रस्ता॑दै॒न्द्रस्य॑ वैश्वदे॒वमाल॑भते वैश्वदे॒वं वा अन्न॒मन्न॑मे॒व पु॒रस्ता᳚द्धत्ते॒ तस्मा᳚त्पु॒रस्ता॒दन्न॑मद्यत ऐ॒न्द्रमा॒लभ्य॑ मारु॒तमा ल॑भते॒ विड्वै म॒रुतो॒ विश॑मे॒वास्मा॒ अनु॑ बध्नाति॒ यदि॑ का॒मये॑त॒ यो\-ऽव॑गतः॒ सो\-ऽप॑ रुध्यतां॒ यो\-ऽप॑रुद्धः॒ सो\-ऽव॑ गच्छ॒त्वित्यै॒न्द्रस्य॑ लो॒के वा॑रु॒णमा ल॑भेत वारु॒णस्य॑ लो॒क ऐ॒न्द्रम्~(२२)

%6.6.5.4
य ए॒वाव॑गतः॒ सो\-ऽप॑ रुध्यते॒ यो\-ऽप॑रुद्धः॒ सो\-ऽव॑ गच्छति॒ यदि॑ का॒मये॑त प्र॒जा मु॑ह्येयु॒रिति॑ प॒शून्व्यति॑षजेत्प्र॒जा ए॒व मो॑हयति॒ यद॑भिवाह॒तो॑\-ऽपां वा॑रु॒णमा॒लभे॑त प्र॒जा वरु॑णो गृह्णीयाद्दक्षिण॒त उद॑ञ्च॒मा ल॑भते\-ऽपवाह॒तो॑\-ऽ\-पां प्र॒जाना॒मव॑रुणग्राहाय॥~(२३)

{\anuvakamend[{रेतो॒ यज॑माने दधाति लो॒क ऐ॒न्द्रꣳ स॒प्तत्रिꣳ॑शच्च}]}%~(५)

%6.6.6.1
इन्द्रः॒ पत्नि॑या॒ मनु॑मयाजय॒त्तां पर्य॑ग्निकृता॒मुद॑सृज॒त्तया॒ मनु॑रार्ध्नो॒द्यत्पर्य॑ग्निकृतं पात्नीव॒तमु॑थ्सृ॒जति॒ यामे॒व मनु॒र्॒\mbox{}ऋद्धि॒\-मार्ध्नो॒त्तामे॒व यज॑मान ऋध्नोति य॒ज्ञस्य॒ वा अप्र॑तिष्ठिताद्य॒ज्ञः परा॑ भवति य॒ज्ञं प॑रा॒भव॑न्तं॒ यज॑मा॒नो\-ऽनु॒ परा॑ भवति॒ यदाज्ये॑न पात्नीव॒तꣳ सꣴ॑स्था॒पय॑ति य॒ज्ञस्य॒ प्रति॑ष्ठित्यै य॒ज्ञं प्र॑ति॒तिष्ठ॑न्तं॒ यज॑मा॒नो\-ऽनु॒ प्रति॑ तिष्ठती॒ष्टं व॒पया᳚~(२४)

%6.6.6.2
भव॒त्यनि॑ष्टं व॒शयाथ॑ पात्नीव॒तेन॒ प्र च॑रति ती॒र्थ ए॒व प्र च॑र॒त्यथो॑ ए॒तर्\mbox{}ह्ये॒वास्य॒ याम॑स्त्वा॒ष्ट्रो भ॑वति॒ त्वष्टा॒ वै रेत॑सः सि॒क्तस्य॑ रू॒पाणि॒ वि क॑रोति॒ तमे॒व वृ॑षाणं॒ पत्नी॒ष्वपि॑ सृजति॒ सो᳚\-ऽस्मै रू॒पाणि॒ वि क॑रोति॥~(२५)

{\anuvakamend[{व॒पया॒ षट्त्रिꣳ॑शच्च}]}%~(६)

%6.6.7.1
घ्नन्ति॒ वा ए॒तथ्सोमं॒ यद॑भिषु॒ण्वन्ति॒ यथ्सौ॒म्यो भव॑ति॒ यथा॑ मृ॒ताया॑नु॒स्तर॑णीं॒ घ्नन्ति॑ ता॒दृगे॒व तद्यदु॑त्तरा॒र्धे वा॒ मध्ये॑ वा जुहु॒याद्दे॒वता᳚भ्यः स॒मदं॑ दध्याद्दक्षिणा॒र्धे जु॑होत्ये॒षा वै पि॑तृ॒णां दिख्स्वाया॑मे॒व दि॒शि पि॒तॄन्नि॒रव॑दयत उद्गा॒तृभ्यो॑ हरन्ति सामदेव॒त्यो॑ वै सौ॒म्यो यदे॒व साम्न॑श्छम्बट्कु॒र्वन्ति॒ तस्यै॒व स शान्ति॒रव॑~(२६)

%6.6.7.2
ई॒क्ष॒न्ते॒ प॒वित्रं॒ वै सौ॒म्य आ॒त्मान॑मे॒व प॑वयन्ते॒ य आ॒त्मानं॒ न प॑रि॒पश्ये॑दि॒तासुः॑ स्यादभिद॒दिं कृ॒त्वावे᳚क्षेत॒ तस्मि॒न् ह्या᳚त्मानं॑ परि॒पश्य॒त्यथो॑ आ॒त्मान॑मे॒व प॑वयते॒ यो ग॒तम॑नाः॒ स्याथ्सो\-ऽवे᳚क्षेत॒ यन्मे॒ मनः॒ परा॑गतं॒ यद्वा॑ मे॒ अप॑रागतम्। राज्ञा॒ सोमे॑न॒ तद्व॒यम॒स्मासु॑ धारयाम॒सीति॒ मन॑ ए॒वा\-ऽऽ\-त्मन्दा॑धार~(२७)

%6.6.7.3
न ग॒तम॑ना भव॒त्यप॒ वै तृ॑तीयसव॒ने य॒ज्ञः क्रा॑मतीजा॒नादनी॑जानम॒भ्या᳚ग्नावैष्ण॒व्यर्चा घृ॒तस्य॑ यजत्य॒ग्निः सर्वा॑ दे॒वता॒ विष्णु॑र्य॒ज्ञो दे॒वता᳚श्चै॒व य॒ज्ञं च॑ दाधारोपा॒ꣳ॒शु य॑जति मिथुन॒त्वाय॑ ब्रह्मवा॒दिनो॑ वदन्ति मि॒त्रो य॒ज्ञस्य॒ स्वि॑ष्टं युवते॒ वरु॑णो॒ दुरि॑ष्टं॒ क्व॑ तर्\mbox{}हि॑ य॒ज्ञः क्व॑ यज॑मानो भव॒तीति॒ यन्मै᳚त्रावरु॒णीं व॒शामा॒लभ॑ते मि॒त्रेणै॒व~(२८)

%6.6.7.4
य॒ज्ञस्य॒ स्वि॑ष्टꣳ शमयति॒ वरु॑णेन॒ दुरि॑ष्टं॒ नार्ति॒मार्च्छ॑ति॒ यज॑मानो॒ यथा॒ वै लाङ्ग॑लेनो॒र्वरां᳚ प्रभि॒न्दन्त्ये॒वमृ॑ख्सा॒मे य॒ज्ञं प्र भि॑न्तो॒ यन्मै᳚त्रावरु॒णीं व॒शामा॒लभ॑ते य॒ज्ञायै॒व प्रभि॑न्नाय म॒त्य॑म॒न्ववा᳚स्यति॒ शान्त्यै॑ या॒तया॑मानि॒ वा ए॒तस्य॒ छन्दाꣳ॑सि॒ य ई॑जा॒नश्छन्द॑सामे॒ष रसो॒ यद्व॒शा यन्मै᳚त्रावरु॒णीं व॒शामा॒लभ॑ते॒ छन्दाꣴ॑स्ये॒व पुन॒रा प्री॑णा॒त्यया॑तयामत्वा॒याथो॒ छन्दः॑स्वे॒व रसं॑ दधाति॥~(२९)

{\anuvakamend[{अव॑ दाधार मि॒त्रेणै॒व प्री॑णाति॒ षट्च॑}]}%~(७)

%6.6.8.1
दे॒वा वा इ॑न्द्रि॒यं वी॒र्यं  व्य॑भजन्त॒ ततो॒ यद॒त्यशि॑ष्यत॒ तद॑तिग्रा॒ह्या॑ अभव॒न्तद॑तिग्रा॒ह्या॑णामतिग्राह्य॒त्वं यद॑तिग्रा॒ह्या॑ गृ॒ह्यन्त॑ इन्द्रि॒यमे॒व तद्वी॒र्यं॑ यज॑मान आ॒त्मन्ध॑त्ते॒ तेज॑ आग्ने॒येने᳚न्द्रि॒यमै॒न्द्रेण॑ ब्रह्मवर्च॒सꣳ सौ॒र्येणो॑प॒स्तम्भ॑नं॒ वा ए॒तद्य॒ज्ञस्य॒ यद॑तिग्रा॒ह्या᳚श्च॒क्रे पृ॒ष्ठानि॒ यत्पृष्ठ्ये॒ न गृ॑ह्णी॒यात्प्राञ्चं॑ य॒ज्ञं पृ॒ष्ठानि॒ सꣳ शृ॑णीयु॒र्यदु॒क्थ्ये᳚~(३०)

%6.6.8.2
गृ॒ह्णी॒यात्प्र॒त्यञ्चं॑ य॒ज्ञम॑तिग्रा॒ह्याः᳚ सꣳ शृ॑णीयुर्विश्व॒जिति॒ सर्व॑पृष्ठे ग्रहीत॒व्या॑ य॒ज्ञस्य॑ सवीर्य॒त्वाय॑ प्र॒जा\-प॑तिर्दे॒वेभ्यो॑ य॒ज्ञान्व्यादि॑श॒थ्स प्रि॒यास्त॒नूरप॒ न्य॑धत्त॒ तद॑तिग्रा॒ह्या॑ अभव॒न्वित॑नु॒स्तस्य॑ य॒ज्ञ इत्या॑हु॒र्यस्या॑तिग्रा॒ह्या॑ न गृ॒ह्यन्त॒ इत्यप्य॑ग्निष्टो॒मे ग्र॑हीत॒व्या॑ य॒ज्ञस्य॑ सतनु॒त्वाय॑ दे॒वता॒ वै सर्वाः᳚ स॒दृशी॑रास॒न्ता न व्या॒वृत᳚मगच्छ॒न्ते दे॒वाः~(३१)

%6.6.8.3
ए॒त ए॒तान्ग्रहा॑नपश्य॒न्तान॑गृह्णताग्ने॒यम॒ग्निरै॒न्द्रमिन्द्रः॑ सौ॒र्यꣳ सूर्य॒स्ततो॒ वै ते᳚\-ऽन्याभि॑र्दे॒वता॑भिर्व्या॒वृत॑मगच्छ॒न्॒ यस्यै॒वं वि॒दुष॑ ए॒ते ग्रहा॑ गृ॒ह्यन्ते᳚ व्या॒वृत॑मे॒व पा॒प्मना॒ भ्रातृ॑व्येण गच्छती॒मे लो॒का ज्योति॑ष्मन्तः स॒माव॑द्वीर्याः का॒र्या॑ इत्या॑हुराग्ने॒येना॒स्मिँल्लो॒के ज्योति॑र्धत्त ऐ॒न्द्रेणा॒न्तरि॑क्ष इन्द्रवा॒यू हि स॒युजौ॑ सौ॒र्येणा॒मुष्मिँ॑ल्लो॒के~(३२)

%6.6.8.4
ज्योति॑र्धत्ते॒ ज्योति॑ष्मन्तो\-ऽस्मा इ॒मे लो॒का भ॑वन्ति स॒माव॑द्वीर्यानेनान्कुरुत ए॒तान् वै ग्रहा᳚न्ब॒म्बावि॒श्वव॑यसाववित्तां॒ ताभ्या॑मि॒मे लो॒काः परा᳚ञ्चश्चा॒र्वाञ्च॑श्च॒ प्राभु॒र्यस्यै॒वं वि॒दुष॑ ए॒ते ग्रहा॑ गृ॒ह्यन्ते॒ प्रास्मा॑ इ॒मे लो॒काः परा᳚ञ्चश्चा॒र्वाञ्च॑श्च भान्ति॥~(३३)

{\anuvakamend[{उ॒क्थ्ये॑ दे॒वा अ॒मुष्मिँ॑ल्लो॒क एका॒न्नच॑त्वारि॒ꣳ॒शच्च॑}]}%~(८)

%6.6.9.1
दे॒वा वै यद्य॒ज्ञे\-ऽकु॑र्वत॒ तदसु॑रा अकुर्वत॒ ते दे॒वा अदा᳚भ्ये॒ छन्दाꣳ॑सि॒ सव॑नानि॒ सम॑स्थापय॒न्ततो॑ दे॒वा अभ॑व॒न्परासु॑रा॒ यस्यै॒वं वि॒दुषो\-ऽदा᳚भ्यो गृ॒ह्यते॒ भव॑त्या॒त्मना॒ परा᳚स्य॒ भ्रातृ॑व्यो भवति॒ यद्वै दे॒वा असु॑रा॒नदा᳚भ्ये॒ना\-द॑भ्नुव॒न्तददा᳚भ्यस्यादाभ्य॒त्वं य ए॒वं वेद॑ द॒भ्नोत्ये॒व भ्रातृ॑व्यं॒ नैन॒म्भ्रातृ॑व्यो दभ्नोति~(३४)

%6.6.9.2
ए॒षा वै प्र॒जा\-प॑तेरतिमो॒क्षिणी॒ नाम॑ त॒नूर्यददा᳚भ्य॒ उप॑नद्धस्य गृह्णा॒त्यति॑मुक्त्या॒ अति॑ पा॒प्मान॒म्भ्रातृ॑व्यं मुच्यते॒ य ए॒वं वेद॒ घ्नन्ति॒ वा ए॒तथ्सोमं॒ यद॑भिषु॒ण्वन्ति॒ सोमे॑ ह॒न्यमा॑ने य॒ज्ञो ह॑न्यते य॒ज्ञे यज॑मानो ब्रह्मवा॒दिनो॑ वदन्ति॒ किं तद्य॒ज्ञे यज॑मानः कुरुते॒ येन॒ जीव᳚न्थ्सुव॒र्गं लो॒कमेतीति॑ जीवग्र॒हो वा ए॒ष यददा॒भ्यो\-ऽन॑भिषुतस्य गृह्णाति॒ जीव॑न्तमे॒वैनꣳ॑ सुव॒र्गं लो॒कं ग॑मयति॒ वि वा ए॒तद्य॒ज्ञं छि॑न्दन्ति॒ यददा᳚भ्ये सꣴस्था॒पय॑न्त्य॒ꣳ॒शूनपि॑ सृजति य॒ज्ञस्य॒ सन्त॑त्यै॥~(३५)

{\anuvakamend[{द॒भ्नो॒त्यन॑भिषुतस्य गृह्णा॒त्येका॒न्नविꣳ॑श॒तिश्च॑}]}%~(९)

%6.6.10.1
दे॒वा वै प्र॒बाहु॒ग्ग्रहा॑नगृह्णत॒ स ए॒तं प्र॒जा\-प॑तिर॒ꣳ॒शुम॑पश्य॒त्तम॑गृह्णीत॒ तेन॒ वै स आ᳚र्ध्नो॒द्यस्यै॒वं वि॒दुषो॒\-ऽꣳ॒शुर्गृ॒ह्यत॑ ऋ॒ध्नोत्ये॒व स॒कृद॑भिषुतस्य गृह्णाति स॒कृद्धि स तेनार्ध्नो॒न्मन॑सा गृह्णाति॒ मन॑ इव॒ हि प्र॒जा\-प॑तिः प्र॒जा\-प॑ते॒राप्त्या॒ औदु॑म्बरेण गृह्णा॒त्यूर्ग्वा उ॑दु॒म्बर॒ ऊर्ज॑मे॒वाव॑ रुन्धे॒ चतुः॑स्रक्ति भवति दि॒क्षु~(३६)

%6.6.10.2
ए॒व प्रति॑ तिष्ठति॒ यो वा अ॒ꣳ॒शोरा॒यत॑नं॒ वेदा॒यत॑नवान्भवति वामदे॒व्यमिति॒ साम॒ तद्वा अ॑स्या॒यत॑नं॒ मन॑सा॒ गाय॑मानो गृह्णात्या॒यत॑नवाने॒व भ॑वति॒ यद॑ध्व॒र्युर॒ꣳ॒शुं गृ॒ह्णन्नार्धये॑दु॒भाभ्यां॒ नर्ध्ये॑ताध्व॒र्यवे॑ च॒ यज॑मानाय च॒ यद॒र्धये॑दु॒भाभ्या॑मृध्ये॒तान॑वानं गृह्णाति॒ सैवास्यर्द्धि॒र्॒\mbox{}हिर॑ण्यम॒भि व्य॑नित्य॒मृतं॒ वै हिर॑ण्य॒मायुः॑ प्रा॒ण आयु॑षै॒वामृत॑म॒भि धि॑नोति श॒तमा॑नं भवति श॒तायुः॒ पुरु॑षः श॒तेन्द्रि॑य॒ आयु॑ष्ये॒वेन्द्रि॒ये प्रति॑ तिष्ठति॥~(३७)

{\anuvakamend[{दि॒क्ष्व॑निति विꣳश॒तिश्च॑}]}%॥10॥

%6.6.11.1
प्र॒जा\-प॑तिर्दे॒वेभ्यो॑ य॒ज्ञान्व्यादि॑श॒थ्स रि॑रिचा॒नो॑\-ऽमन्यत॒ स य॒ज्ञानाꣳ॑ षोडश॒धेन्द्रि॒यं वी॒र्य॑मा॒त्मान॑म॒भि सम॑क्खिद॒त् तथ्षो॑ड॒श्य॑भव॒न्न वै षो॑ड॒शी नाम॑ य॒ज्ञो᳚\-ऽस्ति॒ यद्वाव षो॑ड॒शꣴ स्तो॒त्रꣳ षो॑ड॒शꣳ श॒स्त्रं तेन॑ षोड॒शी तथ्षो॑ड॒शिनः॑ षोडशि॒त्वं यथ्षो॑ड॒शी गृ॒ह्यत॑ इन्द्रि॒यमे॒व तद्वी॒र्यं॑ यज॑मान आ॒त्मन्ध॑त्ते दे॒वेभ्यो॒ वै सु॑व॒र्गो लो॒कः~(३८)

%6.6.11.2
न प्राभ॑व॒त्त ए॒तꣳ षो॑ड॒शिन॑मपश्य॒न्तम॑गृह्णत॒ ततो॒ वै तेभ्यः॑ सुव॒र्गो लो॒कः प्राभ॑व॒द्यथ्षो॑ड॒शी गृ॒ह्यते॑ सुव॒र्गस्य॑ लो॒कस्या॒भिजि॑त्या॒ इन्द्रो॒ वै दे॒वाना॑मानुजाव॒र आ॑सी॒थ्स प्र॒जा\-प॑ति॒मुपा॑धाव॒त्तस्मा॑ ए॒तꣳ षो॑ड॒शिनं॒ प्राय॑च्छ॒त्तम॑गृह्णीत॒ ततो॒ वै सो\-ऽग्रं॑ दे॒वता॑नां॒ पर्यै॒द्यस्यै॒वं वि॒दुषः॑ षोड॒शी गृ॒ह्यते᳚~(३९)

%6.6.11.3
अग्र॑मे॒व स॑मा॒नानां॒ पर्ये॑ति प्रातःसव॒ने गृ॑ह्णाति॒ वज्रो॒ वै षो॑ड॒शी वज्रः॑ प्रातःसव॒नꣴ स्वादे॒वैनं॒ योने॒र्निर्गृ॑ह्णाति॒ सव॑नेसवने॒\-ऽभि गृ॑ह्णाति॒ सव॑नाथ्सवनादे॒वैनं॒ प्र ज॑नयति तृतीयसव॒ने प॒शुका॑मस्य गृह्णीया॒द्वज्रो॒ वै षो॑ड॒शी प॒शव॑स्तृतीयसव॒नं वज्रे॑णै॒वास्मै॑ तृतीयसव॒नात्प॒शूनव॑ रुन्धे॒ नोक्थ्ये॑ गृह्णीयात्प्र॒जा वै प॒शव॑ उ॒क्थानि॒ यदु॒क्थ्ये᳚~(४०)

%6.6.11.4
गृ॒ह्णी॒यात्प्र॒जां प॒शून॑स्य॒ निर्द॑हेदतिरा॒त्रे प॒शुका॑मस्य गृह्णीया॒द्वज्रो॒ वै षो॑ड॒शी वज्रे॑णै॒वास्मै॑ प॒शून॑व॒रुध्य॒ रात्रि॑यो॒परि॑ष्टाच्छमय॒त्यप्य॑ग्निष्टो॒मे रा॑ज॒न्य॑स्य गृह्णीयाद्व्या॒वृत्का॑मो॒ हि रा॑ज॒न्यो॑ यज॑ते सा॒ह्न ए॒वास्मै॒ वज्रं॑ गृह्णाति॒ स ए॑नं॒ वज्रो॒ भूत्या॑ इन्द्धे॒ निर्वा दहत्येकवि॒ꣳ॒शꣴ स्तो॒त्रं भ॑वति॒ प्रति॑ष्ठित्यै॒ हरि॑वच्छस्यत॒ इन्द्र॑स्य प्रि॒यं धाम॑~(४१)

%6.6.11.5
उपा᳚प्नोति॒ कनी॑याꣳसि॒ वै दे॒वेषु॒ छन्दा॒ꣴ॒स्यास॒ञ्ज्याया॒ꣴ॒स्यसु॑रेषु॒ ते दे॒वाः कनी॑यसा॒ छन्द॑सा॒ ज्याय॒श्छन्दो॒\-ऽभि व्य॑शꣳस॒न्ततो॒ वै ते\-ऽसु॑राणां लो॒कम॑वृञ्जत॒ यत्कनी॑यसा॒ छन्द॑सा॒ ज्याय॒श्छन्दो॒\-ऽभि वि॒शꣳस॑ति॒ भ्रातृ॑व्यस्यै॒व तल्लो॒कं वृ॑ङ्क्ते॒ षड॒क्षरा॒ण्यति॑ रेचयन्ति॒ षड्वा ऋ॒तव॑ ऋ॒तूने॒व प्री॑णाति च॒त्वारि॒ पूर्वा॒ण्यव॑ कल्पयन्ति~(४२)

%6.6.11.6
चतु॑ष्पद ए॒व प॒शूनव॑ रुन्धे॒ द्वे उत्त॑रे द्वि॒पद॑ ए॒वाव॑ रुन्धे\-ऽनु॒ष्टुभ॑म॒भि सम्पा॑दयन्ति॒ वाग्वा अ॑नु॒ष्टुप्तस्मा᳚त्प्रा॒णानां॒ वागु॑त्त॒मा स॑मयाविषि॒ते सूर्ये॑ षोड॒शिनः॑ स्तो॒त्रमु॒पाक॑रोत्ये॒तस्मि॒न्वै लो॒क इन्द्रो॑ वृ॒त्रम॑हन्थ्सा॒क्षादे॒व वज्र॒म्भ्रातृ॑व्याय॒ प्र ह॑रत्यरुणपिशं॒गो\-ऽश्वो॒ दक्षि॑णै॒तद्वै वज्र॑स्य रू॒पꣳ समृ॑द्ध्यै~(४३)


{\anuvakamend[{लो॒को वि॒दुषः॑ षोड॒शी गृ॒ह्यते॒ यदु॒क्थ्ये॑ धाम॑ कल्पयन्ति स॒प्तच॑त्वारिꣳशच्च}]}%॥11॥

{\prashnaend[{सु॒व॒र्गाय॒ यद्दा᳚क्षि॒णानि॑ समिष्टय॒जूꣳष्य॑वभृथय॒जूꣳषि॒ स्फ्येन॑ प्र॒जा\-प॑तिरेकाद॒शिनी॒मिन्द्रः॒ पत्नि॑या॒ घ्नन्ति॑ दे॒वा वा इ॑न्द्रि॒यं दे॒वा वा अदा᳚भ्ये दे॒वा वै प्र॒बाहु॑क्प्र॒जा\-प॑तिर्दे॒वेभ्यः॒ स रि॑रिचा॒नः षो॑डश॒धैका॑\-दश॥११॥ सु॒व॒र्गाय॑ यजति प्र॒जाः सौ॒म्येन॑ गृह्णी॒यात्प्र॒त्यञ्चं॑ गृह्णी॒यात्प्र॒जां प॒शून्त्रिच॑त्वारिꣳशत्॥४३॥ सु॒व॒र्गाय॒ वज्र॑स्य रू॒पꣳ समृ॑द्ध्यै॥}]}%%६-६

\centerline{॥हरिः॑ ॐ॥}

\centerline{॥कृष्ण-यजुर्वेदीय-तैत्तिरीय-संहितायां षष्ठकाण्डे षष्ठः प्रश्नः समाप्तः॥६-६॥}
%%% END PRASHNA
