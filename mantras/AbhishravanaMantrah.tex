% !TeX program = XeLaTeX
% !TeX root = ../vedamantrabook.tex
\chapt{अभिश्रवण-मन्त्राः}

\sect{पुरुषसूक्तम्}
\ta{३}{१२--१३}{}

स॒हस्र॑शीर्‌षा॒ पुरु॑षः। 
स॒ह॒स्रा॒क्षः स॒हस्र॑पात्। 
स भूमिं॑ वि॒श्वतो॑ वृ॒त्वा। 
अत्य॑तिष्ठद्दशाङ्गु॒लम्॥ 
% 
पुरु॑ष ए॒वेदꣳ सर्वम्᳚। 
यद्भू॒तं यच्च॒ भव्यम्᳚। 
उ॒तामृ॑त॒त्वस्येशा॑नः। 
यदन्ने॑नाति॒रोह॑ति॥ 
% 
ए॒तावा॑नस्य महि॒मा। 
अतो॒ ज्यायाꣴ॑श्च॒ पूरु॑षः। 
पादो᳚ऽस्य॒ विश्वा॑ भू॒तानि॑। 
त्रि॒पाद॑स्या॒मृतं॑ दि॒वि॥ 
% 
त्रि॒पादू॒र्ध्व उदै॒त्पुरु॑षः। 
पादो᳚ऽस्ये॒हाऽऽभ॑वा॒त्पुनः॑। 
ततो॒ विश्व॒ङ्व्य॑क्रामत्। 
सा॒श॒ना॒न॒श॒ने अ॒भि॥ 
% 
तस्मा᳚द्वि॒राड॑जायत। 
वि॒राजो॒ अधि॒ पूरु॑षः। 
स जा॒तो अत्य॑रिच्यत। 
प॒श्चाद्भूमि॒मथो॑ पु॒रः॥ 
% 
 यत्पुरु॑षेण ह॒विषा᳚। 
दे॒वा य॒ज्ञमत॑न्वत। 
व॒स॒न्तो अ॑स्याऽऽसी॒दाज्यम्᳚। 
ग्री॒ष्म इ॒ध्मः श॒रद्ध॒विः॥ 
% 
 स॒प्तास्या॑ऽऽसन्  परि॒धयः॑। 
त्रिः स॒प्त स॒मिधः॑ कृ॒ताः। 
दे॒वा यद्य॒ज्ञं त॑न्वा॒नाः। 
अब॑ध्न॒न् पुरु॑षं प॒शुम्॥ 
% 
 तं य॒ज्ञं ब॒र्हिषि॒ प्रौक्षन्। 
पुरु॑षं जा॒तम॑ग्र॒तः। 
तेन॑ दे॒वा अय॑जन्त। 
सा॒ध्या ऋष॑यश्च॒ ये॥ 
% 
तस्मा᳚द्य॒ज्ञाथ्स॑र्व॒हुतः॑। 
सम्भृ॑तं पृषदा॒ज्यम्। 
प॒शूꣴस्ताꣴश्च॑क्रे वाय॒व्यान्। 
आ॒र॒ण्यान्ग्रा॒म्याश्च॒ ये॥ 
% 
 तस्मा᳚द्य॒ज्ञाथ्स॑र्व॒हुतः॑। 
ऋचः॒ सामा॑नि जज्ञिरे। 
छन्दाꣳ॑सि जज्ञिरे॒ तस्मा᳚त्। 
यजु॒स्तस्मा॑दजायत॥ 
% 
तस्मा॒दश्वा॑ अजायन्त। 
ये के चो॑भ॒याद॑तः। 
गावो॑ ह जज्ञिरे॒ तस्मा᳚त्। 
तस्मा᳚ज्जा॒ता अ॑जा॒वयः॑॥ 
% 
यत्पुरु॑षं॒ व्य॑दधुः। 
क॒ति॒धा व्य॑कल्पयन्। 
मुखं॒ किम॑स्य॒ कौ बा॒हू। 
कावू॒रू पादा॑वुच्येते॥ 
% 
ब्रा॒ह्म॒णो᳚ऽस्य॒ मुख॑मासीत्। 
बा॒हू रा॑ज॒न्यः॑ कृ॒तः। 
ऊ॒रू तद॑स्य॒ यद्वैश्यः॑। 
प॒द्भ्याꣳ शू॒द्रो अ॑जायत॥ 
% 
च॒न्द्रमा॒ मन॑सो जा॒तः। 
चक्षोः॒ सूर्यो॑ अजायत। 
मुखा॒दिन्द्र॑श्चा॒ग्निश्च॑। 
प्रा॒णाद्वा॒युर॑जायत॥ 
% 
नाभ्या॑ आसीद॒न्तरि॑क्षम्। 
शी॒र्ष्णो द्यौः सम॑वर्तत। 
प॒द्भ्यां भूमि॒र्दिशः॒ श्रोत्रा᳚त्। 
तथा॑ लो॒काꣳ अ॑कल्पयन्॥ 
% 
वेदा॒हमे॒तं पुरु॑षं म॒हान्तम्᳚। 
आ॒दि॒त्यव॑र्णं॒ तम॑स॒स्तु पा॒रे॥ 
% 
सर्वा॑णि रू॒पाणि॑ वि॒चित्य॒ धीरः॑। 
नामा॑नि कृ॒त्वाऽभि॒वद॒न्॒ यदास्ते᳚॥ 
% 
धा॒ता पु॒रस्ता॒द्यमु॑दाज॒हार॑। 
श॒क्रः प्रवि॒द्वान्  प्र॒दिश॒श्चत॑स्रः। 
तमे॒वं वि॒द्वान॒मृत॑ इ॒ह भ॑वति। 
नान्यः पन्था॒ अय॑नाय विद्यते॥ 
% 
य॒ज्ञेन॑ य॒ज्ञम॑यजन्त दे॒वाः। 
तानि॒ धर्मा॑णि प्रथ॒मान्या॑सन्। 
ते ह॒ नाकं॑ महि॒मानः॑ सचन्ते। 
यत्र॒ पूर्वे॑ सा॒ध्याः सन्ति॑ दे॒वाः॥ 
% 
अ॒द्भ्यः सम्भू॑तः पृथि॒व्यै रसा᳚च्च। 
वि॒श्वक॑र्मणः॒ सम॑वर्त॒ताधि॑। 
तस्य॒ त्वष्टा॑ वि॒दध॑द्रू॒पमे॑ति। 
तत्पुरु॑षस्य॒ विश्व॒माजा॑न॒मग्रे᳚॥ 
% 
वेदा॒हमे॒तं पुरु॑षं म॒हान्तम्᳚। 
आ॒दि॒त्यव॑र्णं॒ तम॑सः॒ पर॑स्तात्। 
तमे॒वं वि॒द्वान॒मृत॑ इ॒ह भ॑वति। 
नान्यः पन्था॑ विद्य॒तेऽय॑नाय॥ 
% 
प्र॒जाप॑तिश्चरति॒ गर्भे॑ अ॒न्तः। 
अ॒जाय॑मानो बहु॒धा विजा॑यते। 
तस्य॒ धीराः॒ परि॑जानन्ति॒ योनिम्᳚। 
मरी॑चीनां प॒दमि॑च्छन्ति वे॒धसः॑॥ 
% 
यो दे॒वेभ्य॒ आत॑पति। 
यो दे॒वानां᳚ पु॒रोहि॑तः। 
पूर्वो॒ यो दे॒वेभ्यो॑ जा॒तः। 
नमो॑ रु॒चाय॒ ब्राह्म॑ये॥ 
% 
रुचं॑ ब्रा॒ह्मं ज॒नय॑न्तः। 
दे॒वा अग्रे॒ तद॑ब्रुवन्। 
यस्त्वै॒वं ब्रा᳚ह्म॒णो वि॒द्यात्। 
तस्य॑ दे॒वा अस॒न् वशे᳚॥ 
% 
ह्रीश्च॑ ते ल॒क्ष्मीश्च॒ पत्न्यौ᳚। 
अ॒हो॒रा॒त्रे पा॒र्श्वे। 
नक्ष॑त्राणि रू॒पम्। 
अ॒श्विनौ॒ व्यात्तम्᳚। 
इ॒ष्टं म॑निषाण। 
अ॒मुं म॑निषाण। 
सर्वं॑ मनिषाण॥ 
% 
\centerline{॥ॐ शान्तिः॒ शान्तिः॒ शान्तिः॑॥}

\chapt{नारायणसूक्तम्}
\ta{१०}{१३}{}

स॒ह॒स्र॒शीर्षं दे॒वं॒ वि॒श्वाक्षं॑ वि॒श्वश॑म्भुवम्। विश्वं॑ ना॒राय॑णं दे॒व॒म॒क्षरं॑ पर॒मं प॒दम्। 
वि॒श्वतः॒ पर॑मान्नि॒त्यं॒ वि॒श्वं ना॑राय॒णꣳ ह॑रिम्। विश्व॑मे॒वेदं पुरु॑ष॒स्तद्विश्व॒मुप॑जीवति। 
पतिं॒   विश्व॑स्या॒ऽ॒ऽ॒त्मेश्व॑र॒ꣳ॒ शाश्व॑तꣳ शि॒वम॑च्युतम्। ना॒राय॒णं म॑हाज्ञे॒यं॒ वि॒श्वात्मा॑नं प॒राय॑णम्। ना॒राय॒णप॑रो ज्यो॒ति॒रा॒त्मा ना॑राय॒णः प॑रः। ना॒राय॒ण प॑रं ब्र॒ह्म॒ त॒त्त्वं ना॑राय॒णः प॑रः। ना॒राय॒णप॑रो ध्या॒ता॒ ध्या॒नं ना॑राय॒णः प॑रः। यच्च॑ कि॒ञ्चिज्ज॑गथ्स॒र्वं॒ दृ॒श्यते᳚ श्रूय॒तेऽपि॑ वा॥ 

अन्त॑र्ब॒हिश्च॑ तथ्स॒र्वं॒ व्या॒प्य ना॑राय॒णः स्थि॑तः। अन॑न्त॒मव्य॑यं क॒विꣳ स॑मु॒द्रेऽन्तं॑  वि॒श्वश॑म्भुवम्। प॒द्म॒को॒श प्र॑तीका॒श॒ꣳ॒ हृ॒दयं॑ चाप्य॒धोमु॑खम्। अधो॑ नि॒ष्ट्या वि॑तस्त्या॒न्ते॒ ना॒भ्यामु॑परि॒ तिष्ठ॑ति। ज्वा॒ल॒मा॒लाकु॑लं भा॒ती॒ वि॒श्वस्या॑ऽऽयत॒नं म॑हत्। सन्त॑तꣳ शि॒लाभि॑स्तु॒\-लम्ब॑त्याकोश॒सन्नि॑भम्। तस्यान्ते॑ सुषि॒रꣳ सू॒क्ष्मं तस्मि᳚न्थ्स॒र्वं प्रति॑ष्ठितम्। तस्य॒ मध्ये॑ म॒हान॑\-ग्निर्वि॒श्वार्चि॑र्वि॒श्वतो॑मुखः। सोऽग्र॑भु॒ग्विभ॑जन्ति॒ष्ठ॒न्नाहा॑रमज॒रः क॒विः। ति॒र्य॒गू॒र्ध्वम॑धः शा॒यी॒ र॒श्मय॑स्तस्य॒ सन्त॑ता। स॒न्ता॒पय॑ति स्वं दे॒हमापा॑दतल॒मस्त॑कः। तस्य॒ मध्ये॒ वह्नि॑शिखा अ॒णीयो᳚र्ध्वा व्य॒वस्थि॑तः। नी॒लतो॑यद॑\-मध्य॒स्था॒द्वि॒द्युल्ले॑खेव॒  भास्व॑रा। नी॒वार॒शूक॑वत्त॒न्वी॒ पी॒ता भा᳚स्वत्य॒णूप॑मा। तस्याः᳚ शिखा॒या म॑ध्ये प॒रमा᳚त्मा व्य॒वस्थि॑तः। स ब्रह्म॒ स शिवः॒ स हरिः॒ सेन्द्रः॒ सोऽक्ष॑रः पर॒मः स्व॒राट्॥ 
ऋ॒तꣳ स॒त्यं प॑रं ब्र॒ह्म॒ पु॒रुषं॑ कृष्ण॒पिङ्ग॑लम्। ऊ॒र्ध्वरे॑तं वि॑रूपा॒क्षं॒ वि॒श्वरू॑पाय॒ वै नमो॒ नमः॑। 

ना॒रा॒य॒णाय॑ वि॒द्महे॑ वासुदे॒वाय॑ धीमहि। तन्नो॑ विष्णुः प्रचो॒दया᳚त्। 

विष्णो॒र्नु कं॑ वी॒र्या॑णि॒ प्रवो॑चं॒ यः पार्थि॑वानि विम॒मे रजाꣳ॑सि॒ यो अस्क॑भाय॒दुत्त॑रꣳ स॒धस्थं॑ विचक्रमा॒णस्त्रे॒धोरु॑गा॒यो विष्णो॑र॒राट॑मसि॒ विष्णोः᳚ पृ॒ष्ठम॑सि॒ विष्णोः॒ श्नप्त्रे᳚स्थो॒ विष्णोः॒ स्यूर॑सि॒ विष्णो᳚र्ध्रु॒वम॑सि वैष्ण॒वम॑सि॒ विष्ण॑वे त्वा॥ 

\centerline{॥ॐ शान्तिः॒ शान्तिः॒ शान्तिः॑॥}

\sect{कृणुष्व पाजः}
\ts{१}{२}{१४}{२८--३४}

कृ॒णु॒ष्व पाजः॒ प्रसि॑तिं॒ न पृ॒थ्वीं या॒हि राजे॒वाम॑वा॒ꣳ॒ इभे॑न। तृ॒ष्वीमनु॒ प्रसि॑तिं द्रूणा॒नोऽस्ता॑सि॒ विध्य॑ र॒क्षस॒स्तपि॑ष्ठैः। तव॑ भ्र॒मास॑ आशु॒या प॑त॒न्त्यनु॑ स्पृश धृष॒ता शोशु॑चानः। तपूꣴ॑ष्यग्ने जु॒ह्वा॑ पत॒ङ्गानस॑न्दितो॒ वि सृ॑ज॒ विष्व॑गु॒ल्काः। प्रति॒ स्पशो॒ वि सृ॑ज॒ तूर्णि॑तमो॒ भवा॑ पा॒युर्वि॒शो अ॒स्या अद॑ब्धः। यो नो॑ दू॒रे अ॒घशꣳ॑सो॒  यो अन्त्यग्ने॒ माकि॑ष्टे॒ व्यथि॒रा द॑धर्षीत्॥१॥

उद॑ग्ने तिष्ठ॒ प्रत्याऽऽत॑नुष्व॒ न्य॑मित्राꣳ॑ ओषतात्तिग्महेते। यो नो॒ अरा॑तिꣳ समिधान च॒क्रे नी॒चा तं ध॑क्ष्यत॒सं न शुष्कम्᳚। ऊ॒र्ध्वो भ॑व॒ प्रति॑ वि॒ध्याध्य॒स्मदा॒विष्कृ॑णुष्व॒ दैव्या᳚न्यग्ने। अव॑ स्थि॒रा त॑नुहि यातु॒जूनां᳚ जा॒मिमजा॑मिं॒ प्र मृ॑णीहि॒ शत्रून्॑। स ते॑ जानाति सुम॒तिं य॑विष्ठ॒ य ईव॑ते॒ ब्रह्म॑णे गा॒तुमैर॑त्॥२॥

 विश्वा᳚न्यस्मै सु॒दिना॑नि रा॒यो द्यु॒म्नान्य॒र्यो वि दुरो॑ अ॒भि द्यौ᳚त्। सेद॑ग्ने अस्तु सु॒भगः॑ सु॒दानु॒र्यस्त्वा॒ नित्ये॑न ह॒विषा॒ य उ॒क्थैः। पिप्री॑षति॒ स्व आयु॑षि दुरो॒णे विश्वेद॑स्मै सु॒दिना॒ साऽस॑दि॒ष्टिः। अर्चा॑मि ते सुम॒तिं घोष्य॒र्वाख्सं ते॑ वा॒वाता॑ जरतामि॒यङ्गीः॥३॥
 
स्वश्वा᳚स्त्वा सु॒रथा॑ मर्जयेमा॒स्मे क्ष॒त्राणि॑ धारये॒रनु॒ द्यून्। इ॒ह त्वा॒ भूर्या च॑रे॒दुप॒ त्मन्दोषा॑\-वस्तर्दीदि॒वाꣳ\-स॒मनु॒ द्यून्। कीड॑न्तस्त्वा सु॒मन॑सः सपेमा॒भि द्यु॒म्ना त॑स्थि॒वाꣳसो॒ जना॑नाम्। यस्त्वा॒ स्वश्वः॑ सुहिर॒ण्यो अ॑ग्न उप॒याति॒ वसु॑मता॒ रथे॑न। तस्य॑ त्रा॒ता भ॑वसि॒ तस्य॒ सखा॒ यस्त॑ आति॒थ्यमा॑नु॒षग्जुजो॑षत्। म॒हो रु॑जामि ब॒न्धुता॒ वचो॑भि॒स्तन्मा॑ पि॒तुर्गोत॑मा॒दन्वि॑याय॥४॥

 त्वं नो॑ अ॒स्य वच॑सश्चिकिद्धि॒ होत॑र्यविष्ठ सुक्रतो॒ दमू॑नाः। अस्व॑प्नजस्त॒रण॑यः सु॒शेवा॒ अत॑न्द्रासोऽवृ॒का अश्र॑मिष्ठाः। ते पा॒यवः॑ स॒ध्रिय॑ञ्चो नि॒षद्याऽग्ने॒ तव॑ नः पान्त्वमूर। ये पा॒यवो॑ मामते॒यं ते॑ अग्ने॒ पश्य॑न्तो अ॒न्धं दु॑रि॒तादर॑क्षन्। र॒रक्ष॒ तान्थ्सु॒कृतो॑ वि॒श्ववे॑दा॒ दिफ्स॑न्त॒ इद्रि॒पवो॒ ना ह॑ देभुः॥५॥
 
त्वया॑ व॒यꣳ स॑ध॒न्य॑स्त्वोता॒स्तव॒ प्रणी᳚त्यश्याम॒ वाजान्। उ॒भा शꣳसा॑ सूदय सत्यतातेऽनुष्ठु॒या कृ॑णुह्यह्रयाण। अ॒या ते॑ अग्ने स॒मिधा॑ विधेम॒ प्रति॒ स्तोमꣳ॑ श॒स्यमा॑नं गृभाय। दहा॒ऽ॒शसो॑ र॒क्षसः॑ पा॒ह्य॑स्मान्द्रु॒हो नि॒दो मि॑त्रमहो अव॒द्यात्। र॒क्षो॒हणं॑ वा॒जिन॒माऽऽजि॑घर्मि मि॒त्रं प्रथि॑ष्ठ॒मुप॑ यामि॒ शर्म॑। शिशा॑नो अ॒ग्निः क्रतु॑भिः॒ समि॑द्धः॒ स नो॒ दिवा॒ स रि॒षः पा॑तु॒ नक्तम्᳚॥६॥

 वि ज्योति॑षा बृह॒ता भा᳚त्य॒ग्निरा॒विर्विश्वा॑नि कृणुते महि॒त्वा। प्रादे॑वीर्मा॒याः स॑हते दु॒रेवाः॒ शिशी॑ते॒ शृङ्गे॒ रक्ष॑से वि॒निक्षे᳚। उ॒त स्वा॒नासो॑ दि॒विष॑न्त्व॒ग्नेस्ति॒ग्मायु॑धा॒ रक्ष॑से॒ हन्त॒वा उ॑। मदे॑ चिदस्य॒ प्ररु॑जन्ति॒ भामा॒ न व॑रन्ते परि॒बाधो॒ अदे॑वीः॥७॥

\sect{रक्षोहणो वलगहनः}
\centerline{\scriptsize (तैत्तिरीयसंहिता १.३.२)}

र॒क्षो॒हणो॑ वलग॒हनो॑ वैष्ण॒वान्ख॑नामी॒दम॒हं तं व॑ल॒गमुद्व॑पामि॒ यं नः॑ समा॒नो यमस॑मानो निच॒खाने॒दमे॑न॒मध॑रं करोमि॒ यो नः॑ समा॒नो यो\-ऽस॑मानो\-ऽराती॒यति॑ गाय॒त्रेण॒ छन्द॒सा\-ऽव॑बाढो वल॒गः किमत्र॑ भ॒द्रं तन्नौ॑ स॒ह वि॒राड॑सि सपत्न॒हा स॒म्राड॑सि भ्रातृव्य॒हा स्व॒राड॑स्यभिमाति॒हा वि॑श्वा॒राड॑सि॒ विश्वा॑सां ना॒ष्ट्राणाꣳ॑ ह॒न्ता~(३)

%1.3.2.2
र॑क्षो॒हणो॑ वलग॒हनः॒ प्रोक्षा॑मि वैष्ण॒वान् र॑क्षो॒हणो॑ वलग॒हनो\-ऽव॑ नयामि वैष्ण॒वान् यवो॑\-ऽसि य॒वया॒स्मद्द्वेषो॑ य॒वयारा॑ती रक्षो॒हणो॑ वलग॒हनो\-ऽव॑ स्तृणामि वैष्ण॒वान् र॑क्षो॒हणो॑ वलग॒हनो॒\-ऽभि जु॑होमि वैष्ण॒वान् र॑क्षो॒हणौ॑ वलग॒हना॒वुप॑ दधामि वैष्ण॒वी र॑क्षो॒हणौ॑ वलग॒हनौ॒ पर्यू॑हामि वैष्ण॒वी र॑क्षो॒हणौ॑ वलग॒हनौ॒ परि॑ स्तृणामि वैष्ण॒वी र॑क्षो॒हणौ॑ वलग॒हनौ॑ वैष्ण॒वी बृ॒हन्न॑सि बृ॒हद्ग्रा॑वा बृह॒तीमिन्द्रा॑य॒ वाचं॑ वद॥~(४)

\sect{सोमाय पितृमते}
\centerline{\scriptsize (तैत्तिरीयसंहिता १.८.५)}

सोमा॑य पितृ॒मते॑ पुरो॒डाश॒ꣳ॒ षट्\-क॑पालं॒ निर्व॑पति पि॒तृभ्यो॑ बर्\mbox{}हि॒षद्भ्यो॑ धा॒नाः पि॒तृभ्यो᳚\-ऽग्निष्वा॒त्तेभ्यो॑\-ऽभिवा॒न्या॑यै दु॒ग्धे म॒न्थमे॒तत् ते॑ तत॒ ये च॒ त्वामन्वे॒तत् ते॑ पितामह प्रपितामह॒ ये च॒ त्वामन्वत्र॑ पितरो यथाभा॒गं म॑न्दध्वꣳ सुस॒न्दृशं॑ त्वा व॒यं मघ॑वन् मन्दिषी॒महि॑॥ प्र नू॒नं पू॒र्णव॑न्धुरः स्तु॒तो या॑सि॒ वशा॒ꣳ॒ अनु॑॥ योजा॒ न्वि॑न्द्र ते॒ हरी᳚॥~(७)

%1.8.5.2
अक्ष॒न्नमी॑मदन्त॒ ह्यव॑ प्रि॒या अ॑धूषत॥ अस्तो॑षत॒ स्वभा॑नवो॒ विप्रा॒ नवि॑ष्ठया म॒ती॥ योजा॒ न्वि॑न्द्र ते॒ हरी᳚॥ अक्ष॑न् पि॒तरो\-ऽमी॑मदन्त पि॒तरो\-ऽती॑तृपन्त पि॒तरो\-ऽमी॑मृजन्त पि॒तरः॑॥ परे॑त पितरः सोम्या गम्भी॒रैः प॒थिभिः॑ पू॒र्व्यैः॥ अथा॑ पि॒तॄन्थ्सु॑वि॒दत्रा॒ꣳ॒ अपी॑त य॒मेन॒ ये स॑ध॒मादं॒ मद॑न्ति॥ मनो॒ न्वा हु॑वामहे नाराश॒ꣳ॒सेन॒ स्तोमे॑न पितृ॒णां च॒ मन्म॑भिः॥ आ~(८)

%1.8.5.3
न॑ एतु॒ मनः॒ पुनः॒ क्रत्वे॒ दक्षा॑य जी॒वसे᳚॥ ज्योक् च॒ सूर्यं॑ दृ॒शे॥ पुन॑र्नः पि॒तरो॒ मनो॒ ददा॑तु॒ दैव्यो॒ जनः॑॥ जी॒वं व्रातꣳ॑ सचेमहि॥ यद॒न्तरि॑क्षं पृथि॒वीमु॒त द्यां यन्मा॒तरं॑ पि॒तरं॑ वा जिहिꣳसि॒म॥ अ॒ग्निर्मा॒ तस्मा॒देन॑सो॒ गार्\mbox{}ह॑पत्यः॒ प्र मु॑ञ्चतु दुरि॒ता यानि॑ चकृ॒म क॒रोतु॒ माम॑ने॒नसम्᳚॥~(९)

\sect{उशन्तस्त्वा हवामहे}
\centerline{\scriptsize (तैत्तिरीयसंहिता २.६.१२)}

उ॒शन्त॑स्त्वा हवामह उ॒शन्तः॒ समि॑धीमहि। उ॒शन्नु॑श॒त आ व॑ह॒ पि॒तॄन् ह॒विषे॒ अत्त॑वे। त्वꣳ सो॑म॒ प्रचि॑कितो मनी॒षा त्वꣳ रजि॑ष्ठ॒मनु॑ नेषि॒ पन्था᳚म्। तव॒ प्रणी॑ती पि॒तरो॑ न इन्दो दे॒वेषु॒ रत्न॑मभजन्त॒ धीराः᳚। त्वया॒ हि नः॑ पि॒तरः॑ सोम॒ पूर्वे॒ कर्मा॑णि च॒क्रुः प॑वमान॒ धीराः᳚। व॒न्वन्नवा॑तः परि॒धीꣳ रपो᳚र्णु वी॒रेभि॒रश्वै᳚र्म॒घवा॑ भव~(६५)

%2.6.12.2
नः॒। त्वꣳ सो॑म पि॒तृभिः॑ संविदा॒नो\-ऽनु॒ द्यावा॑पृथि॒वी आ त॑तन्थ। तस्मै॑ त इन्दो ह॒विषा॑ विधेम व॒यꣴ स्या॑म॒ पत॑यो रयी॒णाम्। अग्नि॑ष्वात्ताः पितर॒ एह ग॑च्छत॒ सदः॑सदः सदत सुप्रणीतयः। अ॒त्ता ह॒वीꣳषि॒ प्रय॑तानि ब॒र्\mbox{}हिष्यथा॑ र॒यिꣳ सर्व॑वीरं दधातन। बर्\mbox{}हि॑षदः पितर ऊ॒त्य॑र्वागि॒मा वो॑ ह॒व्या च॑कृमा जु॒षध्वम्᳚। त आ ग॒ताव॑सा॒ शन्त॑मे॒नाथा॒स्मभ्यम्᳚~(६६)

%2.6.12.3
शं योर॑र॒पो द॑धात। आहं पि॒त़ॄन्थ्सु॑वि॒दत्राꣳ॑ अविथ्सि॒ नपा॑तञ्च वि॒क्रम॑णं च॒ विष्णोः᳚। ब॒र्\mbox{}हि॒षदो॒ ये स्व॒धया॑ सु॒तस्य॒ भज॑न्त पि॒त्वस्त इ॒हाग॑मिष्ठाः। उप॑हूताः पि॒तरो॑ बर्\mbox{}हि॒ष्ये॑षु नि॒धिषु॑ प्रि॒येषु॑। त आग॑मन्तु॒ त इ॒ह श्रु॑व॒न्त्वधि॑ ब्रुवन्तु॒ ते अ॑वन्त्व॒स्मान्। उदी॑रता॒मव॑र॒ उत्परा॑स॒ उन्म॑ध्य॒माः पि॒तरः॑ सो॒म्यासः॑। असुम्᳚~(६७)

%2.6.12.4
य ई॒युर॑वृ॒का ऋ॑त॒ज्ञास्ते नो॑\-ऽवन्तु पि॒तरो॒ हवे॑षु। इ॒दम्पि॒तृभ्यो॒ नमो॑ अस्त्व॒द्य ये पूर्वा॑सो॒ य उप॑रास ई॒युः। ये पार्थि॑वे॒ रज॒स्या निष॑त्ता॒ ये वा॑ नू॒नꣳ सु॑वृ॒जना॑सु वि॒क्षु। अधा॒ यथा॑ नः पि॒तरः॒ परा॑सः प्र॒त्नासो॑ अग्न ऋ॒तमा॑शुषा॒णाः। शुचीद॑य॒न्दीधि॑तिमुक्थ॒शासः॒ क्षामा॑ भि॒न्दन्तो॑ अरु॒णीरप॑ व्रन्न्। यद॑ग्ने~(६८)

%2.6.12.5
क॒व्य॒वा॒ह॒न॒ पि॒तॄन् यक्ष्यृ॑ता॒वृधः॑। प्र च॑ ह॒व्यानि॑ वक्ष्यसि दे॒वेभ्य॑श्च पि॒तृभ्य॒ आ। त्वम॑ग्न ईडि॒तो जा॑तवे॒दो\-ऽवा᳚ड्ढ॒व्यानि॑ सुर॒भीणि॑ कृ॒त्वा। प्रादाः᳚ पि॒तृभ्यः॑ स्व॒धया॒ ते अ॑क्षन्न॒द्धि त्वं दे॑व॒ प्रय॑ता ह॒वीꣳषि॑। मात॑ली क॒व्यैर्य॒मो अङ्गि॑रोभि॒र्बृह॒स्पति॒र्\mbox{}ऋक्व॑भिर्वावृधा॒नः। याꣴश्च॑ दे॒वा वा॑वृ॒धुर्ये च॑ दे॒वान्थ्स्वाहा॒न्ये स्व॒धया॒न्ये म॑दन्ति।~(६९)

%2.6.12.6
इ॒मं य॑म प्रस्त॒रमा हि सीदाङ्गि॑रोभिः पि॒तृभिः॑ संविदा॒नः। आ त्वा॒ मन्त्राः᳚ कविश॒स्ता व॑हन्त्वे॒ना रा॑जन् ह॒विषा॑ मादयस्व। अङ्गि॑रोभि॒रा ग॑हि य॒ज्ञिये॑भि॒र्यम॑ वैरू॒पैरि॒ह मा॑दयस्व। विव॑स्वन्तꣳ हुवे॒ यः पि॒ता ते॒\-ऽस्मिन् य॒ज्ञे ब॒र्\mbox{}हिष्या नि॒षद्य॑। अङ्गि॑रसो नः पि॒तरो॒ नव॑ग्वा॒ अथ॑र्वाणो॒ भृग॑वः सो॒म्यासः॑। तेषां᳚ व॒यꣳ सु॑म॒तौ य॒ज्ञिया॑ना॒मपि॑ भ॒द्रे सौ॑मन॒से स्या॑म॥~(७०)

\sect{भक्षेहि मा विश}
\centerline{\scriptsize (तैत्तिरीयसंहिता ३.२.५)}

%3.2.5.1
भक्षेहि॒ मा वि॑श दीर्घायु॒त्वाय॑ शन्तनु॒त्वाय॑ रा॒यस्पोषा॑य॒ वर्च॑से सुप्रजा॒स्त्वायेहि॑ वसो पुरोवसो प्रि॒यो मे॑ हृ॒दो᳚\-ऽस्य॒श्विनो᳚स्त्वा बा॒हुभ्याꣳ॑ सघ्यासम् नृ॒चक्ष॑सं त्वा देव सोम सु॒चक्षा॒ अव॑ ख्येषम् म॒न्द्राभिभू॑तिः के॒तुर्य॒ज्ञानां॒ वाग्जु॑षा॒णा सोम॑स्य तृप्यतु म॒न्द्रा स्व॑र्वा॒च्यदि॑ति॒रना॑हतशीर्ष्णी॒ वाग्जु॑षा॒णा सोम॑स्य तृप्य॒त्वेहि॑ विश्वचर्\mbox{}षणे~(१६)

%3.2.5.2
श॒म्भूर्म॑यो॒भूः स्व॒स्ति मा॑ हरिवर्ण॒ प्र च॑र॒ क्रत्वे॒ दक्षा॑य रा॒यस्पोषा॑य सुवी॒रता॑यै॒ मा मा॑ राज॒न्वि बी॑भिषो॒ मा मे॒ हार्दि॑ त्वि॒षा व॑धीः। वृष॑णे॒ शुष्मा॒यायु॑षे॒ वर्च॑से॥ वसु॑मद्गणस्य सोम देव ते मति॒विदः॑ प्रातःसव॒नस्य॑ गाय॒त्रछ॑न्दस॒ इन्द्र॑पीतस्य॒ नरा॒शꣳस॑पीतस्य पि॒तृपी॑तस्य॒ मधु॑मत॒ उप॑हूत॒स्योप॑हूतो भक्षयामि रु॒द्रव॑द्गणस्य सोम देव ते मति॒विदो॒ माध्यं॑दिनस्य॒ सव॑नस्य त्रि॒ष्टुप्छ॑न्दस॒ इन्द्र॑पीतस्य॒ नरा॒शꣳस॑पीतस्य~(१७)

%3.2.5.3
पि॒तृपी॑तस्य॒ मधु॑मत॒ उप॑हूत॒स्योप॑हूतो भक्षयाम्यादि॒त्यव॑द्गणस्य सोम देव ते मति॒विद॑स्तृ॒तीय॑स्य॒ सव॑नस्य॒ जग॑तीछन्दस॒ इन्द्र॑पीतस्य॒ नरा॒शꣳस॑पीतस्य पि॒तृपी॑तस्य॒ मधु॑मत॒ उप॑हूत॒स्योप॑हूतो भक्षयामि। आ प्या॑यस्व॒ समे॑तु ते वि॒श्वतः॑ सोम॒ वृष्णि॑यम्। भवा॒ वाज॑स्य सङ्ग॒थे। हिन्व॑ मे॒ गात्रा॑ हरिवो ग॒णान्मे॒ मा वि ती॑तृषः। शि॒वो मे॑ सप्त॒र्\mbox{}षीनुप॑ तिष्ठस्व॒ मा मे\-ऽवा॒ङ्नाभि॒मति॑~(१८)

%3.2.5.4
गाः॒। अपा॑म॒ सोम॑म॒मृता॑ अभू॒माद॑र्श्म॒ ज्योति॒रवि॑दाम दे॒वान्। किम॒स्मान्कृ॑णव॒दरा॑तिः॒ किमु॑ धू॒र्तिर॑मृत॒ मर्त्य॑स्य। यन्म॑ आ॒त्मनो॑ मि॒न्दाभू॑द॒ग्निस्तत्पुन॒राहा᳚र्जा॒तवे॑दा॒ विच॑र्\mbox{}षणिः। पुन॑र॒ग्निश्चक्षु॑रदा॒त्पुन॒रिन्द्रो॒ बृह॒स्पतिः॑। पुन॑र्मे अश्विना यु॒वं चक्षु॒रा ध॑त्तम॒क्ष्योः। इ॒ष्टय॑जुषस्ते देव सोम स्तु॒तस्तो॑मस्य~(१९)

%3.2.5.5
श॒स्तोक्थ॑स्य॒ हरि॑वत॒ इन्द्र॑पीतस्य॒ मधु॑मत॒ उप॑हूत॒स्योप॑हूतो भक्षयामि। आ॒पूर्याः॒ स्था मा॑ पूरयत प्र॒जया॑ च॒ धने॑न च। ए॒तत्ते॑ तत॒ ये च॒ त्वामन्वे॒तत्ते॑ पितामह प्रपितामह॒ ये च॒ त्वामन्वत्र॑ पितरो यथाभा॒गम्म॑न्दध्व॒म् नमो॑ वः पितरो॒ रसा॑य॒ नमो॑ वः पितरः॒ शुष्मा॑य॒ नमो॑ वः पितरो जी॒वाय॒ नमो॑ वः पितरः~(२०)

%3.2.5.6
स्व॒धायै॒ नमो॑ वः पितरो म॒न्यवे॒ नमो॑ वः पितरो घो॒राय॒ पित॑रो॒ नमो॑ वो॒ य ए॒तस्मिँ॑ल्लो॒के स्थ यु॒ष्माꣴस्ते\-ऽनु॒ ये᳚\-ऽस्मिँल्लो॒के मां ते\-ऽनु॒ य ए॒तस्मिँ॑ल्लो॒के स्थ यू॒यं तेषां॒ वसि॑ष्ठा भूयास्त॒ ये᳚\-ऽस्मिँल्लो॒के॑\-ऽहं तेषां॒ वसि॑ष्ठो भूयासं॒ प्रजा॑पते॒ न त्वदे॒तान्य॒न्यो विश्वा॑ जा॒तानि॒ परि॒ ता ब॑भूव~(२१)

%3.2.5.7
यत्का॑मास्ते जुहु॒मस्तन्नो॑ अस्तु व॒यꣴ स्या॑म॒ पत॑यो रयी॒णाम्। दे॒वकृ॑त॒स्यैन॑सो\-ऽव॒यज॑नमसि मनु॒ष्य॑कृत॒स्यैन॑सो\-ऽ\-व॒यज॑नमसि पि॒तृकृ॑त॒स्यैन॑सो\-ऽव॒यज॑नमस्य॒फ्सु धौ॒तस्य॑ सोम देव ते॒ नृभिः॑ सु॒तस्ये॒ष्टय॑जुषः स्तु॒तस्तो॑मस्य श॒स्तोक्थ॑स्य॒ यो भ॒क्षो अ॑श्व॒सनि॒र्यो गो॒सनि॒स्तस्य॑ ते पि॒तृभि॑र्भ॒क्षं कृ॑त॒स्योप॑हूत॒स्योप॑हूतो भक्षयामि॥~(२२)

\sect{ध्रुवासि धरुणास्तृता}
\centerline{\scriptsize (तैत्तिरीयसंहिता ४.२.९)}


ध्रु॒वासि॑ ध॒रुणास्तृ॑ता वि॒श्वक॑र्मणा॒ सुकृ॑ता। मा त्वा॑ समु॒द्र उद्व॑धी॒न्मा सु॑प॒र्णो\-ऽव्य॑थमाना पृथि॒वीं दृꣳ॑ह। प्र॒जा\-प॑तिस्त्वा सादयतु पृथि॒व्याः पृ॒ष्ठे व्यच॑स्वती॒म्प्रथ॑स्वती॒म्प्रथो॑\-ऽसि पृथि॒व्य॑सि॒ भूर॑सि॒ भूमि॑र॒स्यदि॑तिरसि वि॒श्वधा॑या॒ विश्व॑स्य॒ भुव॑नस्य ध॒र्त्री पृ॑थि॒वीं य॑च्छ पृथि॒वीं दृꣳ॑ह पृथि॒वीं मा हिꣳ॑सी॒र्विश्व॑स्मै प्रा॒णाया॑पा॒नाय॑ व्या॒नायो॑दा॒नाय॑ प्रति॒ष्ठायै᳚~(३६)

%4.2.9.2
च॒रित्रा॑या॒ग्निस्त्वा॒भि पा॑तु म॒ह्या स्व॒स्त्या छ॒र्दिषा॒ शन्त॑मेन॒ तया॑ दे॒वत॑याङ्गिर॒स्वद्ध्रु॒वा सी॑द। काण्डा᳚त्काण्डात् प्र॒रोह॑न्ती॒ परु॑षःपरुषः॒ परि॑। ए॒वा नो॑ दूर्वे॒ प्र त॑नु स॒हस्रे॑ण श॒तेन॑ च। या श॒तेन॑ प्रत॒नोषि॑ स॒हस्रे॑ण वि॒रोह॑सि। तस्या᳚स्ते देवीष्टके वि॒धेम॑ ह॒विषा॑ व॒यम्। अषा॑ढासि॒ सह॑माना॒ सह॒स्वारा॑तीः॒ सह॑स्वारातीय॒तः सह॑स्व॒ पृत॑नाः॒ सह॑स्व पृतन्य॒तः। स॒हस्र॑वीर्या~(३७)

%4.2.9.3
अ॒सि॒ सा मा॑ जिन्व। मधु॒ वाता॑ ऋताय॒ते मधु॑ क्षरन्ति॒ सिन्ध॑वः। माध्वी᳚र्नः स॒न्त्वोष॑धीः। मधु॒ नक्त॑मु॒तोषसि॒ मधु॑म॒त्पार्थि॑व॒ꣳ॒ रजः। मधु॒ द्यौर॑स्तु नः पि॒ता। मधु॑मान्नो॒ वन॒स्पति॒र्मधु॑माꣳ अस्तु॒ सूर्यः॑। माध्वी॒र्गावो॑ भवन्तु नः। म॒ही द्यौः पृ॑थि॒वी च॑ न इ॒मं य॒ज्ञम्मि॑मिक्षताम्। पि॒पृ॒तां नो॒ भरी॑मभिः। तद्विष्णोः᳚ पर॒मम्~(३८)

%4.2.9.4
प॒दꣳ सदा॑ पश्यन्ति सू॒रयः॑। दि॒वीव॒ चक्षु॒रात॑तम्। ध्रु॒वासि॑ पृथिवि॒ सह॑स्व पृतन्य॒तः। स्यू॒ता दे॒वेभि॑र॒मृते॒नागाः᳚। यास्ते॑ अग्ने॒ सूर्ये॒ रुच॑ उद्य॒तो दिव॑मात॒न्वन्ति॑ र॒श्मिभिः॑। ताभिः॒ सर्वा॑भी रु॒चे जना॑य नस्कृधि। या वो॑ देवाः॒ सूर्ये॒ रुचो॒ गोष्वश्वे॑षु॒ या रुचः॑। इन्द्रा᳚ग्नी॒ ताभिः॒ सर्वा॑भी॒ रुचं॑ नो धत्त बृहस्पते। वि॒राट्~(३९)

%4.2.9.5
ज्योति॑रधारयथ्स॒म्राड्ज्योति॑रधारयथ्स्व॒राड्ज्योति॑रधारयत्। अग्ने॑ यु॒क्ष्वा हि ये तवाश्वा॑सो देव सा॒धवः॑। अरं॒ वह॑न्त्या॒शवः॑। यु॒क्ष्वा हि दे॑व॒हूत॑मा॒ꣳ॒ अश्वाꣳ॑ अग्ने र॒थीरि॑व। नि होता॑ पू॒र्व्यः स॑दः। द्र॒फ्सश्च॑स्कन्द पृथि॒वीमनु॒ द्यामि॒मं च॒ योनि॒मनु॒ यश्च॒ पूर्वः॑। तृ॒तीयं॒ योनि॒मनु॑ स॒ञ्चर॑न्तं द्र॒फ्सं जु॑हो॒म्यनु॑ स॒प्त~(४०)

%4.2.9.6
होत्राः᳚। अभू॑दि॒दं विश्व॑स्य॒ भुव॑नस्य॒ वाजि॑नम॒ग्नेर्वै᳚श्वान॒रस्य॑ च। अ॒ग्निर्ज्योति॑षा॒ ज्योति॑ष्मान्रु॒क्मो वर्च॑सा॒ वर्च॑स्वान्। ऋ॒चे त्वा॑ रु॒चे त्वा॒ समिथ्स्र॑वन्ति स॒रितो॒ न धेनाः᳚। अ॒न्तर्\mbox{}हृ॒दा मन॑सा पू॒यमा॑नाः। घृ॒तस्य॒ धारा॑ अ॒भि चा॑कशीमि। हि॒र॒ण्ययो॑ वेत॒सो मध्य॑ आसाम्। तस्मि᳚न्थ्सुप॒र्णो म॑धु॒कृत्कु॑ला॒यी भज॑न्नास्ते॒ मधु॑ दे॒वता᳚भ्यः। तस्या॑सते॒ हर॑यः स॒प्त तीरे᳚ स्व॒धां दुहा॑ना अ॒मृत॑स्य॒ धारा᳚म्॥~(४१)

\sect{यास्ते अग्ने}
\centerline{\scriptsize (तैत्तिरीयसंहिता ५.७.८)}

%5.7.8.1
यास्ते॑ अग्ने स॒मिधो॒ यानि॒ धाम॒ या जि॒ह्वा जा॑तवेदो॒ यो अ॒र्चिः। ये ते॑ अग्ने मे॒डयो॒ य इन्द॑व॒स्तेभि॑रा॒त्मानं॑ चिनुहि प्रजा॒नन्न्। उ॒थ्स॒न्न॒य॒ज्ञो वा ए॒ष यद॒ग्निः किं वाहै॒तस्य॑ क्रि॒यते॒ किं वा॒ न यद्वा अ॑ध्व॒र्युर॒ग्नेश्चि॒न्वन्न॑न्त॒रेत्या॒त्मनो॒ वै तद॒न्तरे॑ति॒ यास्ते॑ अग्ने स॒मिधो॒ यानि॑~(३३)

%5.7.8.2
धामेत्या॑है॒षा वा अ॒ग्नेः स्व॑यञ्चि॒तिर॒ग्निरे॒व तद॒ग्निं चि॑नोति॒ नाध्व॒र्युरा॒त्मनो॒\-ऽन्तरे॑ति॒ चत॑स्र॒ आशाः॒ प्र च॑रन्त्व॒ग्नय॑ इ॒मं नो॑ य॒ज्ञं न॑यतु प्रजा॒नन्न्। घृ॒तम्पिन्व॑न्न॒जरꣳ॑ सु॒वीरं॒ ब्रह्म॑ स॒मिद्भ॑व॒त्याहु॑तीनाम्। सु॒व॒र्गाय॒ वा ए॒ष लो॒कायोप॑ धीयते॒ यत्कू॒र्मश्चत॑स्र॒ आशाः॒ प्र च॑रन्त्व॒ग्नय॒ इत्या॑ह~(३४)

%5.7.8.3
दिश॑ ए॒वैतेन॒ प्र जा॑नाती॒मं नो॑ य॒ज्ञं न॑यतु प्रजा॒नन्नित्या॑ह सुव॒र्गस्य॑ लो॒कस्या॒भ᳚नी॑त्यै॒ ब्रह्म॑ स॒मिद्भ॑व॒त्याहु॑तीना॒मित्या॑ह॒ ब्रह्म॑णा॒ वै दे॒वाः सु॑व॒र्गं लो॒कमा॑य॒न् यद्ब्रह्म॑ण्वत्योप॒दधा॑ति॒ ब्रह्म॑णै॒व तद्यज॑मानः सुव॒र्गं लो॒कमे॑ति प्र॒जा\-प॑ति॒र्वा ए॒ष यद॒ग्निस्तस्य॑ प्र॒जाः प॒शव॒श्छन्दाꣳ॑सि रू॒पꣳ सर्वा॒न् वर्णा॒निष्ट॑कानां कुर्याद्रू॒पेणै॒व प्र॒जां प॒शूञ्छन्दा॒ꣴ॒स्यव॑ रु॒न्द्धे\-ऽथो᳚ प्र॒जाभ्य॑ ए॒वैन॑म्प॒शुभ्य॒श्छन्दो᳚भ्यो\-ऽव॒रुद्ध्य॑ चिनुते॥~(३५)


\centerline{\scriptsize(तैत्तिरीय-ब्राह्मणम्/अष्टकम्–१/प्रश्नः—२/अनुवाकः–३)}

%1.2.3.1
सन्त॑ति॒र्वा ए॒ते ग्रहाः᳚।
यत्परः॑ सामानः।
वि॒षू॒वान्दि॑वा\-की॒र्त्यम्᳚।
यथा॒ शाला॑यै॒ पक्ष॑सी।
ए॒वꣳ सं॑वथ्स॒रस्य॒ पक्ष॑सी।
यदे॒तेन गृ॒ह्येरन्।
विषू॑ची संवथ्स॒रस्य॒ पक्ष॑सी॒ व्यव॑स्रꣳसेयाताम्।
आर्ति॒मार्च्छे॑युः।
यदे॒ते गृ॒ह्यन्ते᳚।
यथा॒ शाला॑यै॒ पक्ष॑सी मध्य॒मं व॒ꣳ॒शम॒भि स॑मा॒यच्छ॑ति॥३३॥

%1.2.3.2
ए॒वꣳ सं॑वथ्स॒रस्य॒ पक्ष॑सी दिवाकी॒र्त्य॑म॒भि सं त॑न्वन्ति।
नार्ति॒मार्च्छ॑न्ति।
ए॒क॒वि॒ꣳ॒शमह॑र्भवति।
शु॒क्राग्रा॒ ग्रहा॑ गृह्यन्ते।
प्रत्युत्त॑ब्ध्यै सय॒त्वाय॑।
सौ॒र्य॑ ए॒तदहः॑ प॒शुराल॑भ्यते।
सौ॒र्यो॑\-ऽतिग्रा॒ह्यो॑ गृह्यते।
अह॑रे॒व रू॒पेण॒ सम॑र्धयन्ति।
अथो॒ अह्न॑ ए॒वैष ब॒लिर्\mbox{}ह्रि॑यते।
स॒प्तैतदह॑रतिग्रा॒ह्या॑ गृह्यन्ते॥३४॥

%1.2.3.3
स॒प्त वै शी॑र्\mbox{}ष॒ण्या᳚ प्रा॒णाः।
अ॒सावा॑दि॒त्यः शिरः॑ प्र॒जाना᳚म्।
शी॒र्॒षन्ने॒व प्र॒जानां᳚ प्रा॒णान्द॑धाति।
तस्मा᳚थ्स॒प्त शी॒र्॒षन्प्रा॒णाः।
इन्द्रो॑ वृ॒त्रꣳ ह॒त्वा।
असु॑रान्परा॒भाव्य॑।
स इ॒माँल्लो॒कान॒भ्य॑जयत्।
तस्या॒सौ लो॒को\-ऽन॑भिजित आसीत्।
तं वि॒श्वक॑र्मा भू॒त्वा\-ऽभ्य॑जयत्।
यद्वै᳚श्वकर्म॒णो गृ॒ह्यते᳚॥३५॥

%1.2.3.4
सु॒व॒र्गस्य॑ लो॒कस्या॒भिजि॑त्यै।
प्र वा ए॒ते᳚\-ऽस्माल्लो॒काच्च्य॑वन्ते।
ये वै᳚श्वकर्म॒णं गृ॒ह्णते᳚।
आ॒दि॒त्यः श्वो गृ॑ह्यते।
इ॒यं वा अदि॑तिः।
अ॒स्यामे॒व प्रति॑ तिष्ठन्ति।
अ॒न्यो᳚न्यो गृह्येते।
विश्वा᳚न्ये॒वान्येन॒ कर्मा॑णि कुर्वा॒णा य॑न्ति।
अ॒स्याम॒न्येन॒ प्रति॑ तिष्ठन्ति।
तावाऽप॑रा॒र्धाथ्सं॑वथ्स॒रस्या॒न्यो᳚न्यो गृह्येते।
तावु॒भौ स॒ह म॑हाव्र॒ते गृ॑ह्येते।
य॒ज्ञस्यै॒वान्तं॑ ग॒त्वा।
उ॒भयो᳚र्लो॒कयोः॒ प्रति॑ तिष्ठन्ति।
अ॒र्क्य॑मु॒क्थं भ॑वति।
अ॒न्नाद्य॒स्याव॑रुद्ध्यै॥३६॥



\centerline{\scriptsize(तैत्तिरीयकाठकम्/प्रश्नः–३/अनुवाकः–९)}

   ऋ॒चां प्राची॑ मह॒ती दिगु॑च्यते।
   दक्षि॑णामाहु॒र्यजु॑षामपा॒राम्।
   अथ॑र्वणा॒मङ्गि॑रसां प्र॒तीची᳚।
   साम्ना॒मुदी॑ची मह॒ती दिगु॑च्यते।
   ऋ॒ग्भिः पू᳚र्वा॒ह्णे दि॒वि दे॒व ई॑यते।
   य॒जु॒र्वे॒दे ति॑ष्ठति॒ मध्ये॒ अह्नः॑।
   सा॒म॒वे॒देना᳚ऽस्तम॒ये मही॑यते।
   वेदै॒रशू᳚न्यस्त्रि॒भिरे॑ति॒ सूर्यः॑।
   ऋ॒ग्भ्यो जा॒ताꣳ स॑र्व॒शो मूर्ति॑माहुः।
   सर्वा॒ गति॑र्याजु॒षी है॒व शश्व॑त्॥४९॥

   सर्वं॒ तेजः॑ सामरू॒प्यꣳ ह॑ शश्वत्।
   सर्वꣳ॑ हे॒दं ब्रह्म॑णा है॒व सृ॒ष्टम्।
   ऋ॒ग्भ्यो जा॒तं वैश्यं॒ वर्ण॑माहुः।
   य॒जु॒र्वे॒दं क्ष॑त्रि॒यस्या॑ऽऽहु॒र्योनिम्᳚।
   सा॒म॒वे॒दो ब्रा᳚ह्म॒णानां॒ प्रसू॑तिः।
   पूर्वे॒ पूर्वे᳚भ्यो॒ वच॑ ए॒तदू॑चुः।
   आ॒द॒र्\mbox{}शम॒ग्निं चि॑न्वा॒नाः।
   पूर्वे॑ विश्व॒सृजो॒ऽमृताः᳚।
   श॒तं व॑र्‌षसह॒स्राणि॑।
   दी॒क्षि॒ताः स॒त्रमा॑सत॥५०॥

   तप॑ आसीद्गृ॒हप॑तिः।
   ब्रह्म॑ ब्र॒ह्माऽभ॑वथ्स्व॒यम्।
   स॒त्यꣳ ह॒ होतै॑षा॒मासी᳚त्।
   यद्वि॑श्व॒सृज॒ आस॑त।
   अ॒मृत॑मेभ्य॒ उद॑गायत्।
   स॒हस्रं॑ परिवथ्स॒रान्।
   भू॒तꣳ ह॑ प्रस्तो॒तैषा॒मासी᳚त्।
   भ॒वि॒ष्यत्प्रति॑ चाहरत्।
   प्रा॒णो अ॑ध्व॒र्युर॑भवत्।
   इ॒दꣳ सर्व॒ꣳ॒ सिषा॑सताम्॥५१॥

   अ॒पा॒नो वि॒द्वाना॒वृतः॑।
   प्रति॒प्राति॑ष्ठदध्व॒रे।
   आ॒र्त॒वा उ॑पगा॒तारः॑।
   स॒द॒स्या॑ ऋ॒तवो॑ऽभवन्।
   अ॒र्ध॒मा॒साश्च॒ मासा᳚श्च।
   च॒म॒सा॒ध्व॒र्य॒वोऽभ॑वन्।
   अ॒शꣳ॑स॒द्ब्रह्म॑ण॒स्तेजः॑।
   अ॒च्छा॒वा॒कोऽभ॑व॒द्यशः॑।
   ऋ॒तमे॑षां प्रशा॒स्ताऽऽसी᳚त्।
   यद्वि॑श्व॒सृज॒ आस॑त॥५२॥

   ऊर्ग्राजा॑न॒मुद॑वहत्।
   ध्रु॒व॒गो॒पः सहो॑ऽभवत्।
   ओजो॒ऽभ्य॑ष्टौ॒\-द्ग्राव्ण्णः॑।
   यद्वि॑श्व॒सृज॒ आस॑त।
   अप॑चितिः पो॒त्रीया॑मयजत्।
   ने॒ष्ट्रीया॑म\-यज॒त्त्विषिः॑।
   आग्नी᳚द्ध्राद्वि॒दुषी॑ स॒त्यम्।
   श्र॒द्धा है॒वाय॑जथ्स्व॒यम्।
   इरा॒ पत्नी॑ विश्व॒सृजा᳚म्।
   आकू॑तिरपिन\-ड्ढ॒विः॥५३॥

   इ॒ध्मꣳ ह॒ क्षुच्चै᳚भ्य उ॒ग्रे।
   तृ॒ष्णा चाऽऽव॑हतामु॒भे।
   वागे॑षाꣳ सुब्रह्म॒ण्याऽऽसी᳚त्।
   छ॒न्दो॒यो॒गान् वि॑जान॒ती।
   क॒ल्प॒त॒न्त्राणि॑ तन्वा॒नाऽहः॑।
   स॒ꣴ॒स्थाश्च॑ सर्व॒शः।
   अ॒हो॒रा॒त्रे प॑शुपा॒ल्यौ।
   मु॒हू॒र्ताः प्रे॒ष्या॑ अभवन्।
   मृ॒त्युस्तद॑भवद्धा॒ता।
   श॒मि॒तोग्रो वि॒शां पतिः॑॥५४॥

   वि॒श्व॒सृजः॑ प्रथ॒माः स॒त्रमा॑सत।
   स॒हस्र॑समं॒ प्रसु॑तेन॒ यन्तः॑।
   ततो॑ ह जज्ञे॒ भुव॑नस्य गो॒पाः।
   हि॒र॒ण्मयः॑ श॒कुनि॒र्ब्रह्म॒ नाम॑।
   येन॒ सूर्य॒स्तप॑ति॒ तेज॑से॒द्धः।
   पि॒ता पु॒त्रेण॑ पितृ॒मान् योनि॑योनौ।
   नावे॑दविन्मनुते॒ तं बृ॒हन्तम्᳚।
   स॒र्वा॒नु॒भुमा॒त्मानꣳ॑ सम्परा॒ये।
   ए॒ष नि॒त्यो म॑हि॒मा ब्रा᳚ह्म॒णस्य॑।
   न कर्म॑णा वर्धते॒ नो कनी॑यान्॥५५॥

   तस्यै॒वाऽऽत्मा प॑द॒वित्तं वि॑दित्वा।
   न कर्म॑णा लिप्यते॒ पाप॑केन।
   पञ्च॑पञ्चा॒शत॑स्त्रि॒वृतः॑ संवथ्स॒राः।
   पञ्च॑पञ्चा॒शतः॑ पञ्चद॒शाः।
   पञ्च॑पञ्चा॒शतः॑ सप्तद॒शाः।
   पञ्च॑पञ्चा॒शत॑ एकवि॒ꣳ॒शाः।
   वि॒श्व॒सृजाꣳ॑ स॒हस्र॑संवथ्सरम्।
   ए॒तेन॒ वै वि॑श्व॒सृज॑ इ॒दं विश्व॑मसृजन्त।
   यद्विश्व॒मसृ॑जन्त।
   तस्मा᳚द्विश्व॒सृजः॑।
   विश्व॑मेना॒ननु॒ प्रजा॑यते।
   ब्रह्म॑णः॒ सायु॑ज्यꣳ सलो॒कतां᳚ यन्ति।
   ए॒तासा॑मे॒व दे॒वता॑ना॒ꣳ॒ सायु॑ज्यम्।
   सा॒र्ष्टिताꣳ॑ समानलो॒कतां᳚ यन्ति।
   य ए॒तदु॑प॒यन्ति॑।
   ये चै॑न॒त्प्राहुः॑।
   येभ्य॑श्चैन॒त्प्राहुः॑॥५६॥
 ॐ॥


\closesection