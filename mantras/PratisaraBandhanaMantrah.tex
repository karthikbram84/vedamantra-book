% !TeX program = XeLaTeX
% !TeX root = ../vedamantrabook.tex
\chapt{प्रतिसर-बन्धन-मन्त्राः}

ॐ भूः। तथ्स॑वि॒तुर्वरे᳚ण्यम्।\
ॐ भुवः। भर्गो॑ दे॒वस्य॑ धीमहि।\\
ओꣳ सुवः। धियो॒ यो नः॑ प्रचो॒दया᳚त्।\\
ॐ भूः तथ्स॑वि॒तुर्वरे᳚ण्यम्। भर्गो॑ दे॒वस्य॑ धीमहि।\\
ॐ भुवः। धियो॒ यो नः॑ प्रचो॒दया᳚त्।\\
ओꣳ सुवः। तथ्स॑वि॒तुर्वरे᳚ण्यम्। भर्गो॑ दे॒वस्य॑ धीमहि। धियो॒ यो नः॑ प्रचो॒दया᳚त्।\\

\dnsub{वेदादयः}
अ॒ग्निमी᳚ळे पु॒रोहि॑तं य॒ज्ञस्य॑ दे॒वमृ॒त्विजम्᳚। होता᳚रं रत्न॒-धात॑मम्॥ 

इ॒षेत्वो॒र्जे त्वा॑ वा॒यवः॑ स्थो पा॒यवः॑ स्थ दे॒वो वः॑ सवि॒ता प्रार्प॑यतु॒ श्रेष्ठ॑तमाय॒ कर्म॑णे॥

अग्न॒ आया॑हि वी॒तये॑ गृणा॒नो ह॒व्यदा॑तये। नि होता॑ सथ्सि ब॒र्हिषि॑॥ 

शन्नो॑ दे॒वीर॒भिष्ट॑य॒ आपो॑ भवन्तु पी॒तये᳚। शं योर॒भिस्र॑वन्तु नः॥ 


\sect{कृणुष्व पाजः}
\ts{१}{२}{१४}{२८--३४}

कृ॒णु॒ष्व पाजः॒ प्रसि॑तिं॒ न पृ॒थ्वीं या॒हि राजे॒वाम॑वा॒ꣳ॒ इभे॑न। तृ॒ष्वीमनु॒ प्रसि॑तिं द्रूणा॒नोऽस्ता॑सि॒ विध्य॑ र॒क्षस॒स्तपि॑ष्ठैः। तव॑ भ्र॒मास॑ आशु॒या प॑त॒न्त्यनु॑ स्पृश धृष॒ता शोशु॑चानः। तपूꣴ॑ष्यग्ने जु॒ह्वा॑ पत॒ङ्गानस॑न्दितो॒ वि सृ॑ज॒ विष्व॑गु॒ल्काः। प्रति॒ स्पशो॒ वि सृ॑ज॒ तूर्णि॑तमो॒ भवा॑ पा॒युर्वि॒शो अ॒स्या अद॑ब्धः। यो नो॑ दू॒रे अ॒घशꣳ॑सो॒  यो अन्त्यग्ने॒ माकि॑ष्टे॒ व्यथि॒रा द॑धर्षीत्॥१॥

उद॑ग्ने तिष्ठ॒ प्रत्याऽऽत॑नुष्व॒ न्य॑मित्राꣳ॑ ओषतात्तिग्महेते। यो नो॒ अरा॑तिꣳ समिधान च॒क्रे नी॒चा तं ध॑क्ष्यत॒सं न शुष्कम्᳚। ऊ॒र्ध्वो भ॑व॒ प्रति॑ वि॒ध्याध्य॒स्मदा॒विष्कृ॑णुष्व॒ दैव्या᳚न्यग्ने। अव॑ स्थि॒रा त॑नुहि यातु॒जूनां᳚ जा॒मिमजा॑मिं॒ प्र मृ॑णीहि॒ शत्रून्॑। स ते॑ जानाति सुम॒तिं य॑विष्ठ॒ य ईव॑ते॒ ब्रह्म॑णे गा॒तुमैर॑त्॥२॥

 विश्वा᳚न्यस्मै सु॒दिना॑नि रा॒यो द्यु॒म्नान्य॒र्यो वि दुरो॑ अ॒भि द्यौ᳚त्। सेद॑ग्ने अस्तु सु॒भगः॑ सु॒दानु॒र्यस्त्वा॒ नित्ये॑न ह॒विषा॒ य उ॒क्थैः। पिप्री॑षति॒ स्व आयु॑षि दुरो॒णे विश्वेद॑स्मै सु॒दिना॒ साऽस॑दि॒ष्टिः। अर्चा॑मि ते सुम॒तिं घोष्य॒र्वाख्सं ते॑ वा॒वाता॑ जरतामि॒यङ्गीः॥३॥
 
स्वश्वा᳚स्त्वा सु॒रथा॑ मर्जयेमा॒स्मे क्ष॒त्राणि॑ धारये॒रनु॒ द्यून्। इ॒ह त्वा॒ भूर्या च॑रे॒दुप॒ त्मन्दोषा॑\-वस्तर्दीदि॒वाꣳ\-स॒मनु॒ द्यून्। कीड॑न्तस्त्वा सु॒मन॑सः सपेमा॒भि द्यु॒म्ना त॑स्थि॒वाꣳसो॒ जना॑नाम्। यस्त्वा॒ स्वश्वः॑ सुहिर॒ण्यो अ॑ग्न उप॒याति॒ वसु॑मता॒ रथे॑न। तस्य॑ त्रा॒ता भ॑वसि॒ तस्य॒ सखा॒ यस्त॑ आति॒थ्यमा॑नु॒षग्जुजो॑षत्। म॒हो रु॑जामि ब॒न्धुता॒ वचो॑भि॒स्तन्मा॑ पि॒तुर्गोत॑मा॒दन्वि॑याय॥४॥

 त्वं नो॑ अ॒स्य वच॑सश्चिकिद्धि॒ होत॑र्यविष्ठ सुक्रतो॒ दमू॑नाः। अस्व॑प्नजस्त॒रण॑यः सु॒शेवा॒ अत॑न्द्रासोऽवृ॒का अश्र॑मिष्ठाः। ते पा॒यवः॑ स॒ध्रिय॑ञ्चो नि॒षद्याऽग्ने॒ तव॑ नः पान्त्वमूर। ये पा॒यवो॑ मामते॒यं ते॑ अग्ने॒ पश्य॑न्तो अ॒न्धं दु॑रि॒तादर॑क्षन्। र॒रक्ष॒ तान्थ्सु॒कृतो॑ वि॒श्ववे॑दा॒ दिफ्स॑न्त॒ इद्रि॒पवो॒ ना ह॑ देभुः॥५॥
 
त्वया॑ व॒यꣳ स॑ध॒न्य॑स्त्वोता॒स्तव॒ प्रणी᳚त्यश्याम॒ वाजान्। उ॒भा शꣳसा॑ सूदय सत्यतातेऽनुष्ठु॒या कृ॑णुह्यह्रयाण। अ॒या ते॑ अग्ने स॒मिधा॑ विधेम॒ प्रति॒ स्तोमꣳ॑ श॒स्यमा॑नं गृभाय। दहा॒ऽ॒शसो॑ र॒क्षसः॑ पा॒ह्य॑स्मान्द्रु॒हो नि॒दो मि॑त्रमहो अव॒द्यात्। र॒क्षो॒हणं॑ वा॒जिन॒माऽऽजि॑घर्मि मि॒त्रं प्रथि॑ष्ठ॒मुप॑ यामि॒ शर्म॑। शिशा॑नो अ॒ग्निः क्रतु॑भिः॒ समि॑द्धः॒ स नो॒ दिवा॒ स रि॒षः पा॑तु॒ नक्तम्᳚॥६॥

 वि ज्योति॑षा बृह॒ता भा᳚त्य॒ग्निरा॒विर्विश्वा॑नि कृणुते महि॒त्वा। प्रादे॑वीर्मा॒याः स॑हते दु॒रेवाः॒ शिशी॑ते॒ शृङ्गे॒ रक्ष॑से वि॒निक्षे᳚। उ॒त स्वा॒नासो॑ दि॒विष॑न्त्व॒ग्नेस्ति॒ग्मायु॑धा॒ रक्ष॑से॒ हन्त॒वा उ॑। मदे॑ चिदस्य॒ प्ररु॑जन्ति॒ भामा॒ न व॑रन्ते परि॒बाधो॒ अदे॑वीः॥७॥


\sect{आप्यम्}
\ts{५}{७}{४}{१५--१६}

अग्ने॑ यशस्वि॒न्॒ यश॑से॒मम॑र्प॒येन्द्रा॑वती॒मप॑चितीमि॒हा व॑ह। अ॒यं मू॒र्धा प॑रमे॒ष्ठी सु॒वर्चाः᳚ समा॒नाना॑मुत्त॒मश्लो॑को अस्तु। भ॒द्रं पश्य॑न्त॒ उप॑ सेदु॒रग्रे॒ तपो॑ दी॒क्षामृष॑यः सुव॒र्विदः॑। ततः॑ क्ष॒त्रं बल॒मोज॑श्च जा॒तं तद॒स्मै दे॒वा अ॒भि सं न॑मन्तु। धा॒ता वि॑धा॒ता प॑र॒मा~(१५)

%5.7.4.4
उ॒त स॒न्दृक्प्र॒जा\-प॑तिः परमे॒ष्ठी वि॒राजा᳚। स्तोमा॒श्छन्दाꣳ॑सि नि॒विदो॑ म आहुरे॒तस्मै॑ रा॒ष्ट्रम॒भि सं न॑माम। अ॒भ्याव॑र्तध्व॒मुप॒ मेत॑ सा॒कम॒यꣳ शा॒स्ताधि॑पतिर्वो अस्तु। अ॒स्य वि॒ज्ञान॒मनु॒ सꣳ र॑भध्वमि॒मं प॒श्चादनु॑ जीवाथ॒ सर्वे᳚। 


\sect{हिरण्यवर्णीयाः}
\ts{५}{६}{१}{१--४}

हिर॑ण्यवर्णाः॒ शुच॑यः पाव॒का यासु॑ जा॒तः क॒श्यपो॒ यास्विन्द्रः॑। अ॒ग्निं या गर्भं॑ दधि॒रे विरू॑पा॒स्ता न॒ आपः॒ शꣴ स्यो॒ना भ॑वन्तु। यासा॒ꣳ॒ राजा॒ वरु॑णो॒ याति॒ मध्ये॑ सत्यानृ॒ते अ॑व॒पश्य॒ञ्जना॑नाम्। म॒धु॒श्चुतः॒ शुच॑यो॒ याः पा॑व॒कास्ता न॒ आपः॒ शꣴ स्यो॒ना भ॑वन्तु। यासां᳚ दे॒वा दि॒वि कृ॒ण्वन्ति॑ भ॒क्षं या अ॒न्तरि॑क्षे बहु॒धा भव॑न्ति। याः पृ॑थि॒वीं पय॑सो॒न्दन्ति॑~(१)

%5.6.1.2
शु॒क्रास्ता न॒ आपः॒ शꣴ स्यो॒ना भ॑वन्तु। शि॒वेन॑ मा॒ चक्षु॑षा पश्यतापः शि॒वया॑ त॒नुवोप॑ स्पृशत॒ त्वचं॑ मे। सर्वाꣳ॑ अ॒ग्नीꣳ र॑फ्सु॒षदो॑ हुवे वो॒ मयि॒ वर्चो॒ बल॒मोजो॒ नि ध॑त्त। यद॒दः स॑म्प्रय॒तीरहा॒वन॑दता ह॒ते। तस्मा॒दा न॒द्यो॑ नाम॑ स्थ॒ ता वो॒ नामा॑नि सिन्धवः। यत्प्रेषि॑ता॒ वरु॑णेन॒ ताः शीभꣳ॑ स॒मव॑ल्गत।~(२)

%5.6.1.3
तदा᳚प्नो॒दिन्द्रो॑ वो य॒तीस्तस्मा॒दापो॒ अनु॑ स्थन। अ॒प॒का॒मꣴ स्यन्द॑माना॒ अवी॑वरत वो॒ हिकम्᳚। इन्द्रो॑ वः॒ शक्ति॑भिर्देवी॒स्तस्मा॒द्वार्णाम॑ वो हि॒तम्। एको॑ दे॒वो अप्य॑तिष्ठ॒थ्स्यन्द॑माना यथाव॒शम्। उदा॑निषुर्म॒हीरिति॒ तस्मा॑दुद॒कमु॑च्यते। आपो॑ भ॒द्रा घृ॒तमिदाप॑ आसुर॒ग्नी\-षोमौ॑ बिभ्र॒त्याप॒ इत्ताः। ती॒व्रो रसो॑ मधु॒पृचा॑-~(३)

%5.6.1.4
मरं ग॒म आ मा᳚ प्रा॒णेन॑ स॒ह वर्च॑सा गन्न्। आदित्प॑श्याम्यु॒त वा॑ शृणो॒म्या मा॒ घोषो॑ गच्छति॒ वाङ्न॑ आसाम्। मन्ये॑ भेजा॒नो अ॒मृत॑स्य॒ तर्\mbox{}हि॒ हिर॑ण्यवर्णा॒ अतृ॑पं य॒दा वः॑। आपो॒ हि\-ष्ठा म॑यो॒भुव॒स्ता न॑ ऊ॒र्जे द॑धातन। म॒हे रणा॑य॒ चक्ष॑से। यो वः॑ शि॒वत॑मो॒ रस॒स्तस्य॑ भाजयते॒ह नः॑। उ॒श॒तीरि॑व मा॒तरः॑। तस्मा॒ अरं॑ गमाम वो॒ यस्य॒ क्षया॑य॒ जिन्व॑थ। आपो॑ ज॒नय॑था च नः। दि॒वि श्र॑यस्वा॒न्तरि॑क्षे यतस्व पृथि॒व्या सम्भ॑व ब्रह्मवर्च॒सम॑सि ब्रह्मवर्च॒साय॑ त्वा॥~(४)

\sect{पवमानसूक्तम्}
\tb{१}{४}{८}{४६--५१}


पव॑मानः॒ सुव॒र्जनः॑। प॒वित्रे॑ण॒ विच॑र्‌षणिः। यः पोता॒ स पु॑नातु मा। पु॒नन्तु॑ मा देवज॒नाः।
पु॒नन्तु॒ मन॑वो धि॒या। पु॒नन्तु॒ विश्व॑ आ॒यवः॑। जात॑वेदः प॒वित्र॑वत्। प॒वित्रे॑ण पुनाहि मा।
शु॒क्रेण॑ देव॒दीद्य॑त्। अग्ने॒ क्रत्वा॒ क्रतू॒ꣳ॒ रनु॑। यत्ते॑ प॒वित्र॑म॒र्चिषि॑। अग्ने॒ वित॑तमन्त॒रा।
ब्रह्म॒ तेन॑ पुनीमहे। उ॒भाभ्यां᳚ देवसवितः। प॒वित्रे॑ण स॒वेन॑ च। इ॒दं ब्रह्म॑ पुनीमहे।
वै॒श्व॒दे॒वी पु॑न॒ती दे॒व्यागा᳚त्। यस्यै॑ ब॒ह्वीस्त॒नुवो॑ वी॒तपृ॑ष्ठाः।
तया॒ मद॑न्तः सध॒माद्ये॑षु। व॒यꣴ स्या॑म॒ पत॑यो रयी॒णाम्।
वै॒श्वा॒न॒रो र॒श्मिभि॑र्मा पुनातु। वातः॑ प्रा॒णेने॑षि॒रो म॑यो॒ भूः।
द्यावा॑पृथि॒वी पय॑सा॒ पयो॑भिः। ऋ॒ताव॑री य॒ज्ञिये॑ मा पुनीताम्।
बृ॒हद्भिः॑ सवित॒स्तृभिः॑। वर्{}षि॑ष्ठैर्देव॒मन्म॑भिः।
अग्ने॒ दक्षैः᳚ पुनाहि मा। येन॑ दे॒वा अपु॑नत।
येनाऽऽपो॑ दि॒व्यं कशः॑। तेन॑ दि॒व्येन॒ ब्रह्म॑णा। इ॒दं ब्रह्म॑ पुनीमहे। यः पा॑वमा॒नीर॒ध्येति॑।
ऋषि॑भिः॒ सम्भृ॑त॒ꣳ॒ रसम्᳚। सर्व॒ꣳ॒ स पू॒तम॑श्ञाति।
स्व॒दि॒तं मा॑त॒रिश्व॑ना। पा॒व॒मा॒नीर्यो अ॒ध्येति॑।
ऋषि॑भिः॒ सम्भृ॑त॒ꣳ॒ रसम्᳚। तस्मै॒ सर॑स्वती दुहे। क्षी॒रꣳ स॒र्पिर्मधू॑द॒कम्॥
पा॒व॒मा॒नीः स्व॒स्त्यय॑नीः। सु॒दुघा॒हि पय॑स्वतीः।
ऋषि॑भिः॒ सम्भृ॑तो॒ रसः॑। ब्रा॒ह्म॒णेष्व॒मृतꣳ॑ हि॒तम्।
पा॒व॒मा॒नीर्दि॑शन्तु नः। इ॒मं लो॒कमथो॑ अ॒मुम्।
कामा॒न्थ्सम॑र्धयन्तु नः। दे॒वीर्दे॒वैः स॒माभृ॑ताः।
पा॒व॒मा॒नीः स्व॒स्त्यय॑नीः। सु॒दुघा॒हि घृ॑त॒श्चुतः॑।
ऋषि॑भिः॒ सम्भृ॑तो॒ रसः॑। ब्रा॒ह्म॒णेष्व॒मृतꣳ॑ हि॒तम्।
येन॑ दे॒वाः प॒वित्रे॑ण। आ॒त्मानं॑ पु॒नते॒ सदा᳚।
तेन॑ स॒हस्र॑धारेण। पा॒व॒मा॒न्यः पु॑नन्तु मा।
प्रा॒जा॒प॒त्यं प॒वित्रम्᳚। श॒तोद्या॑मꣳ हिर॒ण्मयम्᳚।
तेन॑ ब्रह्म॒ विदो॑ व॒यम्। पू॒तं ब्रह्म॑ पुनीमहे।
इन्द्रः॑ सुनी॒ती स॒ह मा॑ पुनातु। सोमः॑ स्व॒स्त्या वरु॑णः स॒मीच्या᳚।
य॒मो राजा᳚ प्रमृ॒णाभिः॑ पुनातु मा। जा॒तवे॑दा मो॒र्जय॑न्त्या पुनातु। 

\sect{वरुणसूक्तम्}
\ts{१}{५}{११}{४९--५०}

उदु॑त्त॒मं व॑रुण॒ पाश॑\-म॒स्मद\-वा॑ध॒मं वि म॑ध्य॒मꣴ श्र॑थाय। अथा॑ व॒यमा॑दित्यव्र॒ते तवाना॑गसो॒ अदि॑तये स्याम॥


\ts{१}{२}{८}{१५}

अस्त॑भ्ना॒द्द्यामृ॑ष॒भो अ॒न्तरि॑क्ष॒ममि॑मीत वरि॒माणं॑ पृथि॒व्या। आसी॑द॒द्विश्वा॒ भुव॑नानि स॒म्राड्विश्वेत्तानि॒ वरु॑णस्य व्र॒तानि॑।


\ts{३}{४}{११}{४६}

यत्किं चे॒दं व॑रुण॒ दैव्ये॒ जने॑\-ऽभिद्रो॒हं म॑नु॒ष्या᳚श्चरा॑मसि। अचि॑त्ती॒ यत्तव॒ धर्मा॑ युयोपि॒म मा न॒स्तस्मा॒देन॑सो देव रीरिषः। कि॒त॒वासो॒ यद्रि॑रि॒पुर्न दी॒वि यद्वा॑ घा स॒त्यमु॒त यन्न वि॒द्म। सर्वा॒ ता वि ष्य॑ शिथि॒रेव॑ दे॒वाथा॑ ते स्याम वरुण प्रि॒यासः॑॥

\ts{१}{५}{११}{४९--५०}

अव॑ ते॒ हेडो॑ वरुण॒ नमो॑\-भि॒रव॑ य॒ज्ञेभि॑रीमहे ह॒विर्भिः॑। क्षय॑न्न॒स्मभ्य॑मसुर प्रचेतो॒ राज॒न्नेनाꣳ॑सि शिश्रथः कृ॒तानि॑॥

\ts{२}{१}{११}{६५}

तत्त्वा॑ यामि॒ ब्रह्म॑णा॒ वन्द॑मान॒स्तदाऽऽशा᳚स्ते॒ यज॑मानो ह॒विर्भिः॑। अहे॑डमानो वरुणे॒ह बो॒द्ध्युरु॑शꣳस॒ मा न॒ आयुः॒ प्रमो॑षीः॥


\sect{रुद्रसूक्तम्}
\ts{४}{५}{१०}{२४}

परि॑ णो रु॒द्रस्य॑ हे॒तिर्वृ॑णक्तु॒ परि॑ त्वे॒षस्य॑ दुर्म॒तिर॑घा॒योः। अव॑ स्थि॒रा म॒घव॑द्भ्यस्तनुष्व॒ मीढ्व॑स्तो॒काय॒ तन॑याय मृडय।

\ts{४}{५}{१०}{२३--२४}

स्तु॒हि श्रु॒तं ग॑र्त॒सदं॒ युवा॑नं मृ॒गं न भी॒ममु॑पह॒त्नुमु॒ग्रम्। मृ॒डा ज॑रि॒त्रे रु॑द्र॒ स्तवा॑नो अ॒न्यं ते॑ अ॒स्मन्नि व॑पन्तु॒ सेनाः᳚।

\ts{४}{५}{१०}{२४--२५}

मीढु॑ष्टम॒ शिव॑तम शि॒वो नः॑ सु॒मना॑ भव। प॒र॒मे वृ॒क्ष आयु॑धं नि॒धाय॒ कृत्तिं॒ वसा॑न॒ आ च॑र॒ पिना॑कं॒ बिभ्र॒दा ग॑हि।

\ta{४}{५}{१८}

अर्\mbox{}ह॑न्बिभर्\mbox{}षि॒ साय॑कानि॒ धन्व॑। 
अर्\mbox{}हं॑ नि॒ष्कं य॑ज॒तं  वि॒श्वरू॑पम्। 
अर्\mbox{}हं॑ नि॒दन्द॑यसे॒ विश्व॒मब्भु॑वम्। 
न वा ओजी॑यो रुद्र॒ त्वद॑स्ति। 

\ts{१}{३}{१४}{२४}

त्वम॑ग्ने रु॒द्रो असु॑रो म॒हो दि॒वस्त्वꣳ शर्धो॒ मारु॑तं पृ॒क्ष ई॑शिषे। त्वं वातै॑ररु॒णैर्या॑सि शङ्ग॒यस्त्वं पू॒षा वि॑ध॒तः पा॑सि॒ नु त्मना᳚॥
आ वो॒ राजा॑नमध्व॒रस्य॑ रु॒द्रꣳ होता॑रꣳ सत्य॒यज॒ꣳ॒ रोद॑स्योः। अ॒ग्निं पु॒रा त॑नयि॒त्नोर॒चित्ता॒द्धिर॑ण्यरूप॒मव॑से कृणुध्वम्॥



\sect{ब्रह्मसूक्तम्}
\tb{२}{८}{८}{६६--६९}
ब्रह्म॑ जज्ञा॒नं प्र॑थ॒मं पु॒रस्ता᳚त्।
वि सी॑म॒तः सु॒रुचो॑ वे॒न आ॑वः।
स बु॒ध्निया॑ उप॒ मा अ॑स्य वि॒ष्ठाः॥६६॥

%2.8.8.9
स॒तश्च॒ योनि॒मस॑तश्च॒ विवः॑।
पि॒ता वि॒राजा॑मृष॒भो र॑यी॒णाम्।
अ॒न्तरि॑क्षं वि॒श्वरू॑प॒ आवि॑वेश।
तम॒र्कैर॒भ्य॑र्चन्ति व॒थ्सम्।
ब्रह्म॒ सन्तं॒ ब्रह्म॑णा व॒र्धय॑न्तः।
ब्रह्म॑ दे॒वान॑जनयत्।
ब्रह्म॒ विश्व॑मि॒दं जग॑त्।
ब्रह्म॑णः क्ष॒त्रं निर्मि॑तम्।
ब्रह्म॑ ब्राह्म॒ण आ॒त्मना᳚।
अ॒न्तर॑स्मिन्नि॒मे लो॒काः॥६७॥

%2.8.8.10
अ॒न्तर्विश्व॑मि॒दं जग॑त्।
ब्रह्मै॒व भू॒तानां॒ ज्येष्ठम्᳚।
तेन॒ को॑\-ऽर्\mbox{}हति॒ स्पर्धि॑तुम्।
ब्रह्म॑न्दे॒वास्त्रय॑स्त्रिꣳशत्।
ब्रह्म॑न्निन्द्रप्रजाप॒ती।
ब्रह्म॑न् ह॒ विश्वा॑ भू॒तानि॑।
ना॒वीवा॒न्तः स॒माहि॑ता।
चत॑स्र॒ आशाः॒ प्रच॑रन्त्व॒ग्नयः॑।
इ॒मं नो॑ य॒ज्ञं न॑यतु प्रजा॒नन्।
घृ॒तं पिन्व॑न्न॒जरꣳ॑ सु॒वीरम्᳚॥६८॥

%2.8.8.12
ब्रह्म॑ स॒मिद्भ॑व॒त्याहु॑तीनाम्।


\sect{विष्णुसूक्तम् (ऋ०)}
\centerline{(ऋ॰सं॰ १.२२.१६)}

अतो॑ दे॒वा अ॑वन्तु॒ नो यतो विष्णु॑र्विचक्र॒मे । पृ॒थि॒व्याः स॒प्त धाम॑भिः॥ इ॒दं विष्णु॒र्विच॑क्रमे त्रे॒धा नि द॑धे प॒दम्। समू॑ढमस्य पाꣳसु॒रे॥ त्रीणि॑ प॒दा वि च॑क्रमे॒ विष्णु॑र्गो॒पा अदा॑भ्यः। ततो॒ धर्मा॑णि धा॒रयन्॑॥ विष्णोः॒ कर्मा॑णि पश्यत॒ यतो॑ व्र॒तानि॑ पस्प॒शे। इन्द्र॑स्य॒ युज्यः॒ सखा॑॥ तद्विष्णोः॑ पर॒मं प॒दꣳ सदा॑ पश्यन्ति सू॒रयः॑। दि॒वीव॒ चक्षु॒रात॑तम्॥ तद्विप्रा॑सो विप॒न्यवो॑ जागृवाꣳसः॒ समि᳚न्धते। विष्णो॒र्यत् प॑र॒मं प॒दम्॥


\sect{दुर्गा सूक्तम्}
\ta{१०}{२}{}

जा॒तवे॑दसे सुनवाम॒ सोम॑ मरातीय॒तो निद॑हाति॒ वेदः॑।
स नः॑ पर्‌ष॒दति॑ दु॒र्गाणि॒ विश्वा॑ ना॒वेव॒ सिन्धुं॑ दुरि॒ताऽत्य॒ग्निः॥१॥

ताम॒ग्निव॑र्णां॒ तप॑सा ज्वल॒न्तीं वै॑रोच॒नीं क॑र्मफ॒लेषु॒ जुष्टा᳚म्।
दु॒र्गां दे॒वीꣳ शर॑णम॒हं प्रप॑द्ये सु॒तर॑सि तरसे॒ नमः॑॥२॥

अग्ने॒ त्वं पा॑रया॒ नव्यो॑ अ॒स्मान्थ्स्व॒स्तिभि॒रति॑ दु॒र्गाणि॒ विश्वा᳚।
पूश्च॑ पृ॒थ्वी ब॑हु॒ला न॑ उ॒र्वी भवा॑ तो॒काय॒ तन॑याय॒ शं योः॥३॥

विश्वा॑नि नो दु॒र्गहा॑ जातवेदः॒ सिन्धुं॒ न ना॒वा दु॑रि॒ताऽति॑पर्‌\mbox{}षि।
अग्ने॑ अत्रि॒वन्मन॑सा गृणा॒नो᳚ऽस्माकं॑ बोध्यवि॒ता त॒नूना᳚म्॥४॥

पृ॒त॒ना॒ जित॒ꣳ॒ सह॑मानमु॒ग्रम॒ग्निꣳ हु॑वेम पर॒माथ्स॒धस्था᳚त्।
स नः॑ पर्‌ष॒दति॑ दु॒र्गाणि॒ विश्वा॒ क्षाम॑द्दे॒वो अति॑ दुरि॒तात्य॒ग्निः॥५॥

प्र॒त्नोषि॑ क॒मीड्यो॑ अध्व॒रेषु॑ स॒नाच्च॒ होता॒ नव्य॑श्च॒ सथ्सि॑।
स्वाञ्चा᳚ग्ने त॒नुवं॑ पि॒प्रय॑स्वा॒स्मभ्यं॑ च॒ सौभ॑ग॒माय॑जस्व॥६॥

गोभि॒र्जुष्ट॑म॒युजो॒ निषि॑क्तं॒ तवे᳚न्द्र विष्णो॒रनु॒सञ्च॑रेम।
नाक॑स्य पृ॒ष्ठम॒भि सं॒वसा॑नो॒ वैष्ण॑वीं लो॒क इ॒ह मा॑दयन्ताम्॥७॥


\sect{श्रीसूक्तम्}

{\centering
हिर॑ण्यवर्णां॒ हरि॑णीं सुव॒र्णर॑जत॒स्रजाम्।\\
च॒न्द्रां॒ हि॒रण्म॑यीं ल॒क्ष्मीं॒ जात॑वेदो म॒ आव॑ह॥१॥

तां म॒ आव॑ह॒ जात॑वेदो ल॒क्ष्मीमन॑पगा॒मिनी᳚म्।\\
यस्यां॒ हिर॑ण्यं वि॒न्देयं॒ गामश्वं॒ पुरु॑षान॒हम्॥२॥

अ॒श्व॒पू॒र्वां र॑थम॒ध्यां॒ ह॒स्तिना॑दप्र॒बोधि॑नीम्।\\
श्रियं॑ दे॒वीमुप॑ह्वये॒ श्रीर्मा॑दे॒वीर्जु॑षताम्॥३॥

}

कां॒ सो॒ऽ॒स्मि॒तां हिर॑ण्यप्राकारामा॒र्द्रां ज्वल॑न्तीं तृ॒प्तां त॒र्पय॑न्तीम्।
प॒द्मे॒ स्थि॒तां प॒द्मव॑र्णां॒ तामि॒होप॑ह्वये॒ श्रियम्॥४॥

च॒न्द्रां प्र॑भा॒सां य॒शसा॒ ज्वल॑न्तीं॒ श्रियं॑ लो॒के दे॒वजु॑ष्टामुदा॒राम्।
तां प॒द्मिनी॑मीं॒ शर॑णम॒हं प्रप॑द्येऽल॒क्ष्मीर्मे॑ नश्यतां॒ त्वां वृ॑णे॥५॥

आ॒दि॒त्यव॑र्णे॒ तप॒सोऽधि॑जा॒तो वन॒स्पति॒स्तव॑ वृ॒क्षोऽथ बि॒ल्वः।
तस्य॒ फला॑नि॒ तप॒सा नु॑दन्तु मा॒यान्त॑रा॒याश्च॑ बा॒ह्या अ॑ल॒क्ष्मीः॥६॥

उपै॑तु॒ मां दे॑वस॒खः की॒र्तिश्च॒ मणि॑ना स॒ह।
प्रा॒दु॒र्भू॒तोऽस्मि॑ राष्ट्रे॒ऽ॒स्मि॒न् की॒र्तिमृद्धिं॑ ददा॒तु मे॥७॥

क्षुत्पि॑पा॒साम॑लां ज्ये॒ष्ठा॒मल॒क्ष्मीं ना॑शया॒म्यहम्।
अभू॑ति॒\-मस॑मृद्धिं॒ च सर्वां॒ निर्णु॑द मे॒ गृहात्॥८॥

ग॒न्ध॒द्वा॒रां दु॑राध॒र्‌षां॒ नि॒त्यपु॑ष्टां करी॒षिणी᳚म्।
ई॒श्वरीं᳚ सर्व॑भूता॒नां॒ तामि॒होप॑ह्वये॒ श्रियम्॥९॥

मन॑सः॒ काम॒माकू॑तिं वा॒चः स॒त्यम॑शीमहि।
प॒शू॒नां रू॒पमन्न॑स्य॒ मयि॒ श्रीः श्र॑यतां॒ यशः॑॥१०॥

क॒र्दमे॑न प्र॑जाभू॒ता॒ म॒यि॒ सम्भ॑व क॒र्दम।
श्रियं॑ वा॒सय॑ मे कु॒ले मा॒तरं॑ पद्ममा॒लिनीम्॥११॥

आपः॑ सृ॒जन्तु॑ स्निग्धा॒नि॒ चिक्ली॒त व॑स मे॒ गृहे।
नि च॑ दे॒वीं मा॒तरं॒ श्रियं॑ वा॒सय॑ मे कु॒ले॥१२॥

आ॒र्द्रां पु॒ष्करि॑णीं पु॒ष्टिं॒ सु॒व॒र्णां हे॑ममा॒लिनीम्।
सू॒र्यां हि॒रण्म॑यीं ल॒क्ष्मीं॒ जात॑वेदो म॒ आव॑ह॥१३॥

आ॒र्द्रां यः॒ करि॑णीं य॒ष्टिं॒ पि॒ङ्ग॒लां प॑द्ममा॒लिनीम्।
च॒न्द्रां हि॒रण्म॑यीं ल॒क्ष्मीं॒ जात॑वेदो म॒ आव॑ह॥१४॥

तां म॒ आव॑ह॒ जात॑वेदो ल॒क्ष्मीमन॑पगा॒मिनी᳚म्।
यस्यां॒ हि॑रण्यं॒ प्रभू॑तं॒ गावो॑ दा॒स्योऽश्वा᳚न् वि॒न्देयं॒ पुरु॑षान॒हम्॥१५॥

\sect{परिधानीया}

ॐ नमो॒ ब्रह्म॑णे॒ नमो॑ अस्त्व॒ग्नये॒ नमः॑ पृथि॒व्यै नम॒ ओष॑धीभ्यः।
नमो॑ वा॒चे नमो॑ वा॒चस्पत॑ये॒ नमो॒ विष्ण॑वे बृह॒ते क॑रोमि॥


\closesection
