% !TeX program = XeLaTeX
% !TeX root = ../AraNyakabook-kindle.tex
%ꣳ॒ ꣳ॑ ꣳ
%ꣴ ꣴ॒ ꣴ॑
\sect{दशमः प्रश्नः --- महानारायणोपनिषत्}\setcounter{anuvakam}{0}
ॐ स॒ह ना॑ववतु। 
स॒ह नौ॑ भुनक्तु। 
स॒ह वी॒र्यं॑ करवावहै। 
ते॒ज॒स्वि ना॒वधी॑तमस्तु॒ मा वि॑द्विषा॒वहै᳚। 
ॐ शान्तिः॒ शान्तिः॒ शान्तिः॑॥

\dnsub{अम्भस्य पारे}
%६.१.१
अम्भ॑स्य पा॒रे भुव॑नस्य॒ मध्ये॒ नाक॑स्य पृ॒ष्ठे म॑ह॒तो मही॑यान्। 
शु॒क्रेण॒ ज्योतीꣳ॑षि समनु॒प्रवि॑ष्टः प्र॒जाप॑तिश्चरति॒ गर्भे॑ अ॒न्तः॥ 
यस्मि॑न्नि॒दꣳ सं च॒ विचैति॒ सर्वं॒ यस्मि॑न्दे॒वा अधि॒ विश्वे॑ निषे॒दुः। 
तदे॒व भू॒तं तदु॒ भव्य॑मा इ॒दं तद॒क्षरे॑ पर॒मे व्यो॑मन्॥ 
येना॑ऽऽवृ॒तं खं च॒ दिवं॑ म॒हीं च॒ येना॑ऽऽदि॒त्यस्तप॑ति॒ तेज॑सा॒ भ्राज॑सा च। 
यम॒न्तः स॑मु॒द्रे क॒वयो॒ वय॑न्ति॒ यद॒क्षरे॑ पर॒मे प्र॒जाः॥ 
यतः॑ प्रसू॒ता ज॒गतः॑ प्रसूती॒ तोये॑न जी॒वान् व्यस॑सर्ज॒ भूम्या᳚म्। 
यदोष॑धीभिः पु॒रुषा᳚न्प॒शूꣴश्च॒ विवे॑श भू॒तानि॑ चराच॒राणि॑॥ 
अतः॑ परं॒ नान्य॒दणी॑यसꣳ हि॒ परा᳚त्परं॒ यन्मह॑तो म॒हान्तम्᳚। 
यदे॑कम॒व्यक्त॒मन॑न्तरूपं॒ विश्वं॑ पुरा॒णं तम॑सः॒ पर॑स्तात्॥१॥

%६.१.२
तदे॒वर्तं तदु॑ स॒त्यमा॑हु॒स्तदे॒व ब्रह्म॑ पर॒मं क॑वी॒नाम्। 
इ॒ष्टा॒पू॒र्तं ब॑हु॒धा जा॒तं जाय॑मानं वि॒श्वं बि॑भर्ति॒ भुव॑नस्य॒ नाभिः॑॥ 
तदे॒वाग्निस्तद्वा॒युस्तथ्सूर्य॒स्तदु॑ च॒न्द्रमाः᳚। 
तदे॒व शु॒क्रम॒मृतं॒ तद्ब्रह्म॒ तदापः॒ स प्र॒जाप॑तिः॥ 
सर्वे॑ निमे॒षा ज॒ज्ञिरे॑ वि॒द्युतः॒ पुरु॑षा॒दधि॑। 
क॒ला मु॑हू॒र्ताः काष्ठा᳚श्चाहोरा॒त्राश्च॑ सर्व॒शः॥ 
अ॒र्ध॒मा॒सा मासा॑ ऋ॒तवः॑ संवथ्स॒रश्च॑ कल्पन्ताम्। 
स आपः॑ प्रदु॒घे उ॒भे इ॒मे अ॒न्तरि॑क्ष॒मथो॒ सुवः॑॥ 
नैन॑मू॒र्ध्वं न ति॒र्यञ्चं॒ न मध्ये॒ परि॑जग्रभत्। 
न तस्ये॑शे॒ कश्च॒न तस्य॑ नाम म॒हद्यशः॑॥२॥

%६.१.३
न स॒न्दृशे॑ तिष्ठति॒ रूप॑मस्य॒ न चक्षु॑षा पश्यति॒ कश्च॒नैनम्᳚। 
हृ॒दा म॑नी॒षा मन॑सा॒ऽभिक्लृ॑प्तो॒ य ए॑नं वि॒दुरमृ॑ता॒स्ते भ॑वन्ति॥ 
अ॒द्भ्यः सम्भू॑तो हिरण्यग॒र्भ इत्य॒ष्टौ॥ 
ए॒ष हि दे॒वः प्र॒दिशोऽनु॒ सर्वाः॒ पूर्वो॑ हि जा॒तः स उ॒ गर्भे॑ अ॒न्तः। 
स वि॒जाय॑मानः स जनि॒ष्यमा॑णः प्र॒त्यङ्मुखा᳚स्तिष्ठति वि॒श्वतो॑मुखः॥ 
वि॒श्वत॑श्चक्षुरु॒त वि॒श्वतो॑मुखो वि॒श्वतो॑हस्त उ॒त वि॒श्वत॑स्पात्। 
सं बा॒हुभ्यां॒ नम॑ति॒ सं पत॑त्रै॒र्द्यावा॑पृथि॒वी ज॒नय॑न्दे॒व एकः॑॥ 
वे॒नस्तत्पश्य॒न्विश्वा॒ भुव॑नानि वि॒द्वान् यत्र॒ विश्वं॒ भव॒त्येक॑नीळम्। 
यस्मि॑न्नि॒दꣳ सं च॒ विचैक॒ꣳ॒ स ओतः॒ प्रोत॑श्च वि॒भुः प्र॒जासु॑। 
प्र तद्वो॑चे अ॒मृतं॒ नु वि॒द्वान्ग॑न्ध॒र्वो नाम॒ निहि॑तं॒ गुहा॑सु॥३॥

%६.१.४
त्रीणि॑ प॒दा निहि॑ता॒ गुहा॑सु॒ यस्तद्वेद॑ सवि॒तुः पि॒ताऽस॑त्। 
स नो॒ बन्धु॑र्जनि॒ता स वि॑धा॒ता धामा॑नि॒ वेद॒ भुव॑नानि॒ विश्वा᳚। 
यत्र॑ दे॒वा अ॒मृत॑मानशा॒नास्तृ॒तीये॒ धामा᳚न्य॒भ्यैर॑यन्त। 
परि॒ द्यावा॑पृथि॒वी य॑न्ति स॒द्यः परि॑ लो॒कान् परि॒ दिशः॒ परि॒ सुवः॑। 
ऋ॒तस्य॒ तन्तुं॑ विततं वि॒चृत्य॒ तद॑पश्य॒त्तद॑भवत् प्र॒जासु॑। 
प॒रीत्य॑ लो॒कान्प॒रीत्य॑ भू॒तानि॑ प॒रीत्य॒ सर्वाः᳚ प्र॒दिशो॒ दिश॑श्च। 
प्र॒जाप॑तिः प्रथम॒जा ऋ॒तस्या॒ऽ॒ऽ॒त्मना॒ऽ॒ऽ॒त्मान॑म॒भिसम्ब॑भूव। 
सद॑स॒स्पति॒मद्भु॑तं प्रि॒यमिन्द्र॑स्य॒ काम्यम्᳚। 
सनिं॑ मे॒धाम॑यासिषम्। 
उद्दी᳚प्यस्व जातवेदोऽप॒घ्नन्निर्\mbox{}ऋ॑तिं॒ मम॑॥४॥

%६.१.५
प॒शूꣴश्च॒ मह्य॒माव॑ह॒ जीव॑नं च॒ दिशो॑ दिश। 
मा नो॑ हिꣳसीज्जातवेदो॒ गामश्वं॒ पुरु॑षं॒ जग॑त्। 
अबि॑भ्र॒दग्न॒ आग॑हि श्रि॒या मा॒ परि॑पातय।

\dnsub{गायत्रीमन्त्राः}

 पुरु॑षस्य विद्म सहस्रा॒क्षस्य॑ महादे॒वस्य॑ धीमहि। 
 तन्नो॑ रुद्रः प्रचो॒दया᳚त्। 
 तत्पुरु॑षाय वि॒द्महे॑ महादे॒वाय॑ धीमहि। 
 तन्नो॑ रुद्रः प्रचो॒दया᳚त्। 
 तत्पुरु॑षाय वि॒द्महे॑ वक्रतु॒ण्डाय॑ धीमहि। 
 तन्नो॑ दन्तिः प्रचो॒दया᳚त्। 
 तत्पुरु॑षाय वि॒द्महे॑ चक्रतु॒ण्डाय॑ धीमहि॥५॥

 तन्नो॑ नन्दिः प्रचो॒दया᳚त्। 
 तत्पुरु॑षाय वि॒द्महे॑ महासे॒नाय॑ धीमहि। 
 तन्नः॑ षण्मुखः प्रचो॒दया᳚त्। 
 तत्पुरु॑षाय वि॒द्महे॑ सुवर्णप॒क्षाय॑ धीमहि। 
 तन्नो॑ गरुडः प्रचो॒दया᳚त्। 
 वे॒दा॒त्म॒नाय॑ वि॒द्महे॑ हिरण्यग॒र्भाय॑ धीमहि। 
 तन्नो᳚ ब्रह्म॑ प्रचो॒दया᳚त्। 
 ना॒रा॒य॒णाय॑ वि॒द्महे॑ वासुदे॒वाय॑ धीमहि। 
 तन्नो॑ विष्णुः प्रचो॒दया᳚त्। 
 व॒ज्र॒न॒खाय॑ वि॒द्महे॑ तीक्ष्णद॒ꣴ॒ष्ट्राय॑ धीमहि॥६॥
 
तन्नो॑ नारसिꣳहः प्रचो॒दया᳚त्। 
भा॒स्क॒राय॑ वि॒द्महे॑ महद्युतिक॒राय॑ धीमहि। 
तन्नो॑ आदित्यः प्रचो॒दया᳚त्। 
वै॒श्वा॒न॒राय॑ वि॒द्महे॑ लाली॒लाय॑ धीमहि। 
तन्नो॑ अग्निः प्रचो॒दया᳚त्। 
का॒त्या॒य॒नाय॑ वि॒द्महे॑ कन्यकु॒मारि॑ धीमहि। 
तन्नो॑ दुर्गिः प्रचो॒दया᳚त्। 


\dnsub{दूर्वासूक्तम्}
स॒ह॒स्र॒पर॑मा दे॒वी॒ श॒तमू॑ला श॒ताङ्कु॑रा। 
सर्वꣳ॑ हरतु॑ मे पा॒पं॒ दू॒र्वा दुः॑स्वप्न॒नाश॑नी। 
काण्डा᳚त्काण्डात् प्र॒रोह॑न्ती॒ परु॑षः परुषः॒ परि॑॥७॥

ए॒वानो॑ दूर्वे॒ प्रत॑नु स॒हस्रे॑ण श॒तेन॑ च। 
या श॒तेन॑ प्रत॒नोषि॑ स॒हस्रे॑ण वि॒रोह॑सि। 
तस्या᳚स्ते देवीष्टके वि॒धेम॑ ह॒विषा॑ व॒यम्। 
अश्व॑क्रा॒न्ते र॑थक्रा॒न्ते॒ वि॒ष्णुक्रा᳚न्ते व॒सुन्ध॑रा। 
शिरसा॑ धार॑यिष्या॒मि॒ र॒क्ष॒स्व मां᳚ पदे॒ पदे।

\dnsub{मृत्तिकासूक्तम्}
 भूमिर्धेनुर्धरणी लो॑कधा॒रिणी। 
 उ॒द्धृता॑ऽसि व॑राहे॒ण॒ कृ॒ष्णे॒न श॑तबा॒हुना। 
 मृ॒त्तिके॑ हन॑ मे पा॒पं॒ य॒न्म॒या दु॑ष्कृतं॒ कृतम्। 
 मृ॒त्तिके᳚ ब्रह्म॑दत्ता॒ऽ॒सि॒ का॒श्यपे॑नाभि॒मन्त्रि॑ता। 
 मृ॒त्तिके॑ देहि॑ मे पु॒ष्टिं॒ त्व॒यि स॑र्वं प्र॒तिष्ठि॑तम्॥८॥
 
 मृ॒त्तिके᳚ प्रतिष्ठि॑ते स॒र्वं॒ त॒न्मे नि॑र्णुद॒ मृत्ति॑के। 
 तया॑ ह॒तेन॑ पापे॒न॒ ग॒च्छा॒मि प॑रमां॒ गतिम्।

\dnsub{शत्रुजयमन्त्राः}
यत॑ इन्द्र॒ भया॑महे॒ ततो॑ नो॒ अभ॑यं कृधि। 
मघ॑वन्छ॒ग्धि तव॒ तन्न॑ ऊ॒तये॒ विद्विषो॒ विमृधो॑ जहि। 
स्व॒स्ति॒दा वि॒शस्पति॑र्वृत्र॒हा विमृधो॑ व॒शी। 
वृषेन्द्रः॑ पु॒र ए॑तु नः स्वस्ति॒दा अ॑भयङ्क॒रः। 
स्व॒स्ति न॒ इन्द्रो॑ वृ॒द्धश्र॑वाः स्व॒स्ति नः॑ पू॒षा वि॒श्ववे॑दाः। 
स्व॒स्ति न॒स्तार्क्ष्यो॒ अरि॑ष्टनेमिः स्व॒स्ति नो॒ बृह॒स्पति॑र्दधातु। 
आपा᳚न्तमन्युस्तृ॒पल॑प्रभर्मा॒ धुनिः॒ शिमी॑वा॒ञ्छरु॑माꣳ ऋजी॒षी। 
सोमो॒ विश्वा᳚न्यत॒सावना॑नि॒ नार्वागिन्द्रं॑ प्रति॒माना॑निदेभुः॥९॥

ब्रह्म॑जज्ञा॒नं प्र॑थ॒मं पु॒रस्ता॒द्विसी॑म॒तः सु॒रुचो॑ वे॒न आ॑वः। 
सबु॒ध्निया॑ उप॒मा अ॑स्य वि॒ष्ठाः स॒तश्च॒ योनि॒मस॑तश्च॒ विवः॑। 
स्यो॒ना पृ॑थिवि॒ भवा॑ऽनृक्ष॒रा नि॒वेश॑नी। 
यच्छा॑नः॒ शर्म॑ स॒प्रथाः᳚। 
ग॒न्ध॒द्वा॒रां दु॑राध॒र्॒‌षां॒ नि॒त्यपु॑ष्टां करी॒षिणी᳚म्। 
ई॒श्वरीꣳ॑ सर्व॑भूता॒नां॒ तामि॒होप॑ह्वये॒ श्रियम्। 
श्री᳚र्मे भ॒जतु। 
अलक्ष्मी᳚र्मे न॒श्यतु। 
विष्णु॑मुखा॒ वै दे॒वाश्छन्दो॑भिरि॒माँल्लो॒कान॑नप\-ज॒य्यम॒भ्य॑जयन्। 
म॒हाꣳ इन्द्रो॒ वज्र॑बाहुः षोड॒शी शर्म॑ यच्छतु॥१०॥

 स्व॒स्ति नो॑ म॒घवा॑ करोतु॒ हन्तु॑ पा॒प्मानं॒ यो᳚ऽस्मान् द्वेष्टि॑। 
 सो॒मान॒ꣴ॒ स्वर॑णं कृणु॒हि ब्र॑ह्मणस्पते। 
 क॒क्षीव॑न्तं॒ य औ॑शि॒जम्। 
 शरी॑रं यज्ञशम॒लं कुसी॑दं॒ तस्मि᳚न्थ्सीदतु॒ यो᳚ऽस्मान् द्वेष्टि॑। 
 चर॑णं प॒वित्रं॒ वित॑तं पुरा॒णं येन॑ पू॒तस्तर॑ति दुष्कृ॒तानि॑। 
 तेन॑ प॒वित्रे॑ण शु॒द्धेन॑ पू॒ता अति॑ पा॒प्मान॒मरा॑तिं तरेम। 
 स॒जोषा॑ इन्द्र॒ सग॑णो म॒रुद्भिः॒ सोमं॑ पिब वृत्रहञ्छूर वि॒द्वान्। 
 ज॒हि शत्रू॒ꣳ॒ रप॒ मृधो॑ नुद॒स्वाथाभ॑यं कृणुहि वि॒श्वतो॑ नः। 
 सु॒मि॒त्रा न॒ आप॒ ओष॑धयः सन्तु दुर्मि॒त्रास्तस्मै॑ भूयासु॒र्यो᳚ऽस्मान् द्वेष्टि॒ यं च॑ व॒यं द्वि॒ष्मः। 
 आपो॒ हि ष्ठा म॑यो॒ भुव॒स्ता न॑ ऊ॒र्जे द॑धातन॥११॥
 
 म॒हेरणा॑य॒ चक्ष॑से। 
 यो वः॑ शि॒वत॑मो॒ रस॒स्तस्य॑ भाजयते॒ह नः॑। 
 उ॒श॒तीरि॑व मा॒तरः॑। 
 तस्मा॒ अरं॑ गमाम वो॒ यस्य॒ क्षया॑य॒ जिन्व॑थ। 
 आपो॑ ज॒नय॑था च नः।

\dnsub{अघमर्षणसूक्तम्}
 
हिर॑ण्यशृङ्गं॒ वरु॑णं॒ प्रप॑द्ये ती॒र्थं मे॑ देहि॒ याचि॑तः। 
य॒न्मया॑ भु॒क्तम॒साधू॑नां पा॒पेभ्य॑श्च प्र॒तिग्र॑हः। 
यन्मे॒ मन॑सा वा॒चा॒ क॒र्म॒णा वा दु॑ष्कृतं॒ कृतम्। 
तन्न॒ इन्द्रो॒ वरु॑णो॒ बृह॒स्पतिः॑ सवि॒ता च॑ पुनन्तु॒ पुनः॑ पुनः। 
नमो॒ऽग्नये᳚ऽफ्सु॒मते॒ नम॒ इन्द्रा॑य॒ नमो॒ वरु॑णाय॒ नमो वारुण्यै॑ नमो॒ऽद्भ्यः॥१२॥

 यद॒पां क्रू॒रं यद॑मे॒ध्यं यद॑शा॒न्तं तदप॑गच्छतात्। 
 अ॒त्या॒श॒नाद॑ती\-पा॒ना॒द्य॒च्च उ॒ग्रात् प्र॑ति॒ग्रहा᳚त्। 
 तन्नो॒ वरु॑णो रा॒जा॒ पा॒णिना᳚ ह्यव॒मर्\mbox{}श॑तु। 
 सो॑ऽहम॑पा॒पो वि॒रजो॒ निर्मु॒क्तो मु॑क्तकि॒ल्बिषः। 
 नाक॑स्य पृ॒ष्ठमारु॑ह्य॒ गच्छे॒द्ब्रह्म॑सलो॒कताम्। 
 यश्चा॒फ्सु वरु॑णः॒ स पु॒नात्व॑घमर्\mbox{}ष॒णः। 
 इ॒मं मे॑ गङ्गे यमुने सरस्वति॒ शुतु॑द्रि॒ स्तोमꣳ॑ सचता॒ परु॒ष्णिया। 
 अ॒सि॒क्नि॒या म॑रुद्\mbox{}वृधे वि॒तस्त॒याऽऽर्जी॑कीये शृणु॒ह्या सु॒षोम॑या। 
 ऋ॒तं च॑ स॒त्यं चा॒भी᳚द्धा॒त्तप॒सोऽध्य॑जायत। 
 ततो॒ रात्रि॑रजायत॒ ततः॑ समु॒द्रो अ॑र्ण॒वः॥१३॥
 
 स॒मु॒द्राद॑र्ण॒वादधि॑ संवथ्स॒रो अ॑जायत। 
 अ॒हो॒रा॒त्राणि॑ वि॒दध॒द्विश्व॑स्य मिष॒तो व॒शी। 
 सू॒र्या॒च॒न्द्र॒मसौ॑ धा॒ता य॑थापू॒र्वम॑\-कल्प\-यत्। 
 दिवं॑ च पृथि॒वीं चा॒न्तरि॑क्ष॒मथो॒ सुवः॑। 
 यत्पृ॑थि॒व्याꣳ रजः॑ स्व॒मान्तरि॑क्षे वि॒रोद॑सी। 
 इ॒माꣴस्तदा॒पो व॑रुणः पु॒नात्व॑घमर्\mbox{}ष॒णः। 
 पु॒नन्तु॒ वस॑वः पु॒नातु॒ वरु॑णः पु॒नात्व॑घमर्\mbox{}ष॒णः। 
 ए॒ष भू॒तस्य॑ म॒ध्ये भुव॑नस्य गो॒प्ता। 
 ए॒ष पु॒ण्यकृ॑तां लो॒का॒ने॒ष मृ॒त्योर्\mbox{}हि॑र॒ण्मयम्᳚। 
 द्यावा॑पृथि॒व्योर्\mbox{}हि॑र॒ण्मय॒ꣳ॒ सꣴश्रि॑त॒ꣳ॒ सुवः॑॥१४॥

%६.१.१०
स नः॒ सुवः॒ सꣳशि॑शाधि। 
आर्द्रं॒ ज्वल॑ति॒ ज्योति॑र॒हम॑स्मि। 
ज्योति॒र्ज्वल॑ति॒ ब्रह्मा॒हम॑स्मि। 
यो॑ऽहम॑स्मि॒ ब्रह्मा॒हम॑स्मि। 
अ॒हम॑स्मि॒ ब्रहा॒हम॑स्मि। 
अ॒हमे॒वाहं मां जु॑होमि॒ स्वाहा᳚। 
अ॒का॒र्य॒का॒र्य॑वकी॒र्णी स्ते॒नो भ्रू॑ण॒हा गु॑रुत॒ल्पगः। 
वरु॑णो॒ऽपाम॑घ\-मर्\mbox{}ष॒णस्तस्मा᳚त्पा॒पात् प्रमु॑च्यते। 
र॒जो भूमि॑स्त्व॒माꣳ रोद॑यस्व॒ प्रव॑दन्ति॒ धीराः᳚। 
आक्रा᳚न्थ्समु॒द्रः प्र॑थ॒मे विध॑र्मं ज॒नय॑न्प्र॒जा भुव॑नस्य॒ राजा᳚। 
वृषा॑ प॒वित्रे॒ अधि॒ सानो॒ अव्ये॑ बृ॒हथ्सोमो॑ वावृधे सुवा॒न इन्दुः॑॥१५॥
\anuvakamend

\dnsub{दुर्गासूक्तम्}
 जा॒तवे॑दसे सुनवाम॒ सोम॑मरातीय॒तो निद॑हाति॒ वेदः॑। 
 स नः॑ पर्\mbox{}ष॒दति॑ दु॒र्गाणि॒ विश्वा॑ ना॒वेव॒ सिन्धुं॑ दुरि॒ताऽत्य॒ग्निः। 
 ताम॒ग्निव॑र्णां॒ तप॑सा ज्वल॒न्तीं॒ वै॑रोच॒नीं क॑र्मफ॒लेषु॒ जुष्टा᳚म्। 
 दु॒र्गां दे॒वीꣳ शर॑णम॒हं प्रप॑द्ये सु॒तर॑सि तरसे॒ नमः॑। 
 अग्ने॒ त्वं पा॑रया॒ नव्यो॑ अ॒स्मान्थ्स्व॒स्तिभि॒रति॑ दु॒र्गाणि॒ विश्वा᳚। 
 पूश्च॑ पृ॒थ्वी ब॑हु॒ला न॑ उ॒र्वी भवा॑ तो॒काय॒ तन॑याय॒ शं योः। 
 विश्वा॑नि नो दु॒र्गहा॑ जातवेदः॒ सिन्धुं॒ न ना॒वा दु॑रि॒ताति॑ पर्\mbox{}षि। 
 अग्ने॑ अत्रि॒वन्मन॑सा गृणा॒नो᳚ऽस्माकं॑ बोध्यवि॒ता त॒नूना᳚म्। 
 पृ॒त॒ना॒जित॒ꣳ॒ सह॑मानम॒ग्निमु॒ग्रꣳ हु॑वेम पर॒माथ्स॒धस्था᳚त्। 
 स नः॑ पर्\mbox{}ष॒दति॑ दु॒र्गाणि॒ विश्वा॒ क्षाम॑द्दे॒वो अति॑ दुरि॒ताऽत्य॒ग्निः। 
 प्र॒त्नोषि॑ क॒मीड्यो॑ अध्व॒रेषु॑ स॒नाच्च॒ होता॒ नव्य॑श्च॒ सथ्सि॑। 
 स्वाञ्चा᳚ग्ने त॒नुवं॑ पि॒प्रय॑स्वा॒स्मभ्यं॑ च॒ सौभ॑ग॒माय॑जस्व। 
 गोभि॒र्जुष्ट॑म॒युजो॒ निषि॑क्तं॒ तवे᳚न्द्र विष्णो॒रनु॒सञ्च॑रेम। 
 नाक॑स्य पृ॒ष्ठम॒भि सं॒वसा॑नो॒ वैष्ण॑वीं लो॒क इ॒ह मा॑दयन्ताम्॥१६॥
\anuvakamend%[पर॑स्ता॒द्यशो॒ गुहा॑सु॒ मम॑ सुवर्णप॒क्षाय॑ धीमहि शतबा॒हुना पुनः॑ पुनरजायत॒ सुवो॒ राजा॑ स॒धस्था॒त्त्रीणि॑ च]

\dnsub{व्याहृतिहोमन्त्राः}

भूरन्न॑म॒ग्नये॑ पृथि॒व्यै स्वाहा॒ भुवोऽन्नं॑ वा॒यवे॒ऽन्तरि॑क्षाय॒ स्वाहा॒ सुव॒रन्न॑मादि॒त्याय॑ दि॒वे स्वाहा॒ भूर्भुवः॒ सुव॒रन्नं॑ च॒न्द्रम॑से दि॒ग्भ्यः स्वाहा॒ नमो॑ दे॒वेभ्यः॑ स्व॒धा पि॒तृभ्यो॒ भूर्भुवः॒ सुव॒रन्न॒मोम्॥१७॥
\anuvakamend

भूर॒ग्नये॑ पृथि॒व्यै स्वाहा॒ भुवो॑ वा॒यवे॒ऽन्तरि॑क्षाय॒ स्वाहा॒ सुव॑रादि॒त्याय॑ दि॒वे स्वाहा॒ भूर्भुवः॒ सुव॑श्च॒न्द्रम॑से दि॒ग्भ्यः स्वाहा॒ नमो॑ दे॒वेभ्यः॑ स्व॒धा पि॒तृभ्यो॒ भूर्भुवः॒ सुव॒रग्न॒ ओम्॥१८॥
\anuvakamend

भूर॒ग्नये॑ च पृथि॒व्यै च॑ मह॒ते च॒ स्वाहा॒ भुवो॑ वा॒यवे॑ चा॒न्तरि॑क्षाय च मह॒ते च॒ स्वाहा॒ सुव॑रादि॒त्याय॑ च दि॒वे च॑ मह॒ते च॒ स्वाहा॒ भूर्भुवः॒ सुव॑श्च॒न्द्रम॑से च॒ नक्ष॑त्रेभ्यश्च दि॒ग्भ्यश्च॑ मह॒ते च॒ स्वाहा॒ नमो॑ दे॒वेभ्यः॑ स्व॒धा पि॒तृभ्यो॒ भूर्भुवः॒ सुव॒र्मह॒रोम्॥१९॥ 
%६.५.०
\anuvakamend

\dnsub{ज्ञानप्राप्त्यर्थहोममन्त्राः}
%६.५.१
पाहि नो अग्न एन॑से स्वा॒हा। 
पाहि नो विश्ववेद॑से स्वा॒हा। 
यज्ञं पाहि विभाव॑सो स्वा॒हा। 
सर्वं पाहि शतक्र॑तो स्वा॒हा॥२०॥
%६.६.०
\anuvakamend

पा॒हि नो॑ अग्न॒ एक॑या। 
पा॒ह्यु॑त द्वि॒तीय॑या। 
पा॒ह्यूर्जं॑ तृ॒तीय॑या। 
पा॒हि गी॒र्भिश्च॑ त॒सृभि॑र्वसो॒ स्वाहा᳚॥२१॥
\anuvakamend

\dnsub{वेदविस्मरणाय जपमन्त्राः}
यश्छन्द॑सामृष॒भो वि॒श्वरू॑प॒श्छन्दो᳚भ्य॒श्छन्दाꣴ॑स्यावि॒वेश॑। 
सताꣳ शिक्यः पुरोवाचो॑पनि॒षदिन्द्रो᳚ ज्ये॒ष्ठ इ॑न्द्रि॒याय॒ ऋषि॑भ्यो॒ नमो॑ दे॒वेभ्यः॑ स्व॒धा पि॒तृभ्यो॒ भूर्भुवः॒ सुव॒श्छन्द॒ ओम्॥२२॥
%६.७.०
\anuvakamend


%६.७.१
नमो॒ ब्रह्म॑णे धा॒रणं॑ मे अ॒स्त्वनि॑राकरणं धा॒रयि॑ता भूयासं॒ कर्ण॑योः श्रु॒तं मा च्यो᳚ढ्वं॒ ममा॒मुष्य॒ ओम्॥२३॥
%६.८.०
\anuvakamend

\dnsub{तपः प्रशंसा}
%६.८.१
ऋ॒तं तपः॑ स॒त्यं तपः॑ श्रु॒तं तपः॑ शा॒न्तं तपो॒ दम॒स्तपः॒ शम॒स्तपो॒ दानं॒ तपो॒ यज्ञं॒ तपो॒ भूर्भुवः॒ सुव॒र्ब्रह्मै॒तदुपा᳚स्यै॒तत्तपः॑॥२४॥
\anuvakamend


\dnsub{विहिताचरणप्रशंसा निषिद्धाचरणनिन्दा च}
यथा॑ वृ॒क्षस्य॑ स॒म्पुष्पि॑तस्य दू॒राद्ग॒न्धो वा᳚त्ये॒वं पुण्य॑स्य क॒र्मणो॑ दू॒राद्ग॒न्धो वा॑ति॒ यथा॑ऽसिधा॒रां क॒र्तेऽव॑हितामव॒क्रामे॒ यद्युवे॒ युवे॒ ह वा॑ वि॒ह्वयि॑ष्यामि क॒र्तं प॑तिष्या॒मीत्ये॒वम॒नृता॑दा॒त्मानं॑ जु॒गुफ्से᳚त्॥२५॥
\anuvakamend


\dnsub{दहरविद्या}
अ॒णोरणी॑यान्मह॒तो मही॑याना॒त्मा गुहा॑यां॒ निहि॑तोऽस्य ज॒न्तोः। 
तम॑क्रतुं पश्यति वीतशो॒को धा॒तुः प्र॒सादा᳚न्महि॒मान॑\-मीशम्। 
स॒प्त प्रा॒णाः प्र॒भव॑न्ति॒ तस्मा᳚थ्स॒प्तार्चिषः॑ स॒मिधः॑ स॒प्त जि॒ह्वाः। 
स॒प्त इ॒मे लो॒का येषु॒ चर॑न्ति प्रा॒णा गु॒हाश॑यां॒ निहि॑ताः स॒प्त स॑प्त। 
अतः॑ समु॒द्रा गि॒रय॑श्च॒ सर्वे॒ऽस्माथ्स्यन्द॑न्ते॒ सिन्ध॑वः॒ सर्व॑रूपाः। 
अत॑श्च॒ विश्वा॒ ओष॑धयो॒ रसा᳚च्च॒ येनै॑ष भू॒तस्ति॑ष्ठत्यन्तरा॒त्मा। 
ब्र॒ह्मा दे॒वानां᳚ पद॒वीः क॑वी॒नामृषि॒र्विप्रा॑णां महि॒षो मृ॒गाणा᳚म्। 
श्ये॒नो गृध्रा॑णा॒ꣴ॒ स्वधि॑ति॒र्वना॑ना॒ꣳ॒ सोमः॑ प॒वित्र॒मत्ये॑ति॒ रेभन्। 
अ॒जामेकां॒ लोहि॑तशुक्लकृ॒ष्णां ब॒ह्वीं प्र॒जां ज॒नय॑न्ती॒ꣳ॒ सरू॑पाम्। 
अ॒जो ह्येको॑ जु॒षमा॑णोऽनु॒शेते॒ जहा᳚त्येनां भु॒क्तभो॑गा॒मजो᳚ऽन्यः॥२६॥

%६.१०.२
ह॒ꣳ॒सः शु॑चि॒षद्वसु॑रन्तरिक्ष॒सद्धोता॑ वेदि॒षदति॑थिर्दुरोण॒सत्। 
नृ॒षद्व॑र॒सदृ॑त॒सद्व्यो॑म॒सद॒ब्जा गो॒जा ऋ॑त॒जा अ॑द्रि॒जा ऋ॒तं बृ॒हत्। 
घृ॒तं मि॑मिक्षिरे घृ॒तम॑स्य॒ योनि॑र्घृ॒ते श्रि॒तो घृ॒तमु॑वस्य॒ धाम॑। 
अ॒नु॒ष्व॒धमाव॑ह मा॒दय॑स्व॒ स्वाहा॑कृतं वृषभ वक्षि ह॒व्यम्। 
स॒मु॒द्रादू॒र्मिर्मधु॑मा॒ꣳ॒ उदा॑रदुपा॒ꣳ॒शुना॒ सम॑मृत॒त्वमा॑नट्। 
घृ॒तस्य॒ नाम॒ गुह्यं॒ यदस्ति॑ जि॒ह्वा दे॒वाना॑म॒मृत॑स्य॒ नाभिः॑। 
व॒यं नाम॒ प्रब्र॑वामा घृ॒तेना॒स्मिन् य॒ज्ञे धा॑रयामा॒ नमो॑भिः। 
उप॑ ब्र॒ह्मा शृ॑णवच्छ॒स्यमा॑नं॒ चतुः॑ शृङ्गोऽवमीद्गौ॒र ए॒तत्। 
च॒त्वारि॒ शृङ्गा॒ त्रयो॑ अस्य॒ पादा॒ द्वे शी॒र्\mbox{}षे स॒प्त हस्ता॑सो अ॒स्य। 
त्रिधा॑ ब॒द्धो वृ॑ष॒भो रो॑रवीति म॒हो दे॒वो मर्त्या॒ꣳ॒ आवि॑वेश॥२७॥

त्रिधा॑ हि॒तं प॒णिभि॑र्गु॒ह्यमा॑नं॒ गवि॑ दे॒वासो॑ घृ॒तमन्व॑विन्दन्। 
इन्द्र॒ एक॒ꣳ॒ सूर्य॒ एकं॑ जजान वे॒नादेकꣴ॑ स्व॒धया॒ निष्ट॑तक्षुः। 
यो दे॒वानां᳚ प्रथ॒मं पु॒रस्ता॒द्विश्वा॒धियो॑ रु॒द्रो म॒हर्\mbox{}षिः॑। 
हि॒र॒ण्य॒ग॒र्भं प॑श्यत॒ जाय॑मान॒ꣳ॒ स नो॑ दे॒वः शु॒भया॒ स्मृत्या॒ संयु॑नक्तु। 
यस्मा॒त्परं॒ नाप॑र॒मस्ति॒ किञ्चि॒द्यस्मा॒न्नाणी॑यो॒ न ज्यायो᳚ऽस्ति॒ कश्चि॑त्। 
वृ॒क्ष इ॑व स्तब्धो दि॒वि ति॑ष्ठ॒त्येक॒स्तेने॒दं पू॒र्णं पुरु॑षेण॒ सर्वम्᳚। 
न कर्म॑णा न प्र॒जया॒ धने॑न॒ त्यागे॑नैके अमृत॒त्वमा॑न॒शुः। 
परे॑ण॒ नाकं॒ निहि॑तं॒ गुहा॑यां वि॒भ्राज॑ते॒ यद्यत॑यो वि॒शन्ति॑। 
वे॒दा॒न्त॒वि॒ज्ञान॒सुनि॑श्चिता॒र्थाः सन्न्या॑सयो॒गाद्यत॑यः शुद्ध॒सत्त्वाः᳚। 
ते ब्र॑ह्मलो॒के तु॒ परा᳚न्तकाले॒ परा॑मृता॒त्परि॑मुच्यन्ति॒ सर्वे᳚। 
द॒ह्रं॒ वि॒पा॒पं प॒रमे᳚श्मभूतं॒ यत्पु॑ण्डरी॒कं पु॒रम॑ध्यस॒ꣴ॒स्थम्। 
त॒त्रा॒पि॒ द॒ह्रं ग॒गनं॑ विशोक॒स्तस्मि॑न् यद॒न्तस्तदुपा॑सित॒व्यम्। 
यो वेदादौ स्व॑रः प्रो॒क्तो॒ वे॒दान्ते॑ च प्र॒तिष्ठि॑तः। 
तस्य॑ प्र॒कृति॑लीन॒स्य॒ यः॒ परः॑ स म॒हेश्व॑रः॥२८॥
\anuvakamend

\dnsub{नारायणसूक्तम्}
स॒ह॒स्र॒शीर्‌षं दे॒वं॒ वि॒श्वाक्षं॑ वि॒श्वश॑म्भुवम्। 
विश्वं॑ ना॒राय॑णं दे॒व॒म॒क्षरं॑ पर॒मं प॒दम्। 
वि॒श्वतः॒ पर॑मान्नि॒त्यं॒ वि॒श्वं ना॑राय॒णꣳ ह॑रिम्। 
विश्व॑मे॒वेदं पुरु॑ष॒स्तद्विश्व॒मुप॑जीवति। 
पतिं॒ विश्व॑स्या॒ऽ॒ऽ॒त्मेश्व॑र॒ꣳ॒ शाश्व॑तꣳ शि॒वम॑च्युतम्। 
ना॒राय॒णं म॑हाज्ञे॒यं॒ वि॒श्वात्मा॑नं प॒राय॑णम्। 
ना॒राय॒णप॑रो ज्यो॒ति॒रा॒त्मा ना॑राय॒णः प॑रः। 
ना॒राय॒ण प॑रं ब्र॒ह्म॒ त॒त्त्वं ना॑राय॒णः प॑रः। 
ना॒राय॒णप॑रो ध्या॒ता॒ ध्या॒नं ना॑राय॒णः प॑रः। 
यच्च॑ कि॒ञ्चिज्ज॑गथ्स॒र्वं॒ दृ॒श्यते᳚ श्रूय॒तेऽपि॑ वा॥ 
अन्त॑र्ब॒हिश्च॑ तथ्स॒र्वं॒ व्या॒प्य ना॑राय॒णः स्थि॑तः॥२९॥

अन॑न्त॒मव्य॑यं क॒विꣳ स॑मु॒द्रेऽन्तं॑ वि॒श्वश॑म्भुवम्। 
प॒द्म॒को॒श प्र॑तीका॒श॒ꣳ॒ हृ॒दयं॑ चाप्य॒धोमु॑खम्। 
अधो॑ नि॒ष्ट्या वि॑तस्त्या॒न्ते॒ ना॒भ्यामु॑परि॒ तिष्ठ॑ति। 
ज्वा॒ल॒मा॒लाकु॑लं भा॒ती॒ वि॒श्वस्या॑ऽऽयत॒नं म॑हत्। 
सन्त॑तꣳ शि॒लाभि॑स्तु॒\-लम्ब॑त्याकोश॒सन्नि॑भम्। 
तस्यान्ते॑ सुषि॒रꣳ सू॒क्ष्मं तस्मि᳚न्थ्स॒र्वं प्रति॑ष्ठितम्। 
तस्य॒ मध्ये॑ म॒हान॑\-ग्निर्वि॒श्वार्चि॑र्वि॒श्वतो॑मुखः। 
सोऽग्र॑भु॒ग्विभ॑जन्ति॒ष्ठ॒न्नाहा॑रमज॒रः क॒विः। 
ति॒र्य॒गू॒र्ध्वम॑धः शा॒यी॒ र॒श्मय॑स्तस्य॒ सन्त॑ता। 
स॒न्ता॒पय॑ति स्वं दे॒हमा\-पा॑द\-तल॒\-मस्त॑कः। 
तस्य॒ मध्ये॒ वह्नि॑शिखा अ॒णीयो᳚र्ध्वा व्य॒वस्थि॑तः। 
नी॒लतो॑यद॑\-मध्य॒स्था॒द्वि॒द्युल्ले॑खेव॒ भास्व॑रा। 
नी॒वार॒शूक॑\-वत्त॒न्वी॒ पी॒ता भा᳚स्वत्य॒णूप॑मा। 
तस्याः᳚ शिखा॒या म॑ध्ये प॒रमा᳚त्मा व्य॒वस्थि॑तः। 
स ब्रह्म॒ स शिवः॒ स हरिः॒ सेन्द्रः॒ सोऽक्ष॑रः पर॒मः स्व॒राट्॥३०॥
\anuvakamend[ना॒रा॒य॒णः स्थि॑तो व्य॒वस्थि॑तश्च॒त्वारि॑ च]

\dnsub{आदित्यमण्डले परब्रह्मोपासनम्}
आ॒दि॒त्यो वा ए॒ष ए॒तन्म॒ण्डलं॒ तप॑ति॒ तत्र॒ ता ऋच॒स्तदृ॒चा म॑ण्डल॒ꣳ॒ स ऋ॒चां लो॒कोऽथ॒ य ए॒ष ए॒तस्मि॑न्म॒ण्डले॒ऽर्चिर्दी॒प्यते॒ तानि॒ सामा॑नि॒ स सा॒म्नां म॒ण्डल॒ꣳ॒ स सा॒म्नां लो॒कोऽथ॒ य ए॒ष ए॒तस्मि॑न्म॒ण्डले॒ऽर्चिषि॒ पुरु॑ष॒स्तानि॒ यजूꣳ॑षि॒ स यजु॑षा मण्डल॒ꣳ॒ स यजु॑षां लो॒कः सैषा त्र॒य्येव॑ वि॒द्या त॑पति॒ य ए॒षो᳚ऽन्तरा॑दि॒त्ये हि॑र॒ण्मयः॒ पुरु॑षः॥३१॥
%६.१४.०
\anuvakamend


\dnsub{आदित्यपुरुषस्य सर्वात्मकत्वप्रदर्शनम्}
आ॒दि॒त्यो वै तेज॒ ओजो॒ बलं॒ यश॒श्चक्षुः॒ श्रोत्र॑मा॒त्मा मनो॑ म॒न्युर्मनु॑र्मृ॒त्युः स॒त्यो मि॒त्रो वा॒युरा॑का॒शः प्रा॒णो लो॑कपा॒लः कः किं कं तथ्स॒त्यमन्न॑म॒मृतो॑ जी॒वो विश्वः॑ कत॒मः स्व॑य॒म्भु ब्रह्मै॒तदमृ॑त ए॒ष पुरु॑ष ए॒ष भू॒ताना॒मधि॑पति॒र्ब्रह्म॑णः॒ सायु॑ज्यꣳ सलो॒कता॑माप्नोत्ये॒तासा॑मे॒व दे॒वता॑ना॒ꣳ॒ सायु॑ज्यꣳ सा॒र्ष्टिताꣳ॑ समानलो॒कता॑माप्नोति॒ य ए॒वं वेदे᳚त्युप॒निषत्॥३२॥
%६.१५.०
\anuvakamend

\dnsub{शिवोपासनमन्त्राः}
निध॑नपतये॒ नमः। 
निध॑नपतान्तिकाय॒ नमः। 
ऊर्ध्वाय॒ नमः। 
ऊर्ध्वलिङ्गाय॒ नमः। 
हिरण्याय॒ नमः। 
हिरण्यलिङ्गाय॒ नमः। 
सुवर्णाय॒ नमः। 
सुवर्णलिङ्गाय॒ नमः। 
दिव्याय॒ नमः। 
दिव्यलिङ्गाय॒ नमः। 
भवाय॒ नमः। 
भवलिङ्गाय॒ नमः। 
शर्वाय॒ नमः। 
शर्वलिङ्गाय॒ नमः। 
शिवाय॒ नमः। 
शिवलिङ्गाय॒ नमः। 
ज्वलाय॒ नमः। 
ज्वललिङ्गाय॒ नमः। 
आत्माय॒ नमः। 
आत्मलिङ्गाय॒ नमः। 
परमाय॒ नमः। 
परमलिङ्गाय॒ नमः। 
एतथ्सोमस्य॑ सूर्य॒स्य॒ सर्वलिङ्गꣴ॑ स्थाप॒य॒ति॒ पाणिमन्त्रं॑ पवि॒त्रम्॥३३॥
\anuvakamend

\dnsub{पश्चिमवक्त्र-प्रतिपादक-मन्त्रः}
स॒द्योजा॒तं प्र॑पद्या॒मि॒ स॒द्योजा॒ताय॒ वै नमो॒ नमः॑। 
भ॒वे भ॑वे॒ नाति॑ भवे भवस्व॒ माम्। 
भ॒वोद्भ॑वाय॒ नमः॑॥३४॥
\anuvakamend

\dnsub{उत्तरवक्त्र-प्रतिपादक-मन्त्रः}
वा॒म॒दे॒वाय॒ नमो᳚ ज्ये॒ष्ठाय॒ नमः॑ श्रे॒ष्ठाय॒ नमो॑ रु॒द्राय॒ नमः॒ काला॑य॒ नमः॒ कल॑विकरणाय॒ नमो॒ बल॑विकरणाय॒ नमो॒ बला॑य॒ नमो॒ बल॑प्रमथनाय॒ नमः॒ सर्व॑भूतदमनाय॒ नमो॑ म॒नोन्म॑नाय॒ नमः॑॥३५॥\anuvakamend

\dnsub{दक्षिणवक्त्र-प्रतिपादक-मन्त्रः}
अ॒घोरे᳚भ्योऽथ॒ घोरे᳚भ्यो॒ घोर॒घोर॑तरेभ्यः। 
सर्वे᳚भ्यः सर्व॒शर्वे᳚भ्यो॒ नम॑स्ते अस्तु रु॒द्ररू॑पेभ्यः॥३६॥
\anuvakamend

\dnsub{प्राग्वक्त्र-प्रतिपादक-मन्त्रः}
तत्पुरु॑षाय वि॒द्महे॑ महादे॒वाय॑ धीमहि। 
तन्नो॑ रुद्रः प्रचो॒दया᳚त्॥३७॥
\anuvakamend

\dnsub{ऊर्ध्ववक्त्र-प्रतिपादक-मन्त्रः}
ईशानः सर्व॑विद्या॒ना॒मीश्वरः सर्व॑भूता॒नां॒ ब्रह्माधि॑पति॒र्ब्रह्म॒णो\-ऽधि॑पति॒र्ब्रह्मा॑ शि॒वो मे॑ अस्तु सदाशि॒वोम्॥३८॥
\anuvakamend

\dnsub{नमस्कारमन्त्राः}
नमो हिरण्यबाहवे हिरण्यवर्णाय हिरण्यरूपाय हिरण्यपतये\-ऽम्बिकापतय उमापतये पशुपतये॑ नमो॒ नमः॥३९॥
\anuvakamend

ऋ॒तꣳ स॒त्यं प॑रं ब्र॒ह्म॒ पु॒रुषं॑ कृष्ण॒पिङ्ग॑लम्। 
ऊ॒र्ध्वरे॑तं वि॑रूपा॒क्षं॒ वि॒श्वरू॑पाय॒ वै नमो॒ नमः॑॥४०॥
\anuvakamend

सर्वो॒ वै रु॒द्रस्तस्मै॑ रु॒द्राय॒ नमो॑ अस्तु। 
पुरु॑षो॒ वै रु॒द्रः सन्म॒हो नमो॒ नमः॑। 
विश्वं॑ भू॒तं भुव॑नं चि॒त्रं ब॑हु॒धा जा॒तं जाय॑मानं च॒ यत्। 
सर्वो॒ ह्ये॑ष रु॒द्रस्तस्मै॑ रु॒द्राय॒ नमो॑ अस्तु॥४१॥
\anuvakamend


%६.१७.१
कद्रु॒द्राय॒ प्रचे॑तसे मी॒ढुष्ट॑माय॒ तव्य॑से। 
वो॒ चेम॒ शन्त॑मꣳ हृ॒दे। 
सर्वो॒ ह्ये॑ष रु॒द्रस्तस्मै॑ रु॒द्राय॒ नमो॑ अस्तु॥४२॥
\anuvakamend

\dnsub{अग्निहोत्रहवण्याः उपयुक्तस्य वृक्षविशेषस्याभिधानम्}
%६.१९.१
यस्य॒ वैक॑ङ्कत्यग्निहोत्र॒हव॑णी भवति॒ प्रत्ये॒वास्याऽऽहु॑तय\-स्तिष्ठ॒न्त्यथो॒ प्रति॑ष्ठित्यै॥४३॥
\anuvakamend


%६.२०.१
कृ॒णु॒ष्व पाज॒ इति॒ पञ्च॑॥४४॥
%६.२१.०
\anuvakamend

\dnsub{भूदेवताकमन्त्रः}
%६.२१.१
अदि॑तिर्दे॒वा ग॑न्ध॒र्वा म॑नु॒ष्याः᳚ पि॒तरोऽसु॑रा॒स्तेषाꣳ॑ सर्वभू॒तानां᳚ मा॒ता मे॒दिनी॑ मह॒ता म॒ही सा॑वि॒त्री गा॑य॒त्री जग॑त्यु॒र्वी पृ॒थ्वी ब॑हु॒ला विश्वा॑ भू॒ता क॑त॒मा का या सा स॒त्येत्य॒मृतेति॑ वसि॒ष्ठः॥४५॥
%६.२२.०
\anuvakamend

\dnsub{सर्वदेवता आपः}
%६.२२.१
आपो॒ वा इ॒दꣳ सर्वं॒ विश्वा॑ भू॒तान्यापः॑ प्रा॒णा वा आपः॑ प॒शव॒ आपोऽन्न॒मापोऽमृ॑त॒मापः॑ स॒म्राडापो॑ वि॒राडापः॑ स्व॒राडाप॒श्छन्दा॒ꣴ॒स्यापो॒ ज्योती॒ꣴ॒ष्यापो॒ यजू॒ꣴ॒ष्यापः॑ स॒त्यमापः॒ सर्वा॑ दे॒वता॒ आपो॒ भूर्भुवः॒ सुव॒राप॒ ओम्॥४६॥
%६.२३.०
\anuvakamend

\dnsub{सन्ध्यावन्दनमन्त्राः}
%६.२३.१
आपः॑ पुनन्तु पृथि॒वीं पृ॑थि॒वी पू॒ता पु॑नातु॒ माम्। 
पु॒नन्तु॒ ब्रह्म॑ण॒स्पति॒र्ब्रह्म॑पू॒ता पु॑नातु॒ माम्। 
यदुच्छि॑ष्ट॒मभो᳚ज्यं॒ यद्वा॑ दु॒श्चरि॑तं॒ मम॑। 
सर्वं॑ पुनन्तु॒ मामापो॑ऽस॒तां च॑ प्रति॒ग्रह॒ꣴ॒ स्वाहा᳚॥४७॥
%६.२४.०
\anuvakamend


%६.२४.१
अग्निश्च मा मन्युश्च मन्युपतयश्च मन्यु॑कृते॒भ्यः। 
पापेभ्यो॑ रक्ष॒न्ताम्। 
यदह्ना पाप॑मका॒रिषम्। 
मनसा वाचा॑ हस्ता॒भ्याम्। 
पद्भ्यामुदरे॑ण शि॒श्ञा। 
अह॒स्तद॑वलु॒म्पतु। 
यत्किं च॑ दुरि॒तं मयि॑। 
इदमहं माममृ॑तयो॒नौ। 
सत्ये ज्योतिषि जुहो॑मि स्वा॒हा॥४८॥
%६.२५.०
\anuvakamend


%६.२५.१
सूर्यश्च मा मन्युश्च मन्युपतयश्च मन्यु॑कृते॒भ्यः। 
पापेभ्यो॑ रक्ष॒न्ताम्। 
यद्रात्रिया पाप॑मका॒रिषम्। 
मनसा वाचा॑ हस्ता॒भ्याम्। 
पद्भ्यामुदरे॑ण शि॒श्ञा। 
रात्रि॒स्तद॑वलु॒म्पतु। 
यत्किं च॑ दुरि॒तं मयि॑। 
इदमहं माममृ॑तयो॒नौ। 
सूर्ये ज्योतिषि जुहो॑मि स्वा॒हा॥४९॥
\anuvakamend

\dnsub{प्रणवस्य ऋष्यादिविवरणम्}
ओमित्येकाक्ष॑रं ब्र॒ह्म। 
अग्निर्देवता ब्रह्म॑ इत्या॒र्\mbox{}षम्। 
गायत्रं छन्दं परमात्मं॑ सरू॒पम्। 
सायुज्यं वि॑नियो॒गम्॥५०॥
\anuvakamend

\dnsub{गायत्र्यावाहनमन्त्राः}
%६.२६.१
आया॑तु॒ वर॑दा दे॒वी॒ अ॒क्षरं॑ ब्रह्म॒सम्मि॑तम्। 
गा॒य॒त्रीं᳚ छन्द॑सां मा॒तेदं ब्र॑ह्म जु॒षस्व॑ मे। 
यदह्ना᳚त्कुरु॑ते पा॒पं॒ तदह्ना᳚त्प्रति॒मुच्य॑ते। 
यद्रात्रिया᳚त्कुरु॑ते पा॒पं॒ तद्रात्रिया᳚त्प्रति॒मुच्य॑ते। 
सर्व॑ व॒र्णे म॑हादे॒वि॒ स॒न्ध्यावि॑द्ये स॒रस्व॑ति॥५१॥ 
\anuvakamend

ओजो॑ऽसि॒ सहो॑ऽसि॒ बल॑मसि॒ भ्राजो॑ऽसि दे॒वानां॒ धाम॒ नामा॑सि॒ विश्व॑मसि वि॒श्वायुः॒ सर्व॑मसि स॒र्वायुरभिभूरों गायत्रीमावा॑हया॒मि॒ सावित्रीमावा॑हया॒मि॒ सरस्वतीमावा॑ह\-या॒मि॒ छन्दऋषीनावा॑हया॒मि॒ श्रियमावा॑हया॒मि॒ गायत्रिया गायत्रीच्छन्दो विश्वामित्र ऋषिः सविता देवताऽग्निर्मुखं ब्रह्मा शिरो विष्णुर्\mbox{}हृदयꣳ रुद्रः शिखा पृथिवी योनिः प्राणापानव्यानोदानसमाना सप्राणा श्वेतवर्णा साङ्ख्यायनसगोत्रा गायत्री चतुर्विꣳशत्यक्षरा त्रिपदा॑ षट्कु॒क्षिः॒ पञ्चशीर्\mbox{}षोपनयने वि॑नियो॒ग॒ ओं भूः। 
ओं भुव। 
ओꣳ सुव। 
ओं मह। 
ओं जन। 
ओं तप। 
ओꣳ स॒त्यम्। 
ओं तथ्स॑वि॒तुर्वरे᳚ण्यं॒ भर्गो॑ दे॒वस्य॑ धीमहि। 
धियो॒ यो नः॑ प्रचो॒दया᳚त्। 
ओमापो॒ ज्योती॒रसो॒ऽमृतं॒ ब्रह्म॒ भूर्भुवः॒ सुव॒रोम्॥५२॥
\anuvakamend

\dnsub{गायत्री उपस्थानमन्त्राः}
उ॒त्तमे॑ शिख॑रे जा॒ते॒ भू॒म्यां प॑र्वत॒मूर्ध॑नि। 
ब्रा॒ह्म॒णे᳚भ्योऽभ्य॑नु\-ज्ञा॒ता॒ ग॒च्छ दे॑वि य॒थासु॑खम्। 
स्तुतो मया वरदा वे॑दमा॒ता॒ प्रचोदयन्ती पवने᳚ द्विजा॒ता। 
आयुः पृथिव्यां द्रविणं ब्र॑ह्मव॒र्च॒सं॒ मह्यं दत्त्वा प्रजातुं ब्र॑ह्मलो॒कम्॥५३॥
\anuvakamend

\dnsub{आदित्यदेवतामन्त्रः}
घृणिः सूर्य॑ आदि॒त्यो न प्रभा॑ वा॒त्यक्ष॑रम्। 
मधु॑ क्षरन्ति॒ तद्र॑सम्। 
स॒त्यं वै तद्रस॒मापो॒ ज्योती॒रसो॒ऽमृतं॒ ब्रह्म॒ भूर्भुवः॒ सुव॒रोम्॥५४॥\anuvakamend

\dnsub{त्रिसुपर्णमन्त्राः}
ब्रह्म॑मेतु॒ माम्। 
मधु॑मेतु॒ माम्। 
ब्रह्म॑मे॒व मधु॑मेतु॒ माम्। 
यास्ते॑ सोम प्र॒जाव॒थ्सोभि॒ सो अ॒हम्। 
दुःस्व॑प्न॒हन्दु॑रुष्षह। 
यास्ते॑ सोम प्रा॒णाꣴस्तां जु॑होमि। 
त्रिसु॑पर्ण॒मया॑चितं ब्राह्म॒णाय॑ दद्यात्। 
ब्र॒ह्म॒ह॒त्यां वा ए॒ते घ्न॑न्ति। 
ये ब्रा᳚ह्म॒णास्त्रिसु॑पर्णं॒ पठ॑न्ति। 
ते सोमं॒ प्राप्नु॑वन्ति। 
आ॒स॒ह॒स्रात्प॒ङ्क्तिं पुन॑न्ति। 
ओम्॥५५॥
\anuvakamend

ब्रह्म॑ मे॒धया᳚। 
मधु॑ मे॒धया᳚। 
ब्रह्म॑मे॒व मधु॑ मे॒धया᳚। 
अ॒द्या नो॑ देव सवितः प्र॒जाव॑थ्सावीः॒ सौभ॑गम्। 
परा॑ दुः॒ष्वप्नि॑यꣳ सुव। 
विश्वा॑नि देव सवितर्दुरि॒तानि॒ परा॑ सुव। 
यद्भ॒द्रं तन्म॒ आ सु॑व। 
मधु॒ वाता॑ ऋताय॒ते मधु॑ क्षरन्ति॒ सिन्ध॑वः। 
माध्वी᳚र्नः स॒न्त्वोष॑धीः। 
मधु॒ नक्त॑मु॒तोषसि॒ मधु॑म॒त्पार्थि॑व॒ꣳ॒ रजः॑। 
मधु॒ द्यौर॑स्तु नः पि॒ता। 
मधु॑मान्नो॒ वन॒स्पति॒र्मधु॑माꣳ अस्तु॒ सूर्यः॑। 
माध्वी॒र्गावो॑ भवन्तु नः। 
य इ॒मं त्रिसु॑पर्ण॒मया॑चितं ब्राह्म॒णाय॑ दद्यात्। 
भ्रू॒ण॒ह॒त्यां वा ए॒ते घ्न॑न्ति। 
ये ब्रा᳚ह्म॒णास्त्रिसु॑पर्णं॒ पठ॑न्ति। 
ते सोमं॒ प्राप्नु॑वन्ति। 
आ॒स॒ह॒स्रात्प॒ङ्क्तिं पुन॑न्ति। 
ओम्॥५६॥
\anuvakamend


%६.५०.१
ब्रह्म॑ मे॒धवा᳚। 
मधु॑ मे॒धवा᳚। 
ब्रह्म॑मे॒व मधु॑ मे॒धवा᳚। 
ब्र॒ह्मा दे॒वानां᳚ पद॒वीः क॑वी॒नामृषि॒र्विप्रा॑णां महि॒षो मृ॒गाणा᳚म्। 
श्ये॒नो गृध्रा॑णा॒ꣴ॒ स्वधि॑ति॒र्वना॑ना॒ꣳ॒ सोमः॑ प॒वित्र॒मत्ये॑ति॒ रेभन्। 
ह॒ꣳ॒सः शु॑चि॒षद्वसु॑रन्तरिक्ष॒सद्धोता॑ वेदि॒षदति॑थिर्दुरोण॒सत्। 
नृ॒षद्व॑र॒सदृ॑त॒सद्व्यो॑म॒सद॒ब्जा गो॒जा ऋ॑त॒जा अ॑द्रि॒जा ऋ॒तं बृ॒हत्। 
ऋ॒चे त्वा॑ रु॒चे त्वा॒ समिथ्स्र॑वन्ति स॒रितो॒ न धेनाः᳚। 
अ॒न्तर्\mbox{}हृ॒दा मन॑सा पू॒यमा॑नाः। 
घृ॒तस्य॒ धारा॑ अ॒भिचा॑कशीमि। 
हि॒र॒ण्ययो॑ वेत॒सो मध्य॑ आसाम्। 
\mbox{तस्मि ᳚\hspace{-1.25ex}न्थ्सु}\-प॒र्णो म॑धु॒कृत्कु॑ला॒यी भज॑न्नास्ते॒ मधु॑\-दे॒वता᳚भ्यः। 
तस्या॑ऽऽसते॒ हर॑यः स॒प्ततीरे᳚ स्व॒धां दुहा॑ना अ॒मृत॑स्य॒ धारा᳚म्। 
य इ॒दं त्रिसु॑पर्ण॒मया॑चितं ब्राह्म॒णाय॑ दद्यात्। 
वी॒र॒ह॒त्यां वा ए॒ते घ्न॑न्ति। 
ये ब्रा᳚ह्म॒णास्त्रिसु॑पर्णं॒ पठ॑न्ति। 
ते सोमं॒ प्राप्नु॑वन्ति। 
आ॒स॒ह॒स्रात्प॒ङ्क्तिं पुन॑न्ति। 
ओम्॥५७॥
\anuvakamend

\dnsub{मेधासूक्तम्}
मे॒धा दे॒वी जु॒षमा॑णा न॒ आगा᳚द्वि॒श्वाची॑ भ॒द्रा सु॑मन॒स्यमा॑ना। 
त्वया॒ जुष्टा॑ जु॒षमा॑णा दु॒रुक्ता᳚न्बृ॒हद्व॑देम वि॒दथे॑ सु॒वीराः᳚॥ 
त्वया॒ जुष्ट॑ ऋ॒षिर्भ॑वति देवि॒ त्वया॒ ब्रह्मा॑ऽऽग॒तश्री॑रु॒त त्वया᳚। 
त्वया॒ जुष्ट॑श्चि॒त्रं वि॑न्दते वसु॒ सा नो॑ जुषस्व॒ द्रवि॑णो न मेधे॥५८॥
%६.४०.०
\anuvakamend


%६.४०.१
मे॒धां म॒ इन्द्रो॑ ददातु मे॒धां दे॒वी सर॑स्वती। 
मे॒धां मे॑ अ॒श्विना॑वु॒भावाध॑त्तां॒ पुष्क॑रस्रजा। 
अ॒फ्स॒रासु॑ च॒ या मे॒धा ग॑न्ध॒र्वेषु॑ च॒ यन्मनः॑। 
दैवीं᳚ मे॒धा सर॑स्वती॒ सा मां᳚ मे॒धा सु॒रभि॑र्जुषता॒ꣴ॒ स्वाहा᳚॥५९॥
%६.४२.०
\anuvakamend


%६.४२.१
आ मां᳚ मे॒धा सु॒रभि॑र्वि॒श्वरू॑पा॒ हिर॑ण्यवर्णा॒ जग॑ती जग॒म्या। 
ऊर्ज॑स्वती॒ पय॑सा॒ पिन्व॑माना॒ सा मां᳚ मे॒धा सु॒प्रती॑का जुषन्ताम्॥६०॥ 
\anuvakamend

मयि॑ मे॒धां मयि॑ प्र॒जां मय्य॒ग्निस्तेजो॑ दधातु॒ मयि॑ मे॒धां मयि॑ प्र॒जां मयीन्द्र॑ इन्द्रि॒यं द॑धातु॒ मयि॑ मे॒धां मयि॑ प्र॒जां मयि॒ सूर्यो॒ भ्राजो॑ दधातु॥६१॥ 
\anuvakamend

\dnsub{मृत्युनिवारणमन्त्राः}

अपै॑तु मृ॒त्युर॒मृतं॑ न॒ आग॑न्वैवस्व॒तो नो॒ अभ॑यं कृणोतु। 
प॒र्णं वन॒स्पते॑रिवा॒भिनः॑ शीयताꣳ र॒यिः स च॑ तान्नः॒ शची॒पतिः॑॥६२॥%
\anuvakamend

 परं॑ मृत्यो॒ अनु॒ परे॑हि॒ पन्थां॒ यस्ते॒ स्व इत॑रो देव॒याना᳚त्। 
 चक्षु॑ष्मते शृण्व॒ते ते᳚ ब्रवीमि॒ मा नः॑ प्र॒जाꣳ री॑रिषो॒ मोत वी॒रान्॥६३॥%
 \anuvakamend
 
 वातं॑ प्रा॒णं मन॑सा॒ऽन्वा र॑भामहे प्र॒जाप॑तिं॒ यो भुव॑नस्य गो॒पाः। 
 स नो॑ मृ॒त्योस्त्रा॑यतां॒ पात्वꣳह॑सो॒ ज्योग्जी॒वा ज॒राम॑शीमहि॥६४॥
 \anuvakamend
 
 अ॒मु॒त्र॒ भूया॒दध॒ यद्य॒मस्य॒ बृह॑स्पते अ॒भिश॑स्ते॒रमु॑ञ्चः। 
 प्रत्यौ॑हताम॒श्विना॑ मृ॒त्युम॑स्माद्दे॒वाना॑मग्ने भि॒षजा॒ शची॑भिः॥६५॥
 \anuvakamend
 
 हरि॒ꣳ॒ हर॑न्त॒मनु॑यन्ति दे॒वा विश्व॒स्येशा॑नं वृष॒भं म॑ती॒नाम्। 
 ब्रह्म॒ सरू॑प॒मनु॑मे॒दमागा॒दय॑नं॒ मा विव॑धी॒र्विक्र॑मस्व॥६६॥
 \anuvakamend
 
 शल्कै॑र॒ग्निमि॑न्धा॒न उ॒भौ लो॒कौ स॑नेम॒हम्। 
 उ॒भयो᳚र्लो॒कयोर्॑-ऋ॒ध्वाऽति॑ मृ॒त्युं त॑राम्य॒हम्॥६७॥
\anuvakamend
 
मा छि॑दो मृत्यो॒ मा व॑धी॒र्मा मे॒ बलं॒ विवृ॑हो॒ मा प्रमो॑षीः। 
प्र॒जां मा मे॑ रीरिष॒ आयु॑रुग्र नृ॒चक्ष॑सं त्वा ह॒विषा॑ विधेम॥६८॥
\anuvakamend
 
मा नो॑ म॒हान्त॑मु॒त मा नो॑ अर्भ॒कं मा न॒ उक्ष॑न्तमु॒त मा न॑ उक्षि॒तम्। 
मा नो॑ऽवधीः पि॒तरं॒ मोत मा॒तरं॑ प्रि॒या मा न॑स्त॒नुवो॑ रुद्र रीरिषः॥६९॥
\anuvakamend

मा न॑स्तो॒के तन॑ये॒ मा न॒ आयु॑षि॒ मा नो॒ गोषु॒ मा नो॒ अश्वे॑षु रीरिषः। 
वी॒रान्मा नो॑ रुद्र भामि॒तोऽव॑धीर्‌ह॒विष्म॑न्तो॒ नम॑सा विधेम ते॥७०॥
\anuvakamend 

\dnsub{प्रजापतिप्रार्थनामन्त्रः}
प्रजा॑पते॒ न त्वदे॒तान्य॒न्यो विश्वा॑ जा॒तानि॒ परि॒ता ब॑भूव। 
यत्का॑मास्ते जुहु॒मस्तन्नो॑ अस्तु व॒यꣴ स्या॑म॒ पत॑यो रयी॒णाम्॥७१॥%
\anuvakamend 

\dnsub{इन्द्रप्रार्थनामन्त्रः}
स्व॒स्ति॒दा वि॒शस्पति॑र्वृत्र॒हा विमृधो॑ व॒शी। 
वृषेन्द्रः॑ पु॒र ए॑तु नः स्वस्ति॒दा अ॑भयङ्क॒रः॥७२॥
\anuvakamend 

\dnsub{मृत्युञ्जयमन्त्राः}
त्र्य॑म्बकं यजामहे सुग॒न्धिं पु॑ष्टि॒वर्ध॑नम्। 
उ॒र्वा॒रु॒कमि॑व॒ बन्ध॑नान्मृ॒त्योर्मु॑क्षीय॒ माऽमृता᳚त्॥७३॥
\anuvakamend 

ये ते॑ स॒हस्र॑म॒युतं॒ पाशा॒ मृत्यो॒ मर्त्या॑य॒ हन्त॑वे। 
तान् य॒ज्ञस्य॑ मा॒यया॒ सर्वा॒नव॑ यजामहे॥७४॥\anuvakamend
 
मृ॒त्यवे॒ स्वाहा॑ मृ॒त्यवे॒ स्वाहा᳚॥७५॥%
\anuvakamend 

\dnsub{पापनिवारक-मन्त्राः}
दे॒वकृ॑त॒स्यैन॑सोऽव॒\-यज॑न\-मसि॒ स्वाहा᳚। 
म॒नु॒ष्य॑कृत॒स्यैन॑सो\-ऽव॒\-यज॑न\-मसि॒ स्वाहा᳚। 
पि॒तृकृ॑त॒स्यैन॑सो\-ऽव॒\-यज॑न\-मसि॒ स्वाहा᳚। 
आ॒त्मकृ॑त॒स्यैन॑सो\-ऽव॒\-यज॑न\-मसि॒ स्वाहा᳚। 
अ॒न्यकृ॑त॒स्यैन॑सो\-ऽव॒\-यज॑न\-मसि॒ स्वाहा᳚। 
अ॒स्मत्कृ॑त॒स्यैन॑सो\-ऽव॒\-यज॑न\-मसि॒ स्वाहा᳚। 
यद्दि॒वा च॒ नक्तं॒ चैन॑श्चकृ॒म तस्या॑व॒यज॑नमसि॒ स्वाहा᳚। 
यथ्स्व॒पन्त॑श्च॒ जाग्र॑त॒श्चैन॑श्चकृ॒म तस्या॑व॒यज॑नमसि॒ स्वाहा᳚। 
यथ्सु॒षुप्त॑श्च॒ जाग्र॑त॒श्चैन॑श्चकृ॒म तस्या॑व॒यज॑नमसि॒ स्वाहा᳚। 
यद्वि॒द्वाꣳस॒\-श्चा\-वि॑द्वाꣳस॒श्चैन॑श्चकृ॒म तस्या॑व॒यज॑नमसि॒ स्वाहा᳚। 
एनस एनसोऽवयजनम॑सि स्वा॒हा॥७६॥\anuvakamend

\dnsub{वसुप्रार्थनामन्त्रः}
यद्वो॑ देवाश्चकृ॒म जि॒ह्वया॑ गु॒रुमन॑सो वा॒ प्रयु॑ती देव॒ हेड॑नम्। 
अरा॑वा॒ यो नो॑ अ॒भि दु॑च्छुना॒यते॒ तस्मि॒न्तदेनो॑ वसवो॒ निधे॑तन॒ स्वाहा᳚॥७७॥ 
\anuvakamend


\dnsub{कामोऽकार्\mbox{}षीत्-मन्युरकार्\mbox{}षीत् मन्त्रः}

कामोऽकार्\mbox{}षी᳚न्नमो॒ नमः। 
 कामोऽकार्\mbox{}षीत्कामः करोति नाहं करोमि कामः कर्ता नाहं कर्ता कामः॑ कार॒यिता नाहं॑ कार॒यिता एष ते काम कामा॑य स्वा॒हा॥७८॥
\anuvakamend

मन्युरकार्\mbox{}षी᳚न्नमो॒ नमः। 
मन्युरकार्\mbox{}षीन्मन्युः करोति नाहं करोमि मन्युः कर्ता नाहं कर्ता मन्युः॑ कार॒यिता नाहं॑ कार॒यिता एष ते मन्यो मन्य॑वे स्वा॒हा॥७९॥
\anuvakamend

\dnsub{विराजहोममन्त्राः}
तिलाञ्जुहोमि सरसाꣳ सपिष्टान् गन्धार मम चित्ते रम॑न्तु स्वा॒हा। 
गावो हिरण्यं धनमन्नपानꣳ सर्वेषाꣴ श्रि॑यै स्वा॒हा। 
श्रियं च लक्ष्मीं च पुष्टिं च कीर्तिं॑ चानृ॒ण्यताम्। 
ब्रह्मण्यं ब॑हुपु॒त्रताम्। 
श्रद्धामेधे प्रजाः सन्ददा॑तु स्वा॒हा॥८०॥
\anuvakamend

तिलाः कृष्णास्ति॑लाः श्वे॒ता॒स्तिलाः सौम्या व॑शानु॒गाः। 
तिलाः पुनन्तु॑ मे पा॒पं॒ यत्किञ्चिद्दुरितं म॑यि स्वा॒हा। 
चोर॒स्यान्नं न॑वश्रा॒द्धं॒ ब्र॒ह्म॒हा गु॑रुत॒ल्पगः। 
गोस्तेयꣳ सु॑रापा॒नं॒ भ्रूणहत्या तिला शान्तिꣳ शमय॑न्तु स्वा॒हा। 
श्रीश्च लक्ष्मीश्च पुष्टीश्च कीर्तिं॑ चानृ॒ण्यताम्। 
ब्रह्मण्यं ब॑हुपु॒त्रताम्। 
श्रद्धामेधे प्रज्ञा तु जातवेदः सन्ददा॑तु स्वा॒हा॥८१॥
 \anuvakamend

प्राणापानव्यानोदानसमाना मे॑ शुद्ध्य॒न्तां॒ ज्योति॑र॒हं वि॒रजा॑ विपा॒प्मा भू॑यास॒ꣴ॒ स्वाहा᳚।
वाङ्मनश्चक्षुःश्रोत्रजिह्वाघ्राणरेतो\-बुद्ध्याकूतिः\-सङ्कल्पा मे॑ शुद्ध्य॒न्तां॒ ज्योति॑र॒हं वि॒रजा॑ विपा॒प्मा भू॑यास॒ꣴ॒ स्वाहा᳚। 
त्वक्चर्ममाꣳसरुधिरमेदोमज्जास्नायवो\-ऽस्थीनि  मे॑ शुद्ध्य॒न्तां॒ ज्योति॑र॒हं वि॒रजा॑ विपा॒प्मा भू॑यास॒ꣴ॒ स्वाहा᳚। 
शिरःपाणिपादपार्श्वपृष्ठोरूदरजङ्घशिश्ञोपस्थपायवो  मे॑ शुद्ध्य॒न्तां॒ ज्योति॑र॒हं वि॒रजा॑ विपा॒प्मा भू॑यास॒ꣴ॒ स्वाहा᳚।
उत्तिष्ठ पुरुष हरित पिङ्गल लोहिताक्षि देहि देहि ददापयिता मे॑ शुद्ध्य॒न्तां॒ ज्योति॑र॒हं वि॒रजा॑ विपा॒प्मा भू॑यास॒ꣴ॒ स्वाहा᳚॥८२॥ 
\anuvakamend

पृथिव्यापस्तेजोवायुराकाशा मे॑ शुद्ध्य॒न्तां॒ ज्योति॑र॒हं वि॒रजा॑ विपा॒प्मा भू॑यास॒ꣴ॒ स्वाहा᳚। 
शब्दस्पर्शरूपरसगन्धा  मे॑ शुद्ध्य॒न्तां॒ ज्योति॑र॒हं वि॒रजा॑ विपा॒प्मा भू॑यास॒ꣴ॒ स्वाहा᳚। 
मनोवाक्कायकर्माणि  मे॑ शुद्ध्य॒न्तां॒ ज्योति॑र॒हं वि॒रजा॑ विपा॒प्मा भू॑यास॒ꣴ॒ स्वाहा᳚। 
अव्यक्तभावैर॑हङ्का॒रै॒र्ज्योति॑र॒हं वि॒रजा॑ विपा॒प्मा भू॑यास॒ꣴ॒ स्वाहा᳚। 
आत्मा मे॑ शुद्ध्य॒न्तां॒ ज्योति॑र॒हं वि॒रजा॑ विपा॒प्मा भू॑यास॒ꣴ॒ स्वाहा᳚। 
अन्तरात्मा मे॑ शुद्ध्य॒न्तां॒ ज्योति॑र॒हं वि॒रजा॑ विपा॒प्मा भू॑यास॒ꣴ॒ स्वाहा᳚। 
परमात्मा  मे॑ शुद्ध्य॒न्तां॒ ज्योति॑र॒हं वि॒रजा॑ विपा॒प्मा भू॑यास॒ꣴ॒ स्वाहा᳚। 
क्षु॒धे स्वाहा᳚। 
क्षुत्पि॑पासाय॒ स्वाहा᳚। 
विवि॑ट्यै॒ स्वाहा᳚। 
ऋग्वि॑धानाय॒ स्वाहा᳚। 
क॒षो᳚त्काय॒ स्वाहा᳚। 
क्षु॒त्पि॒पा॒साम॑लं ज्ये॒ष्ठा॒म॒ल॒क्ष्मीर्ना॑शया॒म्यहम्। 
अभू॑ति॒मस॑मृद्धिं॒ च॒ सर्वान्निर्णुद मे पाप्मा॑नꣴ स्वा॒हा।
अन्नमय-प्राणमय-मनोमय-विज्ञानमय-मानन्दमय-मात्मा मे॑ शुद्ध्य॒न्तां॒ ज्योति॑र॒हं वि॒रजा॑ विपा॒प्मा भू॑यास॒ꣴ॒ स्वाहा᳚॥८३॥
\anuvakamend

\dnsub{वैश्वदेवमन्त्राः}
अ॒ग्नये॒ स्वाहा᳚। 
विश्वे᳚भ्यो दे॒वेभ्यः॒ स्वाहा᳚। 
ध्रु॒वाय॑ भू॒माय॒ स्वाहा᳚। 
ध्रु॒व॒क्षित॑ये॒ स्वाहा᳚। 
अ॒च्यु॒त॒क्षित॑ये॒ स्वाहा᳚। 
अ॒ग्नये᳚ स्विष्ट॒कृते॒ स्वाहा᳚॥ 
धर्मा॑य॒ स्वाहा᳚। 
अध॑र्माय॒ स्वाहा᳚। 
अ॒द्भ्यः स्वाहा᳚। 
ओ॒ष॒धि॒व॒न॒स्प॒तिभ्यः॒ स्वाहा᳚॥८४॥ 


र॒क्षो॒दे॒व॒ज॒नेभ्यः॒ स्वाहा᳚। 
गृह्या᳚भ्यः॒ स्वाहा᳚। 
अ॒व॒साने᳚भ्यः॒ स्वाहा᳚। 
अ॒व॒सान॑पतिभ्यः॒ स्वाहा᳚। 
स॒र्व॒भू॒तेभ्यः॒ स्वाहा᳚। 
कामा॑य॒ स्वाहा᳚। 
अ॒न्तरि॑क्षाय॒ स्वाहा᳚। 
यदेज॑ति॒ जग॑ति॒ यच्च॒ चेष्ट॑ति॒ नाम्नो॑ भा॒गोऽयं नाम्ने॒ स्वाहा᳚। 
पृ॒थि॒व्यै स्वाहा᳚। 
अ॒न्तरि॑क्षाय॒ स्वाहा᳚॥८५॥ 


दि॒वे स्वाहा᳚। 
सूर्या॑य॒ स्वाहा᳚। 
च॒न्द्रम॑से॒ स्वाहा᳚। 
नक्ष॑त्रेभ्यः॒ स्वाहा᳚। 
इन्द्रा॑य॒ स्वाहा᳚। 
बृह॒स्पत॑ये॒ स्वाहा᳚। 
प्र॒जाप॑तये॒ स्वाहा᳚। 
ब्रह्म॑णे॒ स्वाहा᳚। 
स्व॒धा पि॒तृभ्यः॒ स्वाहा᳚। 
नमो॑ रु॒द्राय॑ पशु॒पत॑ये॒ स्वाहा᳚॥८६॥

दे॒वेभ्यः॒ स्वाहा᳚। 
पि॒तृभ्यः॑ स्व॒धाऽस्तु॑। 
भू॒तेभ्यो॒ नमः॑। 
म॒नु॒ष्ये᳚भ्यो॒ हन्ता᳚। 
प्र॒जाप॑तये॒ स्वाहा᳚। 
प॒र॒मे॒ष्ठिने॒ स्वाहा᳚। 
यथा कू॑पः श॒तधा॑रः स॒हस्र॑धारो॒ अक्षि॑तः। 
ए॒वा मे॑ अस्तु धा॒न्यꣳ स॒हस्र॑धार॒मक्षि॑तम्। 
धन॑धान्यै॒ स्वाहा᳚। 
ये भू॒ताः प्र॒चर॑न्ति॒ दिवा॒नक्तं॒ बलि॑मि॒च्छन्तो॑ वि॒तुद॑स्य॒ प्रेष्याः᳚। 
तेभ्यो॑ ब॒लिं पु॑ष्टि॒कामो॑ हरामि॒ मयि॒ पुष्टिं॒ पुष्टि॑पतिर्दधातु॒ स्वाहा᳚॥८७॥ 
\anuvakamend

ओं᳚ तद्ब्र॒ह्म। 
ओं᳚ तद्वा॒युः। 
ओं᳚ तदा॒त्मा। 
ओं᳚ तथ्स॒त्यम्‌।
ओं᳚ तथ्सर्वम्᳚‌। 
ओं᳚ तत्पुरो॒र्नमः॥
अन्तश्चरति॑ भूते॒षु॒ गुहायां वि॑श्वमू॒र्तिषु।
त्वं यज्ञस्त्वं वषट्कारस्त्वमिन्द्रस्त्वꣳ रुद्रस्त्वं विष्णुस्त्वं ब्रह्म त्वं॑ प्रजा॒पतिः।
त्वं त॑दाप॒ आपो॒ ज्योती॒ रसो॒ऽमृतं॒ ब्रह्म॒ भूर्भुव॒स्सुव॒रोम्‌॥८८॥
\anuvakamend

\dnsub{प्राणाहुतिमन्त्राः}
श्र॒द्धायां᳚ प्रा॒णे निवि॑ष्टो॒\-ऽ\-मृतं॑ जुहोमि। 
श्र॒द्धाया॑\-मपा॒ने निवि॑ष्टो॒\-ऽ\-मृतं॑ जुहोमि। 
श्र॒द्धायां᳚ व्या॒ने निवि॑ष्टो॒\-ऽ\-मृतं॑ जुहोमि। 
श्र॒द्धाया॑\-मुदा॒ने निवि॑ष्टो॒\-ऽ\-मृतं॑ जुहोमि। 
श्र॒द्धायाꣳ॑ समा॒ने निवि॑ष्टो॒\-ऽ\-मृतं॑ जुहोमि। 
ब्रह्म॑णि म आ॒त्माऽमृ॑त॒त्वाय॑॥ 
अ॒मृ॒तो॒प॒स्तर॑णमसि॥ 
श्र॒द्धायां᳚ प्रा॒णे निवि॑ष्टो॒\-ऽ\-मृतं॑ जुहोमि। 
शि॒वो मा॑ वि॒शाप्र॑दाहाय। 
प्रा॒णाय॒ स्वाहा᳚॥ 
श्र॒द्धाया॑\-मपा॒ने निवि॑ष्टो॒\-ऽ\-मृतं॑ जुहोमि। 
शि॒वो मा॑ वि॒शाप्र॑दाहाय। 
अ॒पा॒नाय॒ स्वाहा᳚॥ 
श्र॒द्धायां᳚ व्या॒ने निवि॑ष्टो॒\-ऽ\-मृतं॑ जुहोमि। 
शि॒वो मा॑ वि॒शाप्र॑दाहाय। 
व्या॒नाय॒ स्वाहा᳚॥ 
श्र॒द्धाया॑\-मुदा॒ने निवि॑ष्टो॒\-ऽ\-मृतं॑ जुहोमि। 
शि॒वो मा॑ वि॒शाप्र॑दाहाय। 
उ॒दा॒नाय॒ स्वाहा᳚॥ 
श्र॒द्धायाꣳ॑ समा॒ने निवि॑ष्टो॒\-ऽ\-मृतं॑ जुहोमि। 
शि॒वो मा॑ वि॒शाप्र॑दाहाय। 
स॒मा॒नाय॒ स्वाहा᳚॥ 
ब्रह्म॑णि म आ॒त्माऽमृ॑त॒त्वाय॑। 
अ॒मृ॒ता॒पि॒धा॒नम॑सि॥८९॥
 \anuvakamend


\dnsub{भुक्तान्नाभिमन्त्रणमन्त्राः}
श्र॒द्धायां᳚ प्रा॒णे निवि॑श्या॒मृतꣳ॑ हु॒तम्। 
प्रा॒णमन्ने॑नाप्यायस्व। 
श्र॒द्धाया॑\-मपा॒ने निवि॑श्या॒मृतꣳ॑ हु॒तम्। 
अ॒पा॒नमन्ने॑नाप्यायस्व।
श्र॒द्धायां᳚ व्या॒ने निवि॑श्या॒मृतꣳ॑ हु॒तम्। 
व्या॒नमन्ने॑नाप्यायस्व। 
श्र॒द्धाया॑\-मुदा॒ने निवि॑श्या॒मृतꣳ॑ हु॒तम। 
उ॒दा॒नमन्ने॑नाप्यायस्व।
श्र॒द्धायाꣳ॑ समा॒ने निवि॑श्या॒मृतꣳ॑ हु॒तम्। 
स॒मा॒नमन्ने॑नाप्या\-यस्व॥९०॥\anuvakamend

\dnsub{भोजनान्ते आत्मानुसन्धानमन्त्राः}
अङ्गुष्ठमात्रः पुरुषोऽङ्गुष्ठं च॑ समा॒श्रितः। 
ईशः सर्वस्य जगतः प्रभुः प्रीणाति॑ विश्व॒भुक्॥॥९१॥\anuvakamend

\dnsub{अवयवस्वस्थता-प्रार्थनामन्त्रः}
वाङ्म॑ आ॒सन्। 
न॒सोः प्रा॒णः। 
अ॒क्ष्योश्चक्षुः॑। 
कर्ण॑योः॒ श्रोत्रम्᳚। 
बा॒हु॒वोर्बलम्᳚। 
ऊ॒रु॒वोरोजः॑। 
अरि॑ष्टा॒ विश्वा॒न्यङ्गा॑नि त॒नूः। 
त॒नुवा॑ मे स॒ह नम॑स्ते अस्तु॒ मा मा॑ हिꣳसीः॥९२॥\anuvakamend

\dnsub{इन्द्रसप्तर्षि-संवादमन्त्रः}
वयः॑ सुप॒र्णा उप॑ सेदु॒रिन्द्रं॑ प्रि॒यमे॑धा॒ ऋष॑यो॒ नाध॑मानाः। 
अप॑ ध्वा॒न्तमू᳚र्णु॒हि पू॒र्धि चक्षु॑र्मुमु॒ग्ध्य॑स्मान्नि॒धये॑ऽव ब॒द्धान्।\anuvakamend

\dnsub{हृदयालम्भनमन्त्रः}
प्राणानां ग्रन्थिरसि रुद्रो मा॑ विशा॒न्तकः। 
तेनान्नेना᳚प्या\-य॒स्व॥९३॥\anuvakamend

\dnsub{देवताप्राणनिरूपणमन्त्रः}
नमो रुद्राय विष्णवे मृत्यु॑र्मे पा॒हि॥९४॥\anuvakamend


\dnsub{अग्निस्तुतिमन्त्रः}
त्वम॑ग्ने॒ द्युभि॒स्त्वमा॑शुशु॒क्षणि॒स्त्वम॒द्भ्यस्त्वमश्म॑न॒स्परि॑। 
त्वं वने᳚भ्य॒स्त्वमोष॑धीभ्य॒स्त्वं नृ॒णां नृ॑पते जायसे॒ शुचिः॑॥९५॥\anuvakamend

\dnsub{अभीष्टयाचनामन्त्राः}
शि॒वेन॑ मे॒ सन्ति॑ष्ठस्व स्यो॒नेन॑  मे॒ सन्ति॑ष्ठस्व सुभू॒तेन॑  मे॒ सन्ति॑ष्ठस्व ब्रह्मवर्च॒सेन॑  मे॒ सन्ति॑ष्ठस्व य॒ज्ञस्यर्धि॒मनु॒ सन्ति॑ष्ठ॒स्वोप॑ ते यज्ञ॒ नम॒ उप॑ ते॒ नम॒ उप॑ ते॒ नमः॑॥९६॥\anuvakamend

\newcommand{\sep}{{\small$\circ$} }
\dnsub{परतत्त्व-निरूपणम्}
स॒त्यं परं॒ परꣳ॑ स॒त्यꣳ स॒त्येन॒ न सु॑व॒र्गाल्लो॒काच्च्य॑वन्ते क॒दाच॒न स॒ताꣳ हि स॒त्यं तस्मा᳚थ्स॒त्ये र॑मन्ते॒ \sep
तप॒ इति॒ तपो॒ नानश॑ना॒त्परं॒ यद्धि परं॒ तप॒स्तद्दुर्ध॑र्\mbox{}षं॒ तद्दुरा॑धर्\mbox{}षं॒ तस्मा॒त्तप॑सि रमन्ते॒ \sep
दम॒ इति॒ निय॑तं ब्रह्मचा॒रिण॒स्तस्मा॒द्दमे॑ रमन्ते॒ \sep
शम॒ इत्यर॑ण्ये मु॒नय॒स्तस्मा॒च्छमे॑ रमन्ते \sep
दा॒नमिति॒ सर्वा॑णि भू॒तानि॑ प्र॒शꣳस॑न्ति दा॒नान्नाति॑ दु॒ष्करं॒ तस्मा᳚द्दा॒ने र॑मन्ते \sep
ध॒र्म इति॒ धर्मे॑ण॒ सर्व॑मि॒दं परि॑गृहीतं ध॒र्मान्नाति॑ दु॒ष्करं॒ तस्मा᳚द्ध॒र्मे र॑मन्ते \sep
प्र॒जन॒ इति॒ भूयाꣳ॑स॒स्तस्मा॒द्भूयि॑ष्ठाः॒ प्रजा॑यन्ते॒ तस्मा॒द्भूयि॑ष्ठाः प्र॒जन॑ने रमन्ते॒ऽग्नय॒ \sep
इत्या॑ह॒ तस्मा॑द॒ग्नय॒ आधा॑तव्या अग्निहो॒त्रमित्या॑ह॒ तस्मा॑दग्निहो॒त्रे र॑मन्ते \sep
य॒ज्ञ इति॑ य॒ज्ञो हि दे॒वास्तस्मा᳚द्य॒ज्ञे र॑मन्ते \sep
मान॒समिति॑ वि॒द्वाꣳस॒स्तस्मा᳚द्वि॒द्वाꣳस॑ ए॒व मा॑न॒से र॑मन्ते \sep
न्या॒स इति॑ ब्र॒ह्मा ब्र॒ह्मा हि परः॒ परो॑ हि ब्र॒ह्मा तानि॒ वा ए॒तान्यव॑राणि॒ पराꣳ॑सि न्या॒स ए॒वात्य॑रेचय॒द्य ए॒वं वेदे᳚त्युप॒निषत्॥९७॥
%६.६३.०
\anuvakamend


\dnsub{ज्ञानसाधन-निरूपणम्}
प्रा॒जा॒प॒त्यो हारु॑णिः सुप॒र्णेयः॑ प्र॒जाप॑तिं पि॒तर॒मुप॑ससार॒ किं भ॑गव॒न्तः प॑र॒मं व॑द॒न्तीति॒ तस्मै॒ प्रो॑वाच \sep
स॒त्येन॑ वा॒युरावा॑ति स॒त्येना॑ऽऽदि॒त्यो रो॑चते दि॒वि स॒त्यं वा॒चः प्र॑ति॒ष्ठा स॒त्ये स॒र्वं प्रति॑ष्ठितं॒ तस्मा᳚थ्स॒त्यं प॑र॒मं वद॑न्ति॒ \sep
तप॑सा दे॒वा दे॒वता॒मग्र॑ आय॒न्तप॒सर्\mbox{}ष॑यः॒ सुव॒रन्व॑विन्द॒न् तप॑सा स॒पत्ना॒न् प्रणु॑दा॒मारा॑ती॒स्तप॑सि स॒र्वं प्रति॑ष्ठितं॒ तस्मा॒त्तपः॑ पर॒मं वद॑न्ति॒ \sep
दमे॑न दा॒न्ताः कि॒ल्बिष॑मवधू॒न्वन्ति॒ दमे॑न ब्रह्मचा॒रिणः॒ सुव॑रगच्छ॒न्दमो॑ भू॒तानां᳚ दुरा॒धर्\mbox{}षं॒ दमे॑ स॒र्वं प्रति॑ष्ठितं॒ तस्मा॒द्दमः॑ पर॒मं वद॑न्ति॒ \sep
शमे॑न शा॒न्ताः  शि॒वमा॒चर॑न्ति॒ शमे॑न ना॒कं मु॒नयो॒ऽन्ववि॑न्द॒ञ्छमो॑ भू॒तानां᳚ दुरा॒धर्\mbox{}ष॒ञ्छमे॑ स॒र्वं प्रति॑ष्ठितं॒ तस्मा॒च्छमः॑ पर॒मं वद॑न्ति \sep
दा॒नं य॒ज्ञानां॒ वरू॑थं॒ दक्षि॑णा लो॒के दा॒तारꣳ॑ सर्वभू॒तान्यु॑पजी॒वन्ति॑ दा॒नेनारा॑ती॒रपा॑नुदन्त दा॒नेन॑ द्विष॒न्तो मि॒त्रा भ॑वन्ति दा॒ने स॒र्वं प्रति॑ष्ठितं॒ तस्मा᳚द्दा॒नं प॑र॒मं वद॑न्ति \sep
ध॒र्मो विश्व॑स्य॒ जग॑तः प्रति॒ष्ठा लो॒के ध॒र्मिष्ठं॑ प्र॒जा उ॑पस॒र्पन्ति॑ ध॒र्मेण॑ पा॒पम॑प॒नुद॑ति ध॒र्मे स॒र्वं प्रति॑ष्ठितं॒ तस्मा᳚द्ध॒र्मं प॑र॒मं वद॑न्ति \sep
प्र॒जन॑नं॒ वै प्र॑ति॒ष्ठा लो॒के सा॒धु प्र॒जाया᳚स्त॒न्तुं त॑न्वा॒नः पि॑तृ॒णाम॑नृ॒णो भव॑ति॒ तदे॑व त॒स्यानृ॑णं॒ तस्मा᳚त् प्र॒जन॑नं पर॒मं वद॑न्त्य॒ग्नयो॒ वै त्रयी॑ वि॒द्या \sep
 दे॑व॒यानः॒ पन्था॑ गार्\mbox{}हप॒त्य ऋक्पृ॑थि॒वी र॑थन्त॒रम॑न्वाहार्य॒पच॑नं॒ यजु॑र॒न्तरि॑क्षं वामदे॒व्यमा॑हव॒नीयः॒ साम॑ सुव॒र्गो लो॒को बृ॒हत्तस्मा॑द॒ग्नीन्प॑र॒मं वद॑न्त्यग्निहो॒त्रꣳ सा॑यं प्रा॒तर्गृ॒हाणां॒ निष्कृ॑तिः॒ स्वि॑ष्टꣳ सुहु॒तं य॑ज्ञक्रतू॒नां प्राय॑णꣳ सुव॒र्गस्य॑ लो॒कस्य॒ ज्योति॒स्तस्मा॑दग्निहो॒त्रं प॑र॒मं वद॑न्ति \sep
 य॒ज्ञ इति॑ य॒ज्ञेन॒ हि दे॒वा दिवं॑ ग॒ता य॒ज्ञेनासु॑रा॒नपा॑नुदन्त य॒ज्ञेन॑ द्विष॒न्तो मि॒त्रा भ॑वन्ति य॒ज्ञे स॒र्वं प्रति॑ष्ठितं॒ तस्मा᳚द्य॒ज्ञं प॑र॒मं वद॑न्ति \sep 
 मान॒सं वै प्रा॑जाप॒त्यं प॒वित्रं॑ मान॒सेन॒ मन॑सा सा॒धु प॑श्यति मान॒सा ऋष॑यः प्र॒जा अ॑सृजन्त मान॒से स॒र्वं प्रति॑ष्ठितं॒ तस्मा᳚न्मान॒सं प॑र॒मं वद॑न्ति \sep
 न्या॒स इ॒त्याहु॑र्मनी॒षिणो᳚ ब्र॒ह्माणं॑ ब्र॒ह्मा विश्वः॑ कत॒मः स्व॑य॒म्भुः प्र॒जाप॑तिः संवथ्स॒र इति॑ संवथ्स॒रो॑ऽसावा॑दि॒त्यो य ए॒ष आ॑दि॒त्ये पुरु॑षः॒ स प॑रमे॒ष्ठी ब्रह्मा॒त्मा \sep
 याभि॑रादि॒त्यस्तप॑ति र॒श्मिभि॒स्ताभिः॑ प॒र्जन्यो॑ वर्\mbox{}षति प॒र्जन्ये॑नौषधिवनस्प॒तयः॒ प्रजा॑यन्त ओषधिवनस्प॒तिभि॒रन्नं॑ भव॒त्यन्ने॑न प्रा॒णाः प्रा॒णैर्बलं॒ बले॑न॒ तप॒स्तप॑सा श्र॒द्धा श्र॒द्धया॑ मे॒धा मे॒धया॑ मनी॒षा म॑नी॒षया॒ मनो॒ मन॑सा॒ शान्तिः॒ शान्त्या॑ चि॒त्तं चि॒त्तेन॒ स्मृति॒ꣴ॒ स्मृत्या॒ स्मार॒ꣴ॒ स्मारे॑ण वि॒ज्ञानं॑  वि॒ज्ञाने॑ना॒ऽ॒ऽ॒त्मानं॑ वेदयति॒ तस्मा॑द॒न्नं दद॒न्थ्सर्वा᳚ण्ये॒तानि॑ ददा॒त्यन्ना᳚त् प्रा॒णा भ॑वन्ति \sep
  भू॒तानां᳚ प्रा॒णैर्मनो॒ मन॑सश्च वि॒ज्ञानं॑  वि॒ज्ञाना॑दान॒न्दो ब्र॑ह्मयो॒निः स वा ए॒ष पुरु॑षः पञ्च॒धा प॑ञ्चा॒त्मा येन॒ सर्व॑मि॒दं प्रोतं॑ पृथि॒वी चा॒न्तरि॑क्षं च॒ द्यौश्च॒ दिश॑श्चावान्तरदि॒शाश्च॒ स वै सर्व॑मि॒दं जग॒थ्स च॒ भूतꣳ॑ स भ॒व्यं जि॑ज्ञासकॢ॒प्त ऋ॑त॒जा रयि॑ष्ठा \sep
  श्र॒द्धा स॒त्यो मह॑स्वान्त॒पसो॒ वरि॑ष्ठा॒द्ज्ञात्वा॑ तमे॒वं मन॑सा हृ॒दा च॒ भूयो॑ न मृ॒त्युमुप॑याहि वि॒द्वान्तस्मा᳚न्न्या॒समे॒षां तप॑सामतिरिक्त॒माहु॑र्वसुर॒ण्वो॑ वि॒भूर॑सि प्रा॒णे त्वमसि॑ सन्धा॒ता \sep 
 ब्रह्म॑न् त्वमसि॑ विश्व॒धृत्ते॑जो॒दास्त्वम॑स्य॒ग्निर॑सि वर्चो॒दास्त्वम॑सि॒ सूर्य॑स्य द्युम्नो॒दास्त्वम॑सि च॒न्द्रम॑स उपया॒मगृ॑हीतोऽसि ब्र॒ह्मणे᳚ त्वा॒ \sep
 महस॒ ओमित्या॒त्मानं॑ युञ्जीतै॒तद्वै म॑होप॒निष॑दं दे॒वानां॒ गुह्यं॒ य ए॒वं वेद॑ ब्र॒ह्मणो॑ महि॒मान॑माप्नोति॒ तस्मा᳚द्ब्र॒ह्मणो॑ महि॒मान॑मित्युप॒निषत्॥९८॥\anuvakamend



\dnsub{ज्ञानयज्ञः}
तस्यै॒वं  वि॒दुषो॑ य॒ज्ञस्या॒ऽ॒ऽ॒त्मा यज॑मानः श्र॒द्धा पत्नी॒ शरी॑रमि॒ध्ममुरो॒ वेदि॒र्लोमा॑नि ब॒र्॒‌हिर्वे॒दः शिखा॒ हृद॑यं॒ यूपः॒ काम॒ आज्यं॑ म॒न्युः प॒शुस्तपो॒ऽग्निर्दमः॑ शमयि॒ता दक्षि॑णा॒ वाग्घोता᳚ प्रा॒ण उ॑द्गा॒ता चक्षु॑रध्व॒र्युर्मनो॒ ब्रह्मा॒ \sep 
श्रोत्र॑म॒ग्नीद्याव॒द्ध्रिय॑ते॒ सा दी॒क्षा यदश्ञा॑ति॒ तद्धवि॒र्यत्पिब॑ति॒ तद॑स्य सोमपा॒नं यद्रम॑ते॒ तदु॑प॒सदो॒ यथ्स॒ञ्चर॑त्युप॒विश॑त्यु॒त्तिष्ठ॑ते च॒ स प्र॑व॒र्ग्यो॑ यन्मुखं॒ तदा॑हव॒नीयो॒ या व्याहृ॑तिराहु॒तिर्यद॑स्य \sep 
वि॒ज्ञानं॒ तज्जु॒होति॒ यथ्सा॒यं प्रा॒तर॑त्ति॒ तथ्स॒मिधं॒ यत्प्रा॒तर्म॒ध्यं दि॑नꣳ सा॒यं  च॒ तानि॒ सव॑नानि॒ ये अ॑होरा॒त्रे ते द॑र्\mbox{}शपूर्णमा॒सौ ये᳚ऽर्धमा॒साश्च॒ मासा᳚श्च॒ ते चा॑तुर्मा॒स्यानि॒ य ऋ॒तव॒स्ते प॑शुब॒न्धा ये सं॑वथ्स॒राश्च॑ परिवथ्स॒राश्च॒ तेऽह॑र्ग॒णाः स॑र्ववेद॒सं वा \sep 
ए॒तथ्स॒त्रं यन्मर॑णं॒ तद॑व॒भृथ॑ ए॒तद्वै ज॑रामर्यमग्निहो॒त्रꣳ स॒त्रं य ए॒वं  वि॒द्वानु॑द॒गय॑ने प्र॒मीय॑ते दे॒वाना॑मे॒व म॑हि॒मानं॑ ग॒त्वाऽऽदि॒त्यस्य॒ सायु॑ज्यं गच्छ॒त्यथ॒ \sep
यो द॑क्षि॒णे प्र॒मीय॑ते पितृ॒णामे॒व म॑हि॒मानं॑ ग॒त्वा च॒न्द्रम॑सः॒ सायु॑ज्य सलो॒कता॑माप्नोत्ये॒तौ वै सू᳚र्याचन्द्र॒मसो᳚र्महि॒मानौ᳚ ब्राह्म॒णो वि॒द्वान॒भिज॑यति॒ तस्मा᳚द्ब्र॒ह्मणो॑ महि॒मान॑माप्नोति॒ तस्मा᳚द्ब्र॒ह्मणो॑ महि॒मान॑मित्युप॒निषत्॥९९॥
\anuvakamend

ॐ स॒ह ना॑ववतु। 
स॒ह नौ॑ भुनक्तु। 
स॒ह वी॒र्यं॑ करवावहै। 
ते॒ज॒स्वि ना॒वधी॑तमस्तु॒ मा वि॑द्विषा॒वहै᳚। 
ॐ शान्तिः॒ शान्तिः॒ शान्तिः॑॥

\closesection
\clearpage
